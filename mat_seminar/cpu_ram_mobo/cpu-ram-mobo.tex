\documentclass[a4paper,11pt]{article}

\usepackage[czech]{babel}
% Fonts %
\usepackage{fouriernc}
\usepackage[T1]{fontenc}

% Colors %
\usepackage[dvipsnames]{color}
\usepackage[dvipsnames]{xcolor}

% Page Layout %
\usepackage[margin=1.5in]{geometry}

% Fancy Headers %
\usepackage{fancyhdr}
\fancyhf{}
\cfoot{\thepage}
\rhead{}
\renewcommand{\headrulewidth}{0pt}
\setlength{\headheight}{16pt}

% Math
\usepackage{mathtools}
\usepackage{amssymb}
\usepackage{faktor}
\usepackage{import}
\usepackage{caption}
\usepackage{subcaption}
\usepackage{wrapfig}
\usepackage{enumitem}
\setlist{topsep=0pt}

\usepackage{tikz}
\usetikzlibrary{cd,positioning,babel,shapes,calc}
\usepackage{tkz-base}
\usepackage{tkz-euclide}

% Logic Circuits
\usepackage{circuitikz}
\ctikzset{
 logic ports=ieee,
}

% Theorems
\usepackage[thmmarks, amsmath, thref]{ntheorem}
\usepackage{thmtools}

\theoremsymbol{\ensuremath{\blacksquare}}
\newtheorem*{solution}{Possible solution.}

% Table of Contents %
\usepackage{hyperref}
\hypersetup{
 colorlinks=true,
 linktoc=all,
 linkcolor=blue
}

% Tables %
\usepackage{booktabs}
\usepackage{tabularx}

% Patch for hyphens
\usepackage{regexpatch}
\makeatletter
% Change the `-` delimiter to an active character
\xpatchparametertext\@@@cmidrule{-}{\cA-}{}{}
\xpatchparametertext\@cline{-}{\cA-}{}{}
\makeatother

\newcolumntype{s}{>{\centering\arraybackslash}p{.4\textwidth}}

% Operators %
\DeclareMathOperator{\Ker}{Ker}
\DeclareMathOperator{\Img}{Im}
\DeclareMathOperator{\End}{End}
\DeclareMathOperator{\Aut}{Aut}
\DeclareMathOperator{\Inn}{Inn}

% Common operators %
\newcommand{\R}{\mathbb{R}}
\newcommand{\N}{\mathbb{N}}
\newcommand{\Z}{\mathbb{Z}}
\newcommand{\Q}{\mathbb{Q}}
\newcommand{\C}{\mathbb{C}}

\newcommand{\clr}{\textcolor{red}}
\newcommand{\clb}{\textcolor{blue}}
\newcommand{\clg}{\textcolor{green}}
\newcommand{\clm}{\textcolor{magenta}}
\newcommand{\clv}{\textcolor{violet}}
\newcommand{\clbr}{\textcolor{Sepia}}

% American Paragraph Skip %
\setlength{\parindent}{0pt}
\setlength{\parskip}{1em}

% Document %
\pagestyle{fancy}
\begin{document}

\thispagestyle{fancy}

\section*{CPU}

CPU (\textbf{C}entral \textbf{P}rocessing \textbf{U}nit) je elektrický obvod,
který vykonává instrukce uložené ve vnitřní paměti. Vykonání \textbf{jedné}
instrukce spočívá ve třech krocích
\[
 \text{Fetch} \to \text{Decode} \to \text{Execute}
\]
\begin{itemize}
 \item \textbf{Fetch} je vyzvednutí instrukce z vnitřní paměti. CPU zkrátka
  přečte tolik bitů z paměti, kolik je délka jeho instrukce (mezi různými CPU se
  liší), z~toho místa, odkud četl naposledy.
 \item \textbf{Decode} přeloží vyzvednutou instrukci. Opět, každý procesor má
  svoje specifikace, co se kódování instrukcí týče. Pro jedno CPU může třeba
  10001111 znamenat \uv{sečti}, pro jiný \uv{načti vstup z klávesnice}.
 \item \textbf{Execute} vykoná přeloženou instrukci. Vykonání obvykle spočívá v
  delegaci instrukce jedné z jednotek CPU (popsány níže).
\end{itemize}

CPU má v sobě ukazatel (tzv. \emph{counter} nebo \emph{čítač}), pomocí něhož si
pamatuje, na kterém místě v paměti zrovna je. Po každém vykonání instrukce
posune čítač na následující místo v paměti. Slovo \uv{místo} zde může znamenat
různý počet bitů, v~závislosti na architektuře CPU. 

Několik základních instrukcí, které CPU může z paměti přečíst, je:
\begin{itemize}
 \item \textbf{LOAD} -- načti číslo z paměti do CPU,
 \item \textbf{ADD} -- sečti dvě čísla, která v paměti následují (CPU
  implementuje ostatní operace jako odčítání, násobení a dělení pomocí sčítání),
 \item \textbf{STORE} -- ulož číslo do paměti,
 \item \textbf{COMPARE} -- porovnej dvě čísla, která v paměti následují,
 \item \textbf{JUMP} -- přesuň čítač na místo v paměti určené následujícím
  číslem,
 \item \textbf{JUMP if podmínka} -- přesuň čítač na místo v paměti určené
  následujícím číslem, pokud byla splněna \textbf{podmínka},
 \item \textbf{OUT} -- předej výstup nějakému zařízení (třeba monitor,
  reproduktor apod.),
 \item \textbf{IN} -- získej vstup od nějakého zařízení (třeba myš, klávesnice
  apod.).
\end{itemize}

Původní CPU vykonávaly vždy jednu instrukci za jeden úder \emph{vnitřních
hodin}. Počet úderů za vteřinu značí, jak rychle je CPU schopen vykonávat
instrukce. Moderní CPU mají frekvenci vnitřních hodin měřenou v gigahertzích
(tj. v miliardách úderů za vteřinu). Navíc mají obvykle více \emph{virtuálních
jednotek} (též \emph{jader}) a jsou schopny během jednoho úderu vykonat souběžně
více instrukcí. Konkrétně, CPU, který vykoná jednu instrukci za jeden úder a
frekvenci úderů má 1 Ghz, vykoná přesně miliardu instrukcí za vteřinu.

CPU bývá obvykle rozdělen do několika menších jednotek, z nichž nezanedbatelné
jsou
\begin{itemize}
 \item \textbf{Control Unit} (\uv{řídící jednotka}), která obdrží instrukce z
  vnitřní paměti a poté je rozešle ostatním jednotkám v závislosti na povaze
  instrukce;
 \item \textbf{Arithmetic Logic Unit} (\uv{jednotka výpočetní logiky}), která
  vykonává instrukce (obdržené od Control Unit) představující aritmetické
  ($+$, $-$, $*$, $/$, ...), logické (ne, a, nebo,...) nebo srovnávací ($<$,
  $>$, $=$) operace.
\end{itemize}

Mnoho moderních CPU má navíc ještě \textbf{Memory Management Unit} (\uv{jednotku
správy paměti}), která obstarává instrukce typu \textbf{STORE} nebo
\textbf{JUMP} -- obecně instrukce související s pamětí.

\subsection*{Logické obvody}

Samotné CPU je složeno z pospojovaných logických obvodů, které jsou zase složeny
z logických bran. \textbf{Logická brána} je prostě jen elektrický obvod, který
je schopný \uv{vykonat} logickou operaci. Třeba AND brána vykonávající logickou
operaci \uv{a} se kreslí takhle:
\begin{figure}[ht]
 \centering
 \begin{circuitikz}
  \node[and port] (AND) at (0,0) {};

  \node[ocirc] (i1) at ($(AND.in 1) - (1,0)$) {};
  \node[ocirc] (i2) at ($(AND.in 2) - (1,0)$) {};

  \draw (i1.east) to node[midway,above] {\clb{$i_1$}} (AND.in 1);
  \draw (i2.east) to node[midway,below] {\clb{$i_2$}} (AND.in 2);

  \node[ocirc] (o) at ($(AND.out) + (2,0)$) {};
  \draw (AND.out) to node[midway,above] {$\clr{o} = \clb{i_1} \wedge \clb{i_2}$}
  (o.west);
 \end{circuitikz}
 \caption*{AND brána. Výstupem $\clr{o}$ prochází proud, jedině když jak vstupem
 $\clb{i_1}$, tak $\clb{i_2}$, prochází proud. To odpovídá logické operaci
 \uv{a}. Totiž, když $\clb{i_1} = \clb{i_2} = 1$, pak $\clb{i_1} \wedge \clb{i_2}
 = 1$. Ve všech ostatních případech (tj, $0,1$; $1,0$ a $0,0$) je $\clb{i_1}
 \wedge \clb{i_2} = 0$.}
\end{figure}

Snadný způsob, jak postavit AND bránu je prostě vzít drát se zdrojem a po jeho
délce kamkoli umístit dva transistory. Transistory jsou zařízení, která zesilují
elektrický proud. Tedy, pokud oba transistory budou zapnuté, pak poteče drátem
silný proud, který reprezentuje číslo $1$. Pokud je byť i jeden z nich vypnutý,
bude proud dostatečně slabý, aby byl interpretován jako $0$. Pro jednoduchost si
lze transistory v logických branách představovat tak, že když jsou zapnuté, tak
\uv{pouštějí} proud, jinak ho \uv{zastavují}.

\begin{figure}[ht]
 \centering
 \begin{circuitikz}[scale=0.8,transform shape]
  \draw (0,0) node[npn] (Q) {};
  \draw ($(Q.E) + (0,-0.4)$) node[npn] (P) {};

  \node[ocirc] (i1) at ($(Q.B) - (.5,0)$) {};
  \node[ocirc] (i2) at ($(P.B) - (.5,0)$) {};

  \draw (i1.east) to node[midway,above] {$\clb{i_1}$} (Q.B);
  \draw (i2.east) to node[midway,above] {$\clb{i_2}$} (P.B);
  \node[yshift=-2mm] at (P.E) {$\clr{o}$};
  
 \end{circuitikz}
 \caption*{AND brána jako dvojice transistorů.}
\end{figure}

Další brána, kterou si musíme představit, je NOT brána. Ta odpovídá logické
negaci. Vychází z ní proud, když do ní nepřichází, a naopak z ní nevychází, když
do ní přichází. Kreslí se takhle:

\begin{figure}[ht]
 \centering
 \begin{circuitikz}
  \node[not port] (NOT) at (0,0) {};

  \node[ocirc] (i) at ($(NOT.in) - (1,0)$) {};

  \draw (i.east) to node[midway,above] {\clb{$i_1$}} (NOT.in);

  \node[ocirc] (o) at ($(NOT.out) + (2,0)$) {};
  \draw (NOT.out) to node[midway,above] {$\clr{o} = \neg \clb{i}$}
  (o.west);
 \end{circuitikz}
 \caption*{NOT brána. Výstupem $\clr{o}$ prochází proud, jedině když vstupem
 $\clb{i}$ proud neprochází. Odpovídá logickému \uv{ne}. Totiž, $\neg 0 = 1$ a
 $\neg 1 = 0$.}
\end{figure}

Jednoduchou NOT bránu lze postavit jako drát se zdrojem a jedním rezistorem.
Rezistor si lze v logických obvodech představovat jako zařízení, které
\uv{pouští} proud, když je \textbf{vypnuté} a \uv{nepouští} proud, když je
\textbf{zapnuté}.

Ve skutečnosti, logický obvod vykonávající libovolnou logickou operaci lze
postavit pouze použitím tzv. NAND bran, což jsou jen brány vzniklé spojením AND
a NOT brány. Kreslí se takhle:

\begin{figure}[ht]
 \centering
 \begin{subfigure}[b]{.99\textwidth}
  \centering
  \begin{circuitikz}
   \node[and port] (AND) at (0,0) {};

   \node[ocirc] (i1) at ($(AND.in 1) - (1,0)$) {};
   \node[ocirc] (i2) at ($(AND.in 2) - (1,0)$) {};

   \draw (i1.east) to node[midway,above] {\clb{$i_1$}} (AND.in 1);
   \draw (i2.east) to node[midway,below] {\clb{$i_2$}} (AND.in 2);

   \node[not port] (NOT) at (3,0) {};

   \draw (AND.out) to node[midway,above] {$\clb{i_1} \wedge \clb{i_2}$}
   (NOT.in);
   \node[ocirc] (o) at ($(NOT.out) + (2.3,0)$) {};
   \draw (NOT.out) to node[midway,above] {$\clr{o} = \neg (\clb{i_1} \wedge
   \clb{i_2})$} (o.west);
  \end{circuitikz}
  \caption*{NAND brána jako spojení AND a NOT brány.}
 \end{subfigure}
 \begin{subfigure}[b]{.99\textwidth}
  \centering
  \begin{circuitikz}
   \node[nand port] (NAND) at (0,0) {};

   \node[ocirc] (i1) at ($(NAND.in 1) - (1,0)$) {};
   \node[ocirc] (i2) at ($(NAND.in 2) - (1,0)$) {};

   \draw (i1.east) to node[midway,above] {\clb{$i_1$}} (NAND.in 1);
   \draw (i2.east) to node[midway,below] {\clb{$i_2$}} (NAND.in 2);

   \node[ocirc] (o) at ($(NAND.out) + (2.3,0)$) {};
   \draw (NAND.out) to node[midway,above] {$\clr{o} = \neg (\clb{i_1} \wedge
   \clb{i_2})$}
   (o.west);
  \end{circuitikz}
  \caption*{NAND brána zjednodušeně.}
 \end{subfigure}
 
 \caption*{NAND brána. Odpovídá logickému výrazu $\neg (\clb{i_1} \wedge
 \clb{i_2})$. Tedy, NAND bránou \textbf{neprochází} proud jedině, když oběma
 vstupy $\clb{i_1}$ i $\clb{i_2}$ proud \textbf{prochází}.}
\end{figure}

Jako ukázku toho, že se každý logický obvod dá postavit pouze z NAND bran,
z~nich postavíme OR bránu, tj. logickou bránu implementující logické \uv{nebo}.

\begin{figure}[ht]
 \centering
 \begin{circuitikz}
  \node[nand port] (nand1) at (0,1) {};
  \node[nand port] (nand2) at (0,-1) {};
  \node[circ] (div1) at (-1.5,1) {};
  \node[circ] (div2) at (-1.5,-1) {};

  \node[ocirc] (i1) at (-2.5,1) {};
  \node[ocirc] (i2) at (-2.5,-1) {};

  \draw (i1.east) to node[midway,above] {\clb{$i_1$}} (div1.west);
  \draw (i2.east) to node[midway,above] {\clb{$i_2$}} (div2.west);

  \draw (div1.north) |- (nand1.in 1);
  \draw (div1.south) |- (nand1.in 2);
  \draw (div2.north) |- (nand2.in 1);
  \draw (div2.south) |- (nand2.in 2);

  \node[nand port] (nand3) at (3,0) {};

  \draw (nand1.out) |- (nand3.in 1);
  \draw (nand2.out) |- (nand3.in 2);

  \node[ocirc] (o) at ($(nand3.out) + (2.3,0)$) {};
  \draw (nand3.out) to node[midway,above] {$\clr{o} = \clb{i_1} \vee \clb{i_2}$}
  (o.west);
 \end{circuitikz}
 \caption*{Obvod implementující logickou OR bránu. OR bránou \textbf{neprochází}
 proud jedině, když ani jedním ze vstupu proud rovněž \textbf{neprochází}. Ve
 všech ostatních případech jí proud prochází. Symbol~\textbullet~tady značí
 \emph{rozbočovač}; zkrátka zařízení, které rozpůlí proud, jenž jím prochází.}
\end{figure}

Abychom ověřili, že obvod nahoře je opravdu logické nebo, museli bychom zkusit
za $\clb{i_1}$ a $\clb{i_2}$ dosadit všechny možnosti $0,0; 0,1; 1,0; 1,1$, kde
$0$ (zase) znamená, že vstupem proud neprochází a $1$, že ano.

Zkusíme například $\clb{i_1} = 0$ a $\clb{i_2} = 1$, pro které by měl OR bránou
proud projít. Jednoduchý způsob, jak sledovat průchod proudu logickým obvodem,
je zkrátka kreslit $1$ nebo $0$ nad každý kus drátu. Například takhle:

\begin{figure}[ht]
 \centering
 \begin{circuitikz}
  \node[nand port] (nand1) at (0,1) {};
  \node[nand port] (nand2) at (0,-1) {};
  \node[circ] (div1) at (-1.5,1) {};
  \node[circ] (div2) at (-1.5,-1) {};

  \node[ocirc] (i1) at (-2.5,1) {};
  \node[ocirc] (i2) at (-2.5,-1) {};

  \draw (i1.east) to node[midway,above] {0} (div1.west);
  \draw (i2.east) to node[midway,above] {1} (div2.west);

  \draw (div1.north) |- (nand1.in 1) node[midway,above right,xshift=1mm] {0};
  \draw (div1.south) |- (nand1.in 2) node[midway,below right,xshift=1mm] {0};
  \draw (div2.north) |- (nand2.in 1) node[midway,above right,xshift=1mm] {1};
  \draw (div2.south) |- (nand2.in 2) node[midway,below right,xshift=1mm] {1};

  \node[nand port] (nand3) at (3,0) {};

  \draw (nand1.out) |- (nand3.in 1) node[midway,above right,xshift=3mm] {1};
  \draw (nand2.out) |- (nand3.in 2) node[midway,below right,xshift=3mm] {0};

  \node[ocirc] (o) at ($(nand3.out) + (2.3,0)$) {};
  \draw (nand3.out) to node[midway,above] {1}
  (o.west);
 \end{circuitikz}
 \caption*{Příklad průchodu proudu OR bránou.}
\end{figure}

\section*{Vnitřní paměť}

Vnitřní paměť (tou se myslí paměť, ve které má CPU uloženy instrukce) přichází
ve dvou chutích -- volatilní a nevolatilní (angl. volatile a non-volatile, asi
to překládám blbě...). Volatilní paměť ztrácí veškerá data po vypojení z proudu,
zatímco nevolatilní nikoliv.

Moderní počítače používají jako vnitřní paměť téměř výhradně \textbf{volatilní},
neboť je levnější na výrobu a taky je mnohem kompaktnější. Vlastně jedinou stále
používanou volatilní pamětí je RAM (\textbf{R}andom \textbf{A}ccess
\textbf{M}emory). Výraz \uv{random access} značí, že se jedná o paměť, která umí
zpřístupnit svůj obsah okamžitě, bez ohledu na to, v kterém místě je uložen. To
právě umožňuje operaci \textbf{JUMP} vykonat ihned, stačí znát cílové místo v
paměti (to není ten případ u externích pamětí, které je potřeba obvykle
procházet skoro od začátku).

RAM se používá ve dvou formách:
\begin{itemize}
 \item SRAM nebo (\textbf{S}tatic RAM) je typ volatilní paměti, ve které se data
  za běhu neposouvají, tj. za neustálého přísunu proudu je schopna udržet
  uložené údaje teoreticky nekonečně dlouho. Používá se hlavně pro vnitřní paměť
  samotného CPU (tzv. cache), protože je velmi rychlá díky tomu, že nepotřebuje
  žádnou správu -- uložená data jsou vždy tam, kam byla uložena.
 \item DRAM nebo (\textbf{D}ynamic RAM) je ten druhý typ RAMky, kde se data s
  postupem času ztrácejí (i když do ní stále jde proud). Proto je v ní třeba
  důležitá data (třeba ta nutná pro běh OS) neustále obnovovat. Tato potřeba se
  u DRAMek, jejichž velikost se měří v GB, stala natolik častou, že se pro tyto
  účely začaly vyrábět CPU s jednotkou správy paměti (Memory Management Unit).
  DRAM je nejvíc používaná vnitřní paměť.
\end{itemize}

Z nevolatilních typů pamětí si zmíníme dva:
\begin{itemize}
 \item NVRAM (neboli \textbf{N}on-\textbf{V}olatile RAM) je přesně to, co si
  myslíte, že je. Je to RAMka (buď statická nebo dynamická), která uchovává data
  i po odpojení od proudu. Jako vnitřní paměť počítače se nepoužívá hlavně z
  toho důvodu, že potřebuje baterii, která se tudíž musí časem měnit, a obrovská
  zátěž, kterou CPU klade na RAM, by ji vyčerpala v řádu dní. Největší využití
  zaznamenala jako úložiště BIOSu (jakoby operačního systému základní desky) a
  jako \emph{flash} paměť v SSD discích.
 \item ROM (\textbf{R}ead-\textbf{O}nly \textbf{M}emory) je typ paměti, do které
  nelze ukládat data, ale pouze je z ní číst. Používá se hlavně jako úložiště
  firmwaru v různých typech externích zařízení (myši, klávesnice, tiskárny,
  ...). Firmware je vlastně takový malý operační systém, přes který posílá dané
  zařízení signály do CPU připojeného počítače.
\end{itemize}

Primárním smyslem RAM je být úložištěm pro CPU instrukce. Představíme-li si RAM
jako jeden dlouhý sloupec, kde v řádcích jsou postupně instrukce, může jeden
uložený počítačový program vypadat třeba takhle:

\begin{figure}[ht]
 \centering
 \begin{tikzpicture}
  \node[draw,rectangle,minimum width=2cm,minimum height=6mm] at (0,0) {JUMP};
  \node at (-1.5,0) {\clr{0010}};
  \node[draw,rectangle,minimum width=2cm,minimum height=6mm] at (0,-0.6) {1001};
  \node at (0,-1.2) {$\vdots$};
  \node at (-1.5,-0.6) {\clr{0011}};
  
  \node at (4,0.2) {$\vdots$};
  \node at (2.5,-0.6) {\clr{1001}};
  \node at (2.5,-1.2) {\clr{1010}};
  \node at (2.5,-1.8) {\clr{1011}};
  \node at (2.5,-2.4) {\clr{1100}};
  \node[draw,rectangle,minimum width=2cm,minimum height=6mm] at (4,-0.6) {ADD};
  \draw[->] (1,-0.6) -- (2,-0.6);
  \node[draw,rectangle,minimum width=2cm,minimum height=6mm] at (4,-1.2) {0100};
  \node[draw,rectangle,minimum width=2cm,minimum height=6mm] at (4,-1.8) {1010};
  \node[draw,rectangle,minimum width=2cm,minimum height=6mm] at (4,-2.4) {STORE};
  \node at (4,-3.1) {$\vdots$};
 \end{tikzpicture}
 \caption*{Příklad programu v RAM. \clr{Červená čísla} představují \emph{adresy}
  v paměti -- prostě jenom říkají, kolikátá (v dvojkové soustavě) pozice v
  paměti to je. Po instrukci JUMP přečte CPU další řádek, kde je uloženo, na
  které místo v paměti má skočit. Tam najde instrukci ADD, po které očekává dvě
  čísla, která má sečíst. Nakonec narazí na instrukci STORE a výsledek
  předchozího součtu na následující místo v~paměti uloží.}
\end{figure}

\section*{Základní deska}

Základní deska (angl. motherboard) je zase jen elektrický obvod, který spojuje
všechny komponenty počítače. Obvykle se dělí na dva kusy -- severní (North
Bri\-dge) a jižní (South Bridge).
\begin{itemize}
 \item North Bridge obsahuje sloty pro CPU a RAM a často taky porty pro
  propojení s I/O zařízeními (myšmi, klávesnicemi, tiskárnami, monitorem,...).
 \item South Bridge má sloty pro externí paměti a grafické/zvukové karty a také
  je v něm umístěna NVRAM s BIOSem a její baterkou. BIOS (\textbf{B}asic
  \textbf{I}nput/\textbf{O}ut\-put \textbf{S}ystem) je software zajišťující přenos
  dat mezi komponenty a CPU.
\end{itemize}

\section*{GPU}

GPU (\textbf{G}raphics \textbf{P}rocessing \textbf{U}nit) je vlastně CPU ořezané
o většinu svých funkcí. Primárním účelem GPU je provádět co nejvíce výpočtů
najednou co největší rychlostí. CPU mívá až 32 výpočetních \emph{jader}
(jednotek schopných provádět výpočty najednou během jednoho úderu vnitřních
hodin); moderní GPU mají výpočetních jader tisíce. Často umožňují posílat výstup
přímo do monitoru a mají v sobě též vlastní RAMku, kam ukládají mezivýsledky,
které není potřeba posílat zpět do CPU.

\end{document}
