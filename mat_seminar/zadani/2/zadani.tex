\documentclass[a4paper,12pt]{article}

\usepackage[czech,english]{babel}
% Fonts %
\usepackage{fouriernc}
\usepackage[T1]{fontenc}

% Colors %
\usepackage[dvipsnames]{color}
\usepackage{xcolor}

% Page Layout %
\usepackage[margin=1in]{geometry}

% Fancy Headers %
\usepackage{fancyhdr}
\fancyhf{}
\cfoot{\thepage}
\rhead{}
\renewcommand{\headrulewidth}{0pt}
\setlength{\headheight}{16pt}

% Math
\usepackage{mathtools}
\usepackage{amssymb}
\usepackage{faktor}
\usepackage{import}
\usepackage{caption}
\usepackage{subcaption}
\usepackage{wrapfig}
\usepackage{enumitem}
\usepackage{tikz}
\usetikzlibrary{cd,positioning,babel,shapes}
\usepackage{tkz-base}
\usepackage{tkz-euclide}
\usepackage{circuitikz}
\ctikzset{
 logic ports=ieee,
 logic ports/scale=0.7,
}


% Theorems
\usepackage{amsthm}
\usepackage{thmtools}

% Title %
\title{\Huge\textsf{Linear Equations}\\
 \Large\textsf{Scales, Lines \& Functions}
 \author{Áďa Klepáčů}
 \date{\today}
}

% Table of Contents %
\usepackage{hyperref}
\hypersetup{
 colorlinks=true,
 linktoc=all,
 linkcolor=blue
}

% Tables %
\usepackage{booktabs}
\usepackage{tabularx}

% Patch for hyphens
\usepackage{regexpatch}
\makeatletter
% Change the `-` delimiter to an active character
\xpatchparametertext\@@@cmidrule{-}{\cA-}{}{}
\xpatchparametertext\@cline{-}{\cA-}{}{}
\makeatother

\newcolumntype{s}{>{\centering\arraybackslash}p{.4\textwidth}}

% Operators %
\DeclareMathOperator{\Ker}{Ker}
\DeclareMathOperator{\Img}{Im}
\DeclareMathOperator{\End}{End}
\DeclareMathOperator{\Aut}{Aut}
\DeclareMathOperator{\Inn}{Inn}

% Common operators %
\newcommand{\R}{\mathbb{R}}
\newcommand{\N}{\mathbb{N}}
\newcommand{\Z}{\mathbb{Z}}
\newcommand{\Q}{\mathbb{Q}}
\newcommand{\C}{\mathbb{C}}

\newcommand{\tr}{\textcolor{red}}
\newcommand{\tb}{\textcolor{blue}}
\newcommand{\tg}{\textcolor{green}}
\newcommand{\tm}{\textcolor{magenta}}
\newcommand{\tv}{\textcolor{violet}}

% American Paragraph Skip %
\setlength{\parindent}{0pt}
\setlength{\parskip}{1em}

% Document %
\pagestyle{fancy}
\begin{document}

\thispagestyle{fancy}

\section*{Hardware -- úloha 2}

Určete výstupy (všechny) nakresleného logického obvodu pro uvedené vstupy.
Postup řešení musíte být schopni řádně vysvětlit.

\begin{center}
 \begin{circuitikz}
  \node[circle,draw,thick,inner sep=2pt] (i1) at (0, 2) {};
  \node[left=1mm of i1] {$\tb{0}$};

  \node[circle,draw,thick,inner sep=2pt] (i2) at (0, 0) {};
  \node[left=1mm of i2] {$\tb{1}$};

  \node[circle,draw,thick,inner sep=2pt] (i3) at (0, -2) {};
  \node[left=1mm of i3] {$\tb{0}$};

  \node[and port,anchor=in 1] (AND1) at (3, 2) {};
  \node[nand port,anchor=in 2] (NAND1) at (3, -2) {};
  
  \draw (i1) -- (AND1.in 1);
  \draw (i3) -- (NAND1.in 2);
  
  \node[fill=black,draw,inner sep=3pt] (s1) at (2,0) {};
  \draw (i2) -- (s1);
  \draw (s1) |- (AND1.in 2);
  \draw (s1) |- (NAND1.in 1);

  \node[not port,anchor=in] (NOT1) at (3,0) {};
  \draw (s1) -- (NOT1.in);
  
  \node[nand port,anchor=out] (NAND2) at (6.5,1) {};
  \draw (AND1.out) |- (NAND2.in 1);
  \draw (NOT1.out) |- (NAND2.in 2);
  
  \node[and port,anchor=in 2] (AND2) at (9, -1.8) {};
  \node[fill=black,draw,inner sep=3pt] (s2) at (8,1) {};
  
  \draw (NAND2.out) -- (s2);
  \draw (s2) |- (AND2.in 1);
  \draw (NAND1.out) -- (AND2.in 2);
  
  \node[circle,draw,thick,inner sep=2pt] (o1) at (12, 1) {};
  \node[right=1mm of o1] {$\tr{?}$};

  \node[circle,draw,thick,inner sep=2pt] (o2) at (12, -1.6) {};
  \node[right=1mm of o2] {$\tr{?}$};

  \draw (s2) -- (o1);
  \draw (AND2.out) -- (o2);
 \end{circuitikz}
\end{center}

\begin{center}
 \begin{circuitikz}
  \node[and port] (AND) at (0,0) {};
  \node[right=1mm of AND] {$=$};
  \node[right=6mm of AND] {AND};

  \node[not port] (NOT) at (4,0) {};
  \node[right=1mm of NOT] {$=$};
  \node[right=6mm of NOT] {NOT};

  \node[nand port] (NAND) at (8,0) {};
  \node[right=1mm of NAND] {$=$};
  \node[right=6mm of NAND] {NAND};
 \end{circuitikz}
\end{center}
\end{document}
