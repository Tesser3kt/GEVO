\documentclass[a4paper,11pt]{article}

% Colors %
\usepackage[dvipsnames]{xcolor}

% Page Layout %
\usepackage[margin=1.5in]{geometry}

% Fancy Headers %
\usepackage{fancyhdr}
\fancyhf{}
\cfoot{\thepage}
\rhead{}
\renewcommand{\headrulewidth}{0pt}
\setlength{\headheight}{16pt}

% Math
\usepackage{mathtools}
\usepackage{amssymb}
\usepackage{faktor}
\usepackage{import}
\usepackage{caption}
\usepackage{subcaption}
\usepackage{wrapfig}
\usepackage{transparent}
\usepackage{framed}

% Theorems
\usepackage{amsthm}
\usepackage{thmtools}

% Title %
\title{\Huge\textsf{}\\
 \Large\textsf{}
 \author{}
 \date{}
}

% Table of Contents %
\usepackage{hyperref}
\hypersetup{
 colorlinks=true,
 linktoc=all,
 linkcolor=blue
}

% Tables %
\usepackage{booktabs}

% Enumerate %
\usepackage{enumerate}

% Operators %
\DeclareMathOperator{\Ker}{Ker}
\DeclareMathOperator{\Img}{Im}
\DeclareMathOperator{\End}{End}
\DeclareMathOperator{\Aut}{Aut}
\DeclareMathOperator{\Inn}{Inn}

% Common operators %
\newcommand{\R}{\mathbb{R}}
\newcommand{\N}{\mathbb{N}}
\newcommand{\Z}{\mathbb{Z}}
\newcommand{\Q}{\mathbb{Q}}
\newcommand{\C}{\mathbb{C}}

% American Paragraph Skip %
\setlength{\parindent}{0pt}
\setlength{\parskip}{1em}

% Document %
\pagestyle{fancy}
\begin{document}

\begin{figure}[t]
 \centering
 \def\svgwidth{\textwidth}
 \import{figs/}{triangle.pdf_tex}
 \caption*{Rozkouskouvaný trojúhelník ABC.}
\end{figure}

Ty body dotyku kružnice s trojúhelníkem mi ho rozdělí na tři páry shodných
pravoúhlých trojúhelníků. Protože $a = 6$ a $b = 5$, délky všech kousků stran
můžu vyjádřit pomocí jedné neznámé -- $x$. Stačí mi, když spočítám $x$, protože
strana $c$ měří $(5 - x) + (6 - x) = 11 - 2x$.

Použiju dvě goniometrické identity:
\begin{framed}
 \begin{align*}
  \cot(\pi / 2 - \theta) &= \tan\theta\\
  \tan(\theta + \omega) &= \frac{\cot\theta + \cot\omega}{\cot\theta\cot\omega -
  1}.
 \end{align*}
\end{framed}

Půlky úhlů $\alpha, \beta, \gamma$ si označím $\overline{\alpha} \coloneqq
\alpha / 2$, $\overline{\beta} \coloneqq \beta / 2$ a $\overline{\gamma}
\coloneqq \gamma / 2$. Pak
\[
 \cot \overline{\alpha} = \frac{5-x}{r} = \frac{5-x}{1.5}, \quad 
 \cot \overline{\beta} = \frac{6-x}{1.5}, \quad 
 \cot \overline{\gamma} = \frac{x}{1.5}. \\
\]
Máme $\alpha + \beta + \gamma = \pi$, takže $\overline{\alpha} +
\overline{\beta} + \overline{\gamma} = \pi / 2$. Dál,
\[
 \cot \overline{\gamma} = \cot(\pi / 2 - (\overline{\alpha} + \overline{\beta}))
 = \tan(\overline{\alpha} + \overline{\beta}).
\]
Potom,
\[
 \cot \overline{\gamma} = \tan(\overline{\alpha} + \overline{\beta}) =
 \frac{\cot \overline{\alpha} + \cot \overline{\beta}}{\cot \overline{\alpha}
 \cot \overline{\beta} - 1}.
\]
Položíme
\[
 y \coloneqq \cot \overline{\alpha}, \quad z \coloneqq \cot \overline{\beta}.
\]
Takže dostaneme soustavu
\begin{align*}
 y &= \frac{5 - x}{1.5}, \\
 z &= \frac{6 - x}{1.5}, \\
 \frac{y + z}{yz - 1} &= \frac{x}{1.5}.
\end{align*}
Dosadíme za $y$ a za $z$ do třetí rovnice.
\[
 1.5\frac{\frac{5-x}{1.5} + \frac{6-x}{1.5}}{\frac{5-x}{1.5}\frac{6-x}{1.5} -
 1} = x.
\]
Odtud
\[
 \frac{(1.5)^2(11 - 2x)}{(5 - x)(6 - x) - 2.25} = x.
\]
Po zkrášlení
\[
 99 - 18x = x(4x^2 - 44x + 111).
\]
To dá rovnici
\[
 4x^3 - 44x^2 + 129x - 99 = 0.
\]
Ta má řešení (podle Pythonu) $3$, $4 - \sqrt{31} / 2$ a $4 + \sqrt{31} / 2$.
Protože $c = 11 - 2x$, dostaneme, že $c$ je $5$, $3 + \sqrt{31}$ nebo $3 -
\sqrt{31}$.


\end{document}
