\documentclass[a4paper,11pt]{article}

\usepackage[czech,english]{babel}
% Fonts %
\usepackage{fouriernc}
\usepackage[T1]{fontenc}

% Colors %
\usepackage[dvipsnames]{color}
\usepackage[dvipsnames]{xcolor}

% Page Layout %
\usepackage[margin=1.5in]{geometry}

% Fancy Headers %
\usepackage{fancyhdr}
\fancyhf{}
\cfoot{\thepage}
\rhead{}
\renewcommand{\headrulewidth}{0pt}
\setlength{\headheight}{16pt}

% Math
\usepackage{mathtools}
\usepackage{amssymb}
\usepackage{faktor}
\usepackage{import}
\usepackage{caption}
\usepackage{subcaption}
\usepackage{wrapfig}
\usepackage{enumitem}
\setlist{topsep=0pt}

\usepackage{tikz}
\usetikzlibrary{cd,positioning,babel,shapes,calc}
\usepackage{tkz-base}
\usepackage{tkz-euclide}

% Theorems
\usepackage[thmmarks, amsmath, thref]{ntheorem}
\usepackage{thmtools}

\theoremsymbol{\ensuremath{\blacksquare}}
\newtheorem*{solution}{Possible solution.}

% Title %
\title{\Huge\textsf{Homework -- PreIB 3.AB 3 \& 4}\\
 \Large\textsf{Structures and Operations}
 \author{Áďa Klepáčů}
 \date{\today}
}

% Table of Contents %
\usepackage{hyperref}
\hypersetup{
 colorlinks=true,
 linktoc=all,
 linkcolor=blue
}

% Tables %
\usepackage{booktabs}
\usepackage{tabularx}

% Patch for hyphens
\usepackage{regexpatch}
\makeatletter
% Change the `-` delimiter to an active character
\xpatchparametertext\@@@cmidrule{-}{\cA-}{}{}
\xpatchparametertext\@cline{-}{\cA-}{}{}
\makeatother

\newcolumntype{s}{>{\centering\arraybackslash}p{.4\textwidth}}

% Operators %
\DeclareMathOperator{\Ker}{Ker}
\DeclareMathOperator{\Img}{Im}
\DeclareMathOperator{\End}{End}
\DeclareMathOperator{\Aut}{Aut}
\DeclareMathOperator{\Inn}{Inn}

% Common operators %
\newcommand{\R}{\mathbb{R}}
\newcommand{\N}{\mathbb{N}}
\newcommand{\Z}{\mathbb{Z}}
\newcommand{\Q}{\mathbb{Q}}
\newcommand{\C}{\mathbb{C}}

\newcommand{\clr}{\textcolor{red}}
\newcommand{\clb}{\textcolor{blue}}
\newcommand{\clg}{\textcolor{green}}
\newcommand{\clm}{\textcolor{magenta}}
\newcommand{\clv}{\textcolor{violet}}
\newcommand{\clbr}{\textcolor{Sepia}}

% American Paragraph Skip %
\setlength{\parindent}{0pt}
\setlength{\parskip}{1em}

% Document %
\pagestyle{fancy}
\begin{document}

\maketitle
\thispagestyle{fancy}

\begin{center}
 \hrule
 \textbf{\clr{DON'T FORGET TO EXPLAIN EVERYTHING EVEN IF YOU THINK IT'S
 OBVIOUS!}}
 \vspace{2ex}
 \hrule
\end{center}
 
\section*{Natural Numbers}

\textbf{Exponentiation} of two natural numbers $n,m \in \N$ is defined by the
following two formulae:
\begin{itemize}
 \item $n^{0} = 1$,
 \item $n^{s(m)} = n^{m} \cdot n$.
\end{itemize}
Explain \textbf{very clearly} how to calculate $n^{m}$ using \textbf{only} the
two rules above.\\
\textbf{Hint}: This process is very similar to the definition of \emph{addition}
and \emph{multiplication} on natural numbers.

Answer the following questions:
\begin{enumerate}
 \item Is exponentiation \emph{commutative}, that is, is it true that $n^{m} =
  m^{n}$ for all pairs of natural numbers $n,m \in \N$? If yes, explain why. If
  not, provide a counterexample.
 \item Is exponentiation \emph{associative}, that is, is it true that
  $n^{(m^{k})} = (n^{m})^{k}$ for all triples of natural numbers $n,m,k \in \N$?
  If yes, explain why. If not, provide a counterexample.
 \item Is exponentiation an \emph{operation} (by definition) on natural numbers?
  Explain.
\end{enumerate}

\clearpage
\section*{Operations}

On the set $X = \{a,b,c,d\}$ there are two operations given by the following
picture.

\begin{figure}[ht]
 \centering
 \begin{subfigure}[b]{.48\textwidth}
  \centering
  \begin{tikzpicture}[scale=1.5]
   \tkzDefPoints{0/0/a,1/0/b,1/1/c,0/1/d}
   \tkzDrawPoints[size=4](a,b,c,d)
   \tkzLabelPoint[below left](a){$a$}
   \tkzLabelPoint[below right](b){$b$}
   \tkzLabelPoint[above right](c){$c$}
   \tkzLabelPoint[above left](d){$d$}

   \tkzDefPoint(-0.15,-0.15){o}
   \tkzDrawArc[R,->,thick,color=RoyalBlue](o,0.2)(70,380)
   \tkzDrawLine[->,thick,color=RoyalBlue,add=-.1 and -.1](d,b)
   \tkzDrawLine[->,thick,color=RoyalBlue,add=-.1 and -.1](c,d)
   \tkzDrawLine[->,thick,color=RoyalBlue,add=-.1 and -.1](b,a)
  \end{tikzpicture}
  \caption{First operation.}
 \end{subfigure}
 \begin{subfigure}[b]{.48\textwidth}
  \centering
  \begin{tikzpicture}[scale=1.5]
   \tkzDefPoints{0/0/a,1/0/b,1/1/c,0/1/d}
   \tkzDrawPoints[size=4](a,b,c,d)
   \tkzLabelPoint[below left](a){$a$}
   \tkzLabelPoint[below right](b){$b$}
   \tkzLabelPoint[above right](c){$c$}
   \tkzLabelPoint[above left](d){$d$}

   \tkzDefPoint(-0.15,-0.15){o}
   \tkzDrawArc[R,->,thick,color=RoyalBlue](o,0.2)(70,380)
   \tkzDrawLine[->,thick,color=RoyalBlue,add=-.1 and -.1](d,b)
   \tkzDrawLine[->,thick,color=RoyalBlue,add=-.1 and -.1](c,d)
   \tkzDrawLine[->,thick,color=RoyalBlue,add=-.1 and -.1](b,a)
  \end{tikzpicture}
  \caption{Second operation.}
 \end{subfigure}
\end{figure}

\end{document}
