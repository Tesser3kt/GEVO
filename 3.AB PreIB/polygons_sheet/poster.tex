% Unofficial University of Cambridge Poster Template
% https://github.com/andiac/gemini-cam
% a fork of https://github.com/anishathalye/gemini
% also refer to https://github.com/k4rtik/uchicago-poster

\documentclass[final]{beamer}

% ====================
% Packages
% ====================

\usepackage[T1]{fontenc}
\usepackage{lmodern}
\usepackage[orientation=portrait,size=a2,scale=1.15]{beamerposter}
\usetheme{gemini}
\usepackage[dvipsnames]{xcolor}
\usecolortheme{nott}
\usepackage{graphicx}
\usepackage{booktabs}
\usepackage{tikz}
\usetikzlibrary{patterns,decorations.pathmorphing}
\usepackage{tkz-euclide}
\tikzset{point style/.style = {%
  draw = black,
  inner sep = 0pt,
  shape = circle,
  minimum size = 5pt,
  fill = black
 }
}
\usepackage{pgfplots}
\pgfplotsset{compat=1.14}
\usepackage{anyfontsize}
\usepackage{caption}
\usepackage{subcaption}

% ====================
% Lengths
% ====================

% If you have N columns, choose \sepwidth and \colwidth such that
% (N+1)*\sepwidth + N*\colwidth = \paperwidth
\newlength{\sepwidth}
\newlength{\colwidth}
\setlength{\sepwidth}{0.025\paperwidth}
\setlength{\colwidth}{0.45\paperwidth}

\newcommand{\separatorcolumn}{\begin{column}{\sepwidth}\end{column}}
\newcommand{\bfalert}[1]{\textbf{\alert{#1}}}

% ====================
% Title
% ====================

\title{Polygons \& Transformations Cheatsheet}

\author{3.AB PreIB Math}

\institute[shortinst]{Adam Klepáč}

% ====================
% Footer (optional)
% ====================

% \footercontent{
%   \href{https://utfpr.edu.br/ct/ppgca}{utfpr.edu.br/ct/ppgca} \hfill
%   Mostra de Trabalhos do PPGCA --- TechTalks 2024 \hfill
%   \href{mailto:ppgca-ct@utfpr.edu.br}{ppgca-ct@utfpr.edu.br}}
% (can be left out to remove footer)


% ====================
% Logo (optional)
% ====================

% use this to include logos on the left and/or right side of the header:
\logoright{\includegraphics[height=2.5cm]{logos/logo-white.png}}
% \logoleft{\hspace{20ex}\includegraphics[height=3.5cm]{logos/ppgca-logo.png}}

% ====================
% Body
% ====================

\begin{document}

% Refer to https://github.com/k4rtik/uchicago-poster
% logo: https://www.cam.ac.uk/brand-resources/about-the-logo/logo-downloads
% \addtobeamertemplate{headline}{}
% {
%     \begin{tikzpicture}[remember picture,overlay]
%       \node [anchor=north west, inner sep=3cm] at ([xshift=-2.5cm,yshift=1.75cm]current page.north west)
%       {\includegraphics[height=7cm]{logos/unott-logo.eps}}; 
%     \end{tikzpicture}
% }

\begin{frame}[t]
\begin{columns}[t]
\separatorcolumn

\begin{column}{\colwidth}

  \begin{block}{Polygons}

   Polygon is a \bfalert{closed} 2D shape \bfalert{made only of segments}. We
   call the endpoints of those segments, \bfalert{vertices}, and the segments
   themselves, \bfalert{edges}.

   \textbf{Examples}
   \begin{figure}[H]
    \centering
    \begin{subfigure}[b]{.23\textwidth}
     \centering
     \begin{tikzpicture}
      \tkzDefPoint(0:1){A}
      \tkzDefPoint(110:2){B}
      \tkzDefPoint(180:1){C}

      \tkzDrawPoints(A,B,C);
      \tkzDrawSegments[thick](A,B B,C C,A);
     \end{tikzpicture}
     \caption*{Triangle}
    \end{subfigure}
    \begin{subfigure}[b]{.23\textwidth}
     \centering
     \begin{tikzpicture}
      \tkzDefPoint(0:1){A}
      \tkzDefPoint(60:1.5){B}
      \tkzDefPoint(130:2.5){C}
      \tkzDefPoint(180:1){D}

      \tkzDrawPoints(A,B,C,D);
      \tkzDrawSegments[thick](A,B B,C C,D D,A);
     \end{tikzpicture}
     \caption*{Quadrilateral}
    \end{subfigure}
    \begin{subfigure}[b]{.23\textwidth}
     \centering
     \begin{tikzpicture}
      \tkzDefPoint(0:1){A}
      \tkzDefPoint(50:2){B}
      \tkzDefPoint(90:2){C}
      \tkzDefPoint(120:2){D}
      \tkzDefPoint(180:1){E}

      \tkzDrawPoints(A,B,C,D,E);
      \tkzDrawSegments[thick](A,B B,C C,D D,E E,A);
     \end{tikzpicture}
     \caption*{Pentagon}
    \end{subfigure}
    \begin{subfigure}[b]{.27\textwidth}
     \centering
     \begin{tikzpicture}
      \tkzDefPoint(0:1){A}
      \tkzDefPoint(13:2){B}
      \tkzDefPoint(30:2.5){C}
      \tkzDefPoint(50:2){D}
      \tkzDefPoint(70:1){E}
      \tkzDefPoint(86:2){F}
      \tkzDefPoint(97:1.5){G}
      \tkzDefPoint(109:2){H}
      \tkzDefPoint(121:2){I}
      \tkzDefPoint(140:1.5){J}
      \tkzDefPoint(160:1){K}
      \tkzDefPoint(180:1){L}

      \tkzDrawPoints(A,B,C,D,E,F,G,H,I,J,K,L);
      \tkzDrawSegments[thick](A,B B,C C,D D,E E,F F,G G,H H,I I,J J,K K,L L,A);
     \end{tikzpicture}
     \caption*{Dodecagon}
    \end{subfigure}
   \end{figure}
   Polygons with $n$ sides are called \bfalert{$n$-gons}.

   \textbf{Counterexamples}
   \begin{figure}[H]
    \centering
    \begin{subfigure}[b]{.32\textwidth}
     \centering
     \begin{tikzpicture}
      \tkzDefPoint(0:1){A}
      \tkzDefPoint(90:1){B}
      \tkzDefPoint(150:1){C}

      \tkzDrawPoints(A,B,C);
      \tkzDrawSegments[thick](A,B B,C);
     \end{tikzpicture}
     \caption*{Not closed}
    \end{subfigure}
    \begin{subfigure}[b]{.32\textwidth}
     \centering
     \begin{tikzpicture}[scale=2]
      \tkzDefPoint(0,0){A1}
      \tkzDefPoint(0,1){B1}
      \tkzDefPoint(1,1){C1}
      \tkzDefPoint(1,0){D1}

      \tkzDefPoint(0.5,0.3){A2}
      \tkzDefPoint(0.5,1.3){B2}
      \tkzDefPoint(1.5,1.3){C2}
      \tkzDefPoint(1.5,0.3){D2}
      \tkzDrawPoints(A1,B1,C1,D1,A2,B2,C2,D2)

      \tkzDrawSegments[thick](A1,B1 B1,C1 C1,D1 D1,A1 B2,C2 C2,D2 B1,B2 C1,C2 D1,D2)
      \tkzDrawSegments[dashed,thick](A2,B2 D2,A2 A1,A2)
     \end{tikzpicture}
     \caption*{3D}
    \end{subfigure}
    \begin{subfigure}[b]{.32\textwidth}
     \centering
     \begin{tikzpicture}
      \tkzDefPoint(0:1){A}
      \tkzDefPoint(180:1){B}
      \tkzDefPoint(90:1){P}

      \tkzDrawPoints(A,B)
      \tkzDrawArc[thick](P,A)(B)
      \tkzDrawSegment[thick](A,B)
     \end{tikzpicture}
     \caption*{Not straight}
    \end{subfigure}
   \end{figure}
  \end{block}

  \begin{block}{Convex Polygons}

   A polygon is called \bfalert{convex} if it has no internal angle greater than
   180$^{\circ}$.
   \begin{figure}[H]
    \centering
    \begin{subfigure}[b]{.35\textwidth}
     \centering
     \begin{tikzpicture}[scale=1.5]
      \tkzDefPoint(0:1){A}
      \tkzDefPoint(50:2){B}
      \tkzDefPoint(90:2){C}
      \tkzDefPoint(120:2){D}
      \tkzDefPoint(180:1){E}

      \tkzDrawPoints[color=gevoblue](A,B,C,D,E);
      \tkzDrawSegments[color=gevoblue,thick](A,B B,C C,D D,E E,A);
     \end{tikzpicture}
     \caption*{\textcolor{gevoblue}{Convex}}
    \end{subfigure}
    \begin{subfigure}[b]{.35\textwidth}
     \centering
     \begin{tikzpicture}[scale=1.5]
      \tkzDefPoint(0,0){A}
      \tkzDefPoint(2,0){B}
      \tkzDefPoint(2,2){C}
      \tkzDefPoint(1,1){D}
      \tkzDefPoint(0,2){E}

      \tkzDrawPoints[color=gevored](A,B,C,D,E)
      \tkzDrawSegments[color=gevored,thick](A,B B,C C,D D,E E,A)

      \tkzMarkAngle[color=gevored,size=0.65,thick](E,D,C)
      \tkzLabelAngle[color=gevored,pos=0.25](E,D,C){\footnotesize $>\!\!180^{\circ}$}
     \end{tikzpicture}
     \caption*{\textcolor{gevored}{NOT convex}}
    \end{subfigure}
   \end{figure}

   \textbf{Special types of convex polygons}
   \begin{figure}[H]
    \centering
    \begin{subfigure}[b]{.3\textwidth}
     \begin{tikzpicture}[scale=2]
      \tkzDefPoint(0,0){A}
      \tkzDefPoint(0.3,1){B}
      \tkzDefPoint(1,1){C}
      \tkzDefPoint(2,0){D}

      \tkzDrawPolygon[thick](A,B,C,D)
      \tkzDrawSegments[ultra thick,gevodarkblue](B,C A,D)
      \tkzDrawPoints(A,B,C,D)
     \end{tikzpicture}
     \caption*{\alert{Trapezoid/Trapezium}\\ 
      A convex quadrilateral with at least two parallel sides.}
    \end{subfigure}
    \begin{subfigure}[b]{.3\textwidth}
     \begin{tikzpicture}[scale=2]
      \tkzDefPoint(0,0){A}
      \tkzDefPoint(0.5,0.75){B}
      \tkzDefPoint(2,0.75){C}
      \tkzDefPoint(1.5,0){D}

      \tkzDrawSegments[ultra thick,gevodarkblue](B,C A,D)
      \tkzDrawSegments[ultra thick,gevodarkred](A,B C,D)
      \tkzDrawPoints(A,B,C,D)
     \end{tikzpicture}
     \caption*{\alert{Parallelogram}\\
      A convex quadrilateral with two pairs of parallel sides.}
    \end{subfigure}
    \begin{subfigure}[b]{.3\textwidth}
     \begin{tikzpicture}[scale=2]
      \clip (-0.04,-0.04) rectangle (1.44,0.79);
      \tkzDefPoint(0,0){A}
      \tkzDefPoint(0.5,0.75){B}
      \tkzDefPoint(1.4,0.75){C}
      \tkzDefPoint(0.9,0){D}

      \tkzDrawPolygon[ultra thick,gevodarkblue](A,B,C,D)
      \tkzDrawPoints(A,B,C,D)
      \tkzDrawSegments[thick](A,C B,D)
      \tkzInterLL(A,C)(B,D) \tkzGetPoint{O}
      \tkzMarkAngle[size=0.2](A,O,D)
      \tkzLabelAngle[pos=0.1](A,O,D){$ \cdot $}
     \end{tikzpicture}
     \caption*{\alert{Rhombus}\\
     An \textbf{equilateral} (all sides of the same length) parallelogram.}
    \end{subfigure}
   \end{figure}
  \end{block}

  \begin{exampleblock}{Diagonals \& Triangulations}

   A \bfalert{diagonal} in a \textbf{convex} polygon is a segment connecting two
   of its \textbf{non-adjacent} vertices.
   \begin{figure}[H]
    \centering
    \begin{tikzpicture}
     \tkzDefPoint(0,0){A}
     \tkzDefPoint(-1,0.5){B}
     \tkzDefPoint(0.3,2){C}
     \tkzDefPoint(2,1){D}
     \tkzDefPoint(2,0.5){E}
     \tkzDefPoint(1.5,0){F}
     \tkzDrawPolygon(A,B,C,D,E,F)
     \tkzDrawSegment[color=gevored,ultra thick](A,D)
     \tkzDrawPoints(A,B,C,D,E,F)
    \end{tikzpicture}
    \caption*{\textcolor{gevored}{Diagonal} in a convex hexagon.}
   \end{figure}

   A \bfalert{triangulation} of a \textbf{convex} polygon is its division into
   triangles by \textbf{non-intersecting} diagonals.

   \begin{figure}[H]
    \centering
    \begin{subfigure}[b]{.2\textwidth}
     \centering
     \begin{tikzpicture}[scale=0.875]
      \tkzDefPoint(-45:1){A}
      \tkzDefPoint(-135:1){B}
      \tkzDefPoint(135:1){C}
      \tkzDefPoint(45:1){D}

      \tkzDrawPolygon(A,B,C,D)
      \tkzDrawSegment[ultra thick,gevored](A,C)
      \tkzDrawPoints(A,B,C,D)
     \end{tikzpicture}
    \end{subfigure}
    \begin{subfigure}[b]{.2\textwidth}
     \centering
     \begin{tikzpicture}[scale=0.7]
      \tkzDefPoint(0:1){A}
      \tkzDefPoint(60:1){B}
      \tkzDefPoint(120:1){C}
      \tkzDefPoint(180:1){D}
      \tkzDefPoint(240:1){E}
      \tkzDefPoint(300:1){F}

      \tkzDrawPolygon(A,B,C,D,E,F)
      \tkzDrawSegments[ultra thick,color=gevored](A,C D,F A,D)
      \tkzDrawPoints(A,B,C,D,E,F)
     \end{tikzpicture}
    \end{subfigure}
    \begin{subfigure}[b]{.2\textwidth}
     \centering
     \begin{tikzpicture}[scale=0.7]
      \tkzDefPoint(0:1){A}
      \tkzDefPoint(60:1){B}
      \tkzDefPoint(120:1){C}
      \tkzDefPoint(180:1){D}
      \tkzDefPoint(240:1){E}
      \tkzDefPoint(300:1){F}

      \tkzDrawPolygon(A,B,C,D,E,F)
      \tkzDrawSegments[ultra thick,color=gevored](B,D D,F B,F)
      \tkzDrawPoints(A,B,C,D,E,F)
     \end{tikzpicture}
    \end{subfigure}
    \caption*{Examples of \textcolor{gevored}{triangulations}.}
   \end{figure}

   \begin{figure}[H]
    \centering
    \begin{subfigure}[b]{.2\textwidth}
     \centering
     \begin{tikzpicture}[scale=0.875]
      \tkzDefPoint(-45:1){A}
      \tkzDefPoint(-135:1){B}
      \tkzDefPoint(135:1){C}
      \tkzDefPoint(45:1){D}

      \tkzDrawPolygon(A,B,C,D)
      \tkzDrawSegments[ultra thick,gevored](A,C B,D)
      \tkzDrawPoints(A,B,C,D)
     \end{tikzpicture}
    \end{subfigure}
    \begin{subfigure}[b]{.2\textwidth}
     \centering
     \begin{tikzpicture}[scale=0.7]
      \tkzDefPoint(0:1){A}
      \tkzDefPoint(60:1){B}
      \tkzDefPoint(120:1){C}
      \tkzDefPoint(180:1){D}
      \tkzDefPoint(240:1){E}
      \tkzDefPoint(300:1){F}

      \tkzDrawPolygon(A,B,C,D,E,F)
      \tkzDrawSegments[ultra thick,color=gevored](A,C A,E)
      \tkzDrawPoints(A,B,C,D,E,F)
     \end{tikzpicture}
    \end{subfigure}
    \begin{subfigure}[b]{.2\textwidth}
     \centering
     \begin{tikzpicture}[scale=0.7]
      \tkzDefPoint(0:1){A}
      \tkzDefPoint(60:1){B}
      \tkzDefPoint(120:1){C}
      \tkzDefPoint(180:1){D}
      \tkzDefPoint(240:1){E}
      \tkzDefPoint(300:1){F}

      \tkzDrawPolygon(A,B,C,D,E,F)
      \tkzDrawSegments[ultra thick,color=gevored](B,E C,F B,F)
      \tkzDrawPoints(A,B,C,D,E,F)
     \end{tikzpicture}
    \end{subfigure}
    \caption*{Counterexamples of \textcolor{gevored}{triangulations}.}
   \end{figure}

   The total number of different triangulations of a convex $n$-gon is
   \[
    \frac{n \cdot (n+1) \cdot \ldots \cdot (2n - 4)}{(n-2)!},
   \]
   which you \textbf{of course don't have to remember}. Interestingly enough,
   every triangulation can be transformed into any other by a series of
   \bfalert{flips}.

   A \bfalert{flip} is a swap of one diagonal for the other in a chosen
   quadrilateral so that the \textbf{result is again a triangulation}.
   \begin{figure}[H]
    \centering
    \begin{subfigure}[b]{.4\textwidth}
     \centering
     \begin{tikzpicture}[scale=0.7]
      \begin{scope}
       \tkzDefPoint(0:1){A}
       \tkzDefPoint(60:1){B}
       \tkzDefPoint(120:1){C}
       \tkzDefPoint(180:1){D}
       \tkzDefPoint(240:1){E}
       \tkzDefPoint(300:1){F}

       \tkzDrawPolygon(A,B,C,D,E,F)
       \tkzDrawPolygon[fill,pattern={north west lines},pattern
       color=black!50](B,D,E,F)
       \tkzDrawSegments[ultra thick,color=gevored](B,D B,F)
       \tkzDrawSegment[ultra thick,color=gevodarkblue](D,F)
       \tkzDrawPoints(A,B,C,D,E,F)
      \end{scope}
      \begin{scope}[xshift=2cm]
       \draw[ultra thick,gevodarkblue,->] (0,0) -- (2,0);
      \end{scope}
      \begin{scope}[xshift=6cm]
       \tkzDefPoint(0:1){A}
       \tkzDefPoint(60:1){B}
       \tkzDefPoint(120:1){C}
       \tkzDefPoint(180:1){D}
       \tkzDefPoint(240:1){E}
       \tkzDefPoint(300:1){F}

       \tkzDrawPolygon(A,B,C,D,E,F)
       \tkzDrawPolygon[fill,pattern={north west lines},pattern
       color=black!50](B,D,E,F)
       \tkzDrawSegments[ultra thick,color=gevored](B,D B,F)
       \tkzDrawSegment[ultra thick,color=gevodarkblue](B,E)
       \tkzDrawPoints(A,B,C,D,E,F)
      \end{scope}
     \end{tikzpicture}
    \end{subfigure}
    \begin{subfigure}[b]{.4\textwidth}
     \centering
     \begin{tikzpicture}[scale=0.7]
      \begin{scope}[rotate=18]
       \tkzDefPoint(0:1){A}
       \tkzDefPoint(72:1){B}
       \tkzDefPoint(144:1){C}
       \tkzDefPoint(216:1){D}
       \tkzDefPoint(288:1){E}

       \tkzDrawPolygon(A,B,C,D,E)
       \tkzDrawPolygon[fill,pattern={north west lines},pattern
       color=black!50](A,C,D,E)
       \tkzDrawSegment[ultra thick,color=gevored](A,C)
       \tkzDrawSegment[ultra thick,color=gevodarkblue](A,D)
       \tkzDrawPoints(A,B,C,D,E)
      \end{scope}
      \begin{scope}[xshift=2cm]
       \draw[ultra thick,gevodarkblue,->] (0,0) -- (2,0);
      \end{scope}
      \begin{scope}[xshift=6cm,rotate=18]
       \tkzDefPoint(0:1){A}
       \tkzDefPoint(72:1){B}
       \tkzDefPoint(144:1){C}
       \tkzDefPoint(216:1){D}
       \tkzDefPoint(288:1){E}

       \tkzDrawPolygon(A,B,C,D,E)
       \tkzDrawPolygon[fill,pattern={north west lines},pattern
       color=black!50](A,C,D,E)
       \tkzDrawSegment[ultra thick,color=gevored](A,C)
       \tkzDrawSegment[ultra thick,color=gevodarkblue](C,E)
       \tkzDrawPoints(A,B,C,D,E)
      \end{scope}
     \end{tikzpicture}
    \end{subfigure}
    \caption*{Examples of \textcolor{gevodarkblue}{flips}.}
   \end{figure}

   \begin{figure}[H]
    \centering
    \begin{subfigure}[b]{.4\textwidth}
     \centering
     \begin{tikzpicture}[scale=0.7]
      \begin{scope}
       \tkzDefPoint(0:1){A}
       \tkzDefPoint(60:1){B}
       \tkzDefPoint(120:1){C}
       \tkzDefPoint(180:1){D}
       \tkzDefPoint(240:1){E}
       \tkzDefPoint(300:1){F}

       \tkzDrawPolygon(A,B,C,D,E,F)
       \tkzDrawSegments[ultra thick,color=gevored](B,D B,F)
       \tkzDrawSegment[ultra thick,color=gevodarkblue](D,F)
       \tkzDrawPoints(A,B,C,D,E,F)
      \end{scope}
      \begin{scope}[xshift=2cm]
       \draw[ultra thick,gevodarkblue,->,decorate,decoration={zigzag}] (0,0) --
        (2,0);
       \draw[ultra thick,gevored,rotate around={45:(1,0)}] (0.5,0) -- (1.5,0);
       \draw[ultra thick,gevored,rotate around={-45:(1,0)}] (0.5,0) -- (1.5,0);
      \end{scope}
      \begin{scope}[xshift=6cm]
       \tkzDefPoint(0:1){A}
       \tkzDefPoint(60:1){B}
       \tkzDefPoint(120:1){C}
       \tkzDefPoint(180:1){D}
       \tkzDefPoint(240:1){E}
       \tkzDefPoint(300:1){F}

       \tkzDrawPolygon(A,B,C,D,E,F)
       \tkzDrawSegments[ultra thick,color=gevored](B,D B,F)
       \tkzDrawSegment[ultra thick,color=gevodarkblue](A,E)
       \tkzDrawPoints(A,B,C,D,E,F)
      \end{scope}
     \end{tikzpicture}
    \end{subfigure}
    \begin{subfigure}[b]{.4\textwidth}
     \centering
     \begin{tikzpicture}[scale=0.7]
      \begin{scope}[rotate=18]
       \tkzDefPoint(0:1){A}
       \tkzDefPoint(72:1){B}
       \tkzDefPoint(144:1){C}
       \tkzDefPoint(216:1){D}
       \tkzDefPoint(288:1){E}

       \tkzDrawPolygon(A,B,C,D,E)
       \tkzDrawSegment[ultra thick,color=gevored](A,C)
       \tkzDrawSegment[ultra thick,color=gevodarkblue](A,D)
       \tkzDrawPoints(A,B,C,D,E)
      \end{scope}
      \begin{scope}[xshift=2cm]
       \draw[ultra thick,gevodarkblue,->,decorate,decoration={zigzag}] (0,0) --
        (2,0);
       \draw[ultra thick,gevored,rotate around={45:(1,0)}] (0.5,0) -- (1.5,0);
       \draw[ultra thick,gevored,rotate around={-45:(1,0)}] (0.5,0) -- (1.5,0);
      \end{scope}
      \begin{scope}[xshift=6cm,rotate=18]
       \tkzDefPoint(0:1){A}
       \tkzDefPoint(72:1){B}
       \tkzDefPoint(144:1){C}
       \tkzDefPoint(216:1){D}
       \tkzDefPoint(288:1){E}

       \tkzDrawPolygon(A,B,C,D,E)
       \tkzDrawSegment[ultra thick,color=gevored](A,C)
       \tkzDrawSegment[ultra thick,color=gevodarkblue](B,E)
       \tkzDrawPoints(A,B,C,D,E)
      \end{scope}
     \end{tikzpicture}
    \end{subfigure}
    \caption*{Counterexamples of \textcolor{gevodarkblue}{flips}.}
   \end{figure}
  \end{exampleblock}

\end{column}

\separatorcolumn

\begin{column}{\colwidth}

  \begin{block}{A block containing an enumerated list}

    Vivamus congue volutpat elit non semper. Praesent molestie nec erat ac
    interdum. In quis suscipit erat. \textbf{Phasellus mauris felis, molestie
    ac pharetra quis}, tempus nec ante. Donec finibus ante vel purus mollis
    fermentum. 

    \begin{enumerate}
      \item \textbf{Morbi mauris purus}, egestas at vehicula et, convallis
        accumsan orci. Orci varius natoque penatibus et magnis dis parturient
        montes, nascetur ridiculus mus.
      \item \textbf{Cras vehicula blandit urna ut maximus}. Aliquam blandit nec
        massa ac sollicitudin. Curabitur cursus, metus nec imperdiet bibendum,
        velit lectus faucibus dolor, quis gravida metus mauris gravida turpis.
      \item \textbf{Vestibulum et massa diam}. Phasellus fermentum augue non
        nulla accumsan, non rhoncus lectus condimentum.
    \end{enumerate}

  \end{block}

  \begin{block}{Fusce aliquam magna velit}

    Et rutrum ex euismod vel. Pellentesque ultricies, velit in fermentum
    vestibulum, lectus nisi pretium nibh, sit amet aliquam lectus augue vel
    velit. Suspendisse rhoncus massa porttitor augue feugiat molestie. Sed
    molestie ut orci nec malesuada. Sed ultricies feugiat est fringilla
    posuere.

    \begin{figure}
      \centering
      \begin{tikzpicture}[scale=0.75]
        \begin{axis}[
            scale only axis,
            no markers,
            domain=0:2*pi,
            samples=100,
            axis lines=center,
            axis line style={-},
            ticks=none]
          \addplot[red] {sin(deg(x))};
          \addplot[blue] {cos(deg(x))};
        \end{axis}
      \end{tikzpicture}
      \caption{Another figure caption.}
    \end{figure}

  \end{block}

  \begin{block}{Nam cursus consequat egestas}

    Etiam sit amet tempus lorem, aliquet condimentum velit. Donec et nibh
    consequat, sagittis ex eget, dictum orci. Etiam quis semper ante. Ut eu
    mauris purus. Proin nec consectetur ligula. Mauris pretium molestie
    ullamcorper. Integer nisi neque, aliquet et odio non, sagittis porta justo.

    \begin{itemize}
      \item \textbf{Sed consequat} id ante vel efficitur. Praesent congue massa
        sed est scelerisque, elementum mollis augue iaculis.
        \begin{itemize}
          \item In sed est finibus, vulputate
            nunc gravida, pulvinar lorem. In maximus nunc dolor, sed auctor eros
            porttitor quis.
          \item Fusce ornare dignissim nisi. Nam sit amet risus vel lacus
            tempor tincidunt eu a arcu.
          \item Donec rhoncus vestibulum erat, quis aliquam leo
            gravida egestas.
        \end{itemize}
      \item \textbf{Pellentesque facilisis dolor in leo} bibendum congue.
        Maecenas congue finibus justo, vitae eleifend urna facilisis at.
    \end{itemize}

  \end{block}

  
  \begin{exampleblock}{A highlighted block containing some math}

    A different kind of highlighted block.

    $$
    \int_{-\infty}^{\infty} e^{-x^2}\,dx = \sqrt{\pi}
    $$

    Interdum et malesuada fames $\{1, 4, 9, \ldots\}$ ac ante ipsum primis in
    faucibus. Cras eleifend dolor eu nulla suscipit suscipit. Sed lobortis non
    felis id vulputate.

    \heading{A heading inside a block}

    Praesent consectetur mi $x^2 + y^2$ metus, nec vestibulum justo viverra
    nec. Proin eget nulla pretium, egestas magna aliquam, mollis neque. Vivamus
    dictum $\mathbf{u}^\intercal\mathbf{v}$ sagittis odio, vel porta erat
    congue sed. Maecenas ut dolor quis arcu auctor porttitor.

    \heading{Another heading inside a block}

    Sed augue erat, scelerisque a purus ultricies, placerat porttitor neque.
    Donec $P(y \mid x)$ fermentum consectetur $\nabla_x P(y \mid x)$ sapien
    sagittis egestas. Duis eget leo euismod nunc viverra imperdiet nec id
    justo.

  \end{exampleblock}

 

  \begin{block}{References (opcional)}

    \nocite{*}
    \footnotesize{\bibliographystyle{plain}\bibliography{poster}}

  \end{block}

\end{column}
\separatorcolumn



\end{columns}
\end{frame}

\end{document}
