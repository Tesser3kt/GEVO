\documentclass[a4paper,11pt]{article}

\usepackage[czech,english]{babel}
% Fonts %
\usepackage{fouriernc}
\usepackage[T1]{fontenc}

% Colors %
\usepackage[dvipsnames]{color}
\usepackage[dvipsnames]{xcolor}

% Page Layout %
\usepackage[margin=1.5in]{geometry}

% Fancy Headers %
\usepackage{fancyhdr}
\fancyhf{}
\cfoot{\thepage}
\rhead{}
\renewcommand{\headrulewidth}{0pt}
\setlength{\headheight}{16pt}

% Math
\usepackage{mathtools}
\usepackage{amssymb}
\usepackage{faktor}
\usepackage{import}
\usepackage{caption}
\usepackage{subcaption}
\usepackage{wrapfig}
\usepackage{enumitem}
\setlist{topsep=0pt}

\usepackage{tikz}
\usetikzlibrary{cd,positioning,babel,shapes,calc}
\usepackage{tkz-base}
\usepackage{tkz-euclide}

% Theorems
\usepackage[thmmarks, amsmath, thref]{ntheorem}
\usepackage{thmtools}

\theoremsymbol{\ensuremath{\blacksquare}}
\newtheorem*{solution}{Possible solution.}

% Title %
\title{\Huge\textsf{Bonus Problems -- PreIB 3.AB 3 \& 4}\\
 \Large\textsf{Triangulations and Symmetries of Regular Polygons}
 \author{Áďa Klepáčů}
 \date{\today}
}

% Table of Contents %
\usepackage{hyperref}
\hypersetup{
 colorlinks=true,
 linktoc=all,
 linkcolor=blue
}

% Tables %
\usepackage{booktabs}
\usepackage{tabularx}

% Patch for hyphens
\usepackage{regexpatch}
\makeatletter
% Change the `-` delimiter to an active character
\xpatchparametertext\@@@cmidrule{-}{\cA-}{}{}
\xpatchparametertext\@cline{-}{\cA-}{}{}
\makeatother

\newcolumntype{s}{>{\centering\arraybackslash}p{.4\textwidth}}

% Operators %
\DeclareMathOperator{\Ker}{Ker}
\DeclareMathOperator{\Img}{Im}
\DeclareMathOperator{\End}{End}
\DeclareMathOperator{\Aut}{Aut}
\DeclareMathOperator{\Inn}{Inn}

% Common operators %
\newcommand{\R}{\mathbb{R}}
\newcommand{\N}{\mathbb{N}}
\newcommand{\Z}{\mathbb{Z}}
\newcommand{\Q}{\mathbb{Q}}
\newcommand{\C}{\mathbb{C}}

\newcommand{\clr}{\textcolor{red}}
\newcommand{\clb}{\textcolor{blue}}
\newcommand{\clg}{\textcolor{green}}
\newcommand{\clm}{\textcolor{magenta}}
\newcommand{\clv}{\textcolor{violet}}
\newcommand{\clbr}{\textcolor{Sepia}}

% American Paragraph Skip %
\setlength{\parindent}{0pt}
\setlength{\parskip}{1em}

% Document %
\pagestyle{fancy}
\begin{document}

\maketitle
\thispagestyle{fancy}

\hrule
\textbf{\footnotesize \clr{\uppercase{You will be asked to present your
solutions orally. You don't need to write anything down. Only your
understanding of the problems and the provided solution is evaluated.
Correctness of the results is immaterial.}}}
\vspace{2ex}
\hrule
 
\section*{Triangulations}

The graph of triangulations of a regular heptagon has 42 triangulations. By a
\textbf{cycle} in the graph, I mean a path (of flips) which ends where it
started. That is, it's a sequence of flips which returns to the original
triangulation. Find one of the \clr{\textbf{shortest}} (having the least number
of flips) cycles in the graph of triangulations of the heptagon. \textbf{Explain
very precisely WHY the cycle you've found is the shortest}. Please, do
\textbf{not} draw the whole graph of triangulations...
\clearpage

\section*{Symmetries}
In the regular dodecagon (12 vertices), you're given 3 reflections --
$\clr{s_1}, \clb{s_2}$ and $\clg{s_3}$.
\begin{center}
 \begin{tikzpicture}[scale=1.5]
  \foreach \n/\an in
  {a/90,b/120,c/150,d/180,e/210,f/240,g/270,h/300,i/330,j/0,k/30,l/60}
  {
   \tkzDefPoint(\an:1.5){\n}
  }
  \tkzDrawPolygon(a,b,c,d,e,f,g,h,i,j,k,l)
  \tkzDrawLine[red,thick,dashed](a,g)
  \tkzLabelLine[pos=-0.1,left](a,g){$\clr{s_1}$}
  \tkzDrawLine[blue,thick,dashed](c,i)
  \tkzLabelLine[pos=-0.1,below](c,i){$\clb{s_2}$}
  \tkzDefMidPoint(d,e)\tkzGetPoint{sde}
  \tkzDefMidPoint(k,j)\tkzGetPoint{skj}
  \tkzDrawLine[green,thick,dashed](sde,skj)
  \tkzLabelLine[pos=-0.1,below](sde,skj){$\clg{s_3}$}
  \tkzDrawPoints[size=4](a,b,c,d,e,f,g,h,i,j,k,l)
 \end{tikzpicture}
\end{center}
Can I get $\clg{s_3}$ by combining $\clr{s_1}$ and $\clb{s_2}$? By a
`combination' I mean any sequence (however long) of $\clr{s_1}$'s and
$\clb{s_2}$'s, for example $\clr{s_1s_1}\clb{s_2}\clr{s_1}\clb{s_2}\clb{s_2}$.
If yes, show how. If not, explain why. \textbf{In any case, be precise and
thorough in describing your solution.}

\end{document}
