\documentclass[a4paper,11pt]{article}

\usepackage[czech,english]{babel}
% Fonts %
\usepackage{fouriernc}
\usepackage[T1]{fontenc}

% Colors %
\usepackage[dvipsnames]{color}
\usepackage{xcolor}

% Page Layout %
\usepackage[margin=1in]{geometry}
\usepackage{paracol}

% Fancy Headers %
\usepackage{fancyhdr}
\fancyhf{}
\rhead{\sffamily\large Vocabulary Exam -- 3.AB PreIB 4}
\lhead{\sffamily\large January 17$^{\text{th}}$, 2024}
\renewcommand{\headrulewidth}{0.4pt}
\setlength{\headheight}{16pt}

% Math
\usepackage{mathtools}
\usepackage{amssymb}
\usepackage{faktor}
\usepackage{import}
\usepackage{caption}
\usepackage{subcaption}
\usepackage{wrapfig}
\usepackage{enumitem}
\usepackage{tikz}
\usetikzlibrary{cd,positioning,babel,shapes}
\usepackage{tkz-base}
\usepackage{tkz-euclide}

% Theorems
\usepackage{thmtools}
\usepackage[thmmarks, amsmath, thref]{ntheorem}

% Title %
\title{\Huge\textsf{Vocabulary Exam -- PreIB 3.AB 3}\\
 \Large\textsf{Systems of Linear Equations}
 \author{Áďa Klepáčů}
 \date{\today}
}

% Table of Contents %
\usepackage{hyperref}
\hypersetup{
 colorlinks=true,
 linktoc=all,
 linkcolor=blue
}

% Tables %
\usepackage{booktabs}
\usepackage{tabularx}

% Patch for hyphens
\usepackage{regexpatch}
\makeatletter
% Change the `-` delimiter to an active character
\xpatchparametertext\@@@cmidrule{-}{\cA-}{}{}
\xpatchparametertext\@cline{-}{\cA-}{}{}
\makeatother

\newcolumntype{s}{>{\centering\arraybackslash}p{.4\textwidth}}

% Operators %
\DeclareMathOperator{\Ker}{Ker}
\DeclareMathOperator{\Img}{Im}
\DeclareMathOperator{\End}{End}
\DeclareMathOperator{\Aut}{Aut}
\DeclareMathOperator{\Inn}{Inn}

% Common operators %
\newcommand{\R}{\mathbb{R}}
\newcommand{\N}{\mathbb{N}}
\newcommand{\Z}{\mathbb{Z}}
\newcommand{\Q}{\mathbb{Q}}
\newcommand{\C}{\mathbb{C}}

\newcommand{\tr}{\textcolor{red}}
\newcommand{\tb}{\textcolor{blue}}
\newcommand{\tg}{\textcolor{green}}
\newcommand{\tm}{\textcolor{magenta}}
\newcommand{\tv}{\textcolor{violet}}

\newcommand{\blank}{\underline{\hspace{16ex}}}

% American Paragraph Skip %
\setlength{\parindent}{0pt}
\setlength{\parskip}{1em}

% Document %
\pagestyle{fancy}
 \renewcommand{\baselinestretch}{1.2}
\begin{document}

\thispagestyle{fancy}

\columnratio{.7}
\begin{paracol}{2}
 A \blank is a closed 2D shape made up of points and segments. The points are
 called \blank and the segments are called \blank. If we connect two \blank
 which are not next to each other by a segment, this segment is called a \blank.

 A \blank whose inner angles don't exceed $180^{ \circ }$ is called \blank.
 Those that have only four sides even have special names. For example, a shape
 made of two pairs of parallel sides is called a \blank. When in addition all of
 the sides share the same length, it's a \blank. In general, a \blank with any
 number of sides that are all equally long is called \blank.

 Such shapes are interesting because they're symmetric with respect to a number
 of \blank~~\blank. We extensively studied two of those -- \blank~and \blank. We
 also saw that their composition is again a \blank~or a \blank.

 A \blank with only three sides is called a \blank. Out of those, the ones
 having an inner angle of $90^{ \circ }$ are interesting. They're called
 \blank~~\blank. Their longest side is called a \blank and the two shorter ones
 are \blank.

 \switchcolumn
 \renewcommand{\arraystretch}{2}
 \begin{tabular}{c}
  \textbf{HYPOTHENUSE}\\
  \textbf{CONVEX}\\
  \textbf{RHOMBUS}\\
  \textbf{PLANE}\\
  \textbf{REFLECTION}\\
  \textbf{VERTEX}\\
  \textbf{TRANSFORMATION}\\
  \textbf{POLYGON}\\
  \textbf{PARALLELOGRAM}\\
  \textbf{TRIANGLE}\\
  \textbf{RIGHT}\\
  \textbf{DIAGONAL}\\
  \textbf{REGULAR}\\
  \textbf{ROTATION}\\
  \textbf{EDGE}\\
  \textbf{CATHETUS}
 \end{tabular}
\end{paracol}

\end{document}
