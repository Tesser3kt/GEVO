\documentclass[a4paper,11pt]{article}

\usepackage[czech,english]{babel}
% Fonts %
\usepackage{fouriernc}
\usepackage[T1]{fontenc}

% Colors %
\usepackage[dvipsnames]{color}
\usepackage[dvipsnames]{xcolor}

% Page Layout %
\usepackage[margin=1.5in]{geometry}

% Fancy Headers %
\usepackage{fancyhdr}
\fancyhf{}
\cfoot{\thepage}
\rhead{}
\renewcommand{\headrulewidth}{0pt}
\setlength{\headheight}{16pt}

% Math
\usepackage{mathtools}
\usepackage{amssymb}
\usepackage{faktor}
\usepackage{import}
\usepackage{caption}
\usepackage{subcaption}
\usepackage{wrapfig}
\usepackage{enumitem}
\setlist{topsep=0pt}

\usepackage{tikz}
\usetikzlibrary{cd,positioning,babel,shapes,calc}
\usepackage{tkz-base}
\usepackage{tkz-euclide}

% Theorems
\usepackage[thmmarks, amsmath, thref]{ntheorem}
\usepackage{thmtools}

\theoremsymbol{\ensuremath{\blacksquare}}
\newtheorem*{solution}{Possible solution.}

% Title %
\title{\Huge\textsf{Homework -- PreIB 3.AB 4}\\
 \Large\textsf{Triangulations and Symmetries of Regular Polygons}
 \author{Áďa Klepáčů}
 \date{\today}
}

% Table of Contents %
\usepackage{hyperref}
\hypersetup{
 colorlinks=true,
 linktoc=all,
 linkcolor=blue
}

% Tables %
\usepackage{booktabs}
\usepackage{tabularx}

% Patch for hyphens
\usepackage{regexpatch}
\makeatletter
% Change the `-` delimiter to an active character
\xpatchparametertext\@@@cmidrule{-}{\cA-}{}{}
\xpatchparametertext\@cline{-}{\cA-}{}{}
\makeatother

\newcolumntype{s}{>{\centering\arraybackslash}p{.4\textwidth}}

% Operators %
\DeclareMathOperator{\Ker}{Ker}
\DeclareMathOperator{\Img}{Im}
\DeclareMathOperator{\End}{End}
\DeclareMathOperator{\Aut}{Aut}
\DeclareMathOperator{\Inn}{Inn}

% Common operators %
\newcommand{\R}{\mathbb{R}}
\newcommand{\N}{\mathbb{N}}
\newcommand{\Z}{\mathbb{Z}}
\newcommand{\Q}{\mathbb{Q}}
\newcommand{\C}{\mathbb{C}}

\newcommand{\clr}{\textcolor{red}}
\newcommand{\clb}{\textcolor{blue}}
\newcommand{\clg}{\textcolor{green}}
\newcommand{\clm}{\textcolor{magenta}}
\newcommand{\clv}{\textcolor{violet}}
\newcommand{\clbr}{\textcolor{Sepia}}

% American Paragraph Skip %
\setlength{\parindent}{0pt}
\setlength{\parskip}{1em}

% Document %
\pagestyle{fancy}
\begin{document}

\maketitle
\thispagestyle{fancy}

\begin{center}
 \hrule
 \textbf{\clr{DON'T FORGET TO EXPLAIN EVERYTHING EVEN IF YOU THINK IT'S
 OBVIOUS!}}
 \vspace{2ex}
 \hrule
\end{center}
 
\section*{Triangulations}

Recall that a \textbf{triangulation} of a regular (or generally convex) polygon
is its division into triangles by non-intersecting diagonals. For example, here
is a triangulation of a regular hexagon intro triangles $\clr{T_1,T_2,T_3}$ and
$\clr{T_4}$.

\begin{figure}[ht]
 \centering
 \begin{tikzpicture}
  \foreach \n/\an in {a/0,b/60,c/120,d/180,e/240,f/300} {
   \tkzDefPoint(\an:1.5){\n}
  }
  \tkzDrawPolygon(a,b,c,d,e,f)

  \tkzDrawLine[color=blue,add=0 and 0](a,c)
  \tkzDrawLine[color=blue,add=0 and 0](a,d)
  \tkzDrawLine[color=blue,add=0 and 0](d,f)

  \tkzDefBarycentricPoint(a=1,b=1,c=1) \tkzGetPoint{t1}
  \tkzDefBarycentricPoint(a=1,c=1,d=1) \tkzGetPoint{t2}
  \tkzDefBarycentricPoint(a=1,d=1,f=1) \tkzGetPoint{t3}
  \tkzDefBarycentricPoint(d=1,e=1,f=1) \tkzGetPoint{t4}
 
  \node[yshift=1mm,red] at (t1.center) {$T_1$};
  \node[xshift=-1mm,yshift=1mm,red] at (t2.center) {$T_2$};
  \node[red] at (t3.center) {$T_3$};
  \node[yshift=-1mm,red] at (t4.center) {$T_4$};
  \tkzDrawPoints[size=4](a,b,c,d,e,f)
 \end{tikzpicture}
\end{figure}

Two triangulations of a regular polygon are related by a \textbf{flip} if you
can get one from the other by deleting one edge and replacing it by the other
diagonal in the resulting quadrilateral. See the picture below.

\begin{figure}[ht]
 \centering
 \begin{tikzpicture}
  \foreach \n/\an in {a/0,b/60,c/120,d/180,e/240,f/300} {
   \tkzDefPoint(\an:1){\n}
  }
  \tkzDrawPolygon(a,b,c,d,e,f)

  \tkzDrawLine[color=blue,add=0 and 0](a,c)
  \tkzDrawLine[color=red,add=0 and 0](a,d)
  \tkzDrawLine[color=blue,add=0 and 0](d,f)
  \tkzLabelLine[above](a,d){$\clr{e}$}

  \tkzDrawPoints[size=4](a,b,c,d,e,f)

  \draw[->] (1.25,0) to node[midway,yshift=2.5mm] {\footnotesize delete $\clr{e}$}
   (2.5,0);

  \tkzDefPoint(4.75,0){a1}
  \tkzDefPointsBy[translation= from a to a1](b,c,d,e,f){b1,c1,d1,e1,f1}
  \tkzDrawPolygon(a1,b1,c1,d1,e1,f1)

  \tkzDrawPolygon[thick,green](a1,c1,d1,f1)
  \tkzDrawPoints[size=4](a1,b1,c1,d1,e1,f1)

  \draw[->] (5,0) to node[midway,yshift=4mm,align=center,execute at begin
  node=\setlength{\baselineskip}{10pt}] {\footnotesize draw the other diagonal \\
  \footnotesize of the \clg{quadrilateral}} (8,0);

  \tkzDefPoint(10.25,0){a2}
  \tkzDefPointsBy[translation= from a1 to a2](b1,c1,d1,e1,f1){b2,c2,d2,e2,f2}
  \tkzDrawPolygon(a2,b2,c2,d2,e2,f2)

  \tkzDrawLine[color=blue,add=0 and 0](a2,c2)
  \tkzDrawLine[color=blue,add=0 and 0](d2,f2)
  \tkzDrawLine[color=blue,add=0 and 0](c2,f2)

  \tkzDrawPoints[size=4](a2,b2,c2,d2,e2,f2)
 \end{tikzpicture}
 \caption*{Flip of a \clb{triangulation}.}
\end{figure}

Actually, any triangulation of a regular polygon can be transformed into any
other triangulation just by flipping (often more than once). We can draw a graph
where we connect two different triangulations of a regular polygon if they are
related by a single flip.

The pentagon has 5 different triangulations. Their graph can look like this:
\begin{figure}[ht]
 \centering
  \begin{tikzpicture}
   \tkzDefPoint(0,0){o}
   \foreach \n/\an in {a/18,b/90,c/162,d/234,e/306} {
    \tkzDefPoint(\an:1){\n}
   }
   \foreach \n/\an in {A/18,B/90,C/162,D/234,E/306} {
    \tkzDefPoint(\an:3){\n}
   }
   \tkzDefPointsBy[translation= from o to A](a,b,c,d,e){a1,b1,c1,d1,e1}
   \tkzDrawPolygon(a1,b1,c1,d1,e1)
   \tkzDrawSegment[blue](a1,c1)
   \tkzDrawSegment[blue](a1,d1)
   \tkzDrawPoints[size=4](a1,b1,c1,d1,e1)

   \tkzDefPointsBy[translation= from o to B](a,b,c,d,e){a2,b2,c2,d2,e2}
   \tkzDrawPolygon(a2,b2,c2,d2,e2)
   \tkzDrawSegment[blue](a2,c2)
   \tkzDrawSegment[blue](c2,e2)
   \tkzDrawPoints[size=4](a2,b2,c2,d2,e2)

   \tkzDefPointsBy[translation= from o to C](a,b,c,d,e){a3,b3,c3,d3,e3}
   \tkzDrawPolygon(a3,b3,c3,d3,e3)
   \tkzDrawSegment[blue](b3,e3)
   \tkzDrawSegment[blue](c3,e3)
   \tkzDrawPoints[size=4](a3,b3,c3,d3,e3)

   \tkzDefPointsBy[translation= from o to D](a,b,c,d,e){a4,b4,c4,d4,e4}
   \tkzDrawPolygon(a4,b4,c4,d4,e4)
   \tkzDrawSegment[blue](b4,e4)
   \tkzDrawSegment[blue](b4,d4)
   \tkzDrawPoints[size=4](a4,b4,c4,d4,e4)

   \tkzDefPointsBy[translation= from o to E](a,b,c,d,e){a5,b5,c5,d5,e5}
   \tkzDrawPolygon(a5,b5,c5,d5,e5)
   \tkzDrawSegment[blue](a5,d5)
   \tkzDrawSegment[blue](b5,d5)
   \tkzDrawPoints[size=4](a5,b5,c5,d5,e5)

   \tkzDrawLine[thick,red,dashed,add=-1cm and -1cm](A,B)
   \tkzDrawLine[thick,red,dashed,add=-1cm and -1cm](B,C)
   \tkzDrawLine[thick,red,dashed,add=-1cm and -1cm](C,D)
   \tkzDrawLine[thick,red,dashed,add=-1cm and -1cm](D,E)
   \tkzDrawLine[thick,red,dashed,add=-1cm and -1cm](E,A)
  \end{tikzpicture}
 \caption*{Triangulations of the pentagon. \clb{Blue} segments are diagonals in
 a triangulation and \clr{red} lines are flips.}
\end{figure}

\subsection*{Problems}

\begin{enumerate}
 \item In the graph above, each triangulation is \clr{connected} to two other by
  a flip because a triangulation of a pentagon has two \clb{diagonals}. If we
  drew the same graph of triangulations for the hexagon, how many other
  triangulations would a fixed triangulation be \clr{connected} to? Why?
 \item A regular hexagon has 14 distinct triangulations. Draw its graph of
  triangulations. It's going to take you a while...
 \item Using the graph find the \emph{smallest} possible number of flips to get
  from the triangulation on the left to the triangulation on the right.
  \begin{center}
   \begin{tikzpicture}
    \tkzDefPoint(0,0){o}
    \foreach \n/\an in {a/0,b/60,c/120,d/180,e/240,f/300} {
     \tkzDefPoint(\an:1){\n}
    }
    \tkzDrawPolygon(a,b,c,d,e,f)
    \tkzDrawSegment[blue](a,c)
    \tkzDrawSegment[blue](c,e)
    \tkzDrawSegment[blue](e,a)
    \tkzDrawPoints[size=4](a,b,c,d,e,f)

    \tkzDefPoint(5,0){p2}
    \tkzDefPointsBy[translation= from o to p2](a,b,c,d,e,f){a1,b1,c1,d1,e1,f1}
    \tkzDrawPolygon(a1,b1,c1,d1,e1,f1)
    \tkzDrawSegment[blue](b1,d1)
    \tkzDrawSegment[blue](d1,f1)
    \tkzDrawSegment[blue](f1,b1)
    \tkzDrawPoints[size=4](a1,b1,c1,d1,e1,f1)

    \draw[->] ($(o.center) + (1.2,0)$) to node[midway,yshift=3mm] {\footnotesize
     How many flips?} ($(p2.center) - (1.2,0)$);
   \end{tikzpicture}
  \end{center}
 \item Find at least three other ways how to get from the left triangulation to
  the right one using the same number of flips.
 \item (\clr{\textbf{OPTIONAL}}) How many ways are there using the same number
  of flips?
 \item Can flips be achieved using symmetries of regular polygons (meaning
  rotations and reflections)? If yes, show how on the triangulation graph of the
  pentagon. If not, explain why.
\end{enumerate}

\section*{Symmetries}

Recall the each regular $n$-gon has $n$ rotational symmetries and $n$
reflectional symmetries. The rotational symmetries are rotations by $k \cdot
360^{ \circ } / n$ where $k$ ranges from $1$ to $n$. If $n$ is odd, then all the
reflectional symmetries are given by lines passing through one vertex and the
midpoint of the opposite side. If $n$ is even, then $n / 2$ reflectional
symmetries are over lines connecting opposite vertices and the other $n / 2$
over lines passing through midpoints of opposite sides.

We saw that we can apply symmetries one after another to get new symmetries. We
also determined which types of symmetries we need to create all the other
symmetries. It was the rotation by $k \cdot 360^{ \circ } / n$ if $k / n$ cannot
be simplified and any reflection.

But, not all polygons are created equal. For example, in the pentagon one can
create all symmetries just by using two reflections. Take $\clr{s_1}$ and
$\clb{s_2}$ from the picture below.

\begin{center}
 \begin{tikzpicture}
  \foreach \n/\an in {a/18,b/90,c/162,d/234,e/306} {
   \tkzDefPoint(\an:1){\n}
  }
  \tkzDrawPolygon(a,b,c,d,e)
  \tkzDefMidPoint(d,e) \tkzGetPoint{sde}
  \tkzDefMidPoint(a,e) \tkzGetPoint{sae}

  \tkzDrawLine[red,dashed](b,sde)
  \tkzDrawLine[blue,dashed](c,sae)
 \end{tikzpicture}
\end{center}

We'll denote rotation by $n^{ \circ }$ in the counter-clockwise direction as
$\circlearrowleft n^{ \circ }$. You can check that $\clr{s_1}\clb{s_2} =
\circlearrowleft 144^{ \circ }$. But, $144 = 2 \cdot 360 / 5$ and $2 / 5$ can't
be simplified! This means that we can get all other rotations and reflections
using just $\clr{s_1}$ and $\clb{s_2}$.

Finally, I want you to think about $s_1s_2$ as `multiplication' of symmetries
and of
\begin{equation}
 \label{eq:symmetries}
 \tag{$*$}
 s_1s_2 = \circlearrowleft 144^{ \circ }
\end{equation}
as an `equation'. You can multiply both sides by any symmetry and it's still
going to hold true. But careful! Symmetries \textbf{do not commute}, so it
matters if you multiply from the left or from the right. The angle from
$\clb{s_2}$ to $\clr{s_1}$ is $72^{ \circ }$ in the clockwise direction. Which
means that to get $\clr{s_1}$ from $\clb{s_2}$, we need to rotate by $144^{
\circ }$ counter-clockwise and then use $\clb{s_2}$. Indeed, look what happens
if we multiply~\eqref{eq:symmetries} by $\clb{s_2}$ \textbf{from the right}. We
get
\[
 s_1s_2s_2 = \circlearrowleft 144^{ \circ }s_2.
\]
But $s_2s_2$ does nothing because it's the same reflection repeated twice. In
the end, we get
\[
 s_1 = \circlearrowleft 144^{ \circ }s_2.
\]
Which is precisely what we just said. Rotate by $144^{ \circ }$
counter-clockwise and use $\clb{s_2}$.

\subsection*{Problems}
\begin{enumerate}
 \item In the heptagon
 \begin{center}
  \begin{tikzpicture}
   \foreach \n/\an in
   {a/90,b/141.43,c/192.86,d/244.289,e/295.719,f/347.149,g/38.579} {
    \tkzDefPoint(\an:1){\n}
   }
   \tkzDrawPolygon(a,b,c,d,e,f,g)
   \tkzDefMidPoint(b,c) \tkzGetPoint{sbc}
   \tkzDrawLine[blue,thick,dashed,add=1cm and 1cm](f,sbc)
   \tkzLabelLine[pos=-0.5,above](f,sbc){$\clb{s}$}
   \tkzDrawPoints[size=4](a,b,c,d,e,f,g)
   \tkzLabelPoint[above](a){$A$}
   \tkzLabelPoint[above left](b){$B$}
   \tkzLabelPoint[left](c){$C$}
   \tkzLabelPoint[below left](d){$D$}
   \tkzLabelPoint[below right](e){$E$}
   \tkzLabelPoint[right](f){$F$}
   \tkzLabelPoint[above right](g){$G$}
  \end{tikzpicture}
 \end{center}
 you're given the reflection $\clb{s}$ and the rotation $\clr{r} =
 \circlearrowleft 3 \cdot 360^{ \circ } / 7$.
 \begin{enumerate}
  \item Is it possible to construct all other symmetries of the regular heptagon
   using only $\clr{r}$ and $\clb{s}$? Why?
  \item Using only $\clr{r}$ and $\clb{s}$ find
  \begin{enumerate}
   \item the rotation $\circlearrowleft 4 \cdot 360^{ \circ } / 7$;
   \item the rotation $\circlearrowright 360^{ \circ } / 7$, where
    $\circlearrowright$ means `rotate clockwise';
   \item the reflection over the line passing through $B$ and the midpoint of
    $EF$.
  \end{enumerate}
 \end{enumerate}
 \item In a decagon ($10$ vertices), you're given the reflections $\clr{s_1},
  \clb{s_2}$ and $\clg{s_3}$.
 \begin{center}
  \begin{tikzpicture}
   \foreach \n/\an in {a/90,b/126,c/162,d/198,e/234,f/270,g/306,h/342,i/18,j/54}
   {
    \tkzDefPoint(\an:1.5){\n}
   }
   \tkzDrawPolygon(a,b,c,d,e,f,g,h,i,j)
   \tkzDrawLine[red,thick,dashed](a,f)
   \tkzLabelLine[pos=-0.1,left](a,f){$\clr{s_1}$}
   \tkzDefMidPoint(i,j) \tkzGetPoint{sij}
   \tkzDefMidPoint(d,e) \tkzGetPoint{sde}
   \tkzDrawLine[blue,thick,dashed](sij,sde)
   \tkzLabelLine[pos=-0.1,below right](sij,sde){$\clb{s_2}$}
   \tkzDrawLine[green,thick,dashed](c,h)
   \tkzLabelLine[pos=-0.1,below left](c,h){$\clg{s_3}$}
   \tkzDrawPoints[size=4](a,b,c,d,e,f,g,h,i,j)
  \end{tikzpicture}
 \end{center}
 \begin{enumerate}
  \item Can you generate all the symmetries of the regular decagon using only
   $\clr{s_1}$ and $\clb{s_2}$? Explain.
  \item How about using only $\clr{s_1}$ and $\clg{s_3}$? Explain again.
 \end{enumerate}
\end{enumerate}

\end{document}
