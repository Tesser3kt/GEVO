\documentclass[a4paper,11pt]{article}

\usepackage[czech,english]{babel}
% Fonts %
\usepackage{fouriernc}
\usepackage[T1]{fontenc}

% Colors %
\usepackage[dvipsnames]{color}
\usepackage{xcolor}

% Page Layout %
\usepackage[margin=1in]{geometry}

% Fancy Headers %
\usepackage{fancyhdr}
\fancyhf{}
\cfoot{\thepage}
\rhead{}
\renewcommand{\headrulewidth}{0pt}
\setlength{\headheight}{16pt}

% Math
\usepackage{mathtools}
\usepackage{amssymb}
\usepackage{faktor}
\usepackage{import}
\usepackage{caption}
\usepackage{subcaption}
\usepackage{wrapfig}
\usepackage{enumitem}
\usepackage{tikz}
\usetikzlibrary{cd,positioning,babel,shapes}
\usepackage{tkz-base}
\usepackage{tkz-euclide}

% Theorems
\usepackage{thmtools}
\usepackage[thmmarks, amsmath, thref]{ntheorem}

% Title %
\title{\Huge\textsf{Vocabulary Exam -- PreIB 3.AB 3}\\
 \Large\textsf{Systems of Linear Equations}
 \author{Áďa Klepáčů}
 \date{March 30, 2023}
}

% Table of Contents %
\usepackage{hyperref}
\hypersetup{
 colorlinks=true,
 linktoc=all,
 linkcolor=blue
}

% Tables %
\usepackage{booktabs}
\usepackage{tabularx}

% Patch for hyphens
\usepackage{regexpatch}
\makeatletter
% Change the `-` delimiter to an active character
\xpatchparametertext\@@@cmidrule{-}{\cA-}{}{}
\xpatchparametertext\@cline{-}{\cA-}{}{}
\makeatother

\newcolumntype{s}{>{\centering\arraybackslash}p{.4\textwidth}}

% Operators %
\DeclareMathOperator{\Ker}{Ker}
\DeclareMathOperator{\Img}{Im}
\DeclareMathOperator{\End}{End}
\DeclareMathOperator{\Aut}{Aut}
\DeclareMathOperator{\Inn}{Inn}

% Common operators %
\newcommand{\R}{\mathbb{R}}
\newcommand{\N}{\mathbb{N}}
\newcommand{\Z}{\mathbb{Z}}
\newcommand{\Q}{\mathbb{Q}}
\newcommand{\C}{\mathbb{C}}

\newcommand{\tr}{\textcolor{red}}
\newcommand{\tb}{\textcolor{blue}}
\newcommand{\tg}{\textcolor{green}}
\newcommand{\tm}{\textcolor{magenta}}
\newcommand{\tv}{\textcolor{violet}}

% American Paragraph Skip %
\setlength{\parindent}{0pt}
\setlength{\parskip}{1em}

% Document %
\pagestyle{fancy}
 \renewcommand{\baselinestretch}{1.2}
\begin{document}

\maketitle
\thispagestyle{fancy}

\begin{minipage}{.7\textwidth}
 Below, you see a \underline{\hspace{12ex}} of two \underline{\hspace{12ex}}
 \underline{\hspace{18ex}} in two \underline{\hspace{18ex}}.
 \begin{align*}
  2x + 3y &= 7, \\
  -x + 4y &= 2.
 \end{align*}
 One can solve it for example by \underline{\hspace{22ex}} one of the
 \underline{\hspace{18ex}}. It seems easier to get rid of $x$. If I
 \underline{\hspace{10ex}} the second \underline{\hspace{16ex}} by $2$, I get
 \begin{align*}
  2x + 3y &= 7,\\
  -2x + 8y &= 4.
 \end{align*}
 I can then \underline{\hspace{6ex}} the first \underline{\hspace{16ex}} to the
 second and get
 \[
  11y = 11,
 \]
 which means that $y = 1$. When I \underline{\hspace{20ex}} this result into the
 first \underline{\hspace{16ex}}, I can solve
 \[
  2x + 3 \cdot 1 = 7
 \]
 and obtain $x = 2$. This gives me the \underline{\hspace{16ex}} $(2,1)$.

 If I \underline{\hspace{14ex}} $y$ in the first \underline{\hspace{16ex}}, I
 get
 \[
  y = -\frac{2}{3}x + \frac{7}{3}.
 \]
 This allows me to treat $y$ as a \underline{\hspace{12ex}}
 \underline{\hspace{16ex}}
 \[
  f(x) = -\frac{2}{3}x + \frac{7}{3}.
 \]
 The \underline{\hspace{10ex}} of such a \underline{\hspace{16ex}} is a straight
 line. Which means, that to draw it, I need to find two
 \underline{\hspace{12ex}}. The first coordinates are the
 \underline{\hspace{12ex}} of the \underline{\hspace{16ex}} $f$ and the second
 coordinates are the \underline{\hspace{14ex}}. So, choosing for example $x = 0$ 
 and $x = 2$, I get the two \underline{\hspace{12ex}}, $(0,7 / 3)$ and $(2,1)$ 
 by \underline{\hspace{
\end{minipage}
\hfill
\begin{minipage}{.25\textwidth}
 \renewcommand{\arraystretch}{2}
 \begin{tabular}{c}
  \textbf{SYSTEM}\\
  \textbf{LINEAR}\\
  \textbf{EQUATION}\\
  \textbf{VARIABLE/UNKNOWN}\\
  \textbf{ELIMINATE}\\
  \textbf{SCALE}\\
  \textbf{ADD}\\
  \textbf{SUBSTITUTE}\\
  \textbf{SOLUTION}\\
  \textbf{FUNCTION}\\
  \textbf{POINT}\\
  \textbf{INPUT}\\
  \textbf{OUTPUT}
 \end{tabular}
\end{minipage}

\end{document}
