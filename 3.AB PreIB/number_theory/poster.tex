% Unofficial University of Cambridge Poster Template
% https://github.com/andiac/gemini-cam
% a fork of https://github.com/anishathalye/gemini
% also refer to https://github.com/k4rtik/uchicago-poster
% TeX program = lualatex

\documentclass[final]{beamer}

% ====================
% Packages
% ====================

\usepackage[T1]{fontenc}
\usepackage{lmodern}
\usepackage[orientation=portrait,size=custom,width=160,height=120,scale=1]{beamerposter}
\usetheme{gemini}
\usepackage[dvipsnames]{xcolor}
\usecolortheme{nott}
\usepackage{graphicx}
\usepackage{booktabs}
\usepackage{tikz}
\usetikzlibrary{calc,arrows.meta,patterns,decorations.pathmorphing,shapes.geometric}
\usepackage{tkz-euclide}
\tikzset{point style/.style = {%
  draw = black,
  inner sep = 0pt,
  shape = circle,
  minimum size = 5pt,
  fill = black
 },
 every picture/.append style = {
  scale = 1.5
 },
 every node/.append style={
  scale=1.5
 }
}
\usepackage{pgfplots}
\pgfplotsset{compat=1.14}
\usepackage{anyfontsize}
\usepackage{caption}
\usepackage{subcaption}

% ====================
% Lengths
% ====================

% If you have N columns, choose \sepwidth and \colwidth such that
% (N+1)*\sepwidth + N*\colwidth = \paperwidth
\newlength{\sepwidth}
\newlength{\colwidth}
\setlength{\sepwidth}{0.008\paperwidth}
\setlength{\colwidth}{0.24\paperwidth}

\newcommand{\separatorcolumn}{\begin{column}{\sepwidth}\end{column}}
\newcommand{\bfalert}[1]{\textbf{\alert{#1}}}

% Math shortcuts
\newcommand{\R}{\mathbb{R}}
\newcommand{\N}{\mathbb{N}}
\newcommand{\Z}{\mathbb{Z}}
\newcommand{\Q}{\mathbb{Q}}
\DeclareMathOperator{\s}{succ}
\newcommand{\dv}{\mathbin{\mathrm{div}}}

% Inline shapes
\newcommand{\mysquare}{\tikz[baseline=-7pt]{%
  \node[rectangle,draw,thick,inner sep=6pt] at (0,0) {};
}}
\newcommand{\mytria}{\tikz[baseline=-3.25pt]{%
  \node[isosceles triangle,isosceles triangle apex angle=60,draw,thick,inner
  sep=3.25pt,rotate=90] at (0,0) {};
}}
\newcommand{\mycirc}{\tikz[baseline=-7pt]{%
  \node[circle,draw,thick,inner sep=4.5pt,baseline=0.5ex,rotate=90] at (0,0) {};
}}
\newcommand{\mycross}{\tikz[baseline=-7pt,scale=0.2]{%
  \draw[thick] (-1,1) -- (1,-1);
  \draw[thick] (-1,-1) -- (1,1);
}}

% Colors %
\newcommand{\clr}{\textcolor{BrickRed}}
\newcommand{\clb}{\textcolor{RoyalBlue}}
\newcommand{\clg}{\textcolor{ForestGreen}}
\newcommand{\clm}{\textcolor{Fuchsia}}
\newcommand{\clv}{\textcolor{violet}}
\newcommand{\clbr}{\textcolor{Sepia}}
\newcommand{\cly}{\textcolor{Dandelion}}

% ====================
% Title
% ====================

\title{Number Theory Cheatsheet}

\author{3.AB PreIB Math}

\institute[shortinst]{Adam Klepáč and Jáchym Löwenhöffer}

% ====================
% Footer (optional)
% ====================

% \footercontent{
%   \href{https://utfpr.edu.br/ct/ppgca}{utfpr.edu.br/ct/ppgca} \hfill
%   Mostra de Trabalhos do PPGCA --- TechTalks 2024 \hfill
%   \href{mailto:ppgca-ct@utfpr.edu.br}{ppgca-ct@utfpr.edu.br}}
% (can be left out to remove footer)


% ====================
% Logo (optional)
% ====================

% use this to include logos on the left and/or right side of the header:
\logoright{\includegraphics[height=6cm]{logos/logo-white.png}}
% \logoleft{\hspace{20ex}\includegraphics[height=3.5cm]{logos/ppgca-logo.png}}

% ====================
% Body
% ====================

\begin{document}

% Refer to https://github.com/k4rtik/uchicago-poster
% logo: https://www.cam.ac.uk/brand-resources/about-the-logo/logo-downloads
% \addtobeamertemplate{headline}{}
% {
%     \begin{tikzpicture}[remember picture,overlay]
%       \node [anchor=north west, inner sep=3cm] at ([xshift=-2.5cm,yshift=1.75cm]current page.north west)
%       {\includegraphics[height=7cm]{logos/unott-logo.eps}}; 
%     \end{tikzpicture}
% }

\begin{frame}[t]
\begin{columns}[t]
\separatorcolumn

\begin{column}{\colwidth}

 \begin{exampleblock}{Natural Numbers}
  Natural numbers (denoted $\N$) are defined basically as `sets containing so
  many elements'. This means that the number $0$ is a set with no elements, $1$
  is a set with one element and so on. Formally, we construct them in the
  following way ($\emptyset$ is the empty set):
  \[
   \begin{array}{r c l c l}
    0 & = & \emptyset & &\\
    1 & = & \{0\} & = & \{\emptyset\}\\
    2 & = & \{0, 1\} & = & \{\emptyset, \{\emptyset\}\}\\
    3 & = & \{0, 1, 2\} & = & \{\emptyset, \{\emptyset\}, \{\emptyset,
    \{\emptyset\}\}\}\\
      & \vdots & & & 
   \end{array}
  \]
  In general, the \alert{next natural number} after a number $n$ is defined as
  the set $\{0,\ldots,n\}$. 

  Observe that we can find a formula for the next number after $n$. Since $n =
  \{0,\ldots,n-1\}$ and the next number is $\{0,\ldots,n\}$, we can construct
  the next number after $n$ as a union of two sets: $n \cup \{n\}$. We call this
  number, the \alert{successor} of $n$, and write it as \alert{$\s(n)$}. For
  example, $1 = \{0\} = 0 \cup \{0\} = \s(0)$ or $3 = \{0,1,2\} = \{0,1\} \cup
  \{2\} = 2 \cup \{2\} = \s(2)$.

  \vspace{18pt}

  \textbf{\large Addition of natural numbers (\alert{not examined})}

  We can define the operation of \alert{addition} on natural numbers using two
  simple rules. For two natural numbers $n,m \in \N$,
  \begin{enumerate}[label=(\arabic*),left=12pt]
   \item $n + 1 = \s(n)$,
   \item $\s(n + m) = n + \s(m)$.
  \end{enumerate}
  Rule (1) simply states that $n + 1$ is the next number after $n$. Rule (2) is
  harder to decode. It literally says that by adding the two numbers together
  and then taking the next number one reaches the same answer as by first taking
  the next number and then performing addition. It's visualised on the picture
  below.
  \begin{figure}[H]
   \centering
   \begin{tikzpicture}
    \begin{scope}
     \foreach \x in {-.5, 0, .5} {
      \node[circle,fill=BrickRed,inner sep=3pt] at (\x,.75) {};
     }
     \node[BrickRed] at (0,0) {\footnotesize $n$};

     \foreach \x in {1.5, 2, 2.5, 3} {
      \node[circle,fill=RoyalBlue,inner sep=3pt] at (\x,.75) {};
     }
     \node[RoyalBlue] at (2.25,0) {\footnotesize $m$};
    \end{scope}
    \begin{scope}[xshift=5.5cm]
     \foreach \x in {-.5, 0, .5} {
      \node[circle,fill=BrickRed,inner sep=3pt] at (\x,.75) {};
     }
     \node at (.25,0) {\footnotesize $\clr{n} + \clb{m}$};

     \foreach \x in {-.5, 0, .5, 1} {
      \node[circle,fill=RoyalBlue,inner sep=3pt] at (\x,1.25) {};
     }
    \end{scope}
    \begin{scope}[xshift=8.5cm]
     \foreach \x in {-.5, 0, .5} {
      \node[circle,fill=BrickRed,inner sep=3pt] at (\x,.75) {};
     }
     \node at (.25,0) {\footnotesize $\clg{\s}(\clr{n} + \clb{m})$};

     \foreach \x in {-.5, 0, .5, 1} {
      \node[circle,fill=RoyalBlue,inner sep=3pt] at (\x,1.25) {};
     }
     \node[circle,fill=ForestGreen,inner sep=3pt] at (-.5, 1.75) {};
    \end{scope}
    \begin{scope}[xshift=12.5cm]
     \foreach \x in {-.5, 0, .5, 1} {
      \node[circle,fill=RoyalBlue,inner sep=3pt] at (\x,.75) {};
     }
     \node[circle,fill=ForestGreen,inner sep=3pt] at (-.5, 1.25) {};
     \node at (.25,0) {\footnotesize $\clg{\s}(\clb{m})$};
    \end{scope}
    \begin{scope}[xshift=15.5cm]
     \foreach \x in {-.5, 0, .5} {
      \node[circle,fill=BrickRed,inner sep=3pt] at (\x,.75) {};
     }
     \node[circle,fill=ForestGreen,inner sep=3pt] at (-.5, 1.75) {};
     \node at (.25,0) {\footnotesize $\clr{n} + \clg{\s}(\clb{m})$};

     \foreach \x in {-.5, 0, .5, 1} {
      \node[circle,fill=RoyalBlue,inner sep=3pt] at (\x,1.25) {};
     }
    \end{scope}
   \end{tikzpicture}

   \caption{Visualisation of rule (2) of addition. Both $\clg{\s}(\clr{n} +
    \clb{m})$ and $\clr{n} + \clg{\s}(\clb{m})$ feature the \alert{same number}
    of dots.}
   \label{fig:rule-2-addition}
  \end{figure}

  Rules (1) and (2) combine to give a simple algorithm of computing the sum $n +
  m$ for any two numbers $n,m \in \N$. It goes like this:
  \begin{itemize}[label=\textbullet,left=12pt]
   \item Using rule (1), calculate $n + 1 = \s(n) = n \cup \{n\}$.
   \item Now that we have calculated $n + 1$, we can calculate $n + 2$ because
    $n + 2 = n + \s(1)$ and by rule (2) this equals $n + \s(1) = \s(n + 1)$, so
    just take the next number after $n + 1$.
   \item Continue like this until you calculate $n + m = n + \s(m - 1)$.
  \end{itemize}
  For example, to compute $4 + 2$, we calculate $4 + 1 = \s(4)$ and then $4 + 2
  = 4 + \s(1) = \s(4 + 1) = \s(\s(4))$ so $4 + 2$ is just the next number after
  the next number after $4$.
 \end{exampleblock}

 \begin{exampleblock}{Integers (Whole Numbers)}
  We have defined \alert{addition} on natural numbers, but in order to perform
  \alert{subtraction}, we must move to a `larger' set of numbers -- the
  \alert{integers}. This is because subtraction is \alert{\textbf{not}} an
  operation on natural numbers as its result needn't be a natural number itself.

  The idea behind the definition of integers (labelled $\Z$) is to take
  \alert{pairs of natural numbers}. Fundamentally, we want the pair $(a,b) \in
  \N \times \N$ to \alert{represent} the result of the operation `$a-b$' (which
  we can't yet perform because we need to define the integers \textbf{before}
  defining subtraction).

  To this end, we define an \alert{equivalence} on $\N \times \N$ (i.e. on pairs
  of natural numbers) that makes two pairs equivalent \alert{if they represent
  the same integer}. For example, the pair $(4, 6)$ should represent the number
  $-2$ (as $4 - 6 = -2$) and so should the pairs $(8, 10)$, $(3, 5)$ or just
  about any pair $(a, a + 2)$ for $a \in \N$. The integers will then be the
  \alert{classes of equivalence} of this equivalence relation.

  We label this equivalence by the letter $\clg{E}$. Since we want $(a,b)$ to be
  \clg{equivalent} to $(c,d)$ if `$a - b = c - d$' but we can't use subtraction
  yet, we simply rewrite the equation above to use only addition, like this: $a
  + d = c + b$. Thus, we say that $\clr{(a,b)}\clg{E}\clb{(c,d)}$ if $\clr{a} +
  \clb{d} = \clb{c} + \clr{b}$. This defines an equivalence on $\N \times \N$
  and we let $\Z$ be the classes of equivalence of all pairs of natural numbers:
  \[
   \clm{\Z} = \{[(a,b)]_{\clg{E}} \mid a,b \in \N\}.
  \]
  To give an example, the pair $\clr{(3,5)}$ is \clg{equivalent} to
  $\clb{(7,9)}$ because $\clr{3} + \clb{9} = \clb{7} + \clr{5}$ and they both
  represent the integer $\clm{-2}$. Similarly, both $\clr{(6, 2)}$ and
  $\clb{(8, 4)}$ represent the integer $\clm{4}$. The visualisation of integers
  as pairs of natural numbers is given below.
  \begin{figure}[H]
   \centering
   \begin{tikzpicture}
    \node[ForestGreen] at (-1, 2.5) {\clg{$\cdots$}};
    \draw[ForestGreen,thick] (0, 5) -- (0, 0);
    \begin{scope}[xshift=1.5cm,yshift=1cm]
     \node[BrickRed] at (0, 3) {\footnotesize $(2, 4)$};
     \node[BrickRed] at (2, 3) {\footnotesize $(7, 9)$};
     \node[BrickRed] at (0, 2) {\footnotesize $(0, 2)$};
     \node[BrickRed] at (2, 2) {\footnotesize $(11, 13)$};
     \node at (1, 1) {\footnotesize $\vdots$};
     \node at (1, 0) {\footnotesize all pairs $\clr{(a, a + 2)}$};
     \node at (1, -2) {$\clm{-2}$};
    \end{scope}
    \draw[ForestGreen,thick] (5, 5) -- (5, 0);
    \begin{scope}[xshift=6.5cm,yshift=1cm]
     \node[BrickRed] at (0, 3) {\footnotesize $(3, 4)$};
     \node[BrickRed] at (2, 3) {\footnotesize $(10, 11)$};
     \node[BrickRed] at (0, 2) {\footnotesize $(0, 1)$};
     \node[BrickRed] at (2, 2) {\footnotesize $(22, 23)$};
     \node at (1, 1) {\footnotesize $\vdots$};
     \node at (1, 0) {\footnotesize all pairs $\clr{(a, a + 1)}$};
     \node at (1, -2) {$\clm{-1}$};
    \end{scope}
    \draw[ForestGreen,thick] (10, 5) -- (10, 0);
    \node[ForestGreen] at (11, 2.5) {\clg{$\cdots$}};
    \draw[ForestGreen,thick] (12, 5) -- (12, 0);
    \begin{scope}[xshift=13.5cm,yshift=1cm]
     \node[BrickRed] at (0, 3) {\footnotesize $(8, 5)$};
     \node[BrickRed] at (2, 3) {\footnotesize $(13, 10)$};
     \node[BrickRed] at (0, 2) {\footnotesize $(3, 0)$};
     \node[BrickRed] at (2, 2) {\footnotesize $(7, 4)$};
     \node at (1, 1) {\footnotesize $\vdots$};
     \node at (1, 0) {\footnotesize all pairs $\clr{(a + 3, a)}$};
     \node at (1, -2) {$\clm{3}$};
    \end{scope}
    \draw[ForestGreen,thick] (17, 5) -- (17, 0);
    \node[ForestGreen] at (18, 2.5) {\clg{$\cdots$}};
   \end{tikzpicture}
   \caption{\clm{Integers} as classes of \clg{equivalence} of natural numbers.}
   \label{fig:integers}
  \end{figure}
  The \alert{addition} of integers is defined using the addition of natural
  numbers. Given two classes of equivalence $[(\clr{a},\clr{b})]_{\clg{E}},
  [(\clb{c},\clb{d})]_{\clg{E}} \in \clm{\Z}$, we let
  \[
   [(\clr{a},\clr{b})]_{\clg{E}} + [(\clb{c}, \clb{d})]_{\clg{E}} = [(\clr{a} +
   \clb{c}, \clr{b} + \clb{d})]_{\clg{E}}.
  \]
  Finally, we define the \alert{opposite number} to $[(a,b)]_{\clg{E}}$ as
  $-[(a,b)]_{\clg{E}} = [(b,a)]_{\clg{E}}$ (this is because $-(a-b) = b-a$). The
  \alert{subtraction} of two integers is now just a sum of the first and the
  opposite of the second, that is
  \[
   [(\clr{a},\clr{b})]_{\clg{E}} - [(\clb{c},\clb{d})]_{\clg{E}} =
   [(\clr{a},\clr{b})]_{\clg{E}} + (-[(\clb{c},\clb{d})]_{\clg{E}}) =
   [(\clr{a},\clr{b})]_{\clg{E}} + [(\clb{d},\clb{c})]_{\clg{E}} = [(\clr{a} +
   \clb{d}, \clr{b} + \clb{c})]_{\clg{E}}.
  \]
  For example,
  \[
   [(\clr{3}, \clr{1})]_{\clg{E}} - [(\clb{5},\clb{2})]_{\clg{E}} = [(\clr{3},
   \clr{1})]_{\clg{E}} + [(\clb{2},\clb{5})]_{\clg{E}} = [(\clr{3} + \clb{2},
   \clr{1} + \clb{5})]_{\clg{E}} = [(5, 6)]_{\clg{E}},
  \]
  which is the same as writing
  \[
   \clr{2} - \clb{3} = \clr{2} + (-\clb{3}) = -1.
  \]
 \end{exampleblock}

\end{column}

\separatorcolumn

\begin{column}{\colwidth}

\begin{block}{Multiplication}
 In a way similar to addition, we can define \alert{multiplication} on natural
 numbers by the following two rules.
 \begin{enumerate}[label=(\arabic*),left=12pt]
  \item $n \cdot 1 = n$,
  \item $n \cdot \s(m) = n \cdot m + m$,
 \end{enumerate}
 for $n,m \in \N$. They carry the idea behind an algorithmic way to compute the
 product $n \cdot m$ for any two natural numbers $n,m$. It goes like this:
 \begin{itemize}[left=12pt,label=\textbullet]
  \item Using rule (1), calculate $n \cdot 1 = n$.
  \item Using rule (2), calculate $n \cdot 2 = n \cdot \s(1) = n \cdot 1 + n = n
   + n$.
  \item Continue like this until you calculate
   \[
    n \cdot m = n \cdot \s(m - 1) = n \cdot (m - 1) + n = \underbrace{n + n +
    \ldots + n}_{(m - 1)\text{ times}} + n.
   \]
 \end{itemize}
 For example, to calculate $4 \cdot 3$, we first multiply $4 \cdot 1 = 4$, then
 $4 \cdot 2 = 4 \cdot \s(1) = 4 \cdot 1 + 4 = 4 + 4$, and finally $4 \cdot 3 = 4
 \cdot \s(2) = 4 \cdot 2 + 4 = (4 + 4) + 4$. As you've been taught:
 `multiplication is just repeated addition'.

 Multiplication is easily extended to integers by the formula
 \[
  [(\clr{a},\clr{b})]_{\clg{E}} \cdot [(\clb{c},\clb{d})]_{\clg{E}} = [(\clr{a}
  \cdot \clb{c} + \clr{b} \cdot \clb{d}, \clr{b} \cdot \clb{c} + \clr{a} \cdot
  \clb{d})]_{\clg{E}}.
 \]
 The formula is based on the calculation
 \[
  (\clr{a} - \clr{b}) \cdot (\clb{c} - \clb{d}) = \clr{a} \cdot \clb{c} -
  \clr{b} \cdot \clb{c} - \clr{a} \cdot \clb{d} + \clr{b} \cdot \clb{d} =
  (\clr{a} \cdot \clb{c} + \clr{b} \cdot \clb{d}) - (\clr{b} \cdot \clb{c} +
  \clr{a} \cdot \clb{d}).
 \]

 For example,
 \[
  [(\clr{5},\clr{3})]_{\clg{E}} \cdot [(\clb{1},\clb{5})]_{\clg{E}} =
  [(\clr{5} \cdot \clb{1} + \clr{3} \cdot \clb{5}, \clr{3} \cdot \clb{1} +
  \clr{5} \cdot \clb{5})]_{\clg{E}} = [(20, 28)]_{\clg{E}}.
 \]
 This is the same calculation as
 \[
  \clr{2} \cdot (\clb{-4}) = -8.
 \]
\end{block}

\begin{exampleblock}{Rational Numbers}
 Being able to \alert{multiply integers}, we'd like to divide them as well. As
 was the case with natural numbers and subtraction, \alert{division is not an
 operation on integers} because its result needn't be an integer.

 The idea behind the definition of rational numbers (labelled $\Q$) is pretty
 much the same as the one behind the definition of integers -- rational numbers
 are really just \alert{pairs of integers}. And again, multiple pairs of
 integers \alert{represent the same} rational number. Therefore, given pairs
 $(a,b)$ and $(c,d)$ with $a,b,c,d \in \Z$, we must make sure that $(a,b)$
 \alert{is equivalent to} $(c,d)$ if `the fraction $a / b$ is the same as the
 fraction $c / d$'.

 As we couldn't have defined division yet, we must \alert{rewrite the last
 equation in terms of multiplication only}. This is easy to do because $a / b =
 c / d$ if $a \cdot d = c \cdot b$. This directly leads to the definition of an
 \alert{equivalence} $\clg{Q}$ on pairs of integers: $(\clr{a},
 \clr{b})\clg{Q}(\clb{c},\clb{d})$ if
 \[
  \clr{a} \cdot \clb{d} = \clr{c} \cdot \clb{b}.
 \]
 This is indeed an equivalence on $\Z \times \Z$ and we define $\clm{\Q}$ as
 \[
  \clm{\Q} = \{[(a,b)]_{\clg{Q}} \mid a,b \in \Z\}.
 \]
 We tend to write elements of $\clm{\Q}$ as \textbf{fractions}, that is, instead
 of $[(a,b)]_{\clg{Q}}$, we write $a / b$. We shall adopt this notation
 henceforth.

 It only remains to \alert{extend addition and multiplication} to rational
 numbers. This is easily done using formulae you already know. For example, the
 \alert{product} of two rational numbers $a / b, c / d \in \clm{\Q}$ is defined
 as such:
 \[
  \frac{\clr{a}}{\clr{b}} \cdot \frac{\clb{c}}{\clb{d}} = \frac{\clr{a} \cdot
  \clb{c}}{\clr{b} \cdot \clb{d}}.
 \]
 The \alert{sum} of rational numbers as
 \[
  \frac{\clr{a}}{\clr{b}} + \frac{\clb{c}}{\clb{d}} = \frac{\clr{a} \cdot
  \clb{d} + \clb{c} \cdot \clr{b}}{\clr{b} \cdot \clb{d}}.
 \]
 For example,
 \[
  \frac{\clr{2}}{\clr{5}} \cdot \frac{\clb{3}}{\clb{4}} = \frac{\clr{2} \cdot
  \clb{3}}{\clr{5} \cdot \clb{4}} = \frac{6}{20} \quad \text{and} \quad
  \frac{\clr{2}}{\clr{5}} + \frac{\clb{3}}{\clb{4}} = \frac{\clr{2} \cdot
  \clb{4} + \clb{3} \cdot \clr{5}}{\clr{5} \cdot \clb{4}} = \frac{23}{20}.
 \]
 
 Finally, we're ready to \alert{define division} on rational numbers. We first
 define the \alert{inverse} of a rational numbers $a / b$ as $b / a$. We write
 $b / a = (a / b)^{-1}$. The \alert{operation of division} on rational numbers
 is defined as \alert{multiplication by the inverse element}, that is
 \[
  \frac{\clr{a}}{\clr{b}} : \frac{\clb{c}}{\clb{d}} = \frac{\clr{a}}{\clr{b}}
  \cdot \left( \frac{\clb{c}}{\clb{d}} \right)^{-1} = \frac{\clr{a}}{\clr{b}}
  \cdot \frac{\clb{d}}{\clb{c}} = \frac{\clr{a} \cdot \clb{d}}{\clr{b} \cdot
  \clb{c}}.
 \]
 For example,
 \[
  \frac{\clr{2}}{\clr{5}} : \frac{\clb{3}}{\clb{4}} = \frac{\clr{2}}{\clr{5}}
 \cdot \left(\frac{\clb{3}}{\clb{4}}\right)^{-1} = \frac{\clr{2}}{\clr{5}} \cdot
 \frac{\clb{4}}{\clb{3}} = \frac{\clr{2} \cdot \clb{4}}{\clr{5} \cdot \clb{3}} =
 \frac{8}{15}.
 \]
\end{exampleblock}

\begin{alertblock}{Divisibility}
 There are two more interesting operations on integers -- \alert{integer
 division} and \alert{modulus}. You've learnt about them in elementary school
 and they are basically `a way to do \alert{division on integers}'.

 Given two integers: $a,b \in \Z$, we can ask `\textbf{How many times does $b$
 fit into $a$?}'. This number is called the \alert{quotient} (of $a$ by $b$) and
 denoted \alert{$a \dv b$}. For example, $7 \dv 2 = 3$ because $2$ fits $3$
 times into $7$ or $10 \dv 8 = 1$ because $8$ fits only once into $10$. This
 operation can also be used with integers, for instance $-5 \dv 2 = -2$ because
 $2$ `fits' $-2$ times into $-5$.

 The quantity that is `left after integer division' is called the
 \alert{remainder} and denoted $a \bmod b$. In our first example, $7 \bmod 2 =
 1$ because $7 \dv 3 = 2$ and $2 \cdot 3 = 6$, so the number $1$ is left after
 performing the division. Similarly, $10 \bmod 8 = 2$ since $10 \dv 8 = 1$ and
 $8 \cdot 1 = 8$ and $10 - 8 = 2$. In the last example, we get $-5 \bmod 2 = -1$
 as $-5 \dv 2 = -2$ and $2 \cdot (-2) = -4$.

 Notice that the \alert{remainder must always be smaller (in absolute value)
 than the divisor} because it's the quantity that's left after the divisor no
 longer fits into the dividend. We may thus formalise the idea of \alert{integer
 division as such}: for any two numbers $a,b \in \Z$ there are \textbf{unique}
 numbers $q,r \in \Z$ (called \alert{quotient} and \alert{remainder}) such that
 \[
  a = b \cdot q + r
 \]
 and $0 \leq |r| < |b|$. We write $q = a \dv b$ and $r = a \bmod b$.

 The operation of integer division gives birth to the idea of
 \alert{divisibility}. We say that a number $b$ \alert{divides} the number $a$
 (and write $b \mid a$) if $a \bmod b = 0$ or, equivalently, $a = b \cdot q$ for
 some integer $q \in \Z$. 

 If $b \mid a$, we say that $b$ is a \alert{divisor} of $a$. A number that
 \alert{has exactly two divisors} (the number $1$ and itself) is said to be
 \alert{prime}. If two numbers $x,y \in \Z$ \alert{share no divisors}, they are
 called \emph{coprime}.
\end{alertblock}

\end{column}
\separatorcolumn

\begin{column}{\colwidth}

\begin{block}{Prime Decomposition}
 Each integer $a \in \Z$ has a \alert{unique prime decomposition}, meaning, it
 can be written as a \alert{product of prime numbers}.

 Expressed formally, for every number $a \in \Z$, we can find \textbf{prime}
 numbers $p_1,\ldots,p_n \in \Z$ and powers $k_1,\ldots,k_n \in \N$ satisfying
 \[
  a = p_1^{k_1} \cdot p_2^{k_2} \cdot p_3^{k_3} \cdot \ldots \cdot p_n^{k_n}
 \]
 where $n \in \N$ is an adequate natural number.

 A very important assumption of modern cryptography is that decomposition into
 primes is a very slow process for which no fast algorithm is known. This is the
 basis for several cryptographic methods including the RSA system we shall
 discuss in class.
\end{block}

\begin{exampleblock}{Greatest Common Divisor}
 We ask the question: `What's the \alert{largest number that divides} both $a$
 and $b$?' Such number is called the \alert{greatest common divisor} of $a$ and
 $b$ and written $\mathrm{gcd}(a,b)$.

 There is a nice \alert{algorithmic way} to compute this number: the algorithm
 is called \alert{Euclid's Algorithm}. It uses the following equality: for every
 two numbers, $a,b \in \Z$, it holds that
 \[
  \mathrm{gcd}(a,b) = \mathrm{gcd}(a \bmod b, b).
 \]
 Since $a \bmod b$ is always smaller than both $a$ and $b$, we keep taking the
 remainder after division of the larger number by the smaller until we reach the
 number $0$. Once we do, we have found the greatest common divisor as
 $\mathrm{gcd}(d,0) = d$ ($0$ is divided by every number).

 Let's showcase the algorithm. Suppose we want to calculate $\mathrm{gcd}(3312,
 448)$. We first calculate $3312 \bmod 448 = 176$. Therefore
 \[
  \mathrm{gcd}(3312, 448) = \mathrm{gcd}(176, 448).
 \]
 Next, we compute $448 \bmod 176 = 96$ and thus
 \[
  \mathrm{gcd}(176, 448) = \mathrm{gcd}(176, 96).
 \]
 We are almost done. Computing $176 \bmod 96 = 80$ and $96 \bmod 80 = 16$ gives
 \[
  \mathrm{gcd}(176, 96) = \mathrm{gcd}(80, 96) = \mathrm{gcd}(80, 16).
 \]
 Because $80 \bmod 16 = 0$, we have reached the conclusion that
 $\mathrm{gcd}(3312, 448) = 16$.
\end{exampleblock}

\begin{block}{Congruence}
 Now that we have defined the remainder after division, we want to express the
 idea that two integers (say, $x$ and $y$) \alert{have the same remainder} ($r$)
 \alert{when divided by some $m \in \Z$}. Mathematically, we write this as
 \[
  x \equiv y \pmod{m}.
 \]
 In other words, this means
 \[
  x = k \cdot m + r \quad \text{and} \quad y = l \cdot m + r
 \]
 for some numbers $k,l \in \Z$. Said `naturally', both $x$ and $y$ are greater
 by $r$ than some multiple of $m$. For example, $13$ and $25$ are the same
 modulo $12$ because they share the remainder $1$ when divided by $12$ (formally
 written as  $13 \equiv 25 \pmod{12}$). So, $13$ is greater by $1$ than $12$ (a
 multiple of $12$) and so is $25$ than $24$ (also a multiple of $12$). 

 This, of course, implies that a number $x$ is \alert{divisible by $m$} if
 \alert{it is congruent to $0$} modulo $m$. This alludes to our previous
 definition where we said that $m$ divides $x$ only if there is no remainder
 after the division of $x$ by $m$.

 This idea of congruence might sound unintuitive and artificial at first, yet it
 is all around us. If we, for example, consider the regular old clock. Looking
 only at the clock and seeing both hands up tells us that it is either noon or
 midnight. This is because we use the 12-hour format which describes current
 time modulo $12$.

 \alert{Congruence is very similar to normal equation} (it is also an
 equivalence relation, try to prove it!). We can manipulate it as we would an
 equation. More specifically, if we know that $x \equiv y \pmod{m}$, then both
 of the following congruences hold.
 \begin{itemize}[label=\textbullet,left=24pt]
  \item $x+a \equiv y+a \pmod{m}$ -- \textbf{adding} (also works for subtracting,
   so $a \in \mathbb{Z}$)
  \item $c \cdot x \equiv c \cdot y \pmod{m}$ -- \textbf{multiplying}
 \end{itemize}
 for any $c,k \in \N$.

 One can also \alert{simplify a congruence}, but this one is trickier. If
 \[
  n \cdot a \equiv n \cdot b, \pmod{m}
 \]
 then \alert{it is also true that} $a \equiv b \pmod{m}$ \alert{only in the case
 that} $n$ and $m$ \alert{are coprime}, i.e. $\gcd(m,n) = 1$. Let's make an
 example: if $35 \equiv 15 \pmod{4}$, then we can `divide' both sides by $5$ and
 get $7 \equiv 3 \pmod{4}$ because $\gcd(4,5) = 1$. However, $10 \equiv 2
 \pmod{8}$ but $5 \not\equiv 1 \pmod{8}$ since we have divided by $4$ which is
 \textbf{not} coprime to $8$.

 An interesting thing to note about the equivalence classes created by some
 congruence modulo $m$ is that there are $m$ of them. This is because all the
 possible remainders after diving by $m$ are numbers $0, \ldots, m - 1$. 
\end{block}

\begin{exampleblock}{Solving Congruences}
 We now attempt to \alert{solve the congruence} $7 \cdot x \equiv 5 \pmod{10}$.
 \alert{Solving a congruence} means a similar thing as \alert{solving an
 equation}. We're looking for a number $n \in \Z$ such that $x \equiv n
 \pmod{10}$.

 If this were just a normal equation, how would we solve it? Well, we would
 multiply both sides by such a number that $7$ times that number gives us $1$
 (that would obviously be $1 / 7$ but we care about the idea more than about the
 result). Even though \alert{congruence is a relation between integers} and
 \alert{we mustn't use rational numbers}, the idea is still useful.

The number $n \in \Z$ that satisfies $7 \cdot n \equiv 1 \pmod{10}$ is called
\alert{inverse} of $7$ modulo $10$. There is no straightforward way to find
\alert{an inverse}. To calculate it, we try multiplying $7$ by increasing
integers and see which result is $1$ modulo $10$. 
\begin{equation*}
   \begin{split}
    2 \cdot 7 & \equiv 4 \pmod{10}\\
    3 \cdot 7 & \equiv 1 \pmod{10}
   \end{split}
  \end{equation*}
This concludes that $3$ is the \alert{inverse} to $7$ $\bmod~10$. Now \alert{we
multiply the whole equation by $3$} and get the desired result.
\[
 \begin{array}{r l r}
  7 \cdot x &\equiv 5 & \pmod{10}\\
  3 \cdot 7 \cdot x & \equiv 3 \cdot 5 & \pmod{10}\\
  21 \cdot x & \equiv 15 &\pmod{10}\\
  x & \equiv 5 &\pmod{10}
 \end{array}
\]
We got the fourth congruence from the third by calculating $21 \bmod 10 = 1$ and
$15 \bmod 10 = 5$.

One unfortunate thing about inverses is that they are not guaranteed to exist
for every number. If, for example, we try to find the inverse of $2$ modulo $4$,
we have to conclude that there is no such a number. The \alert{inverse} for $a$
modulo some $m$ (this can be written as: the solution to the congruence $a \cdot
x \equiv 1 \pmod{m}$) \alert{exists if and only if} $a$ and $m$ \alert{are
coprime}.

Now that we can solve a congruence, there is nothing stopping us from solving
more of them.
\end{exampleblock}
\end{column}
\separatorcolumn
\begin{column}{\colwidth}
\begin{alertblock}{Chinese Remainder Theorem}
 Imagine we have the \alert{system of linear congruences}
 \[
  \begin{array}{r l l}
   x & \equiv r_1  &\pmod{m_1}\\
   x & \equiv r_2  &\pmod{m_2}\\
     & \;\vdots &\\
   x & \equiv r_n, &\pmod{m_n}
  \end{array}
 \]
 where all the numbers are natural and $m_1,\ldots, m_n$ are \alert{pairwise
 coprime}. Then the \alert{Chinese Remainder Theorem} tells us that there is
 unique solution smaller than the product $M = m_1 \cdot m_2 \cdot \ldots \cdot
 m_n$ of all the numbers we divide by.

 \alert{Each congruence limits the possible solutions radically}. For instance,
 the congruence $x \equiv r \pmod{m}$ has solutions in the form $k \cdot m + r$
 for any $k \in \N$. To solve the whole system, one can \alert{write down all
 the solutions for all the congruences one by one and then find their
 intersection}.

We can also do this process graphically. If we circle solutions to the
individual congruences, the number with $n$ circles is the solution to the whole
system. For example, we have the linear congruences $\clr{x \equiv 1 \pmod{3}}$,
$\clb{x \equiv 2 \pmod{4}}$. Their solutions are depicted in the picture below.

\begin{center}
 \begin{tikzpicture}
   \def\rows{4}
   \def\cols{8}
   \def\sep{1.4}

   \foreach \x in {1, ..., \cols} {
    \foreach \y in {1, ..., \rows} {
     \pgfmathtruncatemacro\z{(\rows-\y) * \cols + \x}
     \node[black] (\z) at (\x*\sep,\y*\sep) {\small \z};
    }
   }

   \foreach \x/\y in {1/4,4/4, 7/4, 2/3, 5/3, 8/3, 3/2, 6/2, 1/1, 4/1, 7/1 } {
   \node[circle,inner sep=11pt, draw =BrickRed,line width= 2pt]  at
      (\x*\sep,\y*\sep) {};
   }
    
   \foreach \x/\y in {6/4,2/4, 6/3, 2/2, 2/1, 6/1 } {
     \node[circle,inner sep=11pt, draw =RoyalBlue,line width= 2pt] at
      (\x*\sep,\y*\sep) {};
   }

     \node[circle,inner sep=13.5pt, draw =RoyalBlue,fill=yellow,fill opacity=0.2,line width= 2pt] at
      (2*\sep,3*\sep) {};
     
     \node[circle,inner sep=13.5pt, draw =RoyalBlue,fill=yellow,fill opacity=0.2,line width= 2pt] at
      (6*\sep,2*\sep) {};

     \draw[line width = 2pt] (12)+(0.6,0.6) rectangle ++(-0.6,-0.6);

     \draw[line width = 2pt] (24)+(0.6,0.6) rectangle ++(-0.6,-0.6);
 \end{tikzpicture}
\end{center}
Every circle shows a solution to one of the congruence based on their colour.
The overall solution has two circles and is tinted yellow. The box around $12$
and $24$ indicates on which intervals we are guaranteed to have a unique
solution. This is because $M = m_1 \cdot m_2 = 3 \cdot 4 = 12$.

To draw all the solutions to the congruence $x \equiv r \pmod{m}$, it is useful
to note that the first (or the smallest) solution will always be $r$ and then
the solutions always jump by $m$. Hence, the next will be $r+m$, the one after
that $r+2 \cdot m$ and so on.

By the Chinese Remainder Theorem, we are only guaranteed a \alert{unique
solution up to $M$}. But the \alert{system itself has infinitely many
solutions}. To find all of them, we have to start with the smallest solution $x$
(the one smaller than $M$), and then keep adding multiples of $M$. More
precisely, \alert{all solutions are of the form $x + k \cdot M$} for any $k \in
\N$.
\end{alertblock}
 \begin{block}{Solving Congruence Systems}
  Now, to showcase a more scalable but less intuitive method, we shall solve the system
  \[
   \begin{array}{r l l}
    x & \equiv 3  &\pmod{7}\\
    x & \equiv 5  &\pmod{9}\\
    x & \equiv 4. &\pmod{11}
   \end{array}
  \]
  
  \alert{From the first congruence}, we know that $x= 7 \cdot k +3$ for some $k
  \in \N$. We can now substitute for $x$ to the second congruence and solve it.
\begin{equation*}
   \begin{aligned}
    7 \cdot k +3 & \equiv 5 \pmod{9}  && \qquad \text{\footnotesize{\# Substituting}} \\
    7 \cdot k &\equiv 2 \pmod{9}&& \qquad \text{\footnotesize{\# Subtracting $3$}}  \\
    k &\equiv 8 \pmod{9} &&\qquad \text{\footnotesize{\# Multiplying by the
     inverse of $7$ modulo $9$, which is $4$.}}
   \end{aligned}
  \end{equation*}
 Now we know that $k = 9 \cdot l + 8$ for some $l \in \N$. This expression is
 now used to express $x$ in terms of $l$ as $x = 7 \cdot (9 \cdot l+8)+3 = 63
 \cdot l + 59$. There is still one equation that we have not used. We calculate
 $63 \bmod 11 = 8$ and $59 \bmod 11 = 4$. Hence, $x \equiv 8 \cdot l + 4
 \pmod{11}$. Substituting to the last congruence gives us
\begin{equation*}
   \begin{aligned}
    8 \cdot l + 4 & \equiv 4 & \pmod{11}  && \qquad    \text{\footnotesize{\# Substituting}} \\
    l &\equiv 0. & \pmod{11} 
   \end{aligned}
  \end{equation*}
 With this we can express $l = 11 \cdot m$ in terms of some other variable $m$
 and finally see the answer
 \[
  x = 63 \cdot (11 \cdot m) + 59 = 693 \cdot l + 59.
 \]
 Notice that the coefficient before $m$ is equal to $693 = 7\cdot 9 \cdot 11$.
 The CRT tells us that \alert{this will be the case for every system where the
 divisors are coprime}.
 \end{block}

\end{column}
\separatorcolumn

\end{columns}
\end{frame}

\end{document}
