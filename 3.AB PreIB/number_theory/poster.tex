% Unofficial University of Cambridge Poster Template
% https://github.com/andiac/gemini-cam
% a fork of https://github.com/anishathalye/gemini
% also refer to https://github.com/k4rtik/uchicago-poster
% TeX program = lualatex

\documentclass[final]{beamer}

% ====================
% Packages
% ====================

\usepackage[T1]{fontenc}
\usepackage{lmodern}
\usepackage[orientation=portrait,size=custom,width=120,height=100,scale=1]{beamerposter}
\usetheme{gemini}
\usepackage[dvipsnames]{xcolor}
\usecolortheme{nott}
\usepackage{graphicx}
\usepackage{booktabs}
\usepackage{tikz}
\usetikzlibrary{calc,arrows.meta,patterns,decorations.pathmorphing,shapes.geometric}
\usepackage{tkz-euclide}
\tikzset{point style/.style = {%
  draw = black,
  inner sep = 0pt,
  shape = circle,
  minimum size = 5pt,
  fill = black
 },
 every picture/.append style = {
  scale = 1.5
 },
 every node/.append style={
  scale=1.5
 }
}
\usepackage{pgfplots}
\pgfplotsset{compat=1.14}
\usepackage{anyfontsize}
\usepackage{caption}
\usepackage{subcaption}

% ====================
% Lengths
% ====================

% If you have N columns, choose \sepwidth and \colwidth such that
% (N+1)*\sepwidth + N*\colwidth = \paperwidth
\newlength{\sepwidth}
\newlength{\colwidth}
\setlength{\sepwidth}{0.01\paperwidth}
\setlength{\colwidth}{0.32\paperwidth}

\newcommand{\separatorcolumn}{\begin{column}{\sepwidth}\end{column}}
\newcommand{\bfalert}[1]{\textbf{\alert{#1}}}

% Math shortcuts
\newcommand{\R}{\mathbb{R}}

% Inline shapes
\newcommand{\mysquare}{\tikz[baseline=-7pt]{%
  \node[rectangle,draw,thick,inner sep=6pt] at (0,0) {};
}}
\newcommand{\mytria}{\tikz[baseline=-3.25pt]{%
  \node[isosceles triangle,isosceles triangle apex angle=60,draw,thick,inner
  sep=3.25pt,rotate=90] at (0,0) {};
}}
\newcommand{\mycirc}{\tikz[baseline=-7pt]{%
  \node[circle,draw,thick,inner sep=4.5pt,baseline=0.5ex,rotate=90] at (0,0) {};
}}
\newcommand{\mycross}{\tikz[baseline=-7pt,scale=0.2]{%
  \draw[thick] (-1,1) -- (1,-1);
  \draw[thick] (-1,-1) -- (1,1);
}}

% Colors %
\newcommand{\clr}{\textcolor{BrickRed}}
\newcommand{\clb}{\textcolor{RoyalBlue}}
\newcommand{\clg}{\textcolor{ForestGreen}}
\newcommand{\clm}{\textcolor{Fuchsia}}
\newcommand{\clv}{\textcolor{violet}}
\newcommand{\clbr}{\textcolor{Sepia}}
\newcommand{\cly}{\textcolor{Dandelion}}

% ====================
% Definitions
% ====================

\newcommand{\N}{\mathbb{N}}

% ====================
% Title
% ====================

\title{Logic \& Set Theory Cheatsheet}

\author{3.AB PreIB Math}

\institute[shortinst]{Adam Klepáč}

% ====================
% Footer (optional)
% ====================

% \footercontent{
%   \href{https://utfpr.edu.br/ct/ppgca}{utfpr.edu.br/ct/ppgca} \hfill
%   Mostra de Trabalhos do PPGCA --- TechTalks 2024 \hfill
%   \href{mailto:ppgca-ct@utfpr.edu.br}{ppgca-ct@utfpr.edu.br}}
% (can be left out to remove footer)


% ====================
% Logo (optional)
% ====================

% use this to include logos on the left and/or right side of the header:
\logoright{\includegraphics[height=3.5cm]{logos/logo-white.png}}
% \logoleft{\hspace{20ex}\includegraphics[height=3.5cm]{logos/ppgca-logo.png}}

% ====================
% Body
% ====================

\begin{document}

% Refer to https://github.com/k4rtik/uchicago-poster
% logo: https://www.cam.ac.uk/brand-resources/about-the-logo/logo-downloads
% \addtobeamertemplate{headline}{}
% {
%     \begin{tikzpicture}[remember picture,overlay]
%       \node [anchor=north west, inner sep=3cm] at ([xshift=-2.5cm,yshift=1.75cm]current page.north west)
%       {\includegraphics[height=7cm]{logos/unott-logo.eps}}; 
%     \end{tikzpicture}
% }

\begin{frame}[t]
\begin{columns}[t]
\separatorcolumn

\begin{column}{\colwidth}

 \begin{block}{Natural Numbers}
  \alert{Logic} is the language of mathematics. It uses \alert{propositions} to
  talk about sets.

  Propositions are sentences which can be either true or false. For example
  \begin{itemize}[label=\textbullet,left=24pt]
   \item `\textbf{Cats are black.}' is a proposition;
   \item `\textbf{How are you?}' is \emph{not} a proposition;
   \item `\textbf{We will have colonised Mars by 2500.}' is also a proposition.
  \end{itemize}
 \end{block}
 As the third example suggests, we need not necessarily know whether a
 proposition is true or false -- it remains a proposition anyway.

 \vspace{1em}

 \begin{exampleblock}{Whole numbers}
  Propositions can be joined together using \alert{logical conjunctions}. They
  pretty much correspond to the conjunctions of natural language. Let us
  consider two propositions:
  \begin{align*}
   p &= \text{`It's raining outside.'}\\
   q &= \text{`I'll stay at home.'}
  \end{align*}
  \begin{itemize}[left=40pt]
   \item[($ \wedge $)] Logical \alert{and} forms a proposition that is only
    \alert{true} if both of its constituents are also \alert{true}. In natural
    language, the proposition $p \wedge q$ can be expressed as
    \[
     p \alert{ \wedge } q = \text{`It's raining outside \alert{and} I'll stay at
     home.'}
    \]
   \item[($ \vee $)] Logical \alert{or} forms a proposition that is \alert{true}
    if at least one of its constituents is \alert{true}. In natural language,
    the proposition $p \vee q$ can be expressed as
    \[
     p \alert{ \vee } q = \text{`It's raining outside \alert{or} I'll stay at
     home.'}
    \]
    In mathematical logic, \alert{or} is \textbf{not exclusive}! This means that
    $p \alert{ \vee } q$ is true even if both $p$ and $q$ are true.
   \item[($\neg $)] Logical \alert{not} isn't strictly speaking a conjunction
    but I include it anyway. It reverses the truth value of a proposition. For
    example, the proposition $\alert{\neg }p$ can be read as
    \[
     \alert{\neg }p = \text{`It's \alert{not} raining outside.'}
    \]
    It follows that $\alert{\neg }p$ is \alert{true} exactly when $p$ is
    \alert{false} and vice versa.
   \item[($ \Rightarrow $)] Logical \alert{implication} is a conjunction that
    makes the first proposition into an \emph{assumption} or \emph{premise} and
    the second one into a \emph{conclusion}. The proposition $p \alert{
    \Rightarrow } q$ is read in multiple ways, to list a few: 
    \begin{align*}
     p \alert{ \Rightarrow } q &= \text{`\alert{If} it's raining outside,
    \alert{then} I'll stay at home.'}\\
     p \alert{ \Rightarrow } q &= \text{`It raining outside \alert{implies that}
     I'll stay at home.'}\\
     p \alert{ \Rightarrow } q &= \text{`\alert{Assuming} it's raining outside,
     I'll stay at home.'}\\
    \end{align*}
    The implication is tricky. It's true if both $p$ and $q$ are true and false
    if $p$ is true but $q$ is false. However, it is \alert{always true} if $p$
    is \alert{false}. That is because, in mathematical logic, whatever follows
    from a lie is automatically true.
   \item[($ \Leftrightarrow $)] Logical \alert{equivalence} is true only if both
    propositions have the \alert{same truth value} -- they're both true or both
    false. In natural language, it is typically read like this:
    \[
     p \alert{ \Leftrightarrow }q = \text{`It's raining \alert{if and only if}
     I stay at home.'}
    \]
    Equivalence is basically just a two-way implication. The proposition $p$ is
    both a premise and a conclusion to $q$ and $q$ is both a premise and a
    conclusion to $p$. If it's raining outside, I stay at home and if I stay at
    home, then it's raining outside.
  \end{itemize}
 \end{exampleblock}

 \begin{alertblock}{Rational numbers}
  Toto je o racionalnich cislech.
 \end{alertblock}

 \begin{block}{Divisibility}
  A conjunction of propositions being true or false based on whether its
  constituent propositions are true or false can be summarized using so-called
  \alert{truth table}. It is basically just a table that lists all the
  possibilities of $p$ and $q$ being true or false and the resulting truth value
  of their conjunctions.

  For the basic logical conjunctions from above, it can look like this (we
  represent \alert{true} by \alert{1} and \alert{false} by \alert{0}):
  \begin{center}
   \begin{tabular}{c | c | c | c | c | c | c | c}
    $p$ & $q$ & $\neg p$ & $\neg q$ & $p \wedge q$ & $p \vee q$ & $p \Rightarrow
    q$ & $p \Leftrightarrow q$\\
    \toprule
    0 & 0 & 1 & 1 & 0 & 0 & 1 & 1\\
    \midrule
    0 & 1 & 1 & 0 & 0 & 1 & 1 & 0\\
    \midrule
    1 & 0 & 0 & 1 & 0 & 1 & 0 & 0\\
    \midrule
    1 & 1 & 0 & 0 & 1 & 1 & 1 & 1
   \end{tabular}
  \end{center}
 \end{block}
\end{column}

\separatorcolumn

\begin{column}{\colwidth}

\begin{exampleblock}{Prime decomposition}
 \alert{Sets} are the `stuff' that makes up the world of mathematics. Their
 basic characteristics and properties are described using \alert{logic}.

 Sets cannot be defined inside set theory but we interpret them as \emph{groups
 of things}.

 There's only one foundational \emph{proposition} related to set theory -- the
 proposition `\alert{An object is an element of a set.}' If we label the object
 in question $x$ and the set $A$, this proposition is written as $x \in A$ (the
 symbol $ \in $ is just the letter `e' in `element'). Combining these
 propositions using logical conjunctions allows for various set-theoretic
 constructions.

 If a set $A$ has, for example, exactly three elements -- $\mysquare$, $\mytria$
 and $\mycirc$, I can write it as a list of these three elements inside curly
 brackets $\{\}$. In this case,
 \[
  A = \{\mysquare,\mytria,\mycirc\}.
 \]

 A few \alert{warnings} about sets:
 \begin{itemize}[label=\textbullet,left=24pt]
  \item \textbf{Sets are not ordered}. There is nothing like a `first', `second'
   or `last' element of a set. Either an object \textbf{is} inside a set or it
   \textbf{isn't}. Nothing else. For example, the three sets below are
   \alert{exactly the same}, only written differently.
   \[
    \{\mysquare,\mytria,\mycirc\} = \{\mycirc,\mytria,\mysquare\} =
    \{\mytria,\mysquare,\mycirc\}
   \]
  \item \textbf{Elements of sets have no frequency}. Again, an element either is
   inside a set or not. It cannot be \alert{twice} in a set, for example. The
   three sets below are exactly the same.
   \[
    \{\mysquare,\mytria,\mycirc\} =
    \{\mysquare,\mytria,\mycirc,\mytria,\mycirc\} = \{
    \mytria,\mysquare,\mysquare,\mytria,\mycirc,\mytria\}
   \]
 \end{itemize}
\end{exampleblock}

\begin{alertblock}{Euler's algorithm}
\end{alertblock}




\begin{block}{Congruence}
 Now that we have defined the remainder after division we want to express the
 idea that two numbers ($x$ and $y$) \textbf{have the same remainder} ($r$) \textbf{when divided
 by some $m$.} Mathematically we write this as
 \[
  x \equiv y \pmod{m}.
 \]
 In other words this means
 \begin{align*}
  x = km + r && \text{and}  && y = lm + r
 \end{align*}
 for some numbers $l,k \in \N$. This way we can for example say that 13 and 25
 are the same modulo 12, because they share the remainder 1 when divided by 12
 (formally written as  $13 \equiv 25 \pmod{12}$).

 This of course implies that each number divisible by $m$ is congruent to 0
 $\bmod ~m$. This makes sense because we said that $a$ divides $b$ only if there
 is no remainder after the division.

 This idea of congruence might sound unintuitive and artificial at first, yet it
 is all around us. If we for example take the regular old clock. Looking only at
 the clock and seeing both hands up tells us that it is either noon or midnight.
 This is because we use the 12 hours format which gives the time $ \bmod~12$.

Congruence is very similar to normal equation (it is also
equivalence, try to prove it!). Similar to equations we can manipulate it.
More specifically if i have $x \equiv y \pmod{m}$ then all the oncoming
congruences hold.
\begin{itemize}[label=\textbullet,left=24pt]
 \item $x+a \equiv y+a \pmod{m}$ - \textbf{adding} also works for subtracting so
  $a \in \mathbb{Z}$
 \item $x^k \equiv y^k \pmod{m}$ - \textbf{exponentiation}
 \item $cx \equiv cx \pmod{m}$ - \textbf{simplification}
\end{itemize}
for any $k,c \in \N$ and $GCD(m,c)=1$.

The last operation better be explained by an example. If we take $1 \equiv 6
\pmod{5}$ and multiply it by 2, we get $2 \equiv 12 \bmod{5}$. Notice that
multiplying the modulus is optional and the congruence holds in both cases
(because obviously $2 \equiv 12 \bmod{10}$). This
is because $GCD(2,5)=1$.

An interesting thing to note about the equivalence classes created by some
congruence $\bmod~m$ is that there will be $m$ of them. This is because all the
possible remainders after diving by $m$ are numbers $0, \ldots, m-1$. 

\end{block}  
\end{column}

\separatorcolumn
\begin{column}{\colwidth}

\begin{exampleblock}{Solving congruences}
 With this we can now attempt to solve the congruence $7x \equiv 5 \pmod{10}$. 

 If this was just a normal equation how would we solve it? Well we would multiply
both sides by such a number that 7 times that number gives us 1 (that would
obviously be $1 / 7$ but we want the idea more then the result). Even though that we are working
$\bmod~10$ and cannot use rational numbers, the idea is still useful.

The number that satisfies $7x \equiv 1 \pmod{10}$ is called \textbf{inverse} of
7 $\bmod~10$. To calculate the \textbf{inverse} we try
to multiply 7 by increasing integers and see which result is 1 $\bmod~10$. 
\begin{equation*}
   \begin{split}
    2 \cdot 7 & \equiv 4 \pmod{10}\\
    3 \cdot 7 & \equiv 1 \pmod{10}
   \end{split}
  \end{equation*}
This concludes that 3 is the \textbf{inverse} to 7 $\bmod~10$. Now we multiply
the whole equation by 3 and get the final result.
\[
 21x \equiv x \equiv 15 \equiv 5 \pmod{10}
\]

One unfortunate thing about inverses is that they are not guaranteed to exist for
every number. If for example we try to find the inverse of 2 $\bmod~4$ we have
to conclude that there is no such a number. The inverse for $a$ modulo some $m$
(this can be written as: the solution to the congruence $ax \equiv 1 \pmod{m}$)
exists \textbf{if and only if} $a$ and $m$ are \textbf{coprime}.

Now that we can solve an congruence there is nothing stoping us from solving
more of them.
\end{exampleblock}


\begin{alertblock}{Chineese remainder theorem}
 Imagine we have a system of linear congruences:
   \begin{equation*}
   \begin{split}
    x & \equiv r_1 \pmod{m_1}\\
    x & \equiv r_2 \pmod{m_2}\\
      &~~~\vdots\\
    x & \equiv r_n \pmod{m_n}
   \end{split}
  \end{equation*}

 Where all the numbers are natural and $m_1,\ldots m_n$ are mutually coprime.
 Then the CRT tells us that there is unique solution $x$ smaller then the product of
 all the numbers we divide by $M = m_1 \cdot m_2 \ldots m_n$.

Each congruence limits the possible solutions radically. For example the
congruence $x \equiv r \pmod{m}$ has solutions in the form: $km + r$ for any $k
\in \N$. To solve the whole system, one can write down all the solutions for
all the congruences and then find their intersection.

We can do this process also graphically. If we circle solutions to the
individual congruences the number with $n$ circles is the solution to the whole
system. If, for example, we have the linear congruences $\clr{x \equiv 1 \pmod{3}}$,
 $\clb{x \equiv 2 \pmod{4}}$, we can draw

\begin{center}
 \begin{tikzpicture}
   \def\rows{4}
   \def\cols{8}
   \def\sep{1.4}

   \foreach \x in {1, ..., \cols} {
    \foreach \y in {1, ..., \rows} {
     \pgfmathtruncatemacro\z{(\rows-\y) * \cols + \x}
     \node[black] (\z) at (\x*\sep,\y*\sep) {\small \z};
    }
   }

   \foreach \x/\y in {1/4,4/4, 7/4, 2/3, 5/3, 8/3, 3/2, 6/2, 1/1, 4/1, 7/1 } {
   \node[circle,inner sep=11pt, draw =BrickRed,line width= 2pt]  at
      (\x*\sep,\y*\sep) {};
   }
    
   \foreach \x/\y in {6/4,2/4, 6/3, 2/2, 2/1, 6/1 } {
     \node[circle,inner sep=11pt, draw =RoyalBlue,line width= 2pt] at
      (\x*\sep,\y*\sep) {};
   }

     \node[circle,inner sep=13.5pt, draw =RoyalBlue,fill=yellow,fill opacity=0.2,line width= 2pt] at
      (2*\sep,3*\sep) {};
     
     \node[circle,inner sep=13.5pt, draw =RoyalBlue,fill=yellow,fill opacity=0.2,line width= 2pt] at
      (6*\sep,2*\sep) {};

     \draw[line width = 2pt] (12)+(0.6,0.6) rectangle ++(-0.6,-0.6);

     \draw[line width = 2pt] (24)+(0.6,0.6) rectangle ++(-0.6,-0.6);
 \end{tikzpicture}
\end{center}
Where every circle shows a solution to one of the congruence based on their
color. The overall solution has two circles and is tinted yellow. The box
around 12 and 24 indicates on what intervals are we guaranteed to have a unique
solution. This is because $M=m_1 \cdot m_2 = 3 \cdot 4 = 12$.

To draw all the solution to the congruence $x \equiv r \pmod{m}$ it is useful to
note that the first (or the smallest) solution will always be $x$ and then the
solutions always jump by $m$. So the next will be $x+m$, the next will be $x+2m$
and so on.

From the formulation of CFT we are only guaranteed an unique solution up to $M$.
But the system itself has infinitely many of them in similar to congruences. To
find them all we have to start with the smallest solution $x$ (the one smaller
then $M$), then all of them are in the form of $x + kM$ for any $k \in \N$.
 
 \end{alertblock}


 \begin{block}{Solving congruence systems}
  Now, to showcase more scalable but less intuitive method, the system
    \begin{equation*}
   \begin{split}
    x & \equiv 3 \pmod{7}\\
    x & \equiv 5 \pmod{9}\\
    x & \equiv 4 \pmod{11}
   \end{split}
  \end{equation*}
is solved. From the first congruence we know that $x= 7k +3$ for some $k \in
\N$. We can now substitute for $x$ to the second congruence and solve it.
\begin{equation*}
   \begin{aligned}
    7k +3 & \equiv 5 \pmod{9}  && \qquad    \text{\footnotesize{\# Substituting}} \\
    7k &\equiv 2 \pmod{9}&& \qquad \text{\footnotesize{\# Subtracting 3}}  \\
    k &\equiv 8 \pmod{9} &&\qquad \text{\footnotesize{\# Multiplying by inverse of 7 $\bmod~9$
    which is 4.}}
   \end{aligned}
  \end{equation*}
 Now we know that $k = 9l +8$ for some $l \in \N$. This expression is now used
 to express $x$ in terms of $l$ as $x=7(9l+8)+3 = 63l + 59$. There is
 still one equation that we have not used. Substituting to the last congruence
 gives us.
\begin{equation*}
   \begin{aligned}
    8l + 4 & \equiv 4 \pmod{11}  && \qquad    \text{\footnotesize{\# Substituting}} \\
    l &\equiv 0 \pmod{11}   
   \end{aligned}
  \end{equation*}
 With this we can express $l$ in terms of some other constant and finally see
 the answer
 \[
  x = 210m + 59 \text{.}
 \]
 Notice how the coefficient before $m$ is the same $7\cdot 9 \cdot 11$. The CRT
 tell us that this will be the case for every system where the module are
 coprime.


 \end{block}

\end{column}
\separatorcolumn

\end{columns}
\end{frame}

\end{document}
