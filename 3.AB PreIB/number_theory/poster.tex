% Unofficial University of Cambridge Poster Template
% https://github.com/andiac/gemini-cam
% a fork of https://github.com/anishathalye/gemini
% also refer to https://github.com/k4rtik/uchicago-poster
% TeX program = lualatex

\documentclass[final]{beamer}

% ====================
% Packages
% ====================

\usepackage[T1]{fontenc}
\usepackage{lmodern}
\usepackage[orientation=portrait,size=custom,width=120,height=120,scale=1]{beamerposter}
\usetheme{gemini}
\usepackage[dvipsnames]{xcolor}
\usecolortheme{nott}
\usepackage{graphicx}
\usepackage{booktabs}
\usepackage{tikz}
\usetikzlibrary{calc,arrows.meta,patterns,decorations.pathmorphing,shapes.geometric}
\usepackage{tkz-euclide}
\tikzset{point style/.style = {%
  draw = black,
  inner sep = 0pt,
  shape = circle,
  minimum size = 5pt,
  fill = black
 },
 every picture/.append style = {
  scale = 1.5
 },
 every node/.append style={
  scale=1.5
 }
}
\usepackage{pgfplots}
\pgfplotsset{compat=1.14}
\usepackage{anyfontsize}
\usepackage{caption}
\usepackage{subcaption}

% ====================
% Lengths
% ====================

% If you have N columns, choose \sepwidth and \colwidth such that
% (N+1)*\sepwidth + N*\colwidth = \paperwidth
\newlength{\sepwidth}
\newlength{\colwidth}
\setlength{\sepwidth}{0.01\paperwidth}
\setlength{\colwidth}{0.32\paperwidth}

\newcommand{\separatorcolumn}{\begin{column}{\sepwidth}\end{column}}
\newcommand{\bfalert}[1]{\textbf{\alert{#1}}}

% Math shortcuts
\newcommand{\R}{\mathbb{R}}
\newcommand{\N}{\mathbb{N}}
\newcommand{\Z}{\mathbb{Z}}
\newcommand{\Q}{\mathbb{Q}}
\DeclareMathOperator{\s}{succ}

% Inline shapes
\newcommand{\mysquare}{\tikz[baseline=-7pt]{%
  \node[rectangle,draw,thick,inner sep=6pt] at (0,0) {};
}}
\newcommand{\mytria}{\tikz[baseline=-3.25pt]{%
  \node[isosceles triangle,isosceles triangle apex angle=60,draw,thick,inner
  sep=3.25pt,rotate=90] at (0,0) {};
}}
\newcommand{\mycirc}{\tikz[baseline=-7pt]{%
  \node[circle,draw,thick,inner sep=4.5pt,baseline=0.5ex,rotate=90] at (0,0) {};
}}
\newcommand{\mycross}{\tikz[baseline=-7pt,scale=0.2]{%
  \draw[thick] (-1,1) -- (1,-1);
  \draw[thick] (-1,-1) -- (1,1);
}}

% Colors %
\newcommand{\clr}{\textcolor{BrickRed}}
\newcommand{\clb}{\textcolor{RoyalBlue}}
\newcommand{\clg}{\textcolor{ForestGreen}}
\newcommand{\clm}{\textcolor{Fuchsia}}
\newcommand{\clv}{\textcolor{violet}}
\newcommand{\clbr}{\textcolor{Sepia}}
\newcommand{\cly}{\textcolor{Dandelion}}

% ====================
% Title
% ====================

\title{Number Theory Cheatsheet}

\author{3.AB PreIB Math}

\institute[shortinst]{Adam Klepáč and Jáchym Löwenhöffer}

% ====================
% Footer (optional)
% ====================

% \footercontent{
%   \href{https://utfpr.edu.br/ct/ppgca}{utfpr.edu.br/ct/ppgca} \hfill
%   Mostra de Trabalhos do PPGCA --- TechTalks 2024 \hfill
%   \href{mailto:ppgca-ct@utfpr.edu.br}{ppgca-ct@utfpr.edu.br}}
% (can be left out to remove footer)


% ====================
% Logo (optional)
% ====================

% use this to include logos on the left and/or right side of the header:
\logoright{\includegraphics[height=3.5cm]{logos/logo-white.png}}
% \logoleft{\hspace{20ex}\includegraphics[height=3.5cm]{logos/ppgca-logo.png}}

% ====================
% Body
% ====================

\begin{document}

% Refer to https://github.com/k4rtik/uchicago-poster
% logo: https://www.cam.ac.uk/brand-resources/about-the-logo/logo-downloads
% \addtobeamertemplate{headline}{}
% {
%     \begin{tikzpicture}[remember picture,overlay]
%       \node [anchor=north west, inner sep=3cm] at ([xshift=-2.5cm,yshift=1.75cm]current page.north west)
%       {\includegraphics[height=7cm]{logos/unott-logo.eps}}; 
%     \end{tikzpicture}
% }

\begin{frame}[t]
\begin{columns}[t]
\separatorcolumn

\begin{column}{\colwidth}

 \begin{exampleblock}{Natural Numbers}
  Natural numbers (denoted $\N$) are defined basically as `sets containing so
  many elements'. This means that the number $0$ is a set with no elements, $1$
  is a set with one element and so on. Formally, we construct them in the
  following way ($\emptyset$ is the empty set):
  \[
   \begin{array}{r c l c l}
    0 & = & \emptyset & &\\
    1 & = & \{0\} & = & \{\emptyset\}\\
    2 & = & \{0, 1\} & = & \{\emptyset, \{\emptyset\}\}\\
    3 & = & \{0, 1, 2\} & = & \{\emptyset, \{\emptyset\}, \{\emptyset,
    \{\emptyset\}\}\}\\
      & \vdots & & & 
   \end{array}
  \]
  In general, the \alert{next natural number} after a number $n$ is defined as
  the set $\{0,\ldots,n\}$. 

  Observe that we can find a formula for the next number after $n$. Since $n =
  \{0,\ldots,n-1\}$ and the next number is $\{0,\ldots,n\}$, we can construct
  the next number after $n$ as a union of two sets: $n \cup \{n\}$. We call this
  number, the \alert{successor} of $n$, and write it as \alert{$\s(n)$}. For
  example, $1 = \{0\} = 0 \cup \{0\} = \s(0)$ or $3 = \{0,1,2\} = \{0,1\} \cup
  \{2\} = 2 \cup \{2\} = \s(2)$.

  \vspace{18pt}

  \textbf{\large Addition of natural numbers (\alert{not examined})}

  We can define the operation of \alert{addition} on natural numbers using two
  simple rules. For two natural numbers $n,m \in \N$,
  \begin{enumerate}[label=(\arabic*),left=12pt]
   \item $n + 1 = \s(n)$,
   \item $\s(n + m) = n + \s(m)$.
  \end{enumerate}
  Rule (1) simply states that $n + 1$ is the next number after $n$. Rule (2) is
  harder to decode. It literally says that by adding the two numbers together
  and then taking the next number one reaches the same answer as by first taking
  the next number and then performing addition. It's visualised on the picture
  below.
  \begin{figure}[H]
   \centering
   \begin{tikzpicture}
    \begin{scope}
     \foreach \x in {-.5, 0, .5} {
      \node[circle,fill=BrickRed,inner sep=3pt] at (\x,.75) {};
     }
     \node[BrickRed] at (0,0) {\footnotesize $n$};

     \foreach \x in {1.5, 2, 2.5, 3} {
      \node[circle,fill=RoyalBlue,inner sep=3pt] at (\x,.75) {};
     }
     \node[RoyalBlue] at (2.25,0) {\footnotesize $m$};
    \end{scope}
    \begin{scope}[xshift=5.5cm]
     \foreach \x in {-.5, 0, .5} {
      \node[circle,fill=BrickRed,inner sep=3pt] at (\x,.75) {};
     }
     \node at (.25,0) {\footnotesize $\clr{n} + \clb{m}$};

     \foreach \x in {-.5, 0, .5, 1} {
      \node[circle,fill=RoyalBlue,inner sep=3pt] at (\x,1.25) {};
     }
    \end{scope}
    \begin{scope}[xshift=8.5cm]
     \foreach \x in {-.5, 0, .5} {
      \node[circle,fill=BrickRed,inner sep=3pt] at (\x,.75) {};
     }
     \node at (.25,0) {\footnotesize $\clg{\s}(\clr{n} + \clb{m})$};

     \foreach \x in {-.5, 0, .5, 1} {
      \node[circle,fill=RoyalBlue,inner sep=3pt] at (\x,1.25) {};
     }
     \node[circle,fill=ForestGreen,inner sep=3pt] at (-.5, 1.75) {};
    \end{scope}
    \begin{scope}[xshift=12.5cm]
     \foreach \x in {-.5, 0, .5, 1} {
      \node[circle,fill=RoyalBlue,inner sep=3pt] at (\x,.75) {};
     }
     \node[circle,fill=ForestGreen,inner sep=3pt] at (-.5, 1.25) {};
     \node at (.25,0) {\footnotesize $\clg{\s}(\clb{m})$};
    \end{scope}
    \begin{scope}[xshift=15.5cm]
     \foreach \x in {-.5, 0, .5} {
      \node[circle,fill=BrickRed,inner sep=3pt] at (\x,.75) {};
     }
     \node[circle,fill=ForestGreen,inner sep=3pt] at (-.5, 1.75) {};
     \node at (.25,0) {\footnotesize $\clr{n} + \clg{\s}(\clb{m})$};

     \foreach \x in {-.5, 0, .5, 1} {
      \node[circle,fill=RoyalBlue,inner sep=3pt] at (\x,1.25) {};
     }
    \end{scope}
   \end{tikzpicture}

   \caption{Visualisation of rule (2) of addition. Both $\clg{\s}(\clr{n} +
    \clb{m})$ and $\clr{n} + \clg{\s}(\clb{m})$ feature the \alert{same number}
    of dots.}
   \label{fig:rule-2-addition}
  \end{figure}

  Rules (1) and (2) combine to give a simple algorithm of computing the sum $n +
  m$ for any two numbers $n,m \in \N$. It goes like this:
  \begin{itemize}[label=\textbullet,left=12pt]
   \item Using rule (1), calculate $n + 1 = \s(n) = n \cup \{n\}$.
   \item Now that we have calculated $n + 1$, we can calculate $n + 2$ because
    $n + 2 = n + \s(1)$ and by rule (2) this equals $n + \s(1) = \s(n + 1)$, so
    just take the next number after $n + 1$.
   \item Continue like this until you calculate $n + m = n + \s(m - 1)$.
  \end{itemize}
  For example, to compute $4 + 2$, we calculate $4 + 1 = \s(4)$ and then $4 + 2
  = 4 + \s(1) = \s(4 + 1) = \s(\s(4))$ so $4 + 2$ is just the next number after
  the next number after $4$.
 \end{exampleblock}

 \begin{exampleblock}{Integers (Whole Numbers)}
  We have defined \alert{addition} on natural numbers, but in order to perform
  \alert{subtraction}, we must move to a `larger' set of numbers -- the
  \alert{integers}. This is because subtraction is \alert{\textbf{not}} an
  operation on natural numbers as its result needn't be a natural number itself.

  The idea behind the definition of integers (labelled $\Z$) is to take
  \alert{pairs of natural numbers}. Fundamentally, we want the pair $(a,b) \in
  \N \times \N$ to \alert{represent} the result of the operation `$a-b$' (which
  we can't yet perform because we need to define the integers \textbf{before}
  defining subtraction).

  To this end, we define an \alert{equivalence} on $\N \times \N$ (i.e. on pairs
  of natural numbers) that makes two pairs equivalent \alert{if they represent
  the same integer}. For example, the pair $(4, 6)$ should represent the number
  $-2$ (as $4 - 6 = -2$) and so should the pairs $(8, 10)$, $(3, 5)$ or just
  about any pair $(a, a + 2)$ for $a \in \N$. The integers will then be the
  \alert{classes of equivalence} of this equivalence relation.

  We label this equivalence by the letter $\clg{E}$. Since we want $(a,b)$ to be
  \clg{equivalent} to $(c,d)$ if `$a - b = c - d$' but we can't use subtraction
  yet, we simply rewrite the equation above to use only addition, like this: $a
  + d = c + b$. Thus, we say that $\clr{(a,b)}\clg{E}\clb{(c,d)}$ if $\clr{a} +
  \clb{d} = \clb{c} + \clr{b}$. This defines an equivalence on $\N \times \N$
  and we let $\Z$ be the classes of equivalence of all pairs of natural numbers:
  \[
   \clm{\Z} = \{[(a,b)]_{\clg{E}} \mid a,b \in \N\}.
  \]
  To give an example, the pair $\clr{(3,5)}$ is \clg{equivalent} to
  $\clb{(7,9)}$ because $\clr{3} + \clb{9} = \clb{7} + \clr{5}$ and they both
  represent the integer $\clm{-2}$. Similarly, both $\clr{(6, 2)}$ and
  $\clb{(8, 4)}$ represent the integer $\clm{4}$. The visualisation of integers
  as pairs of natural numbers is given below.
  \begin{figure}[H]
   \centering
   \begin{tikzpicture}
    \node[ForestGreen] at (-1, 2.5) {\clg{$\cdots$}};
    \draw[ForestGreen,thick] (0, 5) -- (0, 0);
    \begin{scope}[xshift=1.5cm,yshift=1cm]
     \node[BrickRed] at (0, 3) {\footnotesize $(2, 4)$};
     \node[BrickRed] at (2, 3) {\footnotesize $(7, 9)$};
     \node[BrickRed] at (0, 2) {\footnotesize $(0, 2)$};
     \node[BrickRed] at (2, 2) {\footnotesize $(11, 13)$};
     \node at (1, 1) {\footnotesize $\vdots$};
     \node at (1, 0) {\footnotesize all pairs $\clr{(a, a + 2)}$};
     \node at (1, -2) {$\clm{-2}$};
    \end{scope}
    \draw[ForestGreen,thick] (5, 5) -- (5, 0);
    \begin{scope}[xshift=6.5cm,yshift=1cm]
     \node[BrickRed] at (0, 3) {\footnotesize $(3, 4)$};
     \node[BrickRed] at (2, 3) {\footnotesize $(10, 11)$};
     \node[BrickRed] at (0, 2) {\footnotesize $(0, 1)$};
     \node[BrickRed] at (2, 2) {\footnotesize $(22, 23)$};
     \node at (1, 1) {\footnotesize $\vdots$};
     \node at (1, 0) {\footnotesize all pairs $\clr{(a, a + 1)}$};
     \node at (1, -2) {$\clm{-1}$};
    \end{scope}
    \draw[ForestGreen,thick] (10, 5) -- (10, 0);
    \node[ForestGreen] at (11, 2.5) {\clg{$\cdots$}};
    \draw[ForestGreen,thick] (12, 5) -- (12, 0);
    \begin{scope}[xshift=13.5cm,yshift=1cm]
     \node[BrickRed] at (0, 3) {\footnotesize $(8, 5)$};
     \node[BrickRed] at (2, 3) {\footnotesize $(13, 10)$};
     \node[BrickRed] at (0, 2) {\footnotesize $(3, 0)$};
     \node[BrickRed] at (2, 2) {\footnotesize $(7, 4)$};
     \node at (1, 1) {\footnotesize $\vdots$};
     \node at (1, 0) {\footnotesize all pairs $\clr{(a + 3, a)}$};
     \node at (1, -2) {$\clm{3}$};
    \end{scope}
    \draw[ForestGreen,thick] (17, 5) -- (17, 0);
    \node[ForestGreen] at (18, 2.5) {\clg{$\cdots$}};
   \end{tikzpicture}
   \caption{\clm{Integers} as classes of \clg{equivalence} of natural numbers.}
   \label{fig:integers}
  \end{figure}
  The \alert{addition} of integers is defined using the addition of natural
  numbers. Given two classes of equivalence $[(\clr{a},\clr{b})]_{\clg{E}},
  [(\clb{c},\clb{d})]_{\clg{E}} \in \clm{\Z}$, we let
  \[
   [(\clr{a},\clr{b})]_{\clg{E}} + [(\clb{c}, \clb{d})]_{\clg{E}} = [(\clr{a} +
   \clb{c}, \clr{b} + \clb{d})]_{\clg{E}}.
  \]
  Finally, we define the \alert{opposite number} to $[(a,b)]_{\clg{E}}$ as
  $-[(a,b)]_{\clg{E}} = [(b,a)]_{\clg{E}}$ (this is because $-(a-b) = b-a$). The
  \alert{subtraction} of two integers is now just a sum of the first and the
  opposite of the second, that is
  \[
   [(\clr{a},\clr{b})]_{\clg{E}} - [(\clb{c},\clb{d})]_{\clg{E}} =
   [(\clr{a},\clr{b})]_{\clg{E}} + (-[(\clb{c},\clb{d})]_{\clg{E}}) =
   [(\clr{a},\clr{b})]_{\clg{E}} + [(\clb{d},\clb{c})]_{\clg{E}} = [(\clr{a} +
   \clb{d}, \clr{b} + \clb{c})]_{\clg{E}}.
  \]
  For example,
  \[
   [(\clr{3}, \clr{1})]_{\clg{E}} - [(\clb{5},\clb{2})]_{\clg{E}} = [(\clr{3},
   \clr{1})]_{\clg{E}} + [(\clb{2},\clb{5})]_{\clg{E}} = [(\clr{3} + \clb{2},
   \clr{1} + \clb{5})]_{\clg{E}} = [(5, 6)]_{\clg{E}},
  \]
  which is the same as writing
  \[
   \clr{2} - \clb{3} = \clr{2} + (-\clb{3}) = -1.
  \]
 \end{exampleblock}

\end{column}

\separatorcolumn

\begin{column}{\colwidth}

\begin{block}{Multiplication}
 In a way similar to addition, we can define \alert{multiplication} on natural
 numbers by the following two rules.
 \begin{enumerate}[label=(\arabic*),left=12pt]
  \item $n \cdot 1 = n$,
  \item $n \cdot \s(m) = n \cdot m + m$,
 \end{enumerate}
 for $n,m \in \N$. They carry the idea behind an algorithmic way to compute the
 product $n \cdot m$ for any two natural numbers $n,m$. It goes like this:
 \begin{itemize}[left=12pt,label=\textbullet]
  \item Using rule (1), calculate $n \cdot 1 = n$.
  \item Using rule (2), calculate $n \cdot 2 = n \cdot \s(1) = n \cdot 1 + n = n
   + n$.
  \item Continue like this until you calculate
   \[
    n \cdot m = n \cdot \s(m - 1) = n \cdot (m - 1) + n = \underbrace{n + n +
    \ldots + n}_{(m - 1)\text{ times}} + n.
   \]
 \end{itemize}
 For example, to calculate $4 \cdot 3$, we first multiply $4 \cdot 1 = 4$, then
 $4 \cdot 2 = 4 \cdot \s(1) = 4 \cdot 1 + 4 = 4 + 4$, and finally $4 \cdot 3 = 4
 \cdot \s(2) = 4 \cdot 2 + 4 = (4 + 4) + 4$. As you've been taught:
 `multiplication is just repeated addition'.

 Multiplication is easily extended to integers by the formula
 \[
  [(\clr{a},\clr{b})]_{\clg{E}} \cdot [(\clb{c},\clb{d})]_{\clg{E}} = [(\clr{a}
  \cdot \clb{c} + \clr{b} \cdot \clb{d}, \clr{b} \cdot \clb{c} + \clr{a} \cdot
  \clb{d})]_{\clg{E}}.
 \]
 The formula is based on the calculation
 \[
  (\clr{a} - \clr{b}) \cdot (\clb{c} - \clb{d}) = \clr{a} \cdot \clb{c} -
  \clr{b} \cdot \clb{c} - \clr{a} \cdot \clb{d} + \clr{b} \cdot \clb{d} =
  (\clr{a} \cdot \clb{c} + \clr{b} \cdot \clb{d}) - (\clr{b} \cdot \clb{c} +
  \clr{a} \cdot \clb{d}).
 \]

 For example,
 \[
  [(\clr{5},\clr{3})]_{\clg{E}} \cdot [(\clb{1},\clb{5})]_{\clg{E}} =
  [(\clr{5} \cdot \clb{1} + \clr{3} \cdot \clb{5}, \clr{3} \cdot \clb{1} +
  \clr{5} \cdot \clb{5})]_{\clg{E}} = [(20, 28)]_{\clg{E}}.
 \]
 This is the same calculation as
 \[
  \clr{2} \cdot (\clb{-4}) = -8.
 \]
\end{block}

\begin{exampleblock}{Rational Numbers}
 Being able to \alert{multiply integers}, we'd like to divide them as well. As
 was the case with natural numbers and subtraction, \alert{division is not an
 operation on integers} because its result needn't be an integer.

 The idea behind the definition of rational numbers (labelled $\Q$) is pretty
 much the same as the one behind the definition of integers -- rational numbers
 are really just \alert{pairs of integers}. And again, multiple pairs of
 integers \alert{represent the same} rational number. Therefore, given pairs
 $(a,b)$ and $(c,d)$ with $a,b,c,d \in \Z$, we must make sure that $(a,b)$
 \alert{is equivalent to} $(c,d)$ if `the fraction $a / b$ is the same as the
 fraction $c / d$'.

 As we couldn't have defined division yet, we must \alert{rewrite the last
 equation in terms of multiplication only}. This is easy to do because $a / b =
 c / d$ if $a \cdot d = c \cdot b$. This directly leads to the definition of an
 \alert{equivalence} $\clg{Q}$ on pairs of integers: $(\clr{a},
 \clr{b})\clg{Q}(\clb{c},\clb{d})$ if
 \[
  \clr{a} \cdot \clb{d} = \clr{c} \cdot \clb{b}.
 \]
 This is indeed an equivalence on $\Z \times \Z$ and we define $\clm{\Q}$ as
 \[
  \clm{\Q} = \{[(a,b)]_{\clg{Q}} \mid a,b \in \Z\}.
 \]
 We tend to write elements of $\clm{\Q}$ as \textbf{fractions}, that is, instead
 of $[(a,b)]_{\clg{Q}}$, we write $a / b$. We shall adopt this notation
 henceforth.

 It only remains to \alert{extend addition and multiplication} to rational
 numbers. This is easily done using formulae you already know. For example, the
 \alert{product} of two rational numbers $a / b, c / d \in \clm{\Q}$ is defined
 as such:
 \[
  \frac{\clr{a}}{\clr{b}} \cdot \frac{\clb{c}}{\clb{d}} = \frac{\clr{a} \cdot
  \clb{c}}{\clr{b} \cdot \clb{d}}.
 \]
 The \alert{sum} of rational numbers as
 \[
  \frac{\clr{a}}{\clr{b}} + \frac{\clb{c}}{\clb{d}} = \frac{\clr{a} \cdot
  \clb{d} + \clb{c} \cdot \clr{b}}{\clr{b} \cdot \clb{d}}.
 \]
 For example,
 \[
  \frac{\clr{2}}{\clr{5}} \cdot \frac{\clb{3}}{\clb{4}} = \frac{\clr{2} \cdot
  \clb{3}}{\clr{5} \cdot \clb{4}} = \frac{6}{20} \quad \text{and} \quad
  \frac{\clr{2}}{\clr{5}} + \frac{\clb{3}}{\clb{4}} = \frac{\clr{2} \cdot
  \clb{4} + \clb{3} \cdot \clr{5}}{\clr{5} \cdot \clb{4}} = \frac{23}{20}.
 \]
 
 Finally, we're ready to \alert{define division} on rational numbers. We first
 define the \alert{inverse} of a rational numbers $a / b$ as $b / a$. We write
 $b / a = (a / b)^{-1}$. The \alert{operation of division} on rational numbers
 is defined as \alert{multiplication by the inverse element}, that is
 \[
  \frac{\clr{a}}{\clr{b}} : \frac{\clb{c}}{\clb{d}} = \frac{\clr{a}}{\clr{b}}
  \cdot \left( \frac{\clb{c}}{\clb{d}} \right)^{-1} = \frac{\clr{a}}{\clr{b}}
  \cdot \frac{\clb{d}}{\clb{c}} = \frac{\clr{a} \cdot \clb{d}}{\clr{b} \cdot
  \clb{c}}.
 \]
 For example,
 \[
  \frac{\clr{2}}{\clr{5}} : \frac{\clb{3}}{\clb{4}} = \frac{\clr{2}}{\clr{5}}
 \cdot \left(\frac{\clb{3}}{\clb{4}}\right)^{-1} = \frac{\clr{2}}{\clr{5}} \cdot
 \frac{\clb{4}}{\clb{3}} = \frac{\clr{2} \cdot \clb{4}}{\clr{5} \cdot \clb{3}} =
 \frac{8}{15}.
 \]
\end{exampleblock}

\begin{block}{Drawing Sets}
 Set operations can be visualized using so-called \emph{Venn diagrams}. This
 just means using circles to represent the sets in questions. For example, two
 sets -- $\clr{A}$ and $\clb{B}$ -- can be drawn like this:
 \begin{center}
  \begin{tikzpicture}[scale=0.5]
   \fill[BrickRed,fill opacity=0.25] (180:1.5) circle (2);
   \fill[RoyalBlue,fill opacity=0.25] (0:1.5) circle (2);
   \draw[BrickRed,thick] (180:1.5) circle (2);
   \draw[RoyalBlue,thick] (0:1.5) circle (2);
   \node at (150:4.3) {\footnotesize $\clr{A}$};
   \node at (30:4.3) {\footnotesize $\clb{B}$};
  \end{tikzpicture}
 \end{center}
 In these pictures, one can easily visualize the operations of union and
 intersection. The union $\clg{A \cup B}$ is the entire area covered by
 $\clr{A}$ and $\clb{B}$. It looks like this:
 \begin{center}
  \begin{tikzpicture}[scale=0.5]
   \def\firstcircle{(180:1.5) circle (2)}
   \def\secondcircle{(0:1.5) circle (2)}
   \filldraw[thick,ForestGreen] \firstcircle; 
   \filldraw[thick,ForestGreen] \secondcircle;
   \node at (90:3) {\footnotesize $\clg{A \cup B}$};
  \end{tikzpicture}
 \end{center}
 The intersection $\clm{A \cap B}$ is the `strip' in the middle, the area which
 is shared between both $\clr{A}$ and $\clb{B}$. It can be depicted like this:
 \begin{center}
  \begin{tikzpicture}[scale=0.5]
   \def\firstcircle{(180:1.5) circle (2)}
   \def\secondcircle{(0:1.5) circle (2)}
   
   \fill[BrickRed,fill opacity=0.25] \firstcircle;
   \fill[RoyalBlue,fill opacity=0.25] \secondcircle;
   \draw[BrickRed,thick] \firstcircle;
   \draw[RoyalBlue,thick] \secondcircle;

   \begin{scope}
    \clip \firstcircle;
    \fill[white] \secondcircle;
    \filldraw[thick,Fuchsia] \secondcircle;
   \end{scope}
   \begin{scope}
    \clip \secondcircle;
    \draw[Fuchsia,thick] \firstcircle;
   \end{scope}
   \node at (90:3) {\footnotesize $\clm{A \cap B}$};
  \end{tikzpicture}
 \end{center}
\end{block}
\end{column}
\separatorcolumn

\begin{column}{\colwidth}

\begin{alertblock}{Products of Sets \& Relations}
 Before introducing \emph{products} of sets, we must define a \emph{pair}.
 Simply said, a \alert{pair} of objects $(a,b)$ is just a set containing $a$ and
 $b$ \textbf{with ordering}, that is, $a$ is the \textbf{first} element of
 $(a,b)$ and $b$ is \textbf{second}. This means that $(a,b) \neq (b,a)$ because
 the order is not the same.

 Now, the \alert{product} of sets $A$ and $B$, denoted $A \times B$, is the set
 of all pairs $(a,b)$ where $a \in A$ and $b \in B$. For example, if
 \[
  A = \{\mycirc,\mytria\} \quad \text{and} \quad B = \{\mytria,\mycross, \sim
  \},
 \]
 then
 \[
  A \times B = \{(\mycirc,\mytria),(\mycirc,\mycross),(\mycirc, \sim),
  (\mytria,\mytria), (\mytria,\mycross), (\mytria, \sim)\}.
 \]
 Notice that $A \times B \neq B \times A$ because the \textbf{order} of elements
 in a pair \textbf{matters}. In this case,
 \[
  B \times A = \{(\mytria,\mycirc), (\mytria,\mytria), (\mycross,\mycirc),
  (\mycross,\mytria), ( \sim, \mycirc), ( \sim, \mytria)\}.
 \]

 As another example, consider the \emph{real plane} -- the set of all points
 with two coordinates. That is simply the set of pairs of real numbers, $\R
 \times \R$.

 The mathematical way to define a \alert{relation} between two sets is to simply
 \textbf{list all the elements that are related}. Said formally, a relation $R$
 is a subset $R \subseteq A \times B$. For example,
 \[
  \{(\mytria,\mycross),(\mytria, \sim )\} \subseteq \{\mycirc,\mytria\} \times
  \{\mytria,\mycross, \sim\}
 \]
 is a relation between the sets $A$ and $B$ from above. It literally says that
 $\mytria$ is related to $\mycross$ and $ \sim $. Instead of writing
 $(\mytria,\mycross) \in R$, we write $\mytria R \mycross$ because it's more
 natural. Typical examples of relations include $ \leq $ and $=$ and we don't
 write $(2,3) \in~\leq $ but $2 \leq 3$. 
\end{alertblock}

\begin{block}{Drawing Products And Relations}
 One can draw the product $A \times B$ similarly to the way we draw Cartesian
 coordinates -- by distributing the elements of $A$ horizontally and those of
 $B$ vertically. Each point in the resulting grid represents an element of $A
 \times B$.

 Take, for example, $\clr{A = \{1, 2, 3, 4\}}$ and $\clb{B = \{a, b, c\}}$. We
 can depict the set $\clr{A} \times \clb{B}$ like this:
 \begin{center}
  \begin{tikzpicture}
   \foreach \x in {0, 1, 2, 3} {
    \pgfmathtruncatemacro\z{\x + 1}
    \node[BrickRed] (A\x) at (\x, -1) {\footnotesize $\z$};
   }
   \foreach \y/\n in {0/a, 1/b, 2/c} {
    \node[RoyalBlue] (B\y) at (-1, \y) {\footnotesize $\n$};
   }
   \foreach \x in {0, 1, 2, 3} {
    \foreach \y in {0, 1, 2} {
     \node[circle,inner sep=2pt,fill=black] at (\x,\y) {};
    }
   }
  \end{tikzpicture}
 \end{center}
 Any relation $\clg{R} \subseteq \clr{A} \times \clb{B}$ can now be easily drawn
 into the grid just by marking certain dots. For example, the relation $\clg{R =
 \{}(\clr{1},\clb{c}), (\clr{2},\clb{b}),(\clr{2},\clb{a})\clg{\}} \subseteq
 \clr{A} \times \clb{B}$ looks like this:
 \begin{center}
  \begin{tikzpicture}
   \foreach \x in {0, 1, 2, 3} {
    \pgfmathtruncatemacro\z{\x + 1}
    \node[BrickRed] (A\x) at (\x, -1) {\footnotesize $\z$};
   }
   \foreach \y/\n in {0/a, 1/b, 2/c} {
    \node[RoyalBlue] (B\y) at (-1, \y) {\footnotesize $\n$};
   }
   \foreach \x in {0, 1, 2, 3} {
    \foreach \y in {0, 1, 2} {
     \node[circle,inner sep=2pt,fill=black] at (\x,\y) {};
    }
   }
   \node[ultra thick,circle,inner sep=5pt,fill=none,draw=ForestGreen] at (0, 2)
    {};
   \node[ultra thick,circle,inner sep=5pt,fill=none,draw=ForestGreen] at (1, 0)
    {};
   \node[ultra thick,circle,inner sep=5pt,fill=none,draw=ForestGreen] at (1, 1)
    {};
  \end{tikzpicture}
 \end{center}
 There is another way to draw relations -- as arrows from $\clr{A}$ to
 $\clb{B}$. This style of drawing emphasises the `one-way' nature of relations.
 The same relation $\clg{R} \subseteq \clr{A} \times \clb{B}$ is drawn using
 arrows like this:
 \begin{center}
  \begin{tikzpicture}
   \foreach \y in {0, 1, 2, 3} {
    \pgfmathtruncatemacro\z{4-\y}
    \node[BrickRed] (A\z) at (-2,\y) {\footnotesize $\z$};
   }
   \foreach \y/\n in {0/c, 1/b, 2/a} {
    \node[RoyalBlue] (B\n) at (2,\y) {\footnotesize $\n$};
   }
   \draw[arrows={-Latex[length=16pt,width=12pt]},thick,ForestGreen] (A1) -- (Bc);
   \draw[arrows={-Latex[length=16pt,width=12pt]},thick,ForestGreen] (A2) -- (Bb);
   \draw[arrows={-Latex[length=16pt,width=12pt]},thick,ForestGreen] (A2) -- (Ba);
  \end{tikzpicture}
 \end{center}
\end{block}

\begin{exampleblock}{Equivalence \& Equivalence Classes}
 A relation $\clg{E} \subseteq \clr{A} \times \clr{A}$ (do note that it's a
 relation \textbf{on a set}) is called an \alert{equivalence} if it behaves
 \textbf{something like `equals'}, that is,
 \begin{itemize}[left=2em]
  \item[(R)] it's \alert{reflexive}, i.e. $\clr{a}\clg{E}\clr{a}$ for every
   $\clr{a} \in \clr{A}$ (every element is equivalent to itself);
  \item[(S)] it's \alert{symmetric}, i.e. if $\clr{a_1}\clg{E}\clr{a_2}$, then
   also $\clr{a_2}\clg{E}\clr{a_1}$ (all elements are \textbf{mutually}
   equivalent);
  \item[(T)] it's \alert{transitive}, i.e. if $\clr{a_1}\clg{E}\clr{a_2}$ and
   $\clr{a_2} \clg{E} \clr{a_3}$, then $\clr{a_1} \clg{E} \clr{a_3}$ (it
   \textbf{propagates} through a middle element).
 \end{itemize}
 A good example of equivalence to keep in mind is the relation `\textbf{being
 the same age}' on the set of all people. It's \alert{reflexive} because I'm as
 old as me. It's \alert{symmetric} because if I'm as old as you, then you are as
 old as me. And finally, it's \alert{transitive} because if I'm as old as you
 and you are as old as someone else, then I'm as old as that someone. This
 example also drives home the idea of equivalence `being like equals'. It's not
 that two people of the same age are \emph{equal}, they are \emph{equal by some
 criterion}. This is exactly what equivalence is, the relation of being equal by
 some criterion.

 With an equivalence $\clg{E} \subseteq \clr{A} \times \clr{A}$, we can divide
 the set $\clr{A}$ into `packets' of elements -- each packet consisting of
 elements which are \textbf{equivalent} by $\clg{E}$. For example, I can divide
 the set of all people by putting equally old people to the same packet.
 Formally, we write
 \[
  [\clr{a}]_{\clg{E}} = \{\clr{b} \in \clr{A} \mid \clr{a}\clg{E}\clr{b}\},
 \]
 in other words, $[\clr{a}]_{\clg{E}}$ is the set of all elements from $\clr{A}$
 that are $\clg{equivalent}$ to $\clr{a}$. We call it \alert{an equivalence
 class}. The element $\clr{a}$ is then called a \alert{representative}. 
\end{exampleblock}
\end{column}
\separatorcolumn

\end{columns}
\end{frame}

\end{document}
