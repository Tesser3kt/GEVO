% Unofficial University of Cambridge Poster Template
% https://github.com/andiac/gemini-cam
% a fork of https://github.com/anishathalye/gemini
% also refer to https://github.com/k4rtik/uchicago-poster
% TeX program = lualatex

\documentclass[final]{beamer}

% ====================
% Packages
% ====================

\usepackage[T1]{fontenc}
\usepackage{lmodern}
\usepackage[orientation=portrait,size=custom,width=120,height=100,scale=1]{beamerposter}
\usetheme{gemini}
\usepackage[dvipsnames]{xcolor}
\usecolortheme{nott}
\usepackage{graphicx}
\usepackage{booktabs}
\usepackage{tikz}
\usetikzlibrary{calc,arrows.meta,patterns,decorations.pathmorphing,shapes.geometric}
\usepackage{tkz-euclide}
\tikzset{point style/.style = {%
  draw = black,
  inner sep = 0pt,
  shape = circle,
  minimum size = 5pt,
  fill = black
 },
 every picture/.append style = {
  scale = 1.5
 },
 every node/.append style={
  scale=1.5
 }
}
\usepackage{pgfplots}
\pgfplotsset{compat=1.14}
\usepackage{anyfontsize}
\usepackage{caption}
\usepackage{subcaption}

% ====================
% Lengths
% ====================

% If you have N columns, choose \sepwidth and \colwidth such that
% (N+1)*\sepwidth + N*\colwidth = \paperwidth
\newlength{\sepwidth}
\newlength{\colwidth}
\setlength{\sepwidth}{0.01\paperwidth}
\setlength{\colwidth}{0.32\paperwidth}

\newcommand{\separatorcolumn}{\begin{column}{\sepwidth}\end{column}}
\newcommand{\bfalert}[1]{\textbf{\alert{#1}}}

% Math shortcuts
\newcommand{\R}{\mathbb{R}}

% Inline shapes
\newcommand{\mysquare}{\tikz[baseline=-7pt]{%
  \node[rectangle,draw,thick,inner sep=6pt] at (0,0) {};
}}
\newcommand{\mytria}{\tikz[baseline=-3.25pt]{%
  \node[isosceles triangle,isosceles triangle apex angle=60,draw,thick,inner
  sep=3.25pt,rotate=90] at (0,0) {};
}}
\newcommand{\mycirc}{\tikz[baseline=-7pt]{%
  \node[circle,draw,thick,inner sep=4.5pt,baseline=0.5ex,rotate=90] at (0,0) {};
}}
\newcommand{\mycross}{\tikz[baseline=-7pt,scale=0.2]{%
  \draw[thick] (-1,1) -- (1,-1);
  \draw[thick] (-1,-1) -- (1,1);
}}

% Colors %
\newcommand{\clr}{\textcolor{BrickRed}}
\newcommand{\clb}{\textcolor{RoyalBlue}}
\newcommand{\clg}{\textcolor{ForestGreen}}
\newcommand{\clm}{\textcolor{Fuchsia}}
\newcommand{\clv}{\textcolor{violet}}
\newcommand{\clbr}{\textcolor{Sepia}}
\newcommand{\cly}{\textcolor{Dandelion}}

% ====================
% Definitions
% ====================

\newcommand{\N}{\mathbb{N}}

% ====================
% Title
% ====================

\title{Logic \& Set Theory Cheatsheet}

\author{3.AB PreIB Math}

\institute[shortinst]{Adam Klepáč}

% ====================
% Footer (optional)
% ====================

% \footercontent{
%   \href{https://utfpr.edu.br/ct/ppgca}{utfpr.edu.br/ct/ppgca} \hfill
%   Mostra de Trabalhos do PPGCA --- TechTalks 2024 \hfill
%   \href{mailto:ppgca-ct@utfpr.edu.br}{ppgca-ct@utfpr.edu.br}}
% (can be left out to remove footer)


% ====================
% Logo (optional)
% ====================

% use this to include logos on the left and/or right side of the header:
\logoright{\includegraphics[height=3.5cm]{logos/logo-white.png}}
% \logoleft{\hspace{20ex}\includegraphics[height=3.5cm]{logos/ppgca-logo.png}}

% ====================
% Body
% ====================

\begin{document}

% Refer to https://github.com/k4rtik/uchicago-poster
% logo: https://www.cam.ac.uk/brand-resources/about-the-logo/logo-downloads
% \addtobeamertemplate{headline}{}
% {
%     \begin{tikzpicture}[remember picture,overlay]
%       \node [anchor=north west, inner sep=3cm] at ([xshift=-2.5cm,yshift=1.75cm]current page.north west)
%       {\includegraphics[height=7cm]{logos/unott-logo.eps}}; 
%     \end{tikzpicture}
% }

\begin{frame}[t]
\begin{columns}[t]
\separatorcolumn

\begin{column}{\colwidth}

 \begin{block}{Natural Numbers}
  \alert{Logic} is the language of mathematics. It uses \alert{propositions} to
  talk about sets.

  Propositions are sentences which can be either true or false. For example
  \begin{itemize}[label=\textbullet,left=24pt]
   \item `\textbf{Cats are black.}' is a proposition;
   \item `\textbf{How are you?}' is \emph{not} a proposition;
   \item `\textbf{We will have colonised Mars by 2500.}' is also a proposition.
  \end{itemize}
 \end{block}
 As the third example suggests, we need not necessarily know whether a
 proposition is true or false -- it remains a proposition anyway.

 \vspace{1em}

 \begin{exampleblock}{Whole numbers}
  Propositions can be joined together using \alert{logical conjunctions}. They
  pretty much correspond to the conjunctions of natural language. Let us
  consider two propositions:
  \begin{align*}
   p &= \text{`It's raining outside.'}\\
   q &= \text{`I'll stay at home.'}
  \end{align*}
  \begin{itemize}[left=40pt]
   \item[($ \wedge $)] Logical \alert{and} forms a proposition that is only
    \alert{true} if both of its constituents are also \alert{true}. In natural
    language, the proposition $p \wedge q$ can be expressed as
    \[
     p \alert{ \wedge } q = \text{`It's raining outside \alert{and} I'll stay at
     home.'}
    \]
   \item[($ \vee $)] Logical \alert{or} forms a proposition that is \alert{true}
    if at least one of its constituents is \alert{true}. In natural language,
    the proposition $p \vee q$ can be expressed as
    \[
     p \alert{ \vee } q = \text{`It's raining outside \alert{or} I'll stay at
     home.'}
    \]
    In mathematical logic, \alert{or} is \textbf{not exclusive}! This means that
    $p \alert{ \vee } q$ is true even if both $p$ and $q$ are true.
   \item[($\neg $)] Logical \alert{not} isn't strictly speaking a conjunction
    but I include it anyway. It reverses the truth value of a proposition. For
    example, the proposition $\alert{\neg }p$ can be read as
    \[
     \alert{\neg }p = \text{`It's \alert{not} raining outside.'}
    \]
    It follows that $\alert{\neg }p$ is \alert{true} exactly when $p$ is
    \alert{false} and vice versa.
   \item[($ \Rightarrow $)] Logical \alert{implication} is a conjunction that
    makes the first proposition into an \emph{assumption} or \emph{premise} and
    the second one into a \emph{conclusion}. The proposition $p \alert{
    \Rightarrow } q$ is read in multiple ways, to list a few: 
    \begin{align*}
     p \alert{ \Rightarrow } q &= \text{`\alert{If} it's raining outside,
    \alert{then} I'll stay at home.'}\\
     p \alert{ \Rightarrow } q &= \text{`It raining outside \alert{implies that}
     I'll stay at home.'}\\
     p \alert{ \Rightarrow } q &= \text{`\alert{Assuming} it's raining outside,
     I'll stay at home.'}\\
    \end{align*}
    The implication is tricky. It's true if both $p$ and $q$ are true and false
    if $p$ is true but $q$ is false. However, it is \alert{always true} if $p$
    is \alert{false}. That is because, in mathematical logic, whatever follows
    from a lie is automatically true.
   \item[($ \Leftrightarrow $)] Logical \alert{equivalence} is true only if both
    propositions have the \alert{same truth value} -- they're both true or both
    false. In natural language, it is typically read like this:
    \[
     p \alert{ \Leftrightarrow }q = \text{`It's raining \alert{if and only if}
     I stay at home.'}
    \]
    Equivalence is basically just a two-way implication. The proposition $p$ is
    both a premise and a conclusion to $q$ and $q$ is both a premise and a
    conclusion to $p$. If it's raining outside, I stay at home and if I stay at
    home, then it's raining outside.
  \end{itemize}
 \end{exampleblock}

 \begin{alertblock}{Rational numbers}
  Toto je o racionalnich cislech.
 \end{alertblock}

 \begin{block}{Divisibility}
  A conjunction of propositions being true or false based on whether its
  constituent propositions are true or false can be summarized using so-called
  \alert{truth table}. It is basically just a table that lists all the
  possibilities of $p$ and $q$ being true or false and the resulting truth value
  of their conjunctions.

  For the basic logical conjunctions from above, it can look like this (we
  represent \alert{true} by \alert{1} and \alert{false} by \alert{0}):
  \begin{center}
   \begin{tabular}{c | c | c | c | c | c | c | c}
    $p$ & $q$ & $\neg p$ & $\neg q$ & $p \wedge q$ & $p \vee q$ & $p \Rightarrow
    q$ & $p \Leftrightarrow q$\\
    \toprule
    0 & 0 & 1 & 1 & 0 & 0 & 1 & 1\\
    \midrule
    0 & 1 & 1 & 0 & 0 & 1 & 1 & 0\\
    \midrule
    1 & 0 & 0 & 1 & 0 & 1 & 0 & 0\\
    \midrule
    1 & 1 & 0 & 0 & 1 & 1 & 1 & 1
   \end{tabular}
  \end{center}
 \end{block}
\end{column}

\separatorcolumn

\begin{column}{\colwidth}

\begin{exampleblock}{Prime decomposition}
 \alert{Sets} are the `stuff' that makes up the world of mathematics. Their
 basic characteristics and properties are described using \alert{logic}.

 Sets cannot be defined inside set theory but we interpret them as \emph{groups
 of things}.

 There's only one foundational \emph{proposition} related to set theory -- the
 proposition `\alert{An object is an element of a set.}' If we label the object
 in question $x$ and the set $A$, this proposition is written as $x \in A$ (the
 symbol $ \in $ is just the letter `e' in `element'). Combining these
 propositions using logical conjunctions allows for various set-theoretic
 constructions.

 If a set $A$ has, for example, exactly three elements -- $\mysquare$, $\mytria$
 and $\mycirc$, I can write it as a list of these three elements inside curly
 brackets $\{\}$. In this case,
 \[
  A = \{\mysquare,\mytria,\mycirc\}.
 \]

 A few \alert{warnings} about sets:
 \begin{itemize}[label=\textbullet,left=24pt]
  \item \textbf{Sets are not ordered}. There is nothing like a `first', `second'
   or `last' element of a set. Either an object \textbf{is} inside a set or it
   \textbf{isn't}. Nothing else. For example, the three sets below are
   \alert{exactly the same}, only written differently.
   \[
    \{\mysquare,\mytria,\mycirc\} = \{\mycirc,\mytria,\mysquare\} =
    \{\mytria,\mysquare,\mycirc\}
   \]
  \item \textbf{Elements of sets have no frequency}. Again, an element either is
   inside a set or not. It cannot be \alert{twice} in a set, for example. The
   three sets below are exactly the same.
   \[
    \{\mysquare,\mytria,\mycirc\} =
    \{\mysquare,\mytria,\mycirc,\mytria,\mycirc\} = \{
    \mytria,\mysquare,\mysquare,\mytria,\mycirc,\mytria\}
   \]
 \end{itemize}
\end{exampleblock}

\begin{alertblock}{Euler's algorithm}
 Using logical conjunctions, we form new sets from existing ones. Consider two
 sets -- $A$ and $B$.
 \begin{itemize}[left=40pt]
  \item[($ \cap $)] I can form the set of all objects $x$ that satisfy the
   proposition $x \in A \wedge x \in B$, that is all objects that \alert{lie in
   both $A$ and $B$}. This set is called the \alert{intersection} of $A$ and $B$
   and written $A \cap B$. For example,
   \[
    \{\mycirc,\mytria,\mysquare\} \cap \{\mycross,\mycirc,\mysquare, \sim \} =
    \{\mycirc,\mysquare\}.
   \]
  \item[($ \cup $)] I can form the set of all objects that satisfy the
   proposition $x \in A \vee x \in B$, the set of all objects that \alert{lie in
   $A$ or in $B$}. It is called the \alert{union} of $A$ and $B$ and denoted
   $A \cup B$. All elements of $A \cup B$ can be found \emph{only} in $A$,
   \emph{only} in B or in \emph{both} $A$ and $B$. For example,
   \[
    \{\mycirc,\mytria,\mysquare\} \cup \{\mycross,\mycirc,\mysquare, \sim \} =
    \{\mycirc,\mytria,\mysquare,\mycross, \sim \}.
   \]
  \item[($ \Rightarrow $)] Implication is a little different from intersection
   and union. It describes a lot of different sets with one logical proposition.
   I ask: `Which sets $A$ satisfy the proposition $x \in A \Rightarrow x \in
   B$?' In other words, which sets $A$ \alert{have all their elements contained}
   in the set $B$? The answer is that $A$ must be a subset of $B$ and we denote
   that fact by $A \subseteq B$. The set $A$ is only allowed to have elements
   which also lie in $B$ but not necessarily all of them. All the subsets of $B
   = \{\mytria,\mycirc\}$ are listed below.
   \[
    \emptyset, \{\mytria\}, \{\mycirc\}, \{\mytria,\mycirc\},
   \]
   where $\emptyset$ is the \alert{empty set}, a set containing no elements.
  \item[($ \Leftrightarrow $)] Equivalence defines \alert{equality} on sets. If
   sets $A$ and $B$ must satisfy the proposition $x \in A \Leftrightarrow x \in
   B$, then they must be equal because all the elements of $A$ lie in $B$ and
   all elements of $B$ lie in $A$. That is, $A = B$.
 \end{itemize}
\end{alertblock}
\end{column}


\begin{column}{\colwidth}


\begin{block}{Congruence}
 Now that we have defined the remainder after division we want to express the
 idea that two numbers ($x$ and $y$) have the same remainder ($r$) after diving by some number $m$.
 Mathematically we write it as:
 \[
  x \equiv y \pmod{m}.
 \]
 In other words this means:
 \begin{align*}
  x = km + r && \text{and}  && y = lm + r
 \end{align*}
 for some numbers $l,k \in \N$. This way we can for example say that 13 and 25
 are the same modulo 12, because they share the remainder 1 when divided by 12
 (formally written as  $13 \equiv 25 \pmod{12}$).

 This idea might sound unintuitive and artificial yet it is all around us. If we
 for example take the regular old clock. Looking only at the clock and seeing
 both hands up tells me that it is either noon or midnight. This is because we
 use the 12 hours format which gives the time$ \mod  12$.

Congruence is in sense very similar to normal equations (it is also
equivalence try to prove it!). Similar to equations we can manipulate it.
More specifically:
\begin{itemize}[label=\textbullet,left=24pt]
 \item $x+a \equiv y+a \pmod{m}$ - \textbf{adding} also works for subtracting so
  $a \in \mathbb{Z}$
 \item $xk \equiv yk \pmod{km}$ - \textbf{multiplying}
 \item $x^k \equiv y^k \pmod{m}$ - \textbf{exponentiation}
\end{itemize}
for any $k \in \N$.

An interesting thing to note about the equivalence classes created by some
congruence $\mod m$ is that there will be $m$ of them. This is because all the
possible remainders after diving by $m$ are numbers $1, \ldots, m$. 

Similarly to system of equations we can have also systems of congruences.

\end{block}  
\begin{alertblock}{Chineese remainder theorem}
 Imagine we have a system of linear congruences:
   \begin{equation*}
   \begin{split}
    x & \equiv r_1 \pmod{m_1}\\
    x & \equiv r_2 \pmod{m_2}\\
      &~~~\vdots\\
    x & \equiv r_n \pmod{m_n}
   \end{split}
  \end{equation*}

 Where all the numbers are natural and $m_1,\ldots m_n$ are mutually coprime.
 The CRT tells us that there is unique solution $x$ smaller then the product of
 all the numbers we divide by, $M = m_1 \cdot m_2
 \ldots m_n$.

Each congruence limits the possible solutions radically. For example the
congruence $x \equiv r \pmod{m}$ has solutions in the form: $km + r$ for any $k
\in \N$. So if we write down all the solutions of each congruence up to $M$ and
then find the intersection, we have found $x$ such that it satisfies all the
congruences and thus is the solution to the whole system.

We can do this process also graphically. If we circle solutions to the
individual congruences the number with $n$ circles is the solution to the whole
system.

 If, for example, we have the linear congruences: $\clr{x \equiv 1 \pmod{3}}$,
 $\clb{x \equiv 2 \pmod{4}}$, we can draw:

\begin{center}
 \begin{tikzpicture}
   \def\rows{4}
   \def\cols{8}
   \def\sep{1.4}

   \foreach \x in {1, ..., \cols} {
    \foreach \y in {1, ..., \rows} {
     \pgfmathtruncatemacro\z{(\rows-\y) * \cols + \x}
     \node[black] (\z) at (\x*\sep,\y*\sep) {\small \z};
    }
   }

   \foreach \x/\y in {1/4,4/4, 7/4, 2/3, 5/3, 8/3, 3/2, 6/2, 1/1, 4/1, 7/1 } {
   \node[circle,inner sep=11pt, draw =BrickRed,line width= 2pt]  at
      (\x*\sep,\y*\sep) {};
   }
    
   \foreach \x/\y in {6/4,2/4, 6/3, 2/2, 2/1, 6/1 } {
     \node[circle,inner sep=11pt, draw =RoyalBlue,line width= 2pt] at
      (\x*\sep,\y*\sep) {};
   }

     \node[circle,inner sep=13.5pt, draw =RoyalBlue,fill=yellow,fill opacity=0.2,line width= 2pt] at
      (2*\sep,3*\sep) {};
     
     \node[circle,inner sep=13.5pt, draw =RoyalBlue,fill=yellow,fill opacity=0.2,line width= 2pt] at
      (6*\sep,2*\sep) {};

     \draw[line width = 2pt] (12)+(0.6,0.6) rectangle ++(-0.6,-0.6);

     \draw[line width = 2pt] (24)+(0.6,0.6) rectangle ++(-0.6,-0.6);
 \end{tikzpicture}
\end{center}
The box around 12 and 24 indicates on what intervals are we guaranteed to have a
unique solution. This is because $M=m_1 \cdot m_2 = 3 \cdot 4 = 12$.

If we find the smallest solution $x$ (the one smaller then $M$) then we have
every other because they are all in the form of $x + kM$ for any $k \in \N$. 
 
 \end{alertblock}

\begin{block}{Drawing Products And Relations}
 One can draw the product $A \times B$ similarly to the way we draw Cartesian
 coordinates -- by distributing the elements of $A$ horizontally and those of
 $B$ vertically. Each point in the resulting grid represents an element of $A
 \times B$.

 Take, for example, $\clr{A = \{1, 2, 3, 4\}}$ and $\clb{B = \{a, b, c\}}$. We
 can depict the set $\clr{A} \times \clb{B}$ like this:
 \begin{center}
  \begin{tikzpicture}
   \foreach \x in {0, 1, 2, 3} {
    \pgfmathtruncatemacro\z{\x + 1}
    \node[BrickRed] (A\x) at (\x, -1) {\footnotesize $\z$};
   }
   \foreach \y/\n in {0/a, 1/b, 2/c} {
    \node[RoyalBlue] (B\y) at (-1, \y) {\footnotesize $\n$};
   }
   \foreach \x in {0, 1, 2, 3} {
    \foreach \y in {0, 1, 2} {
     \node[circle,inner sep=2pt,fill=black] at (\x,\y) {};
    }
   }
  \end{tikzpicture}
 \end{center}
 Any relation $\clg{R} \subseteq \clr{A} \times \clb{B}$ can now be easily drawn
 into the grid just by marking certain dots. For example, the relation $\clg{R =
 \{}(\clr{1},\clb{c}), (\clr{2},\clb{b}),(\clr{2},\clb{a})\clg{\}} \subseteq
 \clr{A} \times \clb{B}$ looks like this:
 \begin{center}
  \begin{tikzpicture}
   \foreach \x in {0, 1, 2, 3} {
    \pgfmathtruncatemacro\z{\x + 1}
    \node[BrickRed] (A\x) at (\x, -1) {\footnotesize $\z$};
   }
   \foreach \y/\n in {0/a, 1/b, 2/c} {
    \node[RoyalBlue] (B\y) at (-1, \y) {\footnotesize $\n$};
   }
   \foreach \x in {0, 1, 2, 3} {
    \foreach \y in {0, 1, 2} {
     \node[circle,inner sep=2pt,fill=black] at (\x,\y) {};
    }
   }
   \node[ultra thick,circle,inner sep=5pt,fill=none,draw=ForestGreen] at (0, 2)
    {};
   \node[ultra thick,circle,inner sep=5pt,fill=none,draw=ForestGreen] at (1, 0)
    {};
   \node[ultra thick,circle,inner sep=5pt,fill=none,draw=ForestGreen] at (1, 1)
    {};
  \end{tikzpicture}
 \end{center}
 There is another way to draw relations -- as arrows from $\clr{A}$ to
 $\clb{B}$. This style of drawing emphasises the `one-way' nature of relations.
 The same relation $\clg{R} \subseteq \clr{A} \times \clb{B}$ is drawn using
 arrows like this:
 \begin{center}
  \begin{tikzpicture}
   \foreach \y in {0, 1, 2, 3} {
    \pgfmathtruncatemacro\z{4-\y}
    \node[BrickRed] (A\z) at (-2,\y) {\footnotesize $\z$};
   }
   \foreach \y/\n in {0/c, 1/b, 2/a} {
    \node[RoyalBlue] (B\n) at (2,\y) {\footnotesize $\n$};
   }
   \draw[arrows={-Latex[length=16pt,width=12pt]},thick,ForestGreen] (A1) -- (Bc);
   \draw[arrows={-Latex[length=16pt,width=12pt]},thick,ForestGreen] (A2) -- (Bb);
   \draw[arrows={-Latex[length=16pt,width=12pt]},thick,ForestGreen] (A2) -- (Ba);
  \end{tikzpicture}
 \end{center}
\end{block}

\begin{exampleblock}{Equivalence \& Equivalence Classes}
 A relation $\clg{E} \subseteq \clr{A} \times \clr{A}$ (do note that it's a
 relation \textbf{on a set}) is called an \alert{equivalence} if it behaves
 \textbf{something like `equals'}, that is,
 \begin{itemize}[left=2em]
  \item[(R)] it's \alert{reflexive}, i.e. $\clr{a}\clg{E}\clr{a}$ for every
   $\clr{a} \in \clr{A}$ (every element is equivalent to itself);
  \item[(S)] it's \alert{symmetric}, i.e. if $\clr{a_1}\clg{E}\clr{a_2}$, then
   also $\clr{a_2}\clg{E}\clr{a_1}$ (all elements are \textbf{mutually}
   equivalent);
  \item[(T)] it's \alert{transitive}, i.e. if $\clr{a_1}\clg{E}\clr{a_2}$ and
   $\clr{a_2} \clg{E} \clr{a_3}$, then $\clr{a_1} \clg{E} \clr{a_3}$ (it
   \textbf{propagates} through a middle element).
 \end{itemize}
 A good example of equivalence to keep in mind is the relation `\textbf{being
 the same age}' on the set of all people. It's \alert{reflexive} because I'm as
 old as me. It's \alert{symmetric} because if I'm as old as you, then you are as
 old as me. And finally, it's \alert{transitive} because if I'm as old as you
 and you are as old as someone else, then I'm as old as that someone. This
 example also drives home the idea of equivalence `being like equals'. It's not
 that two people of the same age are \emph{equal}, they are \emph{equal by some
 criterion}. This is exactly what equivalence is, the relation of being equal by
 some criterion.

 With an equivalence $\clg{E} \subseteq \clr{A} \times \clr{A}$, we can divide
 the set $\clr{A}$ into `packets' of elements -- each packet consisting of
 elements which are \textbf{equivalent} by $\clg{E}$. For example, I can divide
 the set of all people by putting equally old people to the same packet.
 Formally, we write
 \[
  [\clr{a}]_{\clg{E}} = \{\clr{b} \in \clr{A} \mid \clr{a}\clg{E}\clr{b}\},
 \]
 in other words, $[\clr{a}]_{\clg{E}}$ is the set of all elements from $\clr{A}$
 that are $\clg{equivalent}$ to $\clr{a}$. We call it \alert{an equivalence
 class}. The element $\clr{a}$ is then called a \alert{representative}. 
\end{exampleblock}

\end{column}
\separatorcolumn

\end{columns}
\end{frame}

\end{document}
