\documentclass[a4paper,12pt]{article}

\usepackage[czech,english]{babel}
% Fonts %
\usepackage{fouriernc}
\usepackage[T1]{fontenc}

% Colors %
\usepackage[dvipsnames]{color}
\usepackage{xcolor}

% Page Layout %
\usepackage[margin=1in]{geometry}

% Fancy Headers %
\usepackage{fancyhdr}
\fancyhf{}
\cfoot{\thepage}
\rhead{}
\renewcommand{\headrulewidth}{0pt}
\setlength{\headheight}{16pt}

% Math
\usepackage{mathtools}
\usepackage{amssymb}
\usepackage{faktor}
\usepackage{import}
\usepackage{caption}
\usepackage{subcaption}
\usepackage{wrapfig}
\usepackage{enumitem}
\usepackage{tikz}
\usetikzlibrary{cd,positioning,babel,shapes}
\usepackage{tkz-base}
\usepackage{tkz-euclide}

% Theorems
\usepackage{thmtools}
\usepackage[thmmarks, amsmath, thref]{ntheorem}

% Title %
\title{\Huge\textsf{Math Exam -- PreIB 3.AB 3}\\
 \Large\textsf{Systems of Linear Equations}
 \author{Áďa Klepáčů}
 \date{March 9, 2023}
}

% Table of Contents %
\usepackage{hyperref}
\hypersetup{
 colorlinks=true,
 linktoc=all,
 linkcolor=blue
}

% Tables %
\usepackage{booktabs}
\usepackage{tabularx}

% Patch for hyphens
\usepackage{regexpatch}
\makeatletter
% Change the `-` delimiter to an active character
\xpatchparametertext\@@@cmidrule{-}{\cA-}{}{}
\xpatchparametertext\@cline{-}{\cA-}{}{}
\makeatother

\newcolumntype{s}{>{\centering\arraybackslash}p{.4\textwidth}}

% Operators %
\DeclareMathOperator{\Ker}{Ker}
\DeclareMathOperator{\Img}{Im}
\DeclareMathOperator{\End}{End}
\DeclareMathOperator{\Aut}{Aut}
\DeclareMathOperator{\Inn}{Inn}

% Common operators %
\newcommand{\R}{\mathbb{R}}
\newcommand{\N}{\mathbb{N}}
\newcommand{\Z}{\mathbb{Z}}
\newcommand{\Q}{\mathbb{Q}}
\newcommand{\C}{\mathbb{C}}

\newcommand{\tr}{\textcolor{red}}
\newcommand{\tb}{\textcolor{blue}}
\newcommand{\tg}{\textcolor{green}}
\newcommand{\tm}{\textcolor{magenta}}
\newcommand{\tv}{\textcolor{violet}}
\newcommand{\tmar}{\textcolor[HTML]{e23636}}
\newcommand{\tav}{\textcolor[HTML]{2e67a0}}

\definecolor{marvel}{HTML}{e23636}
\definecolor{avatar}{HTML}{2e67a0}

% American Paragraph Skip %
\setlength{\parindent}{0pt}
\setlength{\parskip}{1em}

% Document %
\pagestyle{fancy}
\begin{document}

\maketitle
\thispagestyle{fancy}

\begin{center}
 \textbf{\tr{DON'T FORGET TO EXPLAIN EVERYTHING EVEN IF YOU THINK IT'S
 OBVIOUS!}}
\end{center}

You're selling \tmar{Marvel Comics} and \tav{Avatar} toys. As \tav{Avatar} is
trending right now, you sell one \tav{Avatar} toy for \$40. \tmar{Marvel Comics}
toys don't sell as well and you can only cash in \$20 apiece.

Denote the number of sold \tmar{Marvel Comics} toys by $\tmar{m}$ and the number
of sold \tav{Avatar} toys by $\tav{a}$.

\begin{enumerate}[topsep=0pt,label=(\alph*)]
 \item \textbf{Define a linear function} $A(\tmar{m},\tav{a})$ in variables
  $\tmar{m}$ and $\tav{a}$ which calculates \textbf{your total revenue} based on
  the number of sold \tmar{Marvel Comics} and \tav{Avatar} toys.
\end{enumerate}

\newpage

It seems you're not the only one in the neighbourhood selling toys. Your
competitor seems to think that \tmar{Marvel Toys} don't really sell and only
prices them at \$15. On the other hand, he prices \tav{Avatar} toys at a
whopping \$45.

Surprisingly, at the end of the day, \textbf{both of you sold the same number of
each type of toy}. Your competitor's aggressive pricing paid off as he earned
\$1050 and you earned only \$1000.

\begin{enumerate}[label=(\alph*),topsep=0pt]
 \setcounter{enumi}{1}
 \item \textbf{Define a linear function} $B(\tmar{m},\tav{a})$ in variables
  $\tmar{m}$ and $\tav{a}$ which calculates \textbf{your competitor's revenue}
  based on the number of \tmar{Marvel Comics} and \tav{Avatar} toys he sold.
 \item Write down the \textbf{system of linear equations} which allows you to
  calculate the number of sold toys assuming that \textbf{you earned \$1000},
  \textbf{your competitor earned \$1050} and \textbf{both of you sold the same
  number of each type of toy}.\\
  Hint: Remember, total revenues are exactly the outputs of the functions $A$ 
  and $B$.
 \item Solve the system.
\end{enumerate}

\newpage

\begin{enumerate}[label=(\alph*),topsep=0pt]
 \setcounter{enumi}{4}
 \item Interpret the equations of the previous system as linear functions $f$
  and $g$ with output $\tmar{m}$ and input $\tav{a}$. Write down their
  definitions.\\
  Hint: Just isolate the variable $\tmar{m}$ in the equations.
\end{enumerate}

\newpage

In comes yet another competitor. He really doesn't like \tmar{Marvel} and prices
their toys at mere \$10. But he also seems to think that \tav{Avatar} toys are
going to sell no matter what and values them just as you do, at \$40. His
revenue at the end of the day is only \$800. You can see the graph of the
equation determining his revenue in the grid below.

\begin{enumerate}[label=(\alph*),topsep=0pt]
 \setcounter{enumi}{5}
 \item Draw the graphs of the linear functions $f$ and $g$ into the grid below.
  Did the third competitor sell the same number of each type of toy as you
  did? -- \textbf{Read this information from the graphs!}
\end{enumerate}

\begin{center}
 \begin{tikzpicture}[scale=1.25]
  \tkzInit[xmax=30,ymax=90,xmin=0,ymin=0,xstep=5,ystep=10]
  \tkzGrid
  \tkzLabelX[text=avatar,font=\scriptsize]
  \tkzLabelY[text=marvel,font=\scriptsize]
  \tkzDrawX[label=$\tav{a}$]
  \tkzDrawY[label=$\tmar{m}$]
  \tkzDefPoint(0,80){a};
  \tkzDefPoint(20,0){b};
  \tkzDrawPoints[size=6pt](a,b);
  \tkzDrawLine[add=0 and 0](a,b);

  \tkzLabelLine[rotate=-63,yshift=3mm,pos=0.2](a,b){third competitor}
 \end{tikzpicture}
\end{center}

\end{document}
