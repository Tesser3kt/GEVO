\documentclass[a4paper,11pt]{article}

\usepackage[czech,english]{babel}
% Fonts %
\usepackage{fouriernc}
\usepackage[T1]{fontenc}

% Colors %
\usepackage[dvipsnames]{color}
\usepackage{xcolor}

% Page Layout %
\usepackage[margin=1.5in]{geometry}

% Fancy Headers %
\usepackage{fancyhdr}
\fancyhf{}
\cfoot{\thepage}
\rhead{}
\renewcommand{\headrulewidth}{0pt}
\setlength{\headheight}{16pt}

% Math
\usepackage{mathtools}
\usepackage{amssymb}
\usepackage{faktor}
\usepackage{import}
\usepackage{caption}
\usepackage{subcaption}
\usepackage{wrapfig}
\usepackage{enumitem}
\usepackage{tikz}
\usetikzlibrary{cd,positioning,babel,shapes}
\usepackage{tkz-base}
\usepackage{tkz-euclide}

% Theorems
\usepackage{thmtools}
\usepackage[thmmarks, amsmath, thref]{ntheorem}

% Title %
\title{\Huge\textsf{Math Exam -- PreIB 3.AB 2}\\
 \Large\textsf{Systems of Linear Equations}
 \author{Áďa Klepáčů}
 \date{March 6, 2023}
}

% Table of Contents %
\usepackage{hyperref}
\hypersetup{
 colorlinks=true,
 linktoc=all,
 linkcolor=blue
}

% Tables %
\usepackage{booktabs}
\usepackage{tabularx}

% Patch for hyphens
\usepackage{regexpatch}
\makeatletter
% Change the `-` delimiter to an active character
\xpatchparametertext\@@@cmidrule{-}{\cA-}{}{}
\xpatchparametertext\@cline{-}{\cA-}{}{}
\makeatother

\newcolumntype{s}{>{\centering\arraybackslash}p{.4\textwidth}}

% Operators %
\DeclareMathOperator{\Ker}{Ker}
\DeclareMathOperator{\Img}{Im}
\DeclareMathOperator{\End}{End}
\DeclareMathOperator{\Aut}{Aut}
\DeclareMathOperator{\Inn}{Inn}

% Common operators %
\newcommand{\R}{\mathbb{R}}
\newcommand{\N}{\mathbb{N}}
\newcommand{\Z}{\mathbb{Z}}
\newcommand{\Q}{\mathbb{Q}}
\newcommand{\C}{\mathbb{C}}

\newcommand{\tr}{\textcolor{red}}
\newcommand{\tb}{\textcolor{blue}}
\newcommand{\tg}{\textcolor{green}}
\newcommand{\tm}{\textcolor{magenta}}
\newcommand{\tv}{\textcolor{violet}}

% American Paragraph Skip %
\setlength{\parindent}{0pt}
\setlength{\parskip}{1em}

% Document %
\pagestyle{fancy}
\begin{document}

\maketitle
\thispagestyle{fancy}

\begin{center}
 \textbf{\tr{DON'T FORGET TO EXPLAIN EVERYTHING EVEN IF YOU THINK IT'S
 OBVIOUS!}}
\end{center}

I know that train $\tr{A}$ has an average speed of \textbf{60 km/h} and train
$\tb{B}$ travels on average at \textbf{100 km/h}. I also know that train
$\tr{A}$ is at this moment 3 times as far from the depot as train $\tb{B}$ is.

I denote by $y$ \textbf{the distance} (in km) train $\tb{B}$ is from the depot.
I also denote by $x$ \textbf{the time} (in hours) both trains have travelled for
from the moment they started moving.

\begin{enumerate}[label=(\alph*),topsep=0pt]
 \item Express \textbf{the distance from the depot} of these two trains as
  linear functions $\tr{A(x,y)}$ and $\tb{B(x,y)}$ with inputs: 
  \begin{itemize}[topsep=0pt]
   \item the time they have been moving ($x$) and
   \item their initial distance from the depot ($y$).
  \end{itemize}
\end{enumerate}
\newpage

The terminal station of train $\tr{A}$ is 330 km away from the depot and the
terminal station of train $\tb{B}$ is 350 km away from the depot. I know that
both trains \textbf{reached the terminal station at the same time}.
\begin{enumerate}[label=(\alph*),topsep=0pt]
 \setcounter{enumi}{1}
 \item Write a system of two linear equations in two variables which will allow
  me to calculate how long the trains travelled to their terminal station ($x$)
  and their initial distances from the depot ($y$).
 \item Solve the system.
\end{enumerate}

\newpage

\begin{enumerate}[label=(\alph*),topsep=0pt]
 \setcounter{enumi}{3}
 \item Interpret the equations of the system as linear functions in one
  variable. Draw their graphs as lines (it needn't be precise). You may want to
  use the following grid. \textbf{\tr{Make sure you are drawing them correctly.
  Their intersection must be the solution to the system.}}
  \begin{center}
   \begin{tikzpicture}[scale=1.25]
    \tkzInit[xmax=8,ymax=350,xmin=0,ymin=0,ystep=50]
    \tkzGrid
    \tkzLabelX[font=\scriptsize]
    \tkzLabelY[font=\scriptsize]
    \tkzDrawX
    \tkzDrawY
   \end{tikzpicture}
  \end{center}
\end{enumerate}

\newpage

\begin{enumerate}[label=(\alph*),topsep=0pt]
 \setcounter{enumi}{4}
 \item Let's say I add another train, $\tg{C}$. This train starts moving at the
  exact same time as $\tr{A}$ and $\tb{B}$ at \textbf{80 km/h}. Its initial
  distance from the depot is \textbf{twice that of $\tr{A}$} and the distance of
  its terminal station from the depot is \textbf{400 km}.

  As you did before, express the initial distance of $\tg{C}$, that is, the
  variable $y$, as a linear function dependent on its velocity $(x)$. Finally,
  draw the graph of this function (ideally to the same grid as $\tr{A}$ and
  $\tb{B}$).
 \item Does $\tg{C}$ reach its terminal station at the same time as $\tr{A}$ and
  $\tb{B}$? -- \textbf{Read this information from the graph!}

  If not, \textbf{change its velocity} so that it does reach its terminal
  station together with the two other trains (you do \textbf{not} have to draw
  the graph of the changed linear function).
\end{enumerate}


\end{document}
