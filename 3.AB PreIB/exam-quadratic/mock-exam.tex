\documentclass[a4paper,11pt]{article}

\usepackage[czech,english]{babel}
% Fonts %
\usepackage{fouriernc}
\usepackage[T1]{fontenc}

% Colors %
\usepackage[dvipsnames]{color}
\usepackage{xcolor}

% Page Layout %
\usepackage[margin=1.5in]{geometry}

% Fancy Headers %
\usepackage{fancyhdr}
\fancyhf{}
\cfoot{\thepage}
\rhead{}
\renewcommand{\headrulewidth}{0pt}
\setlength{\headheight}{16pt}

% Math
\usepackage{mathtools}
\usepackage{amssymb}
\usepackage{faktor}
\usepackage{import}
\usepackage{caption}
\usepackage{subcaption}
\usepackage{wrapfig}
\usepackage{enumitem}
\setlist{topsep=0pt}

\usepackage{tikz}
\usetikzlibrary{cd,positioning,babel,shapes}
\usepackage{tkz-base}
\usepackage{tkz-euclide}

% Theorems
\usepackage{thmtools}
\usepackage[thmmarks, amsmath, thref]{ntheorem}

\theoremsymbol{\ensuremath{\blacksquare}}
\newtheorem*{solution}{Possible solution.}

% Title %
\title{\Huge\textsf{Mock Exam}\\
 \Large\textsf{Quadratic Functions \& Equations}
 \author{Áďa Klepáčů}
 \date{\today}
}

% Table of Contents %
\usepackage{hyperref}
\hypersetup{
 colorlinks=true,
 linktoc=all,
 linkcolor=blue
}

% Tables %
\usepackage{booktabs}
\usepackage{tabularx}

% Patch for hyphens
\usepackage{regexpatch}
\makeatletter
% Change the `-` delimiter to an active character
\xpatchparametertext\@@@cmidrule{-}{\cA-}{}{}
\xpatchparametertext\@cline{-}{\cA-}{}{}
\makeatother

\newcolumntype{s}{>{\centering\arraybackslash}p{.4\textwidth}}

% Operators %
\DeclareMathOperator{\Ker}{Ker}
\DeclareMathOperator{\Img}{Im}
\DeclareMathOperator{\End}{End}
\DeclareMathOperator{\Aut}{Aut}
\DeclareMathOperator{\Inn}{Inn}

% Common operators %
\newcommand{\R}{\mathbb{R}}
\newcommand{\N}{\mathbb{N}}
\newcommand{\Z}{\mathbb{Z}}
\newcommand{\Q}{\mathbb{Q}}
\newcommand{\C}{\mathbb{C}}

\newcommand{\tr}{\textcolor{red}}
\newcommand{\tb}{\textcolor{blue}}
\newcommand{\tg}{\textcolor{green}}
\newcommand{\tm}{\textcolor{magenta}}
\newcommand{\tv}{\textcolor{violet}}

% American Paragraph Skip %
\setlength{\parindent}{0pt}
\setlength{\parskip}{1em}

% Document %
\pagestyle{fancy}
\begin{document}

\maketitle
\thispagestyle{fancy}

Meet Mr. Newton, the innkeeper. Mr. Newton regularly supplies his inn with high
quality beer he can however only order in huge barrels. Every time he orders a
batch, some litres of beer go to waste.

Mr. Newton is not a maths guy. Sad that such a good beer is being wasted by his
own inadequacy, he's started to experiment. He knows that \textbf{he needs at
least two huge barrels} of beer a day to satisfy his thirsty customers. Of
course, he doesn't need to order new barrels every day but to meet his quality
standards, he mustn't keep the barrelled beer stocked for too long, either.

He measured that when he ordered
\begin{itemize}
 \item \textbf{three barrels} of beer, \textbf{14 litres} of it went to waste;
 \item \textbf{seven barrels} of beer, \textbf{18 litres} of it went to waste;
 \item \textbf{nine barrels} of beer, \textbf{8 litres} of it went to waste.
\end{itemize}

Help make Mr. Newton happy by minimizing the amount of beer that go to waste,
or, at least, tell him what's the worst possible order he could make.

\begin{enumerate}
 \item Given only three pieces of data, the best we can do is model the
  situation using a quadratic function having as input the number of ordered
  barrels and the losses in litres as output. We know that it should match the
  provided data precisely. That is, our function, which we label $f$, must
  satisfy
  \begin{equation*}
   \begin{split}
    f(3) &= 14 \\
    f(7) &= 18 \\
    f(9) &= 8.
   \end{split}
  \end{equation*}
  Determine the definition of such a quadratic function.
 \item Mr. Newton is most afraid of losing even more litres of beer than he
  already did. He demands you first tell him, what would be the \textbf{worst}
  order he could make. That is, calculate the number of barrels whose order would make
  him lose the most litres of beer.
 \item Now that Mr. Newton has calmed down, knowing he won't lose too much beer,
  he also asks you to help him determine how many barrels of beer he should
  order \textbf{to lose no beer at all}. Remember, that he always needs to order
  at least two barrels. \textbf{\uppercase{\tr{Use the Viète formulae to solve
  this!}}}
 \item Use the results from 3 to factor $f$ as a product of two linear
  functions.
 \item Mr. Newton, happy with how his inn is now coming along, opens another.
  But, he's immediately stricken with grief, when his new inn starts wasting
  beer like crazy. He promptly starts to measure how much beer he's losing and,
  using the same method as in 1, you calculate that the beer losses of his
  second inn can be modelled by the quadratic function
  \[
   g(x) = -x^2 + 10x - 21
  \]
  where $x$ is the number of ordered barrels. \textbf{How many barrels} should
  Mr. Newton order \textbf{for both of his inns}, so that the \textbf{total
  losses} (that is, the losses of the first inn + losses of the second inn)
  \textbf{are as small as possible}?
\end{enumerate}

\begin{solution}
 \hfill
 \begin{enumerate}
  \item Since $f$ is supposed to be a quadratic function, we know that its
   general formula is $f(x) = ax^2 + bx + c$ for some real numbers $a,b$ and
   $c$. Plugging the known inputs and outputs into it gives the system of linear
   equations
   \begin{equation*}
    \begin{split}
     f(3) &= a \cdot 3^2 + b \cdot 3 + c = 14\\
     f(7) &= a \cdot 7^2 + b \cdot 7 + c = 18\\
     f(9) &= a \cdot 9^2 + b \cdot 9 + c = 8.
    \end{split}
   \end{equation*}
   If we subtract the first equation from the second and the second equation
   from the third, we get
   \begin{equation*}
    \begin{split}
     40a + 4b &= 4 \\
     32a + 2b &= -10.
    \end{split}
   \end{equation*}
   We can divide the first equation by $4$ and the second by $2$, to further get
   \begin{equation*}
    \begin{split}
     10a + b &= 1 \\
     16a + b &= -5. \\
    \end{split}
   \end{equation*}
   We can then subtract the first equation from the second to arrive at an
   equation with only one variable
   \[
    6a = -6,
   \]
   whose solution is $a = -1$. Plugging this into the first equation above gives
   \[
    -10 + b = 1,
   \]
   that is, $b = 11$. Finally, substituting $a$ and $b$ into, for example, the
   first original equation yields
   \[
    -9 + 33 + c = 14,
   \]
   from which we calculate that $c = 10$. It follows that we can model Mr.
   Newton's losses by the function $f(x) = -x^2 + 11x - 10$.
  \item Since the function $f$ describes Mr. Newton's losses, the higher it is,
   the more litres of beer he loses. Hence, to find the number of barrels which
   causes the most number of litres of beer to go to waste, we're looking for
   the maximal value of the function $f$. The graph of $f$ is a parabola which
   looks like a `hill' in this case because $a = -1 < 0$. This means that $f$
   does have a maximum at its vertex, whose first coordinate we know is $x = -b
   / 2a$. In our case, we get that $x = -11 / (-2) = 5.5$ so Mr. Newton suffers
   the most losses if he orders either 5 or 6 barrels of beer.
  \item In order to compute the number of barrels whose order would cause no
   losses to Mr. Newton, we need to solve the equation
   \[
    f(x) = 0.
   \]
   Using Viète's formulae, we know that if $x_1$ and $x_2$ are the solutions to
   this equation, then
   \begin{equation*}
    \begin{split}
     x_1 + x_2 &= -\frac{b}{a} = \frac{-11}{-1} = 11,\\
     x_1 \cdot x_2 &= \frac{c}{a} = \frac{-10}{-1} = 10.\\
    \end{split}
   \end{equation*}
   An immediate solution that comes to mind is $x_1 = 1$ and $x_2 = 10$. We know
   that Mr. Newton needs to order at least $2$ barrels to keep his customers
   satisfied, so the only relevant solution to his problem is the order of $10$
   barrels.
  \item We know that if $x_1$ and $x_2$ are the roots of a quadratic function
   $f(x) = ax^2 + bx + c$, then this function decomposes as $f(x) = a(x -
   x_1)(x-x_2)$. In our case, this means that
   \[
    f(x) = -(x - 1)(x - 10).
   \]
  \item As the function $f$ describes the losses of Mr. Newton's first inn and
   $g$ describes the losses of his second inn, the total losses are given by the
   quadratic function $f(x) + g(x)$. We want the total losses to be as close to
   $0$ as possible, which entails solving the quadratic equation
   \[
    f(x) + g(x) = 0.
   \]
   Using the definitions of these functions, we get
   \[
    -x^2 + 11x - 10 - x^2 + 10x - 21 = -2x^2 + 21x - 31 = 0.
   \]
   This equation would not be trivial to solve using Viète's formulae, so we use
   the general formula for the solutions to a quadratic equation instead. We get
   \[
    x = \frac{-b \pm \sqrt{b^2 - 4ac}}{2a} = \frac{-21 \pm \sqrt{441 - 248}}{-4}
    = \frac{-21 \pm \sqrt{193}}{-4}.
   \]
   As $\sqrt{193}$ is approximately $14$ (because $14^2 = 196$), the two
   solutions are approximately
   \[
    x_1 = \frac{-21 + 14}{-4} = \frac{7}{4} \quad \text{and} \quad x_2 =
    \frac{-21 - 14}{-4} = \frac{35}{4}.
   \]
   As Mr. Newton needs to order at least 4 barrels for both of his inns
   combined, the first solution is useless. Unfortunately, this time, Mr. Newton
   cannot avoid losing beer completely but the number of barrels whose order
   minimizes the number of wasted beer is $9$ because $35 / 4$ is just $1 / 4$
   short of $9$.
 \end{enumerate}
\end{solution}
\end{document}
