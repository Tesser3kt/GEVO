\documentclass[a4paper,11pt]{article}

\usepackage[czech,english]{babel}
% Fonts %
\usepackage{fouriernc}
\usepackage[T1]{fontenc}

% Colors %
\usepackage[dvipsnames]{color}
\usepackage[dvipsnames]{xcolor}

% Page Layout %
\usepackage[margin=1.5in]{geometry}

% Fancy Headers %
\usepackage{fancyhdr}
\fancyhf{}
\cfoot{\thepage}
\rhead{}
\renewcommand{\headrulewidth}{0pt}
\setlength{\headheight}{16pt}

% Math
\usepackage{mathtools}
\usepackage{amssymb}
\usepackage{faktor}
\usepackage{import}
\usepackage{caption}
\usepackage{subcaption}
\usepackage{wrapfig}
\usepackage{enumitem}
\setlist{topsep=0pt}

\usepackage{tikz}
\usetikzlibrary{cd,positioning,babel,shapes}
\usepackage{tkz-base}
\usepackage{tkz-euclide}

% Theorems
\usepackage{thmtools}
\usepackage[thmmarks, amsmath, thref]{ntheorem}

\theoremsymbol{\ensuremath{\blacksquare}}
\newtheorem*{solution}{Possible solution.}

% Title %
\title{\Huge\textsf{Math Exam -- PreIB 3.AB 3}\\
 \Large\textsf{Quadratic Functions \& Equations}
 \author{Áďa Klepáčů}
 \date{\today}
}

% Table of Contents %
\usepackage{hyperref}
\hypersetup{
 colorlinks=true,
 linktoc=all,
 linkcolor=blue
}

% Tables %
\usepackage{booktabs}
\usepackage{tabularx}

% Patch for hyphens
\usepackage{regexpatch}
\makeatletter
% Change the `-` delimiter to an active character
\xpatchparametertext\@@@cmidrule{-}{\cA-}{}{}
\xpatchparametertext\@cline{-}{\cA-}{}{}
\makeatother

\newcolumntype{s}{>{\centering\arraybackslash}p{.4\textwidth}}

% Operators %
\DeclareMathOperator{\Ker}{Ker}
\DeclareMathOperator{\Img}{Im}
\DeclareMathOperator{\End}{End}
\DeclareMathOperator{\Aut}{Aut}
\DeclareMathOperator{\Inn}{Inn}

% Common operators %
\newcommand{\R}{\mathbb{R}}
\newcommand{\N}{\mathbb{N}}
\newcommand{\Z}{\mathbb{Z}}
\newcommand{\Q}{\mathbb{Q}}
\newcommand{\C}{\mathbb{C}}

\newcommand{\clr}{\textcolor{red}}
\newcommand{\clb}{\textcolor{blue}}
\newcommand{\clg}{\textcolor{green}}
\newcommand{\clm}{\textcolor{magenta}}
\newcommand{\clv}{\textcolor{violet}}
\newcommand{\clbr}{\textcolor{Sepia}}

% American Paragraph Skip %
\setlength{\parindent}{0pt}
\setlength{\parskip}{1em}

% Document %
\pagestyle{fancy}
\begin{document}

\maketitle
\thispagestyle{fancy}

\begin{center}
 \hrule
 \textbf{\clr{DON'T FORGET TO EXPLAIN EVERYTHING EVEN IF YOU THINK IT'S
 OBVIOUS!}}
 \vspace{2ex}
 \hrule
 \emph{Some of the numbers arising in the solutions are quite large. Use a
 calculator (on your phone/tablet) if you wish.}
\end{center}

Meet Don Corleone, a crime boss. Don Corleone runs a fake car rental service to
cover a tiny part of his profits from rigged slot machines and usury. Don
Corleone's personal accountant has stolen a small set of data assembled over a
three weeks period from an actual car rental company. He knows the \clb{average
price} per car per day of rental and the \clm{profit} made each week.

You, being a notorious underground mathematician, have been hired to perform
various calculations on this data. You'd better do your job well!

\begin{enumerate}
 \item Don Corleone's accountant has presented you with the following table.
  \begin{table}[h]
   \renewcommand{\arraystretch}{1.3}
   \centering
   \begin{tabular}{c|cc}
    & \textbf{\clb{Average price per car per day}} & \textbf{\clm{Profit}} \\
    \toprule
    \textbf{Week 1} & \$75 & \$13200\\
    \textbf{Week 2} & \$65 & \$11220\\
    \textbf{Week 3} & \$95 & \$15360
   \end{tabular}
  \end{table}
 
  Since you have only three pieces of information, model Don Corleone's fake
  profits as a quadratic function $\clm{P}$. That is, find a quadratic function
  labelled $\clm{P}$ whose inputs are the \clb{average prices} and outputs are
  the \clm{profits} from the table above.
 \item Don Corleone wishes to cover as large an amount of his profits as
  possible by the fake car rental company. Yet, he needs to make the forged
  accounts believable. \textbf{Find the \clb{average price}} per rental of one
  car per day \textbf{which would generate the most \clm{profit}} based on the
  provided data. Also, \textbf{calculate the actual maximal \clm{profit}}.
\end{enumerate}
\clearpage
\null
\clearpage
\begin{enumerate}
 \setcounter{enumi}{2}
 \item A court decision has forced Don Corleone to supply additional data about
  his fake company's operations; in particular, the accounts of the \clr{average
  maintenance cost} per one car per week.
 
  Denote the \clb{average rental price} variable by $\clb{p}$ and the
  \clbr{number of cars rented} on average per week by $\clbr{c}$. The expected
  situation is that \textbf{the higher the rental price, the fewer cars you
  rent}. That is, you can approximate \textbf{$\clbr{c}$ as a linear function in
  $\clb{p}$}. You know that the total \textbf{profit is then a quadratic
  function in $\clb{p}$}. Concretely,
  \begin{equation*}
   \begin{split}
    \text{profit} &= \text{\clbr{cars rented}} \cdot \text{\clb{price per car}} -
    \text{\clbr{cars rented}} \cdot \text{\clr{maintenance cost}}\\
                  &= \clbr{c}(\clb{p}) \cdot \clb{p} - \clbr{c}(\clb{p}) \cdot
                  \clr{m}, \\
   \end{split}
  \end{equation*}
  where $\clr{m}$ is the unknown \clr{average maintenance cost} per car per
  week. Can you determine the \clr{maintenance cost} without knowing the
  \clbr{number of cars rented}? If yes, how? If not, why?
 \item In the end, Don Corleone's accountant has managed to also steal the data
  containing the \clbr{number of cars rented each week}. It is given in the
  following table.
  \begin{table}[h]
   \centering
   \begin{tabular}{c|c}
    & \textbf{\clbr{Cars rented}} \\
    \toprule
    \textbf{Week 1} & 300\\
    \textbf{Week 2} & 330\\
    \textbf{Week 3} & 240
   \end{tabular}
  \end{table}
 
  Using this table, \textbf{interpret the} \clbr{number of cars rented} each
  week \textbf{as a linear function in \clb{price} $\clb{p}$}, that is, determine
  $\clbr{c}(\clb{p})$.
 \item Finally, now that you know $\clbr{c}(\clb{p})$, \textbf{determine the
  \clr{maintenance cost}} per car per week, $\clr{m}$.\\
  \textbf{Hint:} Can you perhaps factor out $\clbr{c}(\clb{p})$ from the right
  side of the formula
  \[
   \clm{P}(\clb{p}) = \clbr{c}(\clb{p}) \cdot \clb{p} - \clbr{c}(\clb{p}) \cdot
   \clr{m}?
  \]
\end{enumerate}
\clearpage
\null
\clearpage
\end{document}
