% Unofficial University of Cambridge Poster Template
% https://github.com/andiac/gemini-cam
% a fork of https://github.com/anishathalye/gemini
% also refer to https://github.com/k4rtik/uchicago-poster

\documentclass[final]{beamer}

% ====================
% Packages
% ====================

\usepackage[T1]{fontenc}
\usepackage{lmodern}
\usepackage[orientation=portrait,size=custom,width=120,height=120,scale=1]{beamerposter}
\usetheme{gemini}
\usepackage[dvipsnames]{xcolor}
\usecolortheme{nott}
\usepackage{graphicx}
\usepackage{booktabs}
\usepackage{tikz}
\usetikzlibrary{patterns,decorations.pathmorphing}
\usepackage{tkz-euclide}
\tikzset{point style/.style = {%
  draw = black,
  inner sep = 0pt,
  shape = circle,
  minimum size = 5pt,
  fill = black
 },
 every picture/.append style = {
  scale = 1.5
 },
 every node/.append style={
  scale=1.5
 }
}
\usepackage{pgfplots}
\pgfplotsset{compat=1.14}
\usepackage{anyfontsize}
\usepackage{caption}
\usepackage{subcaption}

% ====================
% Lengths
% ====================

% If you have N columns, choose \sepwidth and \colwidth such that
% (N+1)*\sepwidth + N*\colwidth = \paperwidth
\newlength{\sepwidth}
\newlength{\colwidth}
\setlength{\sepwidth}{0.01\paperwidth}
\setlength{\colwidth}{0.32\paperwidth}

\newcommand{\separatorcolumn}{\begin{column}{\sepwidth}\end{column}}
\newcommand{\bfalert}[1]{\textbf{\alert{#1}}}

% Math shortcuts
\newcommand{\R}{\mathbb{R}}

% ====================
% Title
% ====================

\title{Logic \& Set Theory Cheatsheet}

\author{3.AB PreIB Math}

\institute[shortinst]{Adam Klepáč}

% ====================
% Footer (optional)
% ====================

% \footercontent{
%   \href{https://utfpr.edu.br/ct/ppgca}{utfpr.edu.br/ct/ppgca} \hfill
%   Mostra de Trabalhos do PPGCA --- TechTalks 2024 \hfill
%   \href{mailto:ppgca-ct@utfpr.edu.br}{ppgca-ct@utfpr.edu.br}}
% (can be left out to remove footer)


% ====================
% Logo (optional)
% ====================

% use this to include logos on the left and/or right side of the header:
\logoright{\includegraphics[height=3.5cm]{logos/logo-white.png}}
% \logoleft{\hspace{20ex}\includegraphics[height=3.5cm]{logos/ppgca-logo.png}}

% ====================
% Body
% ====================

\begin{document}

% Refer to https://github.com/k4rtik/uchicago-poster
% logo: https://www.cam.ac.uk/brand-resources/about-the-logo/logo-downloads
% \addtobeamertemplate{headline}{}
% {
%     \begin{tikzpicture}[remember picture,overlay]
%       \node [anchor=north west, inner sep=3cm] at ([xshift=-2.5cm,yshift=1.75cm]current page.north west)
%       {\includegraphics[height=7cm]{logos/unott-logo.eps}}; 
%     \end{tikzpicture}
% }

\begin{frame}[t]
\begin{columns}[t]
\separatorcolumn

\begin{column}{\colwidth}

 \begin{block}{Logic}
  \alert{Logic} is the language of mathematics. It uses \alert{propositions} to
  talk about sets.

  Propositions are sentences which can be either true or false. For example
  \begin{itemize}[label=\textbullet,left=24pt]
   \item `\textbf{Cats are black.}' is a proposition;
   \item `\textbf{How are you?}' is \emph{not} a proposition;
   \item `\textbf{We will have colonised Mars by 2500.} is also a proposition.
  \end{itemize}
 \end{block}
 As the third example suggests, we need not necessarily know whether a
 proposition is true or false -- it remains a proposition anyway.

 \vspace{1em}

 \begin{exampleblock}{Logical Conjunctions}
  Propositions can be joined together using \alert{logical conjunctions}. They
  pretty much correspond to the conjunctions of natural language. Let us
  consider two propositions:
  \begin{align*}
   p &= \text{`It's raining outside.'}\\
   q &= \text{`I'll stay at home.'}
  \end{align*}
  \begin{itemize}[left=40pt]
   \item[($ \wedge $)] Logical \alert{and} forms a proposition that is only
    \alert{true} if both of its constituents are also \alert{true}. In natural
    language, the proposition $p \wedge q$ can be expressed as
    \[
     p \alert{ \wedge } q = \text{`It's raining outside \alert{and} I'll stay at
     home.'}
    \]
   \item[($ \vee $)] Logical \alert{or} forms a proposition that is \alert{true}
    if at least one of its constituents is \alert{true}. In natural language,
    the proposition $p \vee q$ can be expressed as
    \[
     p \alert{ \vee } q = \text{`It's raining outside \alert{or} I'll stay at
     home.'}
    \]
    In mathematical logic, \alert{or} is \textbf{not exclusive}! This means that
    $p \alert{ \vee } q$ is true even if both $p$ and $q$ are true.
   \item[($\neg $)] Logical \alert{not} isn't strictly speaking a conjunction
    but I include it anyway. It reverses the truth value of a proposition. For
    example, the proposition $\alert{\neg }p$ can be read as
    \[
     \alert{\neg }p = \text{`It's \alert{not} raining outside.'}
    \]
    It follows that $\alert{\neg }p$ is \alert{true} exactly when $p$ is
    \alert{false} and vice versa.
   \item[($ \Rightarrow $)] Logical \alert{implication} is a conjunction that
    makes the first proposition into an \emph{assumption} or \emph{premise} and
    the second one into a \emph{conclusion}. The proposition $p \alert{
    \Rightarrow } q$ is read in multiple ways, to list a few: 
    \begin{align*}
     p \alert{ \Rightarrow } q &= \text{`\alert{If} it's raining outside,
    \alert{then} I'll stay at home.'}\\
     p \alert{ \Rightarrow } q &= \text{`It raining outside \alert{implies that}
     I'll stay at home.'}\\
     p \alert{ \Rightarrow } q &= \text{`\alert{Assuming} it's raining outside,
     I'll stay at home.'}\\
    \end{align*}
    The implication is tricky. It's true if both $p$ and $q$ are true and false
    if $p$ is true but $q$ is false. However, it is \alert{always true} if $p$
    is \alert{false}. That is because, in mathematical logic, whatever follows
    from a lie is automatically true.
   \item[($ \Leftrightarrow $)] Logical \alert{equivalence} is true only if both
    propositions have the \alert{same truth value} -- they're both true or both
    false. In natural language, it is typically read like this:
    \[
     p \alert{ \Leftrightarrow }q = \text{`It's raining \alert{if and only if}
     I stay at home.'}
    \]
    Equivalence is basically just a two-way implication. The proposition $p$ is
    both a premise and a conclusion to $q$ and $q$ is both a premise and a
    conclusion to $p$. If it's raining outside, I stay at home and if I stay at
    home, then it's raining outside.
  \end{itemize}
 \end{exampleblock}
\end{column}

\separatorcolumn

\begin{column}{\colwidth}




\end{column}
\separatorcolumn

\begin{column}{\colwidth}



\end{column}
\separatorcolumn

\end{columns}
\end{frame}

\end{document}
