% Unofficial University of Cambridge Poster Template
% https://github.com/andiac/gemini-cam
% a fork of https://github.com/anishathalye/gemini
% also refer to https://github.com/k4rtik/uchicago-poster

\documentclass[final]{beamer}

% ====================
% Packages
% ====================

\usepackage[T1]{fontenc}
\usepackage{lmodern}
\usepackage[orientation=portrait,size=custom,width=120,height=100,scale=1]{beamerposter}
\usetheme{gemini}
\usepackage[dvipsnames]{xcolor}
\usecolortheme{nott}
\usepackage{graphicx}
\usepackage{booktabs}
\usepackage{tikz}
\usetikzlibrary{patterns,decorations.pathmorphing,shapes.geometric}
\usepackage{tkz-euclide}
\tikzset{point style/.style = {%
  draw = black,
  inner sep = 0pt,
  shape = circle,
  minimum size = 5pt,
  fill = black
 },
 every picture/.append style = {
  scale = 1.5
 },
 every node/.append style={
  scale=1.5
 }
}
\usepackage{pgfplots}
\pgfplotsset{compat=1.14}
\usepackage{anyfontsize}
\usepackage{caption}
\usepackage{subcaption}

% ====================
% Lengths
% ====================

% If you have N columns, choose \sepwidth and \colwidth such that
% (N+1)*\sepwidth + N*\colwidth = \paperwidth
\newlength{\sepwidth}
\newlength{\colwidth}
\setlength{\sepwidth}{0.01\paperwidth}
\setlength{\colwidth}{0.32\paperwidth}

\newcommand{\separatorcolumn}{\begin{column}{\sepwidth}\end{column}}
\newcommand{\bfalert}[1]{\textbf{\alert{#1}}}

% Math shortcuts
\newcommand{\R}{\mathbb{R}}

% Inline shapes
\newcommand{\mysquare}{\tikz[baseline=-7pt]{%
  \node[rectangle,draw,thick,inner sep=6pt] at (0,0) {};
}}
\newcommand{\mytria}{\tikz[baseline=-3.25pt]{%
  \node[isosceles triangle,isosceles triangle apex angle=60,draw,thick,inner
  sep=3.25pt,rotate=90] at (0,0) {};
}}
\newcommand{\mycirc}{\tikz[baseline=-7pt]{%
  \node[circle,draw,thick,inner sep=4.5pt,baseline=0.5ex,rotate=90] at (0,0) {};
}}
\newcommand{\mycross}{\tikz[baseline=-7pt,scale=0.2]{%
  \draw[thick] (-1,1) -- (1,-1);
  \draw[thick] (-1,-1) -- (1,1);
}}

% Colors %
\newcommand{\clr}{\textcolor{BrickRed}}
\newcommand{\clb}{\textcolor{RoyalBlue}}
\newcommand{\clg}{\textcolor{ForestGreen}}
\newcommand{\clm}{\textcolor{Fuchsia}}
\newcommand{\clv}{\textcolor{violet}}
\newcommand{\clbr}{\textcolor{Sepia}}
\newcommand{\cly}{\textcolor{Dandelion}}

% ====================
% Title
% ====================

\title{Logic \& Set Theory Cheatsheet}

\author{3.AB PreIB Math}

\institute[shortinst]{Adam Klepáč}

% ====================
% Footer (optional)
% ====================

% \footercontent{
%   \href{https://utfpr.edu.br/ct/ppgca}{utfpr.edu.br/ct/ppgca} \hfill
%   Mostra de Trabalhos do PPGCA --- TechTalks 2024 \hfill
%   \href{mailto:ppgca-ct@utfpr.edu.br}{ppgca-ct@utfpr.edu.br}}
% (can be left out to remove footer)


% ====================
% Logo (optional)
% ====================

% use this to include logos on the left and/or right side of the header:
\logoright{\includegraphics[height=3.5cm]{logos/logo-white.png}}
% \logoleft{\hspace{20ex}\includegraphics[height=3.5cm]{logos/ppgca-logo.png}}

% ====================
% Body
% ====================

\begin{document}

% Refer to https://github.com/k4rtik/uchicago-poster
% logo: https://www.cam.ac.uk/brand-resources/about-the-logo/logo-downloads
% \addtobeamertemplate{headline}{}
% {
%     \begin{tikzpicture}[remember picture,overlay]
%       \node [anchor=north west, inner sep=3cm] at ([xshift=-2.5cm,yshift=1.75cm]current page.north west)
%       {\includegraphics[height=7cm]{logos/unott-logo.eps}}; 
%     \end{tikzpicture}
% }

\begin{frame}[t]
\begin{columns}[t]
\separatorcolumn

\begin{column}{\colwidth}

 \begin{block}{Logic}
  \alert{Logic} is the language of mathematics. It uses \alert{propositions} to
  talk about sets.

  Propositions are sentences which can be either true or false. For example
  \begin{itemize}[label=\textbullet,left=24pt]
   \item `\textbf{Cats are black.}' is a proposition;
   \item `\textbf{How are you?}' is \emph{not} a proposition;
   \item `\textbf{We will have colonised Mars by 2500.}' is also a proposition.
  \end{itemize}
 \end{block}
 As the third example suggests, we need not necessarily know whether a
 proposition is true or false -- it remains a proposition anyway.

 \vspace{1em}

 \begin{exampleblock}{Logical Conjunctions}
  Propositions can be joined together using \alert{logical conjunctions}. They
  pretty much correspond to the conjunctions of natural language. Let us
  consider two propositions:
  \begin{align*}
   p &= \text{`It's raining outside.'}\\
   q &= \text{`I'll stay at home.'}
  \end{align*}
  \begin{itemize}[left=40pt]
   \item[($ \wedge $)] Logical \alert{and} forms a proposition that is only
    \alert{true} if both of its constituents are also \alert{true}. In natural
    language, the proposition $p \wedge q$ can be expressed as
    \[
     p \alert{ \wedge } q = \text{`It's raining outside \alert{and} I'll stay at
     home.'}
    \]
   \item[($ \vee $)] Logical \alert{or} forms a proposition that is \alert{true}
    if at least one of its constituents is \alert{true}. In natural language,
    the proposition $p \vee q$ can be expressed as
    \[
     p \alert{ \vee } q = \text{`It's raining outside \alert{or} I'll stay at
     home.'}
    \]
    In mathematical logic, \alert{or} is \textbf{not exclusive}! This means that
    $p \alert{ \vee } q$ is true even if both $p$ and $q$ are true.
   \item[($\neg $)] Logical \alert{not} isn't strictly speaking a conjunction
    but I include it anyway. It reverses the truth value of a proposition. For
    example, the proposition $\alert{\neg }p$ can be read as
    \[
     \alert{\neg }p = \text{`It's \alert{not} raining outside.'}
    \]
    It follows that $\alert{\neg }p$ is \alert{true} exactly when $p$ is
    \alert{false} and vice versa.
   \item[($ \Rightarrow $)] Logical \alert{implication} is a conjunction that
    makes the first proposition into an \emph{assumption} or \emph{premise} and
    the second one into a \emph{conclusion}. The proposition $p \alert{
    \Rightarrow } q$ is read in multiple ways, to list a few: 
    \begin{align*}
     p \alert{ \Rightarrow } q &= \text{`\alert{If} it's raining outside,
    \alert{then} I'll stay at home.'}\\
     p \alert{ \Rightarrow } q &= \text{`It raining outside \alert{implies that}
     I'll stay at home.'}\\
     p \alert{ \Rightarrow } q &= \text{`\alert{Assuming} it's raining outside,
     I'll stay at home.'}\\
    \end{align*}
    The implication is tricky. It's true if both $p$ and $q$ are true and false
    if $p$ is true but $q$ is false. However, it is \alert{always true} if $p$
    is \alert{false}. That is because, in mathematical logic, whatever follows
    from a lie is automatically true.
   \item[($ \Leftrightarrow $)] Logical \alert{equivalence} is true only if both
    propositions have the \alert{same truth value} -- they're both true or both
    false. In natural language, it is typically read like this:
    \[
     p \alert{ \Leftrightarrow }q = \text{`It's raining \alert{if and only if}
     I stay at home.'}
    \]
    Equivalence is basically just a two-way implication. The proposition $p$ is
    both a premise and a conclusion to $q$ and $q$ is both a premise and a
    conclusion to $p$. If it's raining outside, I stay at home and if I stay at
    home, then it's raining outside.
  \end{itemize}
 \end{exampleblock}

 \begin{block}{Truth Tables}
  A conjunction of propositions being true or false based on whether its
  constituent propositions are true or false can be summarized using so-called
  \alert{truth table}. It is basically just a table that lists all the
  possibilities of $p$ and $q$ being true or false and the resulting truth value
  of their conjunctions.

  For the basic logical conjunctions from above, it can look like this (we
  represent \alert{true} by \alert{1} and \alert{false} by \alert{0}):
  \begin{center}
   \begin{tabular}{c | c | c | c | c | c | c | c}
    $p$ & $q$ & $\neg p$ & $\neg q$ & $p \wedge q$ & $p \vee q$ & $p \Rightarrow
    q$ & $p \Leftrightarrow q$\\
    \toprule
    0 & 0 & 1 & 1 & 0 & 0 & 1 & 1\\
    \midrule
    0 & 1 & 1 & 0 & 0 & 1 & 1 & 0\\
    \midrule
    1 & 0 & 0 & 1 & 0 & 1 & 0 & 0\\
    \midrule
    1 & 1 & 0 & 0 & 1 & 1 & 1 & 1
   \end{tabular}
  \end{center}
 \end{block}
\end{column}

\separatorcolumn

\begin{column}{\colwidth}

\begin{exampleblock}{Sets}
 \alert{Sets} are the `stuff' that makes up the world of mathematics. Their
 basic characteristics and properties are described using \alert{logic}.

 Sets cannot be defined inside set theory but we interpret them as \emph{groups
 of things}.

 There's only one foundational \emph{proposition} related to set theory -- the
 proposition `\alert{An object is an element of a set.}' If we label the object
 in question $x$ and the set $A$, this proposition is written as $x \in A$ (the
 symbol $ \in $ is just the letter `e' in `element'). Combining these
 propositions using logical conjunctions allows for various set-theoretic
 constructions.

 If a set $A$ has, for example, exactly three elements -- $\mysquare$, $\mytria$
 and $\mycirc$, I can write it as a list of these three elements inside curly
 brackets $\{\}$. In this case,
 \[
  A = \{\mysquare,\mytria,\mycirc\}.
 \]

 A few \alert{warnings} about sets:
 \begin{itemize}[label=\textbullet,left=24pt]
  \item \textbf{Sets are not ordered}. There is nothing like a `first', `second'
   or `last' element of a set. Either an object \textbf{is} inside a set or it
   \textbf{isn't}. Nothing else. For example, the three sets below are
   \alert{exactly the same}, only written differently.
   \[
    \{\mysquare,\mytria,\mycirc\} = \{\mycirc,\mytria,\mysquare\} =
    \{\mytria,\mysquare,\mycirc\}
   \]
  \item \textbf{Elements of sets have no frequency}. Again, an element either is
   inside a set or not. It cannot be \alert{twice} in a set, for example. The
   three sets below are exactly the same.
   \[
    \{\mysquare,\mytria,\mycirc\} =
    \{\mysquare,\mytria,\mycirc,\mytria,\mycirc\} = \{
    \mytria,\mysquare,\mysquare,\mytria,\mycirc,\mytria\}
   \]
 \end{itemize}
\end{exampleblock}

\begin{alertblock}{Set Operations}
 Using logical conjunctions, we form new sets from existing ones. Consider two
 sets -- $A$ and $B$.
 \begin{itemize}[left=40pt]
  \item[($ \cap $)] I can form the set of all objects $x$ that satisfy the
   proposition $x \in A \wedge x \in B$, that is all objects that \alert{lie in
   both $A$ and $B$}. This set is called the \alert{intersection} of $A$ and $B$
   and written $A \cap B$. For example,
   \[
    \{\mycirc,\mytria,\mysquare\} \cap \{\mycross,\mycirc,\mysquare, \sim \} =
    \{\mycirc,\mysquare\}.
   \]
  \item[($ \cup $)] I can form the set of all objects that satisfy the
   proposition $x \in A \vee x \in B$, the set of all objects that \alert{lie in
   $A$ or in $B$}. It is called the \alert{union} of $A$ and $B$ and denoted
   $A \cup B$. All elements of $A \cup B$ can be found \emph{only} in $A$,
   \emph{only} in B or in \emph{both} $A$ and $B$. For example,
   \[
    \{\mycirc,\mytria,\mysquare\} \cup \{\mycross,\mycirc,\mysquare, \sim \} =
    \{\mycirc,\mytria,\mysquare,\mycross, \sim \}.
   \]
  \item[($ \Rightarrow $)] Implication is a little different from intersection
   and union. It describes a lot of different sets with one logical proposition.
   I ask: `Which sets $A$ satisfy the proposition $x \in A \Rightarrow x \in
   B$?' In other words, which sets $A$ \alert{have all their elements contained}
   in the set $B$? The answer is that $A$ must be a subset of $B$ and we denote
   that fact by $A \subseteq B$. The set $A$ is only allowed to have elements
   which also lie in $B$ but not necessarily all of them. All the subsets of $B
   = \{\mytria,\mycirc\}$ are listed below.
   \[
    \emptyset, \{\mytria\}, \{\mycirc\}, \{\mytria,\mycirc\},
   \]
   where $\emptyset$ is the \alert{empty set}, a set containing no elements.
  \item[($ \Leftrightarrow $)] Equivalence defines \alert{equality} on sets. If
   sets $A$ and $B$ must satisfy the proposition $x \in A \Leftrightarrow x \in
   B$, then they must be equal because all the elements of $A$ lie in $B$ and
   all elements of $B$ lie in $A$. That is, $A = B$.
 \end{itemize}
\end{alertblock}

\begin{block}{Drawing Sets}
 Set operations can be visualized using so-called \emph{Venn diagrams}. This
 just means using circles to represent the sets in questions. For example, two
 sets -- $\clr{A}$ and $\clb{B}$ can be drawn like this:
 \begin{center}
  \begin{tikzpicture}[scale=0.5]
   \fill[BrickRed,fill opacity=0.25] (180:1.5) circle (2);
   \fill[RoyalBlue,fill opacity=0.25] (0:1.5) circle (2);
   \draw[BrickRed,thick] (180:1.5) circle (2);
   \draw[RoyalBlue,thick] (0:1.5) circle (2);
   \node at (150:4.3) {\footnotesize $\clr{A}$};
   \node at (30:4.3) {\footnotesize $\clb{B}$};
  \end{tikzpicture}
 \end{center}
 In these pictures, one can easily visualize the operations of union and
 intersection. The union $\clg{A \cup B}$ is the entire area covered by
 $\clr{A}$ and $\clb{B}$. It looks like this:
 \begin{center}
  \begin{tikzpicture}[scale=0.5]
   \def\firstcircle{(180:1.5) circle (2)}
   \def\secondcircle{(0:1.5) circle (2)}
   \filldraw[thick,ForestGreen] \firstcircle; 
   \filldraw[thick,ForestGreen] \secondcircle;
   \node at (90:3) {\footnotesize $\clg{A \cup B}$};
  \end{tikzpicture}
 \end{center}
 The intersection $\clm{A \cap B}$ is the `strip' in the middle, the area which
 is shared between both $\clr{A}$ and $\clb{B}$. It can be depicted like this:
 \begin{center}
  \begin{tikzpicture}[scale=0.5]
   \def\firstcircle{(180:1.5) circle (2)}
   \def\secondcircle{(0:1.5) circle (2)}
   
   \fill[BrickRed,fill opacity=0.25] \firstcircle;
   \fill[RoyalBlue,fill opacity=0.25] \secondcircle;
   \draw[BrickRed,thick] \firstcircle;
   \draw[RoyalBlue,thick] \secondcircle;

   \begin{scope}
    \clip \firstcircle;
    \fill[white] \secondcircle;
    \filldraw[thick,Fuchsia] \secondcircle;
   \end{scope}
   \begin{scope}
    \clip \secondcircle;
    \draw[Fuchsia,thick] \firstcircle;
   \end{scope}
   \node at (90:3) {\footnotesize $\clm{A \cap B}$};
  \end{tikzpicture}
 \end{center}
\end{block}
\end{column}
\separatorcolumn

\begin{column}{\colwidth}

\begin{alertblock}{Products of Sets \& Relations}
 Before introducing \emph{products} of sets, we must define a \emph{pair}.
 Simply said, a \alert{pair} of objects $(a,b)$ is just a set containing $a$ and
 $b$ \textbf{with ordering}, that is, $a$ is the \textbf{first} element of
 $(a,b)$ and $b$ is \textbf{second}. This means that $(a,b) \neq (b,a)$ because
 the order is not the same.

 Now, the \alert{product} of sets $A$ and $B$, denoted $A \times B$, is the set
 of all pairs $(a,b)$ where $a \in A$ and $b \in B$. For example, if
 \[
  A = \{\mycirc,\mytria\} \quad \text{and} \quad B = \{\mytria,\mycross, \sim
  \},
 \]
 then
 \[
  A \times B = \{(\mycirc,\mytria),(\mycirc,\mycross),(\mycirc, \sim),
  (\mytria,\mytria), (\mytria,\mycross), (\mytria, \sim)\}.
 \]
 Notice that $A \times B \neq B \times A$ because the \textbf{order} of elements
 in a pair \textbf{matters}. In this case,
 \[
  B \times A = \{(\mytria,\mycirc), (\mytria,\mytria), (\mycross,\mycirc),
  (\mycross,\mytria), ( \sim, \mycirc), ( \sim, \mytria)\}.
 \]

 As another example, consider the \emph{real plane} -- the set of all points
 with two coordinates. That is simply the set of pairs of real numbers, $\R
 \times \R$.

 The mathematical way to define a \alert{relation} between two sets is to simply
 \textbf{list all the elements that are related}. Said formally, a relation $R$
 is a subset $R \subseteq A \times B$. For example,
 \[
  \{(\mytria,\mycross),(\mytria, \sim )\} \subseteq \{\mycirc,\mytria\} \times
  \{\mytria,\mycross, \sim\}
 \]
 is a relation between the sets $A$ and $B$ from above. It literally says that
 $\mytria$ is related to $\mycross$ and $ \sim $.
\end{alertblock}

\end{column}
\separatorcolumn

\end{columns}
\end{frame}

\end{document}
