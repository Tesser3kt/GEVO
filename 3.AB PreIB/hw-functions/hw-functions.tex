\documentclass[a4paper,11pt]{article}

\usepackage[czech,english]{babel}
% Fonts %
\usepackage{fouriernc}
\usepackage[T1]{fontenc}

% Colors %
\usepackage[dvipsnames]{color}
\usepackage[dvipsnames]{xcolor}

% Page Layout %
\usepackage[margin=1.5in]{geometry}

% Fancy Headers %
\usepackage{fancyhdr}
\fancyhf{}
\cfoot{\thepage}
\rhead{}
\renewcommand{\headrulewidth}{0pt}
\setlength{\headheight}{16pt}

% Math
\usepackage{mathtools}
\usepackage{amssymb}
\usepackage{faktor}
\usepackage{import}
\usepackage{caption}
\usepackage{subcaption}
\usepackage{wrapfig}
\usepackage{enumitem}
\setlist{topsep=0pt}

\usepackage{tikz}
\usetikzlibrary{cd,positioning,babel,shapes,decorations.text,
 decorations.pathmorphing}
\usepackage{tkz-base}
\usepackage{tkz-euclide}

% Theorems
\usepackage[thmmarks, amsmath, thref]{ntheorem}
\usepackage{thmtools}

\theoremsymbol{\ensuremath{\blacksquare}}
\newtheorem*{solution}{Possible solution.}

% Title %
\title{\Huge\textsf{Math Homework -- PreIB 3.AB 3 \& 4}
 \Large\textsf{Functions \& Linear Equations}
 \author{Áďa Klepáčů}
 \date{\today}
}

% Table of Contents %
\usepackage{hyperref}
\hypersetup{
 colorlinks=true,
 linktoc=all,
 linkcolor=blue
}

% Tables %
\usepackage{booktabs}
\usepackage{tabularx}

% Patch for hyphens
\usepackage{regexpatch}
\makeatletter
% Change the `-` delimiter to an active character
\xpatchparametertext\@@@cmidrule{-}{\cA-}{}{}
\xpatchparametertext\@cline{-}{\cA-}{}{}
\makeatother

\newcolumntype{s}{>{\centering\arraybackslash}p{.4\textwidth}}

% Operators %
\DeclareMathOperator{\Ker}{Ker}
\DeclareMathOperator{\Img}{Im}
\DeclareMathOperator{\End}{End}
\DeclareMathOperator{\Aut}{Aut}
\DeclareMathOperator{\Inn}{Inn}

% Common operators %
\newcommand{\R}{\mathbb{R}}
\newcommand{\N}{\mathbb{N}}
\newcommand{\Z}{\mathbb{Z}}
\newcommand{\Q}{\mathbb{Q}}
\newcommand{\C}{\mathbb{C}}

\newcommand{\clr}{\textcolor{BrickRed}}
\newcommand{\clb}{\textcolor{RoyalBlue}}
\newcommand{\clg}{\textcolor{ForestGreen}}
\newcommand{\clm}{\textcolor{Fuchsia}}
\newcommand{\clv}{\textcolor{violet}}
\newcommand{\clbr}{\textcolor{Sepia}}
\newcommand{\cly}{\textcolor{Dandelion}}

% American Paragraph Skip %
\setlength{\parindent}{0pt}
\setlength{\parskip}{1em}

% Document %
\pagestyle{fancy}
\begin{document}

\maketitle
\thispagestyle{fancy}

\begin{center}
 \hrule
 \textbf{\clr{DON'T FORGET TO EXPLAIN STUFF AND INCLUDE COMPUTATIONS WHERE
 APPROPRIATE!}}
 \vspace{2ex}
 \hrule
\end{center}

\section*{Functions \& Function Composition}

You're given two functions -- a \clr{function} which receives a word and outputs
the number of \emph{vowels} and \emph{consonants} (as a list of two numbers) in
that word and another \clb{function} which receives a list of however many
numbers and simply computes their sum. Meaning, if it receives a list
$a,b,c,d,e$, it outputs $a + b + c + d + e$.
\begin{center}
 \begin{tikzpicture}
  \node[rectangle,draw,inner sep=10pt] (f1) at (0,0) {\clr{function 1}};
  \node[left=1cm of f1] (i1) {word};
  \node[right=1cm of f1] (o1) {vowels,consonants};

  \draw[-latex,thick] (i1) -- (f1);
  \draw[-latex,thick] (f1) -- (o1);

  \node[rectangle,draw,inner sep=10pt,below=.5cm of f1] (f2) {\clb{function 2}};
  \node[left=1cm of f2] (i2) {list of numbers};
  \node[right=1cm of f2] (o2) {sum of those numbers};

  \draw[-latex,thick] (i2) -- (f2);
  \draw[-latex,thick] (f2) -- (o2);
 \end{tikzpicture}
\end{center}
\begin{enumerate}
 \item (10 \%) In easy terms (you need like 4-5 words) describe the output of
  the composition
 \begin{center}
  \begin{tikzpicture}
   \node[rectangle,draw,inner sep=10pt] (f1) at (0,0) {\clr{function 1}};
   \node[rectangle,draw,inner sep=10pt,right=1cm of f1] (f2) {\clb{function 2}};
   
   \node[left=1cm of f1] (i1) {word};
   \node[right=1cm of f2] (o1) {?.};

   \draw[-latex,thick] (i1) -- (f1);
   \draw[-latex,thick] (f1) -- (f2);
   \draw[-latex,thick] (f2) -- (o1);
  \end{tikzpicture}
 \end{center}
\item (15 \%) Find and describe (using a diagram for example) \clg{third
 function} such that the composition 
 \begin{center}
  \begin{tikzpicture}
   \node[rectangle,draw,inner sep=10pt] (f1) at (0,0) {\clr{function 1}};
   \node[rectangle,draw,inner sep=10pt,right=1cm of f1] (f3) {\clg{function 3}};
   \node[rectangle,draw,inner sep=10pt,right=1cm of f3] (f2) {\clb{function 2}};
   
   \node[left=1cm of f1] (i1) {word};
   \node[right=1cm of f2] (o1) {?};

   \draw[-latex,thick] (i1) -- (f1);
   \draw[-latex,thick] (f1) -- (f3);
   \draw[-latex,thick] (f3) -- (f2);
   \draw[-latex,thick] (f2) -- (o1);
  \end{tikzpicture}
 \end{center}
 \textbf{outputs the number 0} whenever the given \emph{word} has the same
 number of \emph{vowels} and \emph{consonants}.
\item (15 \%) Given real functions $f(x) = (x-1)(x-2)$ and $g(x) = x + 3$,
 compute $f+g$, $f \cdot g$, $f \circ g$ and $g \circ f$.
\item (10 \%) Is it true that $(f \circ g)(0) = (g \circ f)(0)$?
\end{enumerate}

\section*{Linear Equations}

Consider the system
\[
 \begin{array}{ccccc}
  3x & + & y & = & 2,\\
  -x & + & 2y & = & -3.
 \end{array}
\]
\begin{enumerate}
 \item (15 \%) Interpret both equations as linear functions in your chosen
  variable and draw their graphs.
 \item (10 \%) Compute (\textbf{both coordinates of}) the intersection of the
  graphs from point 1.
 \item (15 \%) Find another linear function $h$ whose graph intersects the
  graphs of $f$ and $g$ at the point calculated in 2. Draw it.
 \item (10 \%) Using only your \emph{reasoning} (that is, no computation) deduce
  whether the system
  \begin{align*}
   y &= f(x),\\
   y &= h(x)
  \end{align*}
  has the same solution as the original system. Explain.
\end{enumerate}

\end{document}
