\documentclass[a4paper,11pt]{article}

\usepackage[czech,english]{babel}
% Fonts %
\usepackage{fouriernc}
\usepackage[T1]{fontenc}

% Colors %
\usepackage[dvipsnames]{color}
\usepackage{xcolor}

% Page Layout %
\usepackage[margin=1.5in]{geometry}

% Fancy Headers %
\usepackage{fancyhdr}
\fancyhf{}
\cfoot{\thepage}
\rhead{}
\renewcommand{\headrulewidth}{0pt}
\setlength{\headheight}{16pt}

% Math
\usepackage{mathtools}
\usepackage{amssymb}
\usepackage{faktor}
\usepackage{import}
\usepackage{caption}
\usepackage{subcaption}
\usepackage{wrapfig}
\usepackage{enumitem}

% Theorems
\usepackage{amsthm}
\usepackage{thmtools}

% Title %
\title{\Huge\textsf{Addition \& Multiplication}\\
 \Large\textsf{Some `fun' with basic operations}
 \author{Áďa Klepáčů}
 \date{\today}
}

% Table of Contents %
\usepackage{hyperref}
\hypersetup{
 colorlinks=true,
 linktoc=all,
 linkcolor=blue
}

% Tables %
\usepackage{booktabs}
\usepackage{tabularx}

% Patch for hyphens
\usepackage{regexpatch}
\makeatletter
% Change the `-` delimiter to an active character
\xpatchparametertext\@@@cmidrule{-}{\cA-}{}{}
\xpatchparametertext\@cline{-}{\cA-}{}{}
\makeatother

\newcolumntype{s}{>{\centering\arraybackslash}p{.4\textwidth}}

% Operators %
\DeclareMathOperator{\Ker}{Ker}
\DeclareMathOperator{\Img}{Im}
\DeclareMathOperator{\End}{End}
\DeclareMathOperator{\Aut}{Aut}
\DeclareMathOperator{\Inn}{Inn}

% Common operators %
\newcommand{\R}{\mathbb{R}}
\newcommand{\N}{\mathbb{N}}
\newcommand{\Z}{\mathbb{Z}}
\newcommand{\Q}{\mathbb{Q}}
\newcommand{\C}{\mathbb{C}}

\newcommand{\tr}{\textcolor{red}}
\newcommand{\tb}{\textcolor{blue}}
\newcommand{\tg}{\textcolor{green}}
\newcommand{\tm}{\textcolor{magenta}}
\newcommand{\tv}{\textcolor{violet}}

% American Paragraph Skip %
\setlength{\parindent}{0pt}
\setlength{\parskip}{1em}

% Document %
\pagestyle{fancy}
\begin{document}

\maketitle
\thispagestyle{fancy}

Addition (I denote by $\tr{+}$) and multiplication (I denote by $\tv{\cdot}$ or
by nothing) are operations on, let's say real numbers, satisfying the following
properties.

\begin{center}
 \begin{tabular}{s|s}
  \textbf{\tr{Addition}} & \textbf{\tv{Multiplication}} \\
  \toprule
  Commutativity (\tr{$C+$}) & Commutativity (\tv{$C \cdot $})\\
  $a \tr{+} b = b \tr{+} a$ & $a \tv{ \cdot } b = b \tv{ \cdot } a$\\
  \cmidrule(lr){1-2}
  Associativity ($\tr{A+}$) & Associativity (\tv{$A \cdot $})\\
  $a \tr{+} (b \tr{+} c) = (a \tr{+} b) \tr{+} c$ & $a \tv{ \cdot }  (b \tv{
   \cdot } c) = (a \tv{ \cdot } b) \tv{ \cdot } c$\\
  \cmidrule(lr){1-2}
  \multicolumn{2}{c}{Distributivity ($D\tr{+}\tv{ \cdot })$} \\
  \multicolumn{2}{c}{$a \tv{ \cdot } (b \tr{+} c) = (a \tv{ \cdot } b) \tr{+} (a \tv{ \cdot } c)$}
 \end{tabular}
\end{center}

We also automatically assume that \tv{multiplication} has priority over
\tr{addition} and that there exist numbers $0$ and $1$ such that $x \tr{+} 0 =
x$ and $x \tv{ \cdot } 1 = x$ for every number $x$.

We have seen that it's not too easy to deduce the basic bracket expansion
rule, that is, for instance that
\[
 (2 \cdot x + 3) \cdot (3 \cdot x + 4) = 6 \cdot x^2 + 17 \cdot x + 12
\]
from only these properties. As an exercise, I'd like you to think about various
other things concerning addition and multiplication which might seem obvious
but aren't necessarily so.

\textbf{Exercises} (the ones labeled by an \textbf{*} are probably hard):
\begin{enumerate}[label=\arabic*.,topsep=0pt]
 \item \textbf{*} Invent an operation $\tb{\triangle}$ on real numbers which is
  commutative but \textbf{not} associative. This means that
  \[
   a \tb{\triangle} b = b \tb{\triangle} a  \quad \text{but} \quad a
   \tb{\triangle} (b \tb{\triangle} c) \neq (a \tb{\triangle} b) \tb{\triangle}
   c.
  \]
  You'll have to think quite a bit about this one.\\ \textbf{Hint}: linear
  functions in two variables, something like $x\tb{\triangle}y = f(x,y) = 2
  \cdot x + 3 \cdot y$, are good candidates for such an operation. Play with
  them.

 \item Invent an operation \tg{$\blacksquare$} on real numbers which is
  associative but \textbf{not} commutative. This means that
  \[
   a \tg{\blacksquare} (b \tg{\blacksquare} c) = (a \tg{\blacksquare} b)
   \tg{\blacksquare} c \quad \text{but} \quad a \tg{\blacksquare} b \neq b
   \tg{\blacksquare} a.
  \]
  This one is actually much easier than 1.

 \item \textbf{*} Invent two operations $\spadesuit$ and $\clubsuit$ that are
  both commutative \textbf{and} associative but they are \textbf{not}
  distributive in any direction. This means that
  \begin{align*}
   a \spadesuit b = b \spadesuit a  \quad &\text{and} \quad a \spadesuit (b
   \spadesuit c) = (a \spadesuit b) \spadesuit c, \\ 
   a \clubsuit b = b \clubsuit a \quad &\text{and} \quad a \clubsuit (b
   \clubsuit c) = (a \clubsuit b) \clubsuit c
  \end{align*}
  but
  \[
   a \spadesuit (b \clubsuit c) \neq (a \spadesuit b) \clubsuit (a \spadesuit
   c) \quad \text{and} \quad  a \clubsuit (b \spadesuit c) \neq (a \clubsuit b)
   \spadesuit (a \clubsuit c).
  \]
 \item Explain how to get the equality
  \[
   (1 + x \cdot 2) + (y \cdot (3 \cdot 5)) = (15 \cdot y) + (2 \cdot x) + 1
  \]
  using only the rules of addition and multiplication in the table.
 \item Where would the parentheses be in the expression
  \[
   (1 \cdot 2 + 3) \cdot (4 + 5 \cdot 6)
  \]
  if \textbf{addition had priority over multiplication}. I mean, write the
  expression which has the same numerical value as this one assuming that we
  first add and then we multiply.
\end{enumerate}

\end{document}
