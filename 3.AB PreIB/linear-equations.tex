\documentclass[a4paper,11pt]{article}

\usepackage[czech,english]{babel}
% Fonts %
\usepackage{fouriernc}
\usepackage[T1]{fontenc}

% Colors %
\usepackage[dvipsnames]{color}
\usepackage{xcolor}

% Page Layout %
\usepackage[margin=1.5in]{geometry}

% Fancy Headers %
\usepackage{fancyhdr}
\fancyhf{}
\cfoot{\thepage}
\rhead{}
\renewcommand{\headrulewidth}{0pt}
\setlength{\headheight}{16pt}

% Math
\usepackage{mathtools}
\usepackage{amssymb}
\usepackage{faktor}
\usepackage{import}
\usepackage{caption}
\usepackage{subcaption}
\usepackage{wrapfig}
\usepackage{enumitem}
\usepackage{tikz}
\usetikzlibrary{cd,positioning,babel,shapes}
\usepackage{tkz-base}
\usepackage{tkz-euclide}

% Theorems
\usepackage{amsthm}
\usepackage{thmtools}

% Title %
\title{\Huge\textsf{Linear Equations}\\
 \Large\textsf{Scales, Lines \& Functions}
 \author{Áďa Klepáčů}
 \date{\today}
}

% Table of Contents %
\usepackage{hyperref}
\hypersetup{
 colorlinks=true,
 linktoc=all,
 linkcolor=blue
}

% Tables %
\usepackage{booktabs}
\usepackage{tabularx}

% Patch for hyphens
\usepackage{regexpatch}
\makeatletter
% Change the `-` delimiter to an active character
\xpatchparametertext\@@@cmidrule{-}{\cA-}{}{}
\xpatchparametertext\@cline{-}{\cA-}{}{}
\makeatother

\newcolumntype{s}{>{\centering\arraybackslash}p{.4\textwidth}}

% Operators %
\DeclareMathOperator{\Ker}{Ker}
\DeclareMathOperator{\Img}{Im}
\DeclareMathOperator{\End}{End}
\DeclareMathOperator{\Aut}{Aut}
\DeclareMathOperator{\Inn}{Inn}

% Common operators %
\newcommand{\R}{\mathbb{R}}
\newcommand{\N}{\mathbb{N}}
\newcommand{\Z}{\mathbb{Z}}
\newcommand{\Q}{\mathbb{Q}}
\newcommand{\C}{\mathbb{C}}

\newcommand{\tr}{\textcolor{red}}
\newcommand{\tb}{\textcolor{blue}}
\newcommand{\tg}{\textcolor{green}}
\newcommand{\tm}{\textcolor{magenta}}
\newcommand{\tv}{\textcolor{violet}}

% American Paragraph Skip %
\setlength{\parindent}{0pt}
\setlength{\parskip}{1em}

% Document %
\pagestyle{fancy}
\begin{document}

\maketitle
\thispagestyle{fancy}

We have seen three different ways of interpreting linear equations -- as
comparisons of functions, as crossings of lines and as pairs of scales. We'll
review them here and then you'll have to do a few exercises.

\section*{Linear equations as functions}

In this interpretation, it is useful to think of numbers as representing
\emph{prices} of the things, for instance in dollars. A linear functions is some
`box' which eats a thing of some price and gives back a multiple (possibly
negative) of that thing and an amount (possibly negative) of dollars on top.

One can draw such a function for instance like this:
\begin{center}
 \begin{tikzpicture}
  \node[draw,red,isosceles triangle,isosceles triangle apex
  angle=60,anchor=center,rotate=90,xshift=-1mm] (triangle) at (-3,0) {};
  \node[draw,rectangle,minimum width=2cm,minimum height=1cm] (rectangle) at
  (0,0) {BOX};
  \node[draw,red,isosceles triangle,isosceles triangle apex
  angle=60,anchor=center,rotate=90,xshift=-1mm] (triangle2) at (3,0.25) {};
  \node[draw,red,isosceles triangle,isosceles triangle apex
  angle=60,anchor=center,rotate=90,xshift=-1mm] (triangle3) at (3,-0.25) {};
  \node (plus) at (3.6, 0) {$+$};
  \node (dollars) at (4.1, 0) {$\$4$};

  \draw[->,shorten <=10pt,shorten >=5pt] (-3,0) -- (rectangle);
  \draw[->,shorten <=5pt,shorten >=10pt] (rectangle) -- (3,0);
 \end{tikzpicture}
\end{center}

This function receives a $\tr{\triangle}$ of some price and gives back two such
$\tr{\triangle}$'s and $\$ 4$. A \emph{linear equation} is a comparison of the
total price of things two (typically different) functions give back when
receiving \textbf{the same thing}. The \emph{solution} is the \textbf{price of
the thing} for which the total prices of what is given back by these two
functions are equal. One can visualize it like this:
\begin{center}
 \begin{tikzpicture}
  \node[draw,red,isosceles triangle,isosceles triangle apex
  angle=60,anchor=center,rotate=90,xshift=-1mm] (triangle1) at (-6,0) {};
  \node[draw,rectangle,minimum width=2cm,minimum height=1cm] (box1) at
  (-3,0) {BOX \#1};
  \node[draw,circle,align=center] (total price) at (0, 0) {TOTAL\\PRICE};
  \node[draw,rectangle,minimum width=2cm,minimum height=1cm] (box2) at
  (3,0) {BOX \#2};
  \node[draw,red,isosceles triangle,isosceles triangle apex
  angle=60,anchor=center,rotate=90,xshift=-1mm] (triangle2) at (6,0) {};

  \draw[->,shorten <=10pt,shorten >=5pt] (-6,0) -- (box1);
  \draw[->,shorten <=5pt,shorten >=5pt] (box1) -- (total price);

  \draw[->,shorten <=5pt,shorten >=5pt] (box2) -- (total price);
  \draw[->,shorten <=10pt,shorten >=5pt] (6, 0) -- (box2);
 \end{tikzpicture}
\end{center}
Of course, one typically calls BOX \#1 and BOX \#2 by letters, such as $f$ and
$g$ and treats them as linear functions in one variable. Then, the picture above
is simply written as $f(\tr{\triangle}) = g(\tr{\triangle})$.

\subsection*{Exercises.}
\begin{enumerate}[topsep=0pt,label=\arabic*.]
 \item Sir Minkowski and Sir Riemann are pickpockets. Sir Minkowski boasts that
  he robs on average 5 people a day of their wallets and finds on average \$50
  by picking coins from the ground. Sir Riemann takes a nobler approach. He only
  ever robs or cajoles people and deigns not pick money from the ground. On
  average, he robs 7 people a day of their wallets. How much money must an
  average person's wallet contain so that Sir Riemann and Sir Minkowski have
  gained the same amount of money by the end of each day?

  Express the total value both Sir Riemann and Sir Minkowski acquire each day as
  linear functions and compare their outputs. What does the \textbf{variable} of
  these functions represent? What does the \textbf{value/output} of these
  functions represent? Are they the same?

 \item The cryptocurrencies Dogecoin and Freedomcoin are steadily losing their
  value (as expressed in US Dollars). At the time of writing, Dogecoin stands at
  \$0.094 apiece and Freedomcoin stands at \$0.0188 apiece. For every lost
  investor, the value of Dogecoin drops by 0.1 \% of its current value. Assuming
  that Freedomcoin loses no investors, how many investors does Dogecoin have to
  lose so that its value falls to the value of Freedomcoin?

  Express the drop in value of Dogecoin based on the number of lost investors as
  a linear function. Compare it to the value of Freedomcoin. As the numbers are
  ugly, \textbf{you simply have to construct the equation, not solve it}.
\end{enumerate}

\section*{Linear equations as scales}

This interpretation of linear equations is based on the comparison of
\emph{weights} of objects. Unlike in physics, we have no qualms about giving our
objects negative weights. The bowls of the given pair of scales contain one type
of object (whose weight we typically don't know) and then some absolute weights.
The word \emph{equation} is here expressed in the fact that the scales are
always in balance. The \emph{solution} to such an equation is a situation where
in either bowl, there is only one object of unknown value and in the other are
only absolute weights.

For instance, one possible solution to the equation
\begin{center}
 \begin{tikzpicture}
  \draw (-5,2) -- (-2,2);
  \draw (2,2) -- (5,2);
  \draw (-3.5,2) -- (-3.5,1);
  \draw (3.5,2) -- (3.5,1);
  \draw (-3.5,1) -- (3.5,1);
  \node[draw,isosceles triangle,isosceles triangle apex
  angle=60,anchor=center,rotate=90,inner xsep=6mm,xshift=2mm] (triangle) at
  (0,0) {};

  \node[draw,blue,fill,rectangle,minimum width=5mm, minimum height=5mm] at
  (-4.6, 2.3) {};
  \node[draw,blue,fill,rectangle,minimum width=5mm, minimum height=5mm] at (-4,
  2.3) {};
  \node[draw,blue,fill,rectangle,minimum width=5mm, minimum height=5mm] at
  (-3.4, 2.3) {};
  \node[draw,trapezium,inner xsep=0pt] at (-2.6, 2.3) {$-3$};
  \node[draw,blue,fill,rectangle,minimum width=5mm, minimum height=5mm] at (2.7, 2.3) {};
  \node[draw,blue,fill,rectangle,minimum width=5mm, minimum height=5mm] at (3.3, 2.3) {};
  \node[draw,blue,fill,rectangle,minimum width=5mm, minimum height=5mm] at (3.3,
  2.9) {};
  \node[draw,blue,fill,rectangle,minimum width=5mm, minimum height=5mm] at (2.7,
  2.9) {};
  \node[draw,trapezium,inner xsep=0pt] at (4.1, 2.3) {$-5$};
 \end{tikzpicture}
\end{center}
is the following state of the scales:
\begin{center}
 \begin{tikzpicture}
  \draw (-5,2) -- (-2,2);
  \draw (2,2) -- (5,2);
  \draw (-3.5,2) -- (-3.5,1);
  \draw (3.5,2) -- (3.5,1);
  \draw (-3.5,1) -- (3.5,1);
  \node[draw,isosceles triangle,isosceles triangle apex
  angle=60,anchor=center,rotate=90,inner xsep=6mm,xshift=2mm] (triangle) at
  (0,0) {};

  \node[draw,trapezium,inner xsep=2pt] at (-3.5, 2.3) {$2$};
  \node[draw,blue,fill,rectangle,minimum width=5mm, minimum height=5mm] at (3.5, 2.3) {};
 \end{tikzpicture}
\end{center}

\subsection*{Exercises.}
\begin{enumerate}[topsep=0pt,label=\arabic*.]
 \item Pharaoh Amenhotep had been a notoriously known swindler. He would fill
  hollow objects with dirt to increase their weight and thus sell them for more
  gold. Knowing that the following scales are in balance
 \begin{center}
  \begin{tikzpicture}
   \draw (-5,2) -- (-2,2);
   \draw (2,2) -- (5,2);
   \draw (-3.5,2) -- (-3.5,1);
   \draw (3.5,2) -- (3.5,1);
   \draw (-3.5,1) -- (3.5,1);
   \node[draw,isosceles triangle,isosceles triangle apex
   angle=60,anchor=center,rotate=90,inner xsep=6mm,xshift=2mm] (triangle) at
   (0,0) {};

   \node[draw,blue,fill,rectangle,minimum width=5mm, minimum height=5mm] at
   (-4.4, 2.3) {};
   \node[draw,blue,fill,rectangle,minimum width=5mm, minimum height=5mm] at
   (-3.8,
   2.3) {};
   \node[draw,blue,rectangle,minimum width=5mm, minimum height=5mm] at
   (-3.2, 2.3) {};
   \node[draw,blue,rectangle,minimum width=5mm, minimum height=5mm] at
   (-2.6, 2.3) {};
   2.9) {};
   \node[draw,trapezium,inner xsep=0pt] at (3.5, 2.3) {$12$};
  \end{tikzpicture}
 \end{center}
  where $\tb{\square}$ are the hollow objects and $\tb{\blacksquare}$ are those
  filled with dirt, determine the weight of $\tb{\square}$ if one
  $\tb{\blacksquare}$ is 2 kg heavier than one $\tb{\square}$.
 \item The following scales are in balance:
 \begin{center}
  \begin{tikzpicture}
   \draw (-5,2) -- (-2,2);
   \draw (2,2) -- (5,2);
   \draw (-3.5,2) -- (-3.5,1);
   \draw (3.5,2) -- (3.5,1);
   \draw (-3.5,1) -- (3.5,1);
   \node[draw,isosceles triangle,isosceles triangle apex
   angle=60,anchor=center,rotate=90,inner xsep=6mm,xshift=2mm] (triangle) at
   (0,0) {};

   \node[draw,blue,fill,rectangle,minimum width=5mm, minimum height=5mm] at
   (-4.1, 2.3) {};
   \node[draw,red,fill,circle,minimum width=5mm, minimum height=5mm] at
   (-3.5,
   2.3) {};
   \node[draw,blue,fill,rectangle,minimum width=5mm, minimum height=5mm] at
   (-2.9, 2.3) {};
   \node[draw,green,fill,regular polygon,regular polygon sides=3,outer sep=0pt,
   inner sep=1.1mm,yshift=0.2mm] at (-2.9, 2.8) {};
   \node[draw,blue,fill,rectangle,minimum width=5mm, minimum height=5mm] at
   (-3.5, 2.9) {};
   \node[draw,blue,fill,rectangle,minimum width=5mm, minimum height=5mm] at
   (-2.9, 3.5) {};
   \node[draw,green,fill,regular polygon,regular polygon sides=3,outer sep=0pt,
   inner sep=1.1mm,yshift=0.2mm] at (-3.5, 3.4) {};

   \node[draw,trapezium,inner xsep=2pt] at (3, 2.3) {$5$};
   \node[draw,trapezium,inner xsep=0pt] at (4, 2.3) {$10$};
  \end{tikzpicture}
 \end{center}
 Find a way to remove all the absolute weights \textbf{one by one} from the
 right bowl without ever breaking the balance assuming that the scales
 \begin{center}
  \begin{tikzpicture}
   \draw (-5,2) -- (-2,2);
   \draw (2,2) -- (5,2);
   \draw (-3.5,2) -- (-3.5,1);
   \draw (3.5,2) -- (3.5,1);
   \draw (-3.5,1) -- (3.5,1);
   \node[draw,isosceles triangle,isosceles triangle apex
   angle=60,anchor=center,rotate=90,inner xsep=6mm,xshift=2mm] (triangle) at
   (0,0) {};

   \node[draw,blue,fill,rectangle,minimum width=5mm, minimum height=5mm] at
   (3.2, 2.3) {};
   \node[draw,red,fill,circle,minimum width=5mm, minimum height=5mm] at
   (3.8, 2.3) {};
   \node[draw,green,fill,regular polygon,regular polygon sides=3,outer sep=0pt,
   inner sep=1.1mm,yshift=0.2mm] at (-3.5, 2.2) {};
  \end{tikzpicture}
 \end{center}
 are also in balance. \textbf{You are not allowed to add objects to the scales
 or to split weights. You can only remove objects as they are and you must
 maintain the balance of the scales in the process}.
\end{enumerate}

\section*{Linear equations as lines}

Given some linear equation, for example
\[
 \tr{3x - 1} = \tb{2x + 1},
\]
one can look at the left side and the right side separately as linear functions
in one variable. We know that we can visualize linear functions in one variable
as lines. Hence, the \emph{equation} in this case is a `drawing' of two lines in
the plane. We let $\tr{f(x)} = \tr{3x - 1}$ and $\tb{g(x)} = \tb{2x + 1}$ and
draw the values of $x$ on the horizontal axis and the values of both $\tr{f}$
and $\tb{g}$ on the vertical. For each, we calculate two points (that is all we
need to draw a line). We can put them into a table:

\begin{center}
 \begin{tabular}{c|cc}
  $x$ & $0$ & $1$\\
  \cmidrule{1-3}
  $\tr{f(x)}$ & $\tr{-1}$ & $\tr{2}$\\
  \cmidrule{1-3}
  $\tb{g(x)}$ & $\tb{1}$ & $\tb{3}$
 \end{tabular}
\end{center}

Using these points to draw our two lines gives the following picture.

\begin{center}
 \begin{tikzpicture}[scale=0.75]
  \tkzInit[xmax=6,ymax=6,xmin=-6,ymin=-6]
  \tkzGrid
  \tkzLabelX[font=\scriptsize, orig=false]
  \tkzLabelY[font=\scriptsize, orig=false]
  \tkzDrawX
  \tkzDrawY[label={$\tr{f(x)},\tb{g(x)}$}]

  \tkzDefPoint(0, -1){f1}
  \tkzDefPoint(1, 2){f2}
  \tkzDefPoint(1, 3){g1}
  \tkzDefPoint(0, 1){g2}

  \tkzDrawPoints[color=red,size=4pt](f1, f2)
  \tkzDrawPoints[color=blue,size=4pt](g1, g2)

  \tkzDrawLine[color=red,line width=1pt, add=1.5 and 1.3](f1,f2)
  \tkzDrawLine[color=blue,line width=1pt, add=1.5 and 1.5](g1,g2)

  \tkzInterLL(f1,f2)(g1,g2)
   \tkzGetPoint{I}

  \tkzShowPointCoord(I)
  \tkzDrawPoint[color=black,fill=white,size=6pt](I)
  \tkzLabelPoint[below right,font=\small](I){$(2,5)$}
 \end{tikzpicture}
\end{center}
The \emph{solution} to this equation is the point of intersection of the line
representing $\tr{f}$ with the line representing $\tb{g}$.

\subsection*{Exercises.}

\begin{enumerate}[topsep=0pt,label=\arabic*.]
 \item Two competing brothers, Lidl and Penny, decided to go on a vacation
  together. However, one cannot stand the idea of arriving to their chosen
  destination later than the other. They decided they're both going to travel on
  motorcycles this time around. Lidl is the poorer of the two and his motorcycle
  only reaches the velocity of 80 km/h on average. But, Lidl doesn't mind
  getting up early and is thus willing to set off already at 5 AM. Penny, on the
  other hand, owns a motorcycle that tops 120 km/h but is too lazy to get up so
  early in the morning. Assuming that their destination is 720 km away, when
  does Penny have to get up to arrive there \textbf{at the same time as} Lidl
  and prevent a quarrel?

  Represent time traveled on the horizontal axis and distance traveled on the
  vertical axis. Define linear functions $l(x)$ and $p(x)$ representing the
  distance traveled by Lidl and Penny based on $x$ (which is the time traveled).
  Make sure to choose $l(x)$ so that the point of intersection of these two
  lines has $y$-coordinate equal to $720$, which precisely means that they both
  traveled 720 km before meeting each other.

 \item You're given three linear functions,
  \begin{align*}
   \tr{f(x)} &= \tr{x + 1},\\
   \tb{g(x)} &= \tb{-2x + 4},\\
   \tm{h(x)} &= \tm{-\frac{1}{2}x - \frac{1}{2}}.
  \end{align*}
 Represented as lines, they look like this:
 \begin{center}
  \begin{tikzpicture}[scale=0.75]
   \tkzInit[xmax=4,ymax=4,xmin=-4,ymin=-4]
   \tkzGrid
   \tkzLabelX[font=\scriptsize, orig=false]
   \tkzLabelY[font=\scriptsize, orig=false]
   \tkzDrawX
   \tkzDrawY[label={$\tr{f(x)},\tb{g(x)},\tm{h(x)}$}]

   \tkzDefPoint(0, 1){f1}
   \tkzDefPoint(-1, 0){f2}
   \tkzDefPoint(1, 2){g1}
   \tkzDefPoint(2, 0){g2}
   \tkzDefPoint(1, -1){h1}
   \tkzDefPoint(-1, 0){h2}

   \tkzInterLL(f1,f2)(g1,g2)
    \tkzGetPoint{fg}
   \tkzInterLL(f1,f2)(h1,h2)
    \tkzGetPoint{fh}
   \tkzInterLL(g1,g2)(h1,h2)
    \tkzGetPoint{gh}
   \tkzFillPolygon[color=yellow!20,fill opacity=.8](fg, fh, gh)


   \tkzDrawPoints[color=red,size=4pt](f1, f2)
   \tkzDrawPoints[color=blue,size=4pt](g1, g2)
   \tkzDrawPoints[color=magenta,size=4pt](h1, h2)

   \tkzDrawLine[color=red,line width=1pt, add=3 and 3](f1,f2)
   \tkzDrawLine[color=blue,line width=1pt, add=1 and 2](g1,g2)
   \tkzDrawLine[color=magenta,line width=1pt, add=1.5 and 1.5](h1,h2)

   \tkzShowPointCoord(fg)
   \tkzShowPointCoord(fh)
   \tkzShowPointCoord(gh)
   \tkzDrawPoints[color=black,fill=white,size=6pt](fg,fh,gh)

  \end{tikzpicture}
 \end{center}
 As you can see, the linear equation
 \[
  \tr{f(x)} = \tb{g(x)} = \tm{h(x)}
 \]
 has no solution because the three lines don't intersect at a single point. Move
 any one of the lines (\textbf{without changing its slope/steepness}) so that
 all three \emph{do} intersect at a single point. Change also the definition of
 the corresponding linear function. That is, if you choose to move for example
 the line representing $\tm{h}$, then the definition of the function $\tm{h}$ as
 given above must change as well.

 Look at the triangle (filled with yellow color) which is determined by the
 three intersection points. Observe that if the equation $\tr{f(x)} = \tb{g(x)}
 = \tm{h(x)}$ \emph{has no solution}, then the three lines always determine a
 triangle. Deduce the relation between the number of solutions of $\tr{f(x)} =
 \tb{g(x)} = \tm{h(x)}$ (that is, one or no solution) and the area of this
 triangle. In other words, what is the area of this triangle if $\tr{f(x)} =
 \tb{g(x)} = \tm{h(x)}$ \textbf{has} a solution?
\end{enumerate}

\end{document}
