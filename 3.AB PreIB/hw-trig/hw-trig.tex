\documentclass[a4paper,11pt]{article}

\usepackage[czech,english]{babel}
% Fonts %
\usepackage{fouriernc}
\usepackage[T1]{fontenc}

% Colors %
\usepackage[dvipsnames]{color}
\usepackage[dvipsnames]{xcolor}

% Page Layout %
\usepackage[margin=1.5in]{geometry}

% Fancy Headers %
\usepackage{fancyhdr}
\fancyhf{}
\cfoot{\thepage}
\rhead{}
\renewcommand{\headrulewidth}{0pt}
\setlength{\headheight}{16pt}

% Math
\usepackage{mathtools}
\usepackage{amssymb}
\usepackage{faktor}
\usepackage{import}
\usepackage{caption}
\usepackage{subcaption}
\usepackage{wrapfig}
\usepackage{enumitem}
\setlist{topsep=0pt}

\usepackage{tikz}
\usetikzlibrary{cd,positioning,babel,shapes,decorations.text,
 decorations.pathmorphing}
\usepackage{tkz-base}
\usepackage{tkz-euclide}

% Theorems
\usepackage[thmmarks, amsmath, thref]{ntheorem}
\usepackage{thmtools}

\theoremsymbol{\ensuremath{\blacksquare}}
\newtheorem*{solution}{Possible solution.}

% Title %
\title{\Huge\textsf{Math Homework -- PreIB 3.AB 2 \& 3}\\
 \Large\textsf{Trigonometric Functions}
 \author{Áďa Klepáčů}
 \date{\today}
}

% Table of Contents %
\usepackage{hyperref}
\hypersetup{
 colorlinks=true,
 linktoc=all,
 linkcolor=blue
}

% Tables %
\usepackage{booktabs}
\usepackage{tabularx}

% Patch for hyphens
\usepackage{regexpatch}
\makeatletter
% Change the `-` delimiter to an active character
\xpatchparametertext\@@@cmidrule{-}{\cA-}{}{}
\xpatchparametertext\@cline{-}{\cA-}{}{}
\makeatother

\newcolumntype{s}{>{\centering\arraybackslash}p{.4\textwidth}}

% Operators %
\DeclareMathOperator{\Ker}{Ker}
\DeclareMathOperator{\Img}{Im}
\DeclareMathOperator{\End}{End}
\DeclareMathOperator{\Aut}{Aut}
\DeclareMathOperator{\Inn}{Inn}

% Common operators %
\newcommand{\R}{\mathbb{R}}
\newcommand{\N}{\mathbb{N}}
\newcommand{\Z}{\mathbb{Z}}
\newcommand{\Q}{\mathbb{Q}}
\newcommand{\C}{\mathbb{C}}

\newcommand{\clr}{\textcolor{BrickRed}}
\newcommand{\clb}{\textcolor{RoyalBlue}}
\newcommand{\clg}{\textcolor{ForestGreen}}
\newcommand{\clm}{\textcolor{Fuchsia}}
\newcommand{\clv}{\textcolor{violet}}
\newcommand{\clbr}{\textcolor{Sepia}}
\newcommand{\cly}{\textcolor{Dandelion}}

% American Paragraph Skip %
\setlength{\parindent}{0pt}
\setlength{\parskip}{1em}

% Document %
\pagestyle{fancy}
\begin{document}

\maketitle
\thispagestyle{fancy}

\section*{Review of Trig Functions}

Given a right triangle
\begin{center}
 \begin{tikzpicture}
  \node (a) at (0,0) {};
  \node (b) at (4,0) {};
  \node (c) at (4,2.31) {};
  
  \draw (3.6,0) -- (3.6,0.4);
  \draw (3.6,0.4) -- (4,0.4);
  \node[circle,inner sep=1pt,fill=black] at (3.8,0.2) {};

  \draw[thick,Fuchsia] (1.5,0) arc (0:30:1.5);
  \node[Fuchsia,yshift=3mm] (theta) at (1,0) {$\theta$};
   
  \draw[thick,RoyalBlue] (a.center) to node[midway,RoyalBlue,yshift=-2ex]
   {adjacent} (b.center);
  \draw[thick,ForestGreen] (b.center) to
   node[midway,ForestGreen,rotate=90,yshift=-2ex] {opposite} (c.center);
  \draw[thick,BrickRed] (a.center) to node[midway,BrickRed,yshift=2ex,rotate=30]
   {hypotenuse} (c.center);
 \end{tikzpicture}
\end{center}
we define three \textbf{trigonometric functions}
\[
 \sin \clm{\theta} = \frac{\text{\clg{opposite}}}{\text{\clr{hypotenuse}}},
 \quad \cos \clm{\theta} =
 \frac{\text{\clb{adjacent}}}{\text{\clr{hypotenuse}}}, \quad \tan \clm{\theta}
 = \frac{\text{\clg{opposite}}}{\text{\clb{adjacent}}}.
\]
The \textbf{inverse} trigonometric functions to these three are naturally
\[
 \sin^{-1}\left( \frac{\text{\clg{opposite}}}{\text{\clr{hypotenuse}}} \right) =
 \clm{\theta}, \quad \cos^{-1}\left(
 \frac{\text{\clb{adjacent}}}{\text{\clr{hypotenuse}}} \right) = \clm{\theta},
 \quad \tan^{-1}\left( \frac{\text{\clg{opposite}}}{\text{\clb{adjacent}}}
 \right) = \clm{\theta}.
\]
Further, in \textbf{any} triangle
\begin{center}
 \begin{tikzpicture}
  \node (a) at (0,0) {};
  \node (b) at (4,0) {};
  \node (c) at (2.5,3) {};
  
  \draw[thick,BrickRed] (1,0) arc (0:50:1);
  \node[BrickRed,yshift=3mm] (alpha) at (0.6,0) {$\alpha$};
  
  \draw[thick,RoyalBlue] (3,0) arc (180:117:1);
  \node[RoyalBlue,yshift=3mm] (beta) at (3.4,0) {$\beta$};
  
  \draw[thick,ForestGreen] (2.947,2.106) arc (-63:-130:1);
  \node[ForestGreen,yshift=3mm] (gamma) at (2.45,2.1) {$\gamma$};
  

  \draw[thick] (a.center) to node[midway,yshift=-2ex] {$c$} (b.center);
  \draw[thick] (b.center) to node[midway,xshift=2ex] {$a$} (c.center);
  \draw[thick] (a.center) to node[midway,xshift=-1.5ex,yshift=1.5ex] {$b$}
   (c.center);
 \end{tikzpicture}
\end{center}
the \textbf{Law of Sines} holds. It says that
\[
 \frac{a}{\sin \clr{\alpha}} = \frac{b}{\sin \clb{\beta}} = \frac{c}{\sin
  \clg{\gamma}}.
\]
We also have the \textbf{Law of Cosines} which says that
\begin{equation*}
 \begin{split}
  c^2 &= a^2 + b^2 - 2ab \cdot \cos \clg{\gamma}, \\
  b^2 &= a^2 + c^2 - 2ac \cdot \cos \clb{\beta}, \\
  a^2 &= b^2 + c^2 - 2bc \cdot \cos \clr{\alpha}. \\
 \end{split}
\end{equation*}

\subsection*{Exercises}
\begin{enumerate}
 \item Count Dolgorukov is a Russian landlord. He tends to lounge away his
  remaining days in his enormous garden. He is also very pedantic and meticulous
  about precision and symmetry. On one sunny noon, one of his poplar trees
  caught his unforgiving gaze. At first glance it seemed as though it was the
  same height as all the others but it cast a longer shadow.

  Count Dolgorukov, seething with rage, gathered his gardeners and ordered the
  tree shortened. Not knowing which it was, the gardeners had to wait until the
  next noon for the sun to be as high in the sky as it can be, which in Russia
  \textbf{is about 60 \%}, with 100 \% meaning right above the head, in order to
  tell the lengths of the shadows apart.
 
  The gardeners measured that all but one poplar cast a \textbf{shadow of
  approximately 21.7 meters}. The \textbf{highest poplar} cast a \textbf{shadow
  of approximately 22.3 meters}. How much should Count Dolgorukov's gardeners
  shorten the poplar so that it is as high as all the others?
 
  \emph{You can assume for simplicity that the Sun follows a perfect arc.}
 
  \begin{center}
   \begin{tikzpicture}[scale=1.25]
    \node (a) at (0,0) {};
    \node (b) at (2.17,0) {};
    \node (c) at (2.17,3) {};

    \draw[thick,RoyalBlue] (a.center) to node[midway,RoyalBlue,yshift=-2ex]
     {shadow} (b.center);
    \draw[thick,ForestGreen] (b.center) to
     node[midway,ForestGreen,rotate=90,yshift=-2ex] {poplar (height ?)}
     (c.center);

    \draw[thick] (58.7:5) arc (58:90:5);
    \node[draw,circle,fill=black,inner sep=2pt] (100) at (90:5) {};
    \node[above=1mm of 100] {100 \%};

    \draw [thick,Dandelion] (5,0) arc (0:49.3:5);
    \node[draw,circle,Dandelion] (sun) at (54:5) {Sun};
    \node[above right=0mm and 0mm of sun,Dandelion] {60 \%};

    \draw[thick,Dandelion] (0,0) -- (sun);
   \end{tikzpicture}
  \end{center}
 \item Mr. Trump is a sailor. A brutal sea storm has completely thrown him off
  course. After a while, he's found himself approaching an unknown island. The
  sea is still wild and it's raining making Mr. Trump unable to estimate his
  distance to the island. Using his old astrolabe (an angle-measuring device),
  he determines \textbf{he's viewing the island under an angle of about $\pi /
  2$}. Thirty minutes later, he repeats the measurement and \textbf{measures an
  angle of $2\pi / 3$}. Basing himself upon a lifetime's worth of experience,
  Mr. Trump thinks he might have closed the distance to the island \textbf{by
  about 5 km}.
  
  Using these three pieces of data, determine \textbf{how long is reaching the
  island going to take Mr. Trump} and also \textbf{how wide the island
  approximately is}.

  \textbf{Hint:} Use the Law of Sines.
  \begin{center}
   \begin{tikzpicture}[scale=0.5]
    \node[circle,fill=black,inner sep=2pt] (a) at (0,0) {};
    
    \node[circle,fill=black,inner sep=2pt] (b) at (0,5) {};
    \node (c) at (0,11.83) {};
    \node[circle,fill=black,inner sep=2pt] (d) at (-11.83,11.83) {};
    \node[circle,fill=black,inner sep=2pt] (e) at (11.83,11.83) {};

    \draw[thick,ForestGreen,dashed] (a) to node[pos=0.7,ForestGreen,xshift=4ex]
     {5 km} (b);
    \draw[thick] (a) -- (d);
    \draw[thick] (a) -- (e);
    \draw[thick] (b) -- (d);
    \draw[thick] (b) -- (e);

    \draw[thick,decorate,decoration={snake,amplitude=1},BrickRed] (d) to
     node[midway,yshift=2ex,BrickRed] {island (width ?)} (e);
    \draw[thick,<->,ForestGreen,dashed] (b) to
     node[pos=0.6,ForestGreen,xshift=4ex] {? km} (c.center);
    \draw[thick,RoyalBlue] (45:2) arc (45:135:2);
    \node[RoyalBlue,yshift=6mm] at (0,0) {$\pi / 2$};
    \draw[thick,RoyalBlue] (90:5) + (30:2) arc (30:150:2);
    \node[RoyalBlue,yshift=6mm] at (0,5) {$2\pi / 3$};
   \end{tikzpicture}
  \end{center}
\end{enumerate}

\section*{Polar Coordinates}
The Cartesian coordinate system, which you're all used to, determines points in
the plane by their pair of coordinates -- $(x,y)$ -- which tells one how many
steps to the right ($x$) and how many steps upward ($y$) you need to make to get
to this point from the origin $(0,0)$. It's basically a grid.
\begin{center}
 \begin{tikzpicture}[scale=0.8]
  \tkzInit[xmin=-4,xmax=4,ymin=-3,ymax=3]
  \tkzDrawX \tkzDrawY
  \tkzGrid
  \tkzDefPoints{1/2/A,-3/1/B,2/-1/C,0/0/O}
  \tkzDrawPoints[size=4](A,B,C,O)
  \tkzLabelPoint[below right](A){$(1,2)$}
  \tkzLabelPoint[below right](B){$(-3,1)$}
  \tkzLabelPoint[below right](C){$(2,-1)$}
  \tkzLabelPoint[below right](O){$(0,0)$}
 \end{tikzpicture}
\end{center}
As you can see, the point $(1,2)$ is 1 step to the right and 2 steps upward from
$(0,0)$, the point $(-3,1)$ is 3 steps to the left and 1 step upward and the
point $(2,-1)$ is located 2 steps to the right and 1 step downward.

There is another system how to describe points, which is more natural to humans,
called \textbf{polar coordinates}. Besides other things, trigonometry tells us
that every point is determined by its angle `of elevation' and distance from the
origin. We'll write these coordinates as $(r:\theta)$ where $r$ is the distance
and $\theta$ the angle. For example, in order to see the point $(3:\pi / 3)$ I
have to turn my head $\pi / 3$ radians (60$^{\circ}$) from the ground and look 3
meters ahead. The plane in these coordinates looks like a lot of concentric
circles.
\begin{center}
 \begin{tikzpicture}[scale=1.5]
  \tkzInit[xmin=-3,xmax=3,ymin=-3,ymax=3]
  \tkzDrawX[label=] \tkzDrawY[label=]
  \tkzDefPoint(0,0){O}
  \tkzDefPoint(45:1){A}
  \tkzDefPoint(150:2){B}
  \tkzDefPoint(-60:3){C}
  
  \tkzDrawSegment[thick,ForestGreen](O,A)
  \tkzDrawSegment[thick,BrickRed](O,B)
  \tkzDrawSegment[thick,Fuchsia](O,C)

  \tkzDrawPoint[size=4](O)
  \tkzDrawPoint[color=ForestGreen,size=4](A)
  \tkzDrawPoint[color=BrickRed,size=4](B)
  \tkzDrawPoint[color=Fuchsia,size=4](C)

  \tkzDrawCircle[thick,ForestGreen](O,A)
  \tkzDrawCircle[thick,BrickRed](O,B)
  \tkzDrawCircle[thick,Fuchsia](O,C)

  \tkzLabelPoint[below right](O){$(0:0)$}
  \tkzLabelPoint[right,ForestGreen](A){$(1:\frac{\pi}{4})$}
  \tkzLabelPoint[left,BrickRed](B){$(2:\frac{5\pi}{6})$}
  \tkzLabelPoint[below right,Fuchsia](C){$(3:-\frac{\pi}{3})$}
 \end{tikzpicture}
\end{center}
To see the point $\clg{(1:\pi / 4)}$, I need to tilt my head $\pi / 4$ radians
counter-clockwise and look 1 meter ahead and to see for instance $\clm{(3:-\pi /
3)}$ I tilt my head $\pi / 3$ radians clockwise and look 3 meters ahead.

We'll learn how to convert coordinates between Cartesian and polar coordinates.
Let's start with the direction polar $ \to $ Cartesian. Suppose I have a point
$(2:\pi / 4)$ in polar coordinates. Let's draw it in \textbf{Cartesian}
coordinates instead.

\end{document}
