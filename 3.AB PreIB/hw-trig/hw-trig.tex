\documentclass[a4paper,11pt]{article}

\usepackage[czech,english]{babel}
% Fonts %
\usepackage{fouriernc}
\usepackage[T1]{fontenc}

% Colors %
\usepackage[dvipsnames]{color}
\usepackage[dvipsnames]{xcolor}

% Page Layout %
\usepackage[margin=1.5in]{geometry}

% Fancy Headers %
\usepackage{fancyhdr}
\fancyhf{}
\cfoot{\thepage}
\rhead{}
\renewcommand{\headrulewidth}{0pt}
\setlength{\headheight}{16pt}

% Math
\usepackage{mathtools}
\usepackage{amssymb}
\usepackage{faktor}
\usepackage{import}
\usepackage{caption}
\usepackage{subcaption}
\usepackage{wrapfig}
\usepackage{enumitem}
\setlist{topsep=0pt}

\usepackage{tikz}
\usetikzlibrary{cd,positioning,babel,shapes,decorations.text,
 decorations.pathmorphing}
\usepackage{tkz-base}
\usepackage{tkz-euclide}

% Theorems
\usepackage[thmmarks, amsmath, thref]{ntheorem}
\usepackage{thmtools}

\theoremsymbol{\ensuremath{\blacksquare}}
\newtheorem*{solution}{Possible solution.}

% Title %
\title{\Huge\textsf{Math Homework -- PreIB 3.AB 2 \& 3}\\
 \Large\textsf{Trigonometric Functions}
 \author{Áďa Klepáčů}
 \date{\today}
}

% Table of Contents %
\usepackage{hyperref}
\hypersetup{
 colorlinks=true,
 linktoc=all,
 linkcolor=blue
}

% Tables %
\usepackage{booktabs}
\usepackage{tabularx}

% Patch for hyphens
\usepackage{regexpatch}
\makeatletter
% Change the `-` delimiter to an active character
\xpatchparametertext\@@@cmidrule{-}{\cA-}{}{}
\xpatchparametertext\@cline{-}{\cA-}{}{}
\makeatother

\newcolumntype{s}{>{\centering\arraybackslash}p{.4\textwidth}}

% Operators %
\DeclareMathOperator{\Ker}{Ker}
\DeclareMathOperator{\Img}{Im}
\DeclareMathOperator{\End}{End}
\DeclareMathOperator{\Aut}{Aut}
\DeclareMathOperator{\Inn}{Inn}

% Common operators %
\newcommand{\R}{\mathbb{R}}
\newcommand{\N}{\mathbb{N}}
\newcommand{\Z}{\mathbb{Z}}
\newcommand{\Q}{\mathbb{Q}}
\newcommand{\C}{\mathbb{C}}

\newcommand{\clr}{\textcolor{BrickRed}}
\newcommand{\clb}{\textcolor{RoyalBlue}}
\newcommand{\clg}{\textcolor{ForestGreen}}
\newcommand{\clm}{\textcolor{Fuchsia}}
\newcommand{\clv}{\textcolor{violet}}
\newcommand{\clbr}{\textcolor{Sepia}}
\newcommand{\cly}{\textcolor{Dandelion}}

% American Paragraph Skip %
\setlength{\parindent}{0pt}
\setlength{\parskip}{1em}

% Document %
\pagestyle{fancy}
\begin{document}

\maketitle
\thispagestyle{fancy}

\section*{Review of Trig Functions}

Given a right triangle
\begin{center}
 \begin{tikzpicture}
  \node (a) at (0,0) {};
  \node (b) at (4,0) {};
  \node (c) at (4,2.31) {};
  
  \draw (3.6,0) -- (3.6,0.4);
  \draw (3.6,0.4) -- (4,0.4);
  \node[circle,inner sep=1pt,fill=black] at (3.8,0.2) {};

  \draw[thick,Fuchsia] (1.5,0) arc (0:30:1.5);
  \node[Fuchsia,yshift=3mm] (theta) at (1,0) {$\theta$};
   
  \draw[thick,RoyalBlue] (a.center) to node[midway,RoyalBlue,yshift=-2ex]
   {adjacent} (b.center);
  \draw[thick,ForestGreen] (b.center) to
   node[midway,ForestGreen,rotate=90,yshift=-2ex] {opposite} (c.center);
  \draw[thick,BrickRed] (a.center) to node[midway,BrickRed,yshift=2ex,rotate=30]
   {hypotenuse} (c.center);
 \end{tikzpicture}
\end{center}
we define three \textbf{trigonometric functions}
\[
 \sin \clm{\theta} = \frac{\text{\clg{opposite}}}{\text{\clr{hypotenuse}}},
 \quad \cos \clm{\theta} =
 \frac{\text{\clb{adjacent}}}{\text{\clr{hypotenuse}}}, \quad \tan \clm{\theta}
 = \frac{\text{\clg{opposite}}}{\text{\clb{adjacent}}}.
\]
The \textbf{inverse} trigonometric functions to these three are naturally
\[
 \sin^{-1}\left( \frac{\text{\clg{opposite}}}{\text{\clr{hypotenuse}}} \right) =
 \clm{\theta}, \quad \cos^{-1}\left(
 \frac{\text{\clb{adjacent}}}{\text{\clr{hypotenuse}}} \right) = \clm{\theta},
 \quad \tan^{-1}\left( \frac{\text{\clg{opposite}}}{\text{\clb{adjacent}}}
 \right) = \clm{\theta}.
\]
Further, in \textbf{any} triangle
\begin{center}
 \begin{tikzpicture}
  \node (a) at (0,0) {};
  \node (b) at (4,0) {};
  \node (c) at (2.5,3) {};
  
  \draw[thick,BrickRed] (1,0) arc (0:50:1);
  \node[BrickRed,yshift=3mm] (alpha) at (0.6,0) {$\alpha$};
  
  \draw[thick,RoyalBlue] (3,0) arc (180:117:1);
  \node[RoyalBlue,yshift=3mm] (beta) at (3.4,0) {$\beta$};
  
  \draw[thick,ForestGreen] (2.947,2.106) arc (-63:-130:1);
  \node[ForestGreen,yshift=3mm] (gamma) at (2.45,2.1) {$\gamma$};
  

  \draw[thick] (a.center) to node[midway,yshift=-2ex] {$c$} (b.center);
  \draw[thick] (b.center) to node[midway,xshift=2ex] {$a$} (c.center);
  \draw[thick] (a.center) to node[midway,xshift=-1.5ex,yshift=1.5ex] {$b$}
   (c.center);
 \end{tikzpicture}
\end{center}
the \textbf{Law of Sines} holds. It says that
\[
 \frac{a}{\sin \clr{\alpha}} = \frac{b}{\sin \clb{\beta}} = \frac{c}{\sin
  \clg{\gamma}}.
\]
We also have the \textbf{Law of Cosines}, the following equalities:
\begin{equation*}
 \begin{split}
  c^2 &= a^2 + b^2 - 2ab \cdot \cos \clg{\gamma}, \\
  b^2 &= a^2 + c^2 - 2ac \cdot \cos \clb{\beta}, \\
  a^2 &= b^2 + c^2 - 2bc \cdot \cos \clr{\alpha}. \\
 \end{split}
\end{equation*}

\subsection*{Exercises}
\begin{enumerate}
 \item Count Dolgorukov is a Russian landlord. He tends to lounge away his
  remaining days in his enormous garden. He is also very pedantic and meticulous
  about precision and symmetry. On one sunny noon, one of his poplar trees
  caught his unforgiving gaze. At first glance it seemed as though it was the
  same height as all the others but it cast a longer shadow.

  Count Dolgorukov, seething with rage, gathered his gardeners and ordered the
  tree shortened. Not knowing which it was, the gardeners had to wait until the
  next noon for the sun to be as high in the sky as it can be, which in Russia
  \textbf{is about 60 \%}, with 100 \% meaning right above the head, in order to
  tell the lengths of the shadows apart.
 
  The gardeners measured that all but one poplar cast a \textbf{shadow of
  approximately 21.7 meters}. The \textbf{highest poplar} cast a \textbf{shadow
  of approximately 22.3 meters}. How much should Count Dolgorukov's gardeners
  shorten the poplar so that it is as high as all the others?
 
  \emph{You can assume for simplicity that the Sun follows a perfect arc.}
 
  \begin{center}
   \begin{tikzpicture}[scale=1.25]
    \node (a) at (0,0) {};
    \node (b) at (2.17,0) {};
    \node (c) at (2.17,3) {};

    \draw[thick,RoyalBlue] (a.center) to node[midway,RoyalBlue,yshift=-2ex]
     {shadow} (b.center);
    \draw[thick,ForestGreen] (b.center) to
     node[midway,ForestGreen,rotate=90,yshift=-2ex] {poplar (height ?)}
     (c.center);

    \draw[thick] (58.7:5) arc (58:90:5);
    \node[draw,circle,fill=black,inner sep=2pt] (100) at (90:5) {};
    \node[above=1mm of 100] {100 \%};

    \draw [thick,Dandelion] (5,0) arc (0:49.3:5);
    \node[draw,circle,Dandelion] (sun) at (54:5) {Sun};
    \node[above right=0mm and 0mm of sun,Dandelion] {60 \%};

    \draw[thick,Dandelion] (0,0) -- (sun);
   \end{tikzpicture}
  \end{center}
 \item Mr. Trump is a sailor. A brutal sea storm has completely thrown him off
  course. After a while, he's found himself approaching an unknown island. The
  still wild sea and bitter rain render Mr. Trump unable to estimate his
  distance to the island. Using his old astrolabe (an angle-measuring device),
  he determines \textbf{he's viewing the island under an angle of about $\pi /
  2$}. Thirty minutes later, he repeats the measurement and \textbf{measures an
  angle of $2\pi / 3$}. Basing himself upon a lifetime's worth of experience,
  Mr. Trump thinks he might have closed the distance to the island \textbf{by
  about 5 km} in these last 30 minutes.
  
  Using these three pieces of data, determine \textbf{how long is reaching the
  island going to take Mr. Trump} and also \textbf{how wide the island
  approximately is}.

  \textbf{Hint:} Use the Law of Sines.
  \begin{center}
   \begin{tikzpicture}[scale=0.5]
    \node[circle,fill=black,inner sep=2pt] (a) at (0,0) {};
    
    \node[circle,fill=black,inner sep=2pt] (b) at (0,5) {};
    \node (c) at (0,11.83) {};
    \node[circle,fill=black,inner sep=2pt] (d) at (-11.83,11.83) {};
    \node[circle,fill=black,inner sep=2pt] (e) at (11.83,11.83) {};

    \draw[thick,ForestGreen,dashed] (a) to node[pos=0.7,ForestGreen,xshift=4ex]
     {5 km} (b);
    \draw[thick] (a) -- (d);
    \draw[thick] (a) -- (e);
    \draw[thick] (b) -- (d);
    \draw[thick] (b) -- (e);

    \draw[thick,decorate,decoration={snake,amplitude=1},BrickRed] (d) to
     node[midway,yshift=2ex,BrickRed] {island (width ?)} (e);
    \draw[thick,<->,ForestGreen,dashed] (b) to
     node[pos=0.6,ForestGreen,xshift=4ex] {? km} (c.center);
    \draw[thick,RoyalBlue] (45:2) arc (45:135:2);
    \node[RoyalBlue,yshift=6mm] at (0,0) {$\pi / 2$};
    \draw[thick,RoyalBlue] (90:5) + (30:2) arc (30:150:2);
    \node[RoyalBlue,yshift=6mm] at (0,5) {$2\pi / 3$};
   \end{tikzpicture}
  \end{center}
\end{enumerate}

\section*{Polar Coordinates}
The Cartesian coordinate system, which you're all used to, determines points in
the plane by their pair of coordinates -- $(x,y)$ -- which tell one how many
steps to the right ($x$) and how many steps upward ($y$) he is to make to reach
this point from the origin $(0,0)$. It's basically a grid.
\begin{center}
 \begin{tikzpicture}[scale=0.8]
  \tkzInit[xmin=-4,xmax=4,ymin=-3,ymax=3]
  \tkzDrawX \tkzDrawY
  \tkzGrid
  \tkzDefPoints{1/2/A,-3/1/B,2/-1/C,0/0/O}
  \tkzDrawPoints[size=4](A,B,C,O)
  \tkzLabelPoint[below right](A){$(1,2)$}
  \tkzLabelPoint[below right](B){$(-3,1)$}
  \tkzLabelPoint[below right](C){$(2,-1)$}
  \tkzLabelPoint[below right](O){$(0,0)$}
 \end{tikzpicture}
\end{center}
As you can see, the point $(1,2)$ is 1 step to the right and 2 steps upward from
$(0,0)$, the point $(-3,1)$ is 3 steps to the left and 1 step upward and the
point $(2,-1)$ is located 2 steps to the right and 1 step downward.

There exists a different system of point description, which is more natural to
humans, called \textbf{polar coordinates}. Besides other things, trigonometry
tells us that every point is determined by its angle `of elevation' and distance
from the origin. We'll write these coordinates as $(r:\theta)$ where $r$ is the
distance and $\theta$ the angle. For example, in order to see the point $(3:\pi
/ 3)$ I have to tilt my head $\pi / 3$ radians (60$^{\circ}$) toward the sky and
look 3 meters ahead. The plane in these coordinates resembles a bunch of
concentric circles.
\begin{center}
 \begin{tikzpicture}[scale=1.5]
  \tkzInit[xmin=-3,xmax=3,ymin=-3,ymax=3]
  \tkzDrawX[label=] \tkzDrawY[label=]
  \tkzDefPoint(0,0){O}
  \tkzDefPoint(45:1){A}
  \tkzDefPoint(150:2){B}
  \tkzDefPoint(-60:3){C}
  
  \tkzDrawSegment[thick,ForestGreen](O,A)
  \tkzDrawSegment[thick,BrickRed](O,B)
  \tkzDrawSegment[thick,Fuchsia](O,C)

  \tkzDrawPoint[size=4](O)
  \tkzDrawPoint[color=ForestGreen,size=4](A)
  \tkzDrawPoint[color=BrickRed,size=4](B)
  \tkzDrawPoint[color=Fuchsia,size=4](C)

  \tkzDrawCircle[thick,ForestGreen](O,A)
  \tkzDrawCircle[thick,BrickRed](O,B)
  \tkzDrawCircle[thick,Fuchsia](O,C)

  \tkzLabelPoint[below left](O){$(0:0)$}
  \tkzLabelPoint[right,ForestGreen](A){$(1:\frac{\pi}{4})$}
  \tkzLabelPoint[left,BrickRed](B){$(2:\frac{5\pi}{6})$}
  \tkzLabelPoint[below right,Fuchsia](C){$(3:-\frac{\pi}{3})$}
 \end{tikzpicture}
\end{center}
To see the point $\clg{(1:\pi / 4)}$, I need to rotate my head $\pi / 4$ radians
counter-clockwise and look 1 meter ahead and to see for instance $\clm{(3:-\pi /
3)}$ I rotate it $\pi / 3$ radians clockwise and look 3 meters ahead.

We'll learn how to convert coordinates between Cartesian and polar coordinates.
Let's start with the direction polar $ \to $ Cartesian. Suppose I have a point
$(2:\pi / 4)$ in polar coordinates. Let's draw it in \textbf{Cartesian}
coordinates instead.
\begin{center}
 \begin{tikzpicture}[scale=2]
  \tkzInit[xmax=3,ymax=3]
  \tkzDrawX[label=] \tkzDrawY[label=]
  \tkzGrid
  \tkzDefPoint(45:2){A}
  \tkzDefPoint(1.414,0){X}
  \tkzDefPoint(1.214,0){R1}
  \tkzDefPoint(1.214,0.2){R2}
  \tkzDefPoint(1.414,0.2){R3}
  \tkzDefPoint(1.314,0.1){D}

  \tkzDrawSegment(R1,R2)
  \tkzDrawSegment(R2,R3)
  \tkzDrawSegment[thick,color=BrickRed](O,X)
  \tkzDrawSegment[thick,color=Dandelion](X,A)

  \tkzLabelPoint[above right](A){$(\clg{2}:\clm{\frac{\pi}{4}})$}

  \tkzDefPoint(0,0){O}
  \tkzLabelPoint[below left](O){$(0,0)$}

  \tkzDrawSegment[thick,color=ForestGreen](O,A)
  \tkzLabelSegment[color=ForestGreen,yshift=6mm,xshift=-1mm](O,A){\Large $2$}
  \tkzDrawArc[R,thick,color=Fuchsia](O,1)(0,45)

  \tkzDefPoint(35:0.8){theta}
  \tkzLabelPoint[color=Fuchsia](theta){\Large $\frac{\pi}{4}$}

  \tkzDrawPoint[size=4](O)
  \tkzDrawPoint[size=4](A)

  \tkzLabelSegment[below,color=BrickRed](O,X){\Large $x$}
  \tkzLabelSegment[right,color=Dandelion](X,A){\Large $y$}

  \tkzDrawPoint(D)
 \end{tikzpicture}
\end{center}
We \textbf{do} know how to calculate $\clr{x}$ and $\cly{y}$, don't we? Using
the definitions of $\sin$ and $\cos$, we can compute
\[
 \cos \left( \clm{\frac{\pi}{4}} \right) = \frac{\clr{x}}{\clg{2}} \quad
 \text{and} \quad \sin \left( \clm{\frac{\pi}{4}} \right) =
 \frac{\cly{y}}{\clg{2}}.
\]
This means that $\clr{x} = \clg{2} \cdot \cos(\clm{\pi / 4})$ and $\cly{y} =
\clg{2} \cdot \sin(\clm{\pi / 4})$. In this particular case, the coordinates end
up being $\clr{x} = \sqrt{2}$ and $\cly{y} = \sqrt{2}$. In general, a point
given in polar coordinates as $(\clg{r},\clm{\theta})$ is expressed in Cartesian
coordinates as $(\clg{r} \cdot \cos\clm{\theta},\clg{r} \cdot \sin
\clm{\theta})$.

Let's also discuss the opposite direction -- Cartestion $ \to $ polar. What if I
know the coordinates of a point in Cartesian coordinates and I want to know what
they are in polar? It's actually very easy. We need to determine two variables
-- $\clg{r}$ and $\clm{\theta}$. Looking at the triangle above, it seems that
$\clg{r}$ can be calculated via the Pythagoras' Theorem. Indeed, we know that
\[
 \clg{r}^2 = \clr{x}^2 + \cly{y}^2
\]
and so
\[
 \clg{r} = \sqrt{\clr{x}^2 + \cly{y}^2}
\]
because $\clg{r}$ is \textbf{always positive} (remember, it's a
\emph{distance}).

To compute $\clm{\theta}$ we can use the $\tan^{-1}$ trig. function. By
definition, we have
\[
 \tan \clm{\theta} = \frac{\cly{y}}{\clr{x}}
\]
and so
\[
 \clm{\theta} = \tan^{-1}\left( \frac{\cly{y}}{\clr{x}} \right).
\]
Hence, a point $(\clr{x},\cly{y})$ in Cartesian coordinates is actually the
point
\[
 \left(\sqrt{\clr{x}^2 + \cly{y}^2}, \tan^{-1}(\cly{y}/\clr{x})\right)
\]
in polar. Going back to our original example, if $(\clr{x},\cly{y}) =
(\sqrt{2},\sqrt{2})$, then
\[
 \clg{r} = \sqrt{\clr{x}^2 + \cly{y}^2} = \sqrt{(\sqrt{2})^2 + (\sqrt{2})^2} =
 \sqrt{2 + 2} = \sqrt{4} = 2
\]
and
\[
 \clm{\theta} = \tan^{-1}\left( \frac{\cly{y}}{\clr{x}} \right) =
 \tan^{-1}\left( \frac{\sqrt{2}}{\sqrt{2}} \right) = \tan^{-1}(1) =
 \frac{\pi}{4}.
\]
See, everything works.

\subsection*{Exercises}
\begin{enumerate}
 \item One of the reasons you don't hear of polar coordinates before Cartesian
  is that they're naturally \textbf{circular} coordinates. In essence, the
  simplest objects are curved and flat objects like straight lines are functions
  with difficult definitions. To see this, \textbf{determine the equation of a
  straight line in polar coordinates}.
 
  \textbf{Hint}: Start with the equation of a line in Cartesian coordinates,
  $\cly{y} = a \clr{x} + b$, convert $\clr{x}$ and $\cly{y}$ to polar
  coordinates and isolate either $\clg{r}$ or $\clm{\theta}$.
 \item On the other hand, circles (centred at the origin) have very simple
  equations in polar coordinates. More specifically, the equation $\clg{r} = R$
  describes a circle with centre $(0,0)$ and radius $R$ in polar coordinates.
  \textbf{Use this knowledge to find the equation of a circle in Cartesian
  coordinates}.
 
  \textbf{Hint:} Just convert $\clg{r}$ to Cartesian coordinates. Notice that
  some shapes are given by \textbf{functions} in polar coordinates that
  \textbf{cannot be described by a function} in the Cartesian (the circle
  \textbf{is not} a function in Cartesian coordinates, but it \textbf{is} in
  polar).
 \item It's early morning and the Sun and Moon are both still visible on the
  sky. Say you're looking directly at the Moon and have to turn your head about
  $\pi / 3$ radians to the left to look directly at the Sun. \textbf{How far
  away are the Moon and the Sun from one another at this exact moment}?
 
  \textbf{Data:} Assume the Sun is 150,000,000 km away and the Moon is 400,000
  km away from the Earth.
 
  \textbf{Hint:} Use the \textbf{Law of Cosines}.
  \begin{center}
   \begin{tikzpicture}
    \tkzInit
    \tkzDefPoint(0,0){E}
    \tkzDefPoint(70:2){M}
    \tkzDefPoint(130:7){S}
    \tkzDefPoint(100:0.5){a}

    \tkzDrawArc[R,thick,color=Fuchsia](E,1)(70,130)
    \tkzLabelPoint[above,color=Fuchsia,yshift=-2mm](a){$\frac{\pi}{3}$}

    \tkzDrawSegment[thick,color=RoyalBlue](E,M)
    \tkzDrawSegment[thick,color=Dandelion](E,S)
    \tkzDrawSegment[thick](S,M)

    \tkzDrawPoint[size=6](E)
    \tkzDrawPoint[size=4,color=RoyalBlue](M)
    \tkzDrawPoint[size=10,color=Dandelion](S)

    \tkzLabelPoint[below,yshift=-1ex](E){Earth}
    \tkzLabelPoint[right,color=RoyalBlue,xshift=1ex](M){Moon}
    \tkzLabelPoint[above left,color=Dandelion,yshift=1ex,xshift=-1ex](S){Sun}

    \tkzLabelSegment[right,color=RoyalBlue,yshift=-1mm,xshift=1mm](E,M){400,000
    km}
    \tkzLabelSegment[below left,color=Dandelion](E,S){150,000,000 km}
    \tkzLabelSegment[above right](M,S){\Large ? km}
   \end{tikzpicture}
  \end{center}
 \item (\textbf{\clr{OPTIONAL}}) Can you generalize the previous problem and
  \textbf{find a formula for the distance between two points}
  $(\clg{r_1},\clm{\theta_1})$ and $(\clg{r_2},\clm{\theta_2})$ in
  \textbf{polar} coordinates?
\end{enumerate}

\end{document}
