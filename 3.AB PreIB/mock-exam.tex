\documentclass[a4paper,11pt]{article}

\usepackage[czech,english]{babel}
% Fonts %
\usepackage{fouriernc}
\usepackage[T1]{fontenc}

% Colors %
\usepackage[dvipsnames]{color}
\usepackage{xcolor}

% Page Layout %
\usepackage[margin=1.5in]{geometry}

% Fancy Headers %
\usepackage{fancyhdr}
\fancyhf{}
\cfoot{\thepage}
\rhead{}
\renewcommand{\headrulewidth}{0pt}
\setlength{\headheight}{16pt}

% Math
\usepackage{mathtools}
\usepackage{amssymb}
\usepackage{faktor}
\usepackage{import}
\usepackage{caption}
\usepackage{subcaption}
\usepackage{wrapfig}
\usepackage{enumitem}
\usepackage{tikz}
\usetikzlibrary{cd,positioning,babel,shapes}
\usepackage{tkz-base}
\usepackage{tkz-euclide}

% Theorems
\usepackage{amsthm}
\usepackage{thmtools}

% Title %
\title{\Huge\textsf{Mock Exam}\\
 \Large\textsf{Systems of Linear Equations}
 \author{Áďa Klepáčů}
 \date{\today}
}

% Table of Contents %
\usepackage{hyperref}
\hypersetup{
 colorlinks=true,
 linktoc=all,
 linkcolor=blue
}

% Tables %
\usepackage{booktabs}
\usepackage{tabularx}

% Patch for hyphens
\usepackage{regexpatch}
\makeatletter
% Change the `-` delimiter to an active character
\xpatchparametertext\@@@cmidrule{-}{\cA-}{}{}
\xpatchparametertext\@cline{-}{\cA-}{}{}
\makeatother

\newcolumntype{s}{>{\centering\arraybackslash}p{.4\textwidth}}

% Operators %
\DeclareMathOperator{\Ker}{Ker}
\DeclareMathOperator{\Img}{Im}
\DeclareMathOperator{\End}{End}
\DeclareMathOperator{\Aut}{Aut}
\DeclareMathOperator{\Inn}{Inn}

% Common operators %
\newcommand{\R}{\mathbb{R}}
\newcommand{\N}{\mathbb{N}}
\newcommand{\Z}{\mathbb{Z}}
\newcommand{\Q}{\mathbb{Q}}
\newcommand{\C}{\mathbb{C}}

\newcommand{\tr}{\textcolor{red}}
\newcommand{\tb}{\textcolor{blue}}
\newcommand{\tg}{\textcolor{green}}
\newcommand{\tm}{\textcolor{magenta}}
\newcommand{\tv}{\textcolor{violet}}

% American Paragraph Skip %
\setlength{\parindent}{0pt}
\setlength{\parskip}{1em}

% Document %
\pagestyle{fancy}
\begin{document}

\maketitle
\thispagestyle{fancy}

\section*{Problem 1.}

You're baking cookies for a big party. Your cookies are surprisingly pretty good
and each of the guests will have on average eaten 4 of them by the end of the
event. \emph{\tb{Define a function}} which expresses the number of snacks
\textbf{that are left} (on average) after the event ends based on the number of
people that attended the party. Assume you have baked 300 cookies.

In comes your rival. She couldn't stand the idea of the guests eating only your
handmade cookies. So she made hers. However, they are rushed and in general not
very tasty. An average guest eats 2 of them during the night. She is a hard
worker, though, so she made twice as many as you did. \textbf{The number of
cookies you made is now also an unknown}. If $x$ denotes the number of people
that attended the party and $y$ denotes the number of cookies you made,
\emph{\tb{define a function}} which expresses the \textbf{number of cookies made
by your rival that are left} after the event ends with inputs $x$ and $y$.

Finally, the event is over and out of the cookies you made, 100 are left, and
out of the cookies your rival made, 350 are left. \textbf{How many cookies did
you make and how many guests attended the party?} Formulate this problem as a
system of two linear equations in two variables -- $x$ and $y$ -- and solve it.

\end{document}
