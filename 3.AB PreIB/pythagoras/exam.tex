\documentclass[a4paper,11pt]{article}

\usepackage[czech,english]{babel}
% Fonts %
\usepackage{fouriernc}
\usepackage[T1]{fontenc}

% Colors %
\usepackage[dvipsnames]{color}
\usepackage[dvipsnames]{xcolor}

% Page Layout %
\usepackage[margin=1.5in]{geometry}

% Fancy Headers %
\usepackage{fancyhdr}
\fancyhf{}
\cfoot{\thepage}
\rhead{}
\renewcommand{\headrulewidth}{0pt}
\setlength{\headheight}{16pt}

% Math
\usepackage{mathtools}
\usepackage{amssymb}
\usepackage{faktor}
\usepackage{import}
\usepackage{caption}
\usepackage{subcaption}
\usepackage{wrapfig}
\usepackage{enumitem}
\setlist{topsep=0pt}

\usepackage{tikz}
\usetikzlibrary{cd,positioning,babel,shapes}
\usepackage{tkz-base}
\usepackage{tkz-euclide}

% Theorems
\usepackage{thmtools}
\usepackage[thmmarks, amsmath, thref]{ntheorem}

\theoremsymbol{\ensuremath{\blacksquare}}
\newtheorem*{solution}{Possible solution.}

% Title %
\title{\Huge\textsf{Math Exam -- PreIB 3.AB 4}\\
 \Large\textsf{Pythagorean Theorem}
 \author{Áďa Klepáčů}
 \date{November 13, 2023}
}

% Table of Contents %
\usepackage{hyperref}
\hypersetup{
 colorlinks=true,
 linktoc=all,
 linkcolor=blue
}

% Tables %
\usepackage{booktabs}
\usepackage{tabularx}

% Patch for hyphens
\usepackage{regexpatch}
\makeatletter
% Change the `-` delimiter to an active character
\xpatchparametertext\@@@cmidrule{-}{\cA-}{}{}
\xpatchparametertext\@cline{-}{\cA-}{}{}
\makeatother

\newcolumntype{s}{>{\centering\arraybackslash}p{.4\textwidth}}

% Operators %
\DeclareMathOperator{\Ker}{Ker}
\DeclareMathOperator{\Img}{Im}
\DeclareMathOperator{\End}{End}
\DeclareMathOperator{\Aut}{Aut}
\DeclareMathOperator{\Inn}{Inn}

% Common operators %
\newcommand{\R}{\mathbb{R}}
\newcommand{\N}{\mathbb{N}}
\newcommand{\Z}{\mathbb{Z}}
\newcommand{\Q}{\mathbb{Q}}
\newcommand{\C}{\mathbb{C}}

\newcommand{\clr}{\textcolor{BrickRed}}
\newcommand{\clb}{\textcolor{RoyalBlue}}
\newcommand{\clg}{\textcolor{ForestGreen}}
\newcommand{\clm}{\textcolor{Magenta}}
\newcommand{\clv}{\textcolor{violet}}
\newcommand{\clbr}{\textcolor{Sepia}}

% American Paragraph Skip %
\setlength{\parindent}{0pt}
\setlength{\parskip}{1em}

% Document %
\pagestyle{fancy}
\begin{document}

\maketitle
\thispagestyle{fancy}

\begin{enumerate}
 \item You're given the following triangle:
  \begin{center}
   \begin{tikzpicture}
    \tkzDefPoint(0,0){a}
    \tkzDefPoint(2,0){sab}
    \tkzDefPoint(6,0){b}
    \tkzDefPoint(2,4){c}

    \tkzMarkRightAngle[fill=Gray!20,draw=Magenta,size=0.4](c,sab,b)

    \tkzDrawSegment[thick,color=BrickRed](b,c)
    \tkzDrawSegment[thick,color=RoyalBlue](a,c)
    \tkzDrawSegment[thick,color=ForestGreen](a,b)
    \tkzDrawSegment[thick,dashed,color=Magenta](c,sab)

    \tkzDrawPoint[size=4,color=BrickRed](a)
    \tkzDrawPoint[size=4,color=RoyalBlue](b)
    \tkzDrawPoint[size=4,color=ForestGreen](c)

    \tkzLabelPoint[below left,color=BrickRed](a){$A$}
    \tkzLabelPoint[below right,color=RoyalBlue](b){$B$}
    \tkzLabelPoint[above,color=ForestGreen](c){$C$}

    \tkzLabelSegment[above right,color=BrickRed](b,c){$a$}
    \tkzLabelSegment[above left,color=RoyalBlue](a,c){$b$}
    \tkzLabelSegment[below,color=ForestGreen](a,b){$c$}
    \tkzLabelSegment[right,color=Magenta](c,sab){$v_c$}
   \end{tikzpicture}
  \end{center}
  Calculate the length of $\clg{c}$ if you know that $\clr{a} = \sqrt{32},
  \clb{b} = \sqrt{20}$ and $\clm{v_c} = 4$.
 \item Calculate the distance between $(1, 0, 4)$ and $(-1, 3, 5)$ by altering
  the coordinates in the following order:
  \begin{enumerate}
   \item $x,y,z$;
   \item $y,z,x$.
  \end{enumerate}
 Remember, $x$ is the first coordinate, $y$ is the second and $z$ the third.
 \clr{\textbf{Draw the triangles you're using to perform the calculations.}}
\end{enumerate}


\end{document}
