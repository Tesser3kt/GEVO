\documentclass[a4paper,10pt]{article}

\usepackage[czech,english]{babel}
% Fonts %
\usepackage[T1]{fontenc}
\usepackage{FiraSans}
\renewcommand*\oldstylenums[1]{{\firaoldstyle #1}}
\usepackage{PTSerif}

% Colors %
\usepackage[dvipsnames]{color}
\usepackage{xcolor}

% Page Layout %
\usepackage[margin=0.5in]{geometry}

% Fancy Headers %
\usepackage{fancyhdr}
\fancyhf{}
\cfoot{}
\rhead{}
\renewcommand{\headrulewidth}{0pt}
\setlength{\headheight}{16pt}

% Math
\usepackage{mathtools}
\usepackage{amssymb}
\usepackage{faktor}
\usepackage{import}
\usepackage{caption}
\usepackage{subcaption}
\usepackage{wrapfig}
\usepackage{enumitem}
\usepackage{tikz}
\usetikzlibrary{cd,positioning,babel,shapes}
\usepackage{tkz-base}
\usepackage{tkz-euclide}

% Theorems
\usepackage{thmtools}
\usepackage[thmmarks, amsmath, thref]{ntheorem}

% Table of Contents %
\usepackage{hyperref}
\hypersetup{
 colorlinks=true,
 linktoc=all,
 linkcolor=blue
}

% Tables %
\usepackage{booktabs}
\usepackage{tabularx}

% Patch for hyphens
\usepackage{regexpatch}
\makeatletter
% Change the `-` delimiter to an active character
\xpatchparametertext\@@@cmidrule{-}{\cA-}{}{}
\xpatchparametertext\@cline{-}{\cA-}{}{}
\makeatother

\newcolumntype{s}{>{\centering\arraybackslash}p{.4\textwidth}}

% Operators %
\DeclareMathOperator{\Ker}{Ker}
\DeclareMathOperator{\Img}{Im}
\DeclareMathOperator{\End}{End}
\DeclareMathOperator{\Aut}{Aut}
\DeclareMathOperator{\Inn}{Inn}

% Common operators %
\newcommand{\R}{\mathbb{R}}
\newcommand{\N}{\mathbb{N}}
\newcommand{\Z}{\mathbb{Z}}
\newcommand{\Q}{\mathbb{Q}}
\newcommand{\C}{\mathbb{C}}

\newcommand{\tr}{\textcolor{red}}
\newcommand{\tb}{\textcolor{blue}}
\newcommand{\tg}{\textcolor{green}}
\newcommand{\tm}{\textcolor{magenta}}
\newcommand{\tv}{\textcolor{violet}}
\newcommand{\tmar}{\textcolor[HTML]{e23636}}
\newcommand{\tav}{\textcolor[HTML]{2e67a0}}

\definecolor{marvel}{HTML}{e23636}
\definecolor{avatar}{HTML}{2e67a0}

% American Paragraph Skip %
\setlength{\parindent}{0pt}
\setlength{\parskip}{1em}

\renewcommand{\baselinestretch}{1.2}
\renewcommand{\arraystretch}{1.2}
% Document %
\pagestyle{fancy}

\newcolumntype{C}[1]{%
 >{\vbox to 5ex\bgroup\vfill\centering}%
 p{#1}%
 <{\egroup}}

\begin{document}

\thispagestyle{fancy}
\begin{minipage}{.3\textwidth}
 \includegraphics[width=\textwidth]{logo}
\end{minipage}
\hfill
\begin{minipage}{.69\textwidth}
 \centering
 \Large{\sffamily
  GYMNÁZIUM EVOLUTION JIŽNÍ MĚSTO\\
  Posudek vedoucího ročníkové práce\\
  2022/23
 }
\end{minipage}

\begin{center}
 \large{
 \begin{tabular}{ll}
  \textsf{Název práce:} &
  \textsf{
   %%TITLE%%
  }\\
  \textsf{Autor práce, třída:} &
  \textsf{
   %%AUTHOR%%
  }\\
  \textsf{Vedoucí práce:} &
  \textsf{
   %%SUPERVISOR%%
  }
 \end{tabular}
 }
\end{center}

\section*{\sffamily \centering Hodnocení ročníkové práce}

\begin{center}
 \begin{tabular}{c|m{.65\textwidth}|c|c}
  & \multicolumn{1}{c|}{\textsf{Hodnocení ročníkové práce}} & \textsf{Váha} &
  \textsf{Známka}\\
  \toprule
  1. & Problematika a cíl práce jsou zformulovány a odpovídají zadání a názvu
  práce. & 1 &
  %%FORMULATION%%
  \\
  \midrule
  2. & Metodika práce vede k naplnění cílů, je správně a logicky zvolená a
  kvalitně provedená. & 3 &
  %%METHODOLOGY%%
  \\
  \midrule
  3. & Autor se opírá o relevantní prameny a literaturu. & 2 &
  %%BIBLIOGRAPHY%%
  \\
  \midrule
  4. & Struktura práce je logická a vyvážená a předepsané části práce naplňují
  svůj účel i obsah. & 4 &
  %%STRUCTURE%%
  \\
  \midrule
  5. & Práce je originální a obsahuje jednoznačně definovatelný vlastní přínos
  studenta zvolené tématice. & 4 &
  %%CONTRIBUTION%%
  \\
  \midrule
  6. & Práce používá správnou odbornou terminologii, obrazový doprovod je
  kvalitní a odpovídá tématu. & 1 &
  %%TERMINOLOGY%%
  \\
  \midrule
  7. & Formální stránka práce –- autor správně cituje, má správně vedený seznam
  literatury, nic podstatného neopominul. & 3 &
  %%CITATIONS%%
  \\
  \midrule
  8. & Jazyková stránka práce. & 2 &
  %%LANGUAGE%%
  \\
  \midrule
  9. & Grafická stránka práce (formátování). Hodnotí pověřený učitel & 2 &
  %%FORMATTING%%
  \\
  \midrule
  10. & Cíle práce byly splněny. & 4 &
  %%RESULTS%%
 \end{tabular}
\end{center}

\begin{center}
 \fbox{\textbf{Výsledná známka (vážený průměr):} \textbf{
  %%WEIGHTED MEAN%%
 }}
\end{center}

\subsection*{\sffamily\centering Kritéria hodnocení}
\begin{center}
 \begin{tabular}{c|l|l}
  1 & splňuje na výborné úrovni & 100 \%*, 99 -- 85 \% \\
  \midrule
  2 & splňuje chvalitebně & 84 -- 70 \% \\
  \midrule
  3 & splňuje z větší části & 69 -- 55 \% \\
  \midrule
  4 & splňuje z menší části & 54 -- 40 \% \\
  \midrule
  5 & nesplňuje & 39 -- 0 \%
 \end{tabular}\\
 *\footnotesize{mimořádný výkon po všech stránkách}
\end{center}

\section*{\sffamily \centering Slovní hodnocení výsledné práce (klady a
nedostatky)}
%%REVIEW%%

\section*{\sffamily \centering Doporučené otázky k obhajobě}

\begin{enumerate}
  %%QUESTIONS%%
\end{enumerate}

\vfill

\textbf{Datum:
%%DATE%%
}

\end{document}
