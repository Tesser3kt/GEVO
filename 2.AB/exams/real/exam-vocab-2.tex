\documentclass[a4paper,11pt]{article}

\usepackage[czech,english]{babel}
% Fonts %
\usepackage{fouriernc}
\usepackage[T1]{fontenc}

\usepackage{advdate}

% Colors %
\usepackage[dvipsnames]{color}
\usepackage{xcolor}

% Page Layout %
\usepackage[margin=1in]{geometry}
\usepackage{paracol}

% Fancy Headers %
\usepackage{fancyhdr}
\fancyhf{}
\rhead{\sffamily\large Vocabulary Exam -- 2.AB PreIB Math}
\lhead{\sffamily\large \today}
\renewcommand{\headrulewidth}{0.4pt}
\setlength{\headheight}{16pt}

% Math
\usepackage{mathtools}
\usepackage{amssymb}
\usepackage{faktor}
\usepackage{import}
\usepackage{caption}
\usepackage{subcaption}
\usepackage{wrapfig}
\usepackage{enumitem}
\usepackage{tikz}
\usetikzlibrary{cd,positioning,babel,shapes}
\usepackage{tkz-base}
\usepackage{tkz-euclide}

% Theorems
\usepackage{thmtools}
\usepackage[thmmarks, amsmath, thref]{ntheorem}

% Title %
\title{\Huge\textsf{Vocabulary Exam -- 2.AB PreIB Math}\\
  \Large\textsf{Logic \& Set Theory}
  \author{Áďa Klepáčů}
  \date{\today}
}

% Table of Contents %
\usepackage{hyperref}
\hypersetup{
  colorlinks=true,
  linktoc=all,
  linkcolor=blue
}

% Tables %
\usepackage{booktabs}
\usepackage{tabularx}

% Patch for hyphens
\usepackage{regexpatch}
\makeatletter
% Change the `-` delimiter to an active character
\xpatchparametertext\@@@cmidrule{-}{\cA-}{}{}
\xpatchparametertext\@cline{-}{\cA-}{}{}
\makeatother

\newcolumntype{s}{>{\centering\arraybackslash}p{.4\textwidth}}

% Operators %
\DeclareMathOperator{\Ker}{Ker}
\DeclareMathOperator{\Img}{Im}
\DeclareMathOperator{\End}{End}
\DeclareMathOperator{\Aut}{Aut}
\DeclareMathOperator{\Inn}{Inn}

% Common operators %
\newcommand{\R}{\mathbb{R}}
\newcommand{\N}{\mathbb{N}}
\newcommand{\Z}{\mathbb{Z}}
\newcommand{\Q}{\mathbb{Q}}
\newcommand{\C}{\mathbb{C}}

\newcommand{\tr}{\textcolor{red}}
\newcommand{\tb}{\textcolor{blue}}
\newcommand{\tg}{\textcolor{green}}
\newcommand{\tm}{\textcolor{magenta}}
\newcommand{\tv}{\textcolor{violet}}

% American Paragraph Skip %
\setlength{\parindent}{0pt}
\setlength{\parskip}{1em}

% Document %
\pagestyle{fancy}
\renewcommand{\baselinestretch}{1.2}
\begin{document}

\thispagestyle{fancy}

\columnratio{.7}
\begin{paracol}{2}
  In logic, a \underline{\hspace{12ex}} is a declarative statement to which we
  can assign a truth value. Said truth value can be either
  \underline{\hspace{12ex}}, or \underline{\hspace{12ex}}.

  When we want to reverse a truth value, we use \underline{\hspace{12ex}},
  written $\neg p$. To combine statements, we use logical
  \underline{\hspace{12ex}} such as $\wedge$, $\vee$, and $\Rightarrow$. The
  expression $p \Rightarrow q$ is an \underline{\hspace{12ex}} meaning “if $p$
  then $q$”.

  Two propositions are logically \underline{\hspace{12ex}} if they always have
  the same truth value, no matter how the variables are assigned.

  In set theory, the notation $x \in A$ says that $x$ is an
  \underline{\hspace{12ex}} of the set $A$. Applying logical operators to
  statements like $x \in A$ and $x \in B$ leads to familiar set operations: the
  \underline{\hspace{12ex}} contains the objects that lie in $A$ or in $B$ (or
  in both), while the \underline{\hspace{12ex}} of $A$ and $B$ contains exactly
  the objects that lie in both sets. Finally, $A \setminus B$ is the
  \underline{\hspace{12ex}} of $A$ and $B$, containing everything in $A$ but not
  in $B$.

  \switchcolumn
  \renewcommand{\arraystretch}{2}
  \vspace*{-32pt}
  \begin{tabular}{c}
    \textbf{ELEMENT}\\
    \textbf{NEGATION}\\
    \textbf{DIFFERENCE}\\
    \textbf{TRUE}\\
    \textbf{OPERATOR}\\
    \textbf{EQUIVALENT}\\
    \textbf{IMPLICATION}\\
    \textbf{UNION}\\
    \textbf{PROPOSITION}\\
    \textbf{FALSE}\\
    \textbf{INTERSECTION}\\
  \end{tabular}
\end{paracol}

\end{document}
