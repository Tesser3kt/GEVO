\documentclass[a4paper,11pt]{article}

\usepackage[czech,english]{babel}
% Fonts %
\usepackage{fouriernc}
\usepackage[T1]{fontenc}

\usepackage{advdate}

% Colors %
\usepackage[dvipsnames]{color}
\usepackage{xcolor}

% Page Layout %
\usepackage[margin=1in]{geometry}
\usepackage{paracol}

% Fancy Headers %
\usepackage{fancyhdr}
\fancyhf{}
\rhead{\sffamily\large Vocabulary Exam -- 2.AB PreIB Math}
\lhead{\sffamily\large \AdvanceDate[1]\today}
\renewcommand{\headrulewidth}{0.4pt}
\setlength{\headheight}{16pt}

% Math
\usepackage{mathtools}
\usepackage{amssymb}
\usepackage{faktor}
\usepackage{import}
\usepackage{caption}
\usepackage{subcaption}
\usepackage{wrapfig}
\usepackage{enumitem}
\usepackage{tikz}
\usetikzlibrary{cd,positioning,babel,shapes}
\usepackage{tkz-base}
\usepackage{tkz-euclide}

% Theorems
\usepackage{thmtools}
\usepackage[thmmarks, amsmath, thref]{ntheorem}

% Title %
\title{\Huge\textsf{Vocabulary Exam -- 2.AB PreIB Math}\\
  \Large\textsf{Logic \& Set Theory}
  \author{Áďa Klepáčů}
  \date{\today}
}

% Table of Contents %
\usepackage{hyperref}
\hypersetup{
  colorlinks=true,
  linktoc=all,
  linkcolor=blue
}

% Tables %
\usepackage{booktabs}
\usepackage{tabularx}

% Patch for hyphens
\usepackage{regexpatch}
\makeatletter
% Change the `-` delimiter to an active character
\xpatchparametertext\@@@cmidrule{-}{\cA-}{}{}
\xpatchparametertext\@cline{-}{\cA-}{}{}
\makeatother

\newcolumntype{s}{>{\centering\arraybackslash}p{.4\textwidth}}

% Operators %
\DeclareMathOperator{\Ker}{Ker}
\DeclareMathOperator{\Img}{Im}
\DeclareMathOperator{\End}{End}
\DeclareMathOperator{\Aut}{Aut}
\DeclareMathOperator{\Inn}{Inn}

% Common operators %
\newcommand{\R}{\mathbb{R}}
\newcommand{\N}{\mathbb{N}}
\newcommand{\Z}{\mathbb{Z}}
\newcommand{\Q}{\mathbb{Q}}
\newcommand{\C}{\mathbb{C}}

\newcommand{\tr}{\textcolor{red}}
\newcommand{\tb}{\textcolor{blue}}
\newcommand{\tg}{\textcolor{green}}
\newcommand{\tm}{\textcolor{magenta}}
\newcommand{\tv}{\textcolor{violet}}

% American Paragraph Skip %
\setlength{\parindent}{0pt}
\setlength{\parskip}{1em}

% Document %
\pagestyle{fancy}
\renewcommand{\baselinestretch}{1.2}
\begin{document}

\thispagestyle{fancy}

\columnratio{.7}
\begin{paracol}{2}
  Logical \underline{\hspace{12ex}} are sentences that can be either
  \underline{\hspace{12ex}}, or \underline{\hspace{12ex}}. If two such sentences
  are both \underline{\hspace{12ex}} or \underline{\hspace{12ex}} under any
  condition, we say that they are logically \underline{\hspace{12ex}}. If $p$
  and $q$ are \underline{\hspace{12ex}}, the expression $p \Rightarrow q$ is
  called an \underline{\hspace{12ex}}. The symbol $ \Rightarrow $ is an example
  of a logical \underline{\hspace{12ex}}. Another example is $\neg $, which
  reverses truth value and is called \underline{\hspace{12ex}}.

  Let $A$ and $B$ be sets. We can combine $A$ and $B$ using \emph{set
  operations} which arise by applying logical \underline{\hspace{12ex}} to the
  \underline{\hspace{12ex}} ${x \in A}$ and $x \in B$. The expression ${x \in
  A}$ is read as `$x$ is an \underline{\hspace{12ex}} of $A$'. Applying the
  logical `and', that is, taking all $x$ satisfying $x \in A \wedge x \in B$,
  results in a set of all $x$ that are common to both $A$ and $B$. Such set is
  called the \underline{\hspace{12ex}} of $A$ and $B$. If we instead consider
  all $x$ that lie in $A$ \emph{or} in $B$, we obtain a set called the
  \underline{\hspace{12ex}} of $A$ and $B$. Finally, the set which contains all
  \underline{\hspace{12ex}} contained in $A$ but \textbf{not} in $B$, is denoted
  $A \setminus B$ and called the \underline{\hspace{12ex}} of $A$ and $B$.
  \switchcolumn
  \renewcommand{\arraystretch}{2}
  \begin{tabular}{c}
    \textbf{NEGATION}\\
    \textbf{INTERSECTION}\\
    \textbf{TRUE}\\
    \textbf{UNION}\\
    \textbf{ELEMENT}\\
    \textbf{OPERATOR}\\
    \textbf{PROPOSITION}\\
    \textbf{IMPLICATION}\\
    \textbf{FALSE}\\
    \textbf{DIFFERENCE}\\
    \textbf{EQUIVALENT}\\
  \end{tabular}
\end{paracol}

\end{document}
