% Unofficial University of Cambridge Poster Template
% https://github.com/andiac/gemini-cam
% a fork of https://github.com/anishathalye/gemini
% also refer to https://github.com/k4rtik/uchicago-poster
% TeX program = lualatex

\documentclass[final]{beamer}

% ====================
% Packages
% ====================

\usepackage[T1]{fontenc}
\usepackage{lmodern}
\usepackage[orientation=portrait,size=custom,width=120,height=100,scale=1]{beamerposter}
\usetheme{gemini}
\usepackage[dvipsnames]{xcolor}
\usecolortheme{nott}
\usepackage{graphicx}
\usepackage{booktabs}
\usepackage{tikz}
\usetikzlibrary{calc,arrows.meta,patterns,decorations.pathmorphing,shapes.geometric}
\usepackage{tkz-euclide}
\tikzset{point style/.style = {%
    draw = black,
    inner sep = 0pt,
    shape = circle,
    minimum size = 5pt,
    fill = black
  },
  every picture/.append style = {
    scale = 1.5
  },
  every node/.append style={
    scale=1.5
  }
}
\usepackage{pgfplots}
\pgfplotsset{compat=1.14}
\usepackage{anyfontsize}
\usepackage{caption}
\usepackage{subcaption}

% ====================
% Lengths
% ====================

% If you have N columns, choose \sepwidth and \colwidth such that
% (N+1)*\sepwidth + N*\colwidth = \paperwidth
\newlength{\sepwidth}
\newlength{\colwidth}
\setlength{\sepwidth}{0.01\paperwidth}
\setlength{\colwidth}{0.32\paperwidth}

\newcommand{\separatorcolumn}{
  \begin{column}{\sepwidth}
\end{column}}
\newcommand{\bfalert}[1]{\textbf{\alert{#1}}}

% Math shortcuts
\newcommand{\R}{\mathbb{R}}

% Inline shapes
\newcommand{\mysquare}{\tikz[baseline=-7pt]{%
    \node[rectangle,draw,thick,inner sep=6pt] at (0,0) {};
}}
\newcommand{\mytria}{\tikz[baseline=-3.25pt]{%
    \node[isosceles triangle,isosceles triangle apex angle=60,draw,thick,inner
    sep=3.25pt,rotate=90] at (0,0) {};
}}
\newcommand{\mycirc}{\tikz[baseline=-7pt]{%
    \node[circle,draw,thick,inner sep=4.5pt,baseline=0.5ex,rotate=90]
    at (0,0) {};
}}
\newcommand{\mycross}{\tikz[baseline=-7pt,scale=0.2]{%
    \draw[thick] (-1,1) -- (1,-1);
    \draw[thick] (-1,-1) -- (1,1);
}}

% Colors %
\newcommand{\clr}{\textcolor{BrickRed}}
\newcommand{\clb}{\textcolor{RoyalBlue}}
\newcommand{\clg}{\textcolor{ForestGreen}}
\newcommand{\clm}{\textcolor{Fuchsia}}
\newcommand{\clv}{\textcolor{violet}}
\newcommand{\clbr}{\textcolor{Sepia}}
\newcommand{\cly}{\textcolor{Dandelion}}

% ====================
% Title
% ====================

\title{Logic \& Set Theory Cheatsheet}

\author{2.AB PreIB Math}

\institute[shortinst]{Adam Klepáč}

% ====================
% Footer (optional)
% ====================

% \footercontent{
%   \href{https://utfpr.edu.br/ct/ppgca}{utfpr.edu.br/ct/ppgca} \hfill
%   Mostra de Trabalhos do PPGCA --- TechTalks 2024 \hfill
%   \href{mailto:ppgca-ct@utfpr.edu.br}{ppgca-ct@utfpr.edu.br}}
% (can be left out to remove footer)

% ====================
% Logo (optional)
% ====================

% use this to include logos on the left and/or right side of the header:
\logoright{\includegraphics[height=3.5cm]{logos/logo-white.png}}
% \logoleft{\hspace{20ex}\includegraphics[height=3.5cm]{logos/ppgca-logo.png}}

% ====================
% Body
% ====================

\begin{document}

% Refer to https://github.com/k4rtik/uchicago-poster
% logo: https://www.cam.ac.uk/brand-resources/about-the-logo/logo-downloads
% \addtobeamertemplate{headline}{}
% {
%     \begin{tikzpicture}[remember picture,overlay]
%       \node [anchor=north west, inner sep=3cm] at
% ([xshift=-2.5cm,yshift=1.75cm]current page.north west)
%       {\includegraphics[height=7cm]{logos/unott-logo.eps}};
%     \end{tikzpicture}
% }

\begin{frame}[t]
  \begin{columns}[t]
    \separatorcolumn

    \begin{column}{\colwidth}

      \begin{block}{Logic}
        \alert{Logic} is the language of mathematics. It uses
        \alert{propositions} to
        talk about sets.

        Propositions are sentences which can be either true or false.
        For example
        \begin{itemize}[label=\textbullet,left=24pt]
          \item `\textbf{Cats are black.}' is a proposition;
          \item `\textbf{How are you?}' is \emph{not} a proposition;
          \item `\textbf{We will have colonised Mars by 2500.}' is
            also a proposition.
        \end{itemize}
      \end{block}
      As the third example suggests, we need not necessarily know whether a
      proposition is true or false -- it remains a proposition anyway.

      \vspace{1em}

      \begin{exampleblock}{Logical Operators}
        Propositions can be transformed using \alert{logical
        operators}. They pretty
        much correspond to the conjunctions of natural language. Let
        us consider two
        propositions:
        \begin{align*}
          p &= \text{`It's raining outside.'}\\
          q &= \text{`I'll stay at home.'}
        \end{align*}
        \begin{itemize}[left=40pt]
          \item[($ \wedge $)] Logical \alert{and} forms a proposition that is
            only \alert{true} if both of its constituents are also \alert{true}.
            In natural language, the proposition $p \wedge q$ can be expressed
            as
            \[
              p \alert{ \wedge } q = \text{`It's raining outside
                \alert{and} I'll stay at
              home.'}
            \]
          \item[($ \vee $)] Logical \alert{or} forms a proposition
            that is \alert{true}
            if at least one of its constituents is \alert{true}. In natural
            language, the proposition $p \vee q$ can be expressed as
            \[
              p \alert{ \vee } q = \text{`It's raining outside
                \alert{or} I'll stay at
              home.'}
            \]
            In mathematical logic, \alert{or} is \textbf{not
            exclusive}! This means that
            $p \alert{ \vee } q$ is true even if both $p$ and $q$ are true.
          \item[($\neg $)] Logical \alert{not} reverses the truth value of a
            proposition. For example, the proposition $\alert{\neg }p$ can be
            read as
            \[
              \alert{\neg }p = \text{`It's \alert{not} raining outside.'}
            \]
            It follows that $\alert{\neg }p$ is \alert{true} exactly when $p$ is
            \alert{false} and vice versa.
          \item[($ \Rightarrow $)] Logical \alert{implication} is an operator
            that makes the first proposition into an \emph{assumption} or
            \emph{premise} and the second one into a \emph{conclusion}. The
            proposition $p \alert{ \Rightarrow } q$ is read in multiple ways, to
            list a few:
            \begin{align*}
              p \alert{ \Rightarrow } q &= \text{`\alert{If} it's raining
              outside, \alert{then} I'll stay at home.'}\\
              p \alert{ \Rightarrow } q &= \text{`It raining outside
              \alert{implies that} I'll stay at home.'}\\
              p \alert{ \Rightarrow } q &= \text{`\alert{Assuming} it's raining
              outside, I'll stay at home.'}\\
            \end{align*}
            The implication is tricky. It's true if both $p$ and $q$ are true
            and false if $p$ is true but $q$ is false. However, it is
            \alert{always true} if $p$ is \alert{false}. That is because, in
            mathematical logic, whatever follows from a lie is automatically
            true.
          \item[($ \Leftrightarrow $)] Logical \alert{equivalence} is true only
            if both propositions have the \alert{same truth value} -- they're
            both true or both false. In natural language, it is typically read
            like this:
            \[
              p \alert{ \Leftrightarrow }q = \text{`It's raining
                \alert{if and only if}
              I stay at home.'}
            \]
            Equivalence is basically just a two-way implication. The
            proposition $p$ is
            both a premise and a conclusion to $q$ and $q$ is both a
            premise and a
            conclusion to $p$. If it's raining outside, I stay at
            home and if I stay at
            home, then it's raining outside.
        \end{itemize}
      \end{exampleblock}

      \begin{block}{Truth Tables}
        A proposition made up of smaller propositions is true or false based on
        whether its constituent propositions are true or false. All possible
        scenarios can be summarized using so-called \alert{truth table}. It is
        basically just a table that lists all the possibilities of $p$ and $q$
        (or any number of propositions, really) being true or false and the
        resulting truth value of their combinations.

        For the basic logical conjunctions from above, it can look like this (we
        represent \alert{true} by \alert{1} and \alert{false} by \alert{0}):
        \begin{center}
          \begin{tabular}{c | c | c | c | c | c | c | c}
            $p$ & $q$ & $\neg p$ & $\neg q$ & $p \wedge q$ & $p \vee
            q$ & $p \Rightarrow
            q$ & $p \Leftrightarrow q$\\
            \toprule
            0 & 0 & 1 & 1 & 0 & 0 & 1 & 1\\
            \midrule
            0 & 1 & 1 & 0 & 0 & 1 & 1 & 0\\
            \midrule
            1 & 0 & 0 & 1 & 0 & 1 & 0 & 0\\
            \midrule
            1 & 1 & 0 & 0 & 1 & 1 & 1 & 1
          \end{tabular}
        \end{center}
      \end{block}
    \end{column}

    \separatorcolumn

    \begin{column}{\colwidth}

      \begin{exampleblock}{Sets}
        \alert{Sets} are the `stuff' that makes up the world of
        mathematics. Their
        basic characteristics and properties are described using \alert{logic}.

        Sets cannot be defined inside set theory but we interpret
        them as \emph{groups
        of things}.

        There's only one foundational \emph{proposition} related to set theory
        -- the proposition `\alert{An object is an element of a set.}' If we
        label the object in question, $x$, and the set, $A$, this proposition is
        written as $x \in A$ (the symbol $ \in $ is just the letter `e' in
        `element'). Combining these propositions using logical conjunctions
        allows for various set-theoretic constructions.

        If a set $A$ has, for example, exactly three elements --
        $\mysquare$, $\mytria$
        and $\mycirc$, I can write it as a list of these three
        elements inside curly
        brackets $\{\}$. In this case,
        \[
          A = \{\mysquare,\mytria,\mycirc\}.
        \]

        Two \alert{warnings} about sets:
        \begin{itemize}[label=\textbullet,left=24pt]
          \item \textbf{Sets are not ordered}. There is nothing like
            a `first', `second'
            or `last' element of a set. Either an object \textbf{is}
            inside a set or it
            \textbf{isn't}. Nothing else. For example, the three sets below are
            \alert{exactly the same}, only written differently.
            \[
              \{\mysquare,\mytria,\mycirc\} = \{\mycirc,\mytria,\mysquare\} =
              \{\mytria,\mysquare,\mycirc\}
            \]
          \item \textbf{Elements of sets have no frequency}. Again,
            an element either is
            inside a set or not. It cannot be \alert{twice} in a set,
            for example. The
            three sets below are exactly the same.
            \[
              \{\mysquare,\mytria,\mycirc\} =
              \{\mysquare,\mytria,\mycirc,\mytria,\mycirc\} = \{
              \mytria,\mysquare,\mysquare,\mytria,\mycirc,\mytria\}
            \]
        \end{itemize}

        There are various ways of constructing sets. We shall take a look at
        two: \emph{enumeration} and \emph{condition}.

        By \emph{enumeration}, we simply mean that a set is defined by listing
        all its elements. We have already seen this before. The equality $A =
        \{\mysquare,\mytria,\mycirc\}$ is an example of defining a set by
        enumeration.

        A more potent way of creating sets entails using logical propositions.
        Assume $x$ is an object and $p(x)$ is any logical proposition involving
        $x$. For example,
        \begin{align*}
          p(x) &= \text{`$x$ is beautiful.'}\\
          p(x) &= \text{`$x$ is a number.'}
        \end{align*}
        The set $\{x \mid p(x)\}$ is the set of all objects $x$ for which
        $p(x)$ is true. Imagine
        \[
          p(x) = \text{`$x$ is natural number and $x$ is smaller than five.'}.
        \]
        Then,
        \[
          \{x \mid p(x)\} = \{0,1,2,3,4\}.
        \]
      \end{exampleblock}

      \begin{alertblock}{Set Operations}
        Using logical operators, we can form new sets from existing ones or
        establish relations between sets. Consider two sets -- $A$ and $B$.
        \begin{itemize}[left=40pt]
          \item[($ \cap $)] We can form the set $\{x \mid x \in A \wedge x \in
            B\}$, that is, the set of all objects that \alert{lie in both $A$
            and $B$}. This set is called the \alert{intersection} of $A$ and $B$
            and written $A \cap B$. For example,
            \[
              \{\mycirc,\mytria,\mysquare\} \cap
              \{\mycross,\mycirc,\mysquare, \sim \} =
              \{\mycirc,\mysquare\}.
            \]
          \item[($ \cup $)] We can form the set $\{x \mid x \in A \vee x \in
            B\}$, i.e. the set of all objects that \alert{lie in $A$ or in $B$}.
            It is called the \alert{union} of $A$ and $B$ and denoted $A \cup
            B$. All elements of $A \cup B$ can be found \emph{only} in $A$,
            \emph{only} in B or in \emph{both} $A$ and $B$. For example,
            \[
              \{\mycirc,\mytria,\mysquare\} \cup
              \{\mycross,\mycirc,\mysquare, \sim \} =
              \{\mycirc,\mytria,\mysquare,\mycross, \sim \}.
            \]
          \item[($ \neg $)] Negation itself doesn't really do much but we can
            combine it with conjunction $( \wedge )$ to form the
            \alert{difference} of two sets. The set $\{x \mid x \in A \wedge
            \neg (x \in B)\} = \{x \mid x \in A \wedge x \notin B\}$ is the set
            of all elements that belong to $A$ but do \alert{not} belong to $B$
            and is denoted $A \setminus B$. For example,
            \[
              \{\mycirc,\mytria,\mysquare\} \setminus
              \{\mycross,\mycirc,\mysquare, \sim \} = \{\mytria\}.
            \]
            \textbf{Beware:} the difference (unlike union and intersection) is
            not commutative, meaning that $A \setminus B \neq B \setminus A$. In
            the example above,
            \[
              \{\mycross,\mycirc,\mysquare, \sim \} \setminus
              \{\mycirc,\mytria,\mysquare\}= \{\mycross, \sim\}.
            \]
          \item[($ \Rightarrow $)] Implication is a little different
            from intersection
            and union. It describes a lot of different sets with one
            logical proposition.
            I ask: `Which sets $A$ satisfy the proposition $x \in A
            \Rightarrow x \in
            B$?' In other words, which sets $A$ \alert{have all their
            elements contained}
            in the set $B$? The answer is that $A$ must be a subset
            of $B$ and we denote
            that fact by $A \subseteq B$. The set $A$ is only allowed
            to have elements
            which also lie in $B$ but not necessarily all of them.
            All the subsets of $B
            = \{\mytria,\mycirc\}$ are listed below.
            \[
              \{\}, \{\mytria\}, \{\mycirc\}, \{\mytria,\mycirc\},
            \]
            where $\{\}$ is the \alert{empty set}, a set
            containing no elements.
          \item[($ \Leftrightarrow $)] Equivalence defines
            \alert{equality} on sets. If
            sets $A$ and $B$ must satisfy the proposition $x \in A
            \Leftrightarrow x \in
            B$, then they must be equal because all the elements of
            $A$ lie in $B$ and
            all elements of $B$ lie in $A$. That is, $A = B$.
        \end{itemize}
      \end{alertblock}

    \end{column}
    \separatorcolumn

    \begin{column}{\colwidth}
      \begin{block}{Drawing Sets}
        Set operations can be visualized using so-called \emph{Venn
        diagrams}. This
        just means using circles to represent the sets in question.
        For example, two
        sets -- $\clr{A}$ and $\clb{B}$ -- can be drawn like this:
        \begin{center}
          \begin{tikzpicture}[scale=0.5]
            \fill[BrickRed,fill opacity=0.25] (180:1.5) circle (2);
            \fill[RoyalBlue,fill opacity=0.25] (0:1.5) circle (2);
            \draw[BrickRed,thick] (180:1.5) circle (2);
            \draw[RoyalBlue,thick] (0:1.5) circle (2);
            \node at (150:4.3) {\footnotesize $\clr{A}$};
            \node at (30:4.3) {\footnotesize $\clb{B}$};
          \end{tikzpicture}
        \end{center}
        In these pictures, one can easily visualize the operations of union,
        intersection and difference. The union $\clg{A \cup B}$ is the entire
        area covered by $\clr{A}$ and $\clb{B}$. It looks like this:
        \begin{center}
          \begin{tikzpicture}[scale=0.5]
            \def\firstcircle{(180:1.5) circle (2)}
            \def\secondcircle{(0:1.5) circle (2)}
            \filldraw[thick,ForestGreen] \firstcircle;
            \filldraw[thick,ForestGreen] \secondcircle;
            \node at (90:3) {\footnotesize $\clg{A \cup B}$};
          \end{tikzpicture}
        \end{center}
        The intersection $\clm{A \cap B}$ is the `strip' in the
        middle, the area which
        is shared between both $\clr{A}$ and $\clb{B}$. It can be
        depicted like this:
        \begin{center}
          \begin{tikzpicture}[scale=0.5]
            \def\firstcircle{(180:1.5) circle (2)}
            \def\secondcircle{(0:1.5) circle (2)}

            \fill[BrickRed,fill opacity=0.25] \firstcircle;
            \fill[RoyalBlue,fill opacity=0.25] \secondcircle;
            \draw[BrickRed,thick] \firstcircle;
            \draw[RoyalBlue,thick] \secondcircle;

            \begin{scope}
              \clip \firstcircle;
              \fill[white] \secondcircle;
              \filldraw[thick,Fuchsia] \secondcircle;
            \end{scope}
            \begin{scope}
              \clip \secondcircle;
              \draw[Fuchsia,thick] \firstcircle;
            \end{scope}
            \node at (90:3) {\footnotesize $\clm{A \cap B}$};
          \end{tikzpicture}
        \end{center}
        The difference $\clbr{A \setminus B}$ is the area of the red circle
        which doesn't lie in the blue circle. The difference $\cly{B \setminus
        A}$ is the mirror version of that.
        \begin{center}
          \begin{tikzpicture}[scale=0.5]
            \def\firstcircle{(180:1.5) circle (2)}
            \def\secondcircle{(0:1.5) circle (2)}

            \fill[Sepia] \firstcircle;
            \fill[white] \secondcircle;
            \fill[RoyalBlue,fill opacity=0.25] \secondcircle;
            \draw[RoyalBlue,thick] \secondcircle;
            \node at (90:3) {\footnotesize $\clbr{A \setminus B}$};
          \end{tikzpicture}
          \begin{tikzpicture}[scale=0.5,xshift=10cm]
            \def\firstcircle{(180:1.5) circle (2)}
            \def\secondcircle{(0:1.5) circle (2)}

            \fill[Dandelion] \secondcircle;

            \fill[white] \firstcircle;
            \fill[BrickRed,fill opacity=0.25] \firstcircle;
            \draw[BrickRed,thick] \firstcircle;
            \node at (90:3) {\footnotesize $\cly{B \setminus A}$};
          \end{tikzpicture}
        \end{center}

        Adding a third set $\clg{C}$ results in a picture like this.
        \begin{center}
          \begin{tikzpicture}[scale=0.5]
            \fill[BrickRed,fill opacity=0.25] (135:1.5) circle (2);
            \fill[RoyalBlue,fill opacity=0.25] (45:1.5) circle (2);
            \fill[ForestGreen,fill opacity=0.25] (270:1.5) circle (2);
            \draw[BrickRed,thick] (135:1.5) circle (2);
            \draw[RoyalBlue,thick] (45:1.5) circle (2);
            \draw[ForestGreen,thick] (270:1.5) circle (2);
            \node at (150:4.3) {\footnotesize $\clr{A}$};
            \node at (30:4.3) {\footnotesize $\clb{B}$};
            \node at (270:4.3) {\footnotesize $\clg{C}$};
          \end{tikzpicture}
        \end{center}
        Every area in the picture corresponds to a different combination of set
        operations. For example, the isolated part of the \clr{red} circle
        consists of elements found in $\clr{A}$ but not in $\clb{B}$ or
        $\clg{C}$. This set is the result of the operation $(\clr{A} \setminus
        \clb{B}) \setminus \clg{C}$. The area in the middle where all circles
        intersect represents the set $\clr{A} \cap \clb{B} \cap \clg{C}$.

        Let us see this precisely. We choose the sets
        \[
          \clr{A} = \clr{\{2, 3, 5, 6\}}, \quad \clb{B} = \clb{\{1, 2, 3, 4,
          7\}}, \quad \clg{C} = \clg{\{2, 4, 5, 8\}}.
        \]
        If we place the numbers into correct areas of the diagram, we get this
        picture.
        \begin{center}
          \begin{tikzpicture}[scale=1]
            \fill[BrickRed,fill opacity=0.25] (135:1.5) circle (2);
            \fill[RoyalBlue,fill opacity=0.25] (45:1.5) circle (2);
            \fill[ForestGreen,fill opacity=0.25] (270:1.5) circle (2);
            \draw[BrickRed,thick] (135:1.5) circle (2);
            \draw[RoyalBlue,thick] (45:1.5) circle (2);
            \draw[ForestGreen,thick] (270:1.5) circle (2);
            \node at (150:4.3) {\footnotesize $\clr{A}$};
            \node at (30:4.3) {\footnotesize $\clb{B}$};
            \node at (270:4.3) {\footnotesize $\clg{C}$};

            \node at (135:2.5) {$6$};
            \node at (45:2.5) {$1,7$};
            \node at (270:2.5) {$8$};

            \node at (90:2) {$3$};
            \node at (202.5:1.1) {$5$};
            \node at (337.5:1.1) {$4$};

            \node at (0,0) {$2$};
          \end{tikzpicture}
        \end{center}
        As our final example, we draw two areas defined by two different
        combinations of set operations.
        \begin{center}
          \begin{tikzpicture}[scale=0.5]
            \def\acircle{(135:1.5) circle (2)}
            \def\bcircle{(45:1.5) circle (2)}
            \def\ccircle{(270:1.5) circle (2)}
            \fill[BrickRed,fill opacity=0.25] \acircle;
            \fill[RoyalBlue,fill opacity=0.25] \bcircle;
            \fill[ForestGreen,fill opacity=0.25] \ccircle;
            \draw[BrickRed,thick] \acircle;
            \draw[RoyalBlue,thick] \bcircle;
            \draw[ForestGreen,thick] \ccircle;

            \begin{scope}
              \clip \acircle;
              \filldraw[Fuchsia] \bcircle;
            \end{scope}
            \begin{scope}
              \clip \bcircle;
              \filldraw[Fuchsia] \ccircle;
            \end{scope}

            \node at (150:4.3) {\footnotesize $\clr{A}$};
            \node at (30:4.3) {\footnotesize $\clb{B}$};
            \node at (270:4.3) {\footnotesize $\clg{C}$};

            \node at (90:5) {\footnotesize $\clm{(A \cap B) \cup (B \cap C)}$};
          \end{tikzpicture}
          \begin{tikzpicture}[scale=0.5,xshift=15cm]
            \def\acircle{(135:1.5) circle (2)}
            \def\bcircle{(45:1.5) circle (2)}
            \def\ccircle{(270:1.5) circle (2)}
            \fill[BrickRed,fill opacity=0.25] \acircle;
            \fill[RoyalBlue,fill opacity=0.25] \bcircle;
            \draw[BrickRed,thick] \acircle;
            \draw[RoyalBlue,thick] \bcircle;

            \filldraw[Fuchsia] \acircle;
            \filldraw[Fuchsia] \bcircle;
            \fill[white] \ccircle;
            \fill[ForestGreen,fill opacity=0.25] \ccircle;
            \draw[ForestGreen,thick] \ccircle;

            \node at (150:4.3) {\footnotesize $\clr{A}$};
            \node at (30:4.3) {\footnotesize $\clb{B}$};
            \node at (270:4.3) {\footnotesize $\clg{C}$};

            \node at (90:5) {\footnotesize $\clm{(A \cup B) \setminus C}$};
          \end{tikzpicture}
        \end{center}
      \end{block}

    \end{column}
    \separatorcolumn
  \end{columns}
\end{frame}

\end{document}
