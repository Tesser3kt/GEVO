% Unofficial University of Cambridge Poster Template
% https://github.com/andiac/gemini-cam
% a fork of https://github.com/anishathalye/gemini
% also refer to https://github.com/k4rtik/uchicago-poster
% TeX program = lualatex

\documentclass[final]{beamer}

% ====================
% Packages
% ====================

\usepackage[T1]{fontenc}
\usepackage{lmodern}
\usepackage[czech,english]{babel}
\usepackage[orientation=portrait,size=custom,width=120,height=100,scale=1]{beamerposter}
\usetheme{gemini}
\usepackage[dvipsnames]{xcolor}
\usecolortheme{nott}
\usepackage{graphicx}
\usepackage{booktabs}
\usepackage{tikz}
\usetikzlibrary{calc,arrows.meta,patterns,decorations.pathmorphing,shapes.geometric}
\usepackage{tkz-euclide}
\tikzset{point style/.style = {%
    draw = black,
    inner sep = 0pt,
    shape = circle,
    minimum size = 5pt,
    fill = black
  },
  every picture/.append style = {
    scale = 1.5
  },
  every node/.append style={
    scale=1.5
  }
}
\usepackage{pgfplots}
\pgfplotsset{compat=1.14}
\usepackage{anyfontsize}
\usepackage{caption}
\usepackage{subcaption}

% ====================
% Lengths
% ====================

% If you have N columns, choose \sepwidth and \colwidth such that
% (N+1)*\sepwidth + N*\colwidth = \paperwidth
\newlength{\sepwidth}
\newlength{\colwidth}
\setlength{\sepwidth}{0.01\paperwidth}
\setlength{\colwidth}{0.32\paperwidth}

\newcommand{\separatorcolumn}{
  \begin{column}{\sepwidth}
\end{column}}
\newcommand{\bfalert}[1]{\textbf{\alert{#1}}}

% Math shortcuts
\newcommand{\R}{\mathbb{R}}

% Inline shapes
\newcommand{\mysquare}{\tikz[baseline=-7pt]{%
    \node[rectangle,draw,thick,inner sep=6pt] at (0,0) {};
}}
\newcommand{\mytria}{\tikz[baseline=-3.25pt]{%
    \node[isosceles triangle,isosceles triangle apex angle=60,draw,thick,inner
    sep=3.25pt,rotate=90] at (0,0) {};
}}
\newcommand{\mycirc}{\tikz[baseline=-7pt]{%
    \node[circle,draw,thick,inner sep=4.5pt,baseline=0.5ex,rotate=90]
    at (0,0) {};
}}
\newcommand{\mycross}{\tikz[baseline=-7pt,scale=0.2]{%
    \draw[thick] (-1,1) -- (1,-1);
    \draw[thick] (-1,-1) -- (1,1);
}}

% Colors %
\newcommand{\clr}{\textcolor{BrickRed}}
\newcommand{\clb}{\textcolor{RoyalBlue}}
\newcommand{\clg}{\textcolor{ForestGreen}}
\newcommand{\clm}{\textcolor{Fuchsia}}
\newcommand{\clv}{\textcolor{violet}}
\newcommand{\clbr}{\textcolor{Sepia}}
\newcommand{\cly}{\textcolor{Dandelion}}

% ====================
% Title
% ====================

\title{Cheatsheet z logiky a teorie množin}

\author{2.AB PreIB Math}

\institute[shortinst]{Adam Klepáč}

% ====================
% Footer (optional)
% ====================

% \footercontent{
%   \href{https://utfpr.edu.br/ct/ppgca}{utfpr.edu.br/ct/ppgca} \hfill
%   Mostra de Trabalhos do PPGCA --- TechTalks 2024 \hfill
%   \href{mailto:ppgca-ct@utfpr.edu.br}{ppgca-ct@utfpr.edu.br}}
% (can be left out to remove footer)

% ====================
% Logo (optional)
% ====================

% use this to include logos on the left and/or right side of the header:
\logoright{\includegraphics[height=3.5cm]{logos/logo-white.png}}
% \logoleft{\hspace{20ex}\includegraphics[height=3.5cm]{logos/ppgca-logo.png}}

% ====================
% Body
% ====================

\begin{document}

% Refer to https://github.com/k4rtik/uchicago-poster
% logo: https://www.cam.ac.uk/brand-resources/about-the-logo/logo-downloads
% \addtobeamertemplate{headline}{}
% {
%     \begin{tikzpicture}[remember picture,overlay]
%       \node [anchor=north west, inner sep=3cm] at
% ([xshift=-2.5cm,yshift=1.75cm]current page.north west)
%       {\includegraphics[height=7cm]{logos/unott-logo.eps}};
%     \end{tikzpicture}
% }

\begin{frame}[t]
  \begin{columns}[t]
    \separatorcolumn

    \begin{column}{\colwidth}

      \begin{block}{Logika}
        \alert{Logika} je jazykem matematiky. Využívá \alert{výroků}, aby
        mluvila o množinách.

        Výroky jsou věty, které jsou buď pravdivé, nebo lživé.

        Například,
        \begin{itemize}[label=\textbullet,left=24pt]
          \item `\textbf{Kočky jsou černé.}' je výrok.;
          \item `\textbf{Jak se máš?}' \emph{není} výrok;
          \item `\textbf{Do roku 2500 kolonisujeme Mars.}' je taky výrok.
        \end{itemize}
      \end{block}
      Jak třetí příklad napovídá, nemusíme nutně vědět, jestli je nějaký výrok
      pravdivý, nebo ne. Výrok je to stále.

      \vspace{1em}

      \begin{exampleblock}{Logické operátory}
        Výroky umíme přetvářet a slučovat pomocí \alert{logických operátorů}.
        Svým významem víceméně odpovídají spojkám v běžném jazyce. Uvažme dva
        výroky:
        \begin{align*}
          p &= \text{`Venku prší.'}\\
          q &= \text{`Zůstanu doma.'}
        \end{align*}
        \begin{itemize}[left=40pt]
          \item[($\wedge$)] Logické \alert{a} slučuje dva výroky do výroku,
            které je pravda jedině ve chvíli, kdy jsou oba dílčí výroky rovněž
            pravdivé. Přirozeně můžeme výrok $p \alert{ \wedge } q$ vyjádřit větou
            \[
              p \alert{ \wedge } q = \text{`Venku prší \alert{a} zůstanu doma.'}
            \]
          \item[($ \vee $)] Logické \alert{nebo} slučuje dva výroky do výroku,
            který je pravdivý, když je aspoň jeden z dílčích výroků rovněž
            pravdivý. Přirozeně můžeme výrok $p \alert{ \vee } q$ vyjádřit větou
            \[
              p \alert{ \vee } q = \text{`Venku prší \alert{nebo} zůstanu
              doma.'}
            \]
            V matematické logice \textbf{není} \alert{nebo} \textbf{výlučné}! To
            znamená, že $p \vee q$ je pravda, i když oba výroky $p$ i $q$ jsou
            pravdivé.
          \item[($\neg $)] 
            Logické \alert{ne} obrací pravdivostní hodnotu výroku. V jazyce ji
            můžeme vyjádřit prostým záporem:
            \[
              \alert{\neg }p = \text{`Venku \alert{ne}prší.'}
            \]
            Uvědomme si, že $\alert{\neg }p$ je \alert{pravda} přesně ve chvíli,
            kdy $p$ je \alert{lež} a naopak.
          \item[($ \Rightarrow $)] 
            Logická \alert{implikace} je operátor, který z prvního výroku dělá
            \emph{předpoklad} a z druhého \emph{závěr}. Výrok $p \alert{
            \Rightarrow } q$ se čte mnoha způsoby. Například:
            \begin{align*}
              p \alert{ \Rightarrow } q &= \text{`\alert{Když} venku prší,
              \alert{tak} zůstanu doma.'}\\
              p \alert{ \Rightarrow } q &= \text{`To, že venku prší,
              \alert{implikuje}, že zůstanu doma.'}\\
              p \alert{ \Rightarrow } q &= \text{`\alert{Za předpokladu}, že
                venku prší, zůstanu doma.'}
            \end{align*}
            Implikace umí být zákeřná. Je pravdivá, když $p$ i $q$ jsou pravdivé
            a lživá, když $p$ je pravda, ale $q$ ne. Ovšem, implikace je taky
            \alert{vždy pravdivá}, když $p$ je \alert{lež}. V matematické logice
            cokoli, co plyne ze lži, je automaticky pravdivé.
          \item[($ \Leftrightarrow $)] 
            Logická \alert{ekvivalence} je pravdivá, jedině když jsou buď oba
            výroky naráz pravdivé, nebo naráz lživé. V jazyce se většinou čte
            takto:
            \[
              p \alert{ \Leftrightarrow }q = \text{`Prší \alert{právě tehdy,
              když} zůstávám doma.'}
            \]
            Ekvivalence je v zásadě implikace oběma směry. Výrok $p$ je jak
            předpoklad, tak závěr, výroku $q$ a výrok $q$ je jak předpoklad, tak
            závěr, výroku $p$. Když prší, zůstávám doma, a když zůstávám doma,
            tak prší.
        \end{itemize}
      \end{exampleblock}

      \begin{block}{Pravdivostní tabulky}
        Výrok složený z dílčích výroků je pravdivý nebo lživý na základě
        pravdivosti výroku, jež jej tvoří. Všechny možné případy lze shrnout v
        tzv. \alert{pravdivostní tabulce}. Je jí prostě tabulka s výčtem všech
        možných kombinací pravdivostí $p$ a $q$ (nebo vlastně jakéhokoliv
        množství výroků).  

        Pravdivostní tabulka základních logických operátorů vypsaných výše
        vypadá takto (pravdivost výroku označíme číslem \alert{1} a lživost
        číslem \alert{0}).
        \begin{center}
          \begin{tabular}{c | c | c | c | c | c | c | c}
            $p$ & $q$ & $\neg p$ & $\neg q$ & $p \wedge q$ & $p \vee
            q$ & $p \Rightarrow
            q$ & $p \Leftrightarrow q$\\
            \toprule
            0 & 0 & 1 & 1 & 0 & 0 & 1 & 1\\
            \midrule
            0 & 1 & 1 & 0 & 0 & 1 & 1 & 0\\
            \midrule
            1 & 0 & 0 & 1 & 0 & 1 & 0 & 0\\
            \midrule
            1 & 1 & 0 & 0 & 1 & 1 & 1 & 1
          \end{tabular}
        \end{center}
      \end{block}
    \end{column}

    \separatorcolumn

    \begin{column}{\colwidth}

      \begin{exampleblock}{Množiny}
        \alert{Množiny} jsou `hmotou', která tvoří svět matematiky. Jejich
        základní vlastnosti jsou popsány pomocí \alert{logiky}.

        Množiny nelze definovat uvnitř teorie množin, ale obvykle si je
        představujeme jako \emph{skupiny věcí}.

        V teorii množin existuje pouze jediný základní \emph{logický výrok} --
        výrok `\alert{Objekt je prvkem množiny.}' Když označíme zmíněný objekt
        $x$ a množinu $A$, pak tento výrok zapisujeme jako $\alert{x \in A}$
        (symbol $\in$ je jen zkroucené `e' z anglického slova \emph{element},
        \uv{prvek}). Slučování takových výroků pomocí logických operátorů dá
        vzniknout různým konstrukcím v teorii množin.

        Když má množina $A$, například, přesne tři prvky -- $\mysquare$,
        $\mytria$ a $\mycirc$, pak ji můžeme zapsat jako seznam těchto tří prvků
        uvnitř složených závorek $\{\}$. V tomto případě
        \[
          A = \{\mysquare,\mytria,\mycirc\}.
        \]
        
        Dvě výstrahy ohledně množin:
        \begin{itemize}[label=\textbullet,left=24pt]
          \item \textbf{Množiny nejsou uspořádané}. Neexistuje nic jako `první',
            `druhý' nebo `poslední' prvek množiny. Buď nějaký objekt \textbf{je}
            prvkem množiny, nebo \textbf{není}. Nic víc, nic míň. Například,
            tyto tři množiny jsou dokonale stejné, akorát zapsané jinak.
            \[
              \{\mysquare,\mytria,\mycirc\} = \{\mycirc,\mytria,\mysquare\} =
              \{\mytria,\mysquare,\mycirc\}
            \]
          \item \textbf{Prvky množin nemají četnost}. 
            Opět, prvek buď leží v množině, nebo ne. Nemůže v ní ležet dvakrát,
            třikrát nebo milionkrát. Buď jednou, nebo vůbec. Tyto tři množiny
            jsou rovněž dokonale stejné.
            \[
              \{\mysquare,\mytria,\mycirc\} =
              \{\mysquare,\mytria,\mycirc,\mytria,\mycirc\} = \{
              \mytria,\mysquare,\mysquare,\mytria,\mycirc,\mytria\}
            \]
        \end{itemize}

        Existuje více způsobů, jak tvořit množiny. My se podíváme na dva:
        \emph{výčtem prvků} a \emph{podmínkou}.

        \emph{Výčtem} myslíme zkrátka definici množiny přes vypsaní všech jejích
        prvků, jako výše. Rovnost $A = \{\mysquare,\mytria,\mycirc\}$ je
        příkladem tvorby množiny výčtem.

        Mnohem užitečnější způsob tvorby množin užívá logických výroků. Ať $x$
        je nějaký objekt a $p(x)$ libovolný výrok hovořící o $x$. Například,
        \begin{align*}
          p(x) &= \text{`$x$ je nádherný.'}\\
          p(x) &= \text{`$x$ je číslo.'}
        \end{align*}

        Množina $\{x \mid p(x)\}$ je množina všech objektů $x$, pro něž je výrok
        $p(x)$ pravda. Představme si, že
        \[
          p(x) = \text{`$x$ je přirozené číslo, které je menší než $5$'}.
        \]
        Pak,
        \[
          \{x \mid p(x)\} = \{0,1,2,3,4\}.
        \]
      \end{exampleblock}

      \begin{alertblock}{Množinové operace}
        Užitím logických výroků můžeme vytvořit nové množiny z existujících nebo
        stanovit jisté vztahy mezi množinami. Uvažme dvě množiny -- $A$ a $B$.
        \begin{itemize}[left=40pt]
          \item[($ \cap $)] 
            Můžeme vyrobit množinu $\{x \mid x \in A \wedge x \in B\}$, tedy
            množinu všech objektů, které \alert{leží jak v $A$, tak v $B$}. Této
            množině přezdíváme \alert{průnik} $A$ a $B$ a značíme ji $A \cap B$.
            Například,
            \[
              \{\mycirc,\mytria,\mysquare\} \cap
              \{\mycross,\mycirc,\mysquare, \sim \} =
              \{\mycirc,\mysquare\}.
            \]
          \item[($ \cup $)] 
            Můžeme vyrobit množinu $\{x \mid x \in A \vee x \in
            B\}$, tedy množinu všech objektů, které \alert{leží v $A$ nebo v
            $B$}. Říká se jí \alert{sjednocení} $A$ s $B$ a píše se jako $A \cup
            B$. Všechny prvky v $A \cup B$ buď leží \emph{pouze} v $A$,
            \emph{pouze} v $B$, nebo v $A$ i v $B$. Například,
            \[
              \{\mycirc,\mytria,\mysquare\} \cup
              \{\mycross,\mycirc,\mysquare, \sim \} =
              \{\mycirc,\mytria,\mysquare,\mycross, \sim \}.
            \]
          \item[($ \neg $)] 
            S negací samotnou toho moc nesvedeme, ale můžeme ji použít společně
            s konjunkcí ($ \wedge $), abychom vytvořili \alert{rozdíl} dvou
            množin. Množina $\{x \mid x \in A \wedge
            \neg (x \in B)\} = \{x \mid x \in A \wedge x \notin B\}$ obsahuje
            všechny prvky, které leží v $A$, ale \textbf{ne}leží v $B$ a píšeme
            ji $A \setminus B$. Například,
            \[
              \{\mycirc,\mytria,\mysquare\} \setminus
              \{\mycross,\mycirc,\mysquare, \sim \} = \{\mytria\}.
            \]
            \textbf{Pozor:} rozdíl množin (oproti sjednocení a průniku) není
            komutativní, tj. $A \setminus B \neq B \setminus A$. V příkladu výše
            \[
              \{\mycross,\mycirc,\mysquare, \sim \} \setminus
              \{\mycirc,\mytria,\mysquare\}= \{\mycross, \sim\}.
            \]
          \item[($ \Rightarrow $)] 
            Použití implikace v teorii množin se trochu liší od průniku či
            sjednocení. Popisuje mnoho různých množin jediným logickým výrokem.
            Můžeme se ptát: \uv{Které množiny $A$ splňují výrok $x \in A
            \Rightarrow x \in B$?} Jinak řečeno, které množiny $A$ \alert{mají
            všechny své prvky obsaženy} v množině $B$? Odpovědí jsou
            \alert{podmnožiny} $B$.
          
            Fakt, že $A$ je podmnožina $B$ píšeme jako $A \subseteq B$. Množina
            $A$ smí obsahovat pouze prvky, které obsahuje rovněž $B$, ale ne
            nutně všechny. Všechny podmnožiny množiny $B
            = \{\mytria,\mycirc\}$ jsou vypsány níže.
            \[
              \{\}, \{\mytria\}, \{\mycirc\}, \{\mytria,\mycirc\},
            \]
            kde $\{\}$ je \alert{prázdná množina} -- množina bez prvků.
          \item[($ \Leftrightarrow $)] 
            Logická ekvivalence definuje \alert{rovnost} mezi množinami. Když
            množiny $A$ a $B$ splňují výrok ${x \in A \Leftrightarrow x \in B}$,
            pak musejí být stejné, poněvadž všechny prvky $A$ leží též v $B$ a
            všechny prvky $B$ leží též v $A$. Tedy, $A = B$.
        \end{itemize}
      \end{alertblock}

    \end{column}
    \separatorcolumn

    \begin{column}{\colwidth}
      \begin{block}{Kreslení množinových operací}
        Množinové operace lze visualisovat užitím tzv. \emph{Vennových
        diagramů}. Ty spočívají v nakreslení množin pomocí protínajících se
        kruhů. Například, dvě množiny -- $\clr{A}$ a $\clb{B}$ -- lze nakreslit
        takto:
        \begin{center}
          \begin{tikzpicture}[scale=0.5]
            \fill[BrickRed,fill opacity=0.25] (180:1.5) circle (2);
            \fill[RoyalBlue,fill opacity=0.25] (0:1.5) circle (2);
            \draw[BrickRed,thick] (180:1.5) circle (2);
            \draw[RoyalBlue,thick] (0:1.5) circle (2);
            \node at (150:4.3) {\footnotesize $\clr{A}$};
            \node at (30:4.3) {\footnotesize $\clb{B}$};
          \end{tikzpicture}
        \end{center}
        Snadno též nakreslíme operace sjednocení, průniku a rozdílu. Sjednocení
        $\clg{A \cup B}$ je celá oblast pokrytá kruhy $\clr{A}$ a $\clb{B}$.
        Vypadá takto:
        \begin{center}
          \begin{tikzpicture}[scale=0.5]
            \def\firstcircle{(180:1.5) circle (2)}
            \def\secondcircle{(0:1.5) circle (2)}
            \filldraw[thick,ForestGreen] \firstcircle;
            \filldraw[thick,ForestGreen] \secondcircle;
            \node at (90:3) {\footnotesize $\clg{A \cup B}$};
          \end{tikzpicture}
        \end{center}
        Průnik $\clm{A \cap B}$ je ten \uv{proužek} veprostřed -- ta oblast,
        kterou sdílejí kruhy $\clr{A}$ a $\clb{B}$. Můžeme ho vykreslit takto:
        \begin{center}
          \begin{tikzpicture}[scale=0.5]
            \def\firstcircle{(180:1.5) circle (2)}
            \def\secondcircle{(0:1.5) circle (2)}

            \fill[BrickRed,fill opacity=0.25] \firstcircle;
            \fill[RoyalBlue,fill opacity=0.25] \secondcircle;
            \draw[BrickRed,thick] \firstcircle;
            \draw[RoyalBlue,thick] \secondcircle;

            \begin{scope}
              \clip \firstcircle;
              \fill[white] \secondcircle;
              \filldraw[thick,Fuchsia] \secondcircle;
            \end{scope}
            \begin{scope}
              \clip \secondcircle;
              \draw[Fuchsia,thick] \firstcircle;
            \end{scope}
            \node at (90:3) {\footnotesize $\clm{A \cap B}$};
          \end{tikzpicture}
        \end{center}
        Rozdíl $\clbr{A \setminus B}$ je část červeného kruhu $\clr{A}$, která
        neprotíná modrý kruh $\clb{B}$. Rozdíl $\cly{B \setminus A}$ je
        zrcadlením téhož obrázku. 
        \begin{center}
          \begin{tikzpicture}[scale=0.5]
            \def\firstcircle{(180:1.5) circle (2)}
            \def\secondcircle{(0:1.5) circle (2)}

            \fill[Sepia] \firstcircle;
            \fill[white] \secondcircle;
            \fill[RoyalBlue,fill opacity=0.25] \secondcircle;
            \draw[RoyalBlue,thick] \secondcircle;
            \node at (90:3) {\footnotesize $\clbr{A \setminus B}$};
          \end{tikzpicture}
          \begin{tikzpicture}[scale=0.5,xshift=10cm]
            \def\firstcircle{(180:1.5) circle (2)}
            \def\secondcircle{(0:1.5) circle (2)}

            \fill[Dandelion] \secondcircle;

            \fill[white] \firstcircle;
            \fill[BrickRed,fill opacity=0.25] \firstcircle;
            \draw[BrickRed,thick] \firstcircle;
            \node at (90:3) {\footnotesize $\cly{B \setminus A}$};
          \end{tikzpicture}
        \end{center}

        Přidáním třetí množiny $\clg{C}$ dostaneme takovýto obrázek.
        \begin{center}
          \begin{tikzpicture}[scale=0.5]
            \fill[BrickRed,fill opacity=0.25] (135:1.5) circle (2);
            \fill[RoyalBlue,fill opacity=0.25] (45:1.5) circle (2);
            \fill[ForestGreen,fill opacity=0.25] (270:1.5) circle (2);
            \draw[BrickRed,thick] (135:1.5) circle (2);
            \draw[RoyalBlue,thick] (45:1.5) circle (2);
            \draw[ForestGreen,thick] (270:1.5) circle (2);
            \node at (150:4.3) {\footnotesize $\clr{A}$};
            \node at (30:4.3) {\footnotesize $\clb{B}$};
            \node at (270:4.3) {\footnotesize $\clg{C}$};
          \end{tikzpicture}
        \end{center}
        Každá oblast na tomto obrázku odpovídá jiné kombinaci množinových
        operací. Například, isolovaná část \clr{červeného} kruhu obsahuje prvky,
        které leží pouze v $\clr{A}$, ale nikoli v $\clb{B}$ nebo v $\clg{C}$.
        Taková množina je výsledkem operace $(\clr{A} \setminus \clb{B})
        \setminus \clg{C}$. Středová oblast, kde se protínají všechny tři kruhy
        je odpovídá množině $\clr{A} \cap \clb{B} \cap \clg{C}$.

        Uveďme konkrétní příklad. Volme množiny
        \[
          \clr{A} = \clr{\{2, 3, 5, 6\}}, \quad \clb{B} = \clb{\{1, 2, 3, 4,
          7\}}, \quad \clg{C} = \clg{\{2, 4, 5, 8\}}.
        \]
        Když umístíme čísla do správných oblastí diagramu, dostaneme tento
        obrázek.
        \begin{center}
          \begin{tikzpicture}[scale=1]
            \fill[BrickRed,fill opacity=0.25] (135:1.5) circle (2);
            \fill[RoyalBlue,fill opacity=0.25] (45:1.5) circle (2);
            \fill[ForestGreen,fill opacity=0.25] (270:1.5) circle (2);
            \draw[BrickRed,thick] (135:1.5) circle (2);
            \draw[RoyalBlue,thick] (45:1.5) circle (2);
            \draw[ForestGreen,thick] (270:1.5) circle (2);
            \node at (150:4.3) {\footnotesize $\clr{A}$};
            \node at (30:4.3) {\footnotesize $\clb{B}$};
            \node at (270:4.3) {\footnotesize $\clg{C}$};

            \node at (135:2.5) {$6$};
            \node at (45:2.5) {$1,7$};
            \node at (270:2.5) {$8$};

            \node at (90:2) {$3$};
            \node at (202.5:1.1) {$5$};
            \node at (337.5:1.1) {$4$};

            \node at (0,0) {$2$};
          \end{tikzpicture}
        \end{center}

        Nakonec se podíváme na dva příklady různých kombinací množinových
        operací.
        \begin{center}
          \begin{tikzpicture}[scale=0.5]
            \def\acircle{(135:1.5) circle (2)}
            \def\bcircle{(45:1.5) circle (2)}
            \def\ccircle{(270:1.5) circle (2)}
            \fill[BrickRed,fill opacity=0.25] \acircle;
            \fill[RoyalBlue,fill opacity=0.25] \bcircle;
            \fill[ForestGreen,fill opacity=0.25] \ccircle;
            \draw[BrickRed,thick] \acircle;
            \draw[RoyalBlue,thick] \bcircle;
            \draw[ForestGreen,thick] \ccircle;

            \begin{scope}
              \clip \acircle;
              \filldraw[Fuchsia] \bcircle;
            \end{scope}
            \begin{scope}
              \clip \bcircle;
              \filldraw[Fuchsia] \ccircle;
            \end{scope}

            \node at (150:4.3) {\footnotesize $\clr{A}$};
            \node at (30:4.3) {\footnotesize $\clb{B}$};
            \node at (270:4.3) {\footnotesize $\clg{C}$};

            \node at (90:5) {\footnotesize $\clm{(A \cap B) \cup (B \cap C)}$};
          \end{tikzpicture}
          \begin{tikzpicture}[scale=0.5,xshift=15cm]
            \def\acircle{(135:1.5) circle (2)}
            \def\bcircle{(45:1.5) circle (2)}
            \def\ccircle{(270:1.5) circle (2)}
            \fill[BrickRed,fill opacity=0.25] \acircle;
            \fill[RoyalBlue,fill opacity=0.25] \bcircle;
            \draw[BrickRed,thick] \acircle;
            \draw[RoyalBlue,thick] \bcircle;

            \filldraw[Fuchsia] \acircle;
            \filldraw[Fuchsia] \bcircle;
            \fill[white] \ccircle;
            \fill[ForestGreen,fill opacity=0.25] \ccircle;
            \draw[ForestGreen,thick] \ccircle;

            \node at (150:4.3) {\footnotesize $\clr{A}$};
            \node at (30:4.3) {\footnotesize $\clb{B}$};
            \node at (270:4.3) {\footnotesize $\clg{C}$};

            \node at (90:5) {\footnotesize $\clm{(A \cup B) \setminus C}$};
          \end{tikzpicture}
        \end{center}
      \end{block}

    \end{column}
    \separatorcolumn
  \end{columns}
\end{frame}

\end{document}
