\documentclass[a4paper,11pt]{article}

\usepackage[english,czech]{babel}
% Fonts %
\usepackage{fouriernc}
\usepackage[T1]{fontenc}

% Colors %
\usepackage[dvipsnames]{color}
\usepackage[dvipsnames]{xcolor}

% Page Layout %
\usepackage[margin=1in]{geometry}

% Fancy Headers %
\usepackage{fancyhdr}
\fancyhf{}
\cfoot{\thepage}
\rhead{}
\renewcommand{\headrulewidth}{0pt}
\setlength{\headheight}{16pt}

% Math
\usepackage{mathtools}
\usepackage{amssymb}
\usepackage{faktor}
\usepackage{import}
\usepackage{caption}
\usepackage{subcaption}
\usepackage{wrapfig}
\usepackage{enumitem}
\setlist{topsep=0pt}

\usepackage{tikz}
\usetikzlibrary{cd,positioning,babel,shapes,decorations.text,
 decorations.pathmorphing}
\usepackage{tkz-base}
\usepackage{tkz-euclide}

% Theorems
\usepackage[thmmarks, amsmath, thref, amsthm]{ntheorem}
\usepackage{thmtools}

\theoremsymbol{\ensuremath{\blacksquare}}
\newtheorem*{solution}{Possible solution.}

% Title %
\title{\Huge\textsf{Math Homework -- PreIB 3.AB 2 \& 3}\\
 \Large\textsf{Trigonometric Functions}
 \author{Áďa Klepáčů}
 \date{\today}
}

% Table of Contents %
\usepackage{hyperref}
\hypersetup{
 colorlinks=true,
 linktoc=all,
 linkcolor=blue
}

% Tables %
\usepackage{booktabs}
\usepackage{tabularx}

% Patch for hyphens
\usepackage{regexpatch}
\makeatletter
% Change the `-` delimiter to an active character
\xpatchparametertext\@@@cmidrule{-}{\cA-}{}{}
\xpatchparametertext\@cline{-}{\cA-}{}{}
\makeatother

\newcolumntype{s}{>{\centering\arraybackslash}p{.4\textwidth}}

% Operators %
\DeclareMathOperator{\Ker}{Ker}
\DeclareMathOperator{\Img}{Im}
\DeclareMathOperator{\End}{End}
\DeclareMathOperator{\Aut}{Aut}
\DeclareMathOperator{\Inn}{Inn}

% Common operators %
\newcommand{\R}{\mathbb{R}}
\newcommand{\N}{\mathbb{N}}
\newcommand{\Z}{\mathbb{Z}}
\newcommand{\Q}{\mathbb{Q}}
\newcommand{\C}{\mathbb{C}}

\newcommand{\clr}{\textcolor{BrickRed}}
\newcommand{\clb}{\textcolor{RoyalBlue}}
\newcommand{\clg}{\textcolor{ForestGreen}}
\newcommand{\clm}{\textcolor{Fuchsia}}
\newcommand{\clv}{\textcolor{violet}}
\newcommand{\clbr}{\textcolor{Sepia}}
\newcommand{\cly}{\textcolor{Dandelion}}

% American Paragraph Skip %
\setlength{\parindent}{0pt}
\setlength{\parskip}{1em}

% Document %
\pagestyle{fancy}
\begin{document}

Pro $x,y>0$ platí
\[
 \lim_{p \to 0} \sqrt[p]{\frac{x^{p}+y^{p}}{2}} = \sqrt{xy}.
\]
\begin{proof}
 Spočteme nejprve limitu zprava. Položíme $r \coloneqq 1 / p$ a počítáme
 \[
  \lim_{p \to 0^{+}} \sqrt[p]{\frac{x^{p} + y^{p}}{2}} = \lim_{r \to \infty}
  \left( \frac{\sqrt[r]{x} + \sqrt[r]{y}}{2} \right)^r.
 \]
 Upravíme
 \[
  \left( \frac{\sqrt[r]{x} + \sqrt[r]{y}}{2} \right)^{r} = \frac{x}{2^{r}}\left(
  1 + \sqrt[r]{\frac{y}{x}} \right)^{r}.
 \]
 Položme $a \coloneqq y / x$. Ukážeme, že
 \[
  \lim_{r \to \infty} \frac{(1 + \sqrt[r]{a})^{r}}{2^{r}\sqrt{a}} = 1.
 \]
 Protože $\log$ je spojitá funkce na $(0,\infty)$ a výraz v limitě je vždy
 kladný, je tato rovna $1$, právě když
 \[
  \lim_{r \to \infty} \log \frac{(1 + \sqrt[r]{a})^{r}}{2^{r}\sqrt{a}} = 0.
 \]
 Opět upravíme
 \begin{equation*}
  \label{eq:1}
  \tag{$\heartsuit$}
  \log \frac{(1 + \sqrt[r]{a})^{r}}{2^{r}\sqrt{a}} = r \cdot \log \frac{1 +
  \sqrt[r]{a}}{2 \sqrt[2r]{a}} = \frac{\log \frac{1 + \exp \left( \frac{1}{r}
  \log a \right)}{2\exp \left( \frac{1}{2r} \log a \right)}}{\frac{1}{r}}.
 \end{equation*}
 Jelikož $\lim_{r \to \infty} 1 / r = 0$ a $\exp$ je spojitá na $\R$, platí
 $\lim_{r \to \infty} \exp((1 / r) \log a) = 1$. Potom též
 \[
  \lim_{r \to \infty} \log \frac{1 + \exp \left( \frac{1}{r} \log a
  \right)}{2\exp \left( \frac{1}{2r} \log a \right)} = \log \frac{1 + 1}{2} =
  \log 1 = 0.
 \]
 Na limitu pro $r \to \infty$ výrazu \eqref{eq:1} lze proto použít l'Hospitalovo
 pravidlo. Máme
 \begin{align*}
  \left( \frac{1}{r} \right)' &= -\frac{1}{r^2},\\
  \exp'\left( \frac{1}{cr} \log a \right) &= -\frac{1}{cr^2}\log a \cdot \exp
  \left( \frac{1}{cr} \log a \right) \quad \text{pro $c \neq 0$},
 \end{align*}
 čili
 \begin{align*}
  \left( \frac{1 + \exp \left( \frac{1}{r} \log a \right)}{2 \exp \left(
   \frac{1}{2r} \log a \right)} \right)' &= \frac{\frac{-2\log a}{r^2}\exp
   \left( \frac{1}{r}\log a + \frac{1}{2r} \log a \right) + \frac{\log
   a}{r^2}\exp\left(\frac{1}{2r}\log a\right) \cdot \left(1 + \exp \left(
  \frac{1}{r}\log a \right) \right)}{4 \exp \left( \frac{1}{r} \log a \right)}\\
                                         &= \frac{\frac{\log a}{r^2} \left( -2a
                                         - 2 \exp \left( \frac{3}{2r} \right) +
                                       2a + \exp \left( \frac{1}{2r} \right) +
                                     \exp \left( \frac{3}{2r}
                                   \right)\right)}{4\exp \left( \frac{1}{r}\log
                                   a\right)}\\
                                         &= \frac{\frac{\log a}{r^2} \left(
                                         \exp \left( \frac{1}{2r} \right) - \exp
                                       \left( \frac{3}{2r}
                                     \right)\right)}{4\exp \left(
                                     \frac{1}{r}\log a \right)}.
 \end{align*}
 Odtud
 \begin{align*}
  \frac{\left( \log \frac{1 + \exp \left( \frac{1}{r} \log a \right)}{2 \exp
   \left( \frac{1}{2r} \log a \right)} \right)'}{\left( \frac{1}{r} \right)'} 
   &=
   -r^2 \cdot \frac{2\exp \left(\frac{1}{2r} \log a \right)}{1 + \exp \left(
    \frac{1}{r} \log a \right)} \cdot \frac{\frac{\log a}{r^2} \left( \exp
    \left( \frac{1}{2r} \right) - \exp \left( \frac{3}{2r} \right)
  \right)}{4\exp \left( \frac{1}{r}\log a \right)}\\
   &= -\log a \cdot \frac{2\exp \left( \frac{1}{2r} \log a \right)}{1 + \exp
   \left( \frac{1}{r} \log a \right)} \cdot \frac{\exp \left( \frac{1}{2r}
  \right) - \exp \left( \frac{3}{2r} \right)}{4\exp \left( \frac{1}{r} \log a
 \right)}.
 \end{align*}
 Protože
 \[
  \lim_{r \to \infty} \exp \left( \frac{1}{cr}\log a \right) = 1,
 \]
 spočteme
 \[
  \lim_{r \to \infty} -\log a \cdot \frac{2\exp \left( \frac{1}{2r} \log a
  \right)}{1 + \exp \left( \frac{1}{r} \log a \right)} \cdot \frac{\exp \left(
  \frac{1}{2r} \right) - \exp \left( \frac{3}{2r} \right)}{4\exp \left(
\frac{1}{r} \log a \right)} = -\log a \cdot \frac{2 \cdot 1}{1 + 1} \cdot
\frac{1 - 1}{4 \cdot 1} = 0,
 \]
 jak jsme chtěli.

 Teď víme, že
 \[
  \lim_{r \to \infty} \frac{(1 + \sqrt[r]{a})^{r}}{2^{r}\sqrt{a}} = 1.
 \]
 Takže
 \[
  \lim_{p \to 0^{+}} \sqrt[p]{\frac{x^{p} + y^{p}}{2}} = \lim_{r \to \infty}
  \frac{x}{2^{r}}\left(1 + \sqrt[r]{\frac{y}{x}} \right)^{r} = \lim_{r \to
  \infty} \frac{x}{2^r} \cdot 2^{r} \sqrt{\frac{y}{x}} = \sqrt{xy},
 \]
 čímž je důkaz hotov pro limitu zprava.

 Pro limitu zleva platí
 \[
  \lim_{p \to 0^{-}} \sqrt[p]{\frac{x^{p} + y^{p}}{2}} = \lim_{p \to 0^{+}}
  \sqrt[-p]{\frac{x^{-p} + y^{-p}}{2}} = \lim_{p \to 0^{+}}
  \frac{1}{\sqrt[p]{\frac{x^{-p} + y^{-p}}{2}}}.
 \]
 Po substituci $a \coloneqq x^{-1}$ a $b \coloneqq y^{-1}$ dostaneme
 \[
  \lim_{p \to 0^{+}} \frac{1}{\sqrt[p]{\frac{a^{p} + b^{p}}{2}}} =
  \frac{1}{\sqrt{ab}} = \sqrt{xy},
 \]
 kde první rovnost plyne z předchozího výpočtu. Důkaz je hotov.
\end{proof}

\end{document}
