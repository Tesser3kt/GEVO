\documentclass[a4paper,11pt]{article}

% Colors %
\usepackage[dvipsnames]{xcolor}

% Page Layout %
\usepackage[margin=1.5in]{geometry}

% Fancy Headers %
\usepackage{fancyhdr}
\fancyhf{}
\cfoot{\thepage}
\rhead{}
\renewcommand{\headrulewidth}{0pt}
\setlength{\headheight}{16pt}

% Math
\usepackage{mathtools}
\usepackage{amssymb}
\usepackage{faktor}
\usepackage{import}
\usepackage{caption}
\usepackage{subcaption}
\usepackage{wrapfig}
\usepackage{import}

% Theorems
\usepackage{amsthm}
\usepackage{thmtools}

% Title %
\title{\Huge\textsf{}\\
 \Large\textsf{ }
 \author{}
 \date{}
}

% Table of Contents %
\usepackage{hyperref}
\hypersetup{
 colorlinks=true,
 linktoc=all,
 linkcolor=blue
}

% Tables %
\usepackage{booktabs}

% Enumerate %
\usepackage{enumerate}

% Operators %
\DeclareMathOperator{\Ker}{Ker}
\DeclareMathOperator{\Img}{Im}
\DeclareMathOperator{\End}{End}
\DeclareMathOperator{\Aut}{Aut}
\DeclareMathOperator{\Inn}{Inn}

% Common operators %
\newcommand{\R}{\mathbb{R}}
\newcommand{\N}{\mathbb{N}}
\newcommand{\Z}{\mathbb{Z}}
\newcommand{\Q}{\mathbb{Q}}
\newcommand{\C}{\mathbb{C}}

% American Paragraph Skip %
\setlength{\parindent}{0pt}
\setlength{\parskip}{1em}

% Document %
\fancyhf{}
\pagestyle{fancy}
\begin{document}

\section*{Logic}

Logic is the language of mathematics, it basically provides a framework for
formally working with \emph{expressions}. We mean by \emph{expression}, any
sentence which can be said to be either \textbf{true} or \textbf{false}.

For example, sentences `I am thirsty.' and `Everyone hates math.' are
expressions while `How ya doin', mate?' and `I hope this ends soon.' are not.
Also, the sentence `Trump's going to win the next presidential election.' is
also an \emph{expression} even though we don't know whether it's true or false.
Our knowing or not knowing has nothing to do with a sentence being an
expression.

So called first-order logic has only two number/constants and five operations on
expressions.
\begin{itemize}
 \item The two constants are $0$ (interpreted as \textbf{false} or \textbf{lie})
  and $1$ (interpreted as \textbf{true} or \textbf{truth}). These are sometimes
  written as $\bot$ for \textbf{false} and $\top$ for \textbf{true}.
 \item Expressions in logic are typically written as small letters of roman
  alphabet ($a$, $b$, $c$, ...). If for instance $a$ stands for the sentence
  `It's raining.', it can be either $0$ or $1$ depending on the actual weather.
 \item Let's say $a$ stands for `It's raining.' and $b$ for `I'm thirsty.' We
  can perform these logical operations with $a$ and $b$:
  \begin{itemize}
   \item \textbf{negation} (denoted by $\neg)$: $\neg a$ means `It's \emph{not}
    raining.' and $\neg b$ means `I'm \emph{not} thirsty.' 
   \item \textbf{conjunction} (denoted by $ \wedge $ or $\&$): $a \wedge b$ means
    `It's raining \emph{and} I'm thirsty.'
   \item \textbf{disjunction} (denoted by $ \vee $ ): $a \vee b$ means `It's
    raining \emph{or} I'm thirsty.' \textbf{Beware!} $a \vee b$ is true even if
    both $a$ and $b$ are true.
   \item \textbf{implication} (denoted by $ \implies $ or $\rightarrow$) $a
    \implies b$ means `\emph{If} it's raining, \emph{then} I'm thirsty.'
    \textbf{Beware!} Unlike in normal language, if $a$ is false, then $a
    \implies b$ is always \textbf{true}! The sentence `If it's raining, then I'm
    thirsty.' does not say \emph{anything at all} about me being thirsty when
    it's not raining. Which means that $b$ can be true or false, I don't care.
   \item \textbf{equivalence} (denoted by $ \iff $ or $\equiv$):  $a \iff b$ 
    means `It's raining \emph{if and only if} I'm thirsty.' To translate, this
    means that $a \iff b$ if both $a$ and $b$ are true or both $a$ and $b$ are
    false.
  \end{itemize}
 \end{itemize}
 Of course, we can chain these operations just as in basic arithmetic. So, for
 example
 \[
  ((\neg a  \wedge b) \implies (c \vee \neg b)) \iff (a \implies \neg c)
 \]
 is an expression. It may be hard to tell when expressions are true or false.
 For this we use \textbf{truth tables}. Simply put, truth tables are tables
 where you put the studied expression and check whether it is true or false for
 every possible value of the variables inside. For instance, let's take $\neg
 a \implies b$. Then, we might want to write the table
 \begin{center}
  \begin{tabular}{c|c|c|c}
   $a$ & $b$ & $\neg a$ & $\neg a \implies b$ \\
   \hline
   $0$ & $0$ &  $1$ & $0$\\
   $0$ & $1$ & $1$ &  $1$\\
    $1$ &  $0$ &  $0$ &  $1$\\
     $1$ & $1$& $0$ & $1$
  \end{tabular}
 \end{center}
 We simply check the `truthfulness' of $\neg a \implies b$ depending on whether
 $a$ and $b$ are true or false. You may want to think about it for a while.

 \textbf{Warning!} There is no $=$ symbol in logic. No two expressions are ever
 equal. They can be equivalent but never equal.

 \section*{Sets}
 Modern mathematics is the language of set theory. Everything (literally) in
 mathematics is a \emph{set}. This also means that mathematics cannot answer
 the question `What is a set?' as it is the thing whereby math is formulated.
 People typically think of sets as groups of elements which somehow belong
 together. But any other interpretation is just as valid.

 When talking about sets, we often mention their \emph{elements}. The fact that
 an element $x$ lies in a set $A$ is denoted by $x \in A$. The symbol $\in$ is
 a weirdified letter `E', standing for \emph{\textbf{E}lement}.

 There are a few notions and operations concerning sets. We formulate them
 using mathematical logic.
 \begin{itemize} 
  \item \textbf{Inclusion}. If $A,B$ are sets, then $A \subseteq B$ means that
   every element of $A$ is also an element of $B$. In symbols,
   \[
    A \subseteq B \iff (x \in A \implies x \in B).
   \]
  We often say something like `$A$ is a \emph{subset} of $B$' or `$A$ is
   \emph{contained} in $B$ '.
  \item \textbf{Equality}. We write $A = B$ if $A$ and $B$ share all elements.
   That is to say
   \[
    A = B \iff (A \subseteq B \wedge B \subseteq A)
   \]
   or, purely logically,
   \[
    A = B \iff (x \in A \iff x \in B).
   \]
   If we want to specify that $A$ is a subset of $B$ but is not equal to $B$,
   that is, $A \subseteq B \wedge A \neq B$, we write succinctly $A \subsetneq
   B$.
  \item \textbf{Union}. The union $A \cup B$ of $A$ and $B$ is the set that
   contains all elements which are in $A$ \emph{or} in $B$. Logically,
   \[
    x \in A \cup B \iff (x \in A \vee x \in B).
   \]
   \begin{figure}[h]
    \centering
    \begin{subfigure}{.45\textwidth}
     \centering
     \def\svgwidth{.7\textwidth}
     \import{figs/}{union-before.pdf_tex}
     \caption*{Sets \textcolor{green}{$A$} and \textcolor{red}{$B$}.}
    \end{subfigure}
    \begin{subfigure}{.45\textwidth}
     \centering
     \def\svgwidth{.7\textwidth}
     \import{figs/}{union-after.pdf_tex}
     \caption*{The union \textcolor{blue}{$A \cup B$}.}
    \end{subfigure}
   \end{figure}
 \end{itemize}
\end{document}
