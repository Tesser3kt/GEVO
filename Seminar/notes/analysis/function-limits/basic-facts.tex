\section{Základní poznatky o limitě funkce}
\label{sec:zakladni-poznatky-o-limite-funkce}

Počneme nyní shrnovati intuitivně vcelku zřejmé výsledky o limitách reálných
funkcí. Jakž jsme již vícekrát děli, ona \uv{intuitivní zřejmost} pravdivosti
výroků nechce nabodnout k přeskoku či trivializaci jejich důkazů. Vodami
nekonečnými radno broditi se ostražitě, bo tvrzení jako \emph{limita složené
	funkce} % TODO
ráda svědčí, že intuicí bez logiky člověk na břeh nedoplove.

Na první pád není překvapivé, že limita funkce je jednoznačně určena,
pochopitelně za předpokladu její existence. Vyzýváme čtenáře, aby se při čtení
důkazu drželi vizualizace oboustranné limity z
\myref{obrázku}{fig:oboustranna-limita-ve-2D}.

\begin{lemma}{Jednoznačnost limity}{jednoznacnost-limity}
	Limita funkce (ať už jednostranná či oboustranná) je jednoznačně určená, pokud
	existuje.
\end{lemma}

\begin{lemproof}
	Dokážeme lemma pouze pro oboustrannou limitu, důkaz pro limity jednostranné je
	v zásadě totožný.

	Pro spor budeme předpokládat, že $L$ i $L'$ jsou limity $f$ v bodě $a \in
		\R^{*}$. Nejprve ošetříme případ, kdy $L,L' \in \R$. Bez újmy na obecnosti
	smíme předpokládat, že $L > L'$. Volme $\varepsilon \coloneqq (L-L') / 3$. K
	tomuto $\varepsilon$ existují z
	\hyperref[def:oboustranna-limita-funkce]{definice limity} $\delta_1>0,
		\delta_2>0$ takové,
	že
	\[
		\forall x \in R(a,\delta_1): f(x) \in B(L,\varepsilon).
	\]
	a rovněž
	\[
		\forall x \in R(a,\delta_2): f(x) \in B(L',\varepsilon).
	\]
	Volíme-li ovšem $\delta \coloneqq \min(\delta_1,\delta_2)$, pak pro $x \in
		R(a,\delta)$ dostaneme
	\[
		f(x) \in B(L,\varepsilon) \cap B(L',\varepsilon).
	\]
	Poslední vztah lze přepsat do tvaru
	\begin{align*}
		L - \varepsilon  & < f(x) < L + \varepsilon,  \\
		L' - \varepsilon & < f(x) < L' + \varepsilon.
	\end{align*}
	Odtud plyne, že
	\[
		L - \varepsilon < L' + \varepsilon,
	\]
	což po dosazení $\varepsilon = (L - L') / 3$ a následné úpravě vede na
	\[
		2L - L' < 2L' - L,
	\]
	z čehož ihned
	\[
		L < L',
	\]
	což je spor.

	Nyní ať například $L = \infty$ a $L' \in \R$. Z
	\hyperref[def:okoli-a-prstencove-okoli-bodu]{definice okolí} $B(L,\varepsilon)$
	pro $L = \infty$, stačí nalézt $\varepsilon > 0$ takové, že
	\[
		\frac{1}{\varepsilon} > L' + \varepsilon,
	\]
	pak se totiž nemůže stát, že
	\[
		f(x) \in B(\infty,\varepsilon) \cap B(L',\varepsilon).
	\]
	Snadným výpočtem zjistíme, že
	\[
		\frac{1}{\varepsilon} > L' + \varepsilon
	\]
	právě tehdy, když $\varepsilon < (\sqrt{L'^2 + 4} - L') / 2$. Pro libovolné
	takové $\varepsilon$ tudíž dostáváme spor stejně jako v předchozím případě.

	Ostatní případy se ošetří obdobně.
\end{lemproof}

\begin{figure}[ht]
 \centering
 \begin{tikzpicture}
	\def\del{1}
	\def\eps{1.5}

	\tkzDefPoints{0/0/a,4/2/L,4/-2/Lp}
	\tkzLabelPoint[color=BrickRed,below=1mm](a){$a$}
	\tkzLabelPoint[color=ForestGreen,below left=1mm](L){$L$}
	\tkzLabelPoint[color=ForestGreen,below left=1mm](Lp){$L'$}

	\tkzDefCircle[R](a,\del) \tkzGetPoint{ad}
	\tkzDrawCircle[thick](a,ad)

	\tkzDefCircle[R](L,\eps) \tkzGetPoint{Le}
	\tkzDrawCircle[color=Magenta,thick](L,Le)

	\tkzDefPoints{0/1.2/s,2.8/3.2/t,0/-1.2/s2,2.8/-3.2/t2}
	\draw[bend left=45,color=RoyalBlue,-latex] (s) to node[midway,above left]{$f$}
	(t);
	\draw[bend right=45,color=RoyalBlue,-latex] (s2) to node[midway,below
		left]{$f$} (t2);

	\tkzDefPointOnCircle[R = center a angle 150 radius \del] \tkzGetPoint{B}
	\tkzDrawPoint[color=BrickRed,size=3](B)
	\tkzDrawSegment[color=BrickRed,dashed,thick](a,B)
	\tkzLabelSegment[color=BrickRed,above right=-1mm](a,B){$\delta$}
	\tkzDrawPoint[draw=BrickRed,thick,size=6,fill=white](a)

	\tkzDrawSegment[decorate,decoration={brace,
				amplitude=10pt},color=Aquamarine,thick](L,Lp)
	\tkzLabelSegment[right=3mm,color=Aquamarine](L,Lp){$\displaystyle
			\frac{|L-L'|}{2}$}

	\tkzDefPointOnCircle[R = center L angle 150 radius \eps] \tkzGetPoint{C}
	\tkzDrawPoint[color=Magenta,size=3](C)
	\tkzDrawSegment[color=Magenta,dashed,thick](L,C)
	\tkzLabelSegment[color=Magenta,above right=-1mm](L,C){$\varepsilon$}
	\tkzDrawPoint[color=ForestGreen,size=6](L)

	\tkzDefCircle[R](Lp,\eps) \tkzGetPoint{Lpe}
	\tkzDrawCircle[color=Magenta,thick](Lp,Lpe)

	\tkzDefPointOnCircle[R = center Lp angle 150 radius \eps] \tkzGetPoint{D}
	\tkzDrawPoint[color=Magenta,size=3](D)
	\tkzDrawSegment[color=Magenta,dashed,thick](Lp,D)
	\tkzLabelSegment[color=Magenta,above right=-1mm](Lp,D){$\varepsilon$}
	\tkzDrawPoint[color=ForestGreen,size=6](Lp)

 \end{tikzpicture}
 \caption{Spor v důkazu \myref{lemmatu}{lem:jednoznacnost-limity}.}
 \label{fig:jednoznacnost-limity}
\end{figure}

\begin{lemma}{}{ma-limitu-je-omezena}
 Ať reálná funkce $f$ má \textbf{konečnou} limitu $L \in \R$ v bodě $a \in
 \R^{*}$. Pak existuje prstencové okolí $a$, na němž je $f$ omezená.
\end{lemma}
\begin{lemproof}
 Pro dané $\varepsilon>0$ nalezneme z
 \hyperref[def:oboustranna-limita-funkce]{definice limity} $\delta>0$ takové, že
 pro $x \in R(a,\delta)$ platí $f(x) \in B(L,\varepsilon)$. Protože však
 $B(L,\varepsilon) = (L-\varepsilon,L+\varepsilon)$ platí pro $x \in
 R(a,\delta)$ odhady
 \[
  L-\varepsilon \leq f(x) \leq L+\varepsilon,
 \]
 čili je $f$ na $R(a,\delta)$ omezená.
\end{lemproof}

Vzhledem k základním aritmetickým operacím si limity funkcí počínají vychovaně.
Za předpokladu, že výsledný výraz dává smysl, můžeme spočítat limitu součtu,
součinu či podílu funkcí jako součet, součin či podíl limit těchto funkcí.

\begin{theorem}{Aritmetika limit funkcí}{aritmetika-limit-funkci}
 Ať $f,g$ jsou reálné funkce a $a \in \R^{*}$. Předpokládejme, že $\lim_{x \to
 a} f(x)$ i $\lim_{x \to a} g(x)$ existují a označme je po řadě $L_f$ a $L_g$.
 Potom platí
 \begin{enumerate}[label=(\alph*)]
  \item $\lim_{x \to a} (f + g)(x) = L_f + L_g$, dává-li výraz napravo smysl.
  \item $\lim_{x \to a} (f \cdot g)(x) = L_f \cdot L_g$, dává-li výraz napravo
   smysl.
  \item $\lim_{x \to a} (f / g)(x) = L_f / L_g$, dává-li výraz napravo smysl.
 \end{enumerate}
\end{theorem}

\begin{thmproof}
 Dokážeme pouze část (c), neboť je výpočetně nejnáročnější, ač nepřináší mnoho
 intuice. Část (a) je triviální a (b) je lehká. Vyzýváme čtenáře, aby se je
 pokusili dokázat sami.

 Už jen v důkazu samotné části (c) bychom správně měli rozlišit šest různých
 případů:
 \begin{enumerate}
  \item $L_f \in \R, L_g \in \R \setminus \{0\}$,
  \item $L_f \in \R, L_g \in \{-\infty,\infty\}$,
  \item $L_f = \infty, L_g \in (0,\infty)$,
  \item $L_f = \infty, L_g \in (-\infty,0)$,
  \item $L_f = -\infty, L_g \in (0,\infty)$,
  \item $L_f = -\infty, L_g \in (-\infty,0)$.
 \end{enumerate}
 Jelikož se výpočty limit v oněch případech liší vzájemně pramálo a získaná
 intuice je asymptoticky rovna té ze znalosti metod řešení exponenciálních
 rovnic, soustředíme se pouze na (nejzajímavější) případ (1).

 Ať tedy $L_f \in \R, L_g \in \R \setminus \{0\}$. Je nejprve dobré si uvědomit,
 že $L_f / L_g$ není definován \textbf{nikdy}, pokud $L_g = 0$, bez ohledu na
 hodnotu $L_f$. Totiž, hodnoty $g$ se mohou k $L_g$ limitně blížit zprava,
 zleva či střídavě z obou směrů. Nelze tudíž obecně určit, zda dělíme čím dál
 tím menším kladným číslem, či čím dál tím větším záporným číslem.

 Položme $\varepsilon_g = |L_g| / 2$. K tomuto $\varepsilon_g$ existuje z
 \hyperref[def:oboustranna-limita-funkce]{definice limity} $\delta_g$ takové, že
 pro $x \in R(a,\delta_g)$ platí $g(x) \in B(L_g,\varepsilon_g)$. Poslední vztah
 si přepíšeme na
 \begin{align*}
  L_g - \varepsilon_g & < g(x) < L_g + \varepsilon_g,\\
  L_g - \frac{|L_g|}{2} & < g(x) < L_g + \frac{|L_g|}{2}.
 \end{align*}
 Speciálně tedy pro $x \in R(a,\delta_g)$ máme odhad
 \[
  |g(x)| > \left| L_g - \frac{|L_g|}{2} \right| > \frac{|L_g|}{2}.
 \]
 Jelikož poslední výraz je z předpokladu kladný, má výraz $f(x) / g(x)$ smysl
 pro každé $x \in R(a,\delta_g)$, neboť pro tato $x$ platí $g(x) \neq 0$.

 Pro $x \in R(a,\delta_g)$ odhadujme
 \begin{align*}
  \left| \frac{f(x)}{g(x)} - \frac{L_f}{L_g} \right| &= \frac{|f(x)L_g -
  g(x)L_f|}{|g(x)||L_g|} = \frac{|f(x) L_g - L_f L_g + L_f L_g -
  g(x)L_f|}{|g(x)| |L_g|}\\
  																									 & \leq \frac{|L_g| |f(x) -
  																									 L_f| + |L_f| |L_g -
  																									g(x)|}{|g(x)| |L_g|}\\
  																									 &= \frac{1}{|g(x)|}
  																									 |f(x) - L_f| +
  																									 \frac{|L_f|}{|g(x)|
  																									 |L_g|}|L_g - g(x)|\\
  																									 &< \frac{2}{|L_g|}|f(x) -
  																									 L_f| + \frac{2
  																									 |L_f|}{|L_g|^2}|L_g -
  																									 g(x)|\\
  																									 & \leq c(|f(x) - L_f| +
  																									 |L_g - g(x)|)
 \end{align*}
 pro $c \coloneqq \max(2 / |L_g|, 2|L_f|/|L_g|^2)$.

 Ať je nyní dáno $\varepsilon>0$. K číslu $\varepsilon / 2c$ existují z
 \hyperref[def:oboustranna-limita-funkce]{definice limity} $\delta_1,\delta_2>0$
 taková, že
 \begin{align*}
 	\forall x \in R(a,\delta_1)&: |g(x) - L_g| < \frac{\varepsilon}{2c},\\
 	\forall x \in R(a,\delta_2)&: |f(x) - L_f| < \frac{\varepsilon}{2c}.
 \end{align*}
 Položíme-li nyní $\delta \coloneqq \min(\delta_1,\delta_2,\delta_g)$, pak pro
 $x \in R(a,\delta)$ platí
 \[
  \left| \frac{f(x)}{g(x)} - \frac{L_f}{L_g} \right| < c (|f(x) - L_f| + |L_g -
  g(x)|) < c \left(\frac{\varepsilon}{2c} + \frac{\varepsilon}{2c}\right) =
  \varepsilon,
 \]
 což dokazuje rovnost $\lim_{x \to a} (f / g)(x) = L_f / L_g$.
\end{thmproof}

\begin{exercise}{}{aritmetika-limit}
 Dokažte tvrzení (b) a (c) ve \myref{větě}{thm:aritmetika-limit-funkci}.
\end{exercise}
