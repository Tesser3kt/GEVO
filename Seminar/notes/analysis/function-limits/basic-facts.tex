\section{Základní poznatky o limitě funkce}
\label{sec:zakladni-poznatky-o-limite-funkce}

Počneme nyní shrnovati intuitivně vcelku zřejmé výsledky o limitách reálných
funkcí. Jakž jsme již vícekrát děli, ona \uv{intuitivní zřejmost} pravdivosti
výroků nechce nabodnout k přeskoku či trivializaci jejich důkazů. Vodami
nekonečnými radno broditi se ostražitě, bo tvrzení jako \emph{limita složené
	funkce} % TODO
ráda svědčí, že intuicí bez logiky člověk na břeh nedoplove.

Na první pád není překvapivé, že limita funkce je jednoznačně určena,
pochopitelně za předpokladu její existence. Vyzýváme čtenáře, aby se při čtení
důkazu drželi vizualizace oboustranné limity z
\myref{obrázku}{fig:oboustranna-limita-ve-2D}.

\begin{lemma}{Jednoznačnost limity}{jednoznacnost-limity}
	Limita funkce (ať už jednostranná či oboustranná) je jednoznačně určená, pokud
	existuje.
\end{lemma}

\begin{lemproof}
	Dokážeme lemma pouze pro oboustrannou limitu, důkaz pro limity jednostranné je
	v zásadě totožný.

	Pro spor budeme předpokládat, že $L$ i $L'$ jsou limity $f$ v bodě $a \in
		\R^{*}$. Nejprve ošetříme případ, kdy $L,L' \in \R$. Bez újmy na obecnosti
	smíme předpokládat, že $L > L'$. Volme $\varepsilon \coloneqq (L-L') / 3$. K
	tomuto $\varepsilon$ existují z
	\hyperref[def:oboustranna-limita-funkce]{definice limity} $\delta_1>0,
		\delta_2>0$ takové,
	že
	\[
		\forall x \in R(a,\delta_1): f(x) \in B(L,\varepsilon).
	\]
	a rovněž
	\[
		\forall x \in R(a,\delta_2): f(x) \in B(L',\varepsilon).
	\]
	Volíme-li ovšem $\delta \coloneqq \min(\delta_1,\delta_2)$, pak pro $x \in
		R(a,\delta)$ dostaneme
	\[
		f(x) \in B(L,\varepsilon) \cap B(L',\varepsilon).
	\]
	Poslední vztah lze přepsat do tvaru
	\begin{align*}
		L - \varepsilon  & < f(x) < L + \varepsilon,  \\
		L' - \varepsilon & < f(x) < L' + \varepsilon.
	\end{align*}
	Odtud plyne, že
	\[
		L - \varepsilon < L' + \varepsilon,
	\]
	což po dosazení $\varepsilon = (L - L') / 3$ a následné úpravě vede na
	\[
		2L - L' < 2L' - L,
	\]
	z čehož ihned
	\[
		L < L',
	\]
	což je spor.

	Nyní ať například $L = \infty$ a $L' \in \R$. Z
	\hyperref[def:okoli-a-prstencove-okoli-bodu]{definice okolí} $B(L,\varepsilon)$
	pro $L = \infty$, stačí nalézt $\varepsilon > 0$ takové, že
	\[
		\frac{1}{\varepsilon} > L' + \varepsilon,
	\]
	pak se totiž nemůže stát, že
	\[
		f(x) \in B(\infty,\varepsilon) \cap B(L',\varepsilon).
	\]
	Snadným výpočtem zjistíme, že
	\[
		\frac{1}{\varepsilon} > L' + \varepsilon
	\]
	právě tehdy, když $\varepsilon < (\sqrt{L'^2 + 4} - L') / 2$. Pro libovolné
	takové $\varepsilon$ tudíž dostáváme spor stejně jako v předchozím případě.

	Ostatní případy se ošetří obdobně.
\end{lemproof}

\begin{figure}[ht]
 \centering
 \begin{tikzpicture}
	\def\del{1}
	\def\eps{1.5}

	\tkzDefPoints{0/0/a,4/2/L,4/-2/Lp}
	\tkzLabelPoint[color=BrickRed,below=1mm](a){$a$}
	\tkzLabelPoint[color=ForestGreen,below left=1mm](L){$L$}
	\tkzLabelPoint[color=ForestGreen,below left=1mm](Lp){$L'$}

	\tkzDefCircle[R](a,\del) \tkzGetPoint{ad}
	\tkzDrawCircle[thick](a,ad)

	\tkzDefCircle[R](L,\eps) \tkzGetPoint{Le}
	\tkzDrawCircle[color=Magenta,thick](L,Le)

	\tkzDefPoints{0/1.2/s,2.8/3.2/t,0/-1.2/s2,2.8/-3.2/t2}
	\draw[bend left=45,color=RoyalBlue,-latex] (s) to node[midway,above left]{$f$}
	(t);
	\draw[bend right=45,color=RoyalBlue,-latex] (s2) to node[midway,below
		left]{$f$} (t2);

	\tkzDefPointOnCircle[R = center a angle 150 radius \del] \tkzGetPoint{B}
	\tkzDrawPoint[color=BrickRed,size=3](B)
	\tkzDrawSegment[color=BrickRed,dashed,thick](a,B)
	\tkzLabelSegment[color=BrickRed,above right=-1mm](a,B){$\delta$}
	\tkzDrawPoint[draw=BrickRed,thick,size=6,fill=white](a)

	\tkzDrawSegment[decorate,decoration={brace,
				amplitude=10pt},color=Aquamarine,thick](L,Lp)
	\tkzLabelSegment[right=3mm,color=Aquamarine](L,Lp){$\displaystyle
			\frac{|L-L'|}{2}$}

	\tkzDefPointOnCircle[R = center L angle 150 radius \eps] \tkzGetPoint{C}
	\tkzDrawPoint[color=Magenta,size=3](C)
	\tkzDrawSegment[color=Magenta,dashed,thick](L,C)
	\tkzLabelSegment[color=Magenta,above right=-1mm](L,C){$\varepsilon$}
	\tkzDrawPoint[color=ForestGreen,size=6](L)

	\tkzDefCircle[R](Lp,\eps) \tkzGetPoint{Lpe}
	\tkzDrawCircle[color=Magenta,thick](Lp,Lpe)

	\tkzDefPointOnCircle[R = center Lp angle 150 radius \eps] \tkzGetPoint{D}
	\tkzDrawPoint[color=Magenta,size=3](D)
	\tkzDrawSegment[color=Magenta,dashed,thick](Lp,D)
	\tkzLabelSegment[color=Magenta,above right=-1mm](Lp,D){$\varepsilon$}
	\tkzDrawPoint[color=ForestGreen,size=6](Lp)

 \end{tikzpicture}
 \caption{Spor v důkazu \myref{lemmatu}{lem:jednoznacnost-limity}.}
 \label{fig:jednoznacnost-limity}
\end{figure}


