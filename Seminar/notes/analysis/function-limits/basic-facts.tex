\section{Základní poznatky o limitě funkce}
\label{sec:zakladni-poznatky-o-limite-funkce}

Počneme nyní shrnovati intuitivně vcelku zřejmé výsledky o limitách reálných
funkcí. Jakž jsme již vícekrát děli, ona \uv{intuitivní zřejmost} pravdivosti
výroků nechce nabodnout k přeskoku či trivializaci jejich důkazů. Vodami
nekonečnými radno broditi se ostražitě, bo tvrzení jako
\hyperref[thm:limita-slozene-funkce]{limita složené funkce} ráda svědčí, že
intuicí bez logiky člověk na břeh nedoplove.

Na první pád není překvapivé, že limita funkce je jednoznačně určena,
pochopitelně za předpokladu její existence. Vyzýváme čtenáře, aby se při čtení
důkazu drželi vizualizace oboustranné limity z
\myref{obrázku}{fig:oboustranna-limita-ve-2D}.

\begin{lemma}{Jednoznačnost limity}{jednoznacnost-limity}
	Limita funkce (ať už jednostranná či oboustranná) je jednoznačně určená, pokud
	existuje.
\end{lemma}

\begin{lemproof}
	Dokážeme lemma pouze pro oboustrannou limitu, důkaz pro limity jednostranné je
	v zásadě totožný.

	Pro spor budeme předpokládat, že $L$ i $L'$ jsou limity $f$ v bodě $a \in
		\R^{*}$. Nejprve ošetříme případ, kdy $L,L' \in \R$. Bez újmy na obecnosti
	smíme předpokládat, že $L > L'$. Volme $\varepsilon \coloneqq (L-L') / 3$. K
	tomuto $\varepsilon$ existují z
	\hyperref[def:oboustranna-limita-funkce]{definice limity} $\delta_1>0,
		\delta_2>0$ takové,
	že
	\[
		\forall x \in R(a,\delta_1): f(x) \in B(L,\varepsilon).
	\]
	a rovněž
	\[
		\forall x \in R(a,\delta_2): f(x) \in B(L',\varepsilon).
	\]
	Volíme-li ovšem $\delta \coloneqq \min(\delta_1,\delta_2)$, pak pro $x \in
		R(a,\delta)$ dostaneme
	\[
		f(x) \in B(L,\varepsilon) \cap B(L',\varepsilon).
	\]
	Poslední vztah lze přepsat do tvaru
	\begin{align*}
		L - \varepsilon  & < f(x) < L + \varepsilon,  \\
		L' - \varepsilon & < f(x) < L' + \varepsilon.
	\end{align*}
	Odtud plyne, že
	\[
		L - \varepsilon < L' + \varepsilon,
	\]
	což po dosazení $\varepsilon = (L - L') / 3$ a následné úpravě vede na
	\[
		2L - L' < 2L' - L,
	\]
	z čehož ihned
	\[
		L < L',
	\]
	což je spor.

	Nyní ať například $L = \infty$ a $L' \in \R$. Z
	\hyperref[def:okoli-a-prstencove-okoli-bodu]{definice okolí} $B(L,\varepsilon)$
	pro $L = \infty$, stačí nalézt $\varepsilon > 0$ takové, že
	\[
		\frac{1}{\varepsilon} > L' + \varepsilon,
	\]
	pak se totiž nemůže stát, že
	\[
		f(x) \in B(\infty,\varepsilon) \cap B(L',\varepsilon).
	\]
	Snadným výpočtem zjistíme, že
	\[
		\frac{1}{\varepsilon} > L' + \varepsilon
	\]
	právě tehdy, když $\varepsilon < (\sqrt{L'^2 + 4} - L') / 2$. Pro libovolné
	takové $\varepsilon$ tudíž dostáváme spor stejně jako v předchozím případě.

	Ostatní případy se ošetří obdobně.
\end{lemproof}

\begin{figure}[ht]
 \centering
 \begin{tikzpicture}
	\def\del{1}
	\def\eps{1.5}

	\tkzDefPoints{0/0/a,4/2/L,4/-2/Lp}
	\tkzLabelPoint[color=BrickRed,below=1mm](a){$a$}
	\tkzLabelPoint[color=ForestGreen,below left=1mm](L){$L$}
	\tkzLabelPoint[color=ForestGreen,below left=1mm](Lp){$L'$}

	\tkzDefCircle[R](a,\del) \tkzGetPoint{ad}
	\tkzDrawCircle[thick](a,ad)

	\tkzDefCircle[R](L,\eps) \tkzGetPoint{Le}
	\tkzDrawCircle[color=Magenta,thick](L,Le)

	\tkzDefPoints{0/1.2/s,2.8/3.2/t,0/-1.2/s2,2.8/-3.2/t2}
	\draw[bend left=45,color=RoyalBlue,-latex] (s) to node[midway,above left]{$f$}
	(t);
	\draw[bend right=45,color=RoyalBlue,-latex] (s2) to node[midway,below
		left]{$f$} (t2);

	\tkzDefPointOnCircle[R = center a angle 150 radius \del] \tkzGetPoint{B}
	\tkzDrawPoint[color=BrickRed,size=3](B)
	\tkzDrawSegment[color=BrickRed,dashed,thick](a,B)
	\tkzLabelSegment[color=BrickRed,above right=-1mm](a,B){$\delta$}
	\tkzDrawPoint[draw=BrickRed,thick,size=6,fill=white](a)

	\tkzDrawSegment[decorate,decoration={brace,
				amplitude=10pt},color=Aquamarine,thick](L,Lp)
	\tkzLabelSegment[right=3mm,color=Aquamarine](L,Lp){$\displaystyle
			\frac{|L-L'|}{2}$}

	\tkzDefPointOnCircle[R = center L angle 150 radius \eps] \tkzGetPoint{C}
	\tkzDrawPoint[color=Magenta,size=3](C)
	\tkzDrawSegment[color=Magenta,dashed,thick](L,C)
	\tkzLabelSegment[color=Magenta,above right=-1mm](L,C){$\varepsilon$}
	\tkzDrawPoint[color=ForestGreen,size=6](L)

	\tkzDefCircle[R](Lp,\eps) \tkzGetPoint{Lpe}
	\tkzDrawCircle[color=Magenta,thick](Lp,Lpe)

	\tkzDefPointOnCircle[R = center Lp angle 150 radius \eps] \tkzGetPoint{D}
	\tkzDrawPoint[color=Magenta,size=3](D)
	\tkzDrawSegment[color=Magenta,dashed,thick](Lp,D)
	\tkzLabelSegment[color=Magenta,above right=-1mm](Lp,D){$\varepsilon$}
	\tkzDrawPoint[color=ForestGreen,size=6](Lp)

 \end{tikzpicture}
 \caption{Spor v důkazu \myref{lemmatu}{lem:jednoznacnost-limity}.}
 \label{fig:jednoznacnost-limity}
\end{figure}

\begin{lemma}{}{ma-limitu-je-omezena}
 Ať reálná funkce $f$ má \textbf{konečnou} limitu $L \in \R$ v bodě $a \in
 \R^{*}$. Pak existuje prstencové okolí $a$, na němž je $f$ omezená.
\end{lemma}
\begin{lemproof}
 Pro dané $\varepsilon>0$ nalezneme z
 \hyperref[def:oboustranna-limita-funkce]{definice limity} $\delta>0$ takové, že
 pro $x \in R(a,\delta)$ platí $f(x) \in B(L,\varepsilon)$. Protože však
 $B(L,\varepsilon) = (L-\varepsilon,L+\varepsilon)$ platí pro $x \in
 R(a,\delta)$ odhady
 \[
  L-\varepsilon \leq f(x) \leq L+\varepsilon,
 \]
 čili je $f$ na $R(a,\delta)$ omezená.
\end{lemproof}

Vzhledem k základním aritmetickým operacím si limity funkcí počínají vychovaně.
Za předpokladu, že výsledný výraz dává smysl, můžeme spočítat limitu součtu,
součinu či podílu funkcí jako součet, součin či podíl limit těchto funkcí.

\begin{theorem}{Aritmetika limit funkcí}{aritmetika-limit-funkci}
 Ať $f,g$ jsou reálné funkce a $a \in \R^{*}$. Předpokládejme, že $\lim_{x \to
 a} f(x)$ i $\lim_{x \to a} g(x)$ existují a označme je po řadě $L_f$ a $L_g$.
 Potom platí
 \begin{enumerate}[label=(\alph*)]
  \item $\lim_{x \to a} (f + g)(x) = L_f + L_g$, dává-li výraz napravo smysl.
  \item $\lim_{x \to a} (f \cdot g)(x) = L_f \cdot L_g$, dává-li výraz napravo
   smysl.
  \item $\lim_{x \to a} (f / g)(x) = L_f / L_g$, dává-li výraz napravo smysl.
 \end{enumerate}
\end{theorem}

\begin{thmproof}
 Dokážeme pouze část (c), neboť je výpočetně nejnáročnější, ač nepřináší mnoho
 intuice. Část (a) je triviální a (b) je lehká. Vyzýváme čtenáře, aby se je
 pokusili dokázat sami.

 Už jen v důkazu samotné části (c) bychom správně měli rozlišit šest různých
 případů:
 \begin{enumerate}
  \item $L_f \in \R, L_g \in \R \setminus \{0\}$,
  \item $L_f \in \R, L_g \in \{-\infty,\infty\}$,
  \item $L_f = \infty, L_g \in (0,\infty)$,
  \item $L_f = \infty, L_g \in (-\infty,0)$,
  \item $L_f = -\infty, L_g \in (0,\infty)$,
  \item $L_f = -\infty, L_g \in (-\infty,0)$.
 \end{enumerate}
 Jelikož se výpočty limit v oněch případech liší vzájemně pramálo a získaná
 intuice je asymptoticky rovna té ze znalosti metod řešení exponenciálních
 rovnic, soustředíme se pouze na (nejzajímavější) případ (1).

 Ať tedy $L_f \in \R, L_g \in \R \setminus \{0\}$. Je nejprve dobré si uvědomit,
 že $L_f / L_g$ není definován \textbf{nikdy}, pokud $L_g = 0$, bez ohledu na
 hodnotu $L_f$. Totiž, hodnoty $g$ se mohou k $L_g$ limitně blížit zprava,
 zleva či střídavě z obou směrů. Nelze tudíž obecně určit, zda dělíme čím dál
 tím menším kladným číslem, či čím dál tím větším záporným číslem.

 Položme $\varepsilon_g = |L_g| / 2$. K tomuto $\varepsilon_g$ existuje z
 \hyperref[def:oboustranna-limita-funkce]{definice limity} $\delta_g$ takové, že
 pro $x \in R(a,\delta_g)$ platí $g(x) \in B(L_g,\varepsilon_g)$. Poslední vztah
 si přepíšeme na
 \begin{align*}
  L_g - \varepsilon_g & < g(x) < L_g + \varepsilon_g,\\
  L_g - \frac{|L_g|}{2} & < g(x) < L_g + \frac{|L_g|}{2}.
 \end{align*}
 Speciálně tedy pro $x \in R(a,\delta_g)$ máme odhad
 \[
  |g(x)| > \left| L_g - \frac{|L_g|}{2} \right| > \frac{|L_g|}{2}.
 \]
 Jelikož poslední výraz je z předpokladu kladný, má výraz $f(x) / g(x)$ smysl
 pro každé $x \in R(a,\delta_g)$, neboť pro tato $x$ platí $g(x) \neq 0$.

 Pro $x \in R(a,\delta_g)$ odhadujme
 \begin{align*}
  \left| \frac{f(x)}{g(x)} - \frac{L_f}{L_g} \right| &= \frac{|f(x)L_g -
  g(x)L_f|}{|g(x)||L_g|} = \frac{|f(x) L_g - L_f L_g + L_f L_g -
  g(x)L_f|}{|g(x)| |L_g|}\\
  																									 & \leq \frac{|L_g| |f(x) -
  																									 L_f| + |L_f| |L_g -
  																									g(x)|}{|g(x)| |L_g|}\\
  																									 &= \frac{1}{|g(x)|}
  																									 |f(x) - L_f| +
  																									 \frac{|L_f|}{|g(x)|
  																									 |L_g|}|L_g - g(x)|\\
  																									 &< \frac{2}{|L_g|}|f(x) -
  																									 L_f| + \frac{2
  																									 |L_f|}{|L_g|^2}|L_g -
  																									 g(x)|\\
  																									 & \leq c(|f(x) - L_f| +
  																									 |L_g - g(x)|)
 \end{align*}
 pro $c \coloneqq \max(2 / |L_g|, 2|L_f|/|L_g|^2)$.

 Ať je nyní dáno $\varepsilon>0$. K číslu $\varepsilon / 2c$ existují z
 \hyperref[def:oboustranna-limita-funkce]{definice limity} $\delta_1,\delta_2>0$
 taková, že
 \begin{align*}
 	\forall x \in R(a,\delta_1)&: |g(x) - L_g| < \frac{\varepsilon}{2c},\\
 	\forall x \in R(a,\delta_2)&: |f(x) - L_f| < \frac{\varepsilon}{2c}.
 \end{align*}
 Položíme-li nyní $\delta \coloneqq \min(\delta_1,\delta_2,\delta_g)$, pak pro
 $x \in R(a,\delta)$ platí
 \[
  \left| \frac{f(x)}{g(x)} - \frac{L_f}{L_g} \right| < c (|f(x) - L_f| + |L_g -
  g(x)|) < c \left(\frac{\varepsilon}{2c} + \frac{\varepsilon}{2c}\right) =
  \varepsilon,
 \]
 což dokazuje rovnost $\lim_{x \to a} (f / g)(x) = L_f / L_g$.
\end{thmproof}

\begin{warning}{}{definovanost-pravych-stran}
 Předpoklad \emph{definovanosti} výsledného výrazu ve znění
 \hyperref[thm:aritmetika-limit-funkci]{věty o aritmetice limit} je zásadní.

 Uvažme funkce $f(x) = x + c$ pro libovolné $c \in \R$, $g(x) = -x$. Pak platí
 \begin{align*}
  \lim_{x \to \infty} f(x) &= \infty,\\
  \lim_{x \to \infty} g(x) &= -\infty,\\
  \lim_{x \to \infty} (f(x) + g(x)) &= c,
 \end{align*}
 ale $\lim_{x \to \infty} f(x) + \lim_{x \to \infty} g(x)$ není definován.
\end{warning}

\begin{exercise}{}{aritmetika-limit}
 Dokažte tvrzení (b) a (c) ve \myref{větě}{thm:aritmetika-limit-funkci}.
\end{exercise}

\begin{problem}{}{vypocet-pres-aritmetiku-limit}
 Spočtěte
 \[
  \lim_{x \to \infty} \frac{(2x - 3)^{20} (3x + 2)^{30}}{(2x + 1)^{50}}.
 \]
\end{problem}
\begin{probsol}
 Jelikož limitním bodem je $\infty$, neliší se výpočet limity funkce v zásadě
 nijak od výpočtu limity sesterské posloupnosti. Stále je třeba identifikovat a
 vytknout \uv{nejrychleji rostoucí} členy z čitatele a jmenovatele zlomky a poté
 se odkázat na \hyperref[thm:aritmetika-limit-funkci]{aritmetiku limit}.

 Přímým dosazením zjistíme, že bez dalších úprav vychází limitní výraz $\infty /
 \infty$, na jehož základě nelze nic rozhodnout. Upravujeme tudíž následující
 způsobem:
 \[
  \frac{(2x-3)^{20}(3x+2)^{30}}{(2x+1)^{50}} = \frac{x^{20}(2 -
  \frac{3}{x^{20}}) \cdot x^{30}(3 + \frac{2}{x^{30}})}{x^{50}(2 +
 	\frac{1}{x^{50}})} = \frac{x^{50}}{x^{50}} \cdot \frac{(2 - \frac{3}{x^{20}})(3
 	+ \frac{2}{x^{30}})}{2 + \frac{1}{x^{50}}}.
 \]
 Předpokládajíce definovanost výsledného výrazu (již je třeba ověřit až na
 samotném konci výpočtu), smíme z
 \hyperref[thm:aritmetika-limit-funkci]{aritmetiky limit} tvrdit, že
 \[
  \lim_{x \to \infty} \frac{(2x-3)^{20}(3x+2)^{30}}{(2x+1)^{50}} = \lim_{x \to
  \infty} \frac{x^{50}}{x^{50}} \cdot \lim_{x \to \infty} \frac{(2 -
  \frac{3}{x^{20}})(3 + \frac{2}{x^{30}})}{2 + \frac{1}{x^{50}}}.
 \]
 Zřejmě platí
 \[
  \lim_{x \to \infty} \frac{x^{50}}{x^{50}} = 1.
 \]
 Opět použitím \hyperref[thm:aritmetika-limit-funkci]{aritmetiky limit} můžeme
 počítat
 \[
  \lim_{x \to \infty} \frac{(2 - x^{20})(3 + \frac{2}{x^{30}})}{2 +
  \frac{1}{x^{50}}} = \frac{\lim_{x \to \infty} (2 - \frac{1}{x^{20}}) \cdot
 	\lim_{x \to \infty} (3 + \frac{2}{x^{30}})}{\lim_{x \to \infty} (2 +
 	\frac{1}{x^{50}})} = \frac{(2 - 0) \cdot (3 + 0)}{2 + 0} = 3.
 \]
 Celkem tedy
 \[
  \lim_{x \to \infty} \frac{(2x-3)^{20}(3x+2)^{30}}{(2x+1)^{50}} = 1 \cdot 3 =
  3.
 \]
 Protože výsledný výraz je definován, byla
 \hyperref[thm:aritmetika-limit-funkci]{věta o aritmetice limit} použita
 korektně.
\end{probsol}

\begin{problem}{}{vypocet-pres-aritmetiku-limit-2}
 Spočtěte
 \[
  \lim_{x \to 3} \frac{\sqrt{x+13} - 2 \sqrt{x+1}}{x^2 - 9}
 \]
\end{problem}
\begin{probsol}
 Úlohy na výpočet limit funkcí v bodech jiných než $ \pm \infty$ jsou
 fundamentálně rozdílné od výpočtu limit posloupností. Nelze již rozumně hovořit
 o \uv{rychlosti růstu některého členu} či podobných konceptech. Výpočet se
 pochopitelně stále opírá o \hyperref[thm:aritmetika-limit-funkci]{větu o
 aritmetice limit}, ale často dožaduje jiných algebraických úprav včetně dělení
 mnohočlenů.

 Dosazením $x = 3$ do zadaného výrazu získáme $0 / 0$, tedy je třeba pro výpočet
 limity výraz nejprve upravit.

 Zde postupujeme takto:
 \begin{align*}
  \frac{\sqrt{x + 13} - 2 \sqrt{x+1}}{x^2 - 9} &= \frac{\sqrt{x+13} - 2
  \sqrt{x+1}}{(x-3)(x+3)} \cdot \frac{\sqrt{x+13} + 2 \sqrt{x+1}}{\sqrt{x + 13}
 	+ 2 \sqrt{x+1}}\\
 																							 &= \frac{(x+13) -
 																							 4(x+1)}{(x-3)(x+3)(\sqrt{x+13} +
 																							 2 \sqrt{x+1})}.
 \end{align*}
 Nyní,
 \[
  (x + 13) - 4(x + 1) = -3x + 9 = -3(x-3).
 \]
 Pročež,
 \[
 	\frac{(x + 13) - 4(x + 1)}{(x-3)(x+3)(\sqrt{x+13} + 2 \sqrt{x+1})} =
 	\frac{-3}{(x+3)(\sqrt{x+13} + 2 \sqrt{x+1})}.
 \]
 Z \hyperref[thm:aritmetika-limit-funkci]{aritmetiky limit} máme
 \begin{align*}
  \lim_{x \to 3} \frac{\sqrt{x+13} - 2 \sqrt{x+1}}{x^2 - 9} &= \lim_{x \to 3}
  \frac{-3}{(x+3)(\sqrt{x+13} + 2 \sqrt{x+1})}\\
  																													&= \frac{-3}{(3 +
  																													3)(\sqrt{3 + 13} + 2
  																												 \sqrt{3 + 1})} =
  																												 \frac{-3}{48}.
 \end{align*}
 Protože je konečný výraz definovaný, směli jsme použít
 \hyperref[thm:aritmetika-limit-funkci]{větu o aritmetice limit}.
\end{probsol}

V důkazu \hyperref[thm:aritmetika-limit-funkci]{věty o aritmetice limit} jsme
zmínili, že na jejím základě nelze nic rozhodnout v případě, že konečný výraz
vyjde $a / 0$, kde $a \in \R^{*}$. K rozřešení právě těchto situací slouží
následující tvrzení.

\begin{proposition}{}{limita-a/0}
 Ať $f,g$ jsou reálné funkce, $a \in \R^{*}$. Dále ať $\lim_{x \to a} f(x) =
 A \in \R^{*}, A > 0$, $\lim_{x \to a} g(x) = 0$ a existuje prstencové okolí
 bodu $a$, na němž je $g$ kladná.

 Potom $\lim_{x \to a} f(x) / g(x) = \infty$.
\end{proposition}
\begin{propproof}
 Ať je z předpokladu dáno $\eta>0$ takové, že pro $x \in R(a,\eta)$ je $g(x) >
 0$. Rozlišíme dva případy.

 Prvním případ nastává, když $A \in \R$ je číslo. Mějme dáno $\varepsilon>0$.
 Jelikož $\lim_{x \to a} f(x) = A$ a $A > 0$, nalezneme pro $A / 2$ číslo
 $\delta_1 > 0$ takové, že pro $x \in R(a,\delta_1)$ platí
 \[
  f(x) \in B \left( A,\frac{A}{2} \right) = \left( \frac{A}{2}, \frac{3A}{2}
  \right),
 \]
 čili $f(x) > A / 2$. Podobně, za předpokladu $\lim_{x \to a} g(x) = 0$
 nalezneme $\delta_2>0$ takové, že pro $x \in R(a,\delta_2)$ platí
 \[
  g(x) \in B \left( 0,\frac{A}{2\varepsilon} \right) =
  \left(-\frac{A}{2\varepsilon},\frac{A}{2\varepsilon}\right),
 \]
 tedy speciálně $g(x) < A / 2\varepsilon$, z čehož dostáváme $1 / g(x) >
 2\varepsilon / A$. Celkově pro $\delta \coloneqq \min(\delta_1,\delta_2,\eta)$
 a $x \in R(a,\delta)$ můžeme počítat
 \[
  \left| \frac{f(x)}{g(x)} \right| = \frac{f(x)}{g(x)} > \frac{A}{2} \cdot
  \frac{2\varepsilon}{A} = \varepsilon,
 \]
 kde první rovnost plyne z toho, že pro $x \in R(a,\delta)$ platí $f(x) > 0$ i
 $g(x) > 0$. To dokazuje, že $\lim_{x \to a} f(x) / g(x) = \infty$ v případě
 $A \in \R$.

 Ošetřeme případ $A = \infty$. Argumentujíce analogicky předchozímu odstavci
 nalezneme $\delta_1>0$ takové, že pro $R(a,\delta_1)$ platí $f(x) > 1$ a pro
 dané $\varepsilon>0$ nalezneme $\delta_2>0$ takové, že pro $x \in
 R(a,\delta_2)$ platí $g(x) < 1 / \varepsilon$, a tedy $1 / g(x) > \varepsilon$.
 Potom, pro $x \in R(a,\min(\eta,\delta_1,\delta_2))$ platí
 \[
  \left| \frac{f(x)}{g(x)} \right| = \frac{f(x)}{g(x)} > 1 \cdot \varepsilon =
  \varepsilon,
 \]
 což dokazuje opět, že $\lim_{x \to a} f(x) / g(x) =\infty$ i v případě $A =
 \infty$. 

 Tím je důkaz završen.
\end{propproof}

\begin{remark}{}{poznamka-k-a/0}
 \hyperref[prop:limita-a/0]{Předchozí tvrzení} pochopitelně platí i při záměně
 ostrých nerovností v jeho znění. Konkrétně, za předpokladů
 \begin{itemize}
  \item[($< >$)] $\lim_{x \to a} f(x) = A < 0$ a $g(x) > 0$ na $R(a,\eta)$ platí
   $\lim_{x \to a} f(x) / g(x) = -\infty$;
  \item[($> <$)] $\lim_{x \to a} f(x) = A > 0$ a $g(x) < 0$ na $R(a,\eta)$ platí
   $\lim_{x \to a} f(x) / g(x) = -\infty$;
  \item[($< <$)] $\lim_{x \to a} f(x) = A < 0$ a $g(x) < 0$ na $R(a,\eta)$ platí
   $\lim_{x \to a} f(x) / g(x) = \infty$.
 \end{itemize}
 Důkazy všech těchto případů jsou identické důkazu
 \hyperref[prop:limita-a/0]{původního tvrzení}.
\end{remark}

Posledním základním tvrzením o limitách funkcí je vztah limit a uspořádání
reálných čísel, vlastně jakási varianta \myref{lemmatu}{lem:o-dvou-straznicich}
pro posloupnosti.

\begin{theorem}{O srovnání}{o-srovnani}
 Ať $a \in \R^{*}$ a $f,g,h$ jsou reálné funkce.
 \begin{enumerate}[label=(\alph*)]
  \item Pokud
   \[
    \lim_{x \to a} f(x) > \lim_{x \to a} g(x),
   \]
   pak existuje prstencové okolí bodu $a$, na němž $f > g$.
  \item Existuje-li prstencové okolí bodu $a$, na němž platí $f \leq g$, pak
   \[
    \lim_{x \to a} f(x) \leq \lim_{x \to a} g(x).
   \]
  \item Existuje-li prstencové okolí $a$, na němž $f \leq h \leq g$ a $\lim_{x
   \to a} f(x) = \lim_{x \to a} g(x) = A \in \R^{*}$, pak existuje rovněž
   $\lim_{x \to a} h(x)$ a jest rovna $A$.
 \end{enumerate}
\end{theorem}

\begin{thmproof}
 Položme $L_f \coloneqq \lim_{x \to a} f(x)$ a $L_g \coloneqq \lim_{x \to a}
 g(x)$.

 Dokážeme (a). Protože $L_f > L_g$, existuje $\varepsilon > 0$ takové, že $L_f -
 L_g > 2\varepsilon$. K tomuto $\varepsilon$ nalezneme z
 \hyperref[def:oboustranna-limita-funkce]{definice limity} $\delta_f > 0$ a
 $\delta_g > 0$ taková, že
 \begin{align*}
  \forall x \in R(a,\delta_f) &: f(x) \in B(L_f,\varepsilon),\\
  \forall x \in R(a,\delta_g)&: g(x) \in B(L_g,\varepsilon).
 \end{align*}
 To ovšem znamená, že pro $x \in R(a,\min(\delta_f,\delta_g))$ platí jak
 \[
  f(x) > L_f - \varepsilon,
 \]
 tak
 \[
  g(x) < L_g + \varepsilon,
 \]
 čili
 \[
  f(x) - g(x) > L_f - \varepsilon - L_g - \varepsilon = L_f - L_g - 2\varepsilon
  > 0,
 \]
 jak jsme chtěli.

 Část (b) dokážeme sporem. Ať $L_f > L_g$. Podle (a) pak existuje prstencové
 okolí $R(a,\delta)$ bodu $a$, na němž $f > g$. Ovšem, podle předpokladu
 existuje rovněž okolí $R(a,\eta)$ bodu $a$, kde zase $f \leq g$. Vezmeme-li
 tudíž $x \in R(a,\min(\delta,\eta))$, pak $f(x) > g(x) \geq f(x)$, což je spor.

 V důkazu (c) rozlišíme dva případy. Položme $L \coloneqq L_f = L_g$ a ať
 nejprve $L \in \R$. Pro dané $\varepsilon>0$ existují $\delta_f,\delta_g>0$
 taková, že pro $x \in R(a,\delta_f)$ platí
 \[
  L - \varepsilon< f(x) < L + \varepsilon
 \]
 a pro $x \in R(a,\delta_g)$ zas
 \[
  L - \varepsilon < g(x) < L + \varepsilon.
 \]
 Z předpokladu existuje prstencové okolí $R(a,\eta)$, na němž $f \leq h \leq g$.
 Pročež, pro $x \in R(a,\min(\delta_f,\delta_g,\eta))$ máme
 \[
  L - \varepsilon < f(x) \leq h(x) \leq g(x) < L + \varepsilon,
 \]
 z čehož plyne $h(x) \in B(L,\varepsilon)$, neboli $\lim_{x \to a} h(x) = L$.

 Pro $L = \infty$ postupujeme jednodušeji, neboť stačí dolní odhad. K danému
 $\varepsilon>0$ nalezneme $\delta>0$ takové, že pro $x \in R(a,\delta)$ platí
 \[
  f(x) > \frac{1}{\varepsilon}.
 \]
 Pak pro $x \in R(a,\min(\delta,\eta))$ máme odhad
 \[
  \frac{1}{\varepsilon} < f(x) \leq h(x),
 \]
 čili $h(x) \in B(\infty,\varepsilon)$, což dokazuje rovnost $\lim_{x \to a}
 h(x) = \infty$. 

 Případ $L = -\infty$ se ošetří horním odhadem funkcí $g$.
\end{thmproof}

\begin{exercise}{}{}
 Spočtěte následující limity funkcí
 \begin{align*}
  &\lim_{x \to 1} \frac{3x^{4} - 4x^3 + 1}{(x-1)^2},\\
  &\lim_{x \to 0} \frac{3x + \frac{2}{x}}{x + \frac{4}{x}},\\
  &\lim_{x \to -\infty} \sqrt{x^2 + x} - x,\\
  &\lim_{x \to \infty} \frac{\sqrt{x-1}-2x}{x-7}.
 \end{align*}
\end{exercise}

