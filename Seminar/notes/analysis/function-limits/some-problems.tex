\section{Pár příkladů na konec}
\label{sec:par-prikladu-na-konec}

Účelem této \uv{přiložené} sekce je ukázat na dvou zajímavých příkladech obvyklé
metody práce s~limitami funkcí. Doufáme, že čtenářům dobře pomůže uchápění
tohoto tématu, snad náročnějšího k vnětí než limity posloupností a součty řad.

\begin{example}{}{periodicka-slozeno-rostouci}
 Ať $g$ je rostoucí a spojitá funkce na $[1,\infty)$ s $\lim_{x \to \infty} g(x)
 = \infty$ a $f$ je \textbf{nekonstantní periodická} funkce na $\R$. Pak
 $\lim_{x \to \infty} (f \circ g)(x)$ neexistuje.

 Pro důkaz neexistence limity máme (pochopitelně kromě samotné
 \hyperref[def:oboustranna-limita-funkce]{definice}) zatím pouze dva nástroje --
 \hyperref[def:jednostranna-limita-funkce]{jednostranné limity} a
 \hyperref[thm:heineho]{Heineho větu}. Protože limitním bodem je $\infty$,
 použití jednostranných limit není možné. Zkusíme tedy
 \hyperref[thm:heineho]{Heineho větu}.

 Nejprve si uvědomíme, že z \myref{důsledku}{cor:spojita-interval} je
 $g([1,\infty))$ je interval. Tento navíc není shora omezen, bo $\lim_{x \to
 \infty} g(x) = \infty$. Na oba tyto fakty se budeme vícekrát odvolávat. Označme
 rovněž písmenem $p > 0$ periodu funkce $f$.

 Položme nyní $x_0 \coloneqq 1$ a označme $y_0 \coloneqq g(x_0)$, $A \coloneqq
 f(y_0)$. Induktivně sestrojíme posloupnosti $x:\N \to [1,\infty)$ a $y:\N \to
 \R$. Předpokládejme, že jsou dány členy $x_0,\ldots,x_k$ a $y_0,\ldots,y_k$,
 kde $y_i = g(x_i)$ pro každé $i \leq k$. Položíme $y_{k+1} \coloneqq y_k + p$.
 Potom $f(y_{k+1}) = f(y_k)$. Protože $g([1,\infty)) = [y_0,\infty)$ a $y_{k+1}
 > y_0$, nalezneme $x_{k+1} \in [1,\infty)$ takové, že $g(x_{k+1}) = y_{k+1}$.
 Jelikož $g$ je rostoucí a
 \[
  g(x_{k+1}) = y_{k+1} > y_k = g(x_k),
 \]
 rovněž $x_{k+1} > x_k$. Celkem máme $x_{n+1} > x_n$ pro každé $n \in \N$, čili
 $\lim_{n \to \infty} x_n = \infty$. Rovněž
 \[
  \lim_{n \to \infty} (f \circ g)(x_n) = \lim_{n \to \infty} f(y_n) = \lim_{n
  \to \infty} A = A,
 \]
 neboť členy posloupnosti $y$ jsou od sebe vzdáleny o periodu $p$ funkce $f$.

 Konečně, nalezneme jinou posloupnost $\tilde{x}:\N \to [1,\infty)$ takovou, že
 $\lim_{n \to \infty} (f \circ g)(\tilde{x}_n) \neq A$, čímž završíme důkaz.
 Volme libovolné $0 < \varepsilon < p$. K tomuto $\varepsilon$ nalezneme
 $\delta>0$ takové, že
 \[
  |g(x_0 + \delta) - g(x_0)| < \varepsilon.
 \]
 Toto $\delta$ vskutku existuje pro to, že $g$ je rostoucí -- tudíž $g(x_0 +
 \delta) > g(x_0)$ -- a spojitá -- tudíž dvě různé funkční hodnoty lze volit
 nekonečně blízké.

 Položíme $\tilde{x}_0 \coloneqq x_0 + \delta$ a $\tilde{y}_0 \coloneqq
 g(\tilde{x}_0)$. Potom $f(\tilde{y}_0) = B \neq A$, ježto $\tilde{y}_0 \in
 (y_0, y_0 + p)$. Podobně jako dříve sestrojíme posloupnost $\tilde{x}_n$
 takovou, že $\lim_{n \to \infty} \tilde{x}_n = \infty$ a $(f \circ
 g)(\tilde{x}_n) = B$ pro každé $n \in \N$.

 Potom ale platí
 \[
  \lim_{n \to \infty} (f \circ g)(\tilde{x}_n) = B \neq A = \lim_{n \to \infty}
  (f \circ g)(x_n),
 \]
 čili z \hyperref[thm:heineho]{Heineho věty} $\lim_{x \to \infty} (f \circ
 g)(x)$ neexistuje.
\end{example}

\begin{figure}[ht]
 \centering
 \begin{tikzpicture}
  \tkzInit[xmin=-1,xmax=8,ymin=-1.5,ymax=1.5]
  \tkzDrawX[label=]
  \tkzDrawY[label=]

  \draw[color=Fuchsia,thick,smooth,domain=0:7.5,samples=1000] plot
   (\x,{abs(sin(deg(\x)))}) node[above] {$\clm{f}$};
  \tkzDefPoints{1.5708/0/y0,4.7124/0/y1}
  \tkzDefPoints{2.6179/0/yt0,5.7596/0/yt1}
  \tkzDefPoints{1.5708/1/A1,4.7124/1/A2}
  \tkzDefPoints{2.6179/0.5/B1,5.7596/0.5/B2}

  \tkzDrawSegments[dashed](y0,A1 y1,A2 yt0,B1 yt1,B2)

  \tkzDrawPoints[size=4,color=BrickRed](y0,y1)
  \tkzDrawPoints[size=4,color=RoyalBlue](yt0,yt1)

  \tkzLabelPoint[below=1mm](y0){$\clr{y_0}$}
  \tkzLabelPoint[below=1mm](y1){$\clr{y_1}$}
  \tkzLabelPoint[below=1mm](yt0){$\clb{\tilde{y}_0}$}
  \tkzLabelPoint[below=1mm](yt1){$\clb{\tilde{y}_1}$}

  \tkzDrawPoints[size=4,color=Fuchsia](A1,A2,B1,B2)
  \tkzLabelPoint[above=1mm](A1){$\clm{A}$}
  \tkzLabelPoint[above=1mm](A2){$\clm{A}$}
  \tkzLabelPoint[above=1mm](B1){$\clm{B}$}
  \tkzLabelPoint[above=1mm](B2){$\clm{B}$}

  \draw[decorate,decoration={brace,raise=6mm,mirror,amplitude=10pt}] (y0) --
   (y1) node[pos=0.5,below=9mm] {$p$};
  \draw[decorate,decoration={brace,raise=1.5mm,amplitude=5pt}] (y0) -- (yt0)
   node[pos=0.5,above=2.5mm] {$\varepsilon$};
 \end{tikzpicture}

 \caption{Posloupnosti $\clr{y}$ a $\clb{\tilde{y}}$ z
 \myref{příkladu}{exam:periodicka-slozeno-rostouci}.}
 \label{fig:periodicka-slozeno-rostouci}
\end{figure}

\begin{example}{Singularity funkce}{singularity-funkce}
 Ať $a \in M$ a $f:M \to \R$ je reálná funkce definovaná aspoň na prstencovém
 okolí bodu $a$. Předpokládejme, že $f$ není spojitá v $a$. Pak řekneme, že $f$
 má v $a$
 \begin{itemize}
  \item \textbf{odstranitelnou singularitu}, když existuje konečná
   $\lim_{x \to a} f(x)$. V tomto případě můžeme dodefinovat $f(a) \coloneqq
   \lim_{x \to a} f(x)$;
  \item \textbf{pól}, když existují $\lim_{x \to a^{+}} f(x)$ a
   $\lim_{x \to a^{-}} f(x)$, ale nejsou si rovny;
  \item \textbf{neodstranitelnou singularitu}, když aspoň jedna z jednostranných
   limit $f$ v bodě $a$ neexistuje.
 \end{itemize}
 Singularity funkcí tvoří důležitou část komplexní analýzy, kde poskytují obraz
 zejména o stabilitě fyzikálních systému popsaných těmito funkcemi.
 \emph{Odstranitelné singularity} jsou velmi stabilní a většinou způsobeny pouze
 chybami v měření. \emph{Póly} jsou stabilní při vhodné aproximaci, avšak v
 grafu komplexních funkcí jedné proměnné vypadají vlastně jako nekonečné stále
 se zúžující tuby. Lze si je představovat například jako
 \href{https://cs.wikipedia.org/wiki/Gabriel%C5%AFv_roh}{Gabrielův roh}. Vhodnou
  aproximací je zde uříznutí tohoto tělesa ve zvolené \uv{výšce}. Konečně,
  \emph{neodstranitelné singularity} jsou vskutku neodstranitelné. Dokonce platí
  věta, že komplexní funkce jedné proměnné s neodstranitelnou singularitou
  nabývají \textbf{úplně všech} hodnot z $\C$ na libovolně malém okolí této
  singularity. Menší stability již dosáhnout nelze. Neodstranitelné singularity
  si, tvrdíme, nelze ani rozumně představit.

 Ukážeme, že aspoň pro funkce jedné \emph{reálné} proměnné, jimiž se v tomto
 textu zabýváme, jsou množiny všech odstranitelných singularit i pólů spočetné.

 K důkazu prvního tvrzení uvažme množiny
 \[
  A \coloneqq \{x \in \R \mid \lim_{t \to x} f(t) < f(x)\} \quad \text{a} \quad
  B \coloneqq \{x \in \R \mid \lim_{t \to x} f(t) > f(x)\}.
 \]
 Platí, že množina odstranitelných singularit $f$ je rovna $A \cup B$. Zjevně
 tedy stačí dokázat, že každá z obou množin je spočetná. Provedeme onen důkaz
 pro množinu $A$, pro $B$ lze vést analogicky.

 Protože jsou \hyperref[prop:hustota-q-v-r]{$\Q$ hustá v $\R$}, nalezneme pro
 každé $x \in A$ racionální číslo $r_x \in \Q$ takové, že
 \[
  \lim_{t \to x} f(t) < r_x < f(x).
 \]
 Pak platí $A = \bigcup_{r \in \Q} A_r$, kde
 \[
  A_r = \{x \in A \mid r_x = r\}.
 \]
 Dokážeme, že každá z množin $A_r$ je spočetná. Volme $r \in \Q$. Protože
 $\lim_{t \to x} f(t) < r$ pro každé $x \in A_r$, nalezneme $\delta_x$ takové,
 že pro $t \in R(x,\delta_x)$ platí $f(t) < r$. Dále nahlédneme, že pro každé
 $x \neq y \in A_r$ platí
 \[
  (x - \frac{1}{2}\delta_x, x + \frac{1}{2}\delta_x) \cap
  (y-\frac{1}{2}\delta_y,y+\frac{1}{2}\delta_y) = \emptyset.
 \]
 Pro spor ať platí opak. Nalezneme $x \neq y \in A_r$ taková, že $|x-y|<\delta_x
 / 2 + \delta_y / 2$. Jest-li $\delta_x \leq \delta_y$, pak $|x-y| \leq
 \delta_y$, čili $f(x) < r_y = r$, neboť $x \in P(y,\delta_y)$. Z definice $r_x$
 však také $f(x) > r_x = r$, což je spor. Případ $\delta_x > \delta_y$ vede
 rovněž ke sporu na základě obdobného argumentu. Odtud plyne, že $A_r$ je
 spočetná, bo každý bod $A_r$ má kolem sebe okolí, v němž neleží žádné jiné body
 $A_r$. To mimo jiné znamená, že bodů $A_r$ je nejvýše tolik, jak racionálních
 čísel, tj. spočetně. Ježto $A = \bigcup_{r \in \Q} A_r$, je i $A$ spočetná.

 Nyní dokážeme, že $f$ má i spočetně pólů. Definujme opět množiny
 \begin{align*}
  U &\coloneqq \{x \in \R \mid \lim_{t \to x^{-}} f(t) < \lim_{t \to x^{+}}
  f(t)\},\\
   V &\coloneqq \{x \in \R \mid \lim_{t \to x^{-}} f(t) > \lim_{t \to x^{+}}
   f(t)\}.
 \end{align*}
 Množina pólů funkce $f$ je rovna $U \cup V$. Je třeba ukázat, že každá z množin
 $U,V$ je spočetná. Důkaz pro $U$ bude zjevně symetrický důkazu pro $V$, budeme
 se tudíž zabývat pouze množinou $U$. Pro každé $x \in U$ nalezneme $r_x \in \Q$
 takové, že
 \[
  \lim_{t \to x^{-}} f(t) < r_x < \lim_{t \to x^{+}} f(t).
 \]
 Pak jistě $U = \bigcup_{r \in \Q} U_r$, kde
 \[
  U_r \coloneqq \{x \in U \mid r_x = r\}.
 \]
 Ukážeme, že každá $U_r,r \in \Q$, je spočetná. Pro každé $x \in U_r$ nalezneme
 $\delta_x>0$ takové, že
 \begin{align*}
  \forall t \in R_-(x,\delta_x)&: f(t) < r,\\
  \forall t \in R_+(x,\delta_x)&:f(t) > r.
 \end{align*}
 Podobně jako v důkazu spočetnosti množiny odstranitelných singularit lze snadno
 ukázat, že
 \[
  (x,x+\delta_x) \cap (y-\delta_y,y) = \emptyset
 \]
 pro každá dvě $x \neq y \in U_r$, z čehož plyne, že $U_r$ -- a posléze i $U$ --
 je spočetná.
\end{example}
