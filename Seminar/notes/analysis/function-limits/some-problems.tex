\section{Vybrané úlohy o limitách funkcí}
\label{sec:vybrane-ulohy-o-limitach-funkci}

Účelem této \uv{přiložené} sekce je ukázat na vybraných příkladech a úlohách
obvyklé metody práce s limitami funkcí. Doufáme, že čtenářům dobře poslouží při
uchápění tohoto tématu, snad náročnějšího k vnětí než limity posloupností a
součty řad.

\begin{example}{}{periodicka-slozeno-rostouci}
 Ať $g$ je rostoucí a spojitá funkce na $[1,\infty)$ s $\lim_{x \to \infty} g(x)
 = \infty$ a $f$ je \textbf{nekonstantní periodická} funkce na $\R$. Pak
 $\lim_{x \to \infty} (f \circ g)(x)$ neexistuje.

 Pro důkaz neexistence limity máme (pochopitelně kromě samotné
 \hyperref[def:oboustranna-limita-funkce]{definice}) zatím pouze dva nástroje --
 \hyperref[def:jednostranna-limita-funkce]{jednostranné limity} a
 \hyperref[thm:heineho]{Heineho větu}. Protože limitním bodem je $\infty$,
 použití jednostranných limit není možné. Zkusíme tedy
 \hyperref[thm:heineho]{Heineho větu}.

 Nejprve si uvědomíme, že z \myref{důsledku}{cor:spojita-interval} je
 $g([1,\infty))$ je interval. Tento navíc není shora omezen, bo $\lim_{x \to
 \infty} g(x) = \infty$. Na oba tyto fakty se budeme vícekrát odvolávat. Označme
 rovněž písmenem $p > 0$ periodu funkce $f$.

 Položme nyní $x_0 \coloneqq 1$ a označme $y_0 \coloneqq g(x_0)$, $A \coloneqq
 f(y_0)$. Induktivně sestrojíme posloupnosti $x:\N \to [1,\infty)$ a $y:\N \to
 \R$. Předpokládejme, že jsou dány členy $x_0,\ldots,x_k$ a $y_0,\ldots,y_k$,
 kde $y_i = g(x_i)$ pro každé $i \leq k$. Položíme $y_{k+1} \coloneqq y_k + p$.
 Potom $f(y_{k+1}) = f(y_k)$. Protože $g([1,\infty)) = [y_0,\infty)$ a $y_{k+1}
 > y_0$, nalezneme $x_{k+1} \in [1,\infty)$ takové, že $g(x_{k+1}) = y_{k+1}$.
 Jelikož $g$ je rostoucí a
 \[
  g(x_{k+1}) = y_{k+1} > y_k = g(x_k),
 \]
 rovněž $x_{k+1} > x_k$. Celkem máme $x_{n+1} > x_n$ pro každé $n \in \N$, čili
 $\lim_{n \to \infty} x_n = \infty$. Rovněž
 \[
  \lim_{n \to \infty} (f \circ g)(x_n) = \lim_{n \to \infty} f(y_n) = \lim_{n
  \to \infty} A = A,
 \]
 neboť členy posloupnosti $y$ jsou od sebe vzdáleny o periodu $p$ funkce $f$.

 Konečně, nalezneme jinou posloupnost $\tilde{x}:\N \to [1,\infty)$ takovou, že
 $\lim_{n \to \infty} (f \circ g)(\tilde{x}_n) \neq A$, čímž završíme důkaz.
 Volme libovolné $0 < \varepsilon < p$. K tomuto $\varepsilon$ nalezneme
 $\delta>0$ takové, že
 \[
  |g(x_0 + \delta) - g(x_0)| < \varepsilon.
 \]
 Toto $\delta$ vskutku existuje pro to, že $g$ je rostoucí -- tudíž $g(x_0 +
 \delta) > g(x_0)$ -- a spojitá -- tudíž dvě různé funkční hodnoty lze volit
 nekonečně blízké.

 Položíme $\tilde{x}_0 \coloneqq x_0 + \delta$ a $\tilde{y}_0 \coloneqq
 g(\tilde{x}_0)$. Potom $f(\tilde{y}_0) = B \neq A$, ježto $\tilde{y}_0 \in
 (y_0, y_0 + p)$. Podobně jako dříve sestrojíme posloupnost $\tilde{x}_n$
 takovou, že $\lim_{n \to \infty} \tilde{x}_n = \infty$ a $(f \circ
 g)(\tilde{x}_n) = B$ pro každé $n \in \N$.

 Potom ale platí
 \[
  \lim_{n \to \infty} (f \circ g)(\tilde{x}_n) = B \neq A = \lim_{n \to \infty}
  (f \circ g)(x_n),
 \]
 čili z \hyperref[thm:heineho]{Heineho věty} $\lim_{x \to \infty} (f \circ
 g)(x)$ neexistuje.
\end{example}

\begin{figure}[ht]
 \centering
 \begin{tikzpicture}
  \tkzInit[xmin=-1,xmax=8,ymin=-1.5,ymax=1.5]
  \tkzDrawX[label=]
  \tkzDrawY[label=]

  \draw[color=Fuchsia,thick,smooth,domain=0:7.5,samples=1000] plot
   (\x,{abs(sin(deg(\x)))}) node[above] {$\clm{f}$};
  \tkzDefPoints{1.5708/0/y0,4.7124/0/y1}
  \tkzDefPoints{2.6179/0/yt0,5.7596/0/yt1}
  \tkzDefPoints{1.5708/1/A1,4.7124/1/A2}
  \tkzDefPoints{2.6179/0.5/B1,5.7596/0.5/B2}

  \tkzDrawSegments[dashed](y0,A1 y1,A2 yt0,B1 yt1,B2)

  \tkzDrawPoints[size=4,color=BrickRed](y0,y1)
  \tkzDrawPoints[size=4,color=RoyalBlue](yt0,yt1)

  \tkzLabelPoint[below=1mm](y0){$\clr{y_0}$}
  \tkzLabelPoint[below=1mm](y1){$\clr{y_1}$}
  \tkzLabelPoint[below=1mm](yt0){$\clb{\tilde{y}_0}$}
  \tkzLabelPoint[below=1mm](yt1){$\clb{\tilde{y}_1}$}

  \tkzDrawPoints[size=4,color=Fuchsia](A1,A2,B1,B2)
  \tkzLabelPoint[above=1mm](A1){$\clm{A}$}
  \tkzLabelPoint[above=1mm](A2){$\clm{A}$}
  \tkzLabelPoint[above=1mm](B1){$\clm{B}$}
  \tkzLabelPoint[above=1mm](B2){$\clm{B}$}

  \draw[decorate,decoration={brace,raise=6mm,mirror,amplitude=10pt}] (y0) --
   (y1) node[pos=0.5,below=9mm] {$p$};
  \draw[decorate,decoration={brace,raise=1.5mm,amplitude=5pt}] (y0) -- (yt0)
   node[pos=0.5,above=2.5mm] {$\varepsilon$};
 \end{tikzpicture}

 \caption{Posloupnosti $\clr{y}$ a $\clb{\tilde{y}}$ z
 \myref{příkladu}{exam:periodicka-slozeno-rostouci}.}
 \label{fig:periodicka-slozeno-rostouci}
\end{figure}

