\section{Hlubší poznatky o limitě funkce}
\label{sec:hlubsi-poznatky-o-limite-funkce}

Kapitolu o limitách funkcí dovršíme několika -- v dalším textu zásadními --
tvrzeními. Některá z~nich se vížou na pojem
\hyperref[def:spojita-funkce]{spojitosti funkce}, některá nikoliv. Ukážeme si
rovněž souvislost limit funkcí s limitami posloupností; uzříme, že je to v
jistém širém smyslu týž koncept.

Nejprve se pozastavíme nad chováním limit funkcí vzhledem k jejich skládání. Je
vskutku velmi přirozené -- má-li funkce $g$ limitu $A$ v bodě $a$ a funkce $f$
limitu $B$ v bodě $A$, pak $f \circ g$ má limitu $B$ v bodě $a$, jak by jeden
čekal. Toto tvrzení má však své předpoklady; pro libovolné dvě funkce pravdivé
není.

\begin{theorem}{Limita složené funkce}{limita-slozene-funkce}
 Ať $a,A,B \in \R^{*}$ a $f,g$ jsou reálné funkce. Nechť navíc platí
 \[
  \lim_{x \to a} g(x) = A \quad \text{a} \quad \lim_{y \to A} f(y) = B.
 \]
 Je-li splněna \textbf{aspoň jedna} z podmínek:
 \begin{itemize}
  \item [(R)] existuje prstencové okolí $a$, na němž platí $g \neq A$;
  \item [(S)] funkce $f$ je spojitá v $A$,
 \end{itemize}
 pak
 \[
  \lim_{x \to a} (f \circ g)(x) = B.
 \]
\end{theorem}
\begin{thmproof}
 Důkaz tvrzení není příliš obtížný, ale poněkud technický a alfabeticky náročný.
 Potřebujeme ukázat, že pro libovolné $\varepsilon>0$ existuje $\delta>0$, že
 pro $x \in R(a,\delta)$ máme $f(g(x)) \in B(B,\varepsilon)$. Nechť je tedy
 $\varepsilon>0$ dáno.

 Předpokládejme, že platí (R) a máme $\eta>0$ takové, že $g(x) \neq A$ pro $x
 \in R(a,\eta)$. Ježto platí $\lim_{y \to A} f(y) = B$, existuje k danému
 $\varepsilon>0$ číslo $\delta_f>0$ takové, že pro $y \in R(A,\delta_f)$ platí
 $f(y) \in B(B,\varepsilon)$. K tomuto $\delta_f>0$ pak existuje $\delta_g>0$
 takové, že pro $x \in R(a,\delta_g)$ platí $g(x) \in B(A,\delta_f)$. Volíme-li
 nyní $\delta \coloneqq \min(\eta,\delta_g)$, pak pro $x \in R(a,\delta)$ platí
 \[
  g(x) \in B(A,\delta_f) \quad \text{i} \quad g(x) \neq A,
 \]
 kteréžtě dvě podmínce dávajíta celkem $g(x) \in B(A,\delta_f) \setminus \{A\} =
 R(a,\delta_f)$. Potom však pro $x \in R(a,\delta)$ platí $g(x) \in
 R(A,\delta_f)$, tudíž $f(g(x)) \in B(B,\varepsilon)$, jak jsme chtěli.
  
 Ať naopak platí (S). Pak můžeme pro spojitost $f$ v $A$ brát ve výroku
 \[
 \forall y \in R(A,\delta_f): f(y) \in B(B,\varepsilon)
 \]
 místo prstencového okolí $R(A,\delta_f)$ brát plné okolí $B(A,\delta_f)$, neboť
 $f(A) = \lim_{y \to A} f(y) = B \in B(B,\varepsilon)$. Protože pro $x \in
 R(a,\delta_g)$ však platí $g(x) \in B(A,\delta_f)$, máme rovněž $f(g(x)) \in
 B(B,\varepsilon)$, čímž je důkaz hotov.  
\end{thmproof}

\begin{warning}{}{}
 Platnost aspoň jedné z podmínek (R),(S) v
 \hyperref[thm:limita-slozene-funkce]{předchozí větě} je nezanedbatelná.

 Vezměme $f(y) = |\sign y|$ a $g(x) = 0$. Potom platí
 \[
  \lim_{y \to 0} f(y) = 1 \quad \text{a} \quad \lim_{x \to 0} g(x) = 0,
 \]
 ale $\lim_{x \to 0} (f \circ g)(x) = 0 \neq 1$. Závěr
 \hyperref[thm:limita-slozene-funkce]{předchozí věty} je v tomto případě
 neplatný pro to, že neexistuje okolí $a$, na němž $g \neq 0$ (tj. neplatí (R)),
 ani není $f$ spojitá v $0$ (tj. neplatí (S)).
\end{warning}

\begin{figure}[ht]
 \centering
 \begin{tikzpicture}[scale=1.75]
  \def\dg{0.8}
  \def\df{0.7}
  \def\eps{1}

  \tkzDefPoints{0/0/a,2/1/A,4/2/B}
  \tkzDrawPoint[size=4,color=BrickRed](a)
  \tkzDrawPoint[size=4,color=RoyalBlue](A)
  \tkzDrawPoint[size=4,color=ForestGreen](B)

  \tkzLabelPoint[below=1mm](a){$\clr{a}$}
  \tkzLabelPoint[below=1mm](A){$\clb{A}$}
  \tkzLabelPoint[below=1mm](B){$\clg{B}$}
  
  \tkzDefCircle[R](a,\dg) \tkzGetPoint{ac}
  \tkzDrawCircle[thick,color=BrickRed](a,ac)
  \tkzDefCircle[R](A,\df) \tkzGetPoint{Ac}
  \tkzDrawCircle[thick,color=RoyalBlue](A,Ac)
  \tkzDefCircle[R](B,\dg) \tkzGetPoint{Bc}
  \tkzDrawCircle[thick,color=ForestGreen](B,Bc)

  \tkzDefPointOnCircle[through = center a angle 120 point ac]
  \tkzGetPoint{ad}
  \tkzDrawPoint[size=4,color=BrickRed](ad)
  \tkzDrawSegment[color=BrickRed,dashed,thick](a,ad)
  \tkzLabelSegment[below left](a,ad){$\clr{\delta_g}$}

  \tkzDefPointOnCircle[through = center A angle 120 point Ac]
  \tkzGetPoint{Ad}
  \tkzDrawPoint[size=4,color=RoyalBlue](Ad)
  \tkzDrawSegment[color=RoyalBlue,dashed,thick](A,Ad)
  \tkzLabelSegment[below left](A,Ad){$\clb{\delta_f}$}

  \tkzDefPointOnCircle[through = center B angle 120 point Bc]
  \tkzGetPoint{Bd}
  \tkzDrawPoint[size=4,color=ForestGreen](Bd)
  \tkzDrawSegment[color=RoyalBlue,dashed,thick](B,Bd)
  \tkzLabelSegment[below left](B,Bd){$\clg{\varepsilon}$}

  \tkzDefPointOnCircle[through = center a angle 80 point ac]
  \tkzGetPoint{ae}
  \tkzDefPointOnCircle[through = center A angle 150 point Ac]
  \tkzGetPoint{Ae}
  \tkzDefPointOnCircle[through = center A angle 80 point Ac]
  \tkzGetPoint{Af}
  \tkzDefPointOnCircle[through = center B angle 150 point Bc]
  \tkzGetPoint{Be}

  \draw[-latex,shorten <=5pt,shorten >=5pt] (ae) to [bend left=45]
   node[midway,yshift=2mm,xshift=-2mm] {$g$} (Ae);
  \draw[-latex,shorten <=5pt,shorten >=5pt] (Af) to [bend left=45]
   node[midway,yshift=2mm,xshift=-2mm] {$f$} (Be);

  \tkzDefPoints{0.3/0.2/x,1.8/0.8/gx,4.3/2.5/fgx}
  \tkzDrawPoints[size=4,color=black](x,gx,fgx)
  \tkzLabelPoint[below=1mm](x){$x$}
  \tkzLabelPoint[below=1mm](gx){$g(x)$}
  \tkzLabelPoint[below=1mm](fgx){$f(g(x))$}
 \end{tikzpicture}
 \caption{Důkaz \hyperref[thm:limita-slozene-funkce]{věty o limitě složené
 funkce}.}
 \label{fig:limita-slozene-funkce}
\end{figure}

Pokračujeme vztahem limit posloupností a limit funkcí. Ukazuje se, že limity
funkce mohu v principu nahradit limitou posloupnosti jejích funkčních hodnot.
Toto tvrzení je zvlášť užitečné při důkazu neexistence limity funkce. Stačí
totiž najít dvě posloupnosti, funkční hodnoty jejichž členů se blíží k rozdílným
číslům.

\begin{theorem}{Heineho}{heineho}
 Ať $f$ je reálná funkce a $a,L \in \R^{*}$. Pak jsou následující výroky
 ekvivalentní
 \begin{enumerate}
  \item $\lim_{x \to a} f(x) = L$.
  \item Pro \textbf{každou} posloupnost $x:\N \to \R$ takovou, že
  \begin{itemize}
   \item $\lim_{n \to \infty} x_n = a$;
   \item $x_n \neq a$ pro každé $n \in \N$,
  \end{itemize}
  platí $\lim_{n \to \infty} f(x_n) = L$.
 \end{enumerate}
\end{theorem}
\begin{thmproof}
 Dokážeme nejprve $(1) \Rightarrow (2)$. Ať je dáno $\varepsilon>0$. Potřebujeme
 najít $n_0 \in \N$ takové, že pro $n \geq n_0$ je $|f(x_n) - L|<\varepsilon$.
 Uvědomme si, že poslední výrok je ekvivalentní
 \[
  L -\varepsilon< f(x_n)<L+\varepsilon,
 \]
 neboli $f(x_n) \in B(L,\varepsilon)$. Z toho, že $\lim_{x \to a} f(x) = L$,
 existuje $\delta>0$ takové, že pro $x \in R(a,\delta)$ platí $f(x) \in
 B(L,\varepsilon)$. Ježto $\lim_{n \to \infty} x_n = a$, k tomuto $\delta>0$
 existuje $n_0 \in \N$ takové, že
 \[
 \forall n \geq n_0: |x_n-a|<\delta,
 \]
 neboli $x_n \in B(a,\delta)$. Předpokládáme ovšem, že $x_n \neq a$ pro všechna
 $n \in \N$, a tedy $x_n \in R(a,\delta)$, kdykoli $n \geq n_0$. Potom ale pro
 $n \geq n_0$ rovněž $f(x_n) \in B(L,\varepsilon)$, čili $\lim_{n \to \infty}
 f(x_n) = L$.

 Místo přímého důkazu $(2) \Rightarrow (1)$ (který je zdlouhavý), dokážeme $\neg
 (1) \Rightarrow \neg (2)$. Předpokládejme tedy, že $\lim_{x \to a} f(x)$ buď
 neexistuje, nebo není rovna $L$. Chceme najít posloupnost $x:\N \to \R$, která
 konverguje k $a$, žádný její člen není roven $a$, ale přesto $\lim_{n \to
 \infty} f(x_n)$ opět buď neexistuje, nebo není rovna $L$.

 Uvědomíme si nejprve přesně, jak zní \textbf{negace} výroku $\lim_{x \to a}
 f(x) = L$. Tvrdíme, že
 \[
 \exists \varepsilon>0 \; \forall \delta>0 \; \exists x \in R(a,\delta): f(x)
 \text{ není definováno} \vee f(x) \notin B(L,\varepsilon).
 \]
 Nalezněme tedy ono $\varepsilon>0$ z výroku výše. Pak pro každé $n \in \N$
 existuje $x_n \in R(a,1 / n)$ pro které $f(x_n)$ není definováno, nebo neleží v
 $B(L,\varepsilon)$. Vlastně jsme ve výroku výše položili $\delta \coloneqq 1 /
 n$ postupně pro každé $n \in \N$. Sestrojili jsme pročež posloupnost $x_n$
 takovou, že $\lim_{n \to \infty} x_n = a$ a $x_n \neq a$ pro všechna $n \in
 \N$. Nicméně, jak jsme psali výše, pro každé $x_n$ buď $f(x_n)$ není
 definováno, nebo neleží v $B(L,\varepsilon)$. To ovšem znamená, že $\lim_{n \to
 \infty} f(x_n)$ buď neexistuje, nebo není rovna $L$. Platí tudíž předpoklady v
 (2), ale nikoli závěr v (2), tj. platí $\neg (2)$.

 Tím je důkaz hotov.
\end{thmproof}

\begin{figure}[ht]
 \centering
 \begin{tikzpicture}
  \tkzInit[xmin=-1,xmax=5,ymin=-1,ymax=3]
  \tkzDrawX[label=$\clr{x}$]
  \tkzDrawY[label=$\clb{f(x)}$]
  
  \coordinate (f0) at (0,0);
  \coordinate (f1) at (1,1);
  \coordinate (f2) at (2,2);
  \coordinate (f3) at (3,2);
  \coordinate (f4) at (4,1);
  \coordinate (f5) at (5,0);
  \tkzDefPoint(3,0){a}
  \tkzDefPoint(3,2){fa}
  \tkzDefPoint(0,2){f}
  \foreach \n/\d in
  {1/1.9,2/2.3,3/2.4,4/2.5,5/2.65,6/2.7,7/2.85,8/2.88,9/2.91,10/2.94,11/2.96,12/2.98}
  {
   \tkzDefPoint(\d,0){x\n}
   \tkzDrawPoint[color=black](x\n)
  }
  \tkzLabelPoint[below](x1){\footnotesize $x_1$}
  \tkzLabelPoint[below](x2){\footnotesize $x_2$}
  \tkzLabelPoint[below](x6){\footnotesize $\dots$}

  \tkzDrawSegment[dashed](a,fa)
  \tkzDrawPoint[size=4,color=BrickRed](a)
  \tkzLabelPoint[below=1mm](a){$\clr{a}$}
  \tkzDrawPoint[size=4,color=RoyalBlue](fa)
  \draw[thick,color=RoyalBlue] (f0) to [curve through={(f1) (f2) (f3) (f4)}]
   (f5);
  \tkzDrawSegment[dashed](f,fa)
  \tkzDrawPoint[size=4,color=RoyalBlue](f)
  \tkzLabelPoint[left=1mm](f){$L$}

  \tkzDefPoint(1.9,1.95){fx1}
  \tkzDrawPoint[color=black](fx1)
  \tkzDefPoint(2.3,2.09){fx2}
  \tkzDrawPoint[color=black](fx2)
  \tkzDefPoint(2.4,2.11){fx3}
  \tkzDrawPoint[color=black](fx3)
  \tkzDefPoint(2.5,2.11){fx4}
  \tkzDrawPoint[color=black](fx4)
  \tkzDefPoint(2.65,2.1){fx5}
  \tkzDrawPoint[color=black](fx5)
  \tkzDefPoint(2.7,2.09){fx6}
  \tkzDrawPoint[color=black](fx6)
  \tkzDefPoint(2.85,2.06){fx7}
  \tkzDrawPoint[color=black](fx7)
  \tkzDefPoint(2.88,2.05){fx8}
  \tkzDrawPoint[color=black](fx8)

  \tkzLabelPoint[above](fx1){\footnotesize $f(x_1)$}
  \tkzLabelPoint[below](fx2){\footnotesize $f(x_2)$}
 \end{tikzpicture}
 \caption{Tvrzení (2) v \hyperref[thm:heineho]{Heineho větě}.}
 \label{fig:heineho-veta}
\end{figure}

\begin{corollary}{Heineho věta pro spojitost}{heineho-veta-pro-spojitost}
 Ať $a \in \R^{*}$ a $f$ je reálná funkce. Následující tvrzení jsou
 ekvivalentní.
 \begin{enumerate}
  \item Funkce $f$ je spojitá v bodě $a$.
  \item Pro každou posloupnost $x:\N \to \R$ s limitou $\lim_{n \to \infty} x_n
   = a$ platí $\lim_{n \to \infty} f(x_n) = f(a)$.
 \end{enumerate}
\end{corollary}
\begin{corproof}
 Zkrátka použijeme \hyperref[thm:heineho]{Heineho} větu pro $L = f(a)$.
 Poznamenejme pouze, že podmínka $x_n \neq a$ pro každé $n \in \N$ je zde
 zbytečná, neboť $f$ je z předpokladu spojitá v $a$.
\end{corproof}

\begin{example}{}{}
 V zájmu představení, jak se \hyperref[thm:heineho]{Heineho} věta používá pro
 důkaz neexistence limity funkce, budeme drze předpokládat, že čtenáři byli již
 seznámeni s funkcí $\sin:\R \to [-1,1]$. Její formální debut je v kapitole o
 elementárních funkcích. %TODO odkaz

 Ukážeme, že limita $\lim_{x \to \infty} \sin x$ neexistuje. Volme posloupnosti
 \[
  x_n \coloneqq 2\pi n \quad \text{a} \quad y_n \coloneqq 2\pi n + \pi / 2.
 \]
 Pak platí $\lim_{n \to \infty} x_n = \lim_{n \to \infty} y_n = \infty$, zřejmě
 $x_n \neq \infty$ a $y_n \neq \infty$ pro každé $n \in \N$. Máme proto splněny
 předpoklady tvrzení (2) z \hyperref[thm:heineho]{Heineho věty}. Platí však
 \begin{align*}
  \lim_{n \to \infty} \sin(x_n) &= \lim_{n \to \infty} \sin(2\pi n) = 0,\\
  \lim_{n \to \infty} \sin(y_n) &= \lim_{n \to \infty} \sin(2\pi n + \pi / 2) =
  1,
 \end{align*}
 a tedy jsme našli dvě posloupnosti, funkční hodnoty jejichž členů konvergují k
 různým číslům. Podle \hyperref[thm:heineho]{Heineho věty} $\lim_{x \to \infty}
 \sin x$ neexistuje.
\end{example}

Podobně jako monotónní posloupnosti mají vždy limitu (vizte
\myref{lemma}{lem:limita-monotonni-posloupnosti}), stejně tak mono\-tónní funkce
ji vždy mají. Řekneme, že reálná funkce $f$ je \emph{monotónní na intervalu}
$I \subseteq \R$, když
\begin{itemize}
 \item[$(<)$] je $f$ \emph{rostoucí}, tj. $\forall x,y \in I: x<y \Rightarrow
  f(x)<f(y)$,
 \item[$(>)$] je $f$ \emph{klesající}, tj. $ \forall x,y \in I: x < y
  \Rightarrow f(x) > f(y)$,
 \item[$(\leq)$] je $f$ \emph{neklesající}, tj. $ \forall x,y \in I:x<y
  \Rightarrow f(x) \leq f(y)$ nebo
 \item[$(\geq)$] je $f$ \emph{nerostoucí}, tj. $ \forall x,y \in I:x<y
  \Rightarrow f(x) \geq f(y)$.
\end{itemize}
\begin{theorem}{Limita monotónní funkce}{limita-monotonni-funkce}
 Ať $(a,b) \subseteq \R^{*}$ a $f$ je monotónní na $(a,b)$. Potom,
 \begin{enumerate}
  \item je-li $f$ rostoucí nebo neklesající, tak
  \[
   \lim_{x \to a^{+}} f(x) = \inf f((a,b)) \quad \text{a} \quad \lim_{x \to
   b^{-}} f(x) = \sup f((a,b));
  \]
 \item je-li $f$ klesající nebo nerostoucí, tak
  \[
   \lim_{x \to a^{+}} f(x) = \sup f((a,b)) \quad \text{a} \quad \lim_{x \to
   b^{-}} f(x) = \inf f((a,b)).
  \]
 \end{enumerate}
\end{theorem}
\begin{thmproof}
 Dokážeme pouze část (1), důkaz (2) je totožný.

 Budeme nejprve předpokládat, že $f$ je zdola omezená a označíme $m \coloneqq
 \inf f((a,b)) \in \R$. Ať je dáno $\varepsilon>0$. Z definice infima nalezneme
 $y \in f((a,b))$ takové, že $y < m + \varepsilon$. Jelikož $y \in f((a,b))$,
 existuje $x \in (a,b)$, že $f(x) = y$. Z toho, že $f$ je neklesající či
 rostoucí plyne, že pro každé $z \in (a,x)$ je $f(z) \leq f(x) = y$.

 Nalezneme $\delta>0$ takové, že $R_+(a,\delta) = (a,a+\delta) \subseteq (a,x)$.
 Potom ale pro $z \in R_+(a,\delta)$ platí $f(z) \leq y < m+\varepsilon$. Ježto
 nerovnost $f(z) > m-\varepsilon$ je zřejmá ($m$ je dolní závora $f$), máme
 celkem pro $z \in R_+(a,\delta)$
 \[
  m -\varepsilon < f(z) < m + \varepsilon,
 \]
 čili $f(z) \in B(m,\varepsilon)$, jak bylo dokázati.

 Ať nyní $f$ není zdola omezená na $(a,b)$. Pak $\inf f(a,b) = -\infty$ a pro
 každé $\varepsilon>0$ nalezneme $z \in (a,b)$, pro nějž $f(z) < -1 /
 \varepsilon$. To ovšem z definice znamená, že $\lim_{x \to a^{+}} f(x) =
 -\infty = \inf f((a,b))$.

 Důkaz faktu, že $\lim_{x \to b^{-}} f(x) = \sup f((a,b))$ lze vést obdobně.
\end{thmproof}

\begin{exercise}{}{}
 Dokažte bod (2) ve \myref{větě}{thm:limita-monotonni-funkce}.
\end{exercise}

\subsection{Extrémy funkce}
\label{ssec:extremy-funkce}

V mnoha matematických i externích disciplínách jeden často hledá při studiu
reálných funkcí body, v nichž je hodnota funkce největší či nejmenší. Obecně
jsou maximalizační a minimalizační problémy jedny z nejčastěji řešených. Tyto
problémy vedou přímo na výpočet tzv. \emph{derivací} reálných funkcí, jsoucích
dychtivým čtenářům představeny v následující kapitole. Zde pouze definujeme
lokální a globální extrémy funkcí a ukážeme, že spojité funkce na uzavřených
intervalech nutně na týchž nabývají svých nejmenších i největších hodnot.
%TODO odkaz

\begin{definition}{Lokální a globální extrém}{lokalni-a-globalni-extrem}
 Ať $f:M \to \R$ je reálná funkce a $X \subseteq M$. Řekneme, že funkce $f$ v
 bodě $x \in X$ nabývá
 \begin{itemize}
  \item \emph{globálního maxima} na $X$, když pro každé $y \in X$ platí
   $f(y) \leq f(x)$;
  \item \emph{globálního minima} na $X$, když pro každé $y \in X$ platí $f(y)
   \geq f(x)$;
  \item \emph{lokálního maxima} na $X$, když existuje okolí bodu $x$, na němž
   platí $f \leq f(x)$;
  \item \emph{lokálního minima} na $X$, když existuje okolí bodu $x$, na němž
   platí $f \geq f(x)$.
 \end{itemize}
 Souhrnně přezdíváme globálnímu minimu a maximu \emph{globální extrém} a
 lokálnímu maximu a minimu \emph{lokální extrém}.
\end{definition}

\begin{figure}[ht]
 \centering
 \begin{tikzpicture}
  \tkzInit[xmin=-3,xmax=3,ymin=-3,ymax=3]
  \tkzDrawX[-latex]
  \tkzDrawY[-latex]

  \draw[thick,domain=-2.3:3,color=Fuchsia] plot [smooth] (\x, {0.5 * (\x*\x*\x
   - \x*\x - 4*\x + 2)}) node[right] {$\clm{f}$};
  \tkzDefPoints{-2.2/0/a,2.8/0/b}
  \tkzDrawPoints[size=4,color=black](a,b)
  \tkzLabelPoint[above](a){$a$}
  \tkzLabelPoint[below](b){$b$}

  \tkzDefPoints{-2.2/-2.344/fa,2.8/2.456/fb}
  \tkzDrawSegments[dashed,thick](a,fa b,fb)
  \tkzDrawPoints[size=6,color=BrickRed](fa,fb)

  \tkzDefPoints{-0.867/2.032/fM,1.535/-1.44/fm}
  \tkzDefPoints{-0.867/0/M,1.535/0/m}
  \tkzDrawPoints[size=4,color=black](M,m)
  \tkzLabelPoint[below](M){$M$}
  \tkzLabelPoint[above](m){$m$}
  \tkzDrawSegments[dashed,thick](M,fM m,fm)
  \tkzDrawPoints[size=6,color=RoyalBlue](fM,fm)
 \end{tikzpicture}
 \caption{Lokální a globální extrémy funkce $\clm{f}$ na [a,b]. \clb{Lokálních
  extrémů} nabývá $\clm{f}$ v bodech $m$ a $M$ a \clr{globálních extrémů} v
  bodech $a$ a $b$.}
 \label{fig:extremy}
\end{figure}

Jak jsme již zmínili v textu před definicí, spojité funkce nabývají na
uzavřených intervalech globálních extrémů vždy.

\begin{theorem}{Extrémy spojité funkce}{extremy-spojite-funkce}
 Ať $f:[a,b] \to \R$ je spojitá funkce. Pak $f$ nabývá globálního minima a
 maxima na $[a,b]$.
\end{theorem}
\begin{thmproof}
 Dokážeme, že $f$ nabývá na $[a,b]$ globálního maxima. Pro globální minimum lze
 důkaz vést obdobně.

 Využijeme \hyperref[thm:heineho]{Heineho věty}. Nalezneme posloupnost, která se
 uvnitř intervalu $f([a,b])$ blíží k supremu funkce $f$ na $[a,b]$ a ukážeme, že
 vzory členů této posloupnosti posloupnosti z intervalu $[a,b]$ se blíží k bodu,
 kde $f$ nabývá maxima.

 Položme tedy $S \coloneqq \sup f([a,b])$. Sestrojíme posloupnost $y: \N \to
 f([a,b])$, která konverguje k $S$. Je-li $S=\infty$, stačí položit třeba $y_n =
 n$ pro všechna $n \in \N$. Předpokládejme, že $S \in \R$. Z~definice suprema
 existuje pro každé $n \in \N$ prvek $z \in f([a,b])$ takový, že $S-1 / n < z
 \leq S$. Položíme $y_n \coloneqq z$. Tím máme posloupnost $y_n$ s $\lim_{n \to
 \infty} y_n = S$.

 Z definice $f([a,b])$ nalezneme pro každé $y_n$ číslo $x_n \in [a,b]$, pro něž
 $f(x_n) = y_n$. Posloupnost $x_n$ je omezená (leží uvnitř $[a,b]$), a tedy z
 \hyperref[thm:bolzano-weierstrass]{Bolzanovy-Weierstraßovy} věty existuje její
 
\end{thmproof}

