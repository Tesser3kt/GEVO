\section{Základní algebraické struktury}
\label{sec:zakladni-algebraicke-struktury}

První algebraické struktury počali lidé objevovat koncem 19. století, kdy jsme
si všimli, že se mnoho skupin jevů -- geometrických, fyzikálních, ... --
\uv{chová} podobně jako čísla. Dnes bychom řekli, že \uv{vykazují silnou
symetrii}. Například, podobně jako můžeme přirozená čísla násobit, lze zobrazení
\emph{skládat} či křivky v rovině na sebe \emph{napojovat}. Přirozená čísla
\uv{obracíme}, dávajíce vzniknout číslům celým. Po křivce umíme kráčet opačným
směrem.

Taková pozorování vedla na pojem \emph{grupy} -- ve své podstatě množině všech
symetrií nějakého objektu. \emph{Symetrie} v tomto smyslu značí
transformace/proměny tohoto objektu, které jej nemění. Dnes má samozřejmě grupa
svou elegantní formální definici, z níž nelze vůbec poznat, o jakou strukturu
vlastně jde. Uvedeme si ji.

\begin{definition}{Grupa}{grupa}
 Ať $G$ je libovolná neprázdná množina. Platí-li, že
 \begin{itemize}
  \item existuje binární operace $ \cdot :G \times G \to G$, která je
  \textbf{asociativní} (tj. $(g \cdot h) \cdot k = g \cdot (h \cdot k))$,
 \item existuje prvek $1 \in G$ splňující pro každé $g \in G$ rovnost $g \cdot 1
  = 1 \cdot g = g$, zvaný \emph{neutrální}, a
 \item pro každý prvek $g \in G$ existuje prvek $g^{-1} \in G$ splňující
  $g \cdot g^{-1} = g^{-1} \cdot g = 1$, zvaný \emph{inverz},
 \end{itemize}
 pak nazveme čtveřici $\mathbf{G} = (G, \cdot ,^{-1},1)$ \emph{grupou}.
\end{definition}
Tato definice si zaslouží několika poznámek, varování a příkladů. Součástí
definice grupy \textbf{není} komutativita její binární operace. Obecně, v grupě
$\mathbf{G}$ není prvek $g \cdot h$ tentýž jako $h \cdot g$. Mezi algebraiky
platí nepsaná dohoda, že grupy, které jsou \emph{komutativní} (též
\emph{abelovské}) -- tj. ty, kde $g \cdot h = h \cdot g$ opravdu pro všechny
dvojice prvků $g,h \in G$ -- se zapisují jako (tzv. \emph{aditivní}) $\mathbf{G}
= (G,+,-,0)$. Naopak, grupy, které komutativní nutně nejsou, se obvykle píší
stylem z~\myref{definice}{def:grupa}.

Zadruhé, není vůbec zřejmé, proč by taková struktura měla jakýmkoli způsobem
zrcadlit koncept \emph{symetrie}. Ono \uv{zrcadlo} zde sestrojíme.

\begin{example}{Dihedrální grupa}{dihedralni-grupa}
 Ať $P$ je pravidelný šestiúhelník v $\R^2$. Uvažme zobrazení $r:\R^2 \to \R^2$,
 které rotuje body v $\R^2$ o $60^{ \circ }$ (v kladném směru -- proti směru
 hodinových ručiček) podle středu jeho uhlopříček, a zobrazení $s:\R^2 \to
 \R^2$, které reflektuje body v $\R^2$ podle kterékoli (ale fixní) jeho
 uhlopříčky.

 Není těžké nahlédnout, že $r(P) = P$ a $s(P) = P$, čili tato zobrazení
 zachovávají $P$. Tvrdíme, že každé jejich složení je rovněž zobrazení, které
 zachovává $P$. Jinak řečeno, množina všech možných složení zobrazení $r$ se
 zobrazením $s$ tvoří \emph{grupu}, kde binární operací je \emph{složení}
 zobrazení, inverzem je \emph{inverzní zobrazení} (pozřeme, že $r$ i $s$ jsou
 \textbf{bijekce}) a neutrálním prvkem je $\mathds{1}_{\R^2}$ -- \emph{identické
 zobrazení} na $\R^2$.

 Po chvíli přemýšlení zjistíme, že rotace o $60,120,180,240,300$ a $360$ stupňů
 zachovávají $P$. Všechny můžeme dostat jako složení $r$ se sebou samým
 vícekrát. Například $r \circ r \circ r = r^3$ je rotace o $180^{ \circ }$.
 Přirozeně, rotace o $360^{ \circ }$ je identické zobrazení, což lze vyjádřit
 rovností $r^{6} = \mathds{1}_{\R^2}$.

 S reflexemi je to mírně složitější. Jelikož $s$ je reflexe, složení $s \circ s$
 je identické zobrazení. Reflexi podle ostatních dvou uhlopříček dostaneme jeho
 složením s $r$. Například reflexi podle uhlopříčky, která svírá s $s$ úhel
 $60^{ \circ }$ (proti směru hodinových ručiček) je rovna složení $r^2 \circ s$.
 Konečně, šestiúhelník $P$ rovněž zachovávají reflexe podle os stran. Reflexi
 podle osy stran, která svírá s~$s$ úhel $90^{ \circ }$ dostanu (třeba) složením
 $s \circ r^3$.

 Ponecháváme čtenáře, aby si rozmysleli, že různých zobrazení, která mohu dostat
 složením $r$ a $s$ je celkem $12$, všechna jsou bijektivní a zachovávají $P$.
 Označíme-li jejich množinu $D_{12}$ (jako \textbf{d}ihedrální grupa o $12$
 prvcích), pak je $(D_{12}, \circ ,^{-1},\mathds{1}_{\R^2})$
 \textbf{nekomutativní} grupa.

 \begin{figure}[H]
  \centering
  \begin{subfigure}[b]{\textwidth}
   \centering
   \begin{tikzpicture}
    \foreach \an/\name in {0/A,60/B,120/C,180/D,240/E,300/F} {
     \tkzDefPoint(\an:1){\name}
    }
    \tkzDrawPolygon(A,B,C,D,E,F)

    \tkzDrawLine[add=.5 and .5,dashed,RoyalBlue,thick](B,E)
    \tkzDrawPoints[size=6](A,B,C,D,E,F)
    \tkzLabelPoint[right](A){$1$}
    \tkzLabelPoint[above right](B){$2$}
    \tkzLabelPoint[above left](C){$3$}
    \tkzLabelPoint[left](D){$4$}
    \tkzLabelPoint[below left](E){$5$}
    \tkzLabelPoint[below right](F){$6$}

    \draw[->,bend left=30,RoyalBlue] ($(A.center) + (0,0.5)$) to
     node[midway,yshift=2mm]
     {$\clb{s}$} ($(A.center) + (2,0.5)$);

    \tkzDefPoint(0:5){A1}
    \tkzDefPointsBy[translation=from A to A1](B,C,D,E,F){B1,C1,D1,E1,F1}
    \tkzDrawPolygon(A1,B1,C1,D1,E1,F1)

    \tkzLabelPoint[right](A1){$3$}
    \tkzLabelPoint[above right](B1){$2$}
    \tkzLabelPoint[above left](C1){$1$}
    \tkzLabelPoint[left](D1){$6$}
    \tkzLabelPoint[below left](E1){$5$}
    \tkzLabelPoint[below right](F1){$4$}
    \tkzDrawPoints[size=6](A1,B1,C1,D1,E1,F1)
    
    \tkzDefPoint(0:4){O}
    \tkzDrawPoint[size=4,fill=none,draw=BrickRed](O)
    \tkzDrawArc[R,dashed,BrickRed,-latex](O,0.5)(0,120)

    \draw[->,bend left=30,BrickRed] ($(A1.center) + (0,0.5)$) to
     node[midway,yshift=3mm]
     {$\clr{r}^2$} ($(A1.center) + (2,0.5)$);

    \tkzDefPoint(0:9){A2}
    \tkzDefPointsBy[translation=from A1 to A2](B1,C1,D1,E1,F1){B2,C2,D2,E2,F2}
    \tkzDrawPolygon(A2,B2,C2,D2,E2,F2)

    \tkzLabelPoint[right](A2){$5$}
    \tkzLabelPoint[above right](B2){$4$}
    \tkzLabelPoint[above left](C2){$3$}
    \tkzLabelPoint[left](D2){$2$}
    \tkzLabelPoint[below left](E2){$1$}
    \tkzLabelPoint[below right](F2){$6$}

    \tkzDrawLine[thick,dotted,ForestGreen,add=.5 and .5](F2,C2)
    \tkzLabelLine[ForestGreen,pos=1.5,right](F2,C2){$o$}
    \tkzDrawPoints[size=6](A2,B2,C2,D2,E2,F2)
   \end{tikzpicture}
   \caption{Složení $\clr{r}^2 \circ \clb{s} = \text{reflexe podle } \clg{o}$.}
   \label{subfig:slozeni-symetrii-1}
  \end{subfigure}
  \begin{subfigure}[b]{\textwidth}
   \vspace{1em}
   \centering
   \begin{tikzpicture}
    \foreach \an/\name in {0/A,60/B,120/C,180/D,240/E,300/F} {
     \tkzDefPoint(\an:1){\name}
    }
    \tkzDrawPolygon(A,B,C,D,E,F)

    \tkzDrawPoints[size=6](A,B,C,D,E,F)
    \tkzLabelPoint[right](A){$1$}
    \tkzLabelPoint[above right](B){$2$}
    \tkzLabelPoint[above left](C){$3$}
    \tkzLabelPoint[left](D){$4$}
    \tkzLabelPoint[below left](E){$5$}
    \tkzLabelPoint[below right](F){$6$}

    \tkzDefPoint(0,0){O}
    \tkzDrawPoint[size=4,fill=none,draw=BrickRed](O)
    \tkzDrawArc[R,dashed,BrickRed,-latex](O,0.5)(0,180)
    \draw[->,bend left=30,BrickRed] ($(A.center) + (0,0.5)$) to
     node[midway,yshift=3mm]
     {$\clr{r}^3$} ($(A.center) + (2,0.5)$);

    \tkzDefPoint(0:5){A1}
    \tkzDefPointsBy[translation=from A to A1](B,C,D,E,F){B1,C1,D1,E1,F1}
    \tkzDrawPolygon(A1,B1,C1,D1,E1,F1)

    \tkzLabelPoint[right](A1){$4$}
    \tkzLabelPoint[above right](B1){$5$}
    \tkzLabelPoint[above left](C1){$6$}
    \tkzLabelPoint[left](D1){$1$}
    \tkzLabelPoint[below left](E1){$2$}
    \tkzLabelPoint[below right](F1){$3$}

    \tkzDrawLine[add=.5 and .5,dashed,RoyalBlue,thick](E1,B1)
    \tkzDrawPoints[size=6](A1,B1,C1,D1,E1,F1)

    \draw[->,bend left=30,RoyalBlue] ($(A1.center) + (0,0.5)$) to
     node[midway,yshift=3mm]
     {$\clb{s}$} ($(A1.center) + (2,0.5)$);

    \tkzDefPoint(0:9){A2}
    \tkzDefPointsBy[translation=from A1 to A2](B1,C1,D1,E1,F1){B2,C2,D2,E2,F2}
    \tkzDrawPolygon(A2,B2,C2,D2,E2,F2)

    \tkzLabelPoint[right](A2){$6$}
    \tkzLabelPoint[above right](B2){$5$}
    \tkzLabelPoint[above left](C2){$4$}
    \tkzLabelPoint[left](D2){$3$}
    \tkzLabelPoint[below left](E2){$2$}
    \tkzLabelPoint[below right](F2){$1$}

    \tkzDefMidPoint(A2,F2) \tkzGetPoint{M1}
    \tkzDefMidPoint(D2,C2) \tkzGetPoint{M2}
    \tkzDrawLine[thick,dotted,ForestGreen,add=.5 and .5](M1,M2)
    \tkzLabelLine[ForestGreen,pos=-0.4,above right](M1,M2){$o$}
    \tkzDrawPoints[size=6](A2,B2,C2,D2,E2,F2)
   \end{tikzpicture}
   \caption{Složení $\clb{s} \circ \clr{r}^3 = \text{reflexe podle }\clg{o}$.}
   \label{subfig:slozeni-symetrii-2}
  \end{subfigure}
  \caption{Příklady složení \clb{reflexí} a \clr{rotací}.}
  \label{fig:slozeni-symetrii}
 \end{figure}
\end{example}

\begin{example}{Permutační grupa}{permutacni-grupa}
 Ať $X$ je libovolná konečná množina velikosti $n \in \N$. Pak množina všech
 permutací na $X$ (tj. bijekcí $X \leftrightarrow X$) tvoří spolu s operací
 skládání a invertování funkcí \textbf{nekomutativní} grupu. Skutečně, skládání
 funkcí je zřejmě \emph{asociativní}, ke každé bijekci existuje \emph{inverz} a
 \emph{neutrálním} prvkem je $\mathds{1}_{X}$. Z diskrétní matematiky víme, že
 permutací na $n$-prvkové množině je $n!$; označíme-li jejich množinu jako $S_X$
 (ze zaběhlého a zcestného názvu \textbf{s}ymetrická grupa), pak je $(S_X,
 \circ,^{-1},\mathds{1}_{X})$ nekomutativní grupa o $n!$ prvcích. Můžeme se na
 ni dívat jako na množinu všech transformací, které zachovávají množinu $X$.

 Zajímavou otázkou je, kolik potřebujeme nejméně permutací, abychom jejich
 skládáním vyrobili všechny ostatní. V případě dihedrální grupy pravidelného
 šestiúhelníku (\myref{příklad}{exam:dihedralni-grupa}) to byla zobrazení dvě.
 Ukazuje se, a není příliš obtížné to dokázat, že nám stačí všechny transpozice
 $(x \; y)$, kde $x \in X$ je nějaký fixní prvek a $y$ probíhá všechny ostatní
 prvky $X$. Pokud by $X = \{1,\ldots,n\}$, pak by to byly třeba právě
 transpozice $(1 \; 2), (1 \; 3),\ldots,(1 \; n)$. Tento fakt souvisí přímo s
 pozorováním z diskrétní matematiky, že každou permutaci lze rozložit na
 transpozice. 
\end{example}

\begin{example}{Odmocniny jednotky}{odmocniny-jednotky}
 Každé komplexní číslo má přesně $n$ $n$-tých odmocnin. Zapíšeme-li si komplexní
 číslo $z \in \C$ v~tzv. \uv{goniometrickém} tvaru, pak je můžeme snadno najít.
 Totiž, je-li $z = r \cdot (\cos\theta + i \sin\theta)$, kde $r \in \R^{+}$ je
 jeho vzdálenost od počátku, $\theta$ úhel, který svírá s reálnou (typicky
 vodorovnou) osou, a $i$ imaginární jednotka (z definice $i^2=-1$), pak je
 \[
  \left\{\sqrt[n]{r} \cdot \left(\cos\left(\frac{\theta + 2\pi k}{n}\right) + i
  \cdot \sin\left(\frac{\theta + 2\pi k}{n}\right)\right) \mid k \in
  \{0,\ldots,n-1\} \right\}
 \]
 množina všech jeho $n$-tých odmocnin.

 Tato množina obecně \textbf{není} grupa, neboť tím, že vynásobím dvě odmocniny
 komplexního čísla, nedostanu jeho jinou odmocninu -- s jednou výjimkou, a tou
 je číslo $1$. Totiž, $1 = \cos(2\pi) + {i \cdot \sin(2\pi)}$, a tedy všechny
 jeho třeba čtvrté odmocniny jsou
 \begin{align*}
  &\left\{ \cos\left(\frac{\pi}{2}\right)+i \sin \left( \frac{\pi}{2} \right),
  \cos \left( \pi \right) + i \sin \left( \pi \right),
  \cos\left(\frac{3\pi}{2}\right)+i\sin \left( \frac{3\pi}{2} \right), \cos
  \left( 2\pi \right) + i\sin \left( 2\pi \right) \right\}\\
  &= \{i, -1, -i, 1\}.
 \end{align*}
 Důležité pozorování k pochopení tohoto příkladu je, že když spolu násobím dvě
 komplexní čísla, jejich vzdálenosti od počátku ($r$) se násobí a jejich úhly
 svírané s reálnou osou ($\theta$), se sčítají. Z toho plyne, že vzdálenost
 každé odmocniny z $1$ od počátku je vždy $1$ a že vynásobením dvou odmocnin z
 $1$ dostanu další odmocninu z $1$. Vskutku, jsou-li $\cos (2k\pi / n) +
 i\sin(2k\pi / n)$ a $\cos(2l\pi / n) + i\sin(2l\pi / n)$ dvě odmocniny z jedné,
 pak je jejich součin roven
 \[
  \left( \cos \left( \frac{2k\pi}{n} \right) + i\sin \left( \frac{2k\pi}{n}
   \right) \right) \left( \cos \left( \frac{2l\pi}{n} \right) + i\sin \left(
   \frac{2l\pi}{n} \right)\right) = \cos \left( \frac{2(k+l)\pi}{n} \right) +
   i\sin \left( \frac{2(k+l)\pi}{n} \right),
 \]
 což je opět odmocnina z $1$ (za předpokladu, že ztotožňujeme \uv{přetočené
 úhly} v tom smyslu, že třeba $7\pi / 3 = \pi / 3$). Označíme-li $\Omega(n)$
 množinu všech $n$-tých odmocnin z $1$, pak je čtveřice $(\Omega(n), \cdot
 ,^{-1},1)$ \textbf{komutativní} grupa, kde $ \cdot $ značí běžné násobení
 komplexních čísel.
 \begin{figure}[H]
  \centering
  \begin{subfigure}[t]{.31\textwidth}
   \centering
   \begin{tikzpicture}[scale=1.4]
    \tkzInit[xmin=-1.2,xmax=1,ymin=-1.2,ymax=1]
    \tkzDrawX[label=] \tkzDrawY[label=]
    \tkzDefPoint(0,0){o}
    \tkzDefPoint(0:1){a}
    \tkzDrawCircle(o,a)

    \tkzDefPoint(72:1){w1}
    \tkzDrawPoint[size=4,BrickRed](w1)
    \tkzLabelPoint[above right,BrickRed](w1){$\omega_1$}
    \tkzDrawLine[BrickRed,add=0 and 0](o,w1)
    \tkzLabelAngle[pos=0.4,BrickRed](a,o,w1){$\frac{2\pi}{5}$}
    \tkzMarkAngle[size=0.7,BrickRed](a,o,w1)

    \tkzDefPoint(0,1){i}
    \tkzDrawPoints[size=2](a,i)
    \tkzLabelPoint[below right](a){$1$}
    \tkzLabelPoint[above left](i){$i$}
   \end{tikzpicture}
   \label{subfig:odmocniny-z-jedne-1}
  \end{subfigure}
  \hfill
  \begin{subfigure}[t]{.31\textwidth}
   \centering
   \begin{tikzpicture}[scale=1.4]
    \tkzInit[xmin=-1.2,xmax=1,ymin=-1.2,ymax=1]
    \tkzDrawX[label=] \tkzDrawY[label=]
    \tkzDefPoint(0,0){o}
    \tkzDefPoint(0:1){a}
    \tkzDrawCircle(o,a)

    \tkzDefPoint(144:1){w2}
    \tkzDrawPoint[size=4,RoyalBlue](w2)
    \tkzLabelPoint[above left,RoyalBlue](w2){$\omega_2$}
    \tkzDrawLine[RoyalBlue,add=0 and 0](o,w2)
    \tkzLabelAngle[pos=0.35,RoyalBlue](a,o,w2){$\frac{4\pi}{5}$}
    \tkzMarkAngle[size=0.7,RoyalBlue](a,o,w2)

    \tkzDefPoint(0,1){i}
    \tkzDrawPoints[size=2](a,i)
    \tkzLabelPoint[below right](a){$1$}
    \tkzLabelPoint[above left](i){$i$}
   \end{tikzpicture}
  \end{subfigure}
  \hfill
  \begin{subfigure}[t]{.31\textwidth}
   \centering
   \begin{tikzpicture}[scale=1.4]
    \tkzInit[xmin=-1.2,xmax=1,ymin=-1.2,ymax=1]
    \tkzDrawX[label=] \tkzDrawY[label=]
    \tkzDefPoint(0,0){o}
    \tkzDefPoint(0:1){a}
    \tkzDrawCircle(o,a)

    \tkzDefPoint(216:1){w3}
    \tkzDrawPoint[size=4,ForestGreen](w3)
    \tkzLabelPoint[below left,ForestGreen](w3){$\omega_1\omega_2$}
    \tkzDrawLine[ForestGreen,add=0 and 0](o,w3)
    \tkzLabelAngle[pos=0.35,ForestGreen](a,o,w3){$\frac{6\pi}{5}$}
    \tkzMarkAngle[size=0.7,ForestGreen](a,o,w3)

    \tkzDefPoint(0,1){i}
    \tkzDrawPoints[size=2](a,i)
    \tkzLabelPoint[below right](a){$1$}
    \tkzLabelPoint[above left](i){$i$}
   \end{tikzpicture}
  \end{subfigure}
  \caption{Komplexní čísla $\clr{\omega_1},\clb{\omega_2} \in \Omega(5)$ a
   jejich součin $\clg{\omega_1\omega_2}$.}
  \label{fig:odmocniny-z-jedne}
 \end{figure}
\end{example}

Doufáme, že jsme uspěli ve snaze vnímavé čtenáře přesvědčit, že grupy jsou
přirozené struktury v~různém smyslu reprezentující symetrie objektů spolu s
jejich vzájemnými souvislostmi.

Avšak, grupy nezachycují \emph{všechny} transformace, pouze ty, které lze
zvrátit -- tento požadavek je zachycen v podmínce existence inverzu ke každému
prvku grupy. Není přehnané domnívat se, že tímto přístupem přicházíme o řád
informací o studovaných jevech. Vskutku, matematici 19. století souhlasí a
vymýšlejí strukturu \emph{monoidu}, v podstatě jen grupy, u které nepožadujeme,
aby každý prvek bylo lze invertovat. Monoidy jsou tudíž algebraické struktury
objímající \textbf{všechny} transformace -- jak symetrie, tak deformace.

\begin{definition}{Monoid}{monoid}
 Ať $M$ je libovolná neprázdná množina. Platí-li, že
 \begin{itemize}
  \item existuje binární operace $ \cdot :M \times M \to M$, která je
   \textbf{asociativní} a
  \item existuje prvek $1 \in M$ takový, že $1 \cdot m = m \cdot 1 = m$ pro
   každé $m \in M$,
 \end{itemize}
 pak nazýváme trojici $(M, \cdot ,1)$ \emph{monoidem}.
\end{definition}

Přirozeně, pokud má každý prvek monoidu inverz, je tento monoid grupou. Některé
příklady grup se dají zobecnit tak, aby se staly příklady monoidů, které však
nejsou grupami. Vezměme \myref{příklad}{exam:permutacni-grupa}. Uvážíme-li místo
pouhých permutací na $X$ (tj. bijekcí $X \leftrightarrow X$) \textbf{všechna}
zobrazení $X \to X$, pak dostaneme monoid. Vskutku, jak jsme již zmiňovali,
skládání zobrazení je asociativní a máme k~dispozici identické zobrazení
$\mathds{1}_X$, čili je trojice
\[
 (\{f \mid f \text{ je zobrazení } X \to X\}, \circ ,\mathds{1}_X)
\]
monoidem. Tento příklad též ukazuje, že monoidy jsou v jistém smyslu \uv{větší}
než grupy. Je-li $X$ konečná množina velikosti $n$, pak je tento smysl dokonce
absolutní. Všech permutací na $X$ je totiž $n!$, zatímco všechna zobrazení $X
\to X$ čítají $n^{n}$.

Příklady \ref{exam:dihedralni-grupa} a \ref{exam:odmocniny-jednotky} žádných
přirozených zobecnění nenabízejí. Přidáme-li k dihedrální grupě rotace a
reflexe, které nemusejí daný mnohoúhelník zachovat, pak už můžeme rovnou uvážit
úplně všechny rovinné rotace a reflexe. Je sice pravdou, že množina všech rotací
a reflexí dvoudimenzionálního prostoru tvoří monoid, ale již nikterak nesouvisí
s mnohoúhelníky. Podobně, když se nebudeme soustředit na komplexní odmocniny z
$1$, ale na komplexní odmocniny libovolného komplexního čísla, nedostaneme tak
ani monoid -- jak jsme uvedli, součin dvou $n$-tých odmocnin komplexního čísla
obecně není $n$-tá odmocnina téhož čísla.

Předpokládáme, že čtenáři stále nevidí spojitost mezi grupy a monoidy a
číselnými obory. Jedním (pravda zásadním) rozdílem je existence operací součtu a
součinu v každém číselném oboru. Grupy a monoidy z definice dovolují jen jednu
operaci. Pravdať, číselné obory jsou jakýmsi přirozeným \uv{sloučením} monoidu a
grupy, které sluje \emph{okruh}.

Okruhy jsou již vcelku komplikované struktury, jež v sobě mísí symetrie s
destruktivními transformacemi a vlastně je \uv{donucují} ke spolupráci. Z
jiného, více formálního, pohledu jsou prvky okruhů součty násobků všech
transformací objektu.

\begin{definition}{Okruh}{okruh}
 Ať $R$ (od angl. výrazu pro okruh -- \textbf{r}ing) je neprázdná množina, $+,
 \cdot $ jsou operace na $R$ a $0,1 \in R$. Je-li
 \begin{itemize}
  \item $(R,+,-,0)$ \textbf{komutativní} grupa,
  \item $(R, \cdot ,1)$ (ne nutně komutativní) monoid
 \end{itemize}
 a platí-li 
 \begin{equation}
  \label{eq:distributivita}
  \begin{split}
   (r + s) \cdot t &= r \cdot t + s \cdot t,\\
   t \cdot (r + s) &= t \cdot r + t \cdot s
  \end{split}
 \end{equation}
 pro všechna $r,s,t \in R$, nazveme $R$ okruhem. 
\end{definition}

\begin{remark}{}{poznamky-okruhy}
 \begin{itemize}
  \item Symbol $-$ v popisu grupy $(R,+,-,0)$ značí \emph{inverz},
   \textbf{nikoli binární operaci}! Odčítání nemůže být nikdy grupovou (ani
   monoidovou) operací, bo \textbf{není asociativní}. Zápis $r - s$ je pouze
   neformálním zkrácením zápisu $r + (-s)$, podobně jako se třeba $r \cdot
   s^{-1}$ zapisuje jako $r / s$.
  \item \hyperref[def:okruh]{Definice okruhu} vyžaduje, aby byla operace $+$
   komutativní, ale $ \cdot $ nikoli. Mluvíme-li tedy o \textbf{komutativním}
   okruhu, znamená to, že i $ \cdot $ je komutativní, a nemůže dojít ke zmatení,
   kterouže operaci máme na mysli.
  \item V literatuře se občas při definici okruhu nevyžaduje existence jednotky,
   tedy neutrálního prvku k násobení. Dvojice $(R, \cdot )$ je pak pouze tzv.
   \emph{magma}, množina s~binární operací bez žádných dalších předpokladů.
   Našemu pojmu okruhu se v takovém případě říká \emph{okruh s jednotkou}. Možná
   překvapivě je teorie okruhů s jednotkou výrazně odlišná od teorie okruhů bez
   jednotky.
  \item Rovnice \eqref{eq:distributivita} jsou onou \uv{vynucenou} domluvou mezi
   symetrickou operací $+$ a libovolnou transformací $ \cdot $, říkáme jí
   \emph{distributivita}. Je třeba specifikovat distributivitu jak zleva, tak
   zprava, protože $ \cdot $ nemusí být komutativní.
 \end{itemize}
\end{remark}

Jednoduchých příkladů okruhů není mnoho a všechny vyžadují snad nepřirozené
konstrukce. Ty přirozené vyplynou samovolně, až se jmeme tvořiti číselných
oborů, v následující kapitole. S cílem představit jeden velmi naučný
příklad/varování však tyto konstrukce dočasně přeskočíme a budeme předpokládat,
že množina přirozených čísel $\N$ je čtenářům již plně známa.

\begin{warning}{}{delitele-nuly}
 V okruzích (a obecně v monoidech) může nastat situace, že $r \cdot s = 0$,
 přestože $r$ ani $s$ není nulový prvek. Uvažme například množinu přirozených
 čísel $\{0,1,2,3,4,5\}$ se sčítáním a násobením \uv{modulo $6$}. Konkrétně,
 definujme operace $ \oplus $ a $\odot$ předpisy
 \begin{align*}
  m \oplus n &\coloneqq (m + n) \bmod 6,\\
  m \odot n & \coloneqq (m \cdot n) \bmod 6,
 \end{align*}
 a položme $\ominus x \coloneqq (6-x) \bmod 6$, kde $x \bmod y$ značí zbytek $x$
 po dělení $y$. Je poměrně snadné si uvědomit, že
 \[
  (\{0,1,2,3,4,5\}, \oplus ,\ominus, 0, \odot, 1)
 \]
 je (komutativní) okruh. V tomto okruhu platí
 \[
  2 \odot 3 = (2 \cdot 3) \bmod 6 = 0,
 \]
 ačkoli $2$ ani $3$ rovny $0$ zřejmě nejsou.

 Okruhy $(R,+,-,0, \cdot ,1)$  s takovou vlastností jsou z číselného hlediska
 problematické, neboť na nich nelze žádným rozumným (vlastně ani nerozumným)
 způsobem definovat \emph{dělení}, tj. inverz k $\cdot$.

 Představme si totiž, že by na okruhu $(\{0,1,2,3,4,5\}, \oplus ,\ominus, 0,
 \odot, 1)$ existoval k prvku $2$ inverzní prvek $2^{-1}$ vzhledem k $\odot$.
 Pak bychom měli následující rovnosti:
 \begin{align*}
  (2^{-1} \odot 2) \odot 3 &= 1 \odot 3 = 3,\\
  2^{-1} \odot (2 \odot 3) &= 2^{-1} \odot 0 = 0,
 \end{align*}
 čili by operace $\odot$ \textbf{nemohla být asociativní}! To by byl už
 kompletní binec.
\end{warning}

Nepřítomnost takového problému v číselných oborech napovídá, že struktura okruhu
stále ještě není dostatečně striktní, abychom jejím prvkům mohli přezdít
\uv{čísla}. Ukazuje se, že ale stačí zakázat součinu dvou nenulových prvků býti
nulou, abychom se k číslům dostali. Taková struktura slove \emph{obor
integrity}; jmě, jež vrhá světlo na ustálené spojení \emph{číselné obory}.

\begin{definition}{Obor integrity}{obor-integrity}
 Okruh $(R,+,-,0, \cdot ,1)$ nazveme \emph{oborem integrity}, pokud pro každé
 dva $r,s \in R$ platí
 \[
  r \cdot s = 0 \Rightarrow r = 0 \vee s = 0.
 \]
\end{definition}

Čtenáři dobře učiní, vejmou-li, že tato vlastnost číselných oborů je hojně
využívána řekněme při řešení polynomiálních rovnic. Dokáži-li totiž rozložit
polynom na součin jeho lineárních činitelů, pak vím, že řešení rovnice
\[
 (x-a)(x-b)(x-c) = 0
\]
jsou právě čísla $a,b$ a $c$. \textbf{To však není pravda v obecném okruhu!}
Pouze struktura oboru integrity umožňuje činit takový závěr.

V oborech integrity lze \uv{sčítat}, \uv{odčítat} a \uv{násobit}. Nelze v nich
však \uv{dělit}. Součástí definice oboru integrity není existence inverzu k
operaci násobení. Struktury, které toto splňují, se jmenují \emph{tělesa} a
tvoří základ moderní geometrie. Nezamýšlejíce formalizovat zmíníme, že z každého
oboru integrity lze vyrobit těleso vlastně hrubým přidáním inverzů ke všem
prvkům. Tomuto procesu se říká \emph{lokalizace} a výsledné struktuře
\emph{podílové těleso}; lokalizace je způsobem, kterým se mimo jiné tvoří
racionální čísla z čísel celých.

\begin{definition}{Těleso}{teleso}
 Okruh $(F,+,-,0, \cdot ,1)$ (z angl. názvu pro těleso -- \textbf{f}ield)
 nazveme \emph{tělesem}, existuje-li ke každému prvku $f \in F$ inverz vzhledem
 k $ \cdot $, tj. prvek $f^{-1} \in F$ takový, že $f \cdot f^{-1} = f^{-1} \cdot
 f = 1$.
\end{definition}

Pozorní čtenáři jistě sobě povšimli, že v \hyperref[def:teleso]{definici tělesa}
\emph{nepožadujeme}, aby byl výchozí okruh oborem integrity. Existence inverzů
již tuto podmínku implikuje. Důkaz ponecháváme jako cvičení.

\begin{exercise}{}{okruh-inverz-obor}
 Dokažte, že každé těleso je oborem integrity.
\end{exercise}
