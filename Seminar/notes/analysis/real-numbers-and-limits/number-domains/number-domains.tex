\section{Číselné obory}
\label{sec:ciselne-obory}

Konstrukce číselných oborů je symetrizační proces. Přirozená čísla nejsou z
algebraického pohledu \uv{hezký} objekt, nejsou symetrická a všechny operace
jsou destruktivní -- ničí informaci o výchozím stavu. Kupříkladu operace $+$
provedená na dvojici čísel dá číslo $5$. Ovšem, nemám žádný způsob, jak se z
čísla $5$ vrátit zpět do čísel $2$ nebo $3$. V principu, v přirozených číslech
se lze pohybovat pouze jedním směrem a všechny objekty ponechané vzadu upadají v
trvalé zapomnění.

\begin{warning}{}{minus-na-N}
 Nezasvěcený, zmatený a zcela pomýlený čtenář by snad měl odvahu tvrdit, že
 přeci mohu číslo $3$ mohu od čísla $5$ \textbf{\clr{odečíst}} a získat tím zpět
 číslo $2$. Jistě, takové tvrzení by se mohlo stát předmětem vášnivých diskusí v
 anarchistických kroužcích velebitelů teorie polomnožin, v kterékoli algebraické
 teorii však nemá nižádné místo.

 Vyzýváme čtenáře, aby uvážili, že definovat \uv{operaci} $-$ na množině
 přirozených čísel, která vlastně není formálně operací, neboť funguje pouze
 tehdy, když je pravý argument větší nebo roven levému, není komutativní a není
 \textbf{ani asociativní} byl by čin vskutku ohyzdný.

 Znak $-$ bude mít své místo až v celých číslech, kde však rovněž nebude operací
 (stále není asociativní), bude pouze značit inverz vzhledem k operaci $+$.
\end{warning}

Tuto situaci vylepšují čísla celá, která přidávají inverzy k operaci $+$ a
tím tuto operaci symetrizují. Ovšem, operace $ \cdot $ si stále drží svůj
deformační charakter. Podobně jako tomu bylo u přirozených čísel s operacemi $+$
a $ \cdot $, v celých číslech operace $ \cdot $ rovněž není zvratná. Dostat se
ze součinu $-2 \cdot 3$ zpět na číslo $-2$ je nemožné.

Algebraicky nejdokonalejší jsou pak čísla racionální, která jsou již dokonale
symetrickou strukturou -- komutativním tělesem. Obě operace $+$ i $ \cdot $ jsou
již dokonale symetrické, zvratné prostřednictvím $-$ a $^{-1}$. Pozor -- podobně
jako odčítání, ani dělení \textbf{není operace}. Výraz $p / q$ je pohodlným
zápisem formálně korektního $pq^{-1}$ vyjadřujícího součin čísla $p$ s inverzem
k číslu $q$.

Nyní přikročíme ke konstrukcím číselných oborů. Naší první výzvou je konstrukce
množiny přirozených čísel $\N$. Stavebními kameny jsou množiny, tudíž přirozená
čísla sama musejí být rovněž množiny. Existuje mnoho axiomatických systémů (z
nich snad nejoblíbenější tzv.
\href{https://cs.wikipedia.org/wiki/Peanova_aritmetika}{Peanova aritmetika})
popisujících přirozená čísla, avšak, jako je tomu u axiomů vždy, nepodávají
žádnou představu o výsledné struktuře.

My ukážeme jednu konstruktivní definici, jejíž korektnost vyplývá z axiomů
teorie množin (speciálně z axiomu nekonečna), které zde však uvádět nechceme;
žádáme pročež čtenáře o jistou míru tolerance.

\begin{definition}{Přirozená čísla}{prirozena-cisla}
 Definujme $0 \coloneqq \emptyset$ a \uv{funkci následníka} jako $s(a) \coloneqq
 a \cup \{a\}$. Množina $\N$ přirozených čísel je taková množina, že $0 \in \N$
 a $s(n) \in \N$ pro každé $n \in \N$. Konkrétně, $\N$ jsou definována
 iterativně jako
 \begin{align*}
  0 &\coloneqq \emptyset,\\
  1 & \coloneqq 0 \cup \{0\} = \{\emptyset\} = \{0\},\\
  2 & \coloneqq 1 \cup \{1\} = \{\emptyset,\{\emptyset\}\} = \{0,1\},\\
  3 & \coloneqq 2 \cup \{2\} =
  \{\emptyset,\{\emptyset\},\{\emptyset,\{\emptyset\}\}\} = \{0,1,2\},\\
  \vdots
 \end{align*}
\end{definition}

Na přirozených číslech lze definovat operace $+$ a $ \cdot $. Ukážeme si jak.

\begin{definition}{Sčítání}{scitani}
Definujme operaci $+:\N \times \N \to \N$ induktivně předpisy (připomínáme, že
$s(n) \coloneqq n \cup \{n\}$)
 \begin{equation}
  \label{eq:natural-numbers-addition}
  \begin{split}
   n + m &= m + n,\\
   (n + m) + l &= n + (m + l),\\
   n + 0 &= n,\\
   s(n) + m &= s(n + m)
  \end{split}
 \end{equation}
 pro všechna $n,m,l \in \N$.
\end{definition}

Bohužel, definovat $+$ na přirozených číslech pomocí přímočaré množinové
konstrukce není úplně možné; museli jsme se tudíž uchýlit k částečně axiomatické
definici. První dvě rovnosti v~\eqref{eq:natural-numbers-addition} říkají, že
sčítání je komutativní a asociativní. Další dva pak popisují způsob, jak
spočítat součet $n + m$ pro libovolná dvě $n,m \in \N$. Totiž, všimněme si, že
\[
 n + m = 
\]

