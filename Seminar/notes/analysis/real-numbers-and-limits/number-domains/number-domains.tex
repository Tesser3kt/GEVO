\section{Číselné obory}
\label{sec:ciselne-obory}

Konstrukce číselných oborů je symetrizační proces. Přirozená čísla nejsou z
algebraického pohledu \uv{hezký} objekt, nejsou symetrická a všechny operace
jsou destruktivní -- ničí informaci o výchozím stavu. Kupříkladu operace $+$
provedená na dvojici čísel dá číslo $5$. Ovšem, nemám žádný způsob, jak se z
čísla $5$ vrátit zpět do čísel $2$ nebo $3$. V principu, v přirozených číslech
se lze pohybovat pouze jedním směrem a všechny objekty ponechané vzadu upadají v
trvalé zapomnění.

\begin{warning}{}{minus-na-N}
 Nezasvěcený, zmatený a zcela pomýlený čtenář by snad měl odvahu tvrdit, že
 přeci mohu číslo $3$ od čísla $5$ \textbf{\clr{odečíst}} a získat tím zpět
 číslo $2$. Jistě, takové tvrzení by se kvapně stalo předmětem vášnivých diskusí
 v anarchistických kroužcích velebitelů teorie polomnožin, v~kterékoli
 algebraické teorii však nemá nižádné místo.

 Vyzýváme čtenáře, aby uvážili, že definovat \uv{operaci minus} na množině
 přirozených čísel, která vlastně není formálně operací, neboť funguje pouze
 tehdy, když je pravý argument větší nebo roven levému, není komutativní a není
 \textbf{ani asociativní}, byl by čin vskutku ohyzdný.

 Znak $-$ bude mít své místo až v celých číslech, kde však rovněž nebude operací
 (stále není asociativní), bude pouze značit inverz vzhledem k operaci $+$.
\end{warning}

Tuto situaci vylepšují čísla celá, která přidávají inverzy k operaci $+$ a
tím tuto operaci symetrizují. Ovšem, operace $ \cdot $ si stále drží svůj
deformační charakter. Podobně jako tomu bylo u přirozených čísel s operacemi $+$
a $ \cdot $, v celých číslech operace $ \cdot $ rovněž není zvratná. Dostat se
ze součinu $-2 \cdot 3$ zpět na číslo $-2$ je nemožné.

Algebraicky nejdokonalejší jsou pak čísla racionální, která jsou již dokonale
symetrickou strukturou -- komutativním tělesem. Obě operace $+$ i $ \cdot $ jsou
symetrické, zvratné prostřednictvím $-$ a $^{-1}$. Pozor! Podobně jako odčítání,
ani dělení \textbf{není operace}. Výraz $p / q$ je pohodlným zápisem formálně
korektního $pq^{-1}$ vyjadřujícího součin čísla $p$ s inverzem k číslu $q$.

Racionální čísla však stále mají, nikoli z algebraického, nýbrž z analytického
pohledu, jednu podstatnou neduhu. Totiž, nerozumějí si dobře s pojmem
\emph{nekonečna}. Ukazuje se, že racionální čísla mají mezi sebou \uv{nekonečně
malé} díry nejsouce pročež vhodná při modelování fyzického světa, který jsme si
lidé zvykli vnímat jako \emph{souvislý}. Tuto neduhu lze odstranit, a to
konstrukcí čísel \emph{reálných}. Ta však nebude zdaleka tak jednoduchá jako
konstrukce ostatních číselných oborů, neboť z principu věci dožaduje aparátu pro
práci s nekonečně malými vzdálenostmi.

Nyní k samotným konstrukcím. Naší první výzvou je konstrukce množiny přirozených
čísel $\N$. Stavebními kameny jsou množiny, tudíž přirozená čísla sama musejí
být rovněž množiny. Existuje mnoho axiomatických systémů (z nich snad
nejoblíbenější tzv.
\href{https://cs.wikipedia.org/wiki/Peanova_aritmetika}{Peanova aritmetika})
popisujících přirozená čísla, avšak, jako je tomu u axiomů vždy, nepodávají
žádnou představu o výsledné struktuře.

My předvedeme jednu konstruktivní definici, jejíž korektnost vyplývá z axiomů
teorie množin (speciálně z axiomu nekonečna), které zde však uvádět nechceme;
žádáme pročež čtenáře o jistou míru tolerance.

\begin{definition}{Přirozená čísla}{prirozena-cisla}
 Definujme $0 \coloneqq \emptyset$ a \uv{funkci následníka} jako $s(a) \coloneqq
 a \cup \{a\}$. Množina $\N$ přirozených čísel je taková množina, že $0 \in \N$
 a $s(n) \in \N$ pro každé $n \in \N$. Konkrétně, $\N$ jsou definována
 iterativně jako
 \begin{align*}
  0 &\coloneqq \emptyset,\\
  1 & \coloneqq s(0) = 0 \cup \{0\} = \{\emptyset\} = \{0\},\\
  2 & \coloneqq s(1) = 1 \cup \{1\} = \{\emptyset,\{\emptyset\}\} = \{0,1\},\\
  3 & \coloneqq s(2) = 2 \cup \{2\} =
  \{\emptyset,\{\emptyset\},\{\emptyset,\{\emptyset\}\}\} = \{0,1,2\},\\
  \vdots
 \end{align*}
\end{definition}

Na přirozených číslech lze definovat operace $+$ a $ \cdot $. Ukážeme si zběžně
jak.

Přirozená čísla splňují tzv. axiom rekurze, který se obvykle zavádí v
axiomatické definici přirozených čísel. V rámci našeho konstruktivního přístupu
je třeba ho dokázat. My si ho zde však pouze uvedeme, neboť onen důkaz je silně
logický a zdlouhavý.

\begin{proposition}{Axiom rekurze}{axiom-rekurze}
 Ať $X$ je neprázdná množina a $x \in X$. Pak pro každé zobrazení $f:X \to X$
 existuje jednoznačně určené zobrazení $F:\N \to X$ takové, že $F(0) = x$ a
 $F(s(n)) = f(F(n)) \; \forall n \in \N$.
\end{proposition}

Lidsky řečeno, axiom rekurze říká, že přirozenými čísly je možné \uv{číslovat}
opakované (\emph{rekurzivní}) aplikace zobrazení $f$ na prvky množiny $X$
počínaje nějak pevně zvoleným prvkem. Vlastně vyrábíme nekonečný řetěz šipek
zobrazení $f$.

Uvažme například zobrazení na \myref{obrázku}{fig:zobrazeni-axiom-rekurze}.
\begin{figure}[ht]
 \centering
 \begin{tikzpicture}[scale=0.75]
  \foreach \n/\h in {a1/3,b1/2,c1/1,d1/0} {
   \tkzDefPoint(0,\h){\n}
  }
  \tkzDrawPoints[size=4](a1,b1,c1,d1)
  \tkzLabelPoint[left,xshift=-1mm](a1){$a$}
  \tkzLabelPoint[left,xshift=-1mm](b1){$b$}
  \tkzLabelPoint[left,xshift=-1mm](c1){$c$}
  \tkzLabelPoint[left,xshift=-1mm](d1){$d$}

  \tkzLabelPoint[above,yshift=3mm](a1){$X$}

  \tkzDefPoint(3,3){a2}
  \tkzDefPointsBy[translation= from a1 to a2](b1,c1,d1){b2,c2,d2}
  \tkzDrawPoints[size=4](a2,b2,c2,d2)

  \tkzLabelPoint[right,xshift=1mm](a2){$a$}
  \tkzLabelPoint[right,xshift=1mm](b2){$b$}
  \tkzLabelPoint[right,xshift=1mm](c2){$c$}
  \tkzLabelPoint[right,xshift=1mm](d2){$d$}

  \tkzLabelPoint[above,yshift=3mm](a2){$X$}

  \draw[-Latex,red,thick,shorten <= 4pt,shorten >= 4pt] (a1) -- (b2);
  \draw[-Latex,red,thick,shorten <= 4pt,shorten >= 4pt] (b1) -- (b2);
  \draw[-Latex,red,thick,shorten <= 4pt,shorten >= 4pt] (c1) -- (a2);
  \draw[-Latex,red,thick,shorten <= 4pt,shorten >= 4pt] (d1) -- (d2);
  
 \end{tikzpicture}

 \caption{Zobrazení $\clr{f}$ z \hyperref[prop:axiom-rekurze]{axiomu rekurze}.}
 \label{fig:zobrazeni-axiom-rekurze}
\end{figure}

Zde $X = \{a,b,c,d\}$ a za počáteční prvek zvolme třeba $c$. Podle
\myref{tvrzení}{prop:axiom-rekurze} existuje zobrazení $F:\N \to X$ začínající v
$c$ (tj. $F(0) = c$), které zobrazuje číslo $1$ na prvek, na který $f$ zobrazuje
$c$; číslo $2$ na prvek, na který $f$ zobrazuje ten prvek, na který zobrazuje
$c$, atd.


