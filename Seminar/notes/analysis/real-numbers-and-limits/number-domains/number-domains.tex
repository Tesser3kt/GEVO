\section{Číselné obory}
\label{sec:ciselne-obory}

Konstrukce číselných oborů je symetrizační proces. Přirozená čísla nejsou z
algebraického pohledu \uv{hezký} objekt, nejsou symetrická a všechny operace
jsou destruktivní -- ničí informaci o výchozím stavu. Kupříkladu operace $+$
provedená na dvojici čísel dá číslo $5$. Ovšem, nemám žádný způsob, jak se z
čísla $5$ vrátit zpět do čísel $2$ nebo $3$. V principu, v přirozených číslech
se lze pohybovat pouze jedním směrem a všechny objekty ponechané vzadu upadají v
trvalé zapomnění.

\begin{warning}{}{minus-na-N}
 Nezasvěcený, zmatený a zcela pomýlený čtenář by snad měl odvahu tvrdit, že
 přeci mohu číslo $3$ od čísla $5$ \textbf{\clr{odečíst}} a získat tím zpět
 číslo $2$. Jistě, takové tvrzení by se kvapně stalo předmětem vášnivých diskusí
 v anarchistických kroužcích velebitelů teorie polomnožin, v~kterékoli
 algebraické teorii však nemá nižádné místo.

 Vyzýváme čtenáře, aby uvážili, že definovat \uv{operaci minus} na množině
 přirozených čísel, která vlastně není formálně operací, neboť funguje pouze
 tehdy, když je pravý argument větší nebo roven levému, není komutativní a není
 \textbf{ani asociativní}, byl by čin vskutku ohyzdný.

 Znak $-$ bude mít své místo až v celých číslech, kde však rovněž nebude operací
 (stále není asociativní), bude pouze značit inverz vzhledem k operaci $+$.
\end{warning}

Tuto situaci vylepšují čísla celá, která přidávají inverzy k operaci $+$ a
tím tuto operaci symetrizují. Ovšem, operace $ \cdot $ si stále drží svůj
deformační charakter. Podobně jako tomu bylo u přirozených čísel s operacemi $+$
a $ \cdot $, v celých číslech operace $ \cdot $ rovněž není zvratná. Dostat se
ze součinu $-2 \cdot 3$ zpět na číslo $-2$ je nemožné.

Algebraicky nejdokonalejší jsou pak čísla racionální, která jsou již dokonale
symetrickou strukturou -- komutativním tělesem. Obě operace $+$ i $ \cdot $ jsou
symetrické, zvratné prostřednictvím $-$ a $^{-1}$. Pozor! Podobně jako odčítání,
ani dělení \textbf{není operace}. Výraz $p / q$ je pohodlným zápisem formálně
korektního $pq^{-1}$ vyjadřujícího součin čísla $p$ s inverzem k číslu $q$.

Racionální čísla však stále mají, nikoli z algebraického, nýbrž z analytického
pohledu, jednu podstatnou neduhu. Totiž, nerozumějí si dobře s pojmem
\emph{nekonečna}. Ukazuje se, že racionální čísla mají mezi sebou \uv{nekonečně
malé} díry nejsouce pročež vhodná při modelování fyzického světa, který jsme si
lidé zvykli vnímat jako \emph{souvislý}. Tuto neduhu lze odstranit, a to
konstrukcí čísel \emph{reálných}. Ta však nebude zdaleka tak jednoduchá jako
konstrukce ostatních číselných oborů, neboť z principu věci dožaduje aparátu pro
práci s nekonečně malými vzdálenostmi.

Nyní k samotným konstrukcím. Naší první výzvou je konstrukce množiny přirozených
čísel $\N$. Stavebními kameny jsou množiny, tudíž přirozená čísla sama musejí
být rovněž množiny. Existuje mnoho axiomatických systémů (z nich snad
nejoblíbenější tzv.
\href{https://cs.wikipedia.org/wiki/Peanova_aritmetika}{Peanova aritmetika})
popisujících přirozená čísla, avšak, jako je tomu u axiomů vždy, nepodávají
žádnou představu o výsledné struktuře.

My předvedeme jednu konstruktivní definici, jejíž korektnost vyplývá z axiomů
teorie množin (speciálně z axiomu nekonečna), které zde však uvádět nechceme;
žádáme pročež čtenáře o jistou míru tolerance.

\begin{definition}{Přirozená čísla}{prirozena-cisla}
 Definujme $0 \coloneqq \emptyset$ a \uv{funkci následníka} jako $s(a) \coloneqq
 a \cup \{a\}$. Množina $\N$ přirozených čísel je taková množina, že $0 \in \N$
 a $s(n) \in \N$ pro každé $n \in \N$. Konkrétně, $\N$ jsou definována
 iterativně jako
 \begin{align*}
  0 &\coloneqq \emptyset,\\
  1 & \coloneqq s(0) = 0 \cup \{0\} = \{\emptyset\} = \{0\},\\
  2 & \coloneqq s(1) = 1 \cup \{1\} = \{\emptyset,\{\emptyset\}\} = \{0,1\},\\
  3 & \coloneqq s(2) = 2 \cup \{2\} =
  \{\emptyset,\{\emptyset\},\{\emptyset,\{\emptyset\}\}\} = \{0,1,2\},\\
  \vdots
 \end{align*}
\end{definition}

Na přirozených číslech lze definovat operace $+$ a $ \cdot $. Ukážeme si zběžně
jak.

Přirozená čísla splňují tzv. axiom rekurze, který se obvykle zavádí v
axiomatické definici přirozených čísel. V rámci našeho konstruktivního přístupu
je třeba ho dokázat. My si ho zde však pouze uvedeme, neboť onen důkaz je silně
logický a zdlouhavý.

\begin{proposition}{Axiom rekurze}{axiom-rekurze}
 Ať $X$ je neprázdná množina a $x \in X$. Pak pro každé zobrazení $f:X \to X$
 existuje jednoznačně určené zobrazení $F:\N \to X$ takové, že $F(0) = x$ a
 $F(s(n)) = f(F(n)) \; \forall n \in \N$.
\end{proposition}

Lidsky řečeno, axiom rekurze říká, že přirozenými čísly je možné \uv{číslovat}
opakované (\emph{rekurzivní}) aplikace zobrazení $f$ na prvky množiny $X$
počínaje jakýmsi pevně zvoleným prvkem. Vlastně vyrábíme nekonečný řetěz šipek
zobrazení $f$.

Uvažme například zobrazení na \myref{obrázku}{fig:zobrazeni-axiom-rekurze}.
\begin{figure}[ht]
 \centering
 \begin{tikzpicture}[scale=0.75]
  \foreach \n/\h in {a1/3,b1/2,c1/1,d1/0} {
   \tkzDefPoint(0,\h){\n}
  }
  \tkzDrawPoints[size=4](a1,b1,c1,d1)
  \tkzLabelPoint[left,xshift=-1mm](a1){$a$}
  \tkzLabelPoint[left,xshift=-1mm](b1){$b$}
  \tkzLabelPoint[left,xshift=-1mm](c1){$c$}
  \tkzLabelPoint[left,xshift=-1mm](d1){$d$}

  \tkzLabelPoint[above,yshift=3mm](a1){$X$}

  \tkzDefPoint(3,3){a2}
  \tkzDefPointsBy[translation= from a1 to a2](b1,c1,d1){b2,c2,d2}
  \tkzDrawPoints[size=4](a2,b2,c2,d2)

  \tkzLabelPoint[right,xshift=1mm](a2){$a$}
  \tkzLabelPoint[right,xshift=1mm](b2){$b$}
  \tkzLabelPoint[right,xshift=1mm](c2){$c$}
  \tkzLabelPoint[right,xshift=1mm](d2){$d$}

  \tkzLabelPoint[above,yshift=3mm](a2){$X$}

  \draw[-Latex,BrickRed,thick,shorten <= 4pt,shorten >= 4pt] (a1) -- (b2);
  \draw[-Latex,BrickRed,thick,shorten <= 4pt,shorten >= 4pt] (b1) -- (b2);
  \draw[-Latex,BrickRed,thick,shorten <= 4pt,shorten >= 4pt] (c1) -- (a2);
  \draw[-Latex,BrickRed,thick,shorten <= 4pt,shorten >= 4pt] (d1) -- (d2);
  
 \end{tikzpicture}

 \caption{Zobrazení $\clr{f}$ z \hyperref[prop:axiom-rekurze]{axiomu rekurze}.}
 \label{fig:zobrazeni-axiom-rekurze}
\end{figure}

Zde $X = \{a,b,c,d\}$ a za počáteční prvek zvolme třeba $c$. Podle
\myref{tvrzení}{prop:axiom-rekurze} existuje zobrazení $F:\N \to X$ začínající v
$c$ (tj. $F(0) = c$), které zobrazuje číslo $1$ na prvek, na který $f$ zobrazuje
$c$; číslo $2$ na prvek, na který $f$ zobrazuje ten prvek, na který zobrazuje
$c$; číslo $3$ na prvek, na který $f$ zobrazuje ten prvek, na který zobrazuje
ten prvek, na který zobrazuje $c$; číslo $4$ ... radši nic ... Snad lepší
představu poskytne \myref{obrázek}{fig:axiom-rekurze-cislovac}.

\begin{figure}[ht]
 \centering
 \begin{tikzpicture}
  \foreach \n/\h in {a1/3,b1/2,c1/1,d1/0} {
   \tkzDefPoint(0,\h){\n}
  }
  \tkzDrawPoints[size=4](a1,b1,d1)
  \tkzDrawPoint[size=4,RoyalBlue](c1)
  \tkzLabelPoint[left,xshift=-1mm](a1){$a$}
  \tkzLabelPoint[left,xshift=-1mm](b1){$b$}
  \tkzLabelPoint[left,xshift=-1mm](c1){$\clb{F(0)} = c$}
  \tkzLabelPoint[left,xshift=-1mm](d1){$d$}

  \tkzDefPoint(3,3){a2}
  \tkzDefPointsBy[translation= from a1 to a2](b1,c1,d1){b2,c2,d2}
  \tkzDrawPoints[size=4](b2,c2,d2)
  \tkzDrawPoint[size=4,RoyalBlue](a2)

  \tkzLabelPoint[above](a2){$a = \clb{F(1)} = \clr{f(c)}$}
  \tkzLabelPoint[above](b2){$b$}
  \tkzLabelPoint[above](c2){$c$}
  \tkzLabelPoint[above](d2){$d$}

  \tkzDefPoint(6,3){a3}
  \tkzDefPointsBy[translation= from a2 to a3](b2,c2,d2){b3,c3,d3}
  \tkzDrawPoints[size=4](a3,c3,d3)
  \tkzDrawPoint[size=4,RoyalBlue](b3)

  \tkzLabelPoint[above](a3){$a$}
  \tkzLabelPoint[below,yshift=-1mm](b3){$b = \clb{F(2)} = \clr{f(a)}$}
  \tkzLabelPoint[above](c3){$c$}
  \tkzLabelPoint[above](d3){$d$}

  \tkzDefPoint(9,3){a4}
  \tkzDefPointsBy[translation= from a3 to a4](b3,c3,d3){b4,c4,d4}
  \tkzDrawPoints[size=4](a4,c4,d4)
  \tkzDrawPoint[size=4,RoyalBlue](b4)

  \tkzLabelPoint[above](a4){$a$}
  \tkzLabelPoint[below,yshift=-1mm](b4){$b = \clb{F(3)} = \clr{f(b)}$}
  \tkzLabelPoint[above](c4){$c$}
  \tkzLabelPoint[above](d4){$d$}

  \draw[-Latex,BrickRed,thick,shorten <= 4pt,shorten >= 4pt,opacity=0.3] (a1) -- (b2);
  \draw[-Latex,BrickRed,thick,shorten <= 4pt,shorten >= 4pt,opacity=0.3] (b1) -- (b2);
  \draw[-Latex,RoyalBlue,thick,shorten <= 4pt,shorten >= 4pt] (c1) -- (a2);
  \draw[-Latex,BrickRed,thick,shorten <= 4pt,shorten >= 4pt,opacity=0.3] (d1) -- (d2);

  \draw[-Latex,BrickRed,thick,shorten <= 4pt,shorten >= 4pt,opacity=0.3] (b2) -- (b3);
  \draw[-Latex,BrickRed,thick,shorten <= 4pt,shorten >= 4pt,opacity=0.3] (c2) -- (a3);
  \draw[-Latex,BrickRed,thick,shorten <= 4pt,shorten >= 4pt,opacity=0.3] (d2) -- (d3);
  \draw[-Latex,RoyalBlue,thick,shorten <= 4pt,shorten >= 4pt] (a2) -- (b3);

  \draw[-Latex,RoyalBlue,thick,shorten <= 4pt,shorten >= 4pt] (b3) -- (b4);
  \draw[-Latex,BrickRed,thick,shorten <= 4pt,shorten >= 4pt,opacity=0.3] (c3) -- (a4);
  \draw[-Latex,BrickRed,thick,shorten <= 4pt,shorten >= 4pt,opacity=0.3] (d3) -- (d4);
  \draw[-Latex,BrickRed,thick,shorten <= 4pt,shorten >= 4pt,opacity=0.3] (a3) -- (b4);

  \draw[-Latex,RoyalBlue,thick,shorten <= 4pt,shorten >= 4pt] (b4) -- ($(b4) +
   (3,0)$);
  \node[right=2.9cm of b4] {\Large $\ldots$};
  
 \end{tikzpicture}

 \caption{Zobrazení $\clb{F}$ jako \uv{rekurzor} zobrazení $\clr{f}$ s
 počátečním bodem $\clb{F(0)} = c$.}
 \label{fig:axiom-rekurze-cislovac}
\end{figure}

Vybaveni \hyperref[prop:axiom-rekurze]{axiomem rekurze}, můžeme nyní definovat
operaci $+:\N \times \N \to \N$. Začneme tím, že definujeme zobrazení \uv{přičti
$n$}. Zvolme za zobrazení $f$ v \hyperref[prop:axiom-rekurze]{axiomu rekurze}
funkci následníka ${s:\N \to \N}$ definovanou $s(n) = n \cup \{n\}$. Je zřejmé,
že zobrazení \uv{přičti $n$}, pracovně označené $+_n$, musí číslo $0$ zobrazit
na $n$. Podle \hyperref[prop:axiom-rekurze]{axiomu rekurze} však existuje pouze
jediné zobrazení $+_n:\N \to \N$ splňující
\begin{align*}
 +_n(0) &= n,\\
 +_n(s(m)) &= s(+_n(m)) \; \forall m \in \N.
\end{align*}
Uvědomme si, že druhá rovnost je též velmi přirozeným požadavkem pro operaci
sčítání. Říká totiž, že následník čísla $m + n$ je tentýž jako následník čísla
$m$ sečtený s $n$.

Konečně, na $\N$ definujeme operaci $+$ předpisem
\[
 m + n \coloneqq +_n(m).
\]
V každé učebnici základů teorie množin a matematické logiky dá nyní nějakou
práci osvětlit, že takto definovaná operace $+$ je komutativní a asociativní a
že se obdobným způsobem dá definovat operace násobení. Naštěstí! Tento text není
výkladem ani jedné z pokulhávajících disciplín, a tedy těchto několik malých
kroků pro člověka a stejně tak malých kroků pro matematiku přeskočíme a věnovati
sebe dalším oborům číselným budeme.

Zcela striktně vzato, $\N$ ještě nejsou \emph{oborem}. Nejsou vlastně ani
okruhem. Přestože $(\N, \cdot ,1)$ je komutativní monoid, $(\N,+,0)$ zcela jistě
není komutativní grupa, ano rovněž pouze komutativní monoid. Takovým strukturám
se často říká (snad jen proto, aby se jim prostě nějak říkalo, ačkoliv nikoho
zvlášť nezajímají) \emph{polookruhy}. Situaci vylepšují čísla celá.

Podobně jako čísla přirozená, i čísla celá lze definovat mnoha způsoby. Uvedeme
si jeden. Na množině $\N \times \N$ dvojic přirozených čísel definujme relaci
$ \sim _{\Z}$ předpisem
\[
 (a,b) \sim _{\Z} (c,d) \overset{\text{def}}{ \iff } a + d = b +
 c.
\]
Třídám ekvivalence dvojic přirozených čísel podle $ \sim _{\Z}$ budeme říkat
\emph{celá čísla}.

\begin{definition}{Celá čísla}{cela-cisla}
 Množinu celých čísel $\Z$ definujeme jako
 \[
  \Z \coloneqq \{[(a,b)]_{ \sim _{\Z}} \mid (a,b) \in \N \times \N\}.
 \]
 Operace $+$ a $ \cdot $ na $\N$ indukují operace na $\Z$, které budeme
 označovat stejnými symboly. Konkrétně, definujme
 \begin{align*}
  [(a,b)]_{ \sim _{\Z}} + [(c,d)]_{ \sim \Z} &\coloneqq [(a+c,b+d)]_{ \sim
  \Z},\\
   [(a,b)]_{ \sim _{\Z}} \cdot [(c,d)]_{ \sim _{\Z}} & \coloneqq [(a \cdot c +
   b \cdot d,a \cdot d + b \cdot c)]_{ \sim _{\Z}}.
 \end{align*}
 Pro všechna $a,b \in \N$ navíc platí
 \[
  [(a,b)]_{ \sim _{\Z}} + [(b,a)]_{ \sim _{\Z}} = [(a+b,b+a)]_{ \sim _{\Z}} =
  [(0,0)]_{ \sim _{\Z}},
 \]
 kde předposlední rovnost platí, protože $+$ je komutativní. Čili, prvek
 $[(b,a)]_{ \sim _{\Z}}$ je inverzní k~prvku $[(a,b)]_{ \sim _{\Z}}$ vzhledem k
 $+$. Značíme ho $-[(a,b)]_{ \sim _{\Z}}$. Odtud plyne, že $(\Z,+,-,[(0,0)]_{
 \sim _{\Z}})$ je komutativní grupa, pročež je
 \[
  (\Z,+,-,[(0,0)]_{ \sim _{\Z}}, \cdot ,[(1,0)]_{ \sim_{\Z}})
 \]
 komutativní okruh. Je snadné si uvědomit, že je to rovněž obor integrity.
\end{definition}

Čtenáře snad povaha množiny $\Z$ z \hyperref[def:cela-cisla]{předchozí definice}
zaráží. Zcela jistě to není ta \uv{obvyklá}. Ovšem, přechod od této verze celých
čísel k té běžně užívané je zcela bezbolestný. Stačí se totiž dívat na třídy
ekvivalence $[(a,b)]_{ \sim _{\Z}}$ jako na \uv{čísla} $a-b$. Ponecháváme na
čtenáři, aby ověřil, že definice operací $+$ a $-$ v naší verzi $\Z$ odpovídají
těm na celých číslech v jejich obvyklé podobě. My budeme této korespondence drze
využívat bez varování a mluvit o oboru integrity $(\Z,+,-,0, \cdot ,1)$.

\begin{exercise}{Hrátky s celými čísly}{hratky-s-celymi-cisly}
 Množinou $\Z$ zde myslíme tu z \myref{definice}{def:cela-cisla}. Ověřte, že
 \begin{enumerate}
  \item relace $ \sim _{\Z}$ je skutečně ekvivalence;
  \item operace $+$ a $ \cdot $ jsou dobře definované. To znamená, že nezávisí
   na volbě konkrétního reprezentanta z každé třídy ekvivalence. Ještě
   konkrétněji, dobrá definovanost zde značí fakt, že
   \begin{align*}
    [(a,b)]_{ \sim_{\Z}} + [(c,d)]_{ \sim _{\Z}} &= [(a',b')]_{ \sim _{\Z}} +
    [(c',d')]_{ \sim _{\Z}},\\
    [(a,b)]_{ \sim_{\Z}} \cdot [(c,d)]_{ \sim _{\Z}} &= [(a',b')]_{ \sim _{\Z}}
    \cdot [(c',d')]_{ \sim _{\Z}},
   \end{align*}
   kdykoli $(a,b) \sim _{\Z} (a',b')$ a $(c,d) \sim _{\Z} (c',d')$;
  \item operace $+$, $-$ a inverz $-$ podle naší definice souhlasí s operacemi
   danými stejnými symboly na \uv{běžné} verzi celých čísel při korespondenci
   \[
    [(a,b)]_{ \sim_{\Z}} \leftrightarrow a - b.
   \]
   Konkrétně, pro operaci $+$ toto znamená, že platí korespondence
   \[
    [(a,b)]_{ \sim_{\Z}} + [(c,d)]_{ \sim _{\Z}} \leftrightarrow (a - b) + (c -
    d)
   \]
   a nezávisí na výběru reprezentanta z tříd ekvivalence $[(a,b)]_{ \sim _{\Z}}$
   a $[(c,d)]_{ \sim_{\Z}}$.
 \end{enumerate}
\end{exercise}

Přechod od celých čísel k racionálním znamená definovat na celých číslech
\uv{dělení} -- v algebraické hantýrce definovat inverz k operaci $ \cdot $ a
učiniti tímť z oboru $(\Z,+,-,0, \cdot ,1)$ těleso. Ten je překvapivě snadný
úkol a proces \uv{racionalizace}, nazývaný oficiálně \emph{lokalizace}, lze v
podstatě krok po kroku replikovat pro libovolný obor integrity.

