\section{Číselné řady}
\label{sec:ciselne-rady}

Speciální, a pro rozvoj diferenciálního kalkulu zcela nezanedbatelnou, čeledí
posloupností jsou tzv. \emph{číselné řady}. Intuitivně a téměř i formálně jsou
číselné řady vlastně součty nekonečného počtu (reálných) čísel. Přechod od
posloupnosti k číselné řadě je přímočarý -- sestrojíme zkrátka součet všech
prvků oné posloupnosti \emph{zachovajíce jejich pořadí}. Jest ovšem dlužno dbáti
skutku, žeť číselné řady jsou samy posloupnostmi. Totiž, limita posloupnosti,
jejíž $n$-tý člen je právě součtem $n$ prvních členů posloupnosti druhé, je
rovněž přesně součtem číselné řady této druhé posloupnosti. Tímto způsobem se
součty číselných řad definují, což umožňuje k jejich studiu využít dokázaných
tvrzení o limitách posloupností z předchozích oddílů.

\begin{definition}{Posloupnost částečných součtů}{posloupnost-castecnych-souctu}
 Ať $a:\N \to \R$ je posloupnost. \emph{Posloupností částečných součtů}
 posloupnosti $a$ nazveme posloupnost $s:\N \to \R$ definovanou předpisem
 \[
  s_n \coloneqq \sum_{i = 0}^{n} a_i.
 \]
\end{definition}

\begin{example}{}{castecne-soucty}
 Je-li $a:\N \to \R$ posloupnost $0,1,2,3,\ldots$ (tj. $a_n = n$), pak
 posloupnost jejích částečných součtů je $0,0+1,0+1+2,0+1+2+3,\ldots$, neboli
 \[
  s_n = \sum_{i=0}^{n} i.
 \]
\end{example}

\begin{definition}{Součet číselné řady}{soucet-ciselne-rady}
 Je-li $a:\N \to \R$, pak výraz $\sum_{n=0}^{\infty} a_n$ nazýváme
 \emph{číselnou řadou} posloupnosti $a$. Ať je dále $s:\N \to \R$ posloupností
 částečných součtů posloupnosti $a$. Existuje-li $\lim_{n \to \infty} s_n$ (ne
 nutně konečná), pak řkouce, že řada $\sum_{n=0}^{\infty} a_n$ \emph{má součet},
 definujeme
 \[
  \sum_{n=0}^{\infty} a_n \coloneqq \lim_{n \to \infty} s_n.
 \]
 Slovně vyjádřeno, součtem číselné řady míníme limitu částečných součtů
 posloupnosti jejích členů.
\end{definition}

\begin{problem}{}{soucet-rady-pres-castecne-soucty}
 Spočtěte
 \[
  \sum_{n=1}^{\infty} \frac{1}{n(n+1)}.
 \]
\end{problem}
\begin{probsol}
 Posloupnost odpovídající zadané řadě je $a_n = 1 / n(n+1)$. Tudíž, posloupnost
 jejích částečných součtů je dána předpisem
 \[
  s_n = \sum_{i=1}^{n} a_i = \sum_{i=1}^n \frac{1}{i(i+1)}.
 \]
 Naším úkolem je spočítat $\sum_{n=1}^{\infty} a_n = \lim_{n \to \infty} s_n$.
 Za tímto účelem nalezneme \uv{hezčí} vyjádření posloupnosti $s_n$. Snadno
 ověříme, že platí
 \[
  \frac{1}{n(n+1)} = \frac{1}{n} - \frac{1}{n-1}
 \]
 pro každé $n \in \N$. Dále pak
 \[
  \sum_{i = 1}^{n} \frac{1}{i(i+1)} = \sum_{i=1}^n \left( \frac{1}{i} -
  \frac{1}{i+1} \right) = 1 - \frac{1}{2} + \frac{1}{2} - \frac{1}{3} + \ldots +
  \frac{1}{n} - \frac{1}{n + 1} = 1 - \frac{1}{n+1}.
 \]
 Odtud plyne, že
 \[
  \lim_{n \to \infty} s_n = \lim_{n \to \infty} \sum_{i = 1}^{n}
  \frac{1}{i(i+1)} = \lim_{n \to \infty} 1 - \frac{1}{n+1} = 1,
 \]
 čili rovněž $\sum_{n=1}^{\infty} \frac{1}{n(n+1)} = 1$.
\end{probsol}


