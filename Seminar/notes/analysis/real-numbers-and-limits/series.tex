\section{Číselné řady}
\label{sec:ciselne-rady}

Speciální, a pro rozvoj diferenciálního kalkulu zcela nezanedbatelnou, čeledí
posloupností jsou tzv. \emph{číselné řady}. Intuitivně a téměř i formálně jsou
číselné řady vlastně součty nekonečného počtu (reálných) čísel. Přechod od
posloupnosti k číselné řadě je přímočarý -- sestrojíme zkrátka součet všech
prvků oné posloupnosti \emph{zachovajíce jejich pořadí}. Jest ovšem dlužno dbáti
skutku, žeť číselné řady jsou samy posloupnostmi. Totiž, limita posloupnosti,
jejíž $n$-tý člen je právě součtem $n$ prvních členů posloupnosti druhé, je
rovněž přesně součtem číselné řady této druhé posloupnosti. Tímto způsobem se
součty číselných řad definují, což umožňuje k jejich studiu využít dokázaných
tvrzení o limitách posloupností z předchozích oddílů.

\begin{definition}{Posloupnost částečných součtů}{posloupnost-castecnych-souctu}
 Ať $a:\N \to \R$ je posloupnost. \emph{Posloupností částečných součtů}
 posloupnosti $a$ nazveme posloupnost $s:\N \to \R$ definovanou předpisem
 \[
  s_n \coloneqq \sum_{i = 0}^{n} a_i.
 \]
\end{definition}

\begin{example}{}{castecne-soucty}
 Je-li $a:\N \to \R$ posloupnost $0,1,2,3,\ldots$ (tj. $a_n = n$), pak
 posloupnost jejích částečných součtů je $0,0+1,0+1+2,0+1+2+3,\ldots$, neboli
 \[
  s_n = \sum_{i=0}^{n} i.
 \]
\end{example}

\begin{figure}[ht]
 \centering
 \begin{tikzpicture}
  \tkzInit[xmin=0,xmax=7,ymin=0,ymax=3]
  \foreach \n in {0,1,...,11} {
   \tkzDefPoint(0.5 * (\n + 1),0){x\n}
   \tkzDrawPoint[shape=cross](x\n)
   \tkzLabelPoint[below](x\n){$\n$}
  }
  \foreach \y in {1,2,3,4,5} {
   \tkzDefPoint(0,0.5 * \y){y\y}
   \tkzDrawPoint[shape=cross](y\y)
   \tkzLabelPoint[left](y\y){$\y$}
  }
  
  \tkzDefPoint(0.5,0.2){a0}
  \tkzDefPoint(1.0,0.2){a1}
  \tkzDefPoint(1.5,0.2){a2}
  \tkzDefPoint(2,0.2){a3}
  \tkzDefPoint(2.5,0.2){a4}
  \tkzDefPoint(3,0.2){a5}
  \tkzDefPoint(3.5,0.2){a6}
  \tkzDefPoint(4,0.2){a7}
  \tkzDefPoint(4.5,0.2){a8}
  \tkzDefPoint(5,0.2){a9}
  \tkzDefPoint(5.5,0.2){a10}
  \tkzDefPoint(6,0.2){a11}

  \tkzDefPoint(0.5,0.2){s0}
  \tkzDefPoint(1.0,0.4){s1}
  \tkzDefPoint(1.5,0.6){s2}
  \tkzDefPoint(2,0.8){s3}
  \tkzDefPoint(2.5,1){s4}
  \tkzDefPoint(3,1.2){s5}
  \tkzDefPoint(3.5,1.4){s6}
  \tkzDefPoint(4,1.6){s7}
  \tkzDefPoint(4.5,1.8){s8}
  \tkzDefPoint(5,2){s9}
  \tkzDefPoint(5.5,2.2){s10}
  \tkzDefPoint(6,2.4){s11}

  \tkzDrawX[>=latex,label={$n$}]
  \tkzDrawY[>=latex,label={$\clr{a_n}, \clb{s_n}$}]

  \tkzDrawPoints[color=BrickRed](a0,a1,a2,a3,a4,a5,a6,a7,a8,a9,a10,a11)
  \tkzDrawPoints[color=RoyalBlue](s0,s1,s2,s3,s4,s5,s6,s7,s8,s9,s10,s11)

  \foreach \n in {0,1,...,10} {
   \pgfmathparse{\n+1}
   \edef\m{\pgfmathresult}
   \tkzDrawSegment[dashed,color=BrickRed](a\n,a\m)
   \tkzDrawSegment[dashed,color=RoyalBlue](s\n,s\m)
  }
 \end{tikzpicture}

 \caption{Konstantní posloupnost $\clr{a_n}$ a její posloupnost částečných
 součtů $\clb{s_n}$.}
 \label{fig:castecne-soucty}
\end{figure}

\begin{definition}{Součet číselné řady}{soucet-ciselne-rady}
 Je-li $a:\N \to \R$, pak výraz $\sum_{n=0}^{\infty} a_n$ nazýváme
 \emph{číselnou řadou} posloupnosti $a$. Ať je dále $s:\N \to \R$ posloupností
 částečných součtů posloupnosti $a$. Existuje-li $\lim_{n \to \infty} s_n$ (ne
 nutně konečná), pak řkouce, že řada $\sum_{n=0}^{\infty} a_n$ \emph{má součet},
 definujeme
 \[
  \sum_{n=0}^{\infty} a_n \coloneqq \lim_{n \to \infty} s_n.
 \]
 Slovně vyjádřeno, součtem číselné řady míníme limitu částečných součtů
 posloupnosti jejích členů.
\end{definition}

\begin{warning}{}{rada-vs-castecne-soucty}
 Uvědomme sobě, že \textbf{součet číselné řady}, existuje-li, \textbf{závisí na
 všech jejích členech}. Obecně tedy dvě řady lišíce se pouze konečným počtem
 členů mají \textbf{různý} součet. Například platí (ale nedokážeme si to), že
 \[
  \sum_{n=1}^{\infty} \frac{1}{n^2} = \frac{\pi^2}{6}.
 \]
 Pak ale zřejmě
 \[
  \sum_{n=2}^{\infty} \frac{1}{n^2} = \sum_{n=1}^{\infty} \frac{1}{n^2} - 1 =
  \frac{\pi^2}{6} - 1,
 \]
 tedy ony dvě řady mají rozdílné součty.

 Tato situace však \textbf{nenastává} u posloupností částečných součtů $s_n$,
 neboť $n$-tý člen takové posloupnosti je součtem prvních $n$ členů sesterské
 číselné řady. Tudíž platí (jako u každé posloupnosti) $\lim_{n \to \infty}
 s_{n+k} = \lim_{n \to \infty} s_n$ pro libovolnou konstantu $k \in \N$.
\end{warning}

\begin{problem}{}{soucet-rady-pres-castecne-soucty}
 Spočtěte
 \[
  \sum_{n=1}^{\infty} \frac{1}{n(n+1)}.
 \]
\end{problem}
\begin{probsol}
 Posloupnost odpovídající zadané řadě je $a_n = 1 / n(n+1)$. Tudíž, posloupnost
 jejích částečných součtů je dána předpisem
 \[
  s_n = \sum_{i=1}^{n} a_i = \sum_{i=1}^n \frac{1}{i(i+1)}.
 \]
 Naším úkolem je spočítat $\sum_{n=1}^{\infty} a_n = \lim_{n \to \infty} s_n$.
 Za tímto účelem nalezneme \uv{hezčí} vyjádření posloupnosti $s_n$. Snadno
 ověříme, že platí
 \[
  \frac{1}{n(n+1)} = \frac{1}{n} - \frac{1}{n-1}
 \]
 pro každé $n \in \N$. Dále pak
 \[
  \sum_{i = 1}^{n} \frac{1}{i(i+1)} = \sum_{i=1}^n \left( \frac{1}{i} -
  \frac{1}{i+1} \right) = 1 - \frac{1}{2} + \frac{1}{2} - \frac{1}{3} + \ldots +
  \frac{1}{n} - \frac{1}{n + 1} = 1 - \frac{1}{n+1}.
 \]
 Odtud plyne, že
 \[
  \lim_{n \to \infty} s_n = \lim_{n \to \infty} \sum_{i = 1}^{n}
  \frac{1}{i(i+1)} = \lim_{n \to \infty} 1 - \frac{1}{n+1} = 1,
 \]
 čili rovněž $\sum_{n=1}^{\infty} \frac{1}{n(n+1)} = 1$.
\end{probsol}

\begin{problem}{Geometrická řada}{geometricka-rada}
 Dokažte, že řada $\sum_{n=0}^{\infty} q^{n}$, kde $q \in \R$, má
 \textbf{konečný} součet právě tehdy, když $|q| < 1$, a v tomto případě platí
 \[
  \sum_{n=0}^{\infty} q_n = \frac{1}{1-q}.
 \]
\end{problem}
\begin{probsol}
 Ukážeme nejprve, že pro libovolné $n \in \N$ a $q \neq 1$ platí
 \[
  s_n \coloneqq \sum_{k=0}^n q^{k} = \frac{1 - q^{n+1}}{1 - q}.
 \]
 V případě, že $q = 1$, máme zřejmě $s_n = n + 1$, neboť $1^{k} = 1$ pro všechna
 $k \in \N$. Pro $q \neq 1$ potom
 \begin{equation*}
  \begin{array}{ccccccccccccc}
   s_n & = & q^{0} & + & q^{1} & + & q^{2} & + & \ldots & + & q^{n} & &\\
   q \cdot s_n & = & & & q^{1} & + & q^2 & + & \ldots & + & q^{n} & + & q^{n+1},
  \end{array}
 \end{equation*}
 čili $s_n - q \cdot s_n = q^{0} - q^{n+1} = 1 - q^{n+1}$, z čehož snadno
 spočteme, že
 \[
  s_n = \frac{1-q^{n+1}}{1 - q}.
 \]
 Dále víme, že $\lim_{n \to \infty} q^{n} = 0$ pro $|q| < 1$ a pro $|q| > 1$ je
 tato limita nekonečná nebo neexistuje. Je-li $q = 1$, tak $\lim_{n \to \infty}
 q^{n} = 1$, a je-li $q = -1$, tak $\lim_{n \to \infty} q_n$ neexistuje.

 Odtud a z \hyperref[thm:aritmetika-limit]{věty o aritmetice limit},
 \[
  \lim_{n \to \infty} s_n = \lim_{n \to \infty} \frac{1-q^{n+1}}{1-q} =
  \frac{\lim_{n \to \infty} 1-q^{n+1}}{\lim_{n \to \infty} 1-q} = \frac{1}{1-q},
 \]
 když $|q| < 1$. Pro $|q| \geq 1$ plyne z předchozí úvahy, že $\lim_{n \to
 \infty} s_n$ neexistuje nebo je nekonečná. Tím je součet zadané řady spočten.
\end{probsol}

Nyní se jmeme odvodit několik základních tvrzení o existenci konečných součtů
číselných řad, která plynou v podstatě okamžitě z jejich definice jako limit
posloupností částečných součtů.

Asi není překvapivé, že aby součet řady mohl být konečný, musí posloupnost
jejích členů mít limitu $0$. Jinak bychom totiž sčítali nekonečně mnoho čísel s
absolutní hodnotou větší než nějaká konstanta, čímž bychom určitě nedostali
součet konečný.

\begin{lemma}{Nutná podmínka existence součtu
 řady}{nutna-podminka-existence-souctu-rady}
 Ať $\sum_{n=0}^{\infty} a_n$ je číselná řada, která má konečný součet. Pak
 nutně $\lim_{n \to \infty} a_n = 0$.
\end{lemma}
\begin{lemproof}
 Označme $s_n$ posloupnost částečných součtů $a_n$. Z předpokladu existuje
 $\lim_{n \to \infty} s_n$ a je konečná. Jelikož platí $\lim_{n \to \infty} s_n
 = \lim_{n \to \infty} s_{n-1}$, můžeme
 použitím~\hyperref[thm:aritmetika-limit]{věty o aritmetice limit} počítat
 \[
  \lim_{n \to \infty} (s_n - s_{n-1}) = \lim_{n \to \infty} s_n - \lim_{n \to
  \infty} s_{n-1} = 0.
 \]
 Zde je důležité zpozorovat, že výraz $\lim_{n \to \infty} s_n - \lim_{n \to
 \infty} s_{n-1}$ jest z konečnosti $\lim_{n \to \infty} s_n$ vždy definován.

 Snadno vidíme, že platí
 \[
  s_n - s_{n-1} = \sum_{k=0}^n a_k - \sum_{k=0}^{n-1} a_k = a_n.
 \]
 Pak ovšem
 \[
  0 = \lim_{n \to \infty} s_n - \lim_{n \to \infty} s_{n-1} = \lim_{n \to
  \infty} (s_n - s_{n-1}) = \lim_{n \to \infty} a_n,
 \]
 kterak bylo dokázati.
\end{lemproof}

Dalším přímo přenositelným pojmem z teorie limit posloupností je pojem
\emph{konvergence}. O číselné řadě řekneme, že \emph{konverguje}, když se
velikosti součtů členů mezi danými indexy neustále zmenšují. Formální definice
následuje.

\begin{definition}{Konvergence řady}{konvergence-rady}
 Díme, že číselná řada $\sum_{n=0}^{\infty} a_n$ \emph{konverguje}, když
 \[
 \forall \varepsilon > 0 \; \exists n_0 \in \N \; \forall m > n \geq n_0:\left|
 \sum_{k=n+1}^m a_k \right| < \varepsilon.
 \]
\end{definition}

Jak by jeden snad čekal, konvergence a existence konečného součtu řady jsou
víceméně záměnné.

\begin{proposition}{Vztah konvergence a existence
 součtu}{vztah-konvergence-a-existence-souctu}
 Číselná řada $\sum_{n=0}^{\infty} a_n$ má konečný součet právě tehdy, když
 konverguje.
\end{proposition}
\begin{propproof}
 Označme $s_n$ posloupnost částečných součtů $a_n$. Pak to, že
 $\sum_{n=0}^{\infty} a_n$ má konečný součet, znamená, že $s_n$ má konečnou
 limitu. Dále si všimněme, že
 \[
  \sum_{k=n+1}^m a_k = s_m - s_n,
 \]
 a tedy $\sum_{n=0}^{\infty} a_n$ konverguje právě tehdy, když posloupnost $s_n$
 konverguje. Tím jsme převedli původní tvrzení na to, že $s_n$ má konečnou
 limitu právě tehdy, když $s_n$ konverguje. Tuto ekvivalenci jsme dokázali v
 rámci~\myref{sekce}{sec:limity-konvergentnich-posloupnosti}.
\end{propproof}

\hyperref[prop:vztah-konvergence-a-existence-souctu]{Předchozí tvrzení} budeme v
dalším textu používat zcela bez varování a bez explicitního odkazu, zaměňujíce
ekvivalentní výroky \uv{Řada konverguje.} a \uv{Řada má konečný součet.}

Ježto řady jsou též posloupnosti, lze je pochopitelně sčítat, násobit a dělit
\uv{člen po členu}. Pro takto vzniklé řady platí podobná pravidla jako pro
obecné posloupnosti. Pro pořádek si je uvedeme.

\begin{proposition}{Aritmetika číselných řad}{aritmetika-ciselnych-rad}
 Ať $\sum_{n=0}^{\infty} a_n$ a $\sum_{n=0}^{\infty} b_n$ jsou číselné řady.
 \begin{enumerate}
  \item Je-li $c \in \R$ a $\sum_{n=0}^{\infty} a_n$ konverguje, pak
   $\sum_{n=0}^{\infty} c a_n$ konverguje a platí
   \[
    \sum_{n=0}^{\infty} ca_n = c \sum_{n=0}^{\infty} a_n.
   \]
  \item Konvergují-li $\sum_{n=0}^{\infty} a_n$ i $\sum_{n=0}^{\infty} b_n$, pak
   konverguje i $\sum_{n=0}^{\infty} (a_n + b_n)$ a platí
   \[
    \sum_{n=0}^{\infty} (a_n + b_n) = \sum_{n=0}^{\infty} a_n +
    \sum_{n=0}^{\infty} b_n.
   \]
  \item Konvergují-li $\sum_{n=0}^{\infty} a_n$ i $\sum_{n=0}^{\infty} b_n$, pak
   konverguje i $\sum_{n=0}^{\infty} (a_n \cdot b_n)$.
 \end{enumerate}
\end{proposition}

\begin{propproof}
 Označme $s_n,t_n$ posloupnosti částečných součtů $\sum_{n=0}^{\infty} a_n,
 \sum_{n=0}^{\infty} b_n$, respektive. Body (1) a (2) plynou přímo z
 \hyperref[thm:aritmetika-limit]{věty o aritmetice limit}. Vskutku, v bodě (1)
 máme
 \[
  c \cdot s_n = c \cdot \left( \sum_{k=0}^{n} a_k \right) = \sum_{k=0}^{n} c
  \cdot a_k,
 \]
 a tedy je $cs_n$ posloupností částečných součtů řady $\sum_{n=0}^{\infty}
 ca_n$. Pak tedy z \hyperref[thm:aritmetika-limit]{věty o aritmetice limit}
 \[
  \sum_{n=0}^{\infty} ca_n = \lim_{n \to \infty} cs_n = c \lim_{n \to \infty}
  s_n = c \sum_{n=0}^{\infty} a_n.
 \]
 Důkaz konvergence $c \sum_{n=0}^{\infty} a_n$ za předpokladu konvergence
 $\sum_{n=0}^{\infty} a_n$ plyne přímo z toho, že když existuje $\lim_{n \to
 \infty} s_n$, pak existuje i $c \lim_{n \to \infty} s_n$.

 V bodě (2) opět nahlédneme, že $s_n+t_n$ je posloupností částečných součtů
 $\sum_{n=0}^{\infty} (a_n + b_n)$. Pak tedy, opět z
 \hyperref[thm:aritmetika-limit]{věty o aritmetice limit}, dostaneme
 \[
  \sum_{n=0}^{\infty} (a_n+b_n) = \lim_{n \to \infty} (s_n + t_n) = \lim_{n \to
  \infty} s_n + \lim_{n \to \infty} t_n = \sum_{n=0}^{\infty} a_n +
  \sum_{n=0}^{\infty} b_n.
 \]

 V bodě (3) nestačí použít \hyperref[thm:aritmetika-limit]{větu o aritmetice
 limit}, neboť $s_n \cdot t_n$ \textbf{není} posloupností částečných součtů řady
 $\sum_{n=0}^{\infty} a_n \cdot b_n$. Zde použijeme
 \myref{lemma}{lem:nutna-podminka-existence-souctu-rady} a definici konvergence.
 Volme $\varepsilon \coloneqq 1$. K~němu podle tohoto lemmatu existuje $n_1 \in
 \N$ takové, že pro $n \geq n_1$ máme $|a_n| < 1$.

 Ať je nyní libovolné $\varepsilon>0$ dáno. Ježto $\sum_{n=0}^{\infty} a_n$ i
 $\sum_{n=0}^{\infty} b_n$ konvergují, existují $n_a, n_b \in \N$ taková, že pro
 $m > n \geq n_a$ platí $| \sum_{k=n+1}^{m} a_k | < \varepsilon$ a pro $m > n
 \geq n_b$ platí $| \sum_{k=n+1}^{m} b_k| < \varepsilon$. Volme nyní $n_0
 \coloneqq \max(n_1,n_a,n_b)$. Pak pro $m > n > n_0$ platí
 \[
  \left| \sum_{k=n+1}^{m} a_k b_k \right| \leq \left| \sum_{k=n+1}^{m} |a_k|b_k
  \right| < \left| \sum_{k=n+1}^{m} b_k \right| < \varepsilon,
 \]
 kde první nerovnost plyne z toho, že $a_k \leq |a_k|$ pro všechna $k \in \N$ a
 druhá z toho, že pro $k>n_1$ je $|a_k|<1$. Tedy $\sum_{n=0}^{\infty} a_nb_n$
 konverguje. 
\end{propproof}

\begin{warning}{}{soucin-rad}
 Bod (3) v \hyperref[prop:aritmetika-ciselnych-rad]{předchozím tvrzení} zaručuje
 pouze \textbf{existenci} konečného součtu řady $\sum_{n=0}^{\infty} a_nb_n$ za
 předpokladu existence konečného součtu řad $\sum_{n=0}^{\infty} a_n$ a
 $\sum_{n=0}^{\infty} b_n$, ale \textbf{netvrdí nic o jeho hodnotě}! Obecně na
 základě znalosti hodnot součtů obou řad nelze kromě konečnosti usoudit nic o
 hodnotě součtu řady $\sum_{n=0}^{\infty} a_nb_n$. Zcela jistě neplatí, že by
 tato hodnota byla součinem součtů řad $\sum_{n=0}^{\infty} a_n$ a
 $\sum_{n=0}^{\infty} b_n$.

 Uvažme například libovolnou geometrickou řadu $\sum_{n=0}^{\infty} q^{n}$ pro
 $|q| < 1$. Ta má podle \myref{úlohy}{prob:geometricka-rada} součet
 $\frac{1}{1-q}$. Pak tedy platí
 \[
  \left( \sum_{n=0}^{\infty} q^{n} \right) \cdot \left( \sum_{n=0}^{\infty} q^n
  \right) = \left( \frac{1}{1-q} \right)^2.
 \]
 Avšak, řada $\sum_{n=0}^{\infty} q^{n} \cdot q^{n} = \sum_{n=0}^{\infty}
 q^{2n}$ je rovněž geometrická s kvocientem $q^{2} < 1$, a tudíž její součet je
 \[
  \sum_{n=0}^{\infty} q^{2n} = \frac{1}{1-q^2} \neq \left( \frac{1}{1-q}
  \right)^2.
 \]
\end{warning}

\subsection{Řady s nezápornými členy}
\label{ssec:rady-s-nezapornymi-cleny}

V této sekci se budeme zabývat nejsnadněji zpytovaným typem číselných řad --
řadami, jejichž členy jsou pouze nezáporná čísla. Jejich zpyt je vesměs
jednoduchý z toho důvodu, že tyto řady vždycky mají součet, ať už konečný či
nekonečný. Existence záporných členů v číselné řadě totiž vyžaduje, aby jeden
analyzoval jemný vztah mezi její `kladnou' a `zápornou' částí a hodnotil, zda
obě v jistém smyslu `rostou stejně rychle', či nikolivěk.

Součty ani řad s nezápornými členy však není vůbec triviální určit a otázky
jejich konvergence jsou obyčejně řešeny srovnáními s řadami, jejichž součty
známy jsou. Nástrojem k tomu je následující vcelku přímočaré tvrzení.

\begin{proposition}{Srovnávací kritérium}{srovnavaci-kriterium}
 Ať $\sum_{n = 0}^{\infty} a_n, \sum_{n = 0}^{\infty} b_n$ jsou řady \textbf{s
 nezápornými členy}. Ať dále existuje $n_0 \in \N$ takové, že pro $n \geq n_0$
 platí $a_n \leq b_n$. Potom,
 \begin{enumerate}[label=(\alph*)]
  \item konverguje-li $\sum_{n = 0}^{\infty} b_n$, konverguje i $\sum_{n =
   0}^{\infty} a_n$;
  \item je-li $\sum_{n = 0}^{\infty} a_n = \infty$, pak i $\sum_{n = 0}^{\infty}
   b_n = \infty$.
 \end{enumerate}
\end{proposition}
\begin{propproof}
 Položme $s_n \coloneqq \sum_{i=0}^n a_i$ a $t_n \coloneqq \sum_{i=0}^n b_i$. Z
 předpokladu máme $n_0 \in \N$, od kterého dále již platí $a_n \leq b_n$. Pro
 důkaz (a) předpokládejme rovněž, že $\sum_{n = 0}^{\infty} b_n$ konverguje, tj.
 existuje konečná $\lim_{n \to \infty} t_n$.

 Ukážeme nejprve, že $s_n$ je shora omezená. Pro $n \geq n_0$ odhadujme
 \[
  s_n = s_{n_0} + \sum_{i=n_0+1}^n a_i \leq s_{n_0} + \sum_{i=n_0+1}^n b_i \leq
  s_{n_0} + \sum_{i=0}^{n} b_i = s_{n_0} + t_m,
 \]
 kde odhad $\sum_{i=n_0+1}^n b_i \leq \sum_{i=0}^n b_i$ platí díky nezápornosti
 členů $b_n$.

 Nyní, opět pro nezápornost $b_n$, je posloupnost částečných součtů $t_n$
 neklesající. Pročež pro všechna $k \in \N$ máme $t_k
 \leq \lim_{n \to \infty} t_n$. To nám umožňuje pro $n \geq n_0$ dokončit odhad
 \[
  s_n \leq s_{n_0} + t_n \leq s_{n_0} + \lim_{n \to \infty} t_n,
 \]
 který ukazuje, že $s_n$ je omezená. To ovšem zakončuje důkaz části (a), neboť
 $s_n$ je shora omezená neklesající (pro nezápornost $a_n$) posloupnost, a tedy
 má podle \myref{lemmatu}{lem:limita-monotonni-posloupnosti} limitu.

 Část (b) je pouze přepisem části (a) v kontrapozitivní formě, která zní, že
 nekonveruje-li $\sum_{n=0}^{\infty} a_n$, pak nekonverguje ani
 $\sum_{n=0}^{\infty} b_n$. Ovšem, divergentní řady s nezápornými členy mají
 součet $\infty$, odkud již přímo plyne závěr v (b).
\end{propproof}

Pochopitelně, srovnávací kritérium je užitečné pouze ve chvíli, kdy má jeden
\emph{s čím} srovnávat. Jmeme se odvodit divergenci jedné a konvergenci druhé z
takřkouce \uv{učebnicových} řad.

\begin{lemma}{Divergence harmonické řady}{divergence-harmonicke-rady} Číselná
 řada $\sum_{n=1}^{\infty} 1 / n$ diverguje, neboli $\sum_{n=1}^{\infty} 1 / n =
 \infty$.
\end{lemma}
\begin{lemproof}
 Použijeme \myref{tvrzení}{prop:vztah-konvergence-a-existence-souctu} a dokážeme
 negaci výroku o konvergenci $\sum_{n=1}^{\infty} 1 / n$. Konkrétně výrok
 \[
  \exists \varepsilon>0 \, \forall n_0 \in \N \, \exists m > n \geq n_0: \left|
  \sum_{i=n+1}^{m} \frac{1}{i} \right| \geq \varepsilon.
 \]
 
 Ukážeme, že $\varepsilon = 1 / 2$ vyhovuje výroku výše. Ať je $n_0 \in \N$
 dáno. Volme $n \coloneqq n_0$ a $m \coloneqq 2n_0$. Pak pro všechna $i \in \N,
 n_0 \leq i \leq 2n_0$ platí $1 / i \geq 1 / 2n_0$. Součet výše můžeme pročež
 zezdola odhadnout
 \[
  \left| \sum_{i=n_0+1}^{2n_0} \frac{1}{i} \right| = \sum_{i=n_0+1}^{2n_0}
  \frac{1}{i} \geq \sum_{i=n_0+1}^{2n_0} \frac{1}{2n_0} = n_0 \cdot
  \frac{1}{2n_0} = \frac{1}{2},
 \]
 kde první nerovnost plyne z faktu, že $1 / n > 0$ pro všechna $n \in \N$.

 Pro dané $n_0 \in \N$ tudíž platí
 \[
  \left| \sum_{i=n_0+1}^{2n_0} \frac{1}{i} \right| \geq \frac{1}{2} =
  \varepsilon,
 \]
 a tedy řada $\sum_{n=1}^{\infty} 1 / n$ diverguje. Protože jsou však její členy
 nezáporné, znamená toto, že $\sum_{n=1}^{\infty} 1 / n = \infty$, což bylo jest
 dokázati.
\end{lemproof}

\begin{remark}{}{harmonicka-rada}
 Řada $\sum_{n=1}^{\infty} 1 / n$ sluje \emph{harmonická}, protože je úzce
 spojena s pojmem \emph{alikvóty} a harmonie v~hud\-bě. Vlnové délky alikvót
 daného tónu (vlastně \uv{souznivých} tónů) jsou $1 / 2, 1 / 3, 1 / 4$ atd. jeho
 základní frekvence. Každý člen harmonické řady je \emph{harmonickým průměrem}
 svých sousedů, takže trojice členů v této řadě představuje vlnové délky tónů
 tvořících konsonantní akordy v~tónině dané původním frekvencí (jejíž alikvóty
 jsou vyjádřeny členy řady). Vizte např.
 \href{https://en.wikipedia.org/wiki/Harmonic_series_(mathematics)}{stránku na
 Wikipedii}.
\end{remark}

\begin{lemma}{}{divergence-n^2}
 Řada $\sum_{n=1}^{\infty} 1 / n^2$ konverguje.
\end{lemma}
\begin{lemproof}
 Srovnáme řadu $\sum_{n=1}^{\infty} 1 / n^2$ s řadou $\sum_{n=1}^{\infty} 1 /
 n(n+1)$ z \myref{úlohy}{prob:soucet-rady-pres-castecne-soucty}, jejíž součet
 je roven $1$. Protože obě řady mají nezáporné členy, lze použít
 \hyperref[prop:srovnavaci-kriterium]{srovnávací kritérium}.

 Indukcí dokážeme, že platí
 \[
  \frac{2}{n(n+1)} \geq \frac{1}{n^2} \quad \forall n \in \N.
 \]
 Pro $n = 1$ máme
 \[
  \frac{2}{1 \cdot (1 + 1)} = 1 \geq \frac{1}{1^2} = 1.
 \]
 Předpokládejme, že daná rovnost platí pro $n \in \N$. Počítáme
 \[
  \frac{2}{(n+1)(n+2)} = \frac{2}{n+1} \cdot \frac{1}{n+2} \geq \frac{1}{n}
  \cdot \frac{1}{n+2} \geq \frac{1}{(n+1)^2},
 \]
 kde první nerovnost plyne z indukčního předpokladu (po vynásobení obou stran
 číslem $n \in \N$) a poslední nerovnost plyne ze zřejmého vztahu
 \[
  n(n+2) = n^2 + 2n \leq n^2 + 2n + 1 = (n+1)^2.
 \]

 Nyní, z \hyperref[prop:aritmetika-ciselnych-rad]{aritmetiky řad} platí
 \[
  \sum_{n=1}^{\infty} \frac{2}{n(n+1)} = 2 \sum_{n=1}^{\infty} \frac{1}{n(n+1)}
  = 2
 \]
 a již jsme dokázali, že $2 / n(n+1) \geq 1 / n^2$ pro všechna $n \in \N$.
 \hyperref[prop:srovnavaci-kriterium]{Srovnávací kritérium} nyní dává
 \[
  \sum_{n=1}^{\infty} \frac{1}{n^2} \leq \sum_{n=1}^{\infty} \frac{2}{n(n+1)} =
  2,
 \]
 čili je řada $\sum_{n=1}^{\infty} 1 / n^2$ konvergentní.
\end{lemproof}

Na základě $\lim_{n \to \infty} a_n$ nelze obecně o konvergenci řady
$\sum_{n=0}^{\infty} a_n$ rozhodnout. Je-li limita nenulová, pak řada jistě
diverguje, ale je-li nulová, může řada konvergovat i divergovat. Máme-li ovšem
dvě řady, posloupnosti jejichž členů rostou v limitním smyslu \uv{stejně
rychle}, pak jsou i otázky jejich konvergencí ekvivalentní.

\begin{theorem}{Limitní srovnávací kritérium}{limitni-srovnavaci-kriterium}
 Ať $\sum_{n = 0}^{\infty} a_n, \sum_{n = 0}^{\infty} b_n$ jsou řady s
 nezápornými členy a označme $L \coloneqq \lim_{n \to \infty} a_n / b_n$.
 \begin{enumerate}[label=(\alph*)]
  \item Je-li $L \in (0,\infty)$, pak $\sum_{n = 0}^{\infty} a_n$ konverguje
   \textbf{právě tehdy, když} konverguje $\sum_{n = 0}^{\infty} b_n$.
  \item Je-li $L = 0$, pak konvergence $\sum_{n = 0}^{\infty} b_n$ implikuje
   konvergenci $\sum_{n = 0}^{\infty} a_n$.
  \item Je-li $L = \infty$, pak konvergence $\sum_{n = 0}^{\infty} a_n$
   implikuje konvergenci $\sum_{n = 0}^{\infty} b_n$.
 \end{enumerate}
\end{theorem}
\begin{thmproof}
 Před samotným důkazem je dlužno nahlédnout, že nutně $L \in [0,\infty]$, neboť
 posloupnosti $a_n$ i $b_n$ mají pouze nezáporné členy, tedy hodnoty $L$
 rozlišené výše jsou vskutku jediné možné.

 Započněme částí (a) a dokažme prve implikaci $( \Leftarrow )$. Předpokládejme,
 že $L \in (0,\infty)$ a $\sum_{n = 0}^{\infty} b_n$ konverguje. Volme
 $\varepsilon>0$ libovolně. Nalezneme $n_0 \in \N$ takové, že pro $n \geq n_0$
 platí
 \[
  \left| \frac{a_n}{b_n} - L \right| < \varepsilon.
 \]
 První nerovnost lze přepsat na
 \[
  L - \varepsilon < \frac{a_n}{b_n} < L + \varepsilon,
 \]
 čímž dostaneme horní odhad
 \[
  a_n < b_n(L + \varepsilon)
 \]
 pro všechna $n \geq n_0$. Z \hyperref[prop:aritmetika-ciselnych-rad]{aritmetiky
 řad} plyne, že řada $\sum_{n = 0}^{\infty} b_n(L + \varepsilon)$ je
 konvergentní (neboť $\sum_{n = 0}^{\infty} b_n$ je konvergentní) a ze
 \hyperref[prop:srovnavaci-kriterium]{srovnávacího kritéria} dostáváme (použitím
 odhadu výše), že i $\sum_{n = 0}^{\infty} a_n$ je konvergentní.

 K důkazu implikace $( \Rightarrow )$ je nám dán předpoklad $L \in (0,\infty)$ a
 konvergence $\sum_{n = 0}^{\infty} a_n$. Stejně jako v důkazu opačné implikace
 nalezneme ke zvolenému $\varepsilon>0$ číslo $n_0 \in \N$ takové, že pro $n
 \geq n_0$ platí 
 \[
  L - \varepsilon < \frac{a_n}{b_n} < L + \varepsilon.
 \]
 Číslo $\varepsilon$ zde volíme menší než $L$, aby platilo $L - \varepsilon >
 0$. Z dolního odhadu
 \[
  L - \varepsilon < \frac{a_n}{b_n}
 \]
 plyne úpravou
 \[
  \frac{b_n}{a_n} < \frac{1}{L-\varepsilon},
 \]
 čili
 \[
  b_n < \frac{a_n}{L-\varepsilon}
 \]
 pro $n \geq n_0$. Z \hyperref[prop:aritmetika-ciselnych-rad]{aritmetiky řad}
 opět platí, že řada $\sum_{n = 0}^{\infty} a_n / (L-\varepsilon)$ konverguje a
 \hyperref[prop:srovnavaci-kriterium]{srovnávací kritérium} skýtá kýženou
 konvergenci řady $\sum_{n = 0}^{\infty} b_n$.

 Dokážeme část (b). Ke zvolenému $\varepsilon>0$ nalezneme $n_0 \in \N$ takové,
 že pro $n \geq n_0$ platí
 \[
  \left| \frac{a_n}{b_n} - 0 \right| < \varepsilon.
 \]
 Pak ale máme odhad
 \[
  a_n < \varepsilon \cdot b_n
 \]
 a konvergence $\sum_{n = 0}^{\infty} a_n$ plyne z konvergence $\sum_{n =
 0}^{\infty} b_n$ zcela analogickým argumentem jako v~důkazu části (a).

 Důkaz části (c) je zcela obdobný důkazu části (b).
\end{thmproof}

Sekci o řadách s nezápornými členy završíme uvedením tří užitečných kritérií pro
zpyt konvergence takých řad, které o ní rozhodují přímo ze znalosti jejích
členů nevyžadujíce srovnání s řadami jinými. Výsledkem bude mimo jiné důkaz
konvergence jisté řady jsoucí mimořádně užitečnou pro srovnání s řadami, jejichž
členy jsou vyjádřeny jako podíly dvou polynomů.



