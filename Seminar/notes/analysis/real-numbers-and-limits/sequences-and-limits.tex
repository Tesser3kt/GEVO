\chapter{Posloupnosti, limity a reálná čísla}
\label{chap:posloupnosti-limity-a-realna-cisla}

Kritickým opěrným bodem při konstrukci reálných čísel i při jejich následném
studiu je pojem \emph{limity} (v češtině se tomuto slovu přiřazuje ženský rod).
Limita je bod, k němuž se zvolená posloupnost čísel \uv{blíží}, ale nikdy jeho
\uv{nedosáhne}, pokud takový existuje. Přidruženým pojmem je třeba
\emph{asymptota} reálné funkce, se kterou se čtenáři, očekáváme, setkali.

Samotná definice limity je zpočátku poněkud neintuitivní. Vlastně i samotná
představa býti něčemu \uv{nekonečně blízko} je do jisté míry cizí. Pokusíme se
vhodnými obrázky a vysvětlivkami cestu k pochopení dláždit, avšak, jakož tomu
bývá, intuice přichází, až člověk s ideou takřkouce sroste.

\section{Definice limity posloupnosti}
\label{sec:definice-limity-posloupnosti}

Koncept posloupnosti je, na rozdíl od limity, velmi triviální. Je to vlastně
\uv{očíslovaná množina čísel}. Z každé množiny lze vyrobit posloupnost jejích
prvků tím, že jim přiřkneme nějaké pořadí. Tento \emph{přírok} se nejsnadněji
definuje jako zobrazení z přirozených čísel -- to totiž přesně na každý prvek
kodomény zobrazí jeho pořadí.

\begin{definition}{Posloupnost}{posloupnost}
 Ať $X$ je množina. \emph{Posloupností} prvků z $X$ nazveme libovolné zobrazení
 \[
  a:\N \to X.
 \]
 Pro úsporu zápisu budeme psát $a_n$ místo $a(n)$ pro $n \in \N$. Navíc, je-li
 kodoména $X$ zřejmá z~kontextu, říkáme stručně, že $(a_n)_{n=0}^{\infty}$ je
 \emph{posloupnost.}
\end{definition}

\begin{remark}{}{usporadani-na-N}
 Vnímaví čtenáři sobě jistě povšimli, že jsme na $\N$ nedefinovali žádné
 \emph{uspořádání}. Ačkolivěk není tímto \hyperref[def:posloupnost]{definice
 posloupnosti} formálně nijak postižena, neodpovídá přirozenému vnímání, že
 prvek s číslem $1$ stojí před prvkem s číslem $5$ apod.

 Naštěstí, naše konstruktivní \hyperref[def:prirozena-cisla]{definice
 přirozených čísel} nabízí okamžité řešení. Využijeme toho, že každé přirozené
 číslo je podmnožinou svého následníka, a definujeme zkrátka uspořádání $ \leq $
 na $\N$ předpisem
 \[
  a \leq b \overset{\text{def}}{\iff} a \subseteq b.
 \]
 Fakt, že $ \subseteq $ je uspořádání, okamžitě implikuje, že $ \leq $ je rovněž
 uspořádání.
\end{remark}

Rozmyslíme si nyní dva pojmy pevně spjaté s posloupnostmi -- \emph{konvergence}
a \emph{limita}. Brzo si též ukážeme, že tyto dva pojmy jsou záměnné, ale zatím
je vnímáme odděleně. Navíc, budeme se odteď soustředit speciálně na posloupnosti
racionálních čísel, tj. zobrazení $\N \to \Q$, neboť jsou oním klíčem k
sestrojení své reálné bratří.

Ze všech posloupností $\N \to \Q$ nás zajímá jeden konkrétní typ --
posloupnosti, vzdálenosti mezi jejichž prvky se postupně zmenšují. Tyto
posloupnosti, nazývané \emph{konvergentní} (z lat. con-vergere, \uv{ohýbat k
sobě}), se totiž vždy blíží k nějakému konkrétnímu bodu -- ke své \emph{limitě}.
Představa ze života může být například následující: říct, že se blížíme k
nějakému místu, je totéž, co tvrdit, že se vzdálenost mezi námi a oním místem s
každým dalším krokem zmenšuje. V moment, kdy své kroky směřujeme stále stejným
směrem, posloupnost vzdáleností mezi námi a tím místem tvoří konvergentní
posloupnost. Jestliže se pravidelně odkláníme, k místu nikdy nedorazíme a
posloupnost vzdáleností je pak \emph{divergentní} (tj.
\textbf{ne}konvergentní).

Do jazyka matematiky se věta \uv{vzdálenosti postupně zmenšují} překládá
obtížně. Jeden ne příliš elegantní, ale výpočetně užitečný a celkově oblíbený
způsob je následující: řekneme, že prvky posloupnosti jsou k sobě stále blíž,
když pro jakoukoli vzdálenost vždy dokážeme najít krok, od kterého dál jsou již
k sobě dva libovolné prvky u sebe blíž než tato daná vzdálenost. Důrazně
vyzýváme čtenáře, aby předchozí větu přečítali tak dlouho, dokud jim nedává
dobrý smysl. Podobné formulace se totiž vinou matematickou analýzou a jsou
základem uvažování o nekonečnu.

\begin{definition}{Konvergentní posloupnost}{konvergentni-posloupnost}
 Řekneme, že posloupnost $a:\N \to \Q$ je \emph{konvergentní}, když platí výrok
 \[
  \forall \varepsilon \in \Q, \varepsilon>0 \; \exists n_0 \in \N \; \forall m,n
  \geq n_0: |a_m - a_n| < \varepsilon.
 \]
\end{definition}

\begin{figure}[ht]
 \centering
 \begin{tikzpicture}
  \tkzInit[xmin=0,xmax=10,ymin=0,ymax=3]
  \foreach \n in {0,1,...,18} {
   \tkzDefPoint(0.5 * (\n + 1),0){x\n}
   \tkzDrawPoint[shape=cross](x\n)
   \tkzLabelPoint[below,color=RoyalBlue](x\n){$\n$}
  }
  \foreach \y in {1,2,3,4,5,6} {
   \tkzDefPoint(0,0.5 * \y){y\y}
   \tkzDrawPoint[shape=cross](y\y)
   \tkzLabelPoint[left,color=ForestGreen](y\y){$\y$}
  }
  \tkzDefPoint(0.5,2.8){a0}
  \tkzDefPoint(1,0.9){a1}
  \tkzDefPoint(1.5,2.3){a2}
  \tkzDefPoint(2,1){a3}
  \tkzDefPoint(2.5,1.1){a4}
  \tkzDefPoint(3,0.3){a5}
  \tkzDefPoint(3.5,2.4){a6}
  \tkzDefPoint(4,1.2){a7}
  \tkzDefPoint(4.5,3.2){a8}
  \tkzDefPoint(5,2.6){a9}
  \tkzDefPoint(5.5,1.9){a10}
  \tkzDefPoint(6,2.4){a11}
  \tkzDefPoint(6.5,2.2){a12}
  \tkzDefPoint(7,1.9){a13}
  \tkzDefPoint(7.5,2){a14}
  \tkzDefPoint(8,1.8){a15}
  \tkzDefPoint(8.5,2.05){a16}
  \tkzDefPoint(9,1.95){a17}
  \tkzDefPoint(9.5,1.85){a18}

  \tkzDefPoint(0,1.9){O}
  \tkzDefPoint(0,2.1){O1}
  \tkzDefPoint(0,1.7){O2}
  \tkzDefPoint(10.5,2.1){O3}
  \tkzDefPoint(10.5,1.7){O4}
  \tkzDrawPolygon[draw=none,fill=Fuchsia!10](O1,O3,O4,O2)
  \tkzDrawX[>=latex,label={$\clb{n}$}]
  \tkzDrawY[>=latex,label={$\clg{a_n}$}]

  \tkzDrawLine[add=0 and 0.5,dashed,color=Fuchsia](O,a13)
  \tkzDrawLine[add=0 and 0,color=Fuchsia](O1,O3)
  \tkzDrawLine[add=0 and 0,color=Fuchsia](O2,O4)
  \tkzDrawPoints[color=ForestGreen](a0,a1,a2,a3,a4,a5,a6,a7,a8,a9,a10,a11,a12,a13,a14,a15,a16,a17,a18)

  \draw [decorate,decoration={brace,amplitude=5pt},color=Fuchsia]
   (O3) -- (O4) node [color=Fuchsia,midway,xshift=12pt] {$2\varepsilon$};
  \tkzDefPoint(7,0.5){n0}
  \tkzLabelPoint[below,color=BrickRed](n0){$n_0$}
  \tkzDrawSegment[dashed,color=BrickRed](n0,a13)
 \end{tikzpicture}

 \caption{Konvergentní posloupnost. Zde pro $\clm{\varepsilon} = 0.1$ lze volit
  například $\clr{n_0} = 13$. Vodorovná přímka procházející bodem
  $a_{\clr{n_0}}$ je vlastně \uv{středem} pruhu o šíři $\clm{2\varepsilon}$, ve
  kterém se nacházejí všechny členy posloupnosti s pořadím vyšším než $13$.}
 \label{fig:konvergentni-posloupnost}
\end{figure}

\begin{remark}{}{absolutni-hodnota}
 Radíme, aby se čtenáři sžili s intuitivním (přesto velmi přesným) ponětím
 absolutní hodnoty $|x - y|$ jako \emph{vzdálenosti} mezi čísly $x$ a $y$. V
 tomto smyslu je pak $|x| = |x - 0|$ vzdálenost čísla $x$ od čísla $0$, což
 cele odpovídá definici tohoto symbolu.
\end{remark}

\begin{remark}{}{konvergence}
 Aplikujeme intuitivní vysvětlení \emph{zmenšování vzdálenosti} z odstavce nad
 \myref{definicí}{def:konvergentni-posloupnost} na jeho skutečnou definici.

 Výrok
 \[
  \forall \varepsilon \in \Q, \varepsilon>0 \; \exists n_0 \in \N \; \forall m,n
  \geq n_0: |a_m - a_n| < \varepsilon
 \]
 říká, že pro jakoukoli vzdálenost ($\varepsilon$) dokáži najít krok ($n_0$)
 takový, že vzdálenost dvou prvků v~libovolných dvou následujících krocích
 ($m,n$) už je menší než daná vzdálenost ($|a_n -a_m|<\varepsilon$).

 Slovo \uv{krok} je třeba vnímat volně -- myslíme pochopitelně \emph{pořadí} či
 \emph{indexy} prvků v posloupnosti. Pohled na racionální posloupnosti jako na
 \uv{kroky} činěné v racionálních číslech může být ovšem užitečný.
\end{remark}

\begin{exercise}{}{}
 Dokažte, že posloupnost $a:\N \to \Q$ je konvergentní právě tehdy, když
 \[
  \forall \varepsilon>0 \; \exists n_0 \in \N \; \forall m,n \geq n_0: |a_m -
  a_n| < C\varepsilon
 \]
 pro libovolnou \textbf{kladnou} konstantu $C \in \Q$.
\end{exercise}

Pojem \emph{limity}, představuje jakýsi bod, k němuž se posloupnost s každým
dalším krokem přibližuje, je vyjádřen výrazem podobného charakteru. Zde však
přichází na řadu ona \emph{děravost} racionálních čísel. Může se totiž stát, a
příklady zde uvedeme, že limita racionální posloupnosti není racionální číslo.

Učiňmež tedy dočasný obchvat a před samotnou definicí limity vyrobme reálná
čísla jednou z~přehoušlí možných cest.

Ať $\mathcal{C}(\Q)$ značí množinu všech \textbf{konvergentních }racionálních
posloupností. Uvažme ekvivalenci $ \simeq $ na $\mathcal{C}(\Q)$ danou
\[
 a \simeq b \overset{\text{def}}{\iff} \forall \varepsilon>0 \; \exists n_0 \in \N
 \; \forall n \geq n_0: |a_n - b_n| < \varepsilon.
\]
Přeloženo do člověčtiny, $a \simeq b$, právě když se rozdíl mezi prvky těchto
posloupností se stejným pořadím neustále zmenšuje -- řekli bychom, že se
\emph{blíží k nule}. V rámci naší (zatím intuitivní) představy, že konvergentní
posloupnosti se blíží k nějakému bodu, dává smysl ztotožňovat posloupnosti,
které se blíží k bodu \emph{stejnému} -- stav, který vyjadřujeme tak, že se
jejich rozdíl blíží k nule.

Ve výsledku budeme definovat reálná čísla jako limity všech možných
konvergentních racionálních posloupností. Ježto však pozbýváme aparátu, abychom
koncepty limity a konvergence stmelili v~jeden, jsme nuceni učinit mezikrok.

\begin{definition}{Reálná čísla}{realna-cisla}
 Množinu \emph{reálných čísel} tvoří všechny třídy ekvivalence konvergentních
 racionálních posloupností podle $ \simeq $. Symbolicky,
 \[
  \R \coloneqq \{[a]_{ \simeq } \mid a \in \mathcal{C}(\Q)\}.
 \]
\end{definition}

Nyní definujeme pojem limity. Nemělo by snad být příliš překvapivé, že se od
\hyperref[def:konvergentni-posloupnost]{definice konvergence} příliš neliší.
Významný rozdíl odpočívá pouze v předpokladu existence \emph{cílového bodu}.

\begin{definition}{Limita posloupnosti}{limita-posloupnosti}
 Ať $a:\N \to \Q$ je posloupnost. Řekneme, že $a$ \emph{má limitu} $L \in \R$,
 když
 \[
  \forall \varepsilon \in \Q,\varepsilon>0 \; \exists n_0 \in \N \; \forall n
  \geq n_0: |a_n - L|<\varepsilon,
 \]
 neboli, když jsou prvky $a_n$ bodu $L$ s každým krokem stále blíž.

 Fakt, že $L \in \R$ je limitou $a$ značíme jako $\lim_{} a = L$.
\end{definition}

\section{Limity konvergentních posloupností}
\label{sec:limity-konvergentnich-posloupnosti}

V této sekci dokážeme, že konvergentní posloupnosti mají limitu. Opačná
implikace, tj. že posloupnosti jmajíce limitu konvergují, je téměř triviální. K
jejímu důkazu potřebujeme jen jednu vlastnost absolutní hodnoty.

\begin{lemma}{Trojúhelníková nerovnost}{trojuhelnikova-nerovnost}
 Ať $x,y \in \Q$. Pak
 \[
  |x + y| \leq |x| + |y|.
 \]
\end{lemma}
\begin{lemproof}
 Absolutní hodnota $|x+y|$ je rovna buď $x + y$ (když $x+y \geq 0$) nebo $-x-y$
 (když $x+y<0$). Zřejmě $x \leq |x|$ a $-x \leq |x|$, podobně $y \leq |y|$ a
 $-y \leq |y|$.

 Pak je ale $x + y \leq |x| + |y|$ a též $-x+(-y) \leq |x| + |y|$. Tím je důkaz
 hotov.
\end{lemproof}

Trojúhelníková nerovnost poskytuje snadné důkazy mnoha užitečných dílčích
tvrzení o posloupnostech. Příkladem je následující cvičení.

\begin{exercise}{Jednoznačnost limity}{jednoznacnost-limity}
 Dokažte, že každá posloupnost $a:\N \to \Q$ má nejvýše jednu limitu. Hint:
 použijte \hyperref[lem:trojuhelnikova-nerovnost]{trojúhelníkovou nerovnost}.
\end{exercise}

Ježto bychom však rádi dokazovali všechna tvrzení již pro reálná čísla, ukažme
si nejprve, jak se dají sčítat a násobit. Dokážeme rovněž, že $\R$ -- stejně
jako $\Q$ -- tvoří těleso. Začneme tím, že se naučíme sčítat a násobit
konvergentní posloupnosti.

Ať $a,b \in \mathcal{C}(\Q)$ jsou dvě konvergentní racionální posloupností.
Operace $+$ a $ \cdot $ na $\mathcal{C}(\Q)$ definujeme velmi přirozeně.
Zkrátka, $(a+b)(n) \coloneqq a(n) + b(n)$ a $(a \cdot b)(n) \coloneqq a(n) \cdot
b(n)$, tj. prvek na místě $n$ posloupnosti $a+b$ je součet prvků na místech $n$
posloupností $a$ a $b$. Abychom ovšem získali skutečně operace na
$\mathcal{C}(\Q)$, musíme ověřit, že $a+b$ i $a \cdot b$ jsou konvergentní.

Nechť dáno jest $\varepsilon>0$. Chceme ukázat, že umíme najít $n_0 \in \N$, aby
\[
 |(a_n + b_n) - (a_m + b_m)| < \varepsilon,
\]
kdykoli $m,n \geq n_0$. Protože jak $a$ tak $b$ konverguje, již umíme pro
libovolná $\varepsilon_a,\varepsilon_b>0$ najít $n_a$ a $n_b$ taková, že $|a_n -
a_m| < \varepsilon_a$, kdykoli $m,n \geq n_a$, a podobně $|b_n -
b_m|<\varepsilon_b$, kdykoli $m,n \geq n_b$. Položme tedy $\varepsilon_a =
\varepsilon_b \coloneqq \varepsilon / 2$ a $n_0 \coloneqq \max(n_a,n_b)$. Potom
můžeme užitím \hyperref[lem:trojuhelnikova-nerovnost]{trojúhelníkové nerovnosti}
pro $m,n \geq n_0$ odhadnout
\[
 |(a_n + b_n) - (a_m + b_m)| = |(a_n - a_m) + (b_n - b_m)| \leq |a_n - a_m| +
 |b_n - b_m| < \varepsilon_a + \varepsilon_b = \varepsilon,
\]
čili $a + b$ konverguje.

Předchozí odstavec se může snadno zdát šílenou směsicí symbolů. Ve skutečnosti
však formálně vykládá triviální úvahu. Máme najít pořadí, od kterého jsou prvky
součtu $a + b$ u sebe blíž než nějaká daná vzdálenost. Poněvadž $a$ i $b$
konvergují, stačí přeci vzít větší z pořadí, od kterých je jak rozdíl prvků $a$,
tak rozdíl prvků $b$, menší než polovina dané vzdálenosti.

Velmi obdobnou manipulaci lze provést k důkazu konvergence $a \cdot b$.
Ponecháváme jej čtenářům jako (ne zcela snadné) cvičení.

\begin{exercise}{}{}
 Dokažte, že jsou-li $a,b$ konvergentní posloupnosti racionálních čísel, pak je
 posloupnost $a \cdot b$ rovněž konvergentní.
\end{exercise}

Racionální čísla jsou přirozeně součástí reálných prostřednictvím zobrazení
\begin{equation}
 \label{eq:Q-into-R}
 \begin{split}
  \xi: \Q &\hookrightarrow \R,\\
  q &\mapsto [(q)],
 \end{split}
\end{equation}
kde $(q)$ značí posloupnost $a: n \mapsto q$ pro všechna $n \in \N$ a $[(q)]$
její třídu ekvivalence podle $ \simeq $.

\begin{warning}{}{}
 Tvrdíme pouze, že $\Q$ jsou \emph{součástí} $\R$, kde slovu \emph{součást}
 záměrně není dán rigorózní smysl. Racionální čísla totiž (aspoň po dobu naší
 dočasné \hyperref[def:realna-cisla]{definice reálných čísel}) nejsou v žádném
 smyslu podmnožinou čísel reálných.

 Matematici ale často ztotožňujeme doménu prostého zobrazení s jeho obrazem
 (neboť mezi těmito množinami vždy existuje bijekce). V tomto smyslu mohou být
 $\Q$ vnímána jako podmnožina $\R$, ztotožníme-li racionální čísla s obrazem
 zobrazení $\xi$ z \eqref{eq:Q-into-R}. Toto ztotožnění znamená vnímat
 racionální číslo $q \in \Q$ jako konvergentní posloupnost samých čísel $q$.
\end{warning}

\begin{exercise}{}{}
 Dokažte, že zobrazení $\xi$ z \eqref{eq:Q-into-R} je
 \begin{itemize}
  \item dobře definované -- tzn. že když $p = q$, pak $[(p)] = [(q)]$ -- a
  \item prosté.
 \end{itemize}
\end{exercise}

Jelikož $\Q$ je těleso, speciálně tedy obsahuje $0$ a $1$, $\R$ je
(prostřednictvím $\xi$ z \eqref{eq:Q-into-R}) obsahuje rovněž. Pro stručnost
budeme číslem $0 \in \R$ značit třídu ekvivalence posloupnosti samých nul a
číslem $1 \in \R$ třídu ekvivalence posloupnosti samých jednotek. Ověříme, že se
skutečně jedná o neutrální prvky ke sčítání a násobení.

Je třeba si rozmyslet, že pro každou posloupnost $a \in \mathcal{C}(\Q)$ platí
$a + 0 = a$ a $a \cdot 1 = a$, kde, opět, čísla $0$ a $1$ ve skutečnosti
znamenají nekonečné posloupnosti těchto čísel. Obě rovnosti jsou však zřejmé z
definice, neboť $(a+0)(n) = a_n + 0 = a_n = a(n)$ a $(a \cdot 1)(n) = a_n \cdot
1 = a_n = a(n)$ pro všechna $n \in \N$.

Konečně, rozšíříme rovněž $-$ a $^{-1}$ na $\R$. Pro libovolnou posloupnost $x
\in \mathcal{C}(\Q)$ definujeme zkrátka $(-a)(n) \coloneqq -a(n)$. S $^{-1}$ je
situace lehce komplikovanější. Totiž, pouze \textbf{nenulová} racionální čísla
mají svůj inverz k násobení. Zde je třeba zpozorovat, že \textbf{konvergentní}
posloupnost, která by však měla nekonečně mnoho prvků nulových, už musí mít od
nějakého kroku \textbf{všechny} prvky nulové, jinak by totiž nemohla
konvergovat. Vskutku, představme si, že $a$ je posloupnost taková, že $a_n = 0$
pro nekonečně mnoho přirozených čísel $n \in \N$. Pak ale ať zvolím $n_0 \in \N$
jakkoliv, vždy existuje $m \geq n_0$ takové, že $a_m = 0$. Vezměme $n \geq n_0$
libovolné. Pokud $a_n \neq 0$, pak můžeme vzít třeba $\varepsilon \coloneqq
|a_n| / 2$ a bude platit, že $|a_n - a_m| > \varepsilon$, což je dokonalý zápor
\hyperref[def:konvergentni-posloupnost]{definice konvergence}. Z toho plyne, že
$a_n$ musí být $0$ pro $n \geq n_0$ a odtud dále, že $a \simeq 0$. Čili, pouze
nulové posloupnosti v $\R$ nemají inverz vzhledem k $ \cdot $.

Právě provedená úvaha nám umožňuje definovat $^{-1}$ pro posloupnosti $a \in
\mathcal{C}(\Q)$ takové, že $a \not\simeq 0$, následovně:
\[
 (a^{-1})(n) \coloneqq \begin{cases}
  a(n)^{-1},& \text{když } a(n) \neq 0,\\
  0, &\text{když } a(n) = 0.
 \end{cases}
\]
 
Je snadné uvidět, že $-a$ je inverzem k $a$ vzhledem k $+$ a $a^{-1}$ je
inverzem k $a \neq 0$ vzhledem k $ \cdot $. Vskutku, máme
\[
 (a + (-a))(n) = a_n + (-a_n) = 0,
\]
tedy v tomto případě je $(a + (-a))$ přímo \textbf{rovna} nulové posloupnosti. V
případě $^{-1}$ dostáváme pro $a \not\simeq 0$
\[
 (a \cdot a^{-1})(n) = \begin{cases}
  a_n \cdot a_n^{-1} = 1,& \text{když } a_n \neq 0,\\
  a_n \cdot 0 = 0,& \text{když } a_n = 0.
 \end{cases}
\]
Ergo, $a \cdot a^{-1}$ je rovna posloupnosti samých jedniček až na konečně mnoho
nul, protože, jak jsme si již rozmysleli, $a$ nemůže mít nekonečně $0$ a zároveň
nebýt v relaci $ \simeq $ s nulovou posloupností, jinak by nebyla konvergentní.

Shrneme-li řád předchozích úvah, získáme oprávnění tvrdit, že
\[
 (\R,+,-,[(0)], \cdot,^{-1},[(1)])
\]
je těleso. Tento fakt je do budoucna pochopitelně zásadní; teď se však můžeme
těšit znalostí, že jsme přechodem od $\Q$ k $\R$ neztratili symetrické rysy
původní množiny.

Přikročmež již však k důkazu existence limity každé konvergentní posloupnosti.
Fakt, že existence limity implikuje konvergenci, plyne přímo z
\hyperref[lem:trojuhelnikova-nerovnost]{trojúhelníkové nerovnosti}.

\begin{lemma}{}{}
 Každá posloupnost majíc limitu je konvergentní.
\end{lemma}
\begin{lemproof}
 Ať $a:\N \to \Q$ je posloupnost s limitou $L$. Pak pro každé $\varepsilon_L>0$
 existuje $n_L \in \N$ takové, že $|a_n - L| < \varepsilon_L$ pro všechna $n
 \geq n_L$.

 Ať je dáno $\varepsilon>0$. Chceme ukázat, že $|a_m - a_n| < \varepsilon$ pro
 všechna $m,n$ větší než vhodné $n_0 \in \N$. Položme tedy $n_0 \coloneqq n_L$ a
 $\varepsilon_L \coloneqq \varepsilon / 2$. Potom pro všechna $m,n \geq n_0 =
 n_L$ máme
 \[
  |a_m - a_n| = |a_m - a_n - L + L| = |(a_n - L) + (L - a_m)| \leq |a_n - L| +
  |L - a_m| < \varepsilon_L + \varepsilon_L = \varepsilon,
 \]
 čili $a$ konverguje.
\end{lemproof}

\subsection{Úplnost reálných čísel}
\label{ssec:uplnost-realnych-cisel}

K důkazu existence limity každé konvergentní posloupnosti potřebujeme
prozpytovat vztah racionálních a reálných čísel podrobněji. Konkrétně
potřebujeme ukázat, že $\Q$ jsou tzv. \emph{hustá} v $\R$, tj. že ke každému
reálnému číslu existuje racionální číslo, které je mu nekonečně blízko. Zde jsme
opět implicitně ztotožnili racionální čísla s třídami ekvivalence konstantních
posloupností. Na základě toho budeme totiž moci tvrdit, že reálná čísla jsou
tzv. \emph{kompletní}, což přesně znamená, že každá konvergentní posloupnost
reálných čísel má reálnou limitu.

Nejprve si ovšem musíme rozmyslet, co vlastně míníme posloupností
\emph{reálných} čísel. Pochopitelně, zobrazení $x:\N \to \R$ poskytuje validní
definici, ale uvědomme si, že teď vlastně uvažujeme posloupnosti, jejichž prvky
jsou třídy ekvivalence konvergentních racionálních posloupností.

Abychom směli hovořit o konvergentních \emph{reálných} posloupnostech, rozšíříme
absolutní hodnotu $| \cdot |$ z $\Q$ na $\R$ zkrátka předpisem $|[(x_n)]|
\coloneqq [(|x_n|)]$ pro $(x_n) \in \mathcal{C}(\Q)$. Napíšeme-li tedy $|x| \leq
K$ pro reálná čísla $x,K \in \R$, pak tím doslova myslíme $[(|x_n|)] \leq
[(K_n)]$, což ale \textbf{neznamená} $|x_n| \leq K_n$ pro všechna $n \in \N$,
kde $x_n,K_n$ jsou nyní již čísla ryze rozumná čili racionální, anobrž $|x_n| >
K_n$ jen pro \textbf{konečně mnoho} $n \in \N$.

\begin{warning}{}{}
 Důležitá myšlenka, již je dlužno snovat v srdci při práci s třídami ekvivalence
 konvergentních posloupností, je ta, že při porovnávání dvou tříd nás nezajímá
 libovolný \textbf{konečný počet} jejich prvních prvků.

 Konkrétně, vztah $x = y$ pro $x,y \in \R$ znamená, že $x_n = y_n$ pro každé $n
 \in \N$ až na libovolný konečný počet prvních přirozených čísel. To se lépe
 vyjadřuje pomocí negace. Je snazší říct, že $x = y$, když $x_n \neq y_n$ pro
 jenom konečně mnoho $n \in \N$.
\end{warning}

Rozepíšeme-li si tedy podrobně, co znamená, že je posloupnost $x:\N \to \R$
konvergentní, dostaneme pro dané $\varepsilon>0$, vhodné $n_0 \in \N$ a $m,n
\geq n_0$ nerovnost $|x_n - x_m| < \varepsilon$. Ovšem, $x_n$ i $x_m$ jsou samy
o sobě třídy ekvivalence konvergentních \textbf{posloupností} racionálních
čísel, tedy poslední nerovnost plně rozepsána dí
\[
 |[((x_n)_{k} - (x_m)_{k})_{k=0}^{\infty}]| < \varepsilon \; \forall k \in \N,
\]
což lze rovněž vyjádřit tak, že
\[
 |(x_n)_k - (x_m)_k| \geq \varepsilon
\]
jen pro konečně mnoho $k \in \N$.

Nepřináší však žádný hmotný užitek nad konvergencí reálných posloupností
uvažovat takto složitě. Čtenáři dobře učiní, uvědomí-li si plný význam
předchozího odstavce, ovšem zůstanou věrni intuitivnímu vnímání výrazu $|x - y|$
jako \uv{vzdálenosti} čísel $x$ a $y$.

\begin{definition}{Omezená posloupnost}{omezena-posloupnost}
 Řekneme, že posloupnost $x:\N \to \R$ je \emph{omezená}, když existuje $K \in
 \R$ takové, že $|x_n| \leq K$ pro všechna $n \in \N$.
\end{definition}

\begin{lemma}{}{}
 Každá konvergentní posloupnost $x:\N \to \R$ je omezená.
\end{lemma}
\begin{lemproof}
 Ať je $\varepsilon>0$ dáno. Z \hyperref[def:konvergentni-posloupnost]{definice
 konvergence} nalezneme $n_0 \in \N$ takové, že pro každé $m,n \geq n_0$ je
 $|x_m - x_n| < \varepsilon$. Speciálně tedy pro každé $n \geq n_0$ platí
 \[
  |x_n| = |x_n - x_{n_0} + x_{n_0}| \leq |x_n - x_{n_0}| + |x_{n_0}| <
  \varepsilon + |x_{n_0}|,
 \]
 tudíž všechny členy posloupnosti s pořadím větším než $n_0$ jsou omezeny číslem
 $\varepsilon + |x_{n_0}|$. Ovšem, členů posloupnosti s pořadím menším než $n_0$
 je konečně mnoho, a tedy z nich můžeme vzít ten největší -- nazvěme ho $s$.
 Položíme-li $K \coloneqq \max(s,\varepsilon + |x_{n_0}|)$, pak $|x_n| \leq K$
 pro každé $n \in \N$, čili $x$ je omezená číslem $K$.
\end{lemproof}

\begin{proposition}{Hustota $\Q$ v $\R$}{hustota-q-v-r}
 Množina racionálních čísel $\Q$ je hustá v $\R$, tj. ke každému $x \in \R$ a
 každému $\varepsilon>0$ existuje $r \in \Q$ takové, že $|x-r|<\varepsilon$.
\end{proposition}
\begin{propproof}
 Ať $\varepsilon>0$ je dáno a označme $x \coloneqq [(x_n)]$, $(x_n) \in
 \mathcal{C}(\Q)$. Najdeme $n_0 \in \N$ takové, že $ \forall m,n \geq n_0$ je
 $|x_m - x_n|<\varepsilon$. Zvolme $r \coloneqq x_{n_0} \in \Q$. Pak ovšem máme
 \[
  |x_n - r| = |x_n - x_{n_0}| < \varepsilon
 \]
 pro všechna $n \geq n_0$. To přesně znamená, že $|x - r| < \varepsilon$.
\end{propproof}

\begin{lemma}{}{limita-racionalni-posloupnosti}
 Ať $a:\N \to \Q$ je konvergentní posloupnost racionálních čísel. Pak $\lim a =
 [(a)]$.
\end{lemma}
\begin{lemproof}
 Položme $x \coloneqq [(a)]$. Ať je dáno $\varepsilon>0$. Protože $a$ je
 konvergentní, nalezneme $n_0 \in \N$, že $|a_m - a_n|<\varepsilon$ pro všechna
 $m,n \geq n_0$. Potom ale $|a_n - x|<\varepsilon$ pro všechna $n \geq n_0$, což
 z~\hyperref[def:limita-posloupnosti]{definice} znamená, že $\lim_{} a = x$.
\end{lemproof}

\begin{corollary}{$\R$ jsou úplná}{r-jsou-uplna}
 Každá konvergentní reálná posloupnost $x:\N \to \R$ má limitu v $\R$.
\end{corollary}
\begin{corproof}
 Ať $a:\N \to \Q$ je racionální posloupnost taková, že $|x_n - a_n|<1/n$ pro
 všechna $n \in \N$. Tu nalezneme opakovaným použitím
 \myref{tvrzení}{prop:hustota-q-v-r} pro $\varepsilon \coloneqq 1 / n$ a $x
 \coloneqq x_n$. Ukážeme nejprve, že $a$ je konvergentní. Ať je dáno
 $\varepsilon>0$. Zvolme $n_1$ takové, že $ \forall m,n \geq n_1$ platí $1 / m +
 1 / n < \varepsilon$. Dále, $x$ je konvergentní z předpokladu. Čili, pro každé
 $\varepsilon_x>0$ nalezneme $n_2 \in \N$ takové, že $ \forall m,n \geq n_2$
 máme $|x_n - x_m|<\varepsilon_x$. Volme tedy speciálně
 \[
  \varepsilon_x \coloneqq \varepsilon - \frac{1}{m} - \frac{1}{n}.
 \]
 a $n_0 \coloneqq \max(n_1,n_2)$. Potom pro všechna $m,n \geq n_0$ platí
 nerovnosti
 \begin{align*}
  |a_n - a_m| &= |a_n - a_m - x_n + x_n| \leq |a_n - x_n| + |x_n - a_m| = |a_n -
  x_n| + |x_n - a_m - x_m + x_m| \\
              & \leq |a_n - x_n| + |x_n - x_m| + |x_m - a_m| < \frac{1}{n} +
              \varepsilon_x + \frac{1}{m} = \varepsilon,
 \end{align*}
 tedy $a$ konverguje.

 Jistě platí $\lim_{} x - a = 0$, neboť pro každé $\varepsilon>0$ lze najít
 $n \in \N$ takové, že $1 / n < \varepsilon$. Odtud plyne, že $x$ má limitu
 právě tehdy, když $a$ má limitu. Ovšem, podle
 \myref{lemmatu}{lem:limita-racionalni-posloupnosti} má $a$ limitu $[(a)] \in
 \R$. Tím je důkaz hotov.
\end{corproof}

\begin{corollary}{}{}
 Platí
 \[
  \R \cong \{\lim_{} a \mid a \in \mathcal{C}(\Q)\},
 \]
 čili reálná čísla jsou přesně limity všech konvergentních racionálních
 posloupností.
\end{corollary}
\begin{corproof}
 Zkonstruujeme bijekci $f: \R \to \{\lim_{} a \mid a \in \mathcal{C}(\Q)\}$.
 Vezměme $x \in \R$. Pak z definice existuje konvergentní racionální posloupnost
 $a \in \mathcal{C}(\Q)$ taková, že $x = [a]$. Podle
 \myref{lemmatu}{lem:limita-racionalni-posloupnosti} má $a$ limitu v $\R$.
 Definujme tedy $f(x) \coloneqq \lim_{} a$.

 Ověříme, že je $f$ dobře definované, prosté a na.

 Nejprve musíme ukázat, že $f(x)$ nezávisí na volbě konkrétní posloupnosti $a$ z
 třídy ekvivalence $[a]$. Ať tedy $b \simeq a$ a označme $L_a \coloneqq \lim_{}
 a, L_b \coloneqq \lim_{} b$. Pak pro každé $\varepsilon>0$ existuje $n_0 \in
 \N$ takové, že $ \forall n \geq n_0$ platí tři nerovnosti:
 \[
  |a_n-b_n|<\varepsilon, \quad |a_n-L_a|<\varepsilon, \quad
  |b_n-L_b|<\varepsilon.
 \]
 Velmi obdobnou úpravou jako v důkaze \myref{důsledku}{cor:r-jsou-uplna}
 dostaneme,
 že
 \[
  |L_a - L_b| \leq |L_a - a_n| + |a_n - b_n| + |b_n - L_b| < 3\varepsilon,
 \]
 odkud $L_a = L_b$, neboť $L_a,L_b$ jsou třídy ekvivalence konvergentních
 posloupností. Společně s faktem, že každá konvergentní posloupnost má přesně
 jednu limitu (\myref{cvičení}{exer:jednoznacnost-limity}), plyne z~předchozí
 úvahy, že $f$ je dobře definováno.

 Dokážeme, že $f$ je prosté. To je snadné, neboť pokud $[a] = [b]$, neboli $a
 \simeq b$, potom $\lim_{} a = \lim_{} b$, což jsme již vlastně dokázali v
 odstavci výše.

 Nakonec zbývá ověřit, že $f$ je na. Ať tedy $L \coloneqq \lim a$ pro nějakou $a
 \in \mathcal{C}(\Q)$. Potom ovšem $[(a)] \in \R$ a podle
 \myref{lemmatu}{lem:limita-racionalni-posloupnosti} platí $\lim a = [(a)]$. To
 ovšem přesně znamená, že $f([(a)]) = L$.

 Tím je důkaz hotov.
\end{corproof}

\section{Poznatky o limitách posloupností}
\label{sec:poznatky-o-limitach-posloupnosti}

Účelem této sekce je shrnout základní poznatky o limitách posloupností, jež
umožní čtenářům limity konkrétních posloupností efektivně počítat a navíc
široké jejich použití v následujících kapitolách.

Začneme technickým, ale nezbytným, konceptem \emph{rozšířené reálné osy} a
pokračovati budeme jedním z nejdůležitějších a dle našeho názoru též
nejkrásnějších výsledků -- tzv. Bolzano\-vou-Weierstraßo\-vou větou. Ta tvrdí v
podstatě toto: mám-li omezenou posloupnost, pak z ní již umím vybrat nekonečně
mnoho prvků, které tvoří posloupnost \emph{konvergentní}.

Ona krása takového tvrzení spočívá v principu, kterým se podrobně zabývá
kombinatorická disciplína zvaná
\href{https://en.wikipedia.org/wiki/Ramsey_theory}{Ramseyho teorie}; v principu,
že v téměř libovolně chaotické struktuře lze nalézt řád, jakmile jest tato
dostatečně velká. Nejedná se jistě o čistě matematický princip, nýbrž dost možná
o princip vzniku vesmíru a života, popsaný již starým Aristotelem ve výmluvném
výroku, \uv{Celek je více než součet svých částí.} V mnoha zpytech se tomuto
jevu přezdívá
\href{https://www.sciencedirect.com/topics/computer-science/emergent-behavior}{Emergent
Behavior} a představuje stav, kdy chování systému nelze plně popsat pouze
studiem jeho jednotlivých prvků.

Pro důkaz Bolzanovy-Weierstraßovy věty potřebujeme jedné pomocné konstrukce,
tzv. \emph{systému vnořených intervalů}. Nejprve si však pořádně definujeme
samotný pojem \emph{intervalu}. K tomu se nám bude hodit rozšířit množinu
reálných čísel o prvky $-\infty$ a $\infty$.

\subsection{Rozšířená reálná osa}
\label{ssec:rozsirena-realna-osa}

\begin{definition}{Rozšířená reálná osa}{rozsirena-realna-osa}
 Definujme množinu $\R^{*} \coloneqq \R \cup \{-\infty,\infty\}$, kde $\infty$,
 resp. $-\infty$, je z definice prvek takový, že $\infty \geq x$, resp.
 $-\infty \leq x$, pro každé $x \in \R$. Množině $\R^{*}$ budeme někdy říkat
 \emph{rozšířená reálná osa}. Rozšíříme rovněž operace $+$ a $ \cdot $ na prvky
 $\infty$ a $-\infty$ následovně.
 \begin{align*}
  \infty + a = a + \infty = \infty,& \quad \text{pro }a \in \R \cup
  \{\infty\},\\
  -\infty + a = a + (-\infty) = -\infty,& \quad \text{pro }a \in \R \cup
  \{-\infty\},\\
  \infty \cdot a = a \cdot \infty = \infty,& \quad \text{pro }a > 0 \text{ nebo
  }a = \infty,\\
  \infty \cdot a = a \cdot \infty = -\infty,& \quad \text{pro }a < 0 \text{ nebo
  }a = -\infty,\\
  -\infty \cdot a = a \cdot (-\infty) = -\infty,& \quad \text{pro }a > 0 \text{
  nebo }a = \infty,\\
  -\infty \cdot a = a \cdot (-\infty) = \infty,& \quad \text{pro }a < 0 \text{
  nebo }a = -\infty,\\
  a \cdot \infty^{-1} = a \cdot (-\infty)^{-1} = 0,& \quad \text{pro }a \in \R.
 \end{align*}
\end{definition}

\begin{warning}{}{pocitacni-s-nekonecnem}
 \myref{Definice}{def:rozsirena-realna-osa} stručně řečeno říká, že se s prvky
 $\infty$ a $-\infty$ zachází podobně jako s ostatními reálnými čísly. Ovšem,
 následující operace zůstávají nedefinovány.
 \[
  \infty + (-\infty), -\infty + \infty, \pm \infty \cdot 0, 0 \cdot ( \pm
  \infty), ( \pm \infty) \cdot ( \pm \infty)^{-1}.
 \]
\end{warning}

Čtenáři možná zpozorovali, že jsme při své
\hyperref[def:limita-posloupnosti]{definici limity} nerozlišili mezi
posloupnostmi, které nemají limitu, protože jejich prvky \uv{skáčou sem a tam},
a posloupnostmi, které ji nemají naopak pro to, že \uv{stále klesají či
stoupají}. Pro další studium záhodno se tohoto nedostatku zlišit.

\begin{definition}{Limita v nekonečnu}{limita-v-nekonecnu}
 Ať $x:\N \to \R$ je reálná posloupnost. Řekneme, že $x$ má limitu $\infty$,
 resp. $-\infty$, když pro každé $K > 0, K \in \R$, existuje $n_0 \in \N$
 takové, že pro všechna $n \geq n_0$ platí $x_n > K$, resp. $x_n < -K$. Píšeme
 $\lim_{} x = \infty$, resp. $\lim_{} x = -\infty$.
\end{definition}

Na reálných číslech existuje uspořádání $ \leq $, které zdědila z čísel
přirozených, prostřednictvím čísel celých a konečně čísel racionálních. Protože,
vděkem naší konstrukci, jsou celá čísla třídy ekvivalence dvojic čísel
přirozených, čísla racionální třídy ekvivalence dvojic čísel celých a čísla
reálná limity konvergentních racionálních posloupností, bylo by vskutku obtížné
a neproduktivní vypsat konkrétní množinovou definici tohoto uspořádání na
reálných číslech. Přidržíme se pročež intuitivního pohledu na věc a důkaz, že
$ \leq $ je skutečně uspořádání na reálných číslech, necháváme laskavému čtenáři
k promyšlení.

Existence uspořádání umožňuje dívat se na podmnožiny $\R$ z jistého
\uv{souvislého} pohledu. Nemusejí již být vňaty (jako tomu je u ostatních
představených číselných okruhů) jako výčty jednotlivých prvků, ale oprávněně
jako \uv{provázky} či \uv{úsečky}. \hyperref[cor:r-jsou-uplna]{Úplnost reálných
čísel} zaručuje, že z každého reálného čísla mohu plynule dorazit do každého
jiného reálného čísla aniž reálná čísla opustím.

Předchozí odstavec vágně motivuje definici \emph{intervalu} -- \uv{souvislé}
omezené podmnožiny reálných čísel. V souhlasu s definicí intervalu vzniká i
pojem \emph{otevřenosti} a \emph{uzavřenosti} množiny -- pojem, který je klíčem
k definici \emph{topologie} na obecné množině a tím pádem vlastně i základem tak
zhruba poloviny celé moderní matematiky.

Směrem k definici intervalu učiňmež koliksi mezikroků.

\begin{definition}{Maximum a minimum}{maximum-a-minimum}
 Ať $X \subseteq \R$ je množina. Řekneme, že prvek $M \in X$, resp. $m \in X$,
 je \emph{maximem}, resp. \emph{minimem}, množiny $X$, když pro každé $x \in X$
 platí $x \leq M$, resp. $x \geq m$. Píšeme $M = \max X$, resp. $m = \min X$.
\end{definition}

\begin{definition}{Horní a dolní závora}{horni-a-dolni-zavora}
 Ať $X \subseteq \R$ je množina. Řekneme, že prvek $Z \in \R^{*}$ resp. $z \in
 \R^{*}$, je \emph{horní}, resp. \emph{dolní}, \emph{závora} množiny $X$, když
 pro každé $x \in X$ platí $x \leq Z$, resp. $x \geq z$.

 Má-li množina $X$ horní, resp. dolní, závoru, \textbf{která leží v $\R$} (tedy
 není rovna $ \pm \infty$), říkáme, že je \emph{shora}, resp. \emph{zdola},
 \emph{omezená}. Je-li navíc $X$ omezená shora i zdola, říkáme krátce, že je
 \emph{omezená}.
\end{definition}

\begin{definition}{Supremum a infimum}{supremum-a-infimum}
 Ať $X \subseteq \R$ je množina. Řekneme, že prvek $S \in \R^{*}$, resp. $i \in
 \R^{*}$, je \emph{supremum}, resp. \emph{infimum}, množiny $X$, když je to její
 \emph{nejmenší horní závora}, resp. \emph{největší dolní závora}. Píšeme $S =
 \sup X$, resp. $i = \inf X$.

 Vyjádřeno symbolicky, prvek $S \in \R$ je \emph{supremem} množiny $X$, když $x
 \leq S$ pro všechna $x \in X$, a kdykoli $x \leq Z$ pro nějaký prvek $Z \in \R$
 a všechna $x \in X$, pak $S \leq Z$. Prvek $i \in \R$ je \emph{infimem} množiny
 $X$, když $x \geq i$ pro všechna $x \in X$, a kdykoli $x \geq z$ pro nějaký
 prvek $z \in \R$ a všechna $x \in X$, pak $i \geq z$.
\end{definition}

\begin{warning}[topsep at break=0pt]{}{supremum-vs-maximum}
 Vřele radíme čtenářům, aby sobě bedlivě přečetli předchozí tři definice a
 uvědomili si -- velmi zásadní, leč lehko přehlédnuté -- jejich vzájemné
 rozdíly.
 \begin{itemize}
  \item Maximum a minimum množiny $X$ je z
   \hyperref[def:maximum-a-minimum]{definice} \textbf{vždy prvkem této množiny}.
   Maximem množiny $\{1,2,3\}$ je prvek $3$ a jeho minimem je prvek $1$.
  \item Horní, resp. dolní, závora množiny $X$ je \textbf{libovolné
   \clr{rozšířené} reálné číslo} (tedy klidně i $ \pm \infty$), které je větší,
   resp. menší, než všechny prvky $X$. Horní závorou množiny $\{1,2,3\}$ je
   číslo $69$, též $\infty$ a též číslo $3$. Horní a dolní závora \textbf{může,
   ale nemusí}, být prvkem $X$.
  \item Supremum, resp. infimum, množiny $X$ je \textbf{rozšířené reálné číslo},
   které je větší, resp. menší, než všechny prvky $X$, ale \textbf{zároveň
   menší, resp. větší, než každá jeho horní, resp. dolní, závora}. Supremum a
   infimum \textbf{může, ale nemusí, ležet v množině $X$}. Touto vlastností se
   přesně rozlišují \emph{uzavřené} a \emph{otevřené} intervaly -- interval je
   uzavřený, když jeho supremum v~něm leží, kdežto otevřený, když nikoliževěk.
   Supremem množiny $\{1,2,3\}$ je číslo $3$ a jeho infimem je číslo $1$.
 \end{itemize}
 Daná podmnožina $X \subseteq \R$ \textbf{nemusí nutně mít maximum a minimum},
 ale, a to si dokážeme, \textbf{má vždy supremum, resp. infimum}. Je-li navíc
 shora, resp. zdola, omezená, pak toto supremum, resp. infimum, leží v $\R$.
\end{warning}

\begin{exercise}{}{sup-inf-prazdne-mnoziny}
 Určete z \hyperref[def:supremum-a-infimum]{definice suprema a infima} $\inf
 \emptyset$ a $\sup \emptyset$.
\end{exercise}

\begin{exercise}{}{sup-inf-jednoznacne}
 Dokažte, že $\sup X$ a $\inf X$ jsou určeny jednoznačně.
\end{exercise}

\subsubsection{Axiomatická definice reálných čísel}
\label{sssec:axiomaticka-definice-realnych-cisel}

Přestože jsme konstrukci reálných čísel úspěšně dokončili použitím
konvergentních racionálních posloupností, stojí snad za zmínku i jejich
axiomatická definice, která se obvykle uvádí v úvodních učebnicích matematické
analýzy.

Překvapivě není v principu tak odlišná od jejich konstrukce, kromě jednoho
konkrétního axiomu, jenž právě zaručuje úplnost; není z něj však vůbec na první,
v zásadě ani na druhý, pohled vidno, že takovou vlastnost skutečně implikuje.

\begin{definition}{Axiomatická definice reálných
 čísel}{axiomaticka-definice-realnych-cisel}
 Množina $\R$ se v zásadě definuje jako nekonečné uspořádané těleso s vlastností
 úplnosti. Tedy,
 \begin{itemize}
  \item existují prvky $0,1 \in \R$ a operace $+, \cdot :\R^{2} \to \R$ s
   inverzy $-,^{-1}:\R \to \R$ takové, že
   \[
    (\R,+,-,0, \cdot ,^{-1},1)
   \]
   je nekonečné těleso;
  \item existuje uspořádání $ \leq $ na $\R$, které je lineární (každé dva prvky
   lze spolu porovnat);
  \item (\textbf{axiom úplnosti}) každá shora omezená podmnožina $\R$ má
   supremum.
 \end{itemize}
\end{definition}

Je to právě on poslední axiom v
\hyperref[def:axiomaticka-definice-realnych-cisel]{předchozí definici}, jehož
použití jsme se chtěli vyhnout, bo dohlédnout jeho hloubky je obtížné a
neintuitivní.

Dokážeme si zde ovšem, že naše \hyperref[def:realna-cisla]{definice reálných
čísel} odpovídá jejich axiomatické. Otázky nekonečnosti, podmínek tělesa i
uspořádání jsme již zodpověděli. Zbývá dokázat axiom úplnosti. Pro stručnost
vyjádření se nám bude hodit následující definice.

\begin{definition}{Monotónní posloupnost}{monotonni-posloupnost}
 O posloupnosti $x:\N \to \R$ řekneme, že je
 \begin{itemize}
  \item \emph{rostoucí}, když $x_{n+1}>x_n \; \forall n \in \N$;
  \item \emph{klesající}, když $x_{n+1} < x_n \; \forall n \in \N$;
  \item \emph{neklesající}, když $x_{n+1} \geq x_n \; \forall n \in \N$;
  \item \emph{nerostoucí}, když $x_{n+1} \leq x_n \; \forall n \in \N$.
 \end{itemize}
 Ve všech těchto případech díme, že posloupnost $x$ je \emph{monotónní}.
\end{definition}
\begin{proposition}{Axiom úplnosti}{axiom-uplnosti}
 Ať $X \subseteq \R$ je shora omezená množina. Pak existuje $\sup X$.
\end{proposition}
\begin{propproof}
 Ježto naše \hyperref[cor:r-jsou-uplna]{pojetí úplnosti} se překládá do znění,
 \uv{Každá konvergentní posloupnost má limitu}, není snad nečekané, že se důkaz
 \emph{axiomu úplnosti} o tuto vlastnost opírá.

 Je-li $X$ prázdná, pak má supremum podle
 \myref{cvičení}{exer:sup-inf-prazdne-mnoziny}. Ať je tedy $X$ neprázdná a shora
 omezená a $Z \in \R$ je libovolná horní závora $X$. Protože $X$ je neprázdná,
 existuje $q \in \R$ takové, že $q < x$ pro nějaké $x \in X$. Definujeme
 posloupnosti $Z_n$ a $q_n$ podle následujících pravidel.
 \begin{itemize}
  \item Položme $Z_0 \coloneqq Z$ a $q_0 \coloneqq q$.
  \item Uvažme číslo $p_n \coloneqq (Z_n + q_n) / 2$.
  \item Je-li $p_n$ horní závorou $X$, položme $Z_{n+1} \coloneqq p_n$ a
   $q_{n+1} \coloneqq q_n$.
  \item Není-li $p_n$ horní závorou $X$, položme $Z_{n+1} \coloneqq Z_n$ a
   $q_{n+1} \coloneqq p_n$.
 \end{itemize}
 Pak jsou posloupnosti $Z_n$ a $q_n$ konvergentní (\textbf{proč?}) a indukcí lze
 snadno dokázat (\textbf{dokažte!}), že $q_n$ \textbf{není} horní závorou $X$ a
 $Z_n$ \textbf{je} horní závorou $X$ pro všechna $n \in \N$. Navíc platí
 $\lim_{} |Z_n - q_n| = 0$ (\textbf{proč?}), a tedy $\lim_{} Z_n = \lim_{} q_n$.

 Označme $S \coloneqq \lim_{} Z_n = \lim_{} q_n$. Dokážeme, že $S = \sup X$. Je
 třeba ukázat, že
 \begin{enumerate}
  \item $S$ je horní závorou $X$;
  \item $S$ je nejmenší horní závorou.
 \end{enumerate}
 Předpokládejme pro spor, že existuje $x \in X$ takové, že $x > S$. To znamená,
 že existuje konstanta $c > 0$ taková, že $x - S = c$. Volme $\varepsilon
 \coloneqq c / 2$. Pro toto $\varepsilon$ z
 \hyperref[def:limita-posloupnosti]{definice limity} existuje $n_0 \in \N$
 takové, že pro všechna $n \geq n_0$ platí $|Z_n - \lim_{} Z_n| = |Z_n - S| <
 \varepsilon$. Jelikož $(Z_n)$ je nerostoucí a $S \leq Z_n$ pro každé $n \in
 \N$, je absolutní hodnota v předchozím výrazu zbytečná a můžeme zkrátka psát
 $Z_n - S < \varepsilon$. Potom ale pro všechna $n \geq n_0$ máme
 \[
  x - Z_n = x + S - S - Z_n = (x - S) + (S - Z_n) > c - \varepsilon =
  \frac{c}{2},
 \]
 čili speciálně $x > Z_n$, což je ve sporu s tím, že $Z_n$ je horní závora $X$.
 To dokazuje (1).

 Tvrzení (2) lze dokázat obdobně, akorát využitím posloupnosti $(q_n)$ spíše než
 $(Z_n)$. Opět ať pro spor existuje $Z \in \R$, které je horní závorou $X$, a $Z
 < S$. Pak nalezneme konstantu $c > 0$ takovou, že $S - Z = c$. Opět z
 \hyperref[def:limita-posloupnosti]{definice limity} vezmeme $\varepsilon
 \coloneqq c / 2$ a k němu $n_0 \in \N$ takové, že $\forall n \geq n_0$ platí $S
 - q_n < \varepsilon$, kde absolutní hodnotu jsme mohli vynechat, ježto jest
 posloupnost $(q_n)$ neklesající a $S \geq q_n$ pro každé $n \in \N$. Nyní pro
 $n \geq n_0$ platí
 \[
  q_n - Z = q_n - S + S - Z = (q_n - S) + (S - Z) > c - \varepsilon =
  \frac{c}{2},
 \]
 čili speciálně $q_n > Z$, což je ve sporu s tím, že $q_n$ není horní závora $X$
 pro žádné $n \in \N$, zatímco $Z$ je.

 Tím je důkaz dokončen.
\end{propproof}
\begin{figure}[ht]
 \centering
 \begin{subfigure}[b]{.99\textwidth}
  \centering
  \begin{tikzpicture}
   \tkzDefPoints{0/0/a,12/0/b}
   \tkzDrawSegment(a,b)

   \tkzDefPoint(8,0){S}
   \tkzLabelPoint[BrickRed,yshift=-1mm](S){$S$}
   \tkzDrawSegment[ultra thick,BrickRed,latex-|](a,S)
   \tkzLabelSegment[pos=0.5,yshift=-1mm](a,S){$\clr{X}$}
   \tkzDefPoints{3/0/q0,11/0/Z0,7/0/q1}
   \tkzDrawPoint[RoyalBlue,size=6](q0)
   \tkzDrawPoint[ForestGreen,size=6](Z0)
   \tkzDrawPoint[Fuchsia,size=6](q1)
   \tkzLabelPoint[above,yshift=2mm,RoyalBlue](q0){$q_0$}
   \tkzLabelPoint[above,yshift=2mm,ForestGreen](Z0){$Z_1 \coloneqq Z_0$}
   \tkzLabelPoint[above,yshift=2mm,Fuchsia](q1){$q_1 \coloneqq
   \frac{Z_0+q_0}{2}$}
  \end{tikzpicture}
 \end{subfigure}
 \begin{subfigure}[b]{.99\textwidth}
  \centering
  \begin{tikzpicture}
   \tkzDefPoints{0/0/a,12/0/b}
   \tkzDrawSegment(a,b)

   \tkzDefPoint(8,0){S}
   \tkzLabelPoint[BrickRed,yshift=-1mm](S){$S$}
   \tkzDrawSegment[ultra thick,BrickRed,latex-|](a,S)
   \tkzLabelSegment[pos=0.5,yshift=-1mm](a,S){$\clr{X}$}
   \tkzDefPoints{7/0/q1,11/0/Z1,9/0/Z2}
   \tkzDrawPoint[RoyalBlue,size=6](q1)
   \tkzDrawPoint[ForestGreen,size=6](Z1)
   \tkzDrawPoint[Fuchsia,size=6](Z2)
   \tkzLabelPoint[above,yshift=2mm,RoyalBlue](q1){$q_2 \coloneqq q_1$}
   \tkzLabelPoint[above,yshift=2mm,ForestGreen](Z1){$Z_1$}
   \tkzLabelPoint[above,yshift=2mm,Fuchsia](Z2){$Z_2 \coloneqq \frac{Z_1 +
   q_1}{2}$}
   \tkzDefPoints{6/-1/d1,6/-1.2/d2,6/-1.4/d3}
   \tkzDrawPoints(d1,d2,d3)
  \end{tikzpicture}
 \end{subfigure}
 \caption{Důkaz axiomu úplnosti}
 \label{fig:axiom-uplnosti}
\end{figure}
\begin{exercise}{}{axiom-uplnosti}
 Dokažte všechna (\textbf{proč?}) a (\textbf{dokažte!}) v důkazu
 \hyperref[prop:axiom-uplnosti]{předchozího tvrzení}.
\end{exercise}

Jako každé poctivé tvrzení, jmá i \hyperref[prop:axiom-uplnosti]{axiom úplnosti}
svých důsledkův. Tyto bychom pochopitelně dokázati uměli i bez něj, neboť axiom
úplnosti z naší konstrukce reálných čísel přímo plyne. Nicméně, zcela jistě jej
lze použít jako nástroj ke zkrácení některých důkazů.

Nejprve duální tvrzení.

\begin{proposition}{}{axiom-infima}
 Každá zdola omezená podmnožina $\R$ má infimum.
\end{proposition}
\begin{propproof}
 Cvičení. Doporučujeme čtenářům se zamyslet, jak tvrzení snadno plyne z
 \hyperref[prop:axiom-uplnosti]{axiomu úplnosti}, aniž opakují konstrukci z jeho
 důkazu.
\end{propproof}

Jedno, jak bude časem vidno, mimořádně užitečné tvrzení dí, že shora omezené
rostoucí či neklesající posloupnosti a zdola omezené klesající či nerostoucí
posloupnosti mají vždy limitu. To je opět intuitivně zřejmý fakt (jistě?), ale,
kterak čtenáři doufáme již pozřeli, tvrzení o věcech nekonečných řídce radno
nechati pouze intuici.

\begin{lemma}{Limita monotónní posloupnosti}{limita-monotonni-posloupnosti}
 \begin{enumerate}[label=(\alph*)]
  \item Každá rostoucí nebo neklesající shora omezená posloupnost je
   konvergentní.
  \item Každá klesající nebo nerostoucí zdola omezená posloupnost je
   konvergentní.
 \end{enumerate}
\end{lemma}
\begin{lemproof}
 Dokážeme pouze část (a), část (b) je ponechána jako cvičení.

 Ať $x:\N \to \R$ je neklesající posloupnost. Důkaz pro rostoucí posloupnost je
 téměř dokonale stejný, liše se akorát ostrými nerovnostmi v několika výrazech.
 Z předpokladu je $x$ shora omezená, tudíž má množina jejích členů $\{x_n \mid n
 \in \N\}$ horní závoru. Z \hyperref[prop:axiom-uplnosti]{axiomu úplnosti} má
 tato množina též supremum; označíme je $S$.

 Ukážeme, že $\lim x = S$. Ať je $\varepsilon>0$ dáno. Z
 \hyperref[def:supremum-a-infimum]{definice suprema} není $S - \varepsilon$
 horní závora množiny $\{x_n \mid n \in \N\}$. Tedy existuje $n_0 \in \N$
 takové, že $x_{n_0} > S - \varepsilon$. Protože $x$ je neklesající -- tj. $x_n
 \geq x_{n_0}$, kdykoli $n \geq n_0$ -- platí rovněž $x_n > S - \varepsilon$ pro
 všechna $n \geq n_0$. Jelikož $S$ je horní závora množiny členů $x$, platí $S
\geq x_n$ pro všechna $n \in \N$. To však znamená, že $|x_n - S| = S - x_n$, a
tedy z nerovnosti $x_n > S - \varepsilon$ po úpravě plyne, že $\varepsilon > S -
x_n = |x_n - S|$, čili $\lim x = S$.
\end{lemproof}

Posledním důsledkem \hyperref[prop:axiom-uplnosti]{axiomu úplnosti}, který si
uvedeme, je tzv. \emph{Archimédova vlastnost reálných čísel}. Obecně, těleso se
nazývá \emph{Archimédovo}, když vágně řečeno neobsahuje žádné nekonečně velké
ani nekonečně malé prvky \textbf{vzhledem ke zvolené absolutní hodnotě}. Ukazuje
se, že na reálných číslech lze definovat jen dva typy funkcí absolutní hodnoty
-- jednu \uv{obvyklou}, též vyjádřitelnou vztahem $|x| = \sqrt{x^2}$, a pak tzv.
\href{https://en.wikipedia.org/wiki/P-adic_valuation}{$p$-adickou absolutní
hodnotu} pro $p$ prvočíslo. Libovolná další konstrukce absolutní hodnoty
(majíc přirozené vlastnosti) již je ekvivalentní absolutní hodnotě jednoho z
těchto typů. Reálná čísla jsou Archimédova vzhledem k obvyklé absolutní hodnotě,
ale nikoliv vzhledem k libovolné $p$-adické absolutní hodnotě.

\begin{lemma}{Archimédova vlastnost reálných
 čísel}{archimedova-vlastnost-realnych-cisel}
 Pro každé $\varepsilon \in \R, \varepsilon>0$, existuje $n \in \N$ takové, že
 $1 / n < \varepsilon$.
\end{lemma}
\begin{lemproof}
 Stačí dokázat, že
 \[
  \inf \left\{\frac{1}{n} \mid n \in \N\right\} = 0,
 \]
 neboť potom z \hyperref[def:supremum-a-infimum]{definice infima} pro každé
 $\varepsilon>0$ není $0 + \varepsilon = \varepsilon$ dolní závorou $\{1 / n
 \mid n \in \N\}$, čili existuje $n \in \N$ takové, že $1 / n < \varepsilon$.

Číslo $0$ je zřejmě dolní závorou množiny $\{1 / n \mid n \in \N\}$. Podle
\myref{tvrzení}{prop:axiom-infima} má tato množina infimum, označme je $i$. Pro
spor ať $i > 0$. Potom $1 / i \in \R$ a z nerovnosti $1 / n \geq i$ ($i$ je
dolní závora) plyne, že $n \leq 1 / i$ pro všechna $n \in \N$. Potom je ovšem
číslo $1 / i$ horní závorou množiny $\N$ a podle
\hyperref[prop:axiom-uplnosti]{axiomu úplnosti} má množina $\N$ supremum;
označme je $S$. Pro každé $n \in \N$ tudíž platí $n \leq S$. Ovšem, z
\hyperref[def:prirozena-cisla]{definice přirozených čísel} platí $n+1 \in \N$
pro každé $n \in \N$. Speciálně toto tedy znamená, že $n + 1 \leq S$ pro každé
$n \in \N$. Pak je ovšem $S - 1$ horní závorou množiny $\N$, což je spor, neboť
$S$ bylo z předpokladu supremum $\N$.

Musí pročež platit $i = 0$, což bylo dokázati.
\end{lemproof}

\begin{remark}{}{archimedova-vlastnost-realnych-cisel}
 \myref{Lemma}{lem:archimedova-vlastnost-realnych-cisel} v podstatě říká, že
 $\lim_{n \to \infty} 1 / n = 0$.

 Bedliví čtenáři si mohou pamatovat, že jsme ono lemma již v předchozím textu
 bez uvedení použili (například v důkaze \myref{důsledku}{cor:r-jsou-uplna}).
 Jedná se však z naší strany o drzost pouze malou. Totiž, jeho platnost je téměř
 okamžitým důsledkem \myref{tvrzení}{prop:hustota-q-v-r}, jak si čtenáři rádi
 ověří v~následujícím cvičení.
\end{remark}
\begin{exercise}{}{archimedes-z-hustoty}
 Dokažte, že \myref{lemma}{lem:archimedova-vlastnost-realnych-cisel} je
 důsledkem \myref{tvrzení}{prop:hustota-q-v-r}.
\end{exercise}

\subsection{Bolzanova-Weierstraßova věta}
\label{ssec:bolzanova-weierstrassova-veta}

Konečně kráčíme cestou definice intervalu a důkazu slibované
Bolzanovy-Weierstraßovy věty. Vybaveni pojmy
\hyperref[def:maximum-a-minimum]{maxima (minima)} a
\hyperref[def:supremum-a-infimum]{suprema (infima)}, můžeme intuitivní
představě intervalu dát formální ráz. Vágně řečeno je interval \emph{souvislá}
podmnožina $\R$. Formálně je no ... vlastně totéž.

\begin{definition}{Interval}{interval}
 Podmnožinu $I \subseteq \R$ nazveme \emph{intervalem}, pokud pro každé dva
 prvky $x < y \in I$ a $z \in \R$ platí
 \[
  x < z < y \Rightarrow z \in I.
 \]
\end{definition}

Intervaly mohou být otevřené, uzavřené a polouzavřené (či polootevřené?). Tyto
vlastnosti intervalů jsou definovány pomocí existence maxim a minim.

\begin{definition}{Typy intervalů}{typy-intervalu}
 Ať $I \subseteq \R$ je interval. Řekneme, že $I$ je
 \begin{itemize}
  \item \emph{otevřený}, když \textbf{nemá} maximum ani minimum;
  \item \emph{uzavřený}, když \textbf{má} maximum i minimum;
  \item \emph{shora uzavřený}, když má pouze maximum, ale nikoli minimum;
  \item \emph{zdola uzavřený}, když má pouze minimum, ale nikoli maximum.
 \end{itemize}
 Otevřený interval $I$ zapisujeme jako $I = (a,b)$, kde $a = \inf I$ a $b = \sup
 I$. Čísla $a,b$ mohou být i $ \pm \infty$, pokud $I$ není shora či zdola
 omezený.

 Uzavřený interval $I$ zapisujeme jako $I = [a,b]$, kde $a = \min I$ a $b = \max
 I$. \textbf{Pozor!} Zde prvky $a$ i $b$ jsou striktně reálná čísla, tedy
 například $[0,\infty]$ \textbf{není} interval, neboť se nejedná o podmnožinu
 $\R$.
\end{definition}

\begin{definition}{Délka intervalu}{delka-intervalu}
 Délkou intervalu $I \subseteq \R$ s $a \coloneqq \inf I$ a $b \coloneqq \sup I$
 myslíme číslo $\lambda(I) \coloneqq b - a$, je-li toto definováno.
\end{definition}

\begin{remark}{}{delka-intervalu}
 Čtenáře snad mohlo zarazit značení $\lambda(I)$ pro délku intervalu, oproti
 zvyku podlehnuvšímu $|I|$. Písmeno $\lambda$ zde není spojeno s angl. slovem
 \textbf{l}ength, jak by se snad mohlo prve zdát, nýbrž pochází ze jména
 \textbf{L}ebesgue. Totiž, \emph{délka} intervalu je jeho \emph{objemem} či
 \emph{velikostí} vzhledem k tzv. Lebesgueově míře -- mnohem obecnější
 konstrukci umožňující měřit velikosti všemožných podmnožin reálných čísel.
\end{remark}

\begin{example}{Pár intervalů}{par-intervalu}
 Množina
 \begin{itemize}
  \item $I = (4,6)$ je otevřený interval. Zřejmě platí $4 = \inf I$ a $6 = \sup
   I$. Ovšem, $I$ nemá maximum ani minimum.
  \item $I = [-5,4]$ je uzavřený interval. Zřejmě platí $-5 = \min I = \inf I$ a
   $4 = \max I = \sup I$.
  \item $I = [-2,\infty)$ je zdola uzavřený interval. Platí $-2 = \min I = \inf
   I$ a $\infty = \sup I$.
  \item $\R = (-\infty,\infty)$ je otevřený interval. Platí $-\infty = \inf \R$
   a $\infty = \sup \R$.
  \item $I = (4,4)$ je prázdná, neboť je to z definice množina čísel $x \in \R$
   takových, že $4 < x < 4$.
  \item $I = [\exp(\tan(\log^{3}(\sqrt[7]{\pi /
   4}))),\exp(\tan(\log^{3}(\sqrt[7]{\pi / 4})))]$ je rovna
   $\{\exp(\tan(\log^{3}(\sqrt[7]{\pi / 4})))\}$, neboť je to z definice množina
   čísel $x \in \R$ takových, že
   \[
    \exp(\tan(\log^{3}(\sqrt[7]{\pi / 4}))) \leq x \leq
    \exp(\tan(\log^{3}(\sqrt[7]{\pi / 4}))).
   \]
 \end{itemize}
\end{example}

K pojmu intervalu se víže jedna speciální konstrukce zvaná \emph{systém
vnořených intervalů}. Definujeme si ji a ihned poté si povíme, čím je speciální.

\begin{definition}{Systém vnořených intervalů}{system-vnorenych-intervalu}
 \emph{Systém vnořených intervalů} je posloupnost $(I_n)_{n=0}^{\infty}$
 podmnožin $\R$ (čili zobrazení $\N \to 2^{\R}$) splňující následující podmínky:
 \begin{itemize}
  \item $I_n$ je \textbf{uzavřený} interval pro každé $n \in \N$;
  \item $I_{n+1} \subseteq I_n$ pro každé $n \in \N$;
  \item $\lim_{n \to \infty} \lambda(I_n) = 0$.
 \end{itemize}
\end{definition}

Následující tvrzení je dalším ekvivalentem \hyperref[prop:axiom-uplnosti]{axiomu
úplnosti} a \myref{důsledku}{cor:r-jsou-uplna}. V některých definicích reálných
čísel se jím \hyperref[prop:axiom-uplnosti]{axiom úplnosti} nahrazuje.

\begin{proposition}{O vnořených intervalech}{vnorene-intervaly}
 Ať $(I_n)_{n=0}^{\infty}$ je \hyperref[def:system-vnorenych-intervalu]{systém
 vnořených intervalů}. Pak $\# (\bigcap_{n=0}^{\infty} I_n) = 1$, čili v průniku
 všech intervalů $I_n$ leží přesně jeden prvek.
\end{proposition}
\begin{propproof}
 Je třeba dokázat, že takový prvek existuje a že je právě jeden. Začněme
 jednoznačností.

 Předpokládejme, že existují prvky $x,y \in \bigcap_{n=0}^{\infty} I_n$ a $x
 \neq y$. Pak ale existuje konstanta $c>0$ taková, že $|x-y| = c$. Protože však
 $x,y \in I_n$ pro každé $n \in \N$, speciálně platí $\lambda(I_n) \geq c$ pro
 každé $n \in \N$. To je spor s tím, že $\lim_{n \to \infty} \lambda(I_n) = 0$.

 Dokážeme existenci. Označme $I_n = [a_n,b_n]$. Definujme posloupnost $x:\N \to
 \R$, $x_n \coloneqq (a_n + b_n) / 2$. Na volbě čísla $(a_n + b_n) / 2$ není nic
 speciálního. Stačí volit jakékoliv $x_n \in I_n$. Ukážeme, že $x$ konverguje.
 Ať je dáno $\varepsilon>0$. Protože $\lim_{n \to \infty} \lambda(I_n) = 0$,
 nalezneme $n_0 \in \N$ takové, že $\lambda(I_{n_0})<\varepsilon$. Potom ale
 platí $|x_n-x_m|<\varepsilon$ pro všechna $m,n \geq n_0$, neboť $x_n,x_m \in
 I_{n_0}$, což je zaručeno podmínkou $I_n,I_m \subseteq I_{n_0}$.

 Podle \myref{důsledku}{cor:r-jsou-uplna} má $x$ limitu, označme ji $L$. Chceme
 ukázat, že $L \in \bigcap_{n=0}^{\infty}I_n$. K tomu je třeba ověřit, že $L \in
 I_n$ pro každé $n \in \N$. Ať pro spor existuje $n_L \in \N$ takové, že $L
 \notin I_{n_L}$. Protože intervaly jsou vnořené, znamená toto, že $L \notin
 I_n$ pro $n \geq n_L$. Volme libovolné $\varepsilon>0$. K němu nalezneme $n_I
 \in \N$ takové, že $\lambda(I_n)<\varepsilon$ pro $n \geq n_I$. Ať $n_0
 \coloneqq \max(n_L,n_I)$. Pak na jednu stranu pro $n \geq n_0$ platí
 $\lambda(I_n)<\varepsilon$ a na druhou stranu $L \notin I_n$. Sloučením obou
 vztahů dostaneme $|x_n-L| \geq \varepsilon / 2$ pro $n \geq n_0$, neboť $x_n$
 leží v polovině intervalu $I_n$ a $L$ mimo něj pro každé $n \in \N$. To je spor
 s tím, že $\lim x = L$.

 Důkaz je hotov.
\end{propproof}

\begin{figure}[ht]
 \centering
 \begin{subfigure}[b]{.99\textwidth}
  \centering
  \begin{tikzpicture}
   \tkzDefPoints{0/0/L,12/0/R}
   \tkzDrawSegment(L,R)
   \tkzDefPoints{2/0/l,10/0/r}
   \node[BrickRed] at (l) {$[$};
   \node[BrickRed] at (r) {$]$};
   \tkzDrawSegment[thick,BrickRed](l,r)
   \tkzLabelSegment[below,BrickRed,yshift=-2mm](l,r){$I_0$}
   \tkzDefPoint(6,0){x0}
   \tkzDrawPoint[size=4](x0)
   \tkzLabelPoint[above](x0){$x_0$}
  \end{tikzpicture}
 \end{subfigure}
 \begin{subfigure}[b]{.99\textwidth}
  \centering
  \begin{tikzpicture}
   \tkzDefPoints{0/0/L,12/0/R}
   \tkzDrawSegment(L,R)
   \tkzDefPoints{2/0/l,10/0/r}
   \node[BrickRed] at (l) {$[$};
   \node[BrickRed] at (r) {$]$};
   \tkzDrawSegment[BrickRed,dashed,thick](l,r)
   \tkzLabelSegment[below,BrickRed,yshift=-2mm](l,r){$I_0$}
   \tkzDefPoint(6,0){x0}
   \tkzLabelPoint[above](x0){$x_0$}

   \tkzDefPoints{3/0/l2,8/0/r2}
   \node[RoyalBlue] at (l2) {\Large $[$};
   \node[RoyalBlue] at (r2) {\Large $]$};
   \tkzDrawSegment[ultra thick,RoyalBlue](l2,r2)
   \tkzLabelSegment[below,RoyalBlue,yshift=-2mm](l2,r2){$I_1$}
   \tkzDefPoint(5.5,0){x1}
   \tkzDrawPoint[size=4](x1)
   \tkzDrawPoint[size=4](x0)
   \tkzLabelPoint[above](x1){$x_1$}
  \end{tikzpicture}
 \end{subfigure}
 \begin{subfigure}[b]{.99\textwidth}
  \centering
  \begin{tikzpicture}
   \tkzDefPoints{0/0/L,12/0/R}
   \tkzDrawSegment(L,R)
   \tkzDefPoints{2/0/l,10/0/r}
   \node[BrickRed] at (l) {$[$};
   \node[BrickRed] at (r) {$]$};
   \tkzDrawSegment[BrickRed,dashed,thick](l,r)
   \tkzLabelSegment[below,BrickRed,yshift=-2mm](l,r){$I_0$}
   \tkzDefPoint(6,0){x0}
   \tkzDrawPoint[size=4](x0)
   \tkzLabelPoint[above](x0){$x_0$}

   \tkzDefPoints{3/0/l2,8/0/r2}
   \node[RoyalBlue] at (l2) {$[$};
   \node[RoyalBlue] at (r2) {$]$};
   \tkzDrawSegment[thick,RoyalBlue,dashed](l2,r2)
   \tkzLabelSegment[below,RoyalBlue,yshift=-2mm](l2,r2){$I_1$}
   \tkzDrawPoint[size=4](x0)
   \tkzDefPoint(5.5,0){x1}
   \tkzDrawPoint[size=4](x1)
   \tkzLabelPoint[above](x1){$x_1$}

   \tkzDefPoints{3.5/0/l3,5/0/r3}
   \node[ForestGreen] at (l3) {\Large $[$};
   \node[ForestGreen] at (r3) {\Large $]$};
   \tkzDrawSegment[ultra thick,ForestGreen](l3,r3)
   \tkzLabelSegment[below,ForestGreen,yshift=-2mm](l3,r3){$I_2$}
   \tkzDefPoint(4.25,0){x2}
   \tkzDrawPoint[size=4](x2)
   \tkzLabelPoint[above](x2){$x_2$}

   \node at (6,-1) {\Large $\vdots$};
  \end{tikzpicture}
 \end{subfigure}
 \caption{Důkaz \myref{tvrzení}{prop:vnorene-intervaly}.}
 \label{fig:vnorene-intervaly}
\end{figure}

\begin{definition}{Podposloupnost}{podposloupnost}
 Řekneme, že $y:\N \to \R$ je \emph{podposloupností} posloupnosti $x:\N \to \R$,
 když pro každé $n \in \N$ existuje $m \in \N$ takové, že $y_n = x_m$. Jinak
 řečeno, každý prvek $y$ je rovněž prvkem $x$.
\end{definition}

Již máme všechny ingredience k formulaci a důkazu Bolzanovy-Weierstraßovy věty.
Je stěžejním tvrzením pro matematickou analýzu a pro matematiku obecně. Jeho
filosofický význam dlí v poznání, že v \uv{příliš velkých} strukturách přirozeně
vzniká řád.

\begin{theorem}{Bolzanova-Weierstraßova}{bolzanova-weierstrassova}
 Ať $x:\N \to \R$ je \textbf{omezená} posloupnost. Pak existuje podposloupnost
 $y$ posloupnosti $x$, která konverguje.
\end{theorem}
\begin{thmproof}
 Z omezenosti $x$ existují $s,S \in \R$ taková, že $s \leq x_n \leq S$ pro
 všechna $n \in \N$. Induktivně vyrobíme systém vnořených intervalů. Položme
 $I_0 \coloneqq [s,S]$. Za předpokladu, že $I_n = [a_n,b_n]$ je dán, sestrojíme
 $I_{n+1}$ následovně:
 \begin{equation}
  \label{eq:bw-construction}
  I_{n+1} \coloneqq \begin{cases}
   [a_n, (a_n+b_n) / 2], &\text{pokud } x_k \in [a_n, (a_n+b_n) / 2] \text{ pro
   nekonečně mnoho } k \in \N,\\
    [ (a_n+b_n) / 2,b_n ], &\text{jinak}.
  \end{cases}
 \end{equation}
 Rozmyslíme si lehce neformálním použitím matematické indukce, že tato
 konstrukce je korektní. První interval $I_0$ jistě obsahuje nekonečně mnoho
 prvků $x$, neboť obsahuje celou tuto posloupnost. Podobně, pokud $I_n$ obsahuje
 nekonečně mnoho prvků $x$, pak aspoň jedna z jeho polovin musí rovněž obsahovat
 nekonečně mnoho prvků $x$. Z konstrukce~\eqref{eq:bw-construction} pak plyne,
 že rovněž $I_{n+1}$ obsahuje nekonečně mnoho prvků $x$.

 Ověříme, že $(I_n)_{n=0}^{\infty}$ je systém vnořených intervalů podle
 \myref{definice}{def:system-vnorenych-intervalu}.
 \begin{itemize}
  \item Zcela jistě je $I_n$ uzavřený interval pro každé $n \in \N$.
  \item Rovněž zcela jistě $I_{n+1} \subseteq I_n$ pro každé $n \in \N$, neboť
   $I_{n+1}$ je jedna z polovin intervalu $I_n$.
  \item Délky intervalů $I_n$ klesají k $0$, neboť $\lambda(I_{n+1}) =
   \lambda(I_n) / 2$, a tedy $\lambda(I_n) = \lambda(I_0) / 2^{n}$. Zřejmě
   \[
    \lim_{n \to \infty} \lambda(I_n) = \lim_{n \to \infty}
    \frac{\lambda(I_0)}{2^{n}} = 0.
   \]
 \end{itemize}
 Vyberme nyní z $x$ libovolnou podposloupnost $y:\N \to \R$ takovou, že $y_n \in
 I_n$. To jistě lze, neboť každý z intervalů $I_n$ obsahuje nekonečně mnoho
 prvků posloupnosti $x$. Pak ovšem podle \myref{tvrzení}{prop:vnorene-intervaly}
 existuje prvek $L \in \bigcap_{n=0}^{\infty} I_n$ a podle důkazu téhož tvrzení
 platí $\lim y = L$. To však znamená, se znalostí
 \myref{lemmatu}{lem:limita-konvergence}, že $y$ konverguje.
\end{thmproof}

\begin{figure}[ht]
 \centering
 \begin{subfigure}[b]{.99\textwidth}
  \centering
  \begin{tikzpicture}
   \tkzDefPoints{0/0/L,12/0/R}
   \tkzDrawSegment(L,R)
   \tkzDefPoints{2/0/l,10/0/r}
   \node[BrickRed] at (l) {\Large $[$};
   \node[BrickRed] at (r) {\Large $]$};
   \tkzDrawSegment[ultra thick,BrickRed](l,r)
   \tkzLabelPoint[below,BrickRed,yshift=-2mm](l){$s$}
   \tkzLabelPoint[below,BrickRed,yshift=-2mm](r){$S$}
   \foreach \x in {1,2,...,10} {
    \tkzDefPoint(3.5 + \x / 10,0){a\x}
    \tkzDefPoint(4 + \x / 20,0){b\x}
    \tkzDrawPoints[size=4](a\x,b\x)
   }
   \tkzDefPoint(3,0){a}
   \tkzDefPoint(3.7,0){b}
   \tkzDefPoint(4.1,0){c}
   \tkzDefPoint(4.2,0){d}
   \tkzDefPoint(4.3,0){e}
   \tkzDefPoint(4.9,0){f}
   \tkzDefPoint(5.1,0){g}
   \tkzDefPoint(5.3,0){h}
   \tkzDefPoint(5.7,0){i}
   \tkzDefPoint(6.4,0){j}
   \tkzDefPoint(7,0){k}
   \tkzDefPoint(8.4,0){m}
   \tkzDefPoint(8.5,0){n}
   \tkzDrawPoints[size=4](a,b,c,d,e,f,g,h,i,j,k,m,n)
   \tkzLabelSegment[below,BrickRed,yshift=-2mm](l,r){$I_0$}
   \tkzDefPoint(6,0){x0}
   \tkzLabelPoint[above,yshift=2mm](x0){$x$}
  \end{tikzpicture}
 \end{subfigure}
 \begin{subfigure}[b]{.99\textwidth}
  \centering
  \begin{tikzpicture}
   \tkzDefPoints{0/0/L,12/0/R}
   \tkzDrawSegment(L,R)
   \tkzDefPoints{2/0/l,6/0/r}
   \node[RoyalBlue] at (l) {\Large $[$};
   \node[RoyalBlue] at (r) {\Large $]$};
   \tkzDrawSegment[ultra thick,RoyalBlue](l,r)
   \tkzLabelPoint[below,RoyalBlue,yshift=-2mm](l){$a_1$}
   \tkzLabelPoint[below,RoyalBlue,yshift=-2mm](r){$b_1$}
   \foreach \x in {1,2,...,10} {
    \tkzDefPoint(3.5 + \x / 10,0){a\x}
    \tkzDefPoint(4 + \x / 20,0){b\x}
    \tkzDrawPoints[size=4](a\x,b\x)
   }
   \tkzDefPoint(3,0){a}
   \tkzDefPoint(3.7,0){b}
   \tkzDefPoint(4.1,0){c}
   \tkzDefPoint(4.2,0){d}
   \tkzDefPoint(4.3,0){e}
   \tkzDefPoint(4.9,0){f}
   \tkzDefPoint(5.1,0){g}
   \tkzDefPoint(5.3,0){h}
   \tkzDefPoint(5.7,0){i}
   \tkzDefPoint(6.4,0){j}
   \tkzDefPoint(7,0){k}
   \tkzDefPoint(8.4,0){m}
   \tkzDefPoint(8.5,0){n}
   \tkzDrawPoints[size=4](a,b,c,d,e,f,g,h,i,j,k,m,n)
   \tkzLabelSegment[below,RoyalBlue,yshift=-2mm](l,r){$I_1$}
   \tkzDefPoint(6,0){x0}
   \tkzLabelPoint[above,yshift=2mm](x0){$x$}
  \end{tikzpicture}
 \end{subfigure}
 \begin{subfigure}[b]{.99\textwidth}
  \centering
  \begin{tikzpicture}
   \tkzDefPoints{0/0/L,12/0/R}
   \tkzDrawSegment(L,R)
   \tkzDefPoints{4/0/l,6/0/r}
   \tkzDrawSegment[ultra thick,ForestGreen](l,r)
   \tkzLabelPoint[below,ForestGreen,yshift=-2mm](l){$a_2$}
   \tkzLabelPoint[below,ForestGreen,yshift=-2mm](r){$b_2$}
   \foreach \x in {1,2,...,10} {
    \tkzDefPoint(3.5 + \x / 10,0){a\x}
    \tkzDefPoint(4 + \x / 20,0){b\x}
    \tkzDrawPoints[size=4](a\x,b\x)
   }
   \tkzDefPoint(3,0){a}
   \tkzDefPoint(3.7,0){b}
   \tkzDefPoint(4.1,0){c}
   \tkzDefPoint(4.2,0){d}
   \tkzDefPoint(4.3,0){e}
   \tkzDefPoint(4.9,0){f}
   \tkzDefPoint(5.1,0){g}
   \tkzDefPoint(5.3,0){h}
   \tkzDefPoint(5.7,0){i}
   \tkzDefPoint(6.4,0){j}
   \tkzDefPoint(7,0){k}
   \tkzDefPoint(8.4,0){m}
   \tkzDefPoint(8.5,0){n}
   \tkzDrawPoints[size=4](a,b,c,d,e,f,g,h,i,j,k,m,n)
   \node[ForestGreen] at (l) {\Large $[$};
   \node[ForestGreen] at (r) {\Large $]$};
   \tkzLabelSegment[below,ForestGreen,yshift=-2mm](l,r){$I_2$}
   \tkzDefPoint(6,0){x0}
   \tkzLabelPoint[above,yshift=2mm](x0){$x$}
   \node at (6,-1) {\Large $\vdots$};
  \end{tikzpicture}
 \end{subfigure}
 \caption{Důkaz \hyperref[thm:bolzanova-weierstrassova]{Bolzanovy-Weierstraßovy
  věty}.}
 \label{fig:bolzano-weierstrass}
\end{figure}

\section{Metody výpočtů limit}
\label{sec:metody-vypoctu-limit}

Tato sekce je veskrze výpočetní, věnována způsobům určování limit rozličných
posloupností -- primárně těch zadaných vzorcem pro $n$-tý člen. Obecně
neexistuje algoritmus pro výpočet limity posloupnosti a například limity
posloupností zadaných rekurentně (další člen je vypočten jako kombinace
předchozích) je často obtížné určit. K jejich výpočtu bývá užito metod z
lineární algebry a obecně metod teorie diskrétních systémů zcela mimo rozsah
tohoto textu.

Přinejmenším v případě limit zadaných \uv{hezkými} vzorci čítajícími podíly
mnohočlenů a odmocnin je obyčejně možné algebraickými úpravami dojít k výsledku.
Uvedeme si pár stěžejních tvrzení sloužících tomuto účelu.

K důkazu prvního bude užitečná následující nerovnost, kterou přenecháváme
čtenáři jako (snadné) cvičení.

\begin{exercise}{}{random-abs-nerovnost}
 Dokažte, že pro čísla $x,y \in \R$ platí
 \[
  | |x| - |y| | \leq |x - y|.
 \]
\end{exercise}

\begin{theorem}{Aritmetika limit}{aritmetika-limit}
 Ať $a,b:\N \to \R$ jsou reálné posloupnosti mající limitu (ale klidně i
 nekonečnou). Pak
 \begin{enumerate}[label=(\alph*)]
  \item $\lim (a + b) = \lim a + \lim b$, je-li pravá strana definována;
  \item $\lim (a \cdot b) = \lim a \cdot \lim b$, je-li pravá strana definována;
  \item $\lim (a / b) = \lim a / \lim b$, platí-li $b \not\simeq 0$ a pravá
   strana je definována.
 \end{enumerate}
\end{theorem}
\begin{thmproof}
 Důkaz této věty je ryze výpočetního charakteru a využívá vhodně zvolených
 odhadů. Vzhledem k tomu, že povolujeme i nekonečné limity, je třeba důkaz
 každého bodu rozložit na případy. Položme $A \coloneqq \lim a, B \coloneqq \lim
 b$.

 \textbf{\emph{Případ $A,B \in \R$}.}\\
 Nejprve budeme předpokládat, že $A,B \in \R$. Pro dané $\varepsilon>0$ existují
 $n_a,n_b \in \N$ taková, že pro každé $n \geq n_a$ platí $|a_n-A|<\varepsilon$
 a pro každé $n \geq n_b$ zas $|b_n-B|<\varepsilon$. Zvolíme-li $n_0 \coloneqq
 \max(n_a,n_b)$, pak pro $n \geq n_0$ platí oba odhady zároveň. Potom ale,
 použitím \hyperref[lem:trojuhelnikova-nerovnost]{trojúhelníkové nerovnosti},
 dostaneme
 \[
  |(a_n+b_n)-(A+B)| = |(a_n-A)+(b_n-B)| \leq |a_n-A| + |b_n-B| <
  \varepsilon+\varepsilon = 2\varepsilon,
 \]
 čili $\lim (a+b) = A+B$. Pro důkaz vzorce pro součin a podíl, musíme navíc
 využít \myref{lemmatu}{lem:konvergentni-omezena}, tedy faktu, že konvergentní
 posloupnosti jsou omezené. Pročež najdeme $C_b \in \R$ takové, že od určitého
 indexu $n_1 \in \N$ dále platí $|b_n| \leq C_b$. Volme nově $n_0 \coloneqq
 \max(n_a,n_b,n_1)$ a pro $n \geq n_0$ počítejme
 \begin{align*}
  |a_n \cdot b_n - A \cdot B| &= |a_n \cdot b_n - b_n \cdot A + b_n \cdot A -
  A \cdot B| = |b_n(a_n - A) + A(b_n - B)|\\
                              & \leq |b_n(a_n - A)| + |A(b_n-B)| = |b_n| \cdot
                              |a_n-A| + |A| \cdot |b_n-B| \\
                              &< |C_b| \cdot \varepsilon + |A| \cdot
                              \varepsilon = (|C_b| + |A|) \cdot \varepsilon.
 \end{align*}
 Protože $|C_b| + |A|$ je kladná konstanta nezávislá na $\varepsilon$, dokazuje
 odhad výše, že $\lim (a \cdot b) = A \cdot B$. Konečně, v případě podílu volme
 $\varepsilon_b = |B| / 2$. K tomuto $\varepsilon_b$ nalezněme $n'_b \in \N$
 takové, že pro $n \geq n'_b$ platí $|b_n - B| < \varepsilon_b$. Poslední
 nerovnost spolu s \myref{cvičením}{exer:random-abs-nerovnost} znamená, že $|
 |b_n| - |B| | < \varepsilon$. Tento vztah si rozepíšeme na
 \[
  |B| - \varepsilon_b < |b_n| < |B| + \varepsilon_b.
 \]
 Levá z těchto nerovností je pak ekvivalentní $|b_n| > |B| / 2$ neboli $1 /
 |b_n| < 2 / |B|$. Položme $n_0 \coloneqq \max(n_a,n_b,n'_b)$. Potom pro $n \geq
 n_0$ máme
 \begin{align*}
  \left| \frac{a_n}{b_n} - \frac{A}{B} \right| &= \left| \frac{a_nB -
  b_nA}{b_nB} \right| = \left| \frac{a_nB - AB + AB - b_nA}{b_nB} \right| \leq
  \left| \frac{B(a_n-A)}{b_nB} \right| + \left| \frac{A(B - b_n)}{b_nB} \right|
  \\
                                               &= \frac{1}{|b_n|}\left(|a_n-A| +
                                               \frac{|A|}{|B|}|B-b_n| \right) <
                                               \frac{2\varepsilon}{|B|}\left(1 +
                                               \frac{|A|}{|B|}\right).
 \end{align*}
 Protože $|A|$ i $|B|$ jsou konstanty nezávislé na $\varepsilon$, toto znamená,
 že $\lim (a / b) = A / B$.

 \textbf{\emph{Případ $A = \pm \infty, B \in \R \setminus \{0\}$.}}\\
 Předpokládejme, že $\lim a = \infty$; případ $\lim a = -\infty$ se dokáže v
 zásadě identicky. Pak pro dané $\varepsilon_a$ existuje $n_a \in \N$ takové, že
 pro $n \geq n_a$ platí $a_n > \varepsilon_a$. Podle
 \myref{lemmatu}{lem:konvergentni-omezena} je posloupnost $b$ omezená, čili
 existuje $C_b > 0$ takové, že $|b_n| \leq C_b$ pro všechna $n \in \N$. Potom
 pro $n \geq n_a$ máme
 \[
  a_n + b_n \geq a_n - C_b > \varepsilon_a - C_b.
 \]
 Jelikož $C_b$ je konstantní, plyne z tohoto odhadu, že $\lim (a + b) = \infty =
 A + B$.

 Pro důkaz součinu nejprve ať $B > 0$. Pak existuje konstanta $C_b > 0$ a $n_b
 \in \N$ takové, že pro $n \geq n_b$ je $b_n \geq C_b$. Pročež, pro libovolné
 $C_a > 0$ a $n \geq \max(n_a,n_b)$ dostaneme
 \[
  a_n \cdot b_n \geq \varepsilon_a \cdot C_b.
 \]
 čili $\lim (a \cdot b) = \infty = A \cdot B$. Z omezenosti (plynoucí z
 konvergence) $b$ pak zase existují $n_b'$ a $K_b > 0$ takové, že $b_n \leq
 K_b$, čili též $1 / b_n \geq 1 / K_b$, pro $n \geq n'_b$. Pro $n \geq
 \max(n_a,n'_b)$ tedy
 \[
  \frac{a_n}{b_n} \geq \frac{\varepsilon_a}{K_b},
 \]
 což dokazuje $\lim (a / b) = \infty = A / B$. Velmi podobně se řeší případ $B <
 0$.

 Zdlouhavý důkaz zakončíme komentářem, že případ $A \in \R \setminus \{0\}, B =
 \pm \infty$ je symetrický předchozímu a případy $A = 0, B = \pm \infty$, též $A
 = \pm \infty, B = 0$ a konečně $A = \pm \infty, B = \pm \infty$ jsou triviální.
\end{thmproof}

\hyperref[thm:aritmetika-limit]{Věta o aritmetice limit} je zcela nejužitečnější
tvrzení k jejich výpočtu, neboť umožňuje limitu výrazu rozdělit na mnoho menších
\uv{podlimit}, jejichž výpočet je snadný. Další dvě lemmata jsou často též
dobrými sluhy.

\begin{lemma}{Limita odmocniny}{limita-odmocniny}
 Ať $a:\N \to [0,\infty)$ je posloupnost \emph{nezáporných} čísel. Ať též
    $\lim a = A$ (speciálně tedy předpokládáme, že $\lim a$ existuje). Potom
 \[
  \lim_{n \to \infty} \sqrt[k]{a_n} = \sqrt[k]{A}
 \]
 pro každé $k \in \N$.
\end{lemma}
\begin{lemproof}
 Zdlouhavý a technický. Ambiciózní čtenáři jsou zváni, aby se o něj pokusili.
\end{lemproof}

\begin{lemma}{O dvou strážnících}{o-dvou-straznicich}
 Ať $a,b,c:\N \to \R$ jsou posloupnosti reálných čísel a $L \coloneqq \lim a =
 \lim c$. Pokud existuje $n_0 \in \N$ takové, že pro každé $n \geq n_0$ platí
 $a_n \leq b_n \leq c_n$, pak $\lim b = L$.
\end{lemma}
\begin{lemproof}
 Protože $\lim a = L$ a též $\lim c = L$, nalezneme pro dané $\varepsilon>0$
 index $n_1 \in \N$ takový, že pro $n \geq n_1$ platí dva odhady:
 \[
  |a_n - L| < \varepsilon \quad \text{a} \quad |c_n - L| < \varepsilon.
 \]
 Potom ovšem $a_n > L - \varepsilon$ a $c_n < L + \varepsilon$. Z předpokladu
 existuje $n_b \in \N$ takové, že $a_n \leq b_n \leq c_n$ pro $n \geq n_b$.
 Zvolíme-li tedy $n_0 \coloneqq \max(n_1,n_b)$, pak pro $n \geq n_0$ platí
 \[
  L - \varepsilon < a_n \leq b_n \leq c_n < L + \varepsilon.
 \]
 Sloučením obou nerovností dostaneme pro $n \geq n_0$ odhad $|b_n -
 L|<\varepsilon$, čili $\lim b = L$.
\end{lemproof}

\begin{figure}[ht]
 \centering
 \begin{tikzpicture}
  \tkzInit[xmin=0,xmax=10,ymin=0,ymax=3]
  \foreach \n in {0,1,...,18} {
   \tkzDefPoint(0.5 * (\n + 1),0){x\n}
   \tkzDrawPoint[shape=cross](x\n)
   \tkzLabelPoint[below](x\n){$\n$}
  }
  \tkzLabelPoint[below,color=RoyalPurple](x11){$11$}
  \foreach \y in {1,2,3,4,5,6} {
   \tkzDefPoint(0,0.5 * \y){y\y}
   \tkzDrawPoint[shape=cross](y\y)
   \tkzLabelPoint[left](y\y){$\y$}
  }
  \tkzDefPoint(0.5,2.8){c0}
  \tkzDefPoint(1,0.9){c1}
  \tkzDefPoint(1.5,2.3){c2}
  \tkzDefPoint(2,1){c3}
  \tkzDefPoint(2.5,1.1){c4}
  \tkzDefPoint(3,0.3){c5}
  \tkzDefPoint(3.5,2.4){c6}
  \tkzDefPoint(4,1.2){c7}
  \tkzDefPoint(4.5,3.2){c8}
  \tkzDefPoint(5,2.6){c9}
  \tkzDefPoint(5.5,1.9){c10}
  \tkzDefPoint(6,2.4){c11}
  \tkzDefPoint(6.5,2.2){c12}
  \tkzDefPoint(7,1.9){c13}
  \tkzDefPoint(7.5,2){c14}
  \tkzDefPoint(8,1.8){c15}
  \tkzDefPoint(8.5,2.05){c16}
  \tkzDefPoint(9,1.95){c17}
  \tkzDefPoint(9.5,1.85){c18}

  \tkzDefPoint(0.5,1){b0}
  \tkzDefPoint(1,2.1){b1}
  \tkzDefPoint(1.5,0.3){b2}
  \tkzDefPoint(2,1.1){b3}
  \tkzDefPoint(2.5,1.6){b4}
  \tkzDefPoint(3,0.8){b5}
  \tkzDefPoint(3.5,2){b6}
  \tkzDefPoint(4,0.9){b7}
  \tkzDefPoint(4.5,0.2){b8}
  \tkzDefPoint(5,2.2){b9}
  \tkzDefPoint(5.5,0.9){b10}
  \tkzDefPoint(6,2){b11}
  \tkzDefPoint(6.5,1.8){b12}
  \tkzDefPoint(7,1){b13}
  \tkzDefPoint(7.5,1.6){b14}
  \tkzDefPoint(8,1.6){b15}
  \tkzDefPoint(8.5,1.4){b16}
  \tkzDefPoint(9,1.7){b17}
  \tkzDefPoint(9.5,1.65){b18}
  
  \tkzDefPoint(0.5,1.5){a0}
  \tkzDefPoint(1,0.2){a1}
  \tkzDefPoint(1.5,3){a2}
  \tkzDefPoint(2,2.7){a3}
  \tkzDefPoint(2.5,1.7){a4}
  \tkzDefPoint(3,0.2){a5}
  \tkzDefPoint(3.5,2.2){a6}
  \tkzDefPoint(4,1){a7}
  \tkzDefPoint(4.5,0.7){a8}
  \tkzDefPoint(5,0.1){a9}
  \tkzDefPoint(5.5,1.3){a10}
  \tkzDefPoint(6,0.8){a11}
  \tkzDefPoint(6.5,1.5){a12}
  \tkzDefPoint(7,0.6){a13}
  \tkzDefPoint(7.5,1){a14}
  \tkzDefPoint(8,1.4){a15}
  \tkzDefPoint(8.5,0.9){a16}
  \tkzDefPoint(9,1.5){a17}
  \tkzDefPoint(9.5,1.55){a18}

  \tkzDrawX[>=latex,label={$n$}]
  \tkzDrawY[>=latex,label={$\clr{a_n},\clg{b_n},\clb{c_n}$}]

  \tkzDefPoint(0,1.6){O}
  \tkzDrawPoint[size=4,color=RoyalPurple](O)
  \tkzLabelPoint[right,color=RoyalPurple](O){$L$}
  \tkzDefPoint(0.5,1.6){P}
  \tkzDefPoint(6.5,1.6){P2}
  \tkzDrawLine[add=0 and 0.7,dashed,color=RoyalPurple,thick](P,P2)

  \tkzDrawPoints[color=RoyalBlue](c0,c1,c2,c3,c4,c5,c6,c7,c8,c9,c10,c11,c12,c13,c14,c15,c16,c17,c18)
  \tkzDrawPoints[color=ForestGreen](b0,b1,b2,b3,b4,b5,b6,b7,b8,b9,b10,b11,b12,b13,b14,b15,b16,b17,b18)
  \tkzDrawPoints[color=BrickRed](a0,a1,a2,a3,a4,a5,a6,a7,a8,a9,a10,a11,a12,a13,a14,a15,a16,a17,a18)

  \foreach \n in {11,12,...,17} {
   \pgfmathparse{\n+1}
   \edef\m{\pgfmathresult}
   \tkzDrawSegment[dashed,color=BrickRed](a\n,a\m)
   \tkzDrawSegment[dashed,color=ForestGreen](b\n,b\m)
   \tkzDrawSegment[dashed,color=RoyalBlue](c\n,c\m)
  }
  \tkzLabelPoint[above,color=RoyalPurple](x11){$n_0$}
  \tkzDefPoint(6,3){wtv}
  \tkzDrawLine[add=-0.15 and 0,dashed,color=RoyalPurple](x11,wtv)
 \end{tikzpicture}

 \caption{Lemma \hyperref[lem:o-dvou-straznicich]{o dvou strážnících}.}
 \label{fig:dva-straznici}
\end{figure}

Zbytek sekce je věnován výpočtům limit vybraných posloupností s účelem objasnit
použití právě sepsaných tvrzení. Mnoho z nich je ponecháno čtenářům jako
cvičení.

\begin{problem}{}{piplacka-limity}
 Spočtěte
 \[
  \lim_{n \to \infty} \frac{2n^2+n-3}{n^3-1}.
 \]
\end{problem}
\begin{probsol}
 Použijeme \hyperref[thm:aritmetika-limit]{větu o aritmetice limit}. Ta
 vyžaduje, aby výsledná strana rovnosti byla definována. Je tudíž možné (a
 žádoucí) limitu spočítat -- často opakovaným použitím této věty -- a teprve na
 konci výpočtu argumentovat, že její nasazení bylo oprávněné.

 Dobrým prvním krokem při řešení limit zadaných zlomky je najít v čitateli i
 jmenovateli \uv{nejrychleji rostoucí} člen. Spojením \uv{nejrychleji rostoucí}
 zde míníme takový člen, velikost ostatních členů je pro velmi velká $n$ vůči
 jehož zanedbatelná. V čitateli zlomku
 \[
  \frac{2n^2 + n - 3}{n^3 - 1}
 \]
 je nejrychleji rostoucí člen právě $2n^2$. Například, pro $n = 10^9$ je $2n^2 =
 2 \cdot 10^{18}$ zatímco $n = 10^9$ zabírá méně než jednu miliardtinu $2n^2$.
 Ve jmenovateli je naopak jediným rostoucím členem $n^3$. Nejrychleji rostoucí
 členy (pro pohodlí bez koeficientů) z obou částí zlomku vytkneme. Dostaneme
 \[
  \frac{2n^2 + n - 3}{n^3 - 1} = \frac{n^2 \left( 2 + \frac{1}{n} -
  \frac{3}{n^2} \right)}{n^3 \left(1 - \frac{1}{n^3}\right)} = \frac{1}{n} \cdot
  \frac{2 + \frac{1}{n} - \frac{3}{n^2}}{1 - \frac{1}{n^3}}.
 \]
 Část (b) \hyperref[thm:aritmetika-limit]{věty o aritmetice limit} nyní dává
 \begin{equation*}
  \label{eq:soucin-limit}
  \tag{$\triangle$}
  \lim_{n \to \infty} \frac{2n^2 + n - 3}{n^3 - 1} = \lim_{n \to \infty}
  \frac{1}{n} \cdot \lim_{n \to \infty} \frac{2 + \frac{1}{n} - \frac{3}{n^2}}{1
  - \frac{1}{n^3}},
 \end{equation*}
 \textbf{za předpokladu, že součin na pravé straně je definován!}

 Již víme, že platí $\lim_{n \to \infty} \frac{1}{n} = 0$. \textbf{Pozor!} Bylo
 by lákavé prohlásit, že výsledná limita je rovna $0$, bo součin čehokoliv s $0$
 je též $0$. To je pravda pro všechna čísla až na $\pm \infty$. Musíme se
 ujistit, že druhá limita v součinu na pravé straně~\eqref{eq:soucin-limit}
 existuje a není nekonečná.

 S použitím \hyperref[thm:aritmetika-limit]{věty o aritmetice limit} (c)
 počítáme
 \[
  \lim_{n \to \infty} \frac{2 + \frac{1}{n} - \frac{3}{n^2}}{1 - \frac{1}{n^3}}
  = \frac{\lim_{n \to \infty} 2 + \frac{1}{n} - \frac{3}{n^2}}{\lim_{n \to
  \infty} 1 - \frac{1}{n^3}}.
 \]
 Limity v čitateli a jmenovateli zlomku výše spočteme zvlášť. Z
 \hyperref[thm:aritmetika-limit]{věty o aritmetice limit}, části (a), plyne, že
 \[
  \lim_{n \to \infty} 2 + \frac{1}{n} - \frac{3}{n^2} = \lim_{n \to \infty} 2 +
  \lim_{n \to \infty} \frac{1}{n} - \lim_{n \to \infty} \frac{3}{n^2} = 2 + 0 -
  0 = 2.
 \]
 Podle stejného tvrzení též
 \[
  \lim_{n \to \infty} 1 - \frac{1}{n^3} = \lim_{n \to \infty} 1 - \lim_{n \to
  \infty} \frac{1}{n^3} = 1 - 0 = 1.
 \]
 To znamená, že
 \[
  \lim_{n \to \infty} \frac{2 + \frac{1}{n} - \frac{3}{n^2}}{1 - \frac{1}{n^3}}
  = \frac{2}{1} = 2.
 \]
 Odtud pak
 \[
  \lim_{n \to \infty} \frac{2n^2 + n - 3}{n^3 - 1} = \lim_{n \to \infty}
  \frac{1}{n} \cdot \lim_{n \to \infty} \frac{2 + \frac{1}{n} - \frac{3}{n^2}}{1
  - \frac{1}{n^3}} = 0 \cdot 2 = 0.
 \]
 Protože všechny výrazy na konci výpočtů jsou reálná čísla (a tedy speciálně jsou
 dobře definované), bylo lze použít \hyperref[thm:aritmetika-limit]{větu o
 aritmetice limit}.
\end{probsol}

\begin{remark}{}{aritmetika-limit}
 Právě vyřešená \myref{úloha}{prob:piplacka-limity} ukazuje, jak dlouhé se
 limitní úlohy stávají při pedantickém ověřování všech předpokladů. A to jsme
 navíc použili \emph{jen jediné tvrzení} k jejímu výpočtu. Takový postup není, z
 pochopitelného důvodu, obvyklý. Opakovaná použití
 \hyperref[thm:aritmetika-limit]{věty o aritmetice limit} se často schovají pod
 jedno prohlášení a výpočet limity je pak mnohem stručnější. Názorně předvedeme.

 Snadno úpravou zjistíme, že
 \[
  \frac{2n^2 + n - 3}{n^3 - 1} = \frac{1}{n} \cdot \frac{2 + \frac{1}{n} -
  \frac{3}{n^2}}{1 - \frac{1}{n^3}}.
 \]
 Potom z \hyperref[thm:aritmetika-limit]{věty o aritmetice limit} platí
 \[
  \lim_{n \to \infty} \frac{2n^2 + n - 3}{n^3 - 1} = \lim_{n \to \infty}
  \frac{1}{n} \cdot \lim_{n \to \infty} \frac{2 + \frac{1}{n} - \frac{3}{n^2}}{1
  - \frac{1}{n^3}} = 0 \cdot \frac{2 + 0 + 0}{1 + 0} = 0.
 \]
 Protože výsledný výraz je definovaný, byla \hyperref[thm:aritmetika-limit]{věta
 o aritmetice limit} použita korektně.

 My rovněž hodláme v dalším textu bez varování řešit podobné limitní příklady
 tímto \uv{zkráce\-ným} způsobem.
\end{remark}

\begin{warning}{}{aritmetika-limit}
 \hyperref[thm:aritmetika-limit]{Větou o aritmetice limit} \textbf{nelze}
 dokazovat, že limita posloupnosti neexistuje, neboť předpokladem každé její
 části je \emph{definovanost} výsledného výrazu. Zanedbání toho předpokladu může
 snadno vést ke lži. Uvažme následující triviální příklad.

 Prohlásili-li bychom, že z \hyperref[thm:aritmetika-limit]{věty o aritmetice
 limit} platí výpočet
 \[
  \lim_{n \to \infty} \frac{n}{n} = \frac{\lim_{n \to \infty} n}{\lim_{n \to
  \infty} n} = \frac{\infty}{\infty},
 \]
 nabyli bychom práva tvrdit, že $\lim_{n \to \infty} n / n$ neexistuje, přestože
 zřejmě platí $\lim_{n \to \infty} n / n = \lim_{n \to \infty} 1 = 1$.
 \hyperref[thm:aritmetika-limit]{Věta o aritmetice limit} je tudíž zcela prázdné
 tvrzení v případě nedefinovanosti výsledného výrazu.
\end{warning}

\begin{problem}{}{limita-binomicka-veta}
 Spočtěte limitu
 \[
  \lim_{n \to \infty} \frac{(n+4)^{100} - (n+3)^{100}}{(n+2)^{100} - n^{100}}.
 \]
\end{problem}
\begin{probsol}
 Z binomické věty platí
 \[
  (n+m)^{100} = \sum_{k=0}^{100} \binom{100}{k} n^{100-k} m^{k}.
 \]
 Je tudíž snadno vidět, že členy $n^{100}$ v se v čitateli i jmenovateli odečtou
 a \uv{nejrychleji rostoucím} členem v čitateli i jmenovateli stane sebe
 $cn^{99}$ pro vhodné $c \in \N$. Konkrétně, v čitateli koeficient $n^{99}$
 vychází
 \[
  \binom{100}{1} \cdot 4^{1} - \binom{100}{1} \cdot 3^{1} = 400 - 300 = 100
 \]
 a ve jmenovateli zkrátka
 \[
  \binom{100}{1} \cdot 2^{1} = 200.
 \]
 Užitím výpočtu v předešedším odstavci získáme úpravou původního výrazu
 \[
  \frac{(n+4)^{100} - (n+3)^{100}}{(n+2)^{100} - n^{100}} = \frac{100n^{99} +
  \sum_{k=2}^{100} \binom{100}{k}(4^{k} - 3^{k})n^{100 - k}}{200n^{99} +
 \sum_{k=2}^{100} \binom{100}{k} 2^{k}n^{100-k}}.
 \]
 Vytčení $n^{99}$ z obou částí zlomku dá
 \[
  \frac{100n^{99} + \sum_{k=2}^{100} \binom{100}{k}(4^{k} - 3^{k})n^{100 -
  k}}{200n^{99} + \sum_{k=2}^{100} \binom{100}{k} 2^{k}n^{100-k}}
  = \frac{n^{99}\left( 100 + \sum_{k=2}^{100} \binom{100}{k}(4^{k}-3^{k})n^{1-k}
  \right)}{n^{99}\left( 200 + \sum_{k=2}^{100} \binom{100}{k} 2^{k}n^{1-k}
  \right)}.
 \]
 Položme
 \begin{align*}
  f(n) & \coloneqq 100 + \sum_{k=2}^{100} \binom{100}{k} (4^{k} -
  3^{k})n^{1-k},\\
  g(n) & \coloneqq 200 + \sum_{k=2}^{100} \binom{100}{k} 2^{k}n^{1-k}.
 \end{align*}
 Nahlédneme, že $\lim_{n \to \infty} f(n) = 100$. Vskutku, z
 \hyperref[thm:aritmetika-limit]{věty o aritmetice limit}, částí (a) a (b),
 platí
 \begin{align*}
  \lim_{n \to \infty} f(n) &= \lim_{n \to \infty} 100 + \sum_{k=2}^{100}
 \binom{100}{k}(4^{k} - 3^{k})n^{1-k} = 100 + \sum_{k=2}^{100} (4^{k}-3^{k})
 \cdot \lim_{n \to \infty} n^{1-k}\\
                           &= 100 + \sum_{k=2}^{100} (4^{k} - 3^{k}) \cdot 0 =
                           100,
 \end{align*}
 kde $\lim_{n \to \infty} n^{1-k} = 0$ pro $k \geq 2$ zřejmě. Podobně bychom
 byli spočetli i $\lim_{n \to \infty} g(n) = 200$.
 \hyperref[thm:aritmetika-limit]{Větou o aritmetice limit}, částí (b) a (c), pak
 spočteme
 \[
  \lim_{n \to \infty} \frac{(n+4)^{100} - (n+3)^{100}}{(n+2)^{100} - n^{100}} =
  \lim_{n \to \infty} \frac{n^{99}}{n^{99}} \cdot \frac{\lim_{n \to \infty}
  f(n)}{\lim_{n \to \infty} g(n)} = 1 \cdot \frac{100}{200} = \frac{1}{2}.
 \]
 Protože výsledkem je reálné číslo, byla \hyperref[thm:aritmetika-limit]{věta o
 aritmetice limit} použita legálně.
\end{probsol}

\begin{problem}{}{limita-odmocniny}
 Spočtěte
 \[
  \lim_{n \to \infty} \sqrt{n^2 + n} - \sqrt{n^2 + 1}.
 \]
\end{problem}
\begin{probsol}
 Zkusili-li bychom spočítat limitu přímo z \hyperref[thm:aritmetika-limit]{věty
 o aritmetice limit} a \myref{lemmatu}{lem:limita-odmocniny}, dostali bychom
 \[
  \lim_{n \to \infty} \sqrt{n^2 + n} - \lim_{n \to \infty} \sqrt{n^2 + 1} =
  \infty - \infty,
 \]
 anžto výraz není definován. Je pročež třeba jej upravit. Využijeme vzorce
 \[
  a^2 - b^2 = (a+b)(a-b).
 \]
 Pro $a \coloneqq \sqrt{n^2+n}$ a $b \coloneqq \sqrt{n^2+1}$ dostaneme
 \[
  (n^2 + n) - (n^2 + 1) = (\sqrt{n^2 + n} + \sqrt{n^2 + 1})(\sqrt{n^2 + n} -
  \sqrt{n^2 + 1}).
 \]
 Zadaný výraz upravíme posléze na
 \[
  \sqrt{n^2+n} - \sqrt{n^2+1} = \frac{(\sqrt{n^2+n} + \sqrt{n^2+1})(\sqrt{n^2+n}
  - \sqrt{n^2+1})}{\sqrt{n^2+n} + \sqrt{n^2+1}} = \frac{n - 1}{\sqrt{n^2 + n} +
 \sqrt{n^2 + 1}}.
 \]
 Vidíme, že nejrychleji rostoucí člen v čitateli je $n$ a ve jmenovateli
 $\sqrt{n^2} = n$. Jejich vytčením získáme
 \[
  \frac{n-1}{\sqrt{n^2 + n}+\sqrt{n^2+1}} = \frac{n}{n} \cdot \frac{1 -
  \frac{1}{n}}{\sqrt{1 + \frac{1}{n}} + \sqrt{1 + \frac{1}{n^2}}} = \frac{1 -
  \frac{1}{n}}{\sqrt{1 + \frac{1}{n}} + \sqrt{1 + \frac{1}{n^2}}}.
 \]
 Byvše zaštítěni \myref{lemmatu}{lem:limita-odmocniny}, nabyli jsme práva
 tvrdit, že
 \[
  \lim_{n \to \infty} \sqrt{1 + \frac{1}{n}} = \sqrt{\lim_{n \to \infty} 1 +
  \frac{1}{n}}.
 \]
 a podobně pro $\lim_{n \to \infty} \sqrt{1 + 1 / n^2}$. Nyní tedy z
 \hyperref[thm:aritmetika-limit]{věty o aritmetice limit}, částí (a) a (c),
 plyne, že
 \[
  \lim_{n \to \infty}
  \frac{1-\frac{1}{n}}{\sqrt{1+\frac{1}{n}}+\sqrt{1+\frac{1}{n^2}}} =
  \frac{1-0}{\sqrt{1+0} + \sqrt{1+0}} = \frac{1}{2}.
 \]
 Výsledkem je reálné číslo, \hyperref[thm:aritmetika-limit]{věta o aritmetice
 limit} byla užita legálně.
\end{probsol}

\begin{problem}{}{dva-straznici-uloha}
 Spočtěte
 \[
  \lim_{n \to \infty} \sqrt[n]{n^2 + 2^{n} + 3^{n}}.
 \]
\end{problem}
\begin{warning}{}{limita-n-te-odmocniny}
 \textbf{Pozor!} Obecně
 \[
  \lim_{n \to \infty} \sqrt[n]{a_n} \neq \sqrt[n]{\lim_{n \to \infty} a_n}.
 \]
 Taková rovnost by ani nedávala žádný smysl, protože ve výraze napravo je
 odmocnina \textbf{vně} limity, přestože závisí na $n$.

 \myref{Lemma}{lem:limita-odmocniny} předpokládá, že $k \in \N$ je
 \textbf{konstantní}, čili nezávisí na $n$. 
\end{warning}

\begin{probsol}[\myref{úlohy}{prob:dva-straznici-uloha}]
 Na první pohled není zřejmé, kterak výraz $\sqrt[n]{n^2 + 2^{n} + 3^{n}}$
 upravit, aby výpočet mohl pokročit, bo \hyperref[thm:aritmetika-limit]{věta o
 aritmetice limit} není v závěsu \myref{varování}{warn:limita-n-te-odmocniny}
 přímo použitelná.

 V případech, kdy člověk nevidí způsob, jak spočítat konkrétní zadanou limitu,
 jesti pleché uchýliti sebe k odhadům zezdola i seshora jinými posloupnosti se
 snadněji určitelnými limitami. Zvoleny-li ony posloupnosti, bychu měly stejnou
 limitu, závěr \myref{lemmatu}{lem:o-dvou-straznicich} dává limitu i
 posloupnosti zadané.

 K volbě vhodných posloupností je však dlužno prve \uv{tipnout} limitu zadaného
 výrazu. Jelikož $3^{n}$ je jistě nejrychleji rostoucí člen dané posloupnosti, a
 $\sqrt[n]{3^{n}} = 3$, zdá se rozumným pokusit se nejprve odhadnout zadaný
 výraz zezdola i seshora posloupnostmi, jejichž limita je $3$.

 Dolní odhad je triviální a v zásadě jsme ho již uvedli. Totiž, jistě platí
 \[
  3^{n} \leq n^2 + 2^{n} + 3^{n},
 \]
 a tedy i
 \[
  \sqrt[n]{3^{n}} \leq \sqrt[n]{n^2 + 2^{n} + 3^{n}}.
 \]
 Zřejmě $\lim_{n \to \infty} \sqrt[n]{3^{n}} = \lim_{n \to \infty} 3 = 3$.

 Snad méně přímočarý jest horní odhad, jenž však plyne z uvědomění, že $3^{n}$
 je nejrychleji rostoucí člen dané posloupnosti. Speciálně máme odhady $n^2 \leq
 3^{n}$ i $2^{n} \leq 3^{n}$. Můžeme pročež pro všechna $n \in \N$ učinit další
 odhad:
 \[
  n^2 + 2^{n} + 3^{n} \leq 3^{n} + 3^{n} + 3^{n} = 3 \cdot 3^{n}.
 \]
 Z \hyperref[thm:aritmetika-limit]{věty o aritmetice limit} potom platí
 \[
  \lim_{n \to \infty} \sqrt[n]{3 \cdot 3^{n}} = \lim_{n \to \infty} \sqrt[n]{3}
  \cdot \sqrt[n]{3^{n}} = \lim_{n \to \infty} \sqrt[n]{3} \cdot \lim_{n \to
  \infty} 3 = 1 \cdot 3 = 3.
 \]
 Fakt, že $\lim_{n \to \infty} \sqrt[n]{3} = 1$ je snadno dokazatelný a onen
 důkaz ponecháme čtenáři jako cvičení.

 Jelikož pro všechna $n \in \N$ platí
 \[
  \sqrt[n]{3^{n}} \leq \sqrt[n]{n^2 + 2^{n} + 3^{n}} \leq \sqrt[n]{3 \cdot
  3^{n}}
 \]
 a již jsme spočetli, že $\lim_{n \to \infty} \sqrt[n]{3^{n}} = \lim_{n \to
 \infty} \sqrt[n]{3 \cdot 3^{n}} = 3$, můžeme prohlásit s použitím
 \myref{lemmatu}{lem:o-dvou-straznicich}, že
 \[
  \lim_{n \to \infty} \sqrt[n]{n^2 + 2^{n} + 3^{n}} = 3.
 \]
\end{probsol}

\begin{exercise}{}{limita-n-te-odmocniny}
 Dokažte, že pro všechna $a \in \R, a > 0$ platí
 \[
  \lim_{n \to \infty} \sqrt[n]{a} = 1.
 \]
\end{exercise}

Úplný závěr sekce věnujeme výpočtu jistých \emph{speciálních} limit, které je
výhodné znát, neboť v tradičních limitních úlohách vyvstávají často. V principu
jde o limity zadané zlomky, u kterých není na první pohled zřejmé, zda roste
rychleji čitatel, či jmenovatel.

Následující tvrzení spolu s \hyperref[thm:aritmetika-limit]{větou o aritmetice
limit} říká v podstatě, že \uv{Každá polynomiální funkce roste pomaleji než
každá funkce exponenciální.}

\begin{lemma}{}{polynomial-over-exponential}
 Platí
 \[
  \lim_{n \to \infty} \frac{n^{k}}{a^{n}} = 0,
 \]
 kdykoli $a > 1$ a $k \in \N$.
\end{lemma}
\begin{lemproof}
 Dokážeme tvrzení nejprve pro $k = 1$.

 Položme $b \coloneqq a - 1$. Potom z binomické věty
 \[
  a^{n} = (1 + b)^{n} = \sum_{i=0}^n \binom{n}{i}b^{i}.
 \]
 Speciálně tedy platí
 \[
  (1 + b)^{n} \geq \binom{n}{2}b^2 = \frac{n(n-1)}{2}b^2,
 \]
 neboť součet výše obsahuje člen napravo a ještě mnoho dalších členů, z nichž
 všechny jsou kladné. Potom ale
 \[
  \frac{n}{a^{n}} = \frac{n}{(1+b)^{n}} \leq \frac{2n}{b^2n(n-1)} =
  \frac{2}{b^2(n-1)}.
 \]
 Snadno vidíme, že platí $n / a^{n} \geq 0$ pro každé $n \in \N$. Máme tudíž
 oboustranný odhad
 \[
  0 \leq \frac{n}{a^{n}} \leq \frac{2}{b^2(n-1)}.
 \]
 Vzhledem k tomu, že
 \[
  \lim_{n \to \infty} \frac{2}{b^2(n-1)} = 0,
 \]
 jest závěrem \myref{lemmatu}{lem:o-dvou-straznicich}, že $\lim_{n \to \infty} n
 / a^{n} = 0$.

 V obecném případě $k \in \N$ stačí položit $b \coloneqq \sqrt[k]{a}$. Potom $b
 > 1$ (protože $a > 1$) a
 \[
  \lim_{n \to \infty} \frac{n^{k}}{a^{n}} = \lim_{n \to \infty} \left(
  \frac{n}{(\sqrt[k]{a})^{n}} \right)^{k} = \lim_{n \to \infty} \left(
  \frac{n}{b^{n}} \right)^{k} = \left( \lim_{n \to \infty} \frac{n}{b^{n}}
  \right)^{k},
 \]
 kde poslední rovnost platí z \hyperref[thm:aritmetika-limit]{věty o aritmetice
 limit}. Podle již dokázané části tvrzení je pravdou, žeť
 \[
  \left( \lim_{n \to \infty} \frac{n}{b^{n}} \right)^{k} = 0^{k} = 0,
 \]
 což bylo dokázati.
\end{lemproof}
\begin{lemma}{}{faktorial-podle-exponenciala}
 Platí
 \[
  \lim_{n \to \infty} \frac{n!}{n^{n}} = 0.
 \]
\end{lemma}
\begin{lemproof}
 Rozložíme výraz následovně:
 \[
  \frac{n!}{n^{n}} = \frac{1}{n} \cdot \prod_{k=2}^n \frac{k}{n}.
 \]
 Pozorujeme, že pro $2 \leq k \leq n$ platí $k / n \leq 1$. Čili,
 \[
  \frac{n!}{n^{n}} = \frac{1}{n} \cdot \prod_{k=2}^n \frac{k}{n} \leq
  \frac{1}{n} \cdot \prod_{k=2}^n 1 = \frac{1}{n}.
 \]
 Dostáváme pro $n \in \N$ odhady
 \[
  0 \leq \frac{n!}{n^{n}} \leq \frac{1}{n}.
 \]
 Jelikož $\lim_{n \to \infty} 0 = \lim_{n \to \infty} 1 / n = 0$, platí z
 \myref{lemmatu}{lem:o-dvou-straznicich} závěr
 \[
  \lim_{n \to \infty} \frac{n!}{n^{n}} = 0,
 \]
 jak jsme chtěli.
\end{lemproof}
\begin{lemma}{}{exponenciala-podle-faktorialu}
 Pro $ a > 1$ platí
 \[
  \lim_{n \to \infty} \frac{a^{n}}{n!} = 0,
 \]
 čili \uv{faktoriál roste rychleji než exponenciála}.
\end{lemma}
\begin{lemproof}
 Nalezněme $m \in \N$ takové, že $m > a$. Rozložíme
 \[
  \frac{a^{n}}{n!} = \frac{a^{m}}{m!} \cdot \prod_{k=m+1}^n \frac{a}{k}.
 \]
 Všimněme sobě, že pro $k > m$ je $a / k < 1$, ježto $m$ bylo zvoleno ostře
 větší než $a$. Speciálně tedy platí
 \[
  \prod_{k=m+1}^n \frac{a}{k} = \frac{a}{n} \cdot \prod_{k=m+1}^{n-1}
  \frac{a}{k} \leq \frac{a}{n},
 \]
 neboť
 \[
  \prod_{k=m+1}^{n-1} \frac{a}{k} \leq \prod_{k=m+1}^{n-1} 1 = 1.
 \]
 Položme $c_m \coloneqq a^{m} / m!$. Číslo $c_m$ je konstantní (vzhledem k $n$),
 neboť $m$ i $a$ jsou. Můžeme odhadnout
 \[
  0 \leq \frac{a^{n}}{n!} \leq \frac{a^{m}}{m!} \cdot \frac{a}{n} = c_m \cdot
  \frac{a}{n}.
 \]
 Z \hyperref[thm:aritmetika-limit]{věty o aritmetice limit} platí
 \[
  \lim_{n \to \infty} c_m \cdot \frac{a}{n} = ac_m \cdot \lim_{n \to \infty}
  \frac{1}{n} = 0.
 \]
 Podle \myref{lemmatu}{lem:o-dvou-straznicich} tudíž máme i
 \[
  \lim_{n \to \infty} \frac{a^{n}}{n!} = 0,
 \]
 což zakončuje důkaz.
\end{lemproof}

Několik limitních cvičení na závěr.

\begin{exercise}{}{odmocnina-z-faktorialu}
 Dokažte, že
 \[
  \lim_{n \to \infty} \sqrt[n]{n!} = \infty.
 \]
 \textbf{Hint}: Rozložte součin $n!$ na dvě poloviny a tu větší zespodu
 odhadněte vhodnou posloupností jdoucí k $\infty$.
\end{exercise}

\begin{exercise}{}{}
 Spočtěte
 \[
  \lim_{n \to \infty} \sqrt{4n^2 - n} - 2n.
 \]
\end{exercise}

\begin{exercise}{}{}
 Spočtěte
 \[
  \lim_{n \to \infty} (-1)^{n}\sqrt{n}(\sqrt{n+1}-\sqrt{n}).
 \]
\end{exercise}

\begin{exercise}{těžké}{}
 Spočtěte limitu posloupnosti $a:\N \to \R$ zadané rekurentním vztahem
 \begin{align*}
  a_0 &\coloneqq 10,\\
  a_{n+1} & \coloneqq 6 - \frac{5}{a_n} \text{ pro } n \in \N.
 \end{align*}
\end{exercise}

