\chapter{Posloupnosti, limity a reálná čísla}
\label{chap:posloupnosti-limity-a-realna-cisla}

Kritickým opěrným bodem při konstrukci reálných čísel i při jejich následném
studiu je pojem \emph{limity} (v češtině se tomuto slovu přiřazuje ženský rod).
Limita je bod, k němuž se zvolená posloupnost čísel \uv{blíží}, ale nikdy jeho
\uv{nedosáhne}, pokud takový existuje. Přidruženým pojmem je třeba
\emph{asymptota} reálné funkce, se kterou se čtenáři, očekáváme, setkali.

Samotná definice limity je zpočátku poněkud neintuitivní. Vlastně i samotná
představa býti něčemu \uv{nekonečně blízko} je do jisté míry cizí. Pokusíme se
vhodnými obrázky a vysvětlivkami cestu k pochopení dláždit, avšak, jakož tomu
bývá, intuice přichází, až člověk s ideou takřkouce sroste.

\section{Definice limity posloupnosti}
\label{sec:definice-limity-posloupnosti}

Koncept posloupnosti je, na rozdíl od limity, velmi triviální. Je to vlastně
\uv{očíslovaná množina čísel}. Z každé množiny lze vyrobit posloupnost jejích
prvků tím, že jim přiřkneme nějaké pořadí. Toto \emph{přiřčení} se nejsnadněji
definuje jako zobrazení z přirozených čísel -- to totiž přesně na každý prvek
kodomény zobrazí jeho pořadí.

\begin{definition}{Posloupnost}{posloupnost}
 Ať $X$ je množina. \emph{Posloupností} prvků z $X$ nazveme libovolné zobrazení
 \[
  a:\N \to X.
 \]
 Pro úsporu zápisu budeme psát $a_n$ místo $a(n)$ pro $n \in \N$. Navíc, je-li
 kodoména $X$ zřejmá z~kontextu, říkáme stručně, že $(a_n)_{n=0}^{\infty}$ je
 \emph{posloupnost.}
\end{definition}

\begin{remark}{}{usporadani-na-N}
 Vnímaví čtenáři sobě jistě povšimli, že jsme na $\N$ nedefinovali žádné
 \emph{uspořádání}. Ačkolivěk není tímto \hyperref[def:posloupnost]{definice
 posloupnosti} formálně nijak postižena, neodpovídá přirozenému vnímání, že
 prvek s číslem $1$ stojí před prvkem s číslem $5$ apod.

 Naštěstí, naše konstruktivní \hyperref[def:prirozena-cisla]{definice
 přirozených čísel} nabízí okamžité řešení. Využijeme toho, že každé přirozené
 číslo je podmnožinou svého následníka, a definujeme zkrátka uspořádání $ \leq $
 na $\N$ předpisem
 \[
  a \leq b \overset{\text{def}}{\iff} a \subseteq b.
 \]
 Fakt, že $ \subseteq $ je uspořádání, okamžitě implikuje, že $ \leq $ je rovněž
 uspořádání.
\end{remark}

Rozmyslíme si nyní dva pojmy pevně spjaté s posloupnostmi -- \emph{konvergence}
a \emph{limita}. Brzo si též ukážeme, že tyto dva pojmy jsou záměnné, ale zatím
je vnímáme odděleně. Navíc, budeme se odteď soustředit speciálně na posloupnosti
racionálních čísel, tj. zobrazení $\N \to \Q$, neboť jsou oním klíčem k
sestrojení své reálné bratří.

Ze všech posloupností $\N \to \Q$ nás zajímá jeden konkrétní typ --
posloupnosti, vzdálenosti mezi jejichž prvky se postupně zmenšují. Tyto
posloupnosti, nazývané \emph{konvergentní} (z lat. con-vergere, \uv{ohýbat k
sobě}), se totiž vždy blíží k nějakému konkrétnímu bodu -- ke své \emph{limitě}.
Představa ze života může být například následující: říct, že se blížíme k
nějakému místu, je totéž, co tvrdit, že se vzdálenost mezi námi a oním místem s
každým dalším krokem zmenšuje. V moment, kdy své kroky směřujeme stále stejným
směrem, posloupnost vzdáleností mezi námi a tím místem tvoří konvergentní
posloupnost. Jestliže se pravidelně odkláníme, k místu nikdy nedorazíme a
posloupnost vzdáleností je pak \emph{divergentní} (tj.
\textbf{ne}konvergentní).

Do jazyka matematiky se věta \uv{vzdálenosti postupně zmenšují} překládá
obtížně. Jeden ne příliš elegantní, ale výpočetně užitečný a celkově oblíbený
způsob je následující: řekneme, že prvky posloupnosti jsou k sobě stále blíž,
když pro jakoukoli vzdálenost vždy dokážeme najít krok, od kterého dál jsou již
k sobě dva libovolné prvky u sebe blíž než tato daná vzdálenost. Důrazně
vyzýváme čtenáře, aby předchozí větu přečítali tak dlouho, dokud jim nedává
dobrý smysl. Podobné formulace se totiž vinou matematickou analýzou a jsou
základem uvažování o nekonečnu.

\begin{definition}{Konvergentní posloupnost}{konvergentni-posloupnost}
 Řekneme, že posloupnost $a:\N \to \Q$ je \emph{konvergentní}, když platí výrok
 \[
  \forall \varepsilon \in \Q, \varepsilon>0 \; \exists n_0 \in \N \; \forall m,n
  \geq n_0: |a_m - a_n| < \varepsilon.
 \]
\end{definition}

\begin{remark}{}{konvergence}
 Aplikujeme intuitivní vysvětlení \emph{zmenšování vzdálenosti} z odstavce nad
 \myref{definicí}{def:konvergentni-posloupnost} na jeho skutečnou definici.

 Výrok
 \[
  \forall \varepsilon \in \Q, \varepsilon>0 \; \exists n_0 \in \N \; \forall m,n
  \geq n_0: |a_m - a_n| < \varepsilon
 \]
 říká, že pro jakoukoli vzdálenost ($\varepsilon$) dokáži najít krok ($n_0$)
 takový, že vzdálenost dvou prvků v~libovolných dvou následujících krocích
 ($m,n$) už je menší než daná vzdálenost ($|a_n -a_m|<\varepsilon$).

 Slovo \uv{krok} je třeba vnímat volně -- myslíme pochopitelně \emph{pořadí} či
 \emph{indexy} prvků v posloupnosti. Pohled na racionální posloupnosti jako na
 \uv{kroky} činěné v racionálních číslech může být ovšem užitečný.
\end{remark}

\begin{exercise}{}{}
 Dokažte, že posloupnost $a:\N \to \Q$ je konvergentní právě tehdy, když
 \[
  \forall \varepsilon>0 \; \exists n_0 \in \N \; \forall m,n \geq n_0: |a_m -
  a_n| < C\varepsilon
 \]
 pro libovolnou \textbf{kladnou} konstantu $C \in \Q$.
\end{exercise}

Pojem \emph{limity}, představuje jakýsi bod, k němuž se posloupnost s každým
dalším krokem přibližuje, je vyjádřen výrazem podobného charakteru. Zde však
přichází na řadu ona \emph{děravost} racionálních čísel. Může se totiž stát, a
příklady zde uvedeme, že limita racionální posloupnosti není racionální číslo.

Učiňmež tedy dočasný obchvat a před samotnou definicí limity vyrobme reálná
čísla jednou z~přehoušlí možných cest. 

Ať $\mathcal{C}(\Q)$ značí množinu všech \textbf{konvergentních }racionálních
posloupností. Uvažme ekvivalenci $ \simeq $ na $\mathcal{C}(\Q)$ danou
\[
 a \simeq b \overset{\text{def}}{\iff} \forall \varepsilon>0 \; \exists n_0 \in \N
 \; \forall n \geq n_0: |a_n - b_n| < \varepsilon.
\]
Přeloženo do člověčtiny, $a \simeq b$, právě když se rozdíl mezi prvky těchto
posloupností se stejným pořadím neustále zmenšuje -- řekli bychom, že se
\emph{blíží k nule}. V rámci naší (zatím intuitivní) představy, že konvergentní
posloupnosti se blíží k nějakému bodu, dává smysl ztotožňovat posloupnosti,
které se blíží k bodu \emph{stejnému} -- stav, který vyjadřujeme tak, že se
jejich rozdíl blíží k nule.

Ve výsledku budeme definovat reálná čísla jako limity všech možných
konvergentních racionálních posloupností. Ježto však pozbýváme aparátu, abychom
koncepty limity a konvergence stmelili v~jeden, jsme nuceni učinit mezikrok.

\begin{definition}{Reálná čísla}{realna-cisla}
 Množinu \emph{reálných čísel} tvoří všechny třídy ekvivalence konvergentních
 racionálních posloupností podle $ \simeq $. Symbolicky,
 \[
  \R \coloneqq \{[a]_{ \simeq } \mid a \in \mathcal{C}(\Q)\}.
 \]
\end{definition}

Nyní definujeme pojem limity. Nemělo by snad být příliš překvapivé, že se od
\hyperref[def:konvergentni-posloupnost]{definice konvergence} příliš neliší.
Významný rozdíl odpočívá pouze v předpokladu existence \emph{cílového bodu}.

\begin{definition}{Limita posloupnosti}{limita-posloupnosti}
 Ať $a:\N \to \Q$ je posloupnost. Řekneme, že $a$ \emph{má limitu} $L \in \R$,
 když
 \[
  \forall \varepsilon \in \Q,\varepsilon>0 \; \exists n_0 \in \N \; \forall n
  \geq n_0: |a_n - L|<\varepsilon,
 \]
 neboli, když jsou prvky $a_n$ bodu $L$ s každým krokem stále blíž.

 Fakt, že $L \in \R$ je limitou $a$ značíme jako $\lim_{} a = L$.
\end{definition}

\section{Limity konvergentních posloupností}
\label{sec:limity-konvergentnich-posloupnosti}

V této sekci dokážeme, že konvergentní posloupnosti mají limitu. Opačná
implikace, tj. že posloupnosti majíce limitu konvergují, je téměř triviální.
Potřebujeme dokázat jen jednu vlastnost absolutní hodnoty.

\begin{lemma}{Trojúhelníková nerovnost}{trojuhelnikova-nerovnost}
 Ať $x,y \in \Q$. Pak
 \[
  |x + y| \leq |x| + |y|.
 \]
\end{lemma}
\begin{lemproof}
 Absolutní hodnota $|x+y|$ je rovna buď $x + y$ (když $x+y \geq 0$) nebo $-x-y$
 (když $x+y<0$). Zřejmě $x \leq |x|$ a $-x \leq |x|$, podobně $y \leq |y|$ a
 $-y \leq |y|$.

 Pak je ale $x + y \leq |x| + |y|$ a též $-x+(-y) \leq |x| + |y|$. Tím je důkaz
 hotov.
\end{lemproof}

Trojúhelníková nerovnost poskytuje snadné důkazy mnoha užitečných dílčích
tvrzení o posloupnostech. Příkladem je následující cvičení.

\begin{exercise}{Jednoznačnost limity}{jednoznacnost-limity}
 Dokažte, že každá posloupnost $a:\N \to \Q$ má nejvýše jednu limitu. Hint:
 použijte \hyperref[lem:trojuhelnikova-nerovnost]{trojúhelníkovou nerovnost}.
\end{exercise}

Ježto bychom však rádi dokazovali všechna tvrzení již pro reálná čísla, ukažme
si nejprve, jak se dají sčítat a násobit. Dokážeme rovněž, že $\R$ -- stejně
jako $\Q$ -- tvoří těleso. Začneme tím, že se naučíme sčítat a násobit
konvergentní posloupnosti.

Ať $x,y \in \mathcal{C}(\Q)$ jsou dvě konvergentní racionální posloupností.
Operace $+$ a $ \cdot $ na $\mathcal{C}(\Q)$ definujeme velmi přirozeně.
Zkrátka, $(x+y)(n) \coloneqq x(n) + y(n)$ a $(x \cdot y)(n) \coloneqq x(n) \cdot
y(n)$, tj. prvek na místě $n$ posloupnosti $x+y$ je součet prvků na místech $n$
posloupností $x$ a $y$. Abychom ovšem získali skutečně operace na
$\mathcal{C}(\Q)$, musíme ověřit, že $x+y$ i $x \cdot y$ jsou konvergentní.

Nechť dáno jest $\varepsilon>0$. Chceme ukázat, že umíme najít $n_0 \in \N$, aby
\[
 |(x_n + y_n) - (x_m + y_m)| < \varepsilon,
\]
kdykoli $m,n \geq n_0$. Protože jak $x$ tak $y$ konverguje, již umíme pro
libovolná $\varepsilon_x,\varepsilon_y>0$ najít $n_x$ a $n_y$ taková, že $|x_n -
x_m| < \varepsilon_x$, kdykoli $m,n \geq n_x$, a podobně $|y_n -
y_m|<\varepsilon_y$, kdykoli $m,n \geq n_y$. Položme tedy $\varepsilon_x =
\varepsilon_y \coloneqq \varepsilon / 2$ a $n_0 \coloneqq \max(n_x,n_y)$. Potom
můžeme užitím \hyperref[lem:trojuhelnikova-nerovnost]{trojúhelníkové nerovnosti}
odhadnout
\[
 |(x_n + y_n) - (x_m + y_m)| = |(x_n - x_m) + (y_n - y_m)| \leq |x_n - x_m| +
 |y_n - y_m| < \varepsilon_x + \varepsilon_y = \varepsilon,
\]
čili $x + y$ konverguje.

Předchozí odstavec se může snadno zdát šílenou směsicí symbolů. Ve skutečnosti
však formálně vykládá triviální úvahu. Máme najít pořadí, od kterého jsou prvky
součtu $x + y$ u sebe blíž než nějaká daná vzdálenost. Poněvadž $x$ i $y$
konvergují, stačí přeci vzít větší z pořadí, od kterých je jak rozdíl prvků $x$,
tak rozdíl prvků $y$ menší než polovina dané vzdálenosti.

Velmi obdobnou manipulaci lze provést k důkazu konvergence $x \cdot y$.
Ponecháváme jej čtenářům jako (ne zcela snadné) cvičení.

\begin{exercise}{}{}
 Dokažte, že jsou-li $x,y$ konvergentní posloupnosti racionálních čísel, pak je
 posloupnost $x \cdot y$ rovněž konvergentní.
\end{exercise}

Racionální čísla jsou přirozeně součástí reálných prostřednictvím zobrazení
\begin{equation}
 \label{eq:Q-into-R}
 \begin{split}
  \xi: \Q &\hookrightarrow \R,\\
  q &\mapsto [(q)],
 \end{split}
\end{equation}
kde $(q)$ značí posloupnost $n \mapsto q$ pro všechna $n \in \N$ a $[(q)]$ její
třídu ekvivalence podle $ \simeq $.

\begin{warning}{}{}
 Tvrdíme pouze, že $\Q$ jsou \emph{součástí} $\R$, kde slovu \emph{součást}
 záměrně není dán rigorózní smysl. Racionální čísla totiž (aspoň po dobu naší
 dočasné \hyperref[def:realna-cisla]{definice reálných čísel}) nejsou v žádném
 smyslu podmnožinou čísel reálných.

 Matematici ale často ztotožňujeme doménu prostého zobrazení s jeho obrazem
 (neboť mezi těmito množinami vždy existuje bijekce). V tomto smyslu mohou být
 $\Q$ vnímána jako podmnožina $\R$, ztotožníme-li racionální čísla s obrazem
 zobrazení $\xi$ z \eqref{eq:Q-into-R}. Toto ztotožnění znamená vnímat
 racionální číslo $q \in \Q$ jako konvergentní posloupnost samých čísel $q$.
\end{warning}

\begin{exercise}{}{}
 Dokažte, že zobrazení $\xi$ z \eqref{eq:Q-into-R} je
 \begin{itemize}
  \item dobře definované -- tzn. že když $p = q$, pak $[(p)] = [(q)]$ -- a
  \item prosté.
 \end{itemize}
\end{exercise}

Jelikož $\Q$ je těleso, speciálně tedy obsahuje $0$ a $1$, $\R$ je
(prostřednictvím $\xi$ z \eqref{eq:Q-into-R}) obsahuje rovněž. Pro stručnost
budeme číslem $0 \in \R$ značit třídu ekvivalence posloupnosti samých nul a
číslem $1 \in \R$ třídu ekvivalence posloupnosti samých jednotek. Ověříme, že se
skutečně jedná o neutrální prvky ke sčítání a násobení.

Je třeba si rozmyslet, že pro každou posloupnost $x \in \mathcal{C}(\Q)$ platí
$x + 0 = x$ a $x \cdot 1 = x$, kde, opět, čísla $0$ a $1$ ve skutečnosti
znamenají nekonečné posloupnosti těchto čísel. Obě rovnosti jsou však zřejmé z
definice, neboť $(x+0)(n) = x_n + 0 = x_n = x(n)$ a $(x \cdot 1)(n) = x_n \cdot
1 = x_n = x(n)$ pro všechna $n \in \N$.

Konečně, rozšíříme rovněž $-$ a $^{-1}$ na $\R$. Pro libovolnou posloupnost $x
\in \mathcal{C}(\Q)$ definujeme zkrátka $(-x)(n) \coloneqq -x(n)$. S $^{-1}$ je
situace lehce komplikovanější. Totiž, pouze \textbf{nenulová} racionální čísla
mají svůj inverz k násobení. Zde je třeba zpozorovat, že \textbf{konvergentní}
posloupnost, která by však měla nekonečně mnoho prvků nulových, už musí mít od
nějakého kroku \textbf{všechny} prvky nulové, jinak by totiž nemohla
konvergovat. Vskutku, představme si, že $x$ je posloupnost taková, že $x_n = 0$
pro nekonečně mnoho přirozených čísel $n \in \N$. Pak ale ať zvolím $n_0 \in \N$
jakkoliv, vždy existuje $m \geq n_0$ takové, že $x_m = 0$. Vezměme $n \geq n_0$
libovolné. Pokud $x_n \neq 0$, pak můžeme vzít třeba $\varepsilon \coloneqq
|x_n| / 2$ a bude platit, že $|x_n - x_m| > \varepsilon$, což je dokonalý zápor
\hyperref[def:konvergentni-posloupnost]{definice konvergence}. Z toho plyne, že
$x_n$ musí být $0$ pro $n \geq n_0$ a odtud dále, že $x \simeq 0$. Čili, pouze
nulové posloupnosti v $\R$ nemají inverz vzhledem k $ \cdot $.

Právě provedená úvaha nám umožňuje definovat $^{-1}$ pro posloupnosti $x \in
\mathcal{C}(\Q)$ takové, že $x \not\simeq 0$, následovně:
\[
 (x^{-1})(n) \coloneqq \begin{cases}
  x(n)^{-1},& \text{když } x(n) \neq 0,\\
  0, &\text{když } x(n) = 0.
 \end{cases}
\]
 
Je snadné uvidět, že $-x$ je inverzem k $x$ vzhledem k $+$ a $x^{-1}$ je
inverzem k $x \neq 0$ vzhledem k $ \cdot $. Vskutku, máme
\[
 (x + (-x))(n) = x_n + (-x_n) = 0,
\]
tedy v tomto případě je $(x + (-x))$ přímo \textbf{rovna} nulové posloupnosti. V
případě $^{-1}$ dostáváme pro $x \not\simeq 0$
\[
 (x \cdot x^{-1})(n) = \begin{cases}
  x_n \cdot x_n^{-1} = 1,& \text{když } x_n \neq 0,\\
  x_n \cdot 0 = 0,& \text{když } x_n = 0.
 \end{cases}
\]
Ergo, $x \cdot x^{-1}$ je rovna posloupnosti samých jedniček až na konečně mnoho
nul, protože, jak jsme si již rozmysleli, $x$ nemůže mít nekonečně $0$ a zároveň
nebýt v relaci $ \simeq $ s nulovou posloupností, jinak by nebyla konvergentní.

Shrneme-li řád předchozích úvah, získáme oprávnění tvrdit, že
\[
 (\R,+,-,[(0)], \cdot,^{-1},[(1)])
\]
je těleso. Tento fakt je do budoucna pochopitelně zásadní; teď se však můžeme
těšit znalostí, že jsme přechodem od $\Q$ k $\R$ neztratili symetrické rysy
původní množiny.

Přikročmež již však k důkazu existence limity každé konvergentní posloupnosti.
Fakt, že existence limity implikuje konvergenci, plyne přímo z
\hyperref[lem:trojuhelnikova-nerovnost]{trojúhelníkové nerovnosti}.

\begin{lemma}{}{}
 Každá posloupnost majíc limitu je konvergentní.
\end{lemma}
\begin{lemproof}
 Ať $a:\N \to \Q$ je posloupnost s limitou $L$. Pak pro každé $\varepsilon_L>0$
 existuje $n_L \in \N$ takové, že $|a_n - L| < \varepsilon_L$ pro všechna $n
 \geq n_L$.

 Ať je dáno $\varepsilon>0$. Chceme ukázat, že $|a_m - a_n| < \varepsilon$ pro
 všechna $m,n$ větší než vhodné $n_0 \in \N$. Položme tedy $n_0 \coloneqq n_L$ a
 $\varepsilon_L \coloneqq \varepsilon / 2$. Potom pro všechna $m,n \geq n_0 =
 n_L$ máme
 \[
  |a_m - a_n| = |a_m - a_n - L + L| = |(a_n - L) + (L - a_m)| \leq |a_n - L| +
  |L - a_m| < \varepsilon_L + \varepsilon_L = \varepsilon,
 \]
 čili $a$ konverguje.
\end{lemproof}

\subsection{Úplnost reálných čísel}
\label{ssec:uplnost-realnych-cisel}

K důkazu existence limity každé konvergentní posloupnosti potřebujeme
prozpytovat vztah racionálních a reálných čísel podrobněji. Konkrétně
potřebujeme ukázat, že $\Q$ jsou tzv. \emph{hustá} v $\R$, tj. že ke každému
reálnému číslu existuje racionální číslo, které je mu nekonečně blízko. Zde jsme
opět implicitně ztotožnili racionální čísla s třídami ekvivalence konstantních
posloupností. Na základě toho budeme totiž moci tvrdit, že reálná čísla jsou
tzv. \emph{kompletní}, což přesně znamená, že každá konvergentní posloupnost
reálných čísel má reálnou limitu.

Nejprve si ovšem musíme rozmyslet, co vlastně míníme posloupností
\emph{reálných} čísel. Pochopitelně, zobrazení $x:\N \to \R$ poskytuje validní
definici, ale uvědomme si, že teď vlastně uvažujeme posloupnosti, jejichž prvky
jsou třídy ekvivalence konvergentních racionálních posloupností.

Abychom směli hovořit o konvergentních \emph{reálných} posloupnostech, rozšíříme
absolutní hodnotu $| \cdot |$ z $\Q$ na $\R$ zkrátka předpisem $|[(x_n)]|
\coloneqq [(|x_n|)]$. Napíšeme-li tedy $|x| \leq K$ pro reálná čísla $x,K \in
\R$, pak tím doslova myslíme $[(|x_n|)] \leq [(K_n)]$, tj. $|x_n| \leq K_n$ pro
všechna $n \in \N$, kde $x_n,K_n$ jsou nyní již čísla ryze rozumná čili
racionální.

Rozepíšeme-li si tedy podrobně, co znamená, že je posloupnost $x:\N \to \R$
konvergentní, dostaneme pro dané $\varepsilon>0$, vhodné $n_0 \in \N$ a $m,n
\geq n_0$ nerovnost $|x_n - x_m| < \varepsilon$. Ovšem, $x_n$ i $x_m$ jsou samy
o sobě \textbf{posloupnosti} racionálních čísel, tedy poslední nerovnost plně
rozepsána dí
\[
 |(x_n)_{k} - (x_m)_{k}| < \varepsilon \; \forall k \in \N.
\]



\section{Poznatky o limitách posloupností}
\label{sec:poznatky-o-limitach-posloupnosti}

Účelem této sekce je shrnout základní poznatky o limitách posloupností, jež
umožní čtenářům limity konkrétních posloupností efektivně počítat a navíc
široké jejich použití v následujících kapitolách.

Začneme jedním z nejdůležitějších a dle našeho názoru též nejkrásnějších
výsledků -- tzv. Bolzano\-vou-Weierstraßovou větou. Ta tvrdí v podstatě toto:
mám-li omezenou posloupnost, pak z ní již umím vybrat nekonečně mnoho prvků,
které tvoří posloupnost \emph{konvergentní}.

Ona krása takového tvrzení spočívá v principu, kterým se podrobně zabývá
kombinatorická disciplína zvaná
\href{https://en.wikipedia.org/wiki/Ramsey_theory}{Ramseyho teorie}; v principu,
že v téměř libovolně chaotické struktuře lze nalézt řád, jakmile jest tato
dostatečně velká. Nejedná se jistě o čistě matematický princip, nýbrž dost možná
o princip vzniku vesmíru a života, popsaný již starým Aristotelem ve výmluvném
výroku, \uv{Celek je více než součet svých částí.} V mnoha zpytech se tomuto
jevu přezdívá
\href{https://www.sciencedirect.com/topics/computer-science/emergent-behavior}{Emergent
Behavior} a představuje stav, kdy chování systému nelze plně popsat pouze
studiem jeho jednotlivých prvků.

Pro důkaz Bolzanovy-Weierstraßovy věty potřebujeme jedné pomocné konstrukce,
tzv. \emph{systému vnořených intervalů}. Nejprve si však pořádně definujeme
samotný pojem \emph{intervalu}. K tomu se nám bude hodit rozšířit množinu
reálných čísel o prvky $-\infty$ a $\infty$.

\subsection{Rozšířená reálná osa}
\label{ssec:rozsirena-realna-osa}

\begin{definition}{Rozšířená reálná osa}{rozsirena-realna-osa}
 Definujme množinu $\R^{*} \coloneqq \R \cup \{-\infty,\infty\}$, kde $\infty$,
 resp. $-\infty$, je z definice prvek takový, že $\infty \geq x$, resp.
 $-\infty \leq x$, pro každé $x \in \R$. Množině $\R^{*}$ budeme někdy říkat
 \emph{rozšířená reálná osa}. Rozšíříme rovněž operace $+$ a $ \cdot $ na prvky
 $\infty$ a $-\infty$ následovně.
 \begin{align*}
  \infty + a = a + \infty = \infty,& \quad \text{pro }a \in \R \cup
  \{\infty\},\\
  -\infty + a = a + (-\infty) = -\infty,& \quad \text{pro }a \in \R \cup
  \{-\infty\},\\
  \infty \cdot a = a \cdot \infty = \infty,& \quad \text{pro }a > 0 \text{ nebo
  }a = \infty,\\
  \infty \cdot a = a \cdot \infty = -\infty,& \quad \text{pro }a < 0 \text{ nebo
  }a = -\infty,\\
  -\infty \cdot a = a \cdot (-\infty) = -\infty,& \quad \text{pro }a > 0 \text{
  nebo }a = \infty,\\
  -\infty \cdot a = a \cdot (-\infty) = \infty,& \quad \text{pro }a < 0 \text{
  nebo }a = -\infty,\\
  a \cdot \infty^{-1} = a \cdot (-\infty)^{-1} = 0,& \quad \text{pro }a \in \R.
 \end{align*}
\end{definition}

\begin{warning}{}{pocitacni-s-nekonecnem}
 \myref{Definice}{def:rozsirena-realna-osa} stručně řečeno říká, že se s prvky
 $\infty$ a $-\infty$ zachází podobně jako s ostatními reálnými čísly. Ovšem,
 následující operace zůstávají nedefinovány.
 \[
  \infty + (-\infty), -\infty + \infty, \pm \infty \cdot 0, 0 \cdot ( \pm
  \infty), ( \pm \infty) \cdot ( \pm \infty)^{-1}.
 \]
\end{warning}

Čtenáři možná zpozorovali, že jsme při své
\hyperref[def:limita-posloupnosti]{definici limity} nerozlišili mezi
posloupnostmi, které nemají limitu, protože jejich prvky \uv{skáčou sem a tam},
a posloupnostmi, které ji nemají naopak pro to, že \uv{stále klesají či
stoupají}. Pro další studium záhodno se tohoto nedostatku zlišit.

\begin{definition}{Limita v nekonečnu}{limita-v-nekonecnu}
 Ať $x:\N \to \R$ je reálná posloupnost. Řekneme, že $x$ má limitu $\infty$,
 resp. $-\infty$, když pro každé $K > 0, K \in \R$, existuje $n_0 \in \N$
 takové, že pro všechna $n \geq n_0$ platí $x_n > K$, resp. $x_n < -K$. Píšeme
 $\lim_{} x = \infty$, resp. $\lim_{} x = -\infty$.
\end{definition}

Na reálných číslech existuje uspořádání $ \leq $, které zdělila z čísel
přirozených, prostřednictvím čísel celých a konečně čísel racionálních. Protože,
vděkem naší konstrukci, jsou celá čísla třídy ekvivalence dvojic čísel
přirozených, čísla racionální třídy ekvivalence dvojic čísel celých a čísla
reálná limity konvergentních racionálních posloupností, bylo by vskutku obtížné
a neproduktivní vypsat konkrétní množinovou definici tohoto uspořádání na
reálných číslech. Přidržíme se pročež intuitivního pohledu na věc a důkaz, že
$ \leq $ je skutečně uspořádání na reálných číslech, necháváme laskavému čtenáři
k promyšlení.

Existence uspořádání umožňuje dívat se na množiny z jistého \uv{souvislého}
pohledu. Nemusejí již být vňaty (jako tomu je u ostatních představených
číselných okruhů) jako výčty jednotlivých prvků, ale oprávněně jako
\uv{provázky} či \uv{úsečky}. \hyperref[cor:r-jsou-uplna]{Úplnost reálných
čísel} zaručuje, že z každého reálného čísla mohu plynule dorazit do každého
jiné reálného čísla aniž reálná čísla opustím.

Předchozí odstavec vágně motivuje definici \emph{intervalu} -- \uv{souvislé}
omezené podmnožiny reálných čísel. Zároveň s definicí intervalu vzniká i pojem
\emph{otevřenosti} a \emph{uzavřenosti} podmnožiny reálných čísel -- pojem,
který je klíčem k definici \emph{topologie} na obecné množině a tím pádem
vlastně i základem tak zhruba poloviny celé moderní matematiky.

Směrem k definici intervalu učiňmež koliksi mezikroků.

\begin{definition}{Maximum a minimum}{maximum-a-minimum}
 Ať $X \subseteq \R$ je množina. Řekneme, že prvek $M \in X$, resp. $m \in X$,
 je \emph{maximem}, resp. \emph{minimem}, množiny $X$, když pro každé $x \in X$
 platí $x \leq M$, resp. $x \geq m$. Píšeme $M = \max X$, resp. $m = \min X$.
\end{definition}

\begin{definition}{Horní a dolní závora}{horni-a-dolni-zavora}
 Ať $X \subseteq \R$ je množina. Řekneme, že prvek $Z \in \R^{*}$ resp. $z \in
 \R^{*}$, je \emph{horní}, resp. \emph{dolní}, \emph{závora} množiny $X$, když
 pro každé $x \in X$ platí $x \leq Z$, resp. $x \geq z$.

 Má-li množina $X$ horní, resp. dolní, závoru, \textbf{která leží v $\R$} (tedy
 není rovna $ \pm \infty$), říkáme, že je \emph{shora}, resp. \emph{zdola},
 \emph{omezená}. Je-li navíc $X$ omezená shora i zdola, říkáme krátce, že je
 \emph{omezená}.
\end{definition}

\begin{definition}{Supremum a infimum}{supremum-a-infimum}
 Ať $X \subseteq \R$ je množina. Řekneme, že prvek $S \in \R^{*}$, resp. $i \in
 \R^{*}$, je \emph{supremum}, resp. \emph{infimum}, množiny $X$, když je to její
 \emph{nejmenší horní závora}, resp. \emph{největší dolní závora}. Píšeme $S =
 \sup X$, resp. $i = \inf X$.

 Vyjádřeno symbolicky, prvek $S \in \R$ je \emph{supremem} množiny $X$, když $x
 \leq S$ pro všechna $x \in X$, a kdykoli $x \leq Z$ pro nějaký prvek $Z \in \R$
 a všechna $x \in X$, pak $S \leq Z$. Prvek $i \in \R$ je \emph{infimem} množiny
 $X$, když $x \geq i$ pro všechna $x \in X$, a kdykoli $x \geq z$ pro nějaký
 prvek $z \in \R$ a všechna $x \in X$, pak $i \geq z$.
\end{definition}

\begin{warning}[topsep at break=0pt]{}{supremum-vs-maximum}
 Vřele radíme čtenářům, aby sobě bedlivě přečetli předchozí tři definice a
 uvědomili si -- velmi zásadní, leč lehko přehlédnuté -- jejich vzájemné
 rozdíly.
 \begin{itemize}
  \item Maximum a minimum množiny $X$ je z
   \hyperref[def:maximum-a-minimum]{definice} \textbf{vždy prvkem této množiny}.
   Maximem množiny $\{1,2,3\}$ je prvek $3$ a jeho minimem je prvek $1$.
  \item Horní, resp. dolní, závora množiny $X$ je \textbf{libovolné
   \clr{rozšířené} reálné číslo} (tedy klidně i $ \pm \infty$), které je větší,
   resp. menší, než všechny prvky $X$. Horní závorou množiny $\{1,2,3\}$ je
   číslo $69$, též $\infty$ a též číslo $3$. Horní a dolní závora \textbf{může,
   ale nemusí}, být prvkem $X$.
  \item Supremum, resp. infimum, množiny $X$ je \textbf{rozšířené reálné číslo},
   které je větší, resp. menší, než všechny prvky $X$, ale \textbf{zároveň
   menší, resp. větší, než každá jeho horní, resp. dolní, závora}. Supremum a
   infimum \textbf{může, ale nemusí, ležet v množině $X$}. Touto vlastností se
   přesně rozlišují \emph{uzavřené} a \emph{otevřené} intervaly -- interval je
   uzavřený, když jeho supremum v~něm leží, kdežto otevřený, když nikoliževěk.
   Supremem množiny $\{1,2,3\}$ je číslo $3$ a jeho infimem je číslo $1$.
 \end{itemize}
 Daná podmnožina $X \subseteq \R$ \textbf{nemusí nutně mít maximum a minimum},
 ale, a to si dokážeme, jest-li shora, resp. zdola, omezená, \textbf{pak má
 nutně supremum, resp. infimum}.
\end{warning}

\begin{exercise}{}{sup-inf-prazdne-mnoziny}
 Určete z \hyperref[def:supremum-a-infimum]{definice suprema a infima} $\inf
 \emptyset$ a $\sup \emptyset$.
\end{exercise}

\begin{exercise}{}{sup-inf-jednoznacne}
 Dokažte, že $\sup X$ a $\inf X$ jsou určeny jednoznačně.
\end{exercise}

\subsubsection{Axiomatická definice reálných čísel}
\label{sssec:axiomaticka-definice-realnych-cisel}

Přestože jsme konstrukci reálných čísel úspěšně dokončili použitím
konvergentních racionálních posloupností, stojí snad za zmínku i jejich
axiomatická definice, která se obvykle uvádí v úvodních učebnicích matematické
analýzy.

Překvapivě není v principu tak odlišná od jejich konstrukce, kromě jednoho
konkrétního axiomu, jenž právě zaručuje úplnost; není z něj však vůbec na první,
v zásadě ani na druhý, pohled vidno, že takovou vlastnost skutečně implikuje.

\begin{definition}{Axiomatická definice reálných
 čísel}{axiomaticka-definice-realnych-cisel}
 Množina $\R$ se v zásadě definuje jako nekonečné uspořádané těleso s vlastností
 úplnosti. Tedy,
 \begin{itemize}
  \item existují prvky $0,1 \in \R$ a operace $+, \cdot :\R^{2} \to \R$ s
   inverzy $-,^{-1}:\R \to \R$ takové, že
   \[
    (\R,+,-,0, \cdot ,^{-1},1)
   \]
   je nekonečné těleso;
  \item existuje uspořádání $ \leq $ na $\R$, které je lineární (každé dva prvky
   lze spolu porovnat);
  \item (\textbf{axiom úplnosti}) každá shora omezená podmnožina $\R$ má
   supremum.
 \end{itemize}
\end{definition}

Je to právě on poslední axiom v
\hyperref[def:axiomaticka-definice-realnych-cisel]{předchozí definici}, jehož
použití jsme se chtěli vyhnout, bo dohlédnout jeho hloubky je obtížné a
neintuitivní.

Dokážeme si zde ovšem, že naše \hyperref[def:realna-cisla]{definice reálných
čísel} odpovídá jejich axiomatické. Otázky nekonečnosti, podmínek tělesa i
uspořádání jsme již zodpověděli. Zbývá dokázat axiom úplnosti.

K tomu potřebujeme pouze jedno pomocné lemma a definici -- a to konkrétně
\emph{monotónní} posloupnosti. Pěstujeme víru, že ctění čtenáři se s tímto
pojmem již setkali ve spojitosti s funkcemi. Totiž \emph{monotónní}
posloupnosti jsou posloupnosti takové, že velikost jejich prvků buď klesá
(případně jen neroste), či roste (případně jen neklesá). Je přirozené myslet si,
že posloupnosti, které jsou shora omezené a rostou, musejí konvergovat. Je tomu
vskutku tak.

\begin{definition}{Monotónní posloupnost}{monotonni-posloupnost}
 O posloupnosti $x:\N \to \R$ řekneme, že je
 \begin{itemize}
  \item \emph{rostoucí}, když $x_{n+1}>x_n \; \forall n \in \N$;
  \item \emph{klesající}, když $x_{n+1} < x_n \; \forall n \in \N$;
  \item \emph{neklesající}, když $x_{n+1} \geq x_n \; \forall n \in \N$;
  \item \emph{nerostoucí}, když $x_{n+1} \leq x_n \; \forall n \in \N$.
 \end{itemize}
 Ve všech těchto případech díme, že posloupnost $x$ je \emph{monotónní}.
\end{definition}

\begin{lemma}{Limita monotónní posloupnosti}{limita-monotonni-posloupnosti}
 \vspace*{-\parskip}
 \begin{enumerate}[label=(\alph*)]
  \item Každá rostoucí nebo neklesající shora omezená posloupnost je
   konvergentní.
  \item Každá klesající nebo nerostoucí zdola omezená posloupnost je
   konvergentní.
 \end{enumerate}
\end{lemma}
\begin{lemproof}
 Dokážeme pouze část (a), část (b) je ponechána jako cvičení.

 Ať 
\end{lemproof}


\begin{proposition}{Axiom úplnosti}{axiom-uplnosti}
 Ať $X \subseteq \R$ je shora omezená množina. Pak existuje $\sup X$.
\end{proposition}
\begin{propproof}
 Ježto naše \hyperref[cor:r-jsou-uplna]{pojetí úplnosti} se překládá do znění,
 \uv{Každá konvergentní posloupnost má limitu}, není snad nečekané, že se důkaz
 \emph{axiomu úplnosti} o tuto vlastnost opírá.

 Je-li $X$ prázdná, pak má supremum podle
 \myref{cvičení}{exer:sup-inf-prazdne-mnoziny}. Ať je tedy $X$ neprázdná a shora
 omezená a $Z \in \R$ je libovolná horní závora $X$. Protože $X$ je neprázdná,
 existuje $q \in \R$ takové, že $q < x$ pro nějaké $x \in X$. Definujeme
 posloupnosti $Z_n$ a $q_n$ podle následujících pravidel.
 \begin{itemize}
  \item Položme $Z_0 \coloneqq Z$ a $q_0 \coloneqq q$.
  \item Uvažme číslo $p_n \coloneqq (Z_n + q_n) / 2$.
  \item Je-li $p_n$ horní závorou $X$, položme $Z_{n+1} \coloneqq p_n$ a
   $q_{n+1} \coloneqq q_n$.
  \item Není-li $p_n$ horní závorou $X$, položme $Z_{n+1} \coloneqq Z_n$ a
   $q_{n+1} \coloneqq p_n$.
 \end{itemize}
 Pak jsou posloupnosti $Z_n$ a $q_n$ konvergentní (\textbf{proč?}) a indukcí lze
 snadno dokázat (\textbf{dokažte!}), že $q_n$ \textbf{není} horní závorou $X$ a
 $Z_n$ \textbf{je} horní závorou $X$ pro všechna $n \in \N$. Navíc platí
 $\lim_{} |Z_n - q_n| = 0$ (\textbf{proč?}), a tedy $\lim_{} Z_n = \lim_{} q_n$.

 Označme $S \coloneqq \lim_{} Z_n = \lim_{} q_n$. Dokážeme, že $S = \sup X$. Je
 třeba ukázat, že
 \begin{enumerate}
  \item $S$ je horní závorou $X$;
  \item $S$ je nejmenší horní závorou.
 \end{enumerate}
 Předpokládejme pro spor, že existuje $x \in X$ takové, že $x > S$. To znamená,
 že existuje konstanta $c > 0$ taková, že $x - S = c$. Volme $\varepsilon
 \coloneqq c / 2$. Pro toto $\varepsilon$ z
 \hyperref[def:limita-posloupnosti]{definice limity} existuje $n_0 \in \N$
 takové, že pro všechna $n \geq n_0$ platí $|Z_n - \lim_{} Z_n| = |Z_n - S| <
 \varepsilon$. Jelikož $(Z_n)$ je nerostoucí, je absolutní hodnota v předchozím
 výrazu zbytečná a můžeme zkrátka psát $Z_n - S < \varepsilon$. Potom ale pro
 všechna $n \geq n_0$ máme
 \[
  x - Z_n = x + S - S - Z_n = (x - S) + (S - Z_n) > c - \varepsilon =
  \frac{c}{2},
 \]
 čili speciálně $x > Z_n$, což je ve sporu s volbou $Z_n$ jako horní závory $X$.
 To dokazuje (1).

 Tvrzení (2) lze dokázat obdobně, akorát využitím posloupnosti $(q_n)$ spíše než
 $(Z_n)$. Opět ať pro spor existuje $Z \in \R$, které je horní závorou $X$, a $Z
 < S$. Pak nalezneme konstantu $c > 0$ takovou, že $S - Z = c$. Opět z
 \hyperref[def:limita-posloupnosti]{definice limity} vezmeme $\varepsilon
 \coloneqq c / 2$ a k němu $n_0 \in \N$ takové, že $\forall n \geq n_0$ platí $S
 - q_n < \varepsilon$, kde absolutní hodnotu jsem mohli vynechat, bo $S \geq
 q_n$ pro 
\end{propproof}

\section{Metody výpočtů limit}
\label{sec:metody-vypoctu-limit}

Tato sekce je veskrze výpočetní, věnována způsobům určování limit rozličných
posloupností -- primárně těch zadaných vzorcem pro $n$-tý člen. Obecně
neexistuje algoritmus pro výpočet limity posloupnosti a například limity
posloupností zadaných rekurentně (další člen je vypočten jako kombinace
předchozích) je často obtížné určit. K jejich výpočtu bývá užito metod z
lineární algebry a obecně metod teorie diskrétních systémů zcela mimo rozsah
tohoto textu.

Přinejmenším v případě limit zadaných \uv{hezkými} vzorci čítajícími podíly
mnohočlenů a odmocnin je možné obyčejně algebraickými úpravami dojít k výsledku.
Uvedeme si pár stěžejních tvrzení sloužících tomuto účelu.

K důkazu prvního bude užitečná následující nerovnost, kterou přenecháváme
čtenáři jako (snadné) cvičení.

\begin{exercise}{}{random-abs-nerovnost}
 Dokažte, že pro čísla $x,y \in \R$ platí
 \[
  | |x| - |y| | \leq |x - y|.
 \]
\end{exercise}

\begin{theorem}{Aritmetika limit}{aritmetika-limit}
 Ať $a,b:\N \to \R$ jsou reálné posloupnosti mající limitu (ale klidně i
 nekonečnou). Pak
 \begin{enumerate}[label=(\alph*)]
  \item $\lim (a + b) = \lim a + \lim b$, je-li pravá strana definována;
  \item $\lim (a \cdot b) = \lim a \cdot \lim b$, je-li pravá strana definována;
  \item $\lim (a / b) = \lim a / \lim b$, platí-li $b \not\simeq 0$ a pravá
   strana je definována.
 \end{enumerate}
\end{theorem}
\begin{thmproof}
 Důkaz této věty je ryze výpočetního charakteru a využívá vhodně zvolených
 odhadů. Vzhledem k tomu, že povolujeme i nekonečné limity, je třeba důkaz
 každého bodu rozložit na případy. Položme $A \coloneqq \lim a, B \coloneqq \lim
 b$.

 \textbf{\emph{Případ $A,B \in \R$}.}\\
 Nejprve budeme předpokládat, že $A,B \in \R$. Pro dané $\varepsilon>0$ existují
 $n_a,n_b \in \N$ taková, že pro každé $n \geq n_a$ platí $|a_n-A|<\varepsilon$
 a pro každé $n \geq n_b$ zas $|b_n-B|<\varepsilon$. Zvolíme-li $n_0 \coloneqq
 \max(n_a,n_b)$, pak pro $n \geq n_0$ platí oba odhady zároveň. Potom ale,
 použitím \hyperref[lem:trojuhelnikova-nerovnost]{trojúhelníkové nerovnosti},
 dostaneme
 \[
  |(a_n+b_n)-(A+B)| = |(a_n-A)+(b_n-B)| \leq |a_n-A| + |b_n-B| <
  \varepsilon+\varepsilon = 2\varepsilon,
 \]
 čili $\lim (a+b) = A+B$. Pro důkaz vzorce pro součin a podíl, musíme navíc
 využít \myref{lemmatu}{lem:konvergentni-omezena}, tedy faktu, že konvergentní
 posloupnosti jsou omezené. Umíme tudíž najít $C_b \in \R$ takové, že od
 určitého indexu $n_1 \in \N$ dále platí $|b_n| \leq C_b$. Volme tedy nově $n_0
 \coloneqq \max(n_a,n_b,n_1)$ a pro $n \geq n_0$ počítejme
 \begin{align*}
  |a_n \cdot b_n - A \cdot B| &= |a_n \cdot b_n - b_n \cdot A + b_n \cdot A -
  A \cdot B| = |b_n(a_n - A) + A(b_n - B)|\\
                              & \leq |b_n(a_n - A)| + |A(b_n-B)| = |b_n| \cdot
                              |a_n-A| + |A| \cdot |b_n-B| \\
                              &< |C_b| \cdot \varepsilon + |A| \cdot
                              \varepsilon.
 \end{align*}
 Protože $|C_b|$ i $|A|$ jsou konstanty nezávislé na $\varepsilon$, znamená
 toto, že $\lim (a \cdot b) = A \cdot B$. Konečně, v případě podílu volme
 $\varepsilon_b = |B| / 2$. K tomuto $\varepsilon_b$ nalezněme $n'_b \in \N$
 takové, že pro $n \geq n'_b$ platí $|b_n - B| < \varepsilon_b$. Poslední
 nerovnost spolu s \myref{cvičením}{exer:random-abs-nerovnost} znamená, že $|
 |b_n| - |B| | < \varepsilon$. Tento vztah si rozepíšeme na
 \[
  |B| - \varepsilon_b < |b_n| < |B| + \varepsilon_b.
 \]
 Levá z těchto nerovností je pak ekvivalentní $|b_n| > |B| / 2$ neboli $1 /
 |b_n| < 2 / |B|$. Položme $n_0 \coloneqq \max(n_a,n_b,n'_b)$. Potom pro $n \geq
 n_0$ máme
 \begin{align*}
  \left| \frac{a_n}{b_n} - \frac{A}{B} \right| &= \left| \frac{a_nB -
  b_nA}{b_nB} \right| = \left| \frac{a_nB - AB + AB - b_nA}{b_nB} \right| \leq
  \left| \frac{B(a_n-A)}{b_nB} \right| + \left| \frac{A(B - b_n)}{b_nB} \right|
  \\
                                               &= \frac{1}{|b_n|}\left(|a_n-A| +
                                               \frac{|A|}{|B|}|B-b_n| \right) <
                                               \frac{2\varepsilon}{|B|}\left(1 +
                                               \frac{|A|}{|B|}\right).
 \end{align*}
 Protože $|A|$ i $|B|$ jsou konstanty nezávislé na $\varepsilon$, toto znamená,
 že $\lim (a / b) = A / B$.

 \textbf{\emph{Případ $A = \pm \infty, B \in \R \setminus \{0\}$.}}\\
 Předpokládejme, že $\lim a = \infty$; případ $\lim a = -\infty$ se dokáže v
 zásadě identicky. Pak pro dané $\varepsilon_a$ existuje $n_a \in \N$ takové, že
 pro $n \geq n_a$ platí $a_n > \varepsilon_a$. Podle
 \myref{lemmatu}{lem:konvergentni-omezena} je posloupnost $b$ omezená, čili
 existuje $C_b > 0$ takové, že $|b_n| \leq C_b$ pro všechna $n \in \N$. Potom
 pro $n \geq n_a$ máme
 \[
  a_n + b_n \geq a_n - C_b > \varepsilon_a - C_b.
 \]
 Jelikož $C_b$ je konstantní, plyne z tohoto odhadu, že $\lim (a + b) = \infty =
 A + B$.

 Pro důkaz součinu nejprve ať $B > 0$. Pak existuje konstanta $C_b > 0$ a $n_b
 \in \N$ takové, že pro $n \geq n_b$ je $b_n \geq C_b$. Pročež, pro libovolné
 $C_a > 0$ a $n \geq \max(n_a,n_b)$ dostaneme
 \[
  a_n \cdot b_n \geq \varepsilon_a \cdot C_b
 \]
 a rovněž
 \[
  \frac{a_n}{b_n} \geq \frac{\varepsilon_a}{C_b},
 \]
 čili $\lim (a \cdot b) = \infty = A \cdot B$ a $\lim (a / b) = \infty = A / B$.
 Velmi podobně se řeší případ $B < 0$.

 Zdlouhavý důkaz zakončíme komentářem, že případ $A \in \R \setminus \{0\}, B =
 \pm \infty$ je symetrický předchozímu a případy $A = 0, B = \pm \infty$, též $A
 = \pm \infty, B = 0$ a konečně $A = \pm \infty, B = \pm \infty$ jsou triviální.
\end{thmproof}

\hyperref[thm:aritmetika-limit]{Věta o aritmetice limit} je zcela nejužitečnější
tvrzení k jejich výpočtu, neboť umožňuje limitu výrazu rozdělit na mnoho menších
\uv{podlimit}, které jsou často známy. Další dvě lemmata jsou často též dobrými
sluhy.

\begin{lemma}{Limita odmocniny}{limita-odmocniny}
 Ať $a:\N \to [0,\infty)$ je posloupnost \emph{nezáporných} čísel. Ať též
    $\lim a = A$ (speciálně tedy předpokládáme, že $\lim a$ existuje). Potom
 \[
  \lim_{n \to \infty} \sqrt[k]{a_n} = \sqrt[k]{A}
 \]
 pro každé $k \in \N$.
\end{lemma}
\begin{lemproof}
 Zdlouhavý a technický. Ambiciózní čtenáři jsou zváni, aby se o něj pokusili.
\end{lemproof}

\begin{lemma}{O dvou strážnících}{o-dvou-straznicich}
 Ať $a,b,c:\N \to \R$ jsou posloupnosti reálných čísel a $L \coloneqq \lim a =
 \lim c$. Pokud existuje $n_0 \in \N$ takové, že pro každé $n \geq n_0$ platí
 $a_n \leq b_n \leq c_n$, pak $\lim b = L$.
\end{lemma}
\begin{lemproof}
 Protože $\lim a = L$ a též $\lim c = L$, nalezneme pro dané $\varepsilon>0$
 index $n_1 \in \N$ takový, že pro $n \geq n_1$ platí dva odhady:
 \[
  |a_n - L| < \varepsilon \quad \text{a} \quad |c_n - L| < \varepsilon.
 \]
 Potom ovšem $a_n > L - \varepsilon$ a $c_n < L + \varepsilon$. Z předpokladu
 existuje $n_b \in \N$ takové, že $a_n \leq b_n \leq c_n$ pro $n \geq n_b$.
 Zvolíme-li tedy $n_0 \coloneqq \max(n_1,n_b)$, pak pro $n \geq n_0$ platí
 \[
  L - \varepsilon < a_n \leq b_n \leq c_n < L + \varepsilon.
 \]
 Sloučením obou nerovností dostaneme pro $n \geq n_0$ odhad $|b_n -
 L|<\varepsilon$, čili $\lim b = L$.
\end{lemproof}

\begin{figure}[ht]
 \centering
 \begin{tikzpicture}
  \tkzInit[xmin=0,xmax=10,ymin=0,ymax=3]
  \foreach \n in {0,1,...,18} {
   \tkzDefPoint(0.5 * (\n + 1),0){x\n}
   \tkzDrawPoint[shape=cross](x\n)
   \tkzLabelPoint[below](x\n){$\n$}
  }
  \tkzLabelPoint[below,color=RoyalPurple](x11){$11$}
  \foreach \y in {1,2,3,4,5,6} {
   \tkzDefPoint(0,0.5 * \y){y\y}
   \tkzDrawPoint[shape=cross](y\y)
   \tkzLabelPoint[left](y\y){$\y$}
  }
  \tkzDefPoint(0.5,2.8){c0}
  \tkzDefPoint(1,0.9){c1}
  \tkzDefPoint(1.5,2.3){c2}
  \tkzDefPoint(2,1){c3}
  \tkzDefPoint(2.5,1.1){c4}
  \tkzDefPoint(3,0.3){c5}
  \tkzDefPoint(3.5,2.4){c6}
  \tkzDefPoint(4,1.2){c7}
  \tkzDefPoint(4.5,3.2){c8}
  \tkzDefPoint(5,2.6){c9}
  \tkzDefPoint(5.5,1.9){c10}
  \tkzDefPoint(6,2.4){c11}
  \tkzDefPoint(6.5,2.2){c12}
  \tkzDefPoint(7,1.9){c13}
  \tkzDefPoint(7.5,2){c14}
  \tkzDefPoint(8,1.8){c15}
  \tkzDefPoint(8.5,2.05){c16}
  \tkzDefPoint(9,1.95){c17}
  \tkzDefPoint(9.5,1.85){c18}

  \tkzDefPoint(0.5,1){b0}
  \tkzDefPoint(1,2.1){b1}
  \tkzDefPoint(1.5,0.3){b2}
  \tkzDefPoint(2,1.1){b3}
  \tkzDefPoint(2.5,1.6){b4}
  \tkzDefPoint(3,0.8){b5}
  \tkzDefPoint(3.5,2){b6}
  \tkzDefPoint(4,0.9){b7}
  \tkzDefPoint(4.5,0.2){b8}
  \tkzDefPoint(5,2.2){b9}
  \tkzDefPoint(5.5,0.9){b10}
  \tkzDefPoint(6,2){b11}
  \tkzDefPoint(6.5,1.8){b12}
  \tkzDefPoint(7,1){b13}
  \tkzDefPoint(7.5,1.6){b14}
  \tkzDefPoint(8,1.6){b15}
  \tkzDefPoint(8.5,1.4){b16}
  \tkzDefPoint(9,1.7){b17}
  \tkzDefPoint(9.5,1.65){b18}
  
  \tkzDefPoint(0.5,1.5){a0}
  \tkzDefPoint(1,0.2){a1}
  \tkzDefPoint(1.5,3){a2}
  \tkzDefPoint(2,2.7){a3}
  \tkzDefPoint(2.5,1.7){a4}
  \tkzDefPoint(3,0.2){a5}
  \tkzDefPoint(3.5,2.2){a6}
  \tkzDefPoint(4,1){a7}
  \tkzDefPoint(4.5,0.7){a8}
  \tkzDefPoint(5,0.1){a9}
  \tkzDefPoint(5.5,1.3){a10}
  \tkzDefPoint(6,0.8){a11}
  \tkzDefPoint(6.5,1.5){a12}
  \tkzDefPoint(7,0.6){a13}
  \tkzDefPoint(7.5,1){a14}
  \tkzDefPoint(8,1.4){a15}
  \tkzDefPoint(8.5,0.9){a16}
  \tkzDefPoint(9,1.5){a17}
  \tkzDefPoint(9.5,1.55){a18}

  \tkzDrawX[>=latex,label={$n$}]
  \tkzDrawY[>=latex,label={$\clr{a_n},\clg{b_n},\clb{c_n}$}]

  \tkzDefPoint(0,1.6){O}
  \tkzDrawPoint[size=4,color=RoyalPurple](O)
  \tkzLabelPoint[right,color=RoyalPurple](O){$L$}
  \tkzDefPoint(0.5,1.6){P}
  \tkzDefPoint(6.5,1.6){P2}
  \tkzDrawLine[add=0 and 0.7,dashed,color=RoyalPurple,thick](P,P2)

  \tkzDrawPoints[color=RoyalBlue](c0,c1,c2,c3,c4,c5,c6,c7,c8,c9,c10,c11,c12,c13,c14,c15,c16,c17,c18)
  \tkzDrawPoints[color=ForestGreen](b0,b1,b2,b3,b4,b5,b6,b7,b8,b9,b10,b11,b12,b13,b14,b15,b16,b17,b18)
  \tkzDrawPoints[color=BrickRed](a0,a1,a2,a3,a4,a5,a6,a7,a8,a9,a10,a11,a12,a13,a14,a15,a16,a17,a18)

  \foreach \n in {11,12,...,17} {
   \pgfmathparse{\n+1}
   \edef\m{\pgfmathresult}
   \tkzDrawSegment[dashed,color=BrickRed](a\n,a\m)
   \tkzDrawSegment[dashed,color=ForestGreen](b\n,b\m)
   \tkzDrawSegment[dashed,color=RoyalBlue](c\n,c\m)
  }
  \tkzLabelPoint[above,color=RoyalPurple](x11){$n_0$}
  \tkzDefPoint(6,3){wtv}
  \tkzDrawLine[add=-0.15 and 0,dashed,color=RoyalPurple](x11,wtv)
 \end{tikzpicture}

 \caption{Lemma \hyperref[lem:o-dvou-straznicich]{o dvou strážnících}.}
 \label{fig:dva-straznici}
\end{figure}

Zbytek sekce je věnován výpočtům limit vybraných posloupností s účelem objasnit
použití právě sepsaných tvrzení. Mnoho z nich je ponecháno čtenářům jako
cvičení.

\begin{problem}{}{piplacka-limity}
 Spočtěte
 \[
  \lim_{n \to \infty} \frac{2n^2+n-3}{n^3-1}.
 \]
\end{problem}
\begin{probsol}
 Použijeme \hyperref[thm:aritmetika-limit]{větu o aritmetice limit}. Ta
 vyžaduje, aby výsledná strana rovnosti byla definována. Je tudíž možné (a
 žádoucí) limitu spočítat -- často opakovaným použitím této věty -- a teprve na
 konci výpočtu argumentovat, že její nasazení bylo oprávněné.

 Dobrým prvním krokem při řešení limit zadaných zlomky je najít v čitateli i
 jmenovateli \uv{nejrychleji rostoucí} člen. Spojením \uv{nejrychleji rostoucí}
 zde míníme takový člen, velikost ostatních členů je pro velmi velká $n$ vůči
 jehož zanedbatelná. V čitateli zlomku
 \[
  \frac{2n^2 + n - 3}{n^3 - 1}
 \]
 je nejrychleji rostoucí člen právě $2n^2$. Například, pro $n = 10^9$ je $2n^2 =
 2 \cdot 10^{18}$ zatímco $n = 10^9$ zabírá méně než jednu miliardtinu $2n^2$.
 Ve jmenovateli je naopak jediným rostoucím členem $n^3$. Nejrychleji rostoucí
 členy (pro pohodlí bez koeficientů) z obou částí zlomku vytkneme. Dostaneme
 \[
  \frac{2n^2 + n - 3}{n^3 - 1} = \frac{n^2 \left( 2 + \frac{1}{n} -
  \frac{3}{n^2} \right)}{n^3 \left(1 - \frac{1}{n^3}\right)} = \frac{1}{n} \cdot
  \frac{2 + \frac{1}{n} - \frac{3}{n^2}}{1 - \frac{1}{n^3}}.
 \]
 Část (b) \hyperref[thm:aritmetika-limit]{věty o aritmetice limit} nyní dává
 \begin{equation*}
  \label{eq:soucin-limit}
  \tag{$\triangle$}
  \lim_{n \to \infty} \frac{2n^2 + n - 3}{n^3 - 1} = \lim_{n \to \infty}
  \frac{1}{n} \cdot \lim_{n \to \infty} \frac{2 + \frac{1}{n} - \frac{3}{n^2}}{1
  - \frac{1}{n^3}},
 \end{equation*}
 \textbf{za předpokladu, že součin na pravé straně je definován!}

 Již víme, že platí $\lim_{n \to \infty} \frac{1}{n} = 0$. \textbf{Pozor!} Bylo
 by lákavé prohlásit, že výsledná limita je rovna $0$, neboť součin čehokoliv s
 $0$ je též $0$. To je pravda pro všechna čísla až na $\pm \infty$. Musíme se
 ujistit, že druhá limita v součinu na pravé straně~\eqref{eq:soucin-limit}
 existuje a není nekonečná.

 S použitím \hyperref[thm:aritmetika-limit]{věty o aritmetice limit} (c)
 počítáme
 \[
  \lim_{n \to \infty} \frac{2 + \frac{1}{n} - \frac{3}{n^2}}{1 - \frac{1}{n^3}}
  = \frac{\lim_{n \to \infty} 2 + \frac{1}{n} - \frac{3}{n^2}}{\lim_{n \to
  \infty} 1 - \frac{1}{n^3}}.
 \]
 Limity v čitateli a jmenovateli zlomku výše spočteme zvlášť. Z
 \hyperref[thm:aritmetika-limit]{věty o aritmetice limit}, části (a), plyne, že
 \[
  \lim_{n \to \infty} 2 + \frac{1}{n} - \frac{3}{n^2} = \lim_{n \to \infty} 2 +
  \lim_{n \to \infty} \frac{1}{n} - \lim_{n \to \infty} \frac{3}{n^2} = 2 + 0 -
  0 = 2.
 \]
 Podle stejného tvrzení též
 \[
  \lim_{n \to \infty} 1 - \frac{1}{n^3} = \lim_{n \to \infty} 1 - \lim_{n \to
  \infty} \frac{1}{n^3} = 1 - 0 = 1.
 \]
 To znamená, že
 \[
  \lim_{n \to \infty} \frac{2 + \frac{1}{n} - \frac{3}{n^2}}{1 - \frac{1}{n^3}}
  = \frac{2}{1} = 2.
 \]
 Odtud pak
 \[
  \lim_{n \to \infty} \frac{2n^2 + n - 3}{n^3 - 1} = \lim_{n \to \infty}
  \frac{1}{n} \cdot \lim_{n \to \infty} \frac{2 + \frac{1}{n} - \frac{3}{n^2}}{1
  - \frac{1}{n^3}} = 0 \cdot 2 = 0.
 \]
 Protože všechny výrazy na konci výpočtů jsou reálná čísla (a tedy speciálně jsou
 dobře definované), bylo lze použít \hyperref[thm:aritmetika-limit]{větu o
 aritmetice limit}.
\end{probsol}

\begin{remark}{}{aritmetika-limit}
 Právě vyřešená \myref{úloha}{prob:piplacka-limity} ukazuje, jak dlouhé se
 limitní úlohy stávají při ověřování všech předpokladů. A to jsme navíc použil
 \emph{jen jediné tvrzení} k jejímu výpočtu. Takový postup není, s
 pochopitelného důvodu, obvyklý. Opakovaná použití
 \hyperref[thm:aritmetika-limit]{věty o aritmetice limit} se často schovají pod
 jedno prohlášení a výpočet limity je pak mnohem stručnější. Názorně předvedeme.

 Snadno úpravou zjistíme, že
 \[
  \frac{2n^2 + n - 3}{n^3 - 1} = \frac{1}{n} \cdot \frac{2 + \frac{1}{n} -
  \frac{3}{n^2}}{1 - \frac{1}{n^3}}.
 \]
 Potom z \hyperref[thm:aritmetika-limit]{věty o aritmetice limit} platí
 \[
  \lim_{n \to \infty} \frac{2n^2 + n - 3}{n^3 - 1} = \lim_{n \to \infty}
  \frac{1}{n} \cdot \lim_{n \to \infty} \frac{2 + \frac{1}{n} - \frac{3}{n^2}}{1
  - \frac{1}{n^3}} = 0 \cdot \frac{2 + 0 + 0}{1 + 0} = 0.
 \]
 Protože výsledný výraz je definovaný, byla \hyperref[thm:aritmetika-limit]{věta
 o aritmetice limit} použita korektně.

 My rovněž hodláme v dalším textu bez varování řešit podobné limitní příklady
 tímto \uv{zkráce\-ným} způsobem.
\end{remark}

\begin{warning}{}{aritmetika-limit}
 \hyperref[thm:aritmetika-limit]{Větou o aritmetice limit} \textbf{nelze}
 dokazovat, že limita posloupnosti neexistuje, neboť předpokladem každé její
 části je \emph{definovanost} výsledného výrazu. Jeho zanedbání může snadno vést
 ke lži. Uvažme následující triviální příklad.

 Prohlásili-li bychom, že z \hyperref[thm:aritmetika-limit]{věty o aritmetice
 limit} platí výpočet
 \[
  \lim_{n \to \infty} \frac{n}{n} = \frac{\lim_{n \to \infty} n}{\lim_{n \to
  \infty} n} = \frac{\infty}{\infty},
 \]
 nabyli bychom práva tvrdit, že $\lim_{n \to \infty} n / n$ neexistuje, přestože
 zřejmě platí $\lim_{n \to \infty} n / n = \lim_{n \to \infty} 1 = 1$.
 \hyperref[thm:aritmetika-limit]{Věta o aritmetice limit} je tudíž zcela prázdné
 tvrzení v případě nedefinovanosti výsledného výrazu.
\end{warning}



