\subsection{Řady s nezápornými členy}
\label{ssec:rady-s-nezapornymi-cleny}

V této sekci se budeme zabývat nejsnadněji zpytovaným typem číselných řad --
řadami, jejichž členy jsou pouze nezáporná čísla. Jejich zpyt je vesměs
jednoduchý z toho důvodu, že tyto řady vždycky mají součet, ať už konečný či
nekonečný. Existence záporných členů v číselné řadě totiž vyžaduje, aby jeden
analyzoval jemný vztah mezi její `kladnou' a `zápornou' částí a hodnotil, zda
obě v jistém smyslu `rostou stejně rychle', či nikolivěk.

Součty ani řad s nezápornými členy však není vůbec triviální určit a otázky
jejich konvergence jsou obyčejně řešeny srovnáními s řadami, jejichž součty
známy jsou. Nástrojem k tomu je následující vcelku přímočaré tvrzení.

\begin{proposition}{Srovnávací kritérium}{srovnavaci-kriterium}
 Ať $\sum_{n = 0}^{\infty} a_n, \sum_{n = 0}^{\infty} b_n$ jsou řady \textbf{s
 nezápornými členy}. Ať dále existuje $n_0 \in \N$ takové, že pro $n \geq n_0$
 platí $a_n \leq b_n$. Potom,
 \begin{enumerate}[label=(\alph*)]
  \item konverguje-li $\sum_{n = 0}^{\infty} b_n$, konverguje i $\sum_{n =
   0}^{\infty} a_n$;
  \item je-li $\sum_{n = 0}^{\infty} a_n = \infty$, pak i $\sum_{n = 0}^{\infty}
   b_n = \infty$.
 \end{enumerate}
\end{proposition}
\begin{propproof}
 Položme $s_n \coloneqq \sum_{i=0}^n a_i$ a $t_n \coloneqq \sum_{i=0}^n b_i$. Z
 předpokladu máme $n_0 \in \N$, od kterého dále již platí $a_n \leq b_n$. Pro
 důkaz (a) předpokládejme rovněž, že $\sum_{n = 0}^{\infty} b_n$ konverguje, tj.
 existuje konečná $\lim_{n \to \infty} t_n$.

 Ukážeme nejprve, že $s_n$ je shora omezená. Pro $n \geq n_0$ odhadujme
 \[
  s_n = s_{n_0} + \sum_{i=n_0+1}^n a_i \leq s_{n_0} + \sum_{i=n_0+1}^n b_i \leq
  s_{n_0} + \sum_{i=0}^{n} b_i = s_{n_0} + t_m,
 \]
 kde odhad $\sum_{i=n_0+1}^n b_i \leq \sum_{i=0}^n b_i$ platí díky nezápornosti
 členů $b_n$.

 Nyní, opět pro nezápornost $b_n$, je posloupnost částečných součtů $t_n$
 neklesající. Pročež pro všechna $k \in \N$ máme $t_k
 \leq \lim_{n \to \infty} t_n$. To nám umožňuje pro $n \geq n_0$ dokončit odhad
 \[
  s_n \leq s_{n_0} + t_n \leq s_{n_0} + \lim_{n \to \infty} t_n,
 \]
 který ukazuje, že $s_n$ je omezená. To ovšem zakončuje důkaz části (a), neboť
 $s_n$ je shora omezená neklesající (pro nezápornost $a_n$) posloupnost, a tedy
 má podle \myref{lemmatu}{lem:limita-monotonni-posloupnosti} limitu.

 Část (b) je pouze přepisem části (a) v kontrapozitivní formě, která zní, že
 nekonveruje-li $\sum_{n=0}^{\infty} a_n$, pak nekonverguje ani
 $\sum_{n=0}^{\infty} b_n$. Ovšem, divergentní řady s nezápornými členy mají
 součet $\infty$, odkud již přímo plyne závěr v (b).
\end{propproof}

Pochopitelně, srovnávací kritérium je užitečné pouze ve chvíli, kdy má jeden
\emph{s čím} srovnávat. Jmeme se odvodit divergenci jedné a konvergenci druhé z
takřkouce \uv{učebnicových} řad.

\begin{lemma}{Divergence harmonické řady}{divergence-harmonicke-rady} Číselná
 řada $\sum_{n=1}^{\infty} 1 / n$ diverguje, neboli $\sum_{n=1}^{\infty} 1 / n =
 \infty$.
\end{lemma}
\begin{lemproof}
 Použijeme \myref{tvrzení}{prop:vztah-konvergence-a-existence-souctu} a dokážeme
 negaci výroku o konvergenci $\sum_{n=1}^{\infty} 1 / n$. Konkrétně výrok
 \[
  \exists \varepsilon>0 \, \forall n_0 \in \N \, \exists m > n \geq n_0: \left|
  \sum_{i=n+1}^{m} \frac{1}{i} \right| \geq \varepsilon.
 \]
 
 Ukážeme, že $\varepsilon = 1 / 2$ vyhovuje výroku výše. Ať je $n_0 \in \N$
 dáno. Volme $n \coloneqq n_0$ a $m \coloneqq 2n_0$. Pak pro všechna $i \in \N,
 n_0 \leq i \leq 2n_0$ platí $1 / i \geq 1 / 2n_0$. Součet výše můžeme pročež
 zezdola odhadnout
 \[
  \left| \sum_{i=n_0+1}^{2n_0} \frac{1}{i} \right| = \sum_{i=n_0+1}^{2n_0}
  \frac{1}{i} \geq \sum_{i=n_0+1}^{2n_0} \frac{1}{2n_0} = n_0 \cdot
  \frac{1}{2n_0} = \frac{1}{2},
 \]
 kde první nerovnost plyne z faktu, že $1 / n > 0$ pro všechna $n \in \N$.

 Pro dané $n_0 \in \N$ tudíž platí
 \[
  \left| \sum_{i=n_0+1}^{2n_0} \frac{1}{i} \right| \geq \frac{1}{2} =
  \varepsilon,
 \]
 a tedy řada $\sum_{n=1}^{\infty} 1 / n$ diverguje. Protože jsou však její členy
 nezáporné, znamená toto, že $\sum_{n=1}^{\infty} 1 / n = \infty$, což bylo jest
 dokázati.
\end{lemproof}

\begin{remark}{}{harmonicka-rada}
 Řada $\sum_{n=1}^{\infty} 1 / n$ sluje \emph{harmonická}, protože je úzce
 spojena s pojmem \emph{alikvóty} a harmonie v~hud\-bě. Vlnové délky alikvót
 daného tónu (vlastně \uv{souznivých} tónů) jsou $1 / 2, 1 / 3, 1 / 4$ atd. jeho
 základní frekvence. Každý člen harmonické řady je \emph{harmonickým průměrem}
 svých sousedů, takže trojice členů v této řadě představuje vlnové délky tónů
 tvořících konsonantní akordy v~tónině dané původním frekvencí (jejíž alikvóty
 jsou vyjádřeny členy řady). Vizte např.
 \href{https://en.wikipedia.org/wiki/Harmonic_series_(mathematics)}{stránku na
 Wikipedii}.
\end{remark}

\begin{lemma}{}{divergence-n^2}
 Řada $\sum_{n=1}^{\infty} 1 / n^2$ konverguje.
\end{lemma}
\begin{lemproof}
 Srovnáme řadu $\sum_{n=1}^{\infty} 1 / n^2$ s řadou $\sum_{n=1}^{\infty} 1 /
 n(n+1)$ z \myref{úlohy}{prob:soucet-rady-pres-castecne-soucty}, jejíž součet
 je roven $1$. Protože obě řady mají nezáporné členy, lze použít
 \hyperref[prop:srovnavaci-kriterium]{srovnávací kritérium}.

 Indukcí dokážeme, že platí
 \[
  \frac{2}{n(n+1)} \geq \frac{1}{n^2} \quad \forall n \in \N.
 \]
 Pro $n = 1$ máme
 \[
  \frac{2}{1 \cdot (1 + 1)} = 1 \geq \frac{1}{1^2} = 1.
 \]
 Předpokládejme, že daná rovnost platí pro $n \in \N$. Počítáme
 \[
  \frac{2}{(n+1)(n+2)} = \frac{2}{n+1} \cdot \frac{1}{n+2} \geq \frac{1}{n}
  \cdot \frac{1}{n+2} \geq \frac{1}{(n+1)^2},
 \]
 kde první nerovnost plyne z indukčního předpokladu (po vynásobení obou stran
 číslem $n \in \N$) a poslední nerovnost plyne ze zřejmého vztahu
 \[
  n(n+2) = n^2 + 2n \leq n^2 + 2n + 1 = (n+1)^2.
 \]

 Nyní, z \hyperref[prop:aritmetika-ciselnych-rad]{aritmetiky řad} platí
 \[
  \sum_{n=1}^{\infty} \frac{2}{n(n+1)} = 2 \sum_{n=1}^{\infty} \frac{1}{n(n+1)}
  = 2
 \]
 a již jsme dokázali, že $2 / n(n+1) \geq 1 / n^2$ pro všechna $n \in \N$.
 \hyperref[prop:srovnavaci-kriterium]{Srovnávací kritérium} nyní dává
 \[
  \sum_{n=1}^{\infty} \frac{1}{n^2} \leq \sum_{n=1}^{\infty} \frac{2}{n(n+1)} =
  2,
 \]
 čili je řada $\sum_{n=1}^{\infty} 1 / n^2$ konvergentní.
\end{lemproof}

Zbytek sekce o řadách se nezápornými členy je věnován důkazu dvou další
užitečnou kritériou pro srovnání s ostatními řadami. Završíme zesílením závěru
\myref{lemmatu}{lem:divergence-n^2} dokázajíce, že $\sum_{n=1}^{\infty}
\frac{1}{n^{c}}$ je konvergentní právě tehdy, když $c > 1$.

\begin{proposition}{Cauchyho odmocninové kritérium}{cauchyho-odmocninove-kriterium}
 
\end{proposition}
