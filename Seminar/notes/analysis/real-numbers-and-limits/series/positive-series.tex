\subsection{Řady s nezápornými členy}
\label{ssec:rady-s-nezapornymi-cleny}

V této sekci se budeme zabývat nejsnadněji zpytovaným typem číselných řad --
řadami, jejichž členy jsou pouze nezáporná čísla. Jejich zpyt je vesměs
jednoduchý z toho důvodu, že tyto řady vždycky mají součet, ať už konečný či
nekonečný. Existence záporných členů v číselné řadě totiž vyžaduje, aby jeden
analyzoval jemný vztah mezi její `kladnou' a `zápornou' částí a hodnotil, zda
obě v jistém smyslu `rostou stejně rychle', či nikolivěk.

Součty ani řad s nezápornými členy však není vůbec triviální určit a otázky
jejich konvergence jsou obyčejně řešeny srovnáními s řadami, jejichž součty
známy jsou. Nástrojem k tomu je následující vcelku přímočaré tvrzení.

\begin{proposition}{Srovnávací kritérium}{srovnavaci-kriterium}
 Ať $\sum_{n = 0}^{\infty} a_n, \sum_{n = 0}^{\infty} b_n$ jsou řady \textbf{s
 nezápornými členy}. Ať dále existuje $n_0 \in \N$ takové, že pro $n \geq n_0$
 platí $a_n \leq b_n$. Potom,
 \begin{enumerate}[label=(\alph*)]
  \item konverguje-li $\sum_{n = 0}^{\infty} b_n$, konverguje i $\sum_{n =
   0}^{\infty} a_n$;
  \item je-li $\sum_{n = 0}^{\infty} a_n = \infty$, pak i $\sum_{n = 0}^{\infty}
   b_n = \infty$.
 \end{enumerate}
\end{proposition}
\begin{propproof}
 Položme $s_n \coloneqq \sum_{i=0}^n a_i$ a $t_n \coloneqq \sum_{i=0}^n b_i$. Z
 předpokladu máme $n_0 \in \N$, od kterého dále již platí $a_n \leq b_n$. Pro
 důkaz (a) předpokládejme rovněž, že $\sum_{n = 0}^{\infty} b_n$ konverguje, tj.
 existuje konečná $\lim_{n \to \infty} t_n$.

 Ukážeme nejprve, že $s_n$ je shora omezená. Pro $n \geq n_0$ odhadujme
 \[
  s_n = s_{n_0} + \sum_{i=n_0+1}^n a_i \leq s_{n_0} + \sum_{i=n_0+1}^n b_i \leq
  s_{n_0} + \sum_{i=0}^{n} b_i = s_{n_0} + t_m,
 \]
 kde odhad $\sum_{i=n_0+1}^n b_i \leq \sum_{i=0}^n b_i$ platí díky nezápornosti
 členů $b_n$.

 Nyní, opět pro nezápornost $b_n$, je posloupnost částečných součtů $t_n$
 neklesající. Pročež pro všechna $k \in \N$ máme $t_k
 \leq \lim_{n \to \infty} t_n$. To nám umožňuje pro $n \geq n_0$ dokončit odhad
 \[
  s_n \leq s_{n_0} + t_n \leq s_{n_0} + \lim_{n \to \infty} t_n,
 \]
 který ukazuje, že $s_n$ je omezená. To ovšem zakončuje důkaz části (a), neboť
 $s_n$ je shora omezená neklesající (pro nezápornost $a_n$) posloupnost, a tedy
 má podle \myref{lemmatu}{lem:limita-monotonni-posloupnosti} limitu.

 Část (b) je pouze přepisem části (a) v kontrapozitivní formě, která zní, že
 nekonveruje-li $\sum_{n=0}^{\infty} a_n$, pak nekonverguje ani
 $\sum_{n=0}^{\infty} b_n$. Ovšem, divergentní řady s nezápornými členy mají
 součet $\infty$, odkud již přímo plyne závěr v (b).
\end{propproof}

Pochopitelně, srovnávací kritérium je užitečné pouze ve chvíli, kdy má jeden
\emph{s čím} srovnávat. Jmeme se odvodit divergenci jedné a konvergenci druhé z
takřkouce \uv{učebnicových} řad.

\begin{lemma}{Divergence harmonické řady}{divergence-harmonicke-rady} Číselná
 řada $\sum_{n=1}^{\infty} 1 / n$ diverguje, neboli $\sum_{n=1}^{\infty} 1 / n =
 \infty$.
\end{lemma}
\begin{lemproof}
 Použijeme \myref{tvrzení}{prop:vztah-konvergence-a-existence-souctu} a dokážeme
 negaci výroku o konvergenci $\sum_{n=1}^{\infty} 1 / n$. Konkrétně výrok
 \[
  \exists \varepsilon>0 \, \forall n_0 \in \N \, \exists m > n \geq n_0: \left|
  \sum_{i=n+1}^{m} \frac{1}{i} \right| \geq \varepsilon.
 \]
 
 Ukážeme, že $\varepsilon = 1 / 2$ vyhovuje výroku výše. Ať je $n_0 \in \N$
 dáno. Volme $n \coloneqq n_0$ a $m \coloneqq 2n_0$. Pak pro všechna $i \in \N,
 n_0 \leq i \leq 2n_0$ platí $1 / i \geq 1 / 2n_0$. Součet výše můžeme pročež
 zezdola odhadnout
 \[
  \left| \sum_{i=n_0+1}^{2n_0} \frac{1}{i} \right| = \sum_{i=n_0+1}^{2n_0}
  \frac{1}{i} \geq \sum_{i=n_0+1}^{2n_0} \frac{1}{2n_0} = n_0 \cdot
  \frac{1}{2n_0} = \frac{1}{2},
 \]
 kde první nerovnost plyne z faktu, že $1 / n > 0$ pro všechna $n \in \N$.

 Pro dané $n_0 \in \N$ tudíž platí
 \[
  \left| \sum_{i=n_0+1}^{2n_0} \frac{1}{i} \right| \geq \frac{1}{2} =
  \varepsilon,
 \]
 a tedy řada $\sum_{n=1}^{\infty} 1 / n$ diverguje. Protože jsou však její členy
 nezáporné, znamená toto, že $\sum_{n=1}^{\infty} 1 / n = \infty$, což bylo jest
 dokázati.
\end{lemproof}

\begin{remark}{}{harmonicka-rada}
 Řada $\sum_{n=1}^{\infty} 1 / n$ sluje \emph{harmonická}, protože je úzce
 spojena s pojmem \emph{alikvóty} a harmonie v~hud\-bě. Vlnové délky alikvót
 daného tónu (vlastně \uv{souznivých} tónů) jsou $1 / 2, 1 / 3, 1 / 4$ atd. jeho
 základní frekvence. Každý člen harmonické řady je \emph{harmonickým průměrem}
 svých sousedů, takže trojice členů v této řadě představuje vlnové délky tónů
 tvořících konsonantní akordy v~tónině dané původním frekvencí (jejíž alikvóty
 jsou vyjádřeny členy řady). Vizte např.
 \href{https://en.wikipedia.org/wiki/Harmonic_series_(mathematics)}{stránku na
 Wikipedii}.
\end{remark}

\begin{lemma}{}{konvergence-n^2}
 Řada $\sum_{n=1}^{\infty} 1 / n^2$ konverguje.
\end{lemma}
\begin{lemproof}
 Srovnáme řadu $\sum_{n=1}^{\infty} 1 / n^2$ s řadou $\sum_{n=1}^{\infty} 1 /
 n(n+1)$ z \myref{úlohy}{prob:soucet-rady-pres-castecne-soucty}, jejíž součet
 je roven $1$. Protože obě řady mají nezáporné členy, lze použít
 \hyperref[prop:srovnavaci-kriterium]{srovnávací kritérium}.

 Indukcí dokážeme, že platí
 \[
  \frac{2}{n(n+1)} \geq \frac{1}{n^2} \quad \forall n \in \N.
 \]
 Pro $n = 1$ máme
 \[
  \frac{2}{1 \cdot (1 + 1)} = 1 \geq \frac{1}{1^2} = 1.
 \]
 Předpokládejme, že daná rovnost platí pro $n \in \N$. Počítáme
 \[
  \frac{2}{(n+1)(n+2)} = \frac{2}{n+1} \cdot \frac{1}{n+2} \geq \frac{1}{n}
  \cdot \frac{1}{n+2} \geq \frac{1}{(n+1)^2},
 \]
 kde první nerovnost plyne z indukčního předpokladu (po vynásobení obou stran
 číslem $n \in \N$) a poslední nerovnost plyne ze zřejmého vztahu
 \[
  n(n+2) = n^2 + 2n \leq n^2 + 2n + 1 = (n+1)^2.
 \]

 Nyní, z \hyperref[prop:aritmetika-ciselnych-rad]{aritmetiky řad} platí
 \[
  \sum_{n=1}^{\infty} \frac{2}{n(n+1)} = 2 \sum_{n=1}^{\infty} \frac{1}{n(n+1)}
  = 2
 \]
 a již jsme dokázali, že $2 / n(n+1) \geq 1 / n^2$ pro všechna $n \in \N$.
 \hyperref[prop:srovnavaci-kriterium]{Srovnávací kritérium} nyní dává
 \[
  \sum_{n=1}^{\infty} \frac{1}{n^2} \leq \sum_{n=1}^{\infty} \frac{2}{n(n+1)} =
  2,
 \]
 čili je řada $\sum_{n=1}^{\infty} 1 / n^2$ konvergentní.
\end{lemproof}

Na základě $\lim_{n \to \infty} a_n$ nelze obecně o konvergenci řady
$\sum_{n=0}^{\infty} a_n$ rozhodnout. Je-li limita nenulová, pak řada jistě
diverguje, ale je-li nulová, může řada konvergovat i divergovat. Máme-li ovšem
dvě řady, posloupnosti jejichž členů rostou v limitním smyslu \uv{stejně
rychle}, pak jsou i otázky jejich konvergencí ekvivalentní.

\begin{theorem}{Limitní srovnávací kritérium}{limitni-srovnavaci-kriterium}
 Ať $\sum_{n = 0}^{\infty} a_n, \sum_{n = 0}^{\infty} b_n$ jsou řady s
 nezápornými členy a označme $L \coloneqq \lim_{n \to \infty} a_n / b_n$.
 \begin{enumerate}[label=(\alph*)]
  \item Je-li $L \in (0,\infty)$, pak $\sum_{n = 0}^{\infty} a_n$ konverguje
   \textbf{právě tehdy, když} konverguje $\sum_{n = 0}^{\infty} b_n$.
  \item Je-li $L = 0$, pak konvergence $\sum_{n = 0}^{\infty} b_n$ implikuje
   konvergenci $\sum_{n = 0}^{\infty} a_n$.
  \item Je-li $L = \infty$, pak konvergence $\sum_{n = 0}^{\infty} a_n$
   implikuje konvergenci $\sum_{n = 0}^{\infty} b_n$.
 \end{enumerate}
\end{theorem}
\begin{thmproof}
 Před samotným důkazem je dlužno nahlédnout, že nutně $L \in [0,\infty]$, neboť
 posloupnosti $a_n$ i $b_n$ mají pouze nezáporné členy, tedy hodnoty $L$
 rozlišené výše jsou vskutku jediné možné.

 Započněme částí (a) a dokažme prve implikaci $( \Leftarrow )$. Předpokládejme,
 že $L \in (0,\infty)$ a $\sum_{n = 0}^{\infty} b_n$ konverguje. Volme
 $\varepsilon>0$ libovolně. Nalezneme $n_0 \in \N$ takové, že pro $n \geq n_0$
 platí
 \[
  \left| \frac{a_n}{b_n} - L \right| < \varepsilon.
 \]
 První nerovnost lze přepsat na
 \[
  L - \varepsilon < \frac{a_n}{b_n} < L + \varepsilon,
 \]
 čímž dostaneme horní odhad
 \[
  a_n < b_n(L + \varepsilon)
 \]
 pro všechna $n \geq n_0$. Z \hyperref[prop:aritmetika-ciselnych-rad]{aritmetiky
 řad} plyne, že řada $\sum_{n = 0}^{\infty} b_n(L + \varepsilon)$ je
 konvergentní (neboť $\sum_{n = 0}^{\infty} b_n$ je konvergentní) a ze
 \hyperref[prop:srovnavaci-kriterium]{srovnávacího kritéria} dostáváme (použitím
 odhadu výše), že i $\sum_{n = 0}^{\infty} a_n$ je konvergentní.

 K důkazu implikace $( \Rightarrow )$ je nám dán předpoklad $L \in (0,\infty)$ a
 konvergence $\sum_{n = 0}^{\infty} a_n$. Stejně jako v důkazu opačné implikace
 nalezneme ke zvolenému $\varepsilon>0$ číslo $n_0 \in \N$ takové, že pro $n
 \geq n_0$ platí 
 \[
  L - \varepsilon < \frac{a_n}{b_n} < L + \varepsilon.
 \]
 Číslo $\varepsilon$ zde volíme menší než $L$, aby platilo $L - \varepsilon >
 0$. Z dolního odhadu
 \[
  L - \varepsilon < \frac{a_n}{b_n}
 \]
 plyne úpravou
 \[
  \frac{b_n}{a_n} < \frac{1}{L-\varepsilon},
 \]
 čili
 \[
  b_n < \frac{a_n}{L-\varepsilon}
 \]
 pro $n \geq n_0$. Z \hyperref[prop:aritmetika-ciselnych-rad]{aritmetiky řad}
 opět platí, že řada $\sum_{n = 0}^{\infty} a_n / (L-\varepsilon)$ konverguje a
 \hyperref[prop:srovnavaci-kriterium]{srovnávací kritérium} skýtá kýženou
 konvergenci řady $\sum_{n = 0}^{\infty} b_n$.

 Dokážeme část (b). Ke zvolenému $\varepsilon>0$ nalezneme $n_0 \in \N$ takové,
 že pro $n \geq n_0$ platí
 \[
  \left| \frac{a_n}{b_n} - 0 \right| < \varepsilon.
 \]
 Pak ale máme odhad
 \[
  a_n < \varepsilon \cdot b_n
 \]
 a konvergence $\sum_{n = 0}^{\infty} a_n$ plyne z konvergence $\sum_{n =
 0}^{\infty} b_n$ zcela analogickým argumentem jako v~důkazu části (a).

 Důkaz části (c) je zcela obdobný důkazu části (b).
\end{thmproof}

Sekci o řadách s nezápornými členy završíme uvedením tří užitečných kritérií pro
zpyt konvergence takých řad, které o ní rozhodují přímo ze znalosti jejích
členů nevyžadujíce srovnání s řadami jinými. Výsledkem bude mimo jiné důkaz
konvergence jisté řady jsoucí mimořádně užitečnou pro srovnání s řadami, jejichž
členy jsou vyjádřeny jako podíly dvou polynomů.

\begin{theorem}{Cauchyho odmocninové kritérium}{cauchyho-odmocninove-kriterium}
 Ať $\sum_{n = 0}^{\infty} a_n$ je řada s nezápornými členy.
 \begin{enumerate}[label=(\alph*)]
  \item Existují-li $q \in (0,1)$ a $n_0 \in \N$ taková, že pro $n \geq n_0$
   platí $\sqrt[n]{a_n} \leq q$, pak řada $\sum_{n = 0}^{\infty} a_n$
   konverguje.
  \item Platí-li $\lim_{n \to \infty} \sqrt[n]{a_n} < 1$, pak řada $\sum_{n =
   0}^{\infty} a_n$ konverguje.
  \item Platí-li $\lim_{n \to \infty} \sqrt[n]{a_n} > 1$, pak řada $\sum_{n =
   0}^{\infty} a_n$ diverguje.
 \end{enumerate}
\end{theorem}

\begin{thmproof}
 Část (a) se dokáže snadno srovnáním s geometrickou řadou. Mějme tedy dáno $q
 \in (0,1)$ a $n_0 \in \N$ taková, že $\sqrt[n]{a_n} \leq q$ pro $n \geq n_0$.
 Tato nerovnost je ekvivalentní nerovnosti $a_n \leq q^{n}$. Ježto $|q|<1$, je
 řada $\sum_{n = 0}^{\infty} q^n$ podle \myref{úlohy}{prob:geometricka-rada}
 konvergentní. Pak je ale díky \hyperref[prop:srovnavaci-kriterium]{srovnávacímu
 kritériu} i $\sum_{n = 0}^{\infty} a_n$ konvergentní.

 Část (b) lze dokázat užitím závěru již dokázané části (a). Položme $L \coloneqq
 \lim_{n \to \infty} \sqrt[n]{a_n}$ a volme $\varepsilon \coloneqq (1-L) / 2$. Z
 předpokladu je $L < 1$, a tedy vskutku $\varepsilon > 0$. K tomuto
 $\varepsilon$ nalezneme z definice limity $n_0 \in \N$ takové, že pro $n \geq
 n_0$ platí
 \[
  |\sqrt[n]{a_n} - L| < \frac{1-L}{2}.
 \]
 Pak ale
 \[
  \sqrt[n]{a_n} < L + \frac{1-L}{2} = \frac{1+L}{2} < 1,
 \]
 kde poslední nerovnost plyne opět z toho, že $L < 1$. Pro jakoukoli volbu $q
 \in ((1+L) / 2,1)$ pročež platí $\sqrt[n]{a_n} \leq q$, kdykoli $n \geq n_0$. Z
 části (a) víme, že v tomto případě $\sum_{n = 0}^{\infty} a_n$ konverguje.

 Konečně, část (c) dokážeme nepřímo použitím
 \myref{lemmatu}{lem:nutna-podminka-existence-souctu-rady}. Jelikož $\lim_{n \to
 \infty} \sqrt[n]{a_n} > 1$, existuje $n_0 \in \N$ takové, že pro $n \geq n_0$
 máme $\sqrt[n]{a_n} > 1$. Potom ale rovněž $a_n > 1^{n} = 1$ pro $n \geq n_0$.
 To ale znamená, že $\lim_{n \to \infty} a_n \neq 0$, a tedy podle
 \myref{lemmatu}{lem:nutna-podminka-existence-souctu-rady} řada $\sum_{n =
 0}^{\infty} a_n$ diverguje.
\end{thmproof}

\begin{warning}{}{cauchyho-odmocninove-kriterium}
 \hyperref[thm:cauchyho-odmocninove-kriterium]{Předchozí věta} neříká nic o
 konvergenci $\sum_{n = 0}^{\infty} a_n$ v moment, kdy $\lim_{n \to \infty}
 \sqrt[n]{a_n} = 1$.

 Uvažme například řady $\sum_{n=1}^{\infty} 1 / n^2$ a $\sum_{n=1}^{\infty} 1$.
 První konvergentní podle \myref{lemmatu}{lem:konvergence-n^2} a druhá zřejmě
 divergentní. Platí však
 \[
  \lim_{n \to \infty} \sqrt[n]{\frac{1}{n^2}} = \lim_{n \to \infty} \sqrt[n]{1}
  = 1.
 \]
\end{warning}

\begin{theorem}{d'Alembertovo podílové kritérium}{dalembertovo-podilove-kriterium}
 Ať $\sum_{n = 0}^{\infty} a_n$ je řada s nezápornými členy.
 \begin{enumerate}[label=(\alph*)]
  \item Existují-li $q \in (0,1)$ a $n_0 \in \N$ taková, že pro každé $n \geq
   n_0$ platí $a_{n+1} / a_n \leq q$, pak řada $\sum_{n = 0}^{\infty} a_n$
   konverguje.
  \item Platí-li $\lim_{n \to \infty} a_{n+1} / a_n < 1$, pak $\sum_{n =
   0}^{\infty} a_n$ konverguje.
  \item Platí-li $\lim_{n \to \infty} a_{n+1} / a_n > 1$, pak $\sum_{n =
   0}^{\infty} a_n$ diverguje.
 \end{enumerate}
\end{theorem}
\begin{thmproof}
 Začneme částí (a). Ať jsou $q \in (0,1)$ a $n_0 \in \N$ dána. Dokážeme indukcí,
 že pro $n \geq n_0$ platí výrok
 \[
  a_n \leq q^{n-n_0}a_{n_0}.
 \]
 Pro $n = n_0$ máme
 \[
  q^{n_0 - n_0}a_{n_0} = a_{n_0},
 \]
 tedy výrok platí. Dále, z předpokladu věty máme $a_{n+1} / a_n \leq q$, čili
 $a_{n+1} \leq a_nq$. Odtud a z indukčního předpokladu,
 \[
  a_{n+1} \leq a_n q \leq q^{n-n_0}a_{n_0}q = q^{n+1-n_0}a_{n_0},
 \]
 jak jsme chtěli.

 Protože $|q|<1$, je řada $\sum_{n = 0}^{\infty} q^{n-n_0}$ konvergentní a z
 \hyperref[prop:aritmetika-ciselnych-rad]{aritmetiky řad} je i $\sum_{n =
 0}^{\infty} a_{n_0}q^{n-n_0}$ konvergentní. Z nerovnosti $a_n \leq
 a_{n_0}q^{n-n_0}$ plyne použitím
 \hyperref[prop:srovnavaci-kriterium]{srovnávacího kritéria}, že i řada $\sum_{n
 = 0}^{\infty} a_n$ je konvergentní.

 Pro důkaz (b) označme $L \coloneqq \lim_{n \to \infty} a_{n+1} / a_n$ a položme
 $\varepsilon \coloneqq (1-L) / 2$. Z předpokladu $L < 1$, pročež
 $\varepsilon>0$. K tomuto $\varepsilon$ nalezneme $n_0 \in \N$ takové, že pro
 $n \geq n_0$ platí
 \[
  \left| \frac{a_{n+1}}{a_n} - L\right| < \frac{1-L}{2},
 \]
 z čehož úpravou plyne
 \[
  \frac{a_{n+1}}{a_n} < L + \frac{1-L}{2} = \frac{1+L}{2} < 1.
 \]
 Volíme-li tedy $q \in ((1+L) / 2,1)$, pak pro toto $q$ a $n \geq n_0$ platí
 \[
  \frac{a_{n+1}}{a_n} \leq q,
 \]
 čili řada $\sum_{n = 0}^{\infty} a_n$ konverguje podle části (a).

 V části (c) předpokládáme, že $\lim_{n \to \infty} a_{n+1} / a_n > 1$, a tedy
 existuje $n_0 \in \N$, že pro $n \geq n_0$ je
 \[
  \frac{a_{n+1}}{a_n} > 1,
 \]
 odkud $a_{n+1} > a_n$. Posloupnost $a_n$ je tedy od jistého indexu rostoucí a
 nezáporná, a tedy její limita nemůže být rovna $0$. Podle
 \myref{lemmatu}{lem:nutna-podminka-existence-souctu-rady} řada $\sum_{n =
 0}^{\infty} a_n$ diverguje.
\end{thmproof}

\begin{warning}{}{dalembertovo-podilove-kriterium}
 Podobně jako \myref{věta}{thm:cauchyho-odmocninove-kriterium}, ani
 \hyperref[thm:dalembertovo-podilove-kriterium]{předchozí věta} netvrdí nic o
 konvergenci $\sum_{n = 0}^{\infty} a_n$ v případě, kdy $\lim_{n \to \infty}
 a_{n+1} / a_n = 1$. Řady $\sum_{n = 1}^{\infty} 1 / n^2$ a $\sum_{n=1}^{\infty}
 1$ opět ukazují, že ani nic v tomto případě tvrdit obecně nelze.
\end{warning}

Poslední kritérium konvergence řad s nezápornými členy, jež si uvedeme, je
veskrze překvapivým výsledkem. Ukazuje totiž, že (za jisté další podmínky) lze
každých $2^{n}$ (s rostoucím $n$) členů vyměnit za ten nejmenší nebo největší z
nich bez vlivu na konvergenci; na hodnotu součtu pochopitelně ano. Například už
pro $n = 20$ toto znamená, že beztrestně nahrazujeme více než milión čísel.

\begin{theorem}{Kondenzační kritérium}{kondenzacni-kriterium}
 Ať $a:\N \to [0,\infty)$ je \textbf{nerostoucí} a \textbf{nezáporná}
 posloupnost. Pak
 \[
  \sum_{n=0}^{\infty} a_n \text{ konverguje } \Leftrightarrow
  \sum_{n=0}^{\infty} 2^{n}a_{2^{n}} \text{ konverguje}.
 \]
\end{theorem}
\begin{thmproof}
 Máme k důkazu dvě implikace. Položme
 \[
  s_n \coloneqq \sum_{i=0}^{n} a_n, \quad t_n \coloneqq \sum_{i=0}^{n}
  2^{i}a_{2^{i}}.
 \]

 Pro důkaz implikace $(\Leftarrow)$ ověříme, že $s_n$ je shora omezená, což
 spolu s její monotonií zaručí existenci konečné limity. Předpokládáme tedy, že
 existuje konečná $L \coloneqq \lim_{n \to \infty} t_n$. Nechť $m \in \N$ je
 dáno. Nalezneme $k \in \N$ takové, že $2^{k} > m$. Potom
 \begin{align*}
  s_m &= \sum_{i=0}^{m} a_i \leq \sum_{i=0}^{2^{k}} a_i = a_0 + \clr{a_1} +
  \clb{(a_2 + a_3)} + \clg{(a_4 + a_5 + a_6 + a_7)} + \ldots + \clm{(a_{2^{k-1}}
  + \ldots + a_{2^{k} - 1})}\\
      &= a_0 + \clr{\sum_{i=2^{0}}^{2^{1}-1} a_i} + \clb{\sum_{i=2^{1}}^{2^2-1}
      a_i} + \clg{\sum_{i=2^2}^{2^{3}-1} a_i} + \ldots +
      \clm{\sum_{i=2^{k-1}}^{2^{k}-1} a_i} = a_0 + \sum_{j=0}^{k-1}
      \sum_{i=2^{j}}^{2^{j+1}-1} a_i.
 \end{align*}
 Nyní, protože $a$ je nerostoucí, dostáváme
 \[
  \sum_{i=2^{j}}^{2^{j+1}-1} a_i = a_{2^{j}} + a_{2^{j} + 1}\ldots +
  a_{2^{j+1}-1} \leq a_{2^{j}} + a_{2^{j}} + \ldots + a_{2^{j}} = 2^{j}a_{2^{j}}
 \]
 pro každé $j \in \N$. Spolu s předchozí nerovností dostaneme horní odhad
 \[
  s_m \leq a_0 + \sum_{j=0}^{k-1} \sum_{i=2^{j}}^{2^{j+1}-1} a_i \leq
  a_0 + \sum_{j=0}^{k-1} 2^{j}a_{2^{j}} = a_0 + t_{k-1} \leq a_0 + L,
 \]
 kde poslední nerovnost plyne z faktu, že $t_n$ je neklesající, a tedy vždy
 menší než svoje limita. Tím jsme ukázali, že $s_n$ jsouc shora omezená má
 konečnou limitu, a tedy $\sum_{n=0}^{\infty} a_n$ konverguje.

 V důkazu $( \Rightarrow )$ naopak předpokládáme, že existuje konečná $A
 \coloneqq \lim_{n \to \infty} s_n$. Ukážeme omezenost $t_n$ shora. Mějme dáno
 $k \in \N$. Nalezneme $m \in \N$ takové, že $2^{k} \leq m$. Potom
 \begin{align*}
  s_m &= \sum_{i=0}^{m} a_i \geq \sum_{i=0}^{2^{k}} a_i = a_0 + a_1 + \clr{a_2}
  + \clb{(a_3 + a_4)} + \clg{(a_5 + a_6 + a_7 + a_8)} + \ldots +
  \clm{(a_{2^{k-1}+1}+\ldots+a_{2^{k}})}\\
      &= a_0 + a_1 + \clr{\sum_{i=2^{0}+1}^{2^1} a_i} +
      \clb{\sum_{i=2^1+1}^{2^2} a_i} + \clg{\sum_{i=2^2+1}^{2^3} a_i} + \ldots +
      \clm{\sum_{i=2^{k-1}+1}^{2^{k}} a_i} = a_0 + a_1 + \sum_{j=1}^{k}
      \sum_{i=2^{j-1}+1}^{2^{j}} a_i\\
      & \overset{\text{$a$ je nezáporná}}{ \geq } \sum_{j=1}^k
      \sum_{i=2^{j-1}+1}^{2^{j}} a_i \overset{\text{$a$ je nerostoucí}}{ \geq }
      \sum_{j=1}^k 2^{j-1}a_{2^{j}} = 2^{-1} \sum_{j=1}^k 2^{j}a_{2^{j}} =
      2^{-1}t_k.
 \end{align*}
 Tím jsme dokázali, že $t_k \leq 2s_m \leq 2A$, čili $t_k$ je shora omezená a
 má z monotonie konečnou limitu. To znamená, že $\sum_{n=0}^{\infty}
 2^{n}a_{2^{n}}$ konverguje a důkaz je hotov.
\end{thmproof}

Krátké pojednání o řadách s nezápornými členy završíme, kterak jest bylo
zvěstěno, prozkoumáním konvergence jisté významné číselné řady.

\begin{proposition}{}{konvergence-n^c}
 Ať $c \in [0,\infty)$. Pak řada
 \[
  \sum_{n = 1}^{\infty} \frac{1}{n^{c}}
 \]
 konverguje právě tehdy, když $c > 1$.
\end{proposition}
\begin{propproof}
 Nejprve snadno nahlédneme, že řada diverguje, když $c \leq 1$. Z
 \myref{lemmatu}{lem:divergence-harmonicke-rady} víme, že $\sum_{n = 1}^{\infty}
 \frac{1}{n}$ diverguje. Protože pro libovolné $c \in [0,1]$ a každé $n \in \N$
 platí $n^{c} \leq n$, a tudíž $1 / n^{c} \geq 1 / n$, plyne ze
 \hyperref[prop:srovnavaci-kriterium]{srovnávacího kritéria}, že
 $\sum_{n=1}^{\infty} \frac{1}{n^{c}}$ diverguje, když $c \in [0,1]$.

 Pro důkaz konvergence v případě $c > 1$ použijeme
 \hyperref[thm:kondenzacni-kriterium]{kondenzačního kritéria}. Posloupnost $1 /
 n^{c}$ je jistě nezáporná a nerostoucí. Tedy, řada $\sum_{n=1}^{\infty} 1 /
 n^{c}$ konverguje právě tehdy, když konverguje řada
 \[
  \sum_{n=0}^{\infty} 2^{n} \cdot \frac{1}{(2^{n})^{c}} = \sum_{n = 0}^{\infty}
  2^{n-nc} = \sum_{n=0}^{\infty} (2^{1-c})^{n}.
 \]
 Poslední řada je geometrická s kvocientem $2^{1-c}$. Z
 \myref{úlohy}{prob:geometricka-rada} víme, že tato konverguje právě tehdy, když
 \[
  2^{1-c} < 1,
 \]
 což nastává právě tehdy, když $c > 1$.
\end{propproof}

Na závěr několik cvičení na použití výše dokázaných kritérií konvergence.

\begin{exercise}{}{random-rada}
 Vyšetřete konvergenci řady
 \[
  \sum_{n=0}^{\infty} \frac{n^22^{n}+3}{(n^{4}+1)^{6}}.
 \]
\end{exercise}
\begin{exercise}{}{random-rada-2}
 Vyšetřete konvergenci řady
 \[
  \sum_{n = 0}^{\infty} n^{c}(\sqrt{n+1}-\sqrt{n})
 \]
 v závislosti na konstantě $c \in \R$.
\end{exercise}
\begin{exercise}{těžké}{raabeovo-kriterium}
 Dokažte následující tvrzení, známé jako \emph{Raabeovo kritérium}: Nechť
 $\sum_{n = 0}^{\infty} a_n$ je řada s nezápornými členy.
 \begin{enumerate}[label=(\alph*)]
  \item Platí-li $\lim_{n \to \infty} n \left( \frac{a_n}{a_{n+1}} - 1 \right) >
   1$, pak řada $\sum_{n = 0}^{\infty} a_n$ konverguje.
  \item Existuje-li $n_0 \in \N$ takové, že pro $n \geq n_0$ je
  \[
   n \left( \frac{a_n}{a_{n+1}} - 1 \right) \leq 1,
  \]
  pak řada $\sum_{n = 0}^{\infty} a_n$ diverguje.
 \end{enumerate}
\end{exercise}
