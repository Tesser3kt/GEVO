\section{Definice limity posloupnosti}
\label{sec:definice-limity-posloupnosti}

Koncept posloupnosti je, na rozdíl od limity, velmi triviální. Je to vlastně
\uv{očíslovaná množina čísel}. Z každé množiny lze vyrobit posloupnost jejích
prvků tím, že každému přiřkneme nějaké pořadí. Toto \emph{přiřknutí} se
nejsnadněji definuje jako zobrazení z přirozených čísel -- to totiž přesně na
každý prvek kodomény zobrazí jeho pořadí.

\begin{definition}{Posloupnost}{posloupnost}
 Ať $X$ je množina. \emph{Posloupností} prvků z $X$ nazveme libovolné zobrazení
 \[
  a:\N \to X.
 \]
 Pro úsporu zápisu budeme psát $a_n$ místo $a(n)$ pro $n \in \N$. Navíc, je-li
 kodoména $X$ zřejmá z~kontextu, říkáme stručně, že $(a_n)_{n=0}^{\infty}$ je
 \emph{posloupnost.}
\end{definition}

\begin{remark}{}{usporadani-na-N}
 Vnímaví čtenáři sobě jistě povšimli, že jsme na $\N$ nedefinovali žádné
 \emph{uspořádání}. Ačkolivěk není tímto \hyperref[def:posloupnost]{definice
 posloupnosti} formálně nijak postižena, neodpovídá přirozenému vnímání, že
 prvek s číslem $1$ stojí před prvkem s číslem $5$ apod.

 Naštěstí, naše konstruktivní \hyperref[def:prirozena-cisla]{definice
 přirozených čísel} nabízí okamžité řešení. Využijeme toho, že každé přirozené
 číslo je podmnožinou svého následníka, a definujeme zkrátka uspořádání $ \leq $
 na $\N$ předpisem
 \[
  a \leq b \overset{\text{def}}{\iff} a \subseteq b.
 \]
 Fakt, že $ \subseteq $ je uspořádání, okamžitě implikuje, že $ \leq $ je rovněž
 uspořádání.
\end{remark}

Rozmyslíme si nyní dva pojmy pevně spjaté s posloupnostmi -- \emph{konvergence}
a \emph{limita}. Brzo si též ukážeme, že tyto dva pojmy jsou záměnné, ale zatím
je vnímáme odděleně. Navíc, budeme se odteď soustředit speciálně na posloupnosti
racionálních čísel, tj. zobrazení $\N \to \Q$, neboť jsou oním klíčem k
sestrojení reálných čísel.

Ze všech posloupností $\N \to \Q$ nás zajímá jeden konkrétní typ --
posloupnosti, vzdálenosti mezi jejichž prvky se postupně zmenšují. Tyto
posloupnosti, nazývané \emph{konvergentní} (z lat. con-vergere, \uv{ohýbat k
sobě}), se totiž vždy blíží k nějakému konkrétnímu bodu -- ke své \emph{limitě}.
Představa ze života může být například následující: říct, že se blížíme k
nějakému místu, je totéž, co tvrdit, že se vzdálenost mezi námi a oním místem s
každým dalším krokem zmenšuje. V moment, kdy své kroky směřujeme stále stejným
směrem, posloupnost vzdáleností mezi námi a tím místem tvoří konvergentní
posloupnost. Jestliže se pravidelně odkláníme, k místu nikdy nedorazíme a
posloupnost vzdáleností je pak \emph{divergentní} (tj.
\textbf{ne}konvergentní).

Do jazyka matematiky se věta \uv{vzdálenosti postupně zmenšují} překládá
obtížně. Jeden ne příliš elegantní, ale výpočetně užitečný a celkově oblíbený
způsob je následující: řekneme, že prvky posloupnosti jsou k sobě stále blíž,
když pro jakoukoli vzdálenost vždy dokážeme najít krok, od kterého dál jsou již
k sobě dva libovolné prvky u sebe blíž než tato daná vzdálenost. Důrazně
vyzýváme čtenáře, aby předchozí větu přečítali tak dlouho, dokud jim nedává
dobrý smysl. Podobné formulace se totiž vinou matematickou analýzou a jsou
základem uvažování o nekonečnu.

\begin{definition}{Konvergentní posloupnost}{konvergentni-posloupnost}
 Řekneme, že posloupnost $a:\N \to \Q$ je \emph{konvergentní}, když platí výrok
 \[
  \forall \varepsilon \in \Q, \varepsilon>0 \; \exists n_0 \in \N \; \forall m,n
  \geq n_0: |a_m - a_n| < \varepsilon.
 \]
\end{definition}

\begin{remark}{}{konvergence}
 Aplikujeme intuitivní vysvětlení \emph{zmenšování vzdálenosti} z odstavce nad
 \myref{definicí}{def:konvergentni-posloupnost} na jeho skutečnou definici.

 Výrok
 \[
  \forall \varepsilon \in \Q, \varepsilon>0 \; \exists n_0 \in \N \; \forall m,n
  \geq n_0: |a_m - a_n| < \varepsilon
 \]
 říká, že pro jakoukoli vzdálenost ($\varepsilon$) dokáži najít krok ($n_0$)
 takový, že vzdálenost dvou prvků v~libovolných dvou následujících krocích
 ($m,n$) už je menší než daná vzdálenost ($|a_n -a_m|<\varepsilon$).

 Slovo \uv{krok} je třeba vnímat volně -- myslíme pochopitelně \emph{pořadí} či
 \emph{indexy} prvků v posloupnosti. Pohled na racionální posloupnosti jako na
 \uv{kroky} činěné v racionálních číslech může být ovšem užitečný.
\end{remark}

Pojem \emph{limity}, představuje jakýsi bod, k němuž se posloupnost s každým
dalším krokem přibližuje, je vyjádřen výrazem podobného charakteru. Zde však
přichází na řadu ona \emph{děravost} racionálních čísel. Může se totiž stát, a
příklady zde uvedeme, že limita racionální posloupnosti není racionální číslo.

Učiňmež tedy dočasný obchvat a před samotnou definicí limity vyrobme reálná
čísla jednou z~přehoušlí možných cest. 

Ať $\mathcal{C}(\Q)$ značí množinu všech \textbf{konvergentních }racionálních
posloupností. Uvažme ekvivalenci $ \simeq $ na $\mathcal{C}(\Q)$ danou
\[
 a \simeq b \overset{\text{def}}{\iff} \forall \varepsilon>0 \; \exists n_0 \in \N
 \; \forall n \geq n_0: |a_n - b_n| < \varepsilon.
\]
Přeloženo do člověčtiny, $a \simeq b$, právě když se rozdíl mezi prvky těchto
posloupností se stejným pořadím neustále zmenšuje -- řekli bychom, že se
\emph{blíží k nule}. V rámci naší (zatím intuitivní) představy, že konvergentní
posloupnosti se blíží k nějakému bodu, dává smysl ztotožňovat posloupnosti,
které se blíží k bodu \emph{stejnému} -- stav, který vyjadřujeme tak, že se
jejich rozdíl blíží k nule.

Ve výsledku budeme definovat reálná čísla jako limity všech možných
konvergentních racionálních posloupností. Ježto však pozbýváme aparátu, abychom
koncepty limity a konvergence stmelili v jeden, jsme nuceni učinit mezikrok.

\begin{definition}{Reálná čísla}{realna-cisla}
 Množinu \emph{reálných čísel} tvoří všechny třídy ekvivalence konvergentních
 racionálních posloupností podle $ \simeq $. Symbolicky,
 \[
  \R \coloneqq \{[a]_{ \simeq } \mid a \in \mathcal{C}(\Q)\}.
 \]
\end{definition}


