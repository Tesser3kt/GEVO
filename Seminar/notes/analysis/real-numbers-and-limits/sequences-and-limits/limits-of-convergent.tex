\section{Limity konvergentních posloupností}
\label{sec:limity-konvergentnich-posloupnosti}

V této sekci dokážeme, že konvergentní posloupnosti mají limitu. Opačná
implikace, tj. že posloupnosti majíce limitu konvergují, je téměř triviální.
Potřebujeme dokázat jen jednu vlastnost absolutní hodnoty.

\begin{lemma}{Trojúhelníková nerovnost}{trojuhelnikova-nerovnost}
 Ať $x,y \in \Q$. Pak
 \[
  |x + y| \leq |x| + |y|.
 \]
\end{lemma}
\begin{lemproof}
 Absolutní hodnota $|x+y|$ je rovna buď $x + y$ (když $x+y \geq 0$) nebo $-x-y$
 (když $x+y<0$). Zřejmě $x \leq |x|$ a $-x \leq |x|$, podobně $y \leq |y|$ a
 $-y \leq |y|$.

 Pak je ale $x + y \leq |x| + |y|$ a též $-x+(-y) \leq |x| + |y|$. Tím je důkaz
 hotov.
\end{lemproof}

Ježto bychom však rádi dokazovali všechna tvrzení již pro reálná čísla, ukažme
si nejprve, jak se dají sčítat a násobit. Dokážeme rovněž, že $\R$ -- stejně
jako $\Q$ -- tvoří těleso. Začneme tím, že se naučíme sčítat a násobit
konvergentní posloupnosti.

Ať $x,y \in \mathcal{C}(\Q)$ jsou dvě konvergentní racionální posloupností.
Operace $+$ a $ \cdot $ na $\mathcal{C}(\Q)$ definujeme velmi přirozeně.
Zkrátka, $(x+y)(n) = x(n) + y(n)$ a $(x \cdot y)(n) = x(n) \cdot y(n)$, tj.
prvek na místě $n$ posloupnosti $x+y$ je součet prvků na místech $n$
posloupností $x$ a $y$. Abychom ovšem získali skutečně operace na
$\mathcal{C}(\Q)$, musíme ověřit, že $x+y$ i $x \cdot y$ jsou konvergentní.

Ať tedy máme dána dvě kladná čísla $\varepsilon_x$ a $\varepsilon_y$. K nim
najdeme $n_x,n_y \in \N$ taková, že pro všechna $m,n \geq n_x$ platí $|x_n -
x_m|<\varepsilon_x$ a pro všechna $m,n \geq n_y$ zase $|y_n - y_m| <
\varepsilon_y$. Když zvolíme $n_0 \coloneqq \max(n_x,n_y)$ a $\varepsilon
\coloneqq \varepsilon_x + \varepsilon_y$, pak pro všechna $m,n \geq n_0$ je jak
$|x_n - x_m|<\varepsilon$, tak $|y_n - y_m|<\varepsilon$.

Předchozí odstavec se může snadno zdát šílenou směsicí symbolů. Ve skutečnosti
však formálně vykládá triviální úvahu. Řekněme, že máme vzdálenosti
$\varepsilon_x$ a $\varepsilon_y$. Z
\hyperref[def:konvergentni-posloupnost]{definice konvergence} existuje krok
$n_x$, od kterého jsou již od sebe dva prvky posloupnosti $x$ blíže než
$\varepsilon_x$, a též krok $n_y$, od kterého jsou již od sebe prvky posloupnosti
$y$ blíže než $\varepsilon_y.$
