\section{Limity konvergentních posloupností}
\label{sec:limity-konvergentnich-posloupnosti}

V této sekci dokážeme, že konvergentní posloupnosti mají limitu. Opačná
implikace, tj. že posloupnosti majíce limitu konvergují, je téměř triviální.
Potřebujeme dokázat jen jednu vlastnost absolutní hodnoty.

\begin{lemma}{Trojúhelníková nerovnost}{trojuhelnikova-nerovnost}
 Ať $x,y \in \Q$. Pak
 \[
  |x + y| \leq |x| + |y|.
 \]
\end{lemma}
\begin{lemproof}
 Absolutní hodnota $|x+y|$ je rovna buď $x + y$ (když $x+y \geq 0$) nebo $-x-y$
 (když $x+y<0$). Zřejmě $x \leq |x|$ a $-x \leq |x|$, podobně $y \leq |y|$ a
 $-y \leq |y|$.

 Pak je ale $x + y \leq |x| + |y|$ a též $-x+(-y) \leq |x| + |y|$. Tím je důkaz
 hotov.
\end{lemproof}

Trojúhelníková nerovnost poskytuje snadné důkazy mnoha užitečných dílčích
tvrzení o posloupnostech. Příkladem je následující cvičení.

\begin{exercise}{Jednoznačnost limity}{jednoznacnost-limity}
 Dokažte, že každá posloupnost $a:\N \to \Q$ má nejvýše jednu limitu. Hint:
 použijte \hyperref[lem:trojuhelnikova-nerovnost]{trojúhelníkovou nerovnost}.
\end{exercise}

Ježto bychom však rádi dokazovali všechna tvrzení již pro reálná čísla, ukažme
si nejprve, jak se dají sčítat a násobit. Dokážeme rovněž, že $\R$ -- stejně
jako $\Q$ -- tvoří těleso. Začneme tím, že se naučíme sčítat a násobit
konvergentní posloupnosti.

Ať $x,y \in \mathcal{C}(\Q)$ jsou dvě konvergentní racionální posloupností.
Operace $+$ a $ \cdot $ na $\mathcal{C}(\Q)$ definujeme velmi přirozeně.
Zkrátka, $(x+y)(n) \coloneqq x(n) + y(n)$ a $(x \cdot y)(n) \coloneqq x(n) \cdot
y(n)$, tj. prvek na místě $n$ posloupnosti $x+y$ je součet prvků na místech $n$
posloupností $x$ a $y$. Abychom ovšem získali skutečně operace na
$\mathcal{C}(\Q)$, musíme ověřit, že $x+y$ i $x \cdot y$ jsou konvergentní.

Nechť dáno jest $\varepsilon>0$. Chceme ukázat, že umíme najít $n_0 \in \N$, aby
\[
 |(x_n + y_n) - (x_m + y_m)| < \varepsilon,
\]
kdykoli $m,n \geq n_0$. Protože jak $x$ tak $y$ konverguje, již umíme pro
libovolná $\varepsilon_x,\varepsilon_y>0$ najít $n_x$ a $n_y$ taková, že $|x_n -
x_m| < \varepsilon_x$, kdykoli $m,n \geq n_x$, a podobně $|y_n -
y_m|<\varepsilon_y$, kdykoli $m,n \geq n_y$. Položme tedy $\varepsilon_x =
\varepsilon_y \coloneqq \varepsilon / 2$ a $n_0 \coloneqq \max(n_x,n_y)$. Potom
můžeme užitím \hyperref[lem:trojuhelnikova-nerovnost]{trojúhelníkové nerovnosti}
odhadnout
\[
 |(x_n + y_n) - (x_m + y_m)| = |(x_n - x_m) + (y_n - y_m)| \leq |x_n - x_m| +
 |y_n - y_m| < \varepsilon_x + \varepsilon_y = \varepsilon,
\]
čili $x + y$ konverguje.

Předchozí odstavec se může snadno zdát šílenou směsicí symbolů. Ve skutečnosti
však formálně vykládá triviální úvahu. Máme najít pořadí, od kterého jsou prvky
součtu $x + y$ u sebe blíž než nějaká daná vzdálenost. Poněvadž $x$ i $y$
konvergují, stačí přeci vzít větší z pořadí, od kterých je jak rozdíl prvků $x$,
tak rozdíl prvků $y$ menší než polovina dané vzdálenosti.

Velmi obdobnou manipulaci lze provést k důkazu konvergence $x \cdot y$.
Ponecháváme jej čtenářům jako (ne zcela snadné) cvičení.

\begin{exercise}{}{}
 Dokažte, že jsou-li $x,y$ konvergentní posloupnosti racionálních čísel, pak je
 posloupnost $x \cdot y$ rovněž konvergentní.
\end{exercise}

Racionální čísla jsou přirozeně součástí reálných prostřednictvím zobrazení
\begin{equation}
 \label{eq:Q-into-R}
 \begin{split}
  \xi: \Q &\hookrightarrow \R,\\
  q &\mapsto [(q)],
 \end{split}
\end{equation}
kde $(q)$ značí posloupnost $n \mapsto q$ pro všechna $n \in \N$ a $[(q)]$ její
třídu ekvivalence podle $ \simeq $.

\begin{warning}{}{}
 Tvrdíme pouze, že $\Q$ jsou \emph{součástí} $\R$, kde slovu \emph{součást}
 záměrně není dán rigorózní smysl. Racionální čísla totiž (aspoň po dobu naší
 dočasné \hyperref[def:realna-cisla]{definice reálných čísel}) nejsou v žádném
 smyslu podmnožinou čísel reálných.

 Matematici ale často ztotožňujeme doménu prostého zobrazení s jeho obrazem
 (neboť mezi těmito množinami vždy existuje bijekce). V tomto smyslu mohou být
 $\Q$ vnímána jako podmnožina $\R$, ztotožníme-li racionální čísla s obrazem
 zobrazení $\xi$ z \eqref{eq:Q-into-R}. Toto ztotožnění znamená vnímat
 racionální číslo $q \in \Q$ jako konvergentní posloupnost samých čísel $q$.
\end{warning}

\begin{exercise}{}{}
 Dokažte, že zobrazení $\xi$ z \eqref{eq:Q-into-R} je
 \begin{itemize}
  \item dobře definované -- tzn. že když $p = q$, pak $[(p)] = [(q)]$ -- a
  \item prosté.
 \end{itemize}
\end{exercise}

Jelikož $\Q$ je těleso, speciálně tedy obsahuje $0$ a $1$, $\R$ je
(prostřednictvím $\xi$ z \eqref{eq:Q-into-R}) obsahuje rovněž. Pro stručnost
budeme číslem $0 \in \R$ značit třídu ekvivalence posloupnosti samých nul a
číslem $1 \in \R$ třídu ekvivalence posloupnosti samých jednotek. Ověříme, že se
skutečně jedná o neutrální prvky ke sčítání a násobení.

Je třeba si rozmyslet, že pro každou posloupnost $x \in \mathcal{C}(\Q)$ platí
$x + 0 = x$ a $x \cdot 1 = x$, kde, opět, čísla $0$ a $1$ ve skutečnosti
znamenají nekonečné posloupnosti těchto čísel. Obě rovnosti jsou však zřejmé z
definice, neboť $(x+0)(n) = x_n + 0 = x_n = x(n)$ a $(x \cdot 1)(n) = x_n \cdot
1 = x_n = x(n)$ pro všechna $n \in \N$.

Konečně, rozšíříme rovněž $-$ a $^{-1}$ na $\R$. Pro libovolnou posloupnost $x
\in \mathcal{C}(\Q)$ definujeme zkrátka $(-x)(n) \coloneqq -x(n)$. S $^{-1}$ je
situace lehce komplikovanější. Totiž, pouze \textbf{nenulová} racionální čísla
mají svůj inverz k násobení. Zde je třeba zpozorovat, že \textbf{konvergentní}
posloupnost, která by však měla nekonečně mnoho prvků nulových, už musí mít od
nějakého kroku \textbf{všechny} prvky nulové, jinak by totiž nemohla
konvergovat. Vskutku, představme si, že $x$ je posloupnost taková, že $x_n = 0$
pro nekonečně mnoho přirozených čísel $n \in \N$. Pak ale ať zvolím $n_0 \in \N$
jakkoliv, vždy existuje $m \geq n_0$ takové, že $x_m = 0$. Vezměme $n \geq n_0$
libovolné. Pokud $x_n \neq 0$, pak můžeme vzít třeba $\varepsilon \coloneqq
|x_n| / 2$ a bude platit, že $|x_n - x_m| > \varepsilon$, což je dokonalý zápor
\hyperref[def:konvergentni-posloupnost]{definice konvergence}. Z toho plyne, že
$x_n$ musí být $0$ pro $n \geq n_0$ a odtud dále, že $x \simeq 0$. Čili, pouze
nulové posloupnosti v $\R$ nemají inverz vzhledem k $ \cdot $.

Právě provedená úvaha nám umožňuje definovat $^{-1}$ pro posloupnosti $x \in
\mathcal{C}(\Q)$ takové, že $x \not\simeq 0$, následovně:
\[
 (x^{-1})(n) \coloneqq \begin{cases}
  x(n)^{-1},& \text{když } x(n) \neq 0,\\
  0, &\text{když } x(n) = 0.
 \end{cases}
\]
 
Je snadné uvidět, že $-x$ je inverzem k $x$ vzhledem k $+$ a $x^{-1}$ je
inverzem k $x \neq 0$ vzhledem k $ \cdot $. Vskutku, máme
\[
 (x + (-x))(n) = x_n + (-x_n) = 0,
\]
tedy v tomto případě je $(x + (-x))$ přímo \textbf{rovna} nulové posloupnosti. V
případě $^{-1}$ dostáváme pro $x \not\simeq 0$
\[
 (x \cdot x^{-1})(n) = \begin{cases}
  x_n \cdot x_n^{-1} = 1,& \text{když } x_n \neq 0,\\
  x_n \cdot 0 = 0,& \text{když } x_n = 0.
 \end{cases}
\]
Ergo, $x \cdot x^{-1}$ je rovna posloupnosti samých jedniček až na konečně mnoho
nul, protože, jak jsme si již rozmysleli, $x$ nemůže mít nekonečně $0$ a zároveň
nebýt v relaci $ \simeq $ s nulovou posloupností, jinak by nebyla konvergentní.

Shrneme-li řád předchozích úvah, získáme oprávnění tvrdit, že
\[
 (\R,+,-,[(0)], \cdot,^{-1},[(1)])
\]
je těleso. Tento fakt je do budoucna pochopitelně zásadní; teď se však můžeme
těšit znalostí, že jsme přechodem od $\Q$ k $\R$ neztratili symetrické rysy
původní množiny.

Přikročmež již však k důkazu existence limity každé konvergentní posloupnosti.
Fakt, že existence limity implikuje konvergenci, plyne přímo z
\hyperref[lem:trojuhelnikova-nerovnost]{trojúhelníkové nerovnosti}.

\begin{lemma}{}{}
 Každá posloupnost majíc limitu je konvergentní.
\end{lemma}
\begin{lemproof}
 Ať $a:\N \to \Q$ je posloupnost s limitou $L$. Pak pro každé $\varepsilon_L>0$
 existuje $n_L \in \N$ takové, že $|a_n - L| < \varepsilon_L$ pro všechna $n
 \geq n_L$.

 Ať je dáno $\varepsilon>0$. Chceme ukázat, že $|a_m - a_n| < \varepsilon$ pro
 všechna $m,n$ větší než vhodné $n_0 \in \N$. Položme tedy $n_0 \coloneqq n_L$ a
 $\varepsilon_L \coloneqq \varepsilon / 2$. Potom pro všechna $m,n \geq n_0 =
 n_L$ máme
 \[
  |a_m - a_n| = |a_m - a_n - L + L| = |(a_n - L) + (L - a_m)| \leq |a_n - L| +
  |L - a_m| < \varepsilon_L + \varepsilon_L = \varepsilon,
 \]
 čili $a$ konverguje.
\end{lemproof}

\subsection{Úplnost reálných čísel}
\label{ssec:uplnost-realnych-cisel}

K důkazu existence limity každé konvergentní posloupnosti potřebujeme
prozpytovat vztah racionálních a reálných čísel podrobněji. Konkrétně
potřebujeme ukázat, že $\Q$ jsou tzv. \emph{hustá} v $\R$, tj. že ke každému
reálnému číslu existuje racionální číslo, které je mu nekonečně blízko. Zde jsme
opět implicitně ztotožnili racionální čísla s třídami ekvivalence konstantních
posloupností. Na základě toho budeme totiž moci tvrdit, že reálná čísla jsou
tzv. \emph{kompletní}, což přesně znamená, že každá konvergentní posloupnost
reálných čísel má reálnou limitu.

Nejprve si ovšem musíme rozmyslet, co vlastně míníme posloupností
\emph{reálných} čísel. Pochopitelně, zobrazení $x:\N \to \R$ poskytuje validní
definici, ale uvědomme si, že teď vlastně uvažujeme posloupnosti, jejichž prvky
jsou třídy ekvivalence konvergentních racionálních posloupností.

Abychom směli hovořit o konvergentních \emph{reálných} posloupnostech, rozšíříme
absolutní hodnotu $| \cdot |$ z $\Q$ na $\R$ zkrátka předpisem $|[(x_n)]|
\coloneqq [(|x_n|)]$. Napíšeme-li tedy $|x| \leq K$ pro reálná čísla $x,K \in
\R$, pak tím doslova myslíme $[(|x_n|)] \leq [(K_n)]$, tj. $|x_n| \leq K_n$ pro
všechna $n \in \N$, kde $x_n,K_n$ jsou nyní již čísla ryze rozumná čili
racionální.

Rozepíšeme-li si tedy podrobně, co znamená, že je posloupnost $x:\N \to \R$
konvergentní, dostaneme pro dané $\varepsilon>0$, vhodné $n_0 \in \N$ a $m,n
\geq n_0$ nerovnost $|x_n - x_m| < \varepsilon$. Ovšem, $x_n$ i $x_m$ jsou samy
o sobě \textbf{posloupnosti} racionálních čísel, tedy poslední nerovnost plně
rozepsána dí
\[
 |(x_n)_{k} - (x_m)_{k}| < \varepsilon \; \forall k \in \N.
\]


