\section{Metody výpočtů limit}
\label{sec:metody-vypoctu-limit}

Tato sekce je veskrze výpočetní, věnována způsobům určování limit rozličných
posloupností -- primárně těch zadaných vzorcem pro $n$-tý člen. Obecně
neexistuje algoritmus pro výpočet limity posloupnosti a například limity
posloupností zadaných rekurentně (další člen je vypočten jako kombinace
předchozích) je často obtížné určit. K jejich výpočtu bývá užito metod z
lineární algebry a obecně metod teorie diskrétních systémů zcela mimo rozsah
tohoto textu.

Přinejmenším v případě limit zadaných \uv{hezkými} vzorci čítajícími podíly
mnohočlenů a odmocnin je obyčejně možné algebraickými úpravami dojít k výsledku.
Uvedeme si pár stěžejních tvrzení sloužících tomuto účelu.

K důkazu prvního bude užitečná následující nerovnost, kterou přenecháváme
čtenáři jako (snadné) cvičení.

\begin{exercise}{}{random-abs-nerovnost}
 Dokažte, že pro čísla $x,y \in \R$ platí
 \[
  | |x| - |y| | \leq |x - y|.
 \]
\end{exercise}

\begin{theorem}{Aritmetika limit}{aritmetika-limit}
 Ať $a,b:\N \to \R$ jsou reálné posloupnosti mající limitu (ale klidně i
 nekonečnou). Pak
 \begin{enumerate}[label=(\alph*)]
  \item $\lim (a + b) = \lim a + \lim b$, je-li pravá strana definována;
  \item $\lim (a \cdot b) = \lim a \cdot \lim b$, je-li pravá strana definována;
  \item $\lim (a / b) = \lim a / \lim b$, platí-li $b \not\simeq 0$ a pravá
   strana je definována.
 \end{enumerate}
\end{theorem}
\begin{thmproof}
 Důkaz této věty je ryze výpočetního charakteru a využívá vhodně zvolených
 odhadů. Vzhledem k tomu, že povolujeme i nekonečné limity, je třeba důkaz
 každého bodu rozložit na případy. Položme $A \coloneqq \lim a, B \coloneqq \lim
 b$.

 \textbf{\emph{Případ $A,B \in \R$}.}\\
 Nejprve budeme předpokládat, že $A,B \in \R$. Pro dané $\varepsilon>0$ existují
 $n_a,n_b \in \N$ taková, že pro každé $n \geq n_a$ platí $|a_n-A|<\varepsilon$
 a pro každé $n \geq n_b$ zas $|b_n-B|<\varepsilon$. Zvolíme-li $n_0 \coloneqq
 \max(n_a,n_b)$, pak pro $n \geq n_0$ platí oba odhady zároveň. Potom ale,
 použitím \hyperref[lem:trojuhelnikova-nerovnost]{trojúhelníkové nerovnosti},
 dostaneme
 \[
  |(a_n+b_n)-(A+B)| = |(a_n-A)+(b_n-B)| \leq |a_n-A| + |b_n-B| <
  \varepsilon+\varepsilon = 2\varepsilon,
 \]
 čili $\lim (a+b) = A+B$. Pro důkaz vzorce pro součin a podíl, musíme navíc
 využít \myref{lemmatu}{lem:konvergentni-omezena}, tedy faktu, že konvergentní
 posloupnosti jsou omezené. Pročež najdeme $C_b \in \R$ takové, že od určitého
 indexu $n_1 \in \N$ dále platí $|b_n| \leq C_b$. Volme nově $n_0 \coloneqq
 \max(n_a,n_b,n_1)$ a pro $n \geq n_0$ počítejme
 \begin{align*}
  |a_n \cdot b_n - A \cdot B| &= |a_n \cdot b_n - b_n \cdot A + b_n \cdot A -
  A \cdot B| = |b_n(a_n - A) + A(b_n - B)|\\
                              & \leq |b_n(a_n - A)| + |A(b_n-B)| = |b_n| \cdot
                              |a_n-A| + |A| \cdot |b_n-B| \\
                              &< |C_b| \cdot \varepsilon + |A| \cdot
                              \varepsilon = (|C_b| + |A|) \cdot \varepsilon.
 \end{align*}
 Protože $|C_b| + |A|$ je kladná konstanta nezávislá na $\varepsilon$, dokazuje
 odhad výše, že $\lim (a \cdot b) = A \cdot B$. Konečně, v případě podílu volme
 $\varepsilon_b = |B| / 2$. K tomuto $\varepsilon_b$ nalezněme $n'_b \in \N$
 takové, že pro $n \geq n'_b$ platí $|b_n - B| < \varepsilon_b$. Poslední
 nerovnost spolu s \myref{cvičením}{exer:random-abs-nerovnost} znamená, že $|
 |b_n| - |B| | < \varepsilon$. Tento vztah si rozepíšeme na
 \[
  |B| - \varepsilon_b < |b_n| < |B| + \varepsilon_b.
 \]
 Levá z těchto nerovností je pak ekvivalentní $|b_n| > |B| / 2$ neboli $1 /
 |b_n| < 2 / |B|$. Položme $n_0 \coloneqq \max(n_a,n_b,n'_b)$. Potom pro $n \geq
 n_0$ máme
 \begin{align*}
  \left| \frac{a_n}{b_n} - \frac{A}{B} \right| &= \left| \frac{a_nB -
  b_nA}{b_nB} \right| = \left| \frac{a_nB - AB + AB - b_nA}{b_nB} \right| \leq
  \left| \frac{B(a_n-A)}{b_nB} \right| + \left| \frac{A(B - b_n)}{b_nB} \right|
  \\
                                               &= \frac{1}{|b_n|}\left(|a_n-A| +
                                               \frac{|A|}{|B|}|B-b_n| \right) <
                                               \frac{2\varepsilon}{|B|}\left(1 +
                                               \frac{|A|}{|B|}\right).
 \end{align*}
 Protože $|A|$ i $|B|$ jsou konstanty nezávislé na $\varepsilon$, toto znamená,
 že $\lim (a / b) = A / B$.

 \textbf{\emph{Případ $A = \pm \infty, B \in \R \setminus \{0\}$.}}\\
 Předpokládejme, že $\lim a = \infty$; případ $\lim a = -\infty$ se dokáže v
 zásadě identicky. Pak pro dané $\varepsilon_a$ existuje $n_a \in \N$ takové, že
 pro $n \geq n_a$ platí $a_n > \varepsilon_a$. Podle
 \myref{lemmatu}{lem:konvergentni-omezena} je posloupnost $b$ omezená, čili
 existuje $C_b > 0$ takové, že $|b_n| \leq C_b$ pro všechna $n \in \N$. Potom
 pro $n \geq n_a$ máme
 \[
  a_n + b_n \geq a_n - C_b > \varepsilon_a - C_b.
 \]
 Jelikož $C_b$ je konstantní, plyne z tohoto odhadu, že $\lim (a + b) = \infty =
 A + B$.

 Pro důkaz součinu nejprve ať $B > 0$. Pak existuje konstanta $C_b > 0$ a $n_b
 \in \N$ takové, že pro $n \geq n_b$ je $b_n \geq C_b$. Pročež, pro libovolné
 $C_a > 0$ a $n \geq \max(n_a,n_b)$ dostaneme
 \[
  a_n \cdot b_n \geq \varepsilon_a \cdot C_b.
 \]
 čili $\lim (a \cdot b) = \infty = A \cdot B$. Z omezenosti (plynoucí z
 konvergence) $b$ pak zase existují $n_b'$ a $K_b > 0$ takové, že $b_n \leq
 K_b$, čili též $1 / b_n \geq 1 / K_b$, pro $n \geq n'_b$. Pro $n \geq
 \max(n_a,n'_b)$ tedy
 \[
  \frac{a_n}{b_n} \geq \frac{\varepsilon_a}{K_b},
 \]
 což dokazuje $\lim (a / b) = \infty = A / B$. Velmi podobně se řeší případ $B <
 0$.

 Zdlouhavý důkaz zakončíme komentářem, že případ $A \in \R \setminus \{0\}, B =
 \pm \infty$ je symetrický předchozímu a případy $A = 0, B = \pm \infty$, též $A
 = \pm \infty, B = 0$ a konečně $A = \pm \infty, B = \pm \infty$ jsou triviální.
\end{thmproof}

\hyperref[thm:aritmetika-limit]{Věta o aritmetice limit} je zcela nejužitečnější
tvrzení k jejich výpočtu, neboť umožňuje limitu výrazu rozdělit na mnoho menších
\uv{podlimit}, jejichž výpočet je snadný. Další dvě lemmata jsou často též
dobrými sluhy.

\begin{lemma}{Limita odmocniny}{limita-odmocniny}
 Ať $a:\N \to [0,\infty)$ je posloupnost \emph{nezáporných} čísel. Ať též
    $\lim a = A$ (speciálně tedy předpokládáme, že $\lim a$ existuje). Potom
 \[
  \lim_{n \to \infty} \sqrt[k]{a_n} = \sqrt[k]{A}
 \]
 pro každé $k \in \N$.
\end{lemma}
\begin{lemproof}
 Zdlouhavý a technický. Ambiciózní čtenáři jsou zváni, aby se o něj pokusili.
\end{lemproof}

\begin{lemma}{O dvou strážnících}{o-dvou-straznicich}
 Ať $a,b,c:\N \to \R$ jsou posloupnosti reálných čísel a $L \coloneqq \lim a =
 \lim c$. Pokud existuje $n_0 \in \N$ takové, že pro každé $n \geq n_0$ platí
 $a_n \leq b_n \leq c_n$, pak $\lim b = L$.
\end{lemma}
\begin{lemproof}
 Protože $\lim a = L$ a též $\lim c = L$, nalezneme pro dané $\varepsilon>0$
 index $n_1 \in \N$ takový, že pro $n \geq n_1$ platí dva odhady:
 \[
  |a_n - L| < \varepsilon \quad \text{a} \quad |c_n - L| < \varepsilon.
 \]
 Potom ovšem $a_n > L - \varepsilon$ a $c_n < L + \varepsilon$. Z předpokladu
 existuje $n_b \in \N$ takové, že $a_n \leq b_n \leq c_n$ pro $n \geq n_b$.
 Zvolíme-li tedy $n_0 \coloneqq \max(n_1,n_b)$, pak pro $n \geq n_0$ platí
 \[
  L - \varepsilon < a_n \leq b_n \leq c_n < L + \varepsilon.
 \]
 Sloučením obou nerovností dostaneme pro $n \geq n_0$ odhad $|b_n -
 L|<\varepsilon$, čili $\lim b = L$.
\end{lemproof}

\begin{figure}[ht]
 \centering
 \begin{tikzpicture}
  \tkzInit[xmin=0,xmax=10,ymin=0,ymax=3]
  \foreach \n in {0,1,...,18} {
   \tkzDefPoint(0.5 * (\n + 1),0){x\n}
   \tkzDrawPoint[shape=cross](x\n)
   \tkzLabelPoint[below](x\n){$\n$}
  }
  \tkzLabelPoint[below,color=RoyalPurple](x11){$11$}
  \foreach \y in {1,2,3,4,5,6} {
   \tkzDefPoint(0,0.5 * \y){y\y}
   \tkzDrawPoint[shape=cross](y\y)
   \tkzLabelPoint[left](y\y){$\y$}
  }
  \tkzDefPoint(0.5,2.8){c0}
  \tkzDefPoint(1,0.9){c1}
  \tkzDefPoint(1.5,2.3){c2}
  \tkzDefPoint(2,1){c3}
  \tkzDefPoint(2.5,1.1){c4}
  \tkzDefPoint(3,0.3){c5}
  \tkzDefPoint(3.5,2.4){c6}
  \tkzDefPoint(4,1.2){c7}
  \tkzDefPoint(4.5,3.2){c8}
  \tkzDefPoint(5,2.6){c9}
  \tkzDefPoint(5.5,1.9){c10}
  \tkzDefPoint(6,2.4){c11}
  \tkzDefPoint(6.5,2.2){c12}
  \tkzDefPoint(7,1.9){c13}
  \tkzDefPoint(7.5,2){c14}
  \tkzDefPoint(8,1.8){c15}
  \tkzDefPoint(8.5,2.05){c16}
  \tkzDefPoint(9,1.95){c17}
  \tkzDefPoint(9.5,1.85){c18}

  \tkzDefPoint(0.5,1){b0}
  \tkzDefPoint(1,2.1){b1}
  \tkzDefPoint(1.5,0.3){b2}
  \tkzDefPoint(2,1.1){b3}
  \tkzDefPoint(2.5,1.6){b4}
  \tkzDefPoint(3,0.8){b5}
  \tkzDefPoint(3.5,2){b6}
  \tkzDefPoint(4,0.9){b7}
  \tkzDefPoint(4.5,0.2){b8}
  \tkzDefPoint(5,2.2){b9}
  \tkzDefPoint(5.5,0.9){b10}
  \tkzDefPoint(6,2){b11}
  \tkzDefPoint(6.5,1.8){b12}
  \tkzDefPoint(7,1){b13}
  \tkzDefPoint(7.5,1.6){b14}
  \tkzDefPoint(8,1.6){b15}
  \tkzDefPoint(8.5,1.4){b16}
  \tkzDefPoint(9,1.7){b17}
  \tkzDefPoint(9.5,1.65){b18}
  
  \tkzDefPoint(0.5,1.5){a0}
  \tkzDefPoint(1,0.2){a1}
  \tkzDefPoint(1.5,3){a2}
  \tkzDefPoint(2,2.7){a3}
  \tkzDefPoint(2.5,1.7){a4}
  \tkzDefPoint(3,0.2){a5}
  \tkzDefPoint(3.5,2.2){a6}
  \tkzDefPoint(4,1){a7}
  \tkzDefPoint(4.5,0.7){a8}
  \tkzDefPoint(5,0.1){a9}
  \tkzDefPoint(5.5,1.3){a10}
  \tkzDefPoint(6,0.8){a11}
  \tkzDefPoint(6.5,1.5){a12}
  \tkzDefPoint(7,0.6){a13}
  \tkzDefPoint(7.5,1){a14}
  \tkzDefPoint(8,1.4){a15}
  \tkzDefPoint(8.5,0.9){a16}
  \tkzDefPoint(9,1.5){a17}
  \tkzDefPoint(9.5,1.55){a18}

  \tkzDrawX[>=latex,label={$n$}]
  \tkzDrawY[>=latex,label={$\clr{a_n},\clg{b_n},\clb{c_n}$}]

  \tkzDefPoint(0,1.6){O}
  \tkzDrawPoint[size=4,color=RoyalPurple](O)
  \tkzLabelPoint[right,color=RoyalPurple](O){$L$}
  \tkzDefPoint(0.5,1.6){P}
  \tkzDefPoint(6.5,1.6){P2}
  \tkzDrawLine[add=0 and 0.7,dashed,color=RoyalPurple,thick](P,P2)

  \tkzDrawPoints[color=RoyalBlue](c0,c1,c2,c3,c4,c5,c6,c7,c8,c9,c10,c11,c12,c13,c14,c15,c16,c17,c18)
  \tkzDrawPoints[color=ForestGreen](b0,b1,b2,b3,b4,b5,b6,b7,b8,b9,b10,b11,b12,b13,b14,b15,b16,b17,b18)
  \tkzDrawPoints[color=BrickRed](a0,a1,a2,a3,a4,a5,a6,a7,a8,a9,a10,a11,a12,a13,a14,a15,a16,a17,a18)

  \foreach \n in {11,12,...,17} {
   \pgfmathparse{\n+1}
   \edef\m{\pgfmathresult}
   \tkzDrawSegment[dashed,color=BrickRed](a\n,a\m)
   \tkzDrawSegment[dashed,color=ForestGreen](b\n,b\m)
   \tkzDrawSegment[dashed,color=RoyalBlue](c\n,c\m)
  }
  \tkzLabelPoint[above,color=RoyalPurple](x11){$n_0$}
  \tkzDefPoint(6,3){wtv}
  \tkzDrawLine[add=-0.15 and 0,dashed,color=RoyalPurple](x11,wtv)
 \end{tikzpicture}

 \caption{Lemma \hyperref[lem:o-dvou-straznicich]{o dvou strážnících}.}
 \label{fig:dva-straznici}
\end{figure}

Zbytek sekce je věnován výpočtům limit vybraných posloupností s účelem objasnit
použití právě sepsaných tvrzení. Mnoho z nich je ponecháno čtenářům jako
cvičení.

\begin{problem}{}{piplacka-limity}
 Spočtěte
 \[
  \lim_{n \to \infty} \frac{2n^2+n-3}{n^3-1}.
 \]
\end{problem}
\begin{probsol}
 Použijeme \hyperref[thm:aritmetika-limit]{větu o aritmetice limit}. Ta
 vyžaduje, aby výsledná strana rovnosti byla definována. Je tudíž možné (a
 žádoucí) limitu spočítat -- často opakovaným použitím této věty -- a teprve na
 konci výpočtu argumentovat, že její nasazení bylo oprávněné.

 Dobrým prvním krokem při řešení limit zadaných zlomky je najít v čitateli i
 jmenovateli \uv{nejrychleji rostoucí} člen. Spojením \uv{nejrychleji rostoucí}
 zde míníme takový člen, velikost ostatních členů je pro velmi velká $n$ vůči
 jehož zanedbatelná. V čitateli zlomku
 \[
  \frac{2n^2 + n - 3}{n^3 - 1}
 \]
 je nejrychleji rostoucí člen právě $2n^2$. Například, pro $n = 10^9$ je $2n^2 =
 2 \cdot 10^{18}$ zatímco $n = 10^9$ zabírá méně než jednu miliardtinu $2n^2$.
 Ve jmenovateli je naopak jediným rostoucím členem $n^3$. Nejrychleji rostoucí
 členy (pro pohodlí bez koeficientů) z obou částí zlomku vytkneme. Dostaneme
 \[
  \frac{2n^2 + n - 3}{n^3 - 1} = \frac{n^2 \left( 2 + \frac{1}{n} -
  \frac{3}{n^2} \right)}{n^3 \left(1 - \frac{1}{n^3}\right)} = \frac{1}{n} \cdot
  \frac{2 + \frac{1}{n} - \frac{3}{n^2}}{1 - \frac{1}{n^3}}.
 \]
 Část (b) \hyperref[thm:aritmetika-limit]{věty o aritmetice limit} nyní dává
 \begin{equation*}
  \label{eq:soucin-limit}
  \tag{$\triangle$}
  \lim_{n \to \infty} \frac{2n^2 + n - 3}{n^3 - 1} = \lim_{n \to \infty}
  \frac{1}{n} \cdot \lim_{n \to \infty} \frac{2 + \frac{1}{n} - \frac{3}{n^2}}{1
  - \frac{1}{n^3}},
 \end{equation*}
 \textbf{za předpokladu, že součin na pravé straně je definován!}

 Již víme, že platí $\lim_{n \to \infty} \frac{1}{n} = 0$. \textbf{Pozor!} Bylo
 by lákavé prohlásit, že výsledná limita je rovna $0$, bo součin čehokoliv s $0$
 je též $0$. To je pravda pro všechna čísla až na $\pm \infty$. Musíme se
 ujistit, že druhá limita v součinu na pravé straně~\eqref{eq:soucin-limit}
 existuje a není nekonečná.

 S použitím \hyperref[thm:aritmetika-limit]{věty o aritmetice limit} (c)
 počítáme
 \[
  \lim_{n \to \infty} \frac{2 + \frac{1}{n} - \frac{3}{n^2}}{1 - \frac{1}{n^3}}
  = \frac{\lim_{n \to \infty} 2 + \frac{1}{n} - \frac{3}{n^2}}{\lim_{n \to
  \infty} 1 - \frac{1}{n^3}}.
 \]
 Limity v čitateli a jmenovateli zlomku výše spočteme zvlášť. Z
 \hyperref[thm:aritmetika-limit]{věty o aritmetice limit}, části (a), plyne, že
 \[
  \lim_{n \to \infty} 2 + \frac{1}{n} - \frac{3}{n^2} = \lim_{n \to \infty} 2 +
  \lim_{n \to \infty} \frac{1}{n} - \lim_{n \to \infty} \frac{3}{n^2} = 2 + 0 -
  0 = 2.
 \]
 Podle stejného tvrzení též
 \[
  \lim_{n \to \infty} 1 - \frac{1}{n^3} = \lim_{n \to \infty} 1 - \lim_{n \to
  \infty} \frac{1}{n^3} = 1 - 0 = 1.
 \]
 To znamená, že
 \[
  \lim_{n \to \infty} \frac{2 + \frac{1}{n} - \frac{3}{n^2}}{1 - \frac{1}{n^3}}
  = \frac{2}{1} = 2.
 \]
 Odtud pak
 \[
  \lim_{n \to \infty} \frac{2n^2 + n - 3}{n^3 - 1} = \lim_{n \to \infty}
  \frac{1}{n} \cdot \lim_{n \to \infty} \frac{2 + \frac{1}{n} - \frac{3}{n^2}}{1
  - \frac{1}{n^3}} = 0 \cdot 2 = 0.
 \]
 Protože všechny výrazy na konci výpočtů jsou reálná čísla (a tedy speciálně jsou
 dobře definované), bylo lze použít \hyperref[thm:aritmetika-limit]{větu o
 aritmetice limit}.
\end{probsol}

\begin{remark}{}{aritmetika-limit}
 Právě vyřešená \myref{úloha}{prob:piplacka-limity} ukazuje, jak dlouhé se
 limitní úlohy stávají při pedantickém ověřování všech předpokladů. A to jsme
 navíc použili \emph{jen jediné tvrzení} k jejímu výpočtu. Takový postup není, z
 pochopitelného důvodu, obvyklý. Opakovaná použití
 \hyperref[thm:aritmetika-limit]{věty o aritmetice limit} se často schovají pod
 jedno prohlášení a výpočet limity je pak mnohem stručnější. Názorně předvedeme.

 Snadno úpravou zjistíme, že
 \[
  \frac{2n^2 + n - 3}{n^3 - 1} = \frac{1}{n} \cdot \frac{2 + \frac{1}{n} -
  \frac{3}{n^2}}{1 - \frac{1}{n^3}}.
 \]
 Potom z \hyperref[thm:aritmetika-limit]{věty o aritmetice limit} platí
 \[
  \lim_{n \to \infty} \frac{2n^2 + n - 3}{n^3 - 1} = \lim_{n \to \infty}
  \frac{1}{n} \cdot \lim_{n \to \infty} \frac{2 + \frac{1}{n} - \frac{3}{n^2}}{1
  - \frac{1}{n^3}} = 0 \cdot \frac{2 + 0 + 0}{1 + 0} = 0.
 \]
 Protože výsledný výraz je definovaný, byla \hyperref[thm:aritmetika-limit]{věta
 o aritmetice limit} použita korektně.

 My rovněž hodláme v dalším textu bez varování řešit podobné limitní příklady
 tímto \uv{zkráce\-ným} způsobem.
\end{remark}

\begin{warning}{}{aritmetika-limit}
 \hyperref[thm:aritmetika-limit]{Větou o aritmetice limit} \textbf{nelze}
 dokazovat, že limita posloupnosti neexistuje, neboť předpokladem každé její
 části je \emph{definovanost} výsledného výrazu. Zanedbání toho předpokladu může
 snadno vést ke lži. Uvažme následující triviální příklad.

 Prohlásili-li bychom, že z \hyperref[thm:aritmetika-limit]{věty o aritmetice
 limit} platí výpočet
 \[
  \lim_{n \to \infty} \frac{n}{n} = \frac{\lim_{n \to \infty} n}{\lim_{n \to
  \infty} n} = \frac{\infty}{\infty},
 \]
 nabyli bychom práva tvrdit, že $\lim_{n \to \infty} n / n$ neexistuje, přestože
 zřejmě platí $\lim_{n \to \infty} n / n = \lim_{n \to \infty} 1 = 1$.
 \hyperref[thm:aritmetika-limit]{Věta o aritmetice limit} je tudíž zcela prázdné
 tvrzení v případě nedefinovanosti výsledného výrazu.
\end{warning}

\begin{problem}{}{limita-binomicka-veta}
 Spočtěte limitu
 \[
  \lim_{n \to \infty} \frac{(n+4)^{100} - (n+3)^{100}}{(n+2)^{100} - n^{100}}.
 \]
\end{problem}
\begin{probsol}
 Z binomické věty platí
 \[
  (n+m)^{100} = \sum_{k=0}^{100} \binom{100}{k} n^{100-k} m^{k}.
 \]
 Je tudíž snadno vidět, že členy $n^{100}$ v se v čitateli i jmenovateli odečtou
 a \uv{nejrychleji rostoucím} členem v čitateli i jmenovateli stane sebe
 $cn^{99}$ pro vhodné $c \in \N$. Konkrétně, v čitateli koeficient $n^{99}$
 vychází
 \[
  \binom{100}{1} \cdot 4^{1} - \binom{100}{1} \cdot 3^{1} = 400 - 300 = 100
 \]
 a ve jmenovateli zkrátka
 \[
  \binom{100}{1} \cdot 2^{1} = 200.
 \]
 Užitím výpočtu v předešedším odstavci získáme úpravou původního výrazu
 \[
  \frac{(n+4)^{100} - (n+3)^{100}}{(n+2)^{100} - n^{100}} = \frac{100n^{99} +
  \sum_{k=2}^{100} \binom{100}{k}(4^{k} - 3^{k})n^{100 - k}}{200n^{99} +
 \sum_{k=2}^{100} \binom{100}{k} 2^{k}n^{100-k}}.
 \]
 Vytčení $n^{99}$ z obou částí zlomku dá
 \[
  \frac{100n^{99} + \sum_{k=2}^{100} \binom{100}{k}(4^{k} - 3^{k})n^{100 -
  k}}{200n^{99} + \sum_{k=2}^{100} \binom{100}{k} 2^{k}n^{100-k}}
  = \frac{n^{99}\left( 100 + \sum_{k=2}^{100} \binom{100}{k}(4^{k}-3^{k})n^{1-k}
  \right)}{n^{99}\left( 200 + \sum_{k=2}^{100} \binom{100}{k} 2^{k}n^{1-k}
  \right)}.
 \]
 Položme
 \begin{align*}
  f(n) & \coloneqq 100 + \sum_{k=2}^{100} \binom{100}{k} (4^{k} -
  3^{k})n^{1-k},\\
  g(n) & \coloneqq 200 + \sum_{k=2}^{100} \binom{100}{k} 2^{k}n^{1-k}.
 \end{align*}
 Nahlédneme, že $\lim_{n \to \infty} f(n) = 100$. Vskutku, z
 \hyperref[thm:aritmetika-limit]{věty o aritmetice limit}, částí (a) a (b),
 platí
 \begin{align*}
  \lim_{n \to \infty} f(n) &= \lim_{n \to \infty} 100 + \sum_{k=2}^{100}
 \binom{100}{k}(4^{k} - 3^{k})n^{1-k} = 100 + \sum_{k=2}^{100} (4^{k}-3^{k})
 \cdot \lim_{n \to \infty} n^{1-k}\\
                           &= 100 + \sum_{k=2}^{100} (4^{k} - 3^{k}) \cdot 0 =
                           100,
 \end{align*}
 kde $\lim_{n \to \infty} n^{1-k} = 0$ pro $k \geq 2$ zřejmě. Podobně bychom
 byli spočetli i $\lim_{n \to \infty} g(n) = 200$.
 \hyperref[thm:aritmetika-limit]{Větou o aritmetice limit}, částí (b) a (c), pak
 spočteme
 \[
  \lim_{n \to \infty} \frac{(n+4)^{100} - (n+3)^{100}}{(n+2)^{100} - n^{100}} =
  \lim_{n \to \infty} \frac{n^{99}}{n^{99}} \cdot \frac{\lim_{n \to \infty}
  f(n)}{\lim_{n \to \infty} g(n)} = 1 \cdot \frac{100}{200} = \frac{1}{2}.
 \]
 Protože výsledkem je reálné číslo, byla \hyperref[thm:aritmetika-limit]{věta o
 aritmetice limit} použita legálně.
\end{probsol}

\begin{problem}{}{limita-odmocniny}
 Spočtěte
 \[
  \lim_{n \to \infty} \sqrt{n^2 + n} - \sqrt{n^2 + 1}.
 \]
\end{problem}
\begin{probsol}
 Zkusili-li bychom spočítat limitu přímo z \hyperref[thm:aritmetika-limit]{věty
 o aritmetice limit} a \myref{lemmatu}{lem:limita-odmocniny}, dostali bychom
 \[
  \lim_{n \to \infty} \sqrt{n^2 + n} - \lim_{n \to \infty} \sqrt{n^2 + 1} =
  \infty - \infty,
 \]
 anžto výraz není definován. Je pročež třeba jej upravit. Využijeme vzorce
 \[
  a^2 - b^2 = (a+b)(a-b).
 \]
 Pro $a \coloneqq \sqrt{n^2+n}$ a $b \coloneqq \sqrt{n^2+1}$ dostaneme
 \[
  (n^2 + n) - (n^2 + 1) = (\sqrt{n^2 + n} + \sqrt{n^2 + 1})(\sqrt{n^2 + n} -
  \sqrt{n^2 + 1}).
 \]
 Zadaný výraz upravíme posléze na
 \[
  \sqrt{n^2+n} - \sqrt{n^2+1} = \frac{(\sqrt{n^2+n} + \sqrt{n^2+1})(\sqrt{n^2+n}
  - \sqrt{n^2+1})}{\sqrt{n^2+n} + \sqrt{n^2+1}} = \frac{n - 1}{\sqrt{n^2 + n} +
 \sqrt{n^2 + 1}}.
 \]
 Vidíme, že nejrychleji rostoucí člen v čitateli je $n$ a ve jmenovateli
 $\sqrt{n^2} = n$. Jejich vytčením získáme
 \[
  \frac{n-1}{\sqrt{n^2 + n}+\sqrt{n^2+1}} = \frac{n}{n} \cdot \frac{1 -
  \frac{1}{n}}{\sqrt{1 + \frac{1}{n}} + \sqrt{1 + \frac{1}{n^2}}} = \frac{1 -
  \frac{1}{n}}{\sqrt{1 + \frac{1}{n}} + \sqrt{1 + \frac{1}{n^2}}}.
 \]
 Byvše zaštítěni \myref{lemmatem}{lem:limita-odmocniny}, nabyli jsme práva
 tvrdit, že
 \[
  \lim_{n \to \infty} \sqrt{1 + \frac{1}{n}} = \sqrt{\lim_{n \to \infty} 1 +
  \frac{1}{n}}.
 \]
 a podobně pro $\lim_{n \to \infty} \sqrt{1 + 1 / n^2}$. Nyní tedy z
 \hyperref[thm:aritmetika-limit]{věty o aritmetice limit}, částí (a) a (c),
 plyne, že
 \[
  \lim_{n \to \infty}
  \frac{1-\frac{1}{n}}{\sqrt{1+\frac{1}{n}}+\sqrt{1+\frac{1}{n^2}}} =
  \frac{1-0}{\sqrt{1+0} + \sqrt{1+0}} = \frac{1}{2}.
 \]
 Výsledkem je reálné číslo, \hyperref[thm:aritmetika-limit]{věta o aritmetice
 limit} byla užita legálně.
\end{probsol}

\begin{problem}{}{dva-straznici-uloha}
 Spočtěte
 \[
  \lim_{n \to \infty} \sqrt[n]{n^2 + 2^{n} + 3^{n}}.
 \]
\end{problem}
\begin{warning}{}{limita-n-te-odmocniny}
 \textbf{Pozor!} Obecně
 \[
  \lim_{n \to \infty} \sqrt[n]{a_n} \neq \sqrt[n]{\lim_{n \to \infty} a_n}.
 \]
 Taková rovnost by ani nedávala žádný smysl, protože ve výraze napravo je
 odmocnina \textbf{vně} limity, přestože závisí na $n$.

 \myref{Lemma}{lem:limita-odmocniny} předpokládá, že $k \in \N$ je
 \textbf{konstantní}, čili nezávisí na $n$. 
\end{warning}

\begin{probsol}[\myref{úlohy}{prob:dva-straznici-uloha}]
 Na první pohled není zřejmé, kterak výraz $\sqrt[n]{n^2 + 2^{n} + 3^{n}}$
 upravit, aby výpočet mohl pokročit, bo \hyperref[thm:aritmetika-limit]{věta o
 aritmetice limit} není v závěsu \myref{varování}{warn:limita-n-te-odmocniny}
 přímo použitelná.

 V případech, kdy člověk nevidí způsob, jak spočítat konkrétní zadanou limitu,
 jesti pleché uchýliti sebe k odhadům zezdola i seshora jinými posloupnosti se
 snadněji určitelnými limitami. Zvoleny-li ony posloupnosti, bychu měly stejnou
 limitu, závěr \myref{lemmatu}{lem:o-dvou-straznicich} dává limitu i
 posloupnosti zadané.

 K volbě vhodných posloupností je však dlužno prve \uv{tipnout} limitu zadaného
 výrazu. Jelikož $3^{n}$ je jistě nejrychleji rostoucí člen dané posloupnosti, a
 $\sqrt[n]{3^{n}} = 3$, zdá se rozumným pokusit se nejprve odhadnout zadaný
 výraz zezdola i seshora posloupnostmi, jejichž limita je $3$.

 Dolní odhad je triviální a v zásadě jsme ho již uvedli. Totiž, jistě platí
 \[
  3^{n} \leq n^2 + 2^{n} + 3^{n},
 \]
 a tedy i
 \[
  \sqrt[n]{3^{n}} \leq \sqrt[n]{n^2 + 2^{n} + 3^{n}}.
 \]
 Zřejmě $\lim_{n \to \infty} \sqrt[n]{3^{n}} = \lim_{n \to \infty} 3 = 3$.

 Snad méně přímočarý jest horní odhad, jenž však plyne z uvědomění, že $3^{n}$
 je nejrychleji rostoucí člen dané posloupnosti. Speciálně máme odhady $n^2 \leq
 3^{n}$ i $2^{n} \leq 3^{n}$. Můžeme pročež pro všechna $n \in \N$ učinit další
 odhad:
 \[
  n^2 + 2^{n} + 3^{n} \leq 3^{n} + 3^{n} + 3^{n} = 3 \cdot 3^{n}.
 \]
 Z \hyperref[thm:aritmetika-limit]{věty o aritmetice limit} potom platí
 \[
  \lim_{n \to \infty} \sqrt[n]{3 \cdot 3^{n}} = \lim_{n \to \infty} \sqrt[n]{3}
  \cdot \sqrt[n]{3^{n}} = \lim_{n \to \infty} \sqrt[n]{3} \cdot \lim_{n \to
  \infty} 3 = 1 \cdot 3 = 3.
 \]
 Fakt, že $\lim_{n \to \infty} \sqrt[n]{3} = 1$ je snadno dokazatelný a onen
 důkaz ponecháme čtenáři jako cvičení.

 Jelikož pro všechna $n \in \N$ platí
 \[
  \sqrt[n]{3^{n}} \leq \sqrt[n]{n^2 + 2^{n} + 3^{n}} \leq \sqrt[n]{3 \cdot
  3^{n}}
 \]
 a již jsme spočetli, že $\lim_{n \to \infty} \sqrt[n]{3^{n}} = \lim_{n \to
 \infty} \sqrt[n]{3 \cdot 3^{n}} = 3$, můžeme prohlásit s použitím
 \myref{lemmatu}{lem:o-dvou-straznicich}, že
 \[
  \lim_{n \to \infty} \sqrt[n]{n^2 + 2^{n} + 3^{n}} = 3.
 \]
\end{probsol}

\begin{exercise}{}{limita-n-te-odmocniny}
 Dokažte, že pro všechna $a \in \R, a > 0$ platí
 \[
  \lim_{n \to \infty} \sqrt[n]{a} = 1.
 \]
\end{exercise}

Úplný závěr sekce věnujeme výpočtu jistých \emph{speciálních} limit, které je
výhodné znát, neboť v tradičních limitních úlohách vyvstávají často. V principu
jde o limity zadané zlomky, u kterých není na první pohled zřejmé, zda roste
rychleji čitatel, či jmenovatel.

Následující tvrzení spolu s \hyperref[thm:aritmetika-limit]{větou o aritmetice
limit} říká v podstatě, že \uv{Každá polynomiální funkce roste pomaleji než
každá funkce exponenciální.}

\begin{lemma}{}{polynomial-over-exponential}
 Platí
 \[
  \lim_{n \to \infty} \frac{n^{k}}{a^{n}} = 0,
 \]
 kdykoli $a > 1$ a $k \in \N$.
\end{lemma}
\begin{lemproof}
 Dokážeme tvrzení nejprve pro $k = 1$.

 Položme $b \coloneqq a - 1$. Potom z binomické věty
 \[
  a^{n} = (1 + b)^{n} = \sum_{i=0}^n \binom{n}{i}b^{i}.
 \]
 Speciálně tedy platí
 \[
  (1 + b)^{n} \geq \binom{n}{2}b^2 = \frac{n(n-1)}{2}b^2,
 \]
 neboť součet výše obsahuje člen napravo a ještě mnoho dalších členů, z nichž
 všechny jsou kladné. Potom ale
 \[
  \frac{n}{a^{n}} = \frac{n}{(1+b)^{n}} \leq \frac{2n}{b^2n(n-1)} =
  \frac{2}{b^2(n-1)}.
 \]
 Snadno vidíme, že platí $n / a^{n} \geq 0$ pro každé $n \in \N$. Máme tudíž
 oboustranný odhad
 \[
  0 \leq \frac{n}{a^{n}} \leq \frac{2}{b^2(n-1)}.
 \]
 Vzhledem k tomu, že
 \[
  \lim_{n \to \infty} \frac{2}{b^2(n-1)} = 0,
 \]
 jest závěrem \myref{lemmatu}{lem:o-dvou-straznicich}, že $\lim_{n \to \infty} n
 / a^{n} = 0$.

 V obecném případě $k \in \N$ stačí položit $b \coloneqq \sqrt[k]{a}$. Potom $b
 > 1$ (protože $a > 1$) a
 \[
  \lim_{n \to \infty} \frac{n^{k}}{a^{n}} = \lim_{n \to \infty} \left(
  \frac{n}{(\sqrt[k]{a})^{n}} \right)^{k} = \lim_{n \to \infty} \left(
  \frac{n}{b^{n}} \right)^{k} = \left( \lim_{n \to \infty} \frac{n}{b^{n}}
  \right)^{k},
 \]
 kde poslední rovnost platí z \hyperref[thm:aritmetika-limit]{věty o aritmetice
 limit}. Podle již dokázané části tvrzení je pravdou, žeť
 \[
  \left( \lim_{n \to \infty} \frac{n}{b^{n}} \right)^{k} = 0^{k} = 0,
 \]
 což bylo dokázati.
\end{lemproof}
\begin{lemma}{}{faktorial-podle-exponenciala}
 Platí
 \[
  \lim_{n \to \infty} \frac{n!}{n^{n}} = 0.
 \]
\end{lemma}
\begin{lemproof}
 Rozložíme výraz následovně:
 \[
  \frac{n!}{n^{n}} = \frac{1}{n} \cdot \prod_{k=2}^n \frac{k}{n}.
 \]
 Pozorujeme, že pro $2 \leq k \leq n$ platí $k / n \leq 1$. Čili,
 \[
  \frac{n!}{n^{n}} = \frac{1}{n} \cdot \prod_{k=2}^n \frac{k}{n} \leq
  \frac{1}{n} \cdot \prod_{k=2}^n 1 = \frac{1}{n}.
 \]
 Dostáváme pro $n \in \N$ odhady
 \[
  0 \leq \frac{n!}{n^{n}} \leq \frac{1}{n}.
 \]
 Jelikož $\lim_{n \to \infty} 0 = \lim_{n \to \infty} 1 / n = 0$, platí z
 \myref{lemmatu}{lem:o-dvou-straznicich} závěr
 \[
  \lim_{n \to \infty} \frac{n!}{n^{n}} = 0,
 \]
 jak jsme chtěli.
\end{lemproof}
\begin{lemma}{}{exponenciala-podle-faktorialu}
 Pro $ a > 1$ platí
 \[
  \lim_{n \to \infty} \frac{a^{n}}{n!} = 0,
 \]
 čili \uv{faktoriál roste rychleji než exponenciála}.
\end{lemma}
\begin{lemproof}
 Nalezněme $m \in \N$ takové, že $m > a$. Rozložíme
 \[
  \frac{a^{n}}{n!} = \frac{a^{m}}{m!} \cdot \prod_{k=m+1}^n \frac{a}{k}.
 \]
 Všimněme sobě, že pro $k > m$ je $a / k < 1$, ježto $m$ bylo zvoleno ostře
 větší než $a$. Speciálně tedy platí
 \[
  \prod_{k=m+1}^n \frac{a}{k} = \frac{a}{n} \cdot \prod_{k=m+1}^{n-1}
  \frac{a}{k} \leq \frac{a}{n},
 \]
 neboť
 \[
  \prod_{k=m+1}^{n-1} \frac{a}{k} \leq \prod_{k=m+1}^{n-1} 1 = 1.
 \]
 Položme $c_m \coloneqq a^{m} / m!$. Číslo $c_m$ je konstantní (vzhledem k $n$),
 neboť $m$ i $a$ jsou. Můžeme odhadnout
 \[
  0 \leq \frac{a^{n}}{n!} \leq \frac{a^{m}}{m!} \cdot \frac{a}{n} = c_m \cdot
  \frac{a}{n}.
 \]
 Z \hyperref[thm:aritmetika-limit]{věty o aritmetice limit} platí
 \[
  \lim_{n \to \infty} c_m \cdot \frac{a}{n} = ac_m \cdot \lim_{n \to \infty}
  \frac{1}{n} = 0.
 \]
 Podle \myref{lemmatu}{lem:o-dvou-straznicich} tudíž máme i
 \[
  \lim_{n \to \infty} \frac{a^{n}}{n!} = 0,
 \]
 což zakončuje důkaz.
\end{lemproof}

Několik limitních cvičení na závěr.

\begin{exercise}{}{odmocnina-z-faktorialu}
 Dokažte, že
 \[
  \lim_{n \to \infty} \sqrt[n]{n!} = \infty.
 \]
 \textbf{Hint}: Rozložte součin $n!$ na dvě poloviny a tu větší zespodu
 odhadněte vhodnou posloupností jdoucí k $\infty$.
\end{exercise}

\begin{exercise}{}{}
 Spočtěte
 \[
  \lim_{n \to \infty} \sqrt{4n^2 - n} - 2n.
 \]
\end{exercise}

\begin{exercise}{}{}
 Spočtěte
 \[
  \lim_{n \to \infty} (-1)^{n}\sqrt{n}(\sqrt{n+1}-\sqrt{n}).
 \]
\end{exercise}

\begin{exercise}{těžké}{}
 Spočtěte limitu posloupnosti $a:\N \to \R$ zadané rekurentním vztahem
 \begin{align*}
  a_0 &\coloneqq 10,\\
  a_{n+1} & \coloneqq 6 - \frac{5}{a_n} \text{ pro } n \in \N.
 \end{align*}
\end{exercise}
