\subsection{Relace}
\label{ssec:relace}

Relace, jak jejich název snad napovídá, jsou množiny, které kódují \emph{vztahy}
mezi prvky dvou různých množin. Jejich matematické pojetí je přímočaré a
elegantní, ano trochu neintuitivní. Totiž, relace (či \uv{vztah}) je zkrátka jen
výpis prvků, které v něm jsou, nikoli žádný nezávislý popis jeho vlastností.
Musíme-li načrtnout sociální rovnoběžku, můžeme si představit, že význam vztahu
přátelství je přesně určen seznamem všech párů přátel. I když ona rovnoběžka
vzbudila v mnohých čtenářích jistě značnou nedůvěru, taková definice vztahu v
matematice je jednoduchá a téměř nesrovnatelně užitečná.

\begin{definition}{Relace}{relace}
 Ať $A,B$ jsou množiny. \emph{Relací} $R$ mezi $A$ a $B$ míníme kteroukoli
 podmnožinu $R \subseteq A \times B$.
\end{definition}

Milí čtenáři se již jistě s několika relacemi setkali. Například samotný vztah
rovnosti ($=$) je relací. Podobně, vztahy \uv{býti menší nebo rovno} ($ \leq $)
či \uv{býti ostře větší} ($>$) jsou relacemi ve všech číselných oborech. Jak
tyto příklady naznačují, historicky se symboly relací píšou obvykle \emph{mezi}
prvky v oné relaci jsoucí. Tohoto úzu se držíme i my a pro relaci $R \subseteq A
\times B$ píšeme $aRb$, kdykoli $(a,b) \in R$. Jelikož je však zápis
$a\cancel{R}b$ poněkud neuhlazený a nadevše obtížně rozpoznatelný od svého
pozitivního protipólu, používáme množinové značení $(a,b) \notin R$, kdykoli
prvek $a$ není v relaci $R$ s prvkem $b$.

Máme-li k dispozici výčty prvků množin $A$ a $B$, je pro představu dobré kreslit
relace $R \subseteq A \times B$ do tzv. mříží. Ty tvoříme tak, že nakreslíme
doslovnou mříž teček s $\# A$ sloupci, resp. $\# B$ řádky, pro každý prvek
množiny $A$, resp. množiny $B$, a tečky na pozicích odpovídající párům $(a,b)
\in A \times B$, které rovněž leží v $R$, například kroužkujeme. Zvolíme-li
třeba
\[
 \clr{A} = \clr{\{1,2,3,4\}}, \clb{B} = \clb{\{a,b,c\}}
\]
a mezi nimi relaci
\[
 \clg{R} = \clg{\{(1,a),(1,b),(2,b),(3,a),(3,b),(3,c),(4,a)\}},
\]
bude jejím vyobrazením mříž na \myref{obrázku}{fig:mriz-relace}.

\begin{figure}[ht]
 \centering
 \begin{tikzpicture}
  \foreach \x in {1,2,3,4} {
   \foreach \y in {-1,-2,-3} {
    \node[vertex] at (\x,\y) {};
   }
  }
  \node (one) at (1,-0.5) {$\clr{1}$};
  \node (two) at (2,-0.5) {$\clr{2}$};
  \node (three) at (3,-0.5) {$\clr{3}$};
  \node (four) at (4,-0.5) {$\clr{4}$};
  \node (a) at (0.5,-1) {$\clb{a}$};
  \node (b) at (0.5,-2) {$\clb{b}$};
  \node (c) at (0.5,-3) {$\clb{c}$};
  
  \draw[ForestGreen,thick] (1,-1) circle (2mm);
  \draw[ForestGreen,thick] (1,-2) circle (2mm);
  \draw[ForestGreen,thick] (2,-2) circle (2mm);
  \draw[ForestGreen,thick] (3,-1) circle (2mm);
  \draw[ForestGreen,thick] (3,-2) circle (2mm);
  \draw[ForestGreen,thick] (3,-3) circle (2mm);
  \draw[ForestGreen,thick] (4,-1) circle (2mm);
 \end{tikzpicture}

 \caption{Mříž relace $\clg{R} \subseteq \clr{A} \times \clb{B}$.}
 \label{fig:mriz-relace}
\end{figure}

Relace zvláště zásadního významu jsou ty mezi dvěma totožnými množinami. Je-li
$R$ relace mezi $A$ a $A$ říkáme, výrazně lidštěji, že $R$ je relace \emph{na}
$A$. U relací na množině stavíme na piedestal ty s~jistými speciálními
vlastnostmi. Konkrétně, řekneme, že relace $R \subseteq A \times A$ je
\begin{itemize}
 \item \emph{reflexivní}, když je každý prvek v relaci $R$ sám se sebou, tj. $
  \forall x \in A:xRx$;
 \item \emph{antireflexivní}, když žádný prvek není v relaci $R$ sám se sebou,
  tj. $ \forall x:(x,x) \notin R$;
 \item \emph{symetrická}, když s každým párem prvků obsahuje i ten obrácený, tj.
  $ \forall x,y \in A:xRy \Rightarrow yRx$;
 \item \emph{antisymetrická}, když z každého páru dvojic $(x,y)$ a $(y,x)$ je v
  $R$ vždy jen jedna, tj. $ \forall x,y \in A: xRy \Rightarrow (y,x) \notin R$
  (všimněme si, že antisymetrická relace je automaticky antireflexivní);
 \item \emph{\textbf{slabě} antisymetrická}, když z každého páru dvojic $(x,y)$
  a $(y,x)$ je v $R$ vždy jen jedna za předpokladu, že $x$ a $y$ jsou od sebe
  různé, tj. $ \forall x,y \in A: (xRy \wedge yRx) \Rightarrow (x = y)$;
 \item \emph{transitivní}, když se přirozeně \uv{přenáší} přes prostřední prvek,
  tj. $ \forall x,y,z \in A:(xRy \wedge yRz) \Rightarrow xRz$.
\end{itemize}

\begin{figure}[ht]
 \centering
 \begin{subfigure}{.3\textwidth}
  \centering
  \begin{tikzpicture}[scale=0.75]
   \foreach \x in {1,2,3,4} {
    \foreach \y in {-1,-2,-3,-4} {
     \node[vertex] at (\x,\y) {};
    }
   }
   \foreach \x in {1,2,3,4} {
    \node at (\x,-0.25) {$\clr{\x}$};
   }
   \foreach \y in {1,2,3,4} {
    \node at (0.25,-\y) {$\clr{\y}$};
   }

   \foreach \x in {1,2,3,4} {
    \draw[ForestGreen,thick] (\x,-\x) circle (2.5mm);
   }
   \draw[ForestGreen,thick] (1,-3) circle (2.5mm);
   \draw[ForestGreen,thick] (2,-1) circle (2.5mm);
   \draw[ForestGreen,thick] (4,-2) circle (2.5mm);
   \draw[ForestGreen,thick] (3,-4) circle (2.5mm);
  \end{tikzpicture}
  \caption{Reflexivní a slabě antisymetrická, nikoli transitivní.}
  \label{subfig:relace-na-mnozine-a}
 \end{subfigure}
 \hfill
 \begin{subfigure}{.3\textwidth}
  \centering
  \begin{tikzpicture}[scale=0.75]
   \foreach \x in {1,2,3,4} {
    \foreach \y in {-1,-2,-3,-4} {
     \node[vertex] at (\x,\y) {};
    }
   }
   \foreach \x in {1,2,3,4} {
    \node at (\x,-0.25) {$\clr{\x}$};
   }
   \foreach \y in {1,2,3,4} {
    \node at (0.25,-\y) {$\clr{\y}$};
   }

   \draw[ForestGreen,thick] (1,-2) circle (2.5mm);
   \draw[ForestGreen,thick] (2,-1) circle (2.5mm);
   \draw[ForestGreen,thick] (1,-4) circle (2.5mm);
   \draw[ForestGreen,thick] (4,-1) circle (2.5mm);
   \draw[ForestGreen,thick] (2,-4) circle (2.5mm);
   \draw[ForestGreen,thick] (4,-2) circle (2.5mm);
  \end{tikzpicture}
  \caption{Antireflexivní, symetrická a transitivní.}
  \label{subfig:relace-na-mnozine-b}
 \end{subfigure}
 \hfill
 \begin{subfigure}{.3\textwidth}
  \centering
  \begin{tikzpicture}[scale=0.75]
   \foreach \x in {1,2,3,4} {
    \foreach \y in {-1,-2,-3,-4} {
     \node[vertex] at (\x,\y) {};
    }
   }
   \foreach \x in {1,2,3,4} {
    \node at (\x,-0.25) {$\clr{\x}$};
   }
   \foreach \y in {1,2,3,4} {
    \node at (0.25,-\y) {$\clr{\y}$};
   }

   \draw[ForestGreen,thick] (1,-2) circle (2.5mm);
   \draw[ForestGreen,thick] (1,-3) circle (2.5mm);
   \draw[ForestGreen,thick] (1,-4) circle (2.5mm);
   \draw[ForestGreen,thick] (2,-3) circle (2.5mm);
   \draw[ForestGreen,thick] (2,-4) circle (2.5mm);
   \draw[ForestGreen,thick] (3,-4) circle (2.5mm);
  \end{tikzpicture}
  \caption{Antisymetrická a transitivní (též \emph{ostré uspořádání}, zde $<$).}
  \label{subfig:relace-na-mnozine-c}
 \end{subfigure}
 \caption{Příklady relací $\clg{R}$ na množině $\clr{A} = \clr{\{1,2,3,4\}}$.}
 \label{fig:relace-na-mnozine}
\end{figure}

Naprosto klíčovými typy relací pro rozvoj další teorie jsou \emph{ekvivalence},
\emph{uspořádání} a \emph{zobrazení}. Každému typu je věnována jedna z
následujících sekcí.
