\subsection{Relace uspořádání}
\label{ssec:relace-usporadani}

V úvodu do \hyperref[sec:zakladni-pojmy-z-teorie-mnozin]{této sekce} jsme
zdůraznili fakt, že množiny \textbf{nejsou} v žádném smyslu uspořádané, tedy
nelze o jejich prvcích tvrdit, který jde první a poslední. Ovšem, čtenáři jsou
si jistě vědomi, že kupříkladu přirozená čísla uspořádána \emph{jsou} -- číslo
$1$ je menší než číslo $5$, $8$ větší než $3$. Tento výrok je však mírně
nepřesný. Samotná množina přirozených čísel uspořádaná \emph{není}, lze na ní
však definovat jistý všudypřítomný typ relace (v tomto případě $ \leq $) zvaný
\emph{uspořádání}. Definujeme si, které vlastnosti musí relace uspořádání na
množině mít.

\begin{definition}{Relace uspořádání}{relace-usporadani}
 Ať $A$ je množina a $R \subseteq A \times A$ je relace na $A$. Řekneme, že $R$
 je \emph{uspořádání} (někdy též nesoucí přívlastek \emph{neostré}), když je
 \begin{itemize}
  \item reflexivní (tj. $ \forall x \in A:xRx$),
  \item slabě antisymetrické (tj. $ \forall x,y \in A:(xRy \wedge yRx)
   \Rightarrow (x=y))$,
  \item transitivní (tj. $ \forall x,y,z \in A:(xRy \wedge yRz) \Rightarrow
   xRz)$.
 \end{itemize}
 Je-li naopak $R$ (silně) antisymetrické (tj. $ \forall x,y \in A:xRy
 \Rightarrow (y,x) \notin R$), a tudíž antireflexivní (tj. $ \forall x \in
 A:(x,x) \notin R$), řekneme, že je $R$ \emph{\textbf{ostré} uspořádání}.

 Je-li $A$ množina a $R$ (ostré) uspořádání na $A$, říkáme, že dvojice $(A,R)$ je
 \emph{(ostře) uspořádaná množina} (podle $R$).
\end{definition}

Zcela nejobyčejnější příklady (neostrých) uspořádání jsou relace $ \leq $ a $
\geq $ v číselných oborech. Vskut\-ku, každý prvek je menší/větší nebo roven sám
sobě, z každé dvojice prvek je buď jeden menší/větší než ten druhý, nebo jsou
stejné, a když je prvek $x$ menší/větší než prvek $y$ a ten zas menší/větší než
prvek $z$, pak je $x$ menší/větší než $z$. Speciálně, $(\N, \leq )$ a $(\N, \geq
)$ jsou uspořádané množiny.

Příkladem \emph{ostrých} uspořádání jsou relace $<$ a $>$, které se liší tím, že
žádný prvek není ostře menší/větší než on sám.

\begin{warning}{}{linearni-usporadani}
 Součástí definice uspořádání \textbf{není} podmínka, že \textbf{každé} dva
 prvky lze spolu porovnat. Pročež, v uspořádaných množinách obecně existují
 dvojice prvků, kde první není ani menší ani větší než ten druhý.

 Uspořádáním (ať už ostrým či neostrým) naopak \emph{splňujícím} onu podmínku
 říkáme \emph{lineární}.
\end{warning}

Za definicí uspořádání samozřejmě stojí myšlenka, že taková relace určuje
\emph{pořadí} prvků na množině. Z toho důvodu se obvykle uspořádání nekreslí
jako obecné relace do mříží, ale do tzv. Hasseho (po číselném teoretiku Helmutu
Hassem) diagramů, které získáme tím způsobem, že prvky nakreslíme jako puntíky a
každé dva puntíky prvků, které jsou v daném uspořádání porovnatelné, spojíme
úsečkou (nejsou-li již spojeny přes nějaký další puntík) a větší z nich
nakreslíme nad menší. Chováme na vědomí, že kreslit obrázky je více
uspokojující, než je popisovat, a tedy si závěrem této krátké sekce nakreslíme
(Hasseho diagramy) tři takřkouce \uv{učebnicové} příklady uspořádání.

\begin{example}{}{usporadani}
 \begin{enumerate}
  \item Uvažme jednoduchý příklad uspořádané množiny $(\N, \leq )$. Už jsme si
   rozmysleli, že $ \leq $ je na $\N$ skutečně uspořádáním. Je triviální
   nahlédnout, že je rovněž lineární, neboť z~každých dvou přirozených čísel lze
   vybrat to menší.
  
   Lineární uspořádání mají velmi nezajímavý Hasseho diagram -- řetěz prvků,
   který může končit nahoře nebo dole. V tomto případě je nejmenším prvkem $0$ a
   nejvyšší neexistuje, tedy řetěz pokračuje nekonečně dlouho směrem nahoru.
   \begin{figure}[H]
    \centering
    \begin{tikzpicture}
     \foreach \y in {0,1,2,3} {
      \node[vertex] (v\y) at (0,\y) {};
      \node[right=1mm of v\y] {$\y$};
     }
     \foreach \y in {0,1,2} {
      \draw[thick] (0,\y) -- (0,\y+1);
     }
     \draw[thick] (0,3) -- (0,3.5);
     
     \node[above=5mm of v3] {$\vdots$};
    \end{tikzpicture}

    \caption{Hasseho diagram uspořádané množiny $(\N, \leq )$.}
    \label{fig:hasseho-diagram-N}
   \end{figure}
  \item O něco komplikovanější příklad je uspořádání inkluzí $ \subseteq $. Pro
   libovolnou množinu $A$ můžeme uvážit uspořádanou množinu $(2^{A}, \subseteq
   )$. Je téměř samozřejmé, že $ \subseteq $ je skutečně (neostré) uspořádání,
   které však \textbf{není} lineární. Vizme příklad.
  
   Ať $A \coloneqq \{1,2,3\}$. Pak jsou její podmnožiny $\{1,2\}$ a $\{2,3\}$
   neporovnatelné pomocí $ \subseteq $ -- ani jedna není podmnožinou té druhé.
   Zajímavým geometrickým faktem je, že Hasseho diagramem $(2^{A}, \subseteq )$
   pro $n$-prvkovou množinu $A$ je síť vrcholů krychle dimenze $n$.
   \begin{figure}[H]
    \centering
    \begin{tikzpicture}[scale=1.5]
     \node[vertex] (v0) at (0,0) {};
     \node[vertex] (v1) at (-1,1) {};
     \node[vertex] (v2) at (0,1) {};
     \node[vertex] (v3) at (1,1) {};
     \node[vertex] (v12) at (-1,2) {};
     \node[vertex] (v13) at (0,2) {};
     \node[vertex] (v23) at (1,2) {};
     \node[vertex] (v123) at (0,3) {};

     \draw[thick] (v0) -- (v1);
     \draw[thick] (v0) -- (v2);
     \draw[thick] (v0) -- (v3);
     \draw[thick] (v1) -- (v12);
     \draw[thick] (v1) -- (v13);
     \draw[thick] (v2) -- (v12);
     \draw[thick] (v2) -- (v23);
     \draw[thick] (v3) -- (v13);
     \draw[thick] (v3) -- (v23);
     \draw[thick] (v12) -- (v123);
     \draw[thick] (v13) -- (v123);
     \draw[thick] (v23) -- (v123);

     \node[below=0mm of v0] {$\emptyset$};
     \node[above=0mm of v123] {$\{1,2,3\}$};
     \node[below left=-1mm and -1mm of v1] {$\{1\}$};
     \node[below left=-1mm and -1mm of v2] {$\{2\}$};
     \node[below right=-1mm and -1mm of v3] {$\{3\}$};
     \node[above left=-1mm and -1mm of v12] {$\{1,2\}$};
     \node[above left=-1mm and -1mm of v13] {$\{1,3\}$};
     \node[above right=-1mm and -1mm of v23] {$\{2,3\}$};
    \end{tikzpicture}

    \caption{Hasseho diagram uspořádané množiny $(2^{A}, \subseteq )$ pro $A =
    \{1,2,3\}$.}
    \label{fig:hasseho-diagram-power-set}
   \end{figure}
  \item Poněkud méně pravidelný příklad než oba předchozí je uspořádání na
   přirozených číslech \emph{dělitelností}. Dělitelnost jednoho přirozeného
   čísla jiným se definuje samozřejmě. Řekneme, že $n$ \emph{dělí} $m$, píšeme
   $n \mid m$, když existuje přirozené číslo $k$ splňující $m = kn$. Kupříkladu
   $3 \mid 6$, protože existuje přirozené číslo -- konkrétně $2$ -- takové, že
   $6 = 2 \cdot 3$.

   Nyní, povšimněme sobě, že $ \mid $ je uspořádání na $\N$. Vskutku, zřejmě
   každé číslo dělí samo sebe. Když $n$ dělí $m$ a $m$ dělí $n$, pak jde nutně o
   téže číslo. Konečně, když $n$ dělí $m$ a $m$ dělí $l$, pak v podstatě
   triviálně $n$ dělí $l$. Nejde však jistě o uspořádání lineární, neboť
   například číslo $3$ nedělí číslo $10$ ani obráceně.
  
   Uspořádání dělitelností na $\N$ má Hasseho diagram, který je nekonečný jak do
   šířky, tak do výšky. Totiž, žádné prvočíslo není dělitelné žádným číslem,
   které šlo před ním (pouze $1$), a tedy je musíme umístit do druhé řady (nad
   $1$) vedle ostatních prvočísel. Do výšky stoupá nekonečně pro to, že když
   spolu vynásobíme dvě čísla $k$-té úrovně, dostaneme číslo úrovně $(k+1)$-ní.
  
   Pro jednoduchost se spokojíme s Hasseho diagramem množiny přirozených čísel
   od $1$ do $15$ uspořádané relací dělitelnosti.
   \begin{figure}[H]
    \centering
    \begin{tikzpicture}[xscale=2,yscale=1.5]
     \node[vertex] (v1) at (0,0) {};
     \node[vertex] (v2) at (-2.5,1) {};
     \node[vertex] (v3) at (-1.5,1) {};
     \node[vertex] (v5) at (-0.5,1) {};
     \node[vertex] (v7) at (0.5,1) {};
     \node[vertex] (v11) at (1.5,1) {};
     \node[vertex] (v13) at (2.5,1) {};
     
     \node[vertex] (v4) at (-2.5,2) {};
     \node[vertex] (v8) at (-2.5,3) {};

     \node[vertex] (v9) at (-1.5,2) {};

     \node[vertex] (v6) at (-2,1.5) {};
     \node[vertex] (v10) at (-1,1.5) {};
     \node[vertex] (v15) at (0,2) {};
     \node[vertex] (v14) at (-1,2) {};

     \foreach \n in {2,3,5,7,11,13} {
      \draw[thick] (v1) -- (v\n);
     }

     \draw[thick] (v2) -- (v4);
     \draw[thick] (v4) -- (v8);
     \draw[thick] (v3) -- (v9);
     
     \draw[thick] (v2) -- (v6);
     \draw[thick] (v3) -- (v6);
     
     \draw[thick] (v2) -- (v10);
     \draw[thick] (v5) -- (v10);
     
     \draw[thick] (v3) -- (v15);
     \draw[thick] (v5) -- (v15);
     
     \draw[thick] (v2) -- (v14);
     \draw[thick] (v7) -- (v14);
     
     \node[below=0mm of v1] {$1$};
     \node[below left=-1mm and -1mm of v2] {$2$};
     \node[below left=-1mm and -1mm of v3] {$3$};
     \node[below left=-1mm and -1mm of v5] {$5$};
     \node[below right=-1mm and -1mm of v7] {$7$};
     \node[below right=-1mm and -1mm of v11] {$11$};
     \node[below right=-1mm and -1mm of v13] {$13$};
     
     \node[left=0mm of v4] {$4$};
     \node[left=0mm of v8] {$8$};
     
     \node[above=0mm of v9] {$9$};
     \node[above=0mm of v6] {$6$};
     \node[above=0mm of v14] {$14$};
     
     \node[above right=-1mm and -1mm of v10] {$10$};
     \node[above right=-1mm and -1mm of v15] {$15$};
     
    \end{tikzpicture}

    \caption{Hasseho diagram uspořádané množiny $(\{1,2,\ldots,15\}, \mid )$.}
    \label{fig:hasse-delitelnost}
   \end{figure}
 \end{enumerate}
\end{example}
