\subsection{Relace uspořádání}
\label{ssec:relace-usporadani}

V úvodu do \hyperref[sec:zakladni-pojmy-z-teorie-mnozin]{této sekce} jsme
zdůraznili fakt, že množiny \textbf{nejsou} v žádném smyslu uspořádané, tedy
nelze o jejich prvcích tvrdit, který jde první a poslední. Ovšem, čtenáři jsou
si jistě vědomi, že kupříkladu přirozená čísla uspořádána \emph{jsou} -- číslo
$1$ je menší než číslo $5$, $8$ větší než $3$. Tento výrok je však mírně
nepřesný. Samotná množina přirozených čísel uspořádaná \emph{není}, lze na ní
však definovat jistý všudypřítomný typ relace (v tomto případě $ \leq $) zvaný
\emph{uspořádání}. Definujeme si, které vlastnosti musí relace uspořádání na
množině mít.

\begin{definition}{Relace uspořádání}{relace-usporadani}
 Ať $A$ je množina a $R \subseteq A \times A$ je relace na $A$. Řekneme, že $R$
 je \emph{uspořádání} (někdy též nesoucí přívlastek \emph{neostré}), když je
 \begin{itemize}
  \item reflexivní (tj. $ \forall x \in A:xRx$),
  \item slabě antisymetrické (tj. $ \forall x,y \in A:(xRy \wedge yRx)
   \Rightarrow (x=y))$,
  \item transitivní (tj. $ \forall x,y,z \in A:(xRy \wedge yRz) \Rightarrow
   xRz)$.
 \end{itemize}
 Je-li naopak $R$ (silně) antisymetrické (tj. $ \forall x,y \in A:xRy
 \Rightarrow (y,x) \notin R$), a tudíž antireflexivní (tj. $ \forall x \in
 A:(x,x) \notin R$), řekneme, že je $R$ \emph{\textbf{ostré} uspořádání}.

 Je-li $A$ množina a $R$ (ostré) uspořádání na $A$, říkáme, že dvojice $(A,R)$ je
 \emph{(ostře) uspořádaná množina} (podle $R$).
\end{definition}

Zcela nejobyčejnější příklady (neostrých) uspořádání jsou relace $ \leq $ a $
\geq $ v číselných oborech. Vskut\-ku, každý prvek je menší/větší nebo roven sám
sobě, z každé dvojice prvek je buď jeden menší/větší než ten druhý, nebo jsou
stejné, a když je prvek $x$ menší/větší než prvek $y$ a ten zas menší/větší než
prvek $z$, pak je $x$ menší/větší než $z$.

Příkladem \emph{ostrých} uspořádání jsou relace $<$ a $>$, které se liší tím, že
žádný prvek není ostře menší/větší než on sám.


