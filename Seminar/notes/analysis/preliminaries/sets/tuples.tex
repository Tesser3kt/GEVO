\subsection{Kartézský součin a uspořádané n-tice}
\label{ssec:kartezsky-soucin-a-usporadane-n-tice}

Anobrž holý pojem množiny neobsahuje v žádném smyslu koncept \emph{pořadí}
prvku, je tento však pochopitelně užitečný, a proto si jej definujeme.

Zápisem $(x,y)$ myslíme tzv. \emph{uspořádanou dvojici} prvků $x$ a $y$. V této
dvojici je $x$ prvním prvkem a $y$ druhým. Je proto odlišná od množiny
$\{x,y\}$, protože $\{x,y\} = \{y,x\}$, ale $(x,y) \neq (y,x)$. V úvodu do této
kapitoly jsme tvrdili, že teorie množin tvoří svět matematiky. Pojem
\emph{dvojice}, který není shodný s pojmem \emph{množiny} tudíž musel bedlivé
čtenáře vylekat. Darmo se však lekati, uspořádané dvojice jsou rovněž množiny.
Než jej odhalíme, vybízíme čtenáře, aby našli způsob, kterak definovat
uspořádanou dvojici jako množinu.

Běžně užívaná definice (prve formulována Kazimierzem Kuratowskim) je
\[
 (x,y) \coloneqq \{\{x\},\{x,y\}\}.
\]
Ta říká, že uspořádaná dvojice $(x,y)$ je vlastně množina obsahující množinu s
jediným prvkem $x$ a množinu $\{x,y\}$. Ta je pak rozdílná od dvojice $(y,x)$,
což je množina $\{\{y\},\{x,y\}\}$. Tato definice je z~formálního hlediska
nutná, protože vytvářet matematické objekty nedefinovatelné v teorii množin rodí
chaos, ale intuitivně zcela nepoužitelná. Představme si dva lidi za sebou ve
frontě vnímat jako skupinu, v níž je skupina jenom s tím prvním a pak ještě
skupina obou. Tvrdíme, že je dobré udržet si pročež představu uspořádané dvojice
jakožto množiny, ve které mají navíc prvky svá pořadí.

Definice uspořádané dvojice se rekurzivně rozšíří na libovolný počet prvků.
Kupříkladu, uspořádanou trojici definujeme předpisem
\[
 (x,y,z) \coloneqq (x,(y,z)),
\]
to jest jako uspořádanou dvojici obsahující prvek $x$ a uspořádanou dvojici
$(y,z)$. Zápis pomocí množin se s rostoucím počtem prvků velmi rychle
komplikuje. Všimněte si, že už v drobném případě tří prvků dostaneme
\[
 (x,y,z) = (x,(y,z)) = (x,\{\{y\},\{y,z\}\}) = \{\{x\},\{x,
 \{\{y\},\{y,z\}\}\}\}.
\]
Máme-li $n$ prvků $x_1,\ldots,x_n$, pak jejich uspořádanou $n$-tici zapisujeme
$(x_1,\ldots,x_n)$.

Existence uspořádaných $n$-tic dává vzniknout jedné další množinové operaci --
kartézskému součinu. Kartézský součin dvou množin $A,B$, zapsaný $A \times B$,
definujeme jako množinu všech uspořádaných dvojic $(a,b)$, kde $a \in A$ a $b
\in B$. Název \emph{kartézský} (po Reném Descartesovi) ona nese pro zřejmou
souvislost se stejnojmenným systémem souřadnic. Totiž, souřadnice v rovině jsou
přesně dvojice $(x,y) \in \R \times \R$, kde $\R$ značí množinu reálných čísel.

Rovina skýtá navíc užitečný způsob přemýšlení o kartézském součinu jako o
\uv{dimenz\-ní} operaci. Má-li totiž $A$ dimenzi (v intuitivním slova smyslu)
$n$ a $B$ dimenzi $m$, pak $A \times B$ má dimenzi $n + m$. Například, je-li $A$
čtverec o délce strany $1$, pak $A^2 = A \times A$ je krychle o délce strany
$1$.

\begin{figure}[ht]
 \centering
 \begin{tikzpicture}[3d view,perspective,scale=1.5]
  \node (a11) at (0,0,0) {};
  \node (a12) at (0,-2,0) {};
  \node (a13) at (-2,-2,0) {};
  \node (a14) at (-2,0,0) {};
  
  \path [draw,BrickRed,thick,fill opacity=0.25,pattern=north east lines,pattern
   color=BrickRed] (a11.center) to (a12.center) to (a13.center) to (a14.center)
   to (a11.center);
  \node[circle,fill=white,inner sep=2pt] (a1label) at (-1,-1,0) {$\clr{A}$};

  \node (a21) at (0,0,0) {};
  \node (a22) at (0,-2,0) {};
  \node (a23) at (0,-2,2) {};
  \node (a24) at (0,0,2) {};
  
  \path [draw,RoyalBlue,thick,fill opacity=0.25,pattern=north east lines,pattern
   color=RoyalBlue] (a21.center) to (a22.center) to (a23.center) to (a24.center)
   to (a21.center);
  \node[circle,fill=white,inner sep=2pt] (a2label) at (0,-1,1) {$\clb{A}$};

  \node[vertex,BrickRed] (v1) at (-1,-0.5,0) {};
  
  \node[vertex,RoyalBlue] (v2) at (0,-0.5,1) {};
  \node[vertex,minimum size=9pt] (x) at (-1,-0.5,1) {};

  \draw[thick,dashed,BrickRed] (v1) -- (x);
  \draw[thick,dashed,RoyalBlue] (v2) -- (x);
  \node[left=1mm of x] {$(\clr{a_1},\clb{a_2})$};
  
 \end{tikzpicture}

 \caption{Vizualizace kartézského součinu $\clr{A} \times \clb{A}$.}
 \label{fig:kartezsky-soucin}
\end{figure}

Jakož tomu bylo i v případě sjednocení a průniku, lze definovat kartézský součin
více množin než dvou. Zde je však dlužen zřetel -- u kartézského součinu rovněž
(jako u rozdílu) záleží na pořadí násobení množin. Není možné definovat
kartézský součin množin $A_i$ pro libovolnou množinu indexů $I$, ale pouze pro
množinu \emph{uspořádanou} (tento pojem rovněž posléze připomeneme). Pro teď
budeme předpokládat, že $I = \{1,\ldots,n\}$ a zápisem
\[
 \prod_{i=1}^n A_i
\]
myslet množinu uspořádaných $n$-tic $(a_1,\ldots,a_n)$, kde $a_i \in A_i$ pro
každé $i \in \{1,\ldots,n\}$. Kartézský součin nekonečného množství množin lze
též definovat, ano akt to není přímočarý a naší věci zatím zbytečný.

Jsou-li všechny $A_i$, $i \in \{1,\ldots,n\}$, vskutku jednou množinou, $A$,
ujalo se -- poněkud přirozeně -- pro jejich kartézský součin mocninné značení.
To jest,
\[
 A^{n} \coloneqq \prod_{i = 1}^{n} A.
\]
