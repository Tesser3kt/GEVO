\subsection{Relace zobrazení}
\label{ssec:relace-zobrazeni}

Přišel na řadu poslední klíčový typ relace, a pro algebraika snad nejdůležitější
myšlenka matematiky vůbec. Na rozdíl od relací uspořádání a ekvivalence, není
zobrazení nutně relace na téže množině. Jak název napovídá, zobrazení má něco
někam ... \uv{zobrazovat}. Matematik zřídkakdy přemýšlí o zobrazení jako o
relaci, nýbrž o popisu způsobu, kterak se prvky jedné množiny \uv{přetvářejí v}
či \uv{zobrazují na} prvky množiny druhé.

Zobrazení je dáno vlastně jednou přímočarou podmínkou -- každý prvek první
množiny se může zobrazit na maximálně jeden prvek množiny druhé. Lidově řečeno,
není prvku povoleno, aby se rozdvojil, roztřetil, rozčtvrtil, rozpětil,
rozšestil, rozsedmil, rozosmil, rozdevítil a indukcí dále.

Ať tedy $R$ je relace mezi $A$ a $B$. Elegantním logickým zápisem podmínky, aby
byl každý prvek $a \in A$ v relaci $R$ s \emph{maximálně jedním} prvkem $b \in
B$, je výrok, že jsou-li dva prvky z $A$ v relaci s týmž prvkem z $B$, pak se
vskutku jedná o jediný prvek. Formální definice následuje.

\begin{definition}{Zobrazení}{zobrazeni}
 Ať $R \subseteq A \times B$ je relace mezi $A$ a $B$. Nazveme $R$
 \emph{zobrazením} (z $A$ do $B$), pokud splňuje
 \[
  \forall a_1,a_2 \in A \; \forall b \in B: a_1Rb \wedge a_2Rb \Rightarrow a_1 =
  a_2.
 \]
 Zobrazení mezi množinami obvykle zapisujeme malými písmeny latinské abecedy
 počínaje písmenem $f$ (od \textbf{f}unkce, jak se některým speciálním typům
 zobrazení často říká). Fakt, že $f$ je zobrazení z $A$ do $B$, symbolizujeme
 zápisem $f:A \to B$ nebo též $A \overset{f}{ \longrightarrow } B$.

 Když $(a,b) \in f$, nepíšeme $afb$ jako u obecné relace, ale spíše $f(a) = b$
 či $f:a \mapsto b$.
\end{definition}

Stejně jako uspořádání, i zobrazení se kreslí svým osobitým způsobem, který v
sobě nese jejich mimořádnou povahu. Protože, opakujeme, bývají zobrazení vnímána
vlastně jako \uv{šipky} mezi množinami nesoucí prvky z první do té druhé, i se
tak kreslí. Konkrétně, zobrazení $f:A \to B$ nakreslíme například tak, že si
prvky $A$ postavíme nalevo, prvky $B$ napravo, a pro každý vztah $f(a) = b$
nakreslíme šipku z $a$ do $b$. Z definici zobrazení povede z každého $a \in A$
nejvýše jedna šipka. Vizte \myref{obrázek}{fig:priklady-zobrazeni}.

\begin{figure}[ht]
 \centering
 \begin{subfigure}[t]{.45\textwidth}
  \centering
  \begin{tikzpicture}
   \foreach \y in {1,2,3} {
    \node[vertex,BrickRed] (a\y) at (0,-\y) {};
    \node[left=0mm of a\y,BrickRed] {$\y$};
    
    \node[vertex,RoyalBlue] (b\y) at (2,-\y) {};
   }
   \node[vertex,BrickRed] (a4) at (0,-4) {};
   \node[left=0mm of a4,BrickRed] {$4$};

   \node[right=0mm of b1,RoyalBlue] {$a$};
   \node[right=0mm of b2,RoyalBlue] {$b$};
   \node[right=0mm of b3,RoyalBlue] {$c$};

   \draw[thick,ForestGreen,->,shorten <=2pt,shorten >=2pt] (a1) -- (b2);
   \draw[thick,ForestGreen,->,shorten <=2pt,shorten >=2pt] (a2) -- (b1);
   \draw[thick,ForestGreen,->,shorten <=2pt,shorten >=2pt] (a3) -- (b2);
   \draw[thick,ForestGreen,->,shorten <=2pt,shorten >=2pt] (a4) -- (b2);
  \end{tikzpicture}
  \caption{Zobrazení $\clg{f} =
  \{(\clr{1},\clb{b}),(\clr{2},\clb{a}),(\clr{3},\clb{b}),(\clr{4},\clb{b})\}$.}
  \label{subfig:priklady-zobrazeni-1}
 \end{subfigure}
 \begin{subfigure}[t]{.45\textwidth}
  \centering
  \begin{tikzpicture}
   \foreach \y in {1,2,3,4} {
    \node[vertex,BrickRed] (a\y) at (0,-\y) {};
    \node[vertex,BrickRed] (b\y) at (2,-\y) {};
    \node[left=0mm of a\y,BrickRed] {$\y$};
    \node[right=0mm of b\y,BrickRed] {$\y$};
   }

   \draw[thick,ForestGreen,->,shorten <=2pt,shorten >=2pt] (a1) -- (b2);
   \draw[thick,ForestGreen,->,shorten <=2pt,shorten >=2pt] (a2) -- (b3);
   \draw[thick,ForestGreen,->,shorten <=2pt,shorten >=2pt] (a3) -- (b1);
   \draw[thick,ForestGreen,->,shorten <=2pt,shorten >=2pt] (a4) -- (b4);
  \end{tikzpicture}
  \caption{Zobrazení $\clg{g} =
  \{(\clr{1},\clr{2}),(\clr{2},\clr{3}),(\clr{3},\clr{1}),(\clr{4},\clr{4})\}$.
 Zobrazením \uv{prohazujícím} prvky na množině se říká \emph{permutace}.}
  \label{subfig:priklady-zobrazeni-2}
 \end{subfigure}
 \caption{Příklady zobrazení.}
 \label{fig:priklady-zobrazeni}
\end{figure}

K zobrazením se tokem historie navázaly přehoušle názvosloví. Zopakujeme je zde,
bo pěstujeme zámysl je v dalším textu bez výstrah dštíti.

Ať $f$ do konce sekce značí zobrazení $A \to B$. Množinu $A$ nazýváme
\emph{doménou} zobrazení $f$, značíme $\dom f$, a množinu $B$ jeho
\emph{kodoménou}, značíme $\codom f$. Množinu všech prvků z $B$, na které se
zobrazuje nějaký prvek z $A$, nazýváme \emph{obrazem} $f$, značíme $\img f$.
Formálně
\[
 \img f \coloneqq \{b \in B \mid \exists a \in A: f(a) = b\}.
\]
Triviálně platí $\img f \subseteq \codom f$, ale tyto množiny nemusejí být
stejné. Například, zobrazení $\clg{f}$
z~\myref{obrázku}{subfig:priklady-zobrazeni-1} má obraz $\{\clb{a},\clb{b}\}$,
ale kodoménu celou množinu $\{\clb{a},\clb{b},\clb{c}\}$.

Pro každý prvek $b \in \codom f$ definujeme jeho \emph{vzor} (při zobrazení
$f$), značíme $f^{-1}(b)$, jako množinu \textbf{všech} prvků z $A$, které se na
něj zobrazují. Formálně, pro $b \in \codom f$
\[
 f^{-1}(b) \coloneqq \{a \in A \mid f(a) = b\}.
\]
Triviálně $f^{-1}(b) \subseteq \dom f$ pro každé $b \in \codom f$, ale tyto
množiny jistě mohou být různé. Pro zobrazení $\clg{f}$ z
\myref{obrázku}{subfig:priklady-zobrazeni-1} platí $\clg{f}^{-1}(\clb{b}) =
\{\clr{1},\clr{2},\clr{4}\}$ a $\clg{f}^{-1}(\clb{c})=\emptyset$ a pro zobrazení
$\clg{g}$ z \myref{obrázku}{subfig:priklady-zobrazeni-2} platí
$\clg{g}^{-1}(\clr{2}) = \{\clr{1}\}$. Přestože je vzor prvku při zobrazení
oficiálně \textbf{množina}, budeme v případech jako byl tento poslední, kdy je
vzorem množina jednoprvková, psát většinou $\clg{g}^{-1}(\clr{2}) = \clr{1}$ a
tvrdit, že prvek $\clr{1}$ je \emph{vzorem} prvku $\clr{2}$ při zobrazení
$\clg{g}$.

Sekci završíme zopakováním speciálních typů zobrazení. Říkáme, že zobrazení
$f:A \to B$ je
\begin{itemize}
 \item \emph{prosté} (či \emph{injektivní}), když na každý prvek $B$ zobrazuje
  nejvýše jeden prvek $A$. Formálně lze psát, že
  \[
   \forall a_1,a_2 \in A: f(a_1) = f(a_2) \Rightarrow a_1 = a_2.
  \]
  Ekvivalentně můžeme tvrdit, že zobrazení je prosté, když $f^{-1}(b)$ má
  nejvýš jeden prvek pro každé $b \in B$ a, \textbf{je-li $A$ konečná}, pak je
  prostota $f$ vyjádřena též v podmínce $\# \img f = \# A$. Nabádáme čtenáře,
  aby si ekvivalenci těchto výroků rozmysleli. Symbolicky budeme občas zapisovat
  prostá zobrazení jako $f:A \hookrightarrow B$.
 \item \emph{na} (či \emph{surjektivní}), když na každý prvek $B$ zobrazuje
  nějaký prvek $A$. Formálně lze psát, že
  \[
   \forall b \in B \; \exists a \in A: f(a) = b.
  \]
  Ekvivalentně je $f$ na, když $f^{-1}(b)$ není prázdná pro žádný prvek $b \in
  B$ anebo, \textbf{ať už je $B$ konečná či ne}, když $\img f = B$. Opět
  nabádáme čtenáře k ověření této ekvivalence. Symbolicky budeme občas zapisovat
  surjektivní zobrazení jako $f:A \twoheadrightarrow B$.
 \item \emph{vzájemně jednoznačné} (též \emph{bijektivní}), když je prosté i na.
  Množiny, mezi kterými existuje bijektivní zobrazení, lze v mnoha situacích
  považovat za stejné, neboť mají vlastně tytéž prvky, akorát pod jinými
  \uv{jmény}. Z ekvivalentních definic prostých a surjektivních zobrazení plyne,
  že zobrazení je bijektivní, právě když $f^{-1}(b)$ je jednoprvková pro každý
  prvek $b \in B$ a též, \textbf{jsou-li $A$ i $B$ konečné}, když $\# \img f =
  \# A = \# B$. Symbolicky budeme občas zapisovat bijektivní zobrazení jako $f:A
  \leftrightarrow B$.
\end{itemize}

Pojmu bijekce se používá při formální definici velikosti množiny. Pro libovolnou
množinu $X$ je její \emph{velikost} definována jako takové přirozené číslo $n
\in \N$, že existuje bijekce $\{1,\ldots,n\} \leftrightarrow X$. Píšeme $\# X =
n$. Neexistuje-li takové přirozené číslo, říkáme, že $X$ je nekonečná, a píšeme
$\# X = \infty$.
