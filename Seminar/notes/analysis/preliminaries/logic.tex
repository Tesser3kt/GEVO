\section{Základní pojmy z logiky}
\label{sec:zakladni-pojmy-z-logiky}

Obyčejnou podobou matematické logiky je jazyk o
\begin{itemize}
 \item dvou konstantách:
  \begin{itemize}
   \item $0$ (též $\bot$) -- \textbf{lež},
   \item $1$ (též $\top$) -- \textbf{pravda};
  \end{itemize}
 \item dvou binárních operátorech $ \wedge $ (\textbf{a}, též
  \textbf{konjunkce}) a $ \vee $ (\textbf{nebo}, též \textbf{disjunkce})
  definovaných rovnostmi
  \begin{itemize}
   \item $(0 \wedge 0 = 0 \wedge 1 = 1 \wedge 0 = 0)$ a $(1 \wedge 1 = 1)$,
   \item $(0 \vee 0 = 0)$ a $(0 \vee 1 = 1 \vee 0 = 1 \vee 1 = 1)$.
  \end{itemize}
 \item unárním operátoru $\neg $ (\textbf{ne} či \textbf{negace}) definovaném
  rovnostmi $(\neg 1 = 0)$ a $(\neg 0 = 1)$;
 \item proměnných;
 \item dvou kvantifikátorech $ \forall $ (\textbf{pro všechny}, též
  \textbf{universální}) a $ \exists $ (\textbf{existuje}, též
  \textbf{existenční}).
\end{itemize}

K množině binárních operátorů se často též pro praktické účely přidávají $
\Rightarrow $ (\textbf{impli\-kace}, \textbf{když ..., tak ...}) a $
\Leftrightarrow $ (\textbf{ekvivalence}, \textbf{... právě tehdy, když ...})
definované rovnostmi
\[
 (x \Rightarrow y) = (\neg x \vee y) \quad \text{a} \quad (x \Leftrightarrow y)
 = ((x \Rightarrow y) \wedge (y \Rightarrow x)).
\]
Užitím konstant $0$ a $1$ by implikace byla definována rovnostmi
\[
 (0 \Rightarrow 0 = 0 \Rightarrow 1 = 1 \Rightarrow 1 = 1) \quad \text{a} \quad
 (1 \Rightarrow 0 = 0),
\]
zatímco ekvivalence rovnostmi
\[
 (0 \Leftrightarrow 0 = 1 \Leftrightarrow 1 = 1) \quad \text{a} \quad (0
 \Leftrightarrow 1 = 1 \Leftrightarrow 0 = 0).
\]

K matematické logice se též váže pojem \emph{výroku}. Výrokem nepřesně řečeno
míníme jakoukoli větu, o které lze tvrdit, že platí, nebo neplatí. Formálně se
výrok definuje poněkud obtížněji a význam natolik abstraktní definice pro účely
tohoto textu je přinejmenším sporný. Pochopitelně, výraz vzniklý z kratších
výroků užitím logických operátorů a kvantifikátorů je rovněž výrokem.

\begin{example}{}{vyroky}
 Ať $x$ je výrok \uv{Mám hlad.} a $y$ je \uv{Jdu do hospody.} Pak
 \begin{itemize}
  \item výrok $x \wedge y$ zní \uv{Mám hlad a jdu do hospody.}
  \item výrok $x \vee y$ zní \uv{Mám hlad nebo jdu do hospody.}
  \item výrok $\neg x$ zní \uv{Nemám hlad.} a $\neg y$ zní \uv{Nejdu do
   hospody.}
  \item výrok $x \Rightarrow y$ zní \uv{Když mám hlad, tak jdu do hospody.}
  \item výrok $x \Leftrightarrow y$ zní \uv{Mám hlad právě tehdy, když jdu do
   hospody.}
 \end{itemize}
 Sémantická hodnota uvedených výroků se pochopitelně liší.
\end{example}
\begin{warning}{}{nebo-implikace}
 \begin{itemize}
  \item Operátor $ \vee $ \textbf{není} výlučný. To jest, výrok $x \vee y$ je
   pravdivý i v případě, že $x$ je pravdivý a $y$ je pravdivý.
  \item Jazykové vyjádření výroku $x \Rightarrow y$ je v mírném rozporu s běžnou
   intuicí. Totiž, $x \Rightarrow y$ je pravdivý, kdykoli $x$ je lživý, neboť na
   základě lži nelze rozhodnout o pravdivosti žádného výroku. To znamená, že
   výrok \uv{Když mám hlad, tak jdu do hospody.} je pravdivý i tehdy, když nemám
   hlad, a přesto do hospody jdu.
 \end{itemize}
\end{warning}

Při zjišťování pravdivosti výroků na základě pravdivosti \uv{elementárních
výroků} (tedy výroků, které již nelze více dělit), které je tvoří, je užitečná
tzv. \emph{pravdivostní tabulka}. Jde o tabulku, která ve sloupcích obsahuje
stále složitější spojení elementárních výroků, a v posledním onen původní výrok.
V řádcích pak obsahuje pravdivostní hodnoty. Není obtížné si rozmyslet, že je-li
výrok složen z $n$ elementárních výroků spojených logickými operátory, pak má
jeho pravdivostní tabulka $2^{n}$ řádků -- každý pro jedno možné přiřazení $n$
pravdivostních hodnot (tj. 0 nebo 1) jeho elementárním výrokům.

Pro práci s výroky je užitečné si pamatovat (a není ani těžké si rozmyslet), že
negace výroku způsobí nahrazení každého elementárního výroku jeho negací,
prohození všech operátorů $ \wedge $ a $ \vee $ a rovněž prohození
kvantifikátorů $ \exists $ a $ \forall $. Například
\[
 \neg ( \exists x:(x \vee y \wedge \neg z)) = ( \forall x:(\neg x \wedge \neg y
 \vee z)).
\]
V případě implikace $( \Rightarrow )$ pracujeme zkrátka s definicí a dostaneme
\[
 \neg (x \Rightarrow y) = (x \wedge \neg y).
\]
Negovat ekvivalenci je mírně složitější, bo je konjunkcí dvou implikací.
Výpočtem dostaneme
\begin{equation*}
 \begin{split} 
  \neg (x \Leftrightarrow y) &= \neg ((x \Rightarrow y) \wedge (y \Rightarrow
  x))\\
                             &= (\neg (x \Rightarrow y) \vee \neg (y \Rightarrow
  x))\\
                             &= ((x \wedge \neg y) \vee (y \wedge \neg x)).
 \end{split}
\end{equation*}

Ověříme si pravdivostní tabulkou na příkladě ekvivalence, že tento \uv{selský}
přístup k~negování výroků funguje (to samozřejmě \textbf{není} důkaz, že funguje
obecně).

Ekvivalence $x \Leftrightarrow y$ je pravdivá tehdy, když $x$ má stejnou hodnotu
jako $y$. Její negace je tudíž pravdivá, když $x$ a $y$ nabývají hodnot
opačných. Sestrojíme pravdivostní tabulku pro výrok $(x \wedge \neg y) \vee (y
\wedge \neg x)$.

\begin{table}[h]
 \centering
 \begin{tabular}{c|c|c|c|c|c|c}
  $x$ & $y$ & $\neg x$ & $\neg y$ & $x \wedge \neg y$ & $y \wedge \neg x$ & $(x
  \wedge \neg y) \vee (y \wedge \neg x)$\\
  \toprule
  0 & 0 & 1 & 1 & 0 & 0 & 0\\
  0 & 1 & 1 & 0 & 0 & 1 & 1\\
  1 & 0 & 0 & 1 & 1 & 0 & 1\\
  1 & 1 & 0 & 0 & 0 & 0 & 0
 \end{tabular}
 \caption{Pravdivostní tabulka výroku $(x \wedge \neg y) \vee (y \wedge \neg
 x)$.}
 \label{table:negace-ekvivalence}
\end{table}

Vidíme, že $(x \wedge \neg y) \vee (y \wedge \neg x)$ je skutečně negací
ekvivalence, neboť je pravdivý přesně ve chvíli, kdy $x$ a $y$ mají navzájem
opačné pravdivostní hodnoty.
