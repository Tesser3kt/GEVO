\chapter*{Seznam cvičení}
\addcontentsline{toc}{chapter}{Seznam cvičení}

\section*{\nameref*{chap:ciselne-obory}}

\begin{enumerate}
 \item Dokažte, že každé těleso je oborem integrity.
 \item Množinou $\Z$ zde myslíme tu z \myref{definice}{def:cela-cisla}. Ověřte,
  že
  \begin{enumerate}
  \item relace $ \sim _{\Z}$ je skutečně ekvivalence;
  \item operace $+$ a $ \cdot $ jsou dobře definované. To znamená, že nezávisí
   na volbě konkrétního reprezentanta z každé třídy ekvivalence. Ještě
   konkrétněji, dobrá definovanost zde značí fakt, že
   \begin{align*}
    [(a,b)]_{ \sim_{\Z}} + [(c,d)]_{ \sim _{\Z}} &= [(a',b')]_{ \sim _{\Z}} +
    [(c',d')]_{ \sim _{\Z}},\\
    [(a,b)]_{ \sim_{\Z}} \cdot [(c,d)]_{ \sim _{\Z}} &= [(a',b')]_{ \sim _{\Z}}
    \cdot [(c',d')]_{ \sim _{\Z}},
   \end{align*}
   kdykoli $(a,b) \sim _{\Z} (a',b')$ a $(c,d) \sim _{\Z} (c',d')$;
  \item operace $+$, $-$ a inverz $-$ podle naší definice souhlasí s operacemi
   danými stejnými symboly na \uv{běžné} verzi celých čísel při korespondenci
   \[
    [(a,b)]_{ \sim_{\Z}} \leftrightarrow a - b.
   \]
   Konkrétně, pro operaci $+$ toto znamená, že platí korespondence
   \[
    [(a,b)]_{ \sim_{\Z}} + [(c,d)]_{ \sim _{\Z}} \leftrightarrow (a - b) + (c -
    d)
   \]
   a nezávisí na výběru reprezentanta z tříd ekvivalence $[(a,b)]_{ \sim _{\Z}}$
   a $[(c,d)]_{ \sim_{\Z}}$.
 \end{enumerate}
\end{enumerate}


