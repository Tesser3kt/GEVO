\chapter*{Seznam cvičení}
\addcontentsline{toc}{chapter}{Seznam cvičení}

\section*{\nameref*{chap:ciselne-obory}}

\begin{enumerate}
 \item Dokažte, že každé těleso je oborem integrity.
 \item Množinou $\Z$ zde myslíme tu z \myref{definice}{def:cela-cisla}. Ověřte,
  že
  \begin{enumerate}
  \item relace $ \sim _{\Z}$ je skutečně ekvivalence;
  \item operace $+$ a $ \cdot $ jsou dobře definované. To znamená, že nezávisí
   na volbě konkrétního reprezentanta z každé třídy ekvivalence. Ještě
   konkrétněji, dobrá definovanost zde značí fakt, že
   \begin{align*}
    [(a,b)]_{ \sim_{\Z}} + [(c,d)]_{ \sim _{\Z}} &= [(a',b')]_{ \sim _{\Z}} +
    [(c',d')]_{ \sim _{\Z}},\\
    [(a,b)]_{ \sim_{\Z}} \cdot [(c,d)]_{ \sim _{\Z}} &= [(a',b')]_{ \sim _{\Z}}
    \cdot [(c',d')]_{ \sim _{\Z}},
   \end{align*}
   kdykoli $(a,b) \sim _{\Z} (a',b')$ a $(c,d) \sim _{\Z} (c',d')$;
  \item operace $+$, $-$ a inverz $-$ podle naší definice souhlasí s operacemi
   danými stejnými symboly na \uv{běžné} verzi celých čísel při korespondenci
   \[
    [(a,b)]_{ \sim_{\Z}} \leftrightarrow a - b.
   \]
   Konkrétně, pro operaci $+$ toto znamená, že platí korespondence
   \[
    [(a,b)]_{ \sim_{\Z}} + [(c,d)]_{ \sim _{\Z}} \leftrightarrow (a - b) + (c -
    d)
   \]
   a nezávisí na výběru reprezentanta z tříd ekvivalence $[(a,b)]_{ \sim _{\Z}}$
   a $[(c,d)]_{ \sim_{\Z}}$.
 \end{enumerate}
 \item Ověřte, že $ \sim _{\Q}$ z \myref{definice}{def:racionalni-cisla} je
  skutečně ekvivalence a že operace $+$ a $ \cdot $ na $\Q$ jsou dobře
  definované (nezávisejí na výběru reprezentanta) a odpovídají \uv{obvyklým}
  operacím zlomků.
\end{enumerate}

\section*{\nameref*{chap:posloupnosti-limity-a-realna-cisla}}

\begin{enumerate}
 \item Dokažte, že posloupnost $a:\N \to \Q$ je konvergentní právě tehdy, když
  \[
   \forall \varepsilon>0 \; \exists n_0 \in \N \; \forall m,n \geq n_0: |x_m -
   x_n| < C\varepsilon
  \]
  pro libovolnou \textbf{kladnou} konstantu $C \in \Q$.
 \item Dokažte, že každá posloupnost $a:\N \to \Q$ má nejvýše jednu limitu.
  Hint: použijte \hyperref[lem:trojuhelnikova-nerovnost]{trojúhelníkovou
  nerovnost}.
 \item Dokažte, že jsou-li $x,y$ konvergentní posloupnosti racionálních čísel,
  pak je posloupnost $x \cdot y$ rovněž konvergentní. Kromě
  \hyperref[lem:trojuhelnikova-nerovnost]{trojúhelníkové nerovnosti} je zde
  třeba použít i \myref{lemma}{lem:konvergentni-omezena}.
 \item Dokažte, že zobrazení $\xi$ z \eqref{eq:Q-into-R} je
  \begin{itemize}
   \item dobře definované -- tzn. že když $p = q$, pak $[(p)] = [(q)]$ -- a
   \item prosté.
  \end{itemize}
 \item Určete z \hyperref[def:supremum-a-infimum]{definice suprema a infima}
  $\inf \emptyset$ a $\sup \emptyset$.
 \item Dokažte, že $\sup X$ a $\inf X$ jsou určeny jednoznačně.
 \item Dokažte všechna (\textbf{proč?}) a (\textbf{dokažte!}) v důkazu
  \myref{tvrzení}{prop:axiom-uplnosti}.
 \item Dokažte \myref{tvrzení}{prop:axiom-infima}. Doporučujeme čtenářům se
  zamyslet, jak tvrzení snadno plyne z \hyperref[prop:axiom-uplnosti]{axiomu
  úplnosti}, aniž opakují konstrukci z jeho důkazu.
 \item Dokažte část (b) \myref{lemmatu}{lem:limita-monotonni-posloupnosti}.
 \item Dokažte, že \myref{lemma}{lem:archimedova-vlastnost-realnych-cisel} je
  důsledkem \myref{tvrzení}{prop:hustota-q-v-r}.
 \item Dokažte, že pro čísla $x,y \in \R$ platí
  \[
   | |x| - |y| | \leq |x - y|.
  \]
 \item Dokažte, že pro všechna $a \in \R, a > 0$ platí
  \[
   \lim_{n \to \infty} \sqrt[n]{a} = 1.
  \]
 \item Dokažte, že
  \[
   \lim_{n \to \infty} \sqrt[n]{n!} = \infty.
  \]
  \textbf{Hint}: Rozložte součin $n!$ na dvě poloviny a tu větší zespodu
  odhadněte vhodnou posloupností jdoucí k $\infty$.
 \item Spočtěte
  \[
   \lim_{n \to \infty} \sqrt{4n^2 - n} - 2n.
  \]
 \item Spočtěte
  \[
   \lim_{n \to \infty} (-1)^{n}\sqrt{n}(\sqrt{n+1}-\sqrt{n}).
  \]
 \item (těžké) Spočtěte limitu posloupnosti $a:\N \to \R$ zadané rekurentním
  vztahem
  \begin{align*}
   a_0 &\coloneqq 10,\\
   a_{n+1} & \coloneqq 6 - \frac{5}{a_n} \text{ pro } n \in \N.
  \end{align*}
 \item Ať $\sum_{n = 0}^{\infty} a_n$ je \textbf{divergentní} řada (tj. nemá
  žádný nebo konečný součet). Ať $s_n$ je posloupnost částečných součtů
  posloupnosti $a_n$. Dokažte, že i řada $\sum_{n = 0}^{\infty} a_n / s_n$ je
  divergentní.
 \item Vyšetřete konvergenci řady
  \[
   \sum_{n=0}^{\infty} \frac{n^22^{n}+3}{(n^{4}+1)^{6}}.
  \]
 \item Vyšetřete konvergenci řady
  \[
   \sum_{n = 0}^{\infty} n^{c}(\sqrt{n+1}-\sqrt{n})
  \]
  v závislosti na konstantě $c \in \R$.
 \item (těžké) Dokažte následující tvrzení, známé jako \emph{Raabeovo
  kritérium}: Nechť $\sum_{n = 0}^{\infty} a_n$ je řada s~nezápornými členy.
  \begin{enumerate}[label=(\alph*)]
   \item Platí-li $\lim_{n \to \infty} n \left( \frac{a_n}{a_{n+1}} - 1 \right) >
    1$, pak řada $\sum_{n = 0}^{\infty} a_n$ konverguje.
   \item Existuje-li $n_0 \in \N$ takové, že pro $n \geq n_0$ je
   \[
    n \left( \frac{a_n}{a_{n+1}} - 1 \right) \leq 1,
   \]
   pak řada $\sum_{n = 0}^{\infty} a_n$ diverguje.
  \end{enumerate}
\end{enumerate}

\section*{\nameref*{chap:limity-funkci}}

\begin{enumerate}
 \item Dokažte, že pro reálnou funkci $f$ a $a \in \R^{*}$ platí
  \[
   \lim_{x \to a} f(x) = 0 \Leftrightarrow \lim_{x \to a} |f(x)| = 0.
  \]
 \item Dokažte tvrzení (b) a (c) ve \myref{větě}{thm:aritmetika-limit-funkci}.
 \item Spočtěte následující limity funkcí
  \begin{align*}
   &\lim_{x \to 1} \frac{3x^{4} - 4x^3 + 1}{(x-1)^2},\\
   &\lim_{x \to 0} \frac{3x + \frac{2}{x}}{x + \frac{4}{x}},\\
   &\lim_{x \to -\infty} \sqrt{x^2 + x} - x,\\
   &\lim_{x \to \infty} \frac{\sqrt{x-1}-2x}{x-7}.
  \end{align*}
 \item Dokažte bod (2) ve \myref{větě}{thm:limita-monotonni-funkce}.
\end{enumerate}

\section*{\nameref*{chap:derivace}}
\begin{enumerate}
 \item Dokažte, že derivace funkce $f(x) = |x|$ v bodě $0$ neexistuje.
 \item Připomeňme, že funkce \emph{signum} je definována následovně.
  \[
   \sign(x) \coloneqq \begin{cases}
    1,& \text{když } x > 0,\\
    0,& \text{když } x = 0,\\
    -1,& \text{když } x < 0.
   \end{cases}
  \]
  Dokažte, že $\sign'(0) = \infty$.
 \item Dokažte body (1) a (2) ve \myref{větě}{thm:aritmetika-derivaci}.
 \item Nalezněte lokální i globální extrémy funkce $f(x) = 2 x^{3} - 3 x^{2} - 12 x -
  4$.
 \item Použijte \myref{důsledek}{cor:vztah-derivace-a-monotonie} k důkazu, že
  spojitá funkce $f:I \to \R$ mající nulovou derivaci na $I$, je konstantní.
\end{enumerate}
