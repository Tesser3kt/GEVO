\chapter*{Předmluva}

Matematická analýza je věda o reálných číslech; tuším ovšem, že kolegové
analytici mě za ono nedůstojně zjednodušující tvrzení rádi mít příliš nebudou.
Snad mohou nicméně souhlasit, že v~jejím jádru je pojem \emph{nekonečna}. Nikoli
nutně ve smyslu čísla, jež převyšuje všechna ostatní, ale spíše myšlenky, jež
zaštiťuje přirozené jevy jako \emph{okamžitá změna}, \emph{blížení} či
\emph{kontinuum}.

O zrod matematické analýzy, jež zvláště v zámoří sluje též \emph{kalkulus}, se
bez pochyb podělili (nezávisle na sobě) Sir Isaac Newton a Gottfried Wilhelm
Leibniz v 17. století po Kristu. Sir Isaac Newton se tou dobou zajímal o dráhy
vesmírných těles a učinil dvě zásadní pozorování -- zemská tíže působí na
objekty zrychlením a zrychlení je \emph{velikost okamžité změny} rychlosti.
Potřeboval tedy metodu, jak onu velikost spočítat. Vynález takové metody po
přirozeném zobecnění vede ihned na teorii tzv. \emph{limit}, které právě tvoří
srdce kalkulu. Pozoruhodné je, že Gottfried Leibniz, nejsa fyzik, dospěl ke
stejným výsledkům zpytem geometrických vlastností křivek. V jistém přirozeném
smyslu, který se zavazujeme rozkrýt, jsou totiž tečny \emph{limitami} křivek. Ve
sledu těchto rozdílů v přístupu obou vědců se v teoretické matematice dodnes, s
mírnými úpravami, používá při studiu limit značení Leibnizovo, zatímco ve fyzice
a diferenciální geometrii spíše Newtonovo.

Následující text je shrnutím -- lingvistickým, vizuálním a didaktickým
pozlacením -- teorie limit. Hloubka i šíře této teorie ovšem přesáhla původní
očekávání a kalkulus se stal součástí nespočtu matematických (samozřejmě i
fyzikálních) odvětví bádání. První kapitola je věnována osvěžení nutných pojmů k
pochopení textu. Pokračují pojednání o limitách posloupností a reálných číslech,
limitách součtů, limitách funkcí a, konečně, derivacích. Tento sled není volen
náhodně, nýbrž, kterak bude vidno, znalost předšedších kapitol je nutná k
porozumění příchozích.

Jelikož se jedná o text průběžně doplňovaný a upravovaný, autor vyzývá čtenáře,
by četli okem kritickým a myslí čistou, poskytovali připomínky a návrhy ke
zlepšení.
