\chapter{Derivace}
\label{chap:derivace}

% TODO
\clr{\textbf{Tato kapitola se nachází v pracovní verzi. Neočekávejte obrázky,
naopak očekávejte chyby a podivné formulace.}}

Derivace funkce je vlastně \uv{velikost změny} funkce v daný okamžik. Takový
pojem se může zdát dost neintuitivní -- \uv{Jak se může cosi měnit v jeden
okamžik?} -- ale vskutku se s ním setkáváme běžně a divné nám to ani nepřijde.

Kupříkladu rychlost je derivací polohy, tj. představuje velikost změny polohy v
čase. Když automobilový tachometr ukazuje, že jedeme rychlostí 60 km/h, co to
přesně znamená? Jak si mám představit, že \emph{teď} se moje poloha mění o 60
kilometrů v rámci jedné hodiny? Jeden způsob vnímání dlí v obraze, že za příští
hodinu urazím 60 km. Jako hrubá aproximace snad postačuje, ale je snad zřejmé,
že pokud se moje rychlost během oné hodiny mění, změní se i výsledně uražená
vzdálenost.

Mnoho lidí, kteří se dívají za jízdy na tachometr, si pravděpodobně neuvědomuje,
že rychlost jízdy je limitní pojem. Fakt, že \emph{přesně v tuto chvíli} jedu
například právě 60 km/h znamená, že čím menší zlomek jedné hodiny nahlédnu do
budoucnosti (či do minulosti), tím blíže bude velikost změny mojí polohy
odpovídající zlomek 60 km. Za předpokladu, že moje rychlost není konstantní,
nebude tato změna nikdy dokonale taková. Tudíž, vím-li, že se pohybuji rychlostí
60 km/h, nevím toho v~principu mnoho. Nabývám tím pouze práva tvrdit, že se
zmenšujícím se časovým rozdílem mezi přítomností a jistým okamžikem v budoucnu
(či v minulu) klesá i rozdíl mezi skutečně překonanou vzdáleností a tou
odpovídající uvedené rychlosti.

Formalizace \emph{velikosti okamžité změny} není užitím
\hyperref[def:oboustranna-limita-funkce]{pojmu limity funkce} nikterak obtížná.
Před formální definicí si ji ukážeme na příkladě z předšedších odstavců. Ať je
moje poloha v čase daná funkcí $p:[0,1] \to \R$, kde $p(0)$ je moje poloha na
začátku cesty a $p(1)$ na jejím konci. Zanedbáme pro jednoduchost fakt, že
polohu určuje pouze jediné reálné číslo místo své příslušné vícedimenzionální
varianty.

Co přesně myslíme rychlostí změny? Napovědí již použité jednotky -- km/h.
\emph{Změna polohy} je zde rozdíl výsledné polohy od nynější za daný čas. Pro
daný čas $t \in [0,1]$, pak \emph{změna polohy} v tomto čase za danou dobu $h
\in [0,1]$ je
\[
 z_t(h) = \frac{p(t + h) - p(t)}{h}.
\]

Vskutku, ať například $t = 0$, tedy jsem na začátku cesty, třeba na třicátém
kilometru dálnice, čili $p(0) = 30$. Pojedu jednu hodinu. V půli cesty se
podívám na postranní milník a vidím, že jsem na stém kilometru téže dálnice,
čili $p(1 / 2) = 100$. Urazil jsem tudíž za půl hodiny přesně 70 km. Z~těchto
údajů soudím, že byla-li by moje rychlost konstantní, urazil bych 140 km za
jednu hodinu. Vyjádřeno pomocí funkce změny, mám
\[
 z_0(1 / 2) = \frac{p(0 + 1 / 2) - p(0)}{1 / 2} = \frac{100 - 30}{1 / 2} = 140.
\]
Velikost okamžité změny funkce $p$ v daném čase $t$ bude pročež limita uvedené
funkce $z$, jak se bude doba $h$ blížit $0$.

Odstavce výše shrneme v následujících definicích.

\begin{definition}{Funkce změny}{funkce-zmeny}
 Ať $f:M \to \R$ je reálná funkce. \emph{Funkcí změny} $f$ v bodě $a \in M$
 myslíme funkci $z_a:\R \to \R$ danou předpisem
 \[
  z_a(h) = \frac{f(a + h) - f(a)}{h}.
 \]
\end{definition}

\begin{definition}{Derivace funkce}{derivace-funkce}
 Ať $f:M \to \R$ je reálná funkce. \emph{Derivací}, nebo též \emph{funkcí
 okamžité změny}, funkce $f$ v bodě $a \in M$ myslíme limitu
 \[
  \lim_{h \to 0} z_a(h) = \lim_{h \to 0} \frac{f(a + h) - f(a)}{h},
 \]
 pokud tato existuje. V takovém případě ji značíme $f'(a)$.
\end{definition}

\begin{remark}{Jednostranné derivace}{jednostranne-derivace}
 Podobně lze rovněž definovat derivaci funkce $f$ v bodě $a$ zleva, resp.
 zprava, jako
 \[
  \lim_{h \to 0^{-}} z_a(h), \text{ resp. } \lim_{h \to 0^{+}} z_a(h).
 \]
 Intuitivně tato aproximuje, jak se funkce $f$ před okamžikem měnila, resp. bude
 za okamžik měnit. Přirozeně, derivace funkce existuje právě tehdy, když
 existují jednostranné derivace a jsou si rovny.
\end{remark}

Přirozená substituce vede na alternativní vzorec pro výpočet $f'(a)$.

\begin{lemma}{Alternativní vzorec derivace}{alternativni-vzorec-derivace}
 Ať $f:M \to \R$ a $a \in M$. Pak $f'(a)$ existuje právě tehdy, když existuje
 \[
  \lim_{x \to a} \frac{f(x) - f(a)}{x - a}
 \]
 a v tomto případě se rovnají.
\end{lemma}
\begin{lemproof}
 Dokážeme implikaci $( \Rightarrow )$. Ať existuje $f'(a)$. Položme
 \[
  F(h) \coloneqq \frac{f(a + h) - f(a)}{h},
 \]
 tj. $\lim_{h \to 0} F(h) = f'(a)$, a rovněž
 \[
  g(x) = x - a.
 \]
 Pak platí $\lim_{x \to a} g(x) = 0$ a $g$ je na okolí $a$ prostá. Z
 \hyperref[thm:limita-slozene-funkce]{věty o limitě složené funkce} platí
 \[
  f'(a) = \lim_{h \to 0} F(h) = \lim_{x \to a} (F \circ g)(x) = \lim_{x \to a}
  \frac{f(x) - f(a)}{x - a}.
 \]

 Pro důkaz $( \Leftarrow )$ předpokládejme, že existuje
 \[
  L \coloneqq \lim_{x \to a} \frac{f(x) - f(a)}{x - a}.
 \]
 Položme
 \[
  F(x) \coloneqq \frac{f(x) - f(a)}{x - a}
 \]
 a $g(h) = h + a$. Pak $\lim_{x \to a} F(x) = L$, $\lim_{h \to 0} g(h) = a$ a
 $g$ je prostá na okolí $a$. Opět z \hyperref[thm:limita-slozene-funkce]{věty o
 limitě složené funkce} dostaneme rovnost
 \[
  L = \lim_{x \to a} F(x) = \lim_{h \to 0} (F \circ g)(h) = \lim_{h \to 0}
  \frac{f(a + h) - f(a)}{h} = f'(a),
 \]
 což zakončuje důkaz.
\end{lemproof}

\begin{problem}{Derivace konstantní funkce}{derivace-konstantni-funkce}
 Ať $f(x) = c \in \R$ pro každé $x \in M$. Dokažte, že pak platí $f'(a) = 0$ pro
 každé $a \in M$.
\end{problem}
\begin{probsol}
 Platí
 \[
  f'(a) = \lim_{x \to a} \frac{f(x) - f(a)}{x - a} = \lim_{x \to a} \frac{c -
  c}{x - a} = \lim_{x \to a} \frac{0}{x-a} = 0,
 \]
 jak jsme chtěli.
\end{probsol}

\begin{problem}{Derivace polynomu}{derivace-polynomu}
 Ať $f(x) \coloneqq x^{n}$, kde $n \in \N$. Dokažte, že $f'(a) = na^{n-1}$.
\end{problem}
\begin{probsol}
 Počítáme
 \[
  f'(a) = \lim_{x \to a} \frac{f(x) - f(a)}{x - a} = \lim_{x \to a} \frac{x^{n}
  - a^{n}}{x - a}.
 \]
 Snadno ověříme, že platí
 \[
  x^{n} - a^{n} = (x - a)\left( \sum_{i=0}^{n-1} x^{i}a^{n-i} \right).
 \]
 Z čehož ihned
 \[
  \frac{x^{n} - a^{n}}{x - a} = \sum_{i=0}^{n-1} x^{i}a^{n-i},
 \]
 a tedy
 \[
  \lim_{x \to a} \frac{x^{n} - a^{n}}{x-a} = \lim_{x \to a} \sum_{i=0}^{n-1}
  x^{i}a^{n-i} = \sum_{i=0}^{n-1} a^{i}a^{n-i} = \sum_{i=0}^{n-1} a^{n-1} =
  na^{n-1},
 \]
 což bylo dokázati.
\end{probsol}

\begin{exercise}{}{derivace-absolutni-hodnoty}
 Dokažte, že derivace funkce $f(x) = |x|$ v bodě $0$ neexistuje.
\end{exercise}

\begin{exercise}{}{derivace-signum}
 Připomeňme, že funkce \emph{signum} je definována následovně.
 \[
  \sign(x) \coloneqq \begin{cases}
   1,& \text{když } x > 0,\\
   0,& \text{když } x = 0,\\
   -1,& \text{když } x < 0.
  \end{cases}
 \]
 Dokažte, že $\sign'(0) = \infty$.
\end{exercise}

\begin{lemma}{Vztah derivace a spojitosti}{vztah-derivace-a-spojitosti}
 Ať $f:M \to \R$ je reálná funkce, $a \in M$ a existuje \textbf{konečná}
 $f'(a)$. Pak je $f$ v bodě $a$ spojitá. 
\end{lemma}
\begin{lemproof}
 Jelikož $f'(a)$ existuje konečná, z \hyperref[thm:aritmetika-limit]{věty o
 aritmetice limit} plyne, že
 \[
  \lim_{x \to a} f'(a) \cdot (x - a) = 0.
 \]
 To ovšem znamená, že
 \[
  \lim_{x \to a} \frac{f(x) - f(a)}{x-a} \cdot (x-a) = \lim_{x \to a} f(x) -
  f(a) = 0,
 \]
 z čehož ihned
 \[
  \lim_{x \to a} f(x) = f(a),
 \]
 čili $f$ je spojitá v $a$.
\end{lemproof}

\begin{warning}{}{spojitost-a-derivace}
 Předpoklad nejen existence, ale i \textbf{konečnosti} derivace ve znění
 \myref{lemmatu}{lem:vztah-derivace-a-spojitosti} nelze vynechat. Z
 \myref{cvičení}{exer:derivace-signum} plyne, že $\sign'(0)$ existuje, ale
 funkce $\sign$ zcela jistě není v bodě $0$ spojitá.
\end{warning}

\section{Základní poznatky o derivaci}
\label{sec:zakladni-poznatky-o-derivaci}

Tato sekce shrnuje základní tvrzení, která činí z výpočtu derivací překvapivě
silně algoritmický proces, přirozeně za předpokladu znalosti derivací jistých
\uv{běžných} funkcí.

Začneme tím, jak se derivace chová vzhledem ke součtu, součinu a podílu funkcí.

\begin{theorem}{Aritmetika derivací}{aritmetika-derivaci}
 Ať $f,g:M \to \R$ jsou reálné funkce a $a \in M$. Pak platí
 \begin{enumerate}
  \item $(f + g)'(a) = f'(a) + g'(a)$, dává-li pravá strana smysl;
  \item $(f \cdot g)'(a) = f'(a)g(a) + f(a)g'(a)$, dává-li pravá strana smysl,
   $g$ je spojitá v $a$ platí $g(a) \neq 0$;
  \item $(f / g)'(a) = \frac{f'(a)g(a) - f(a)g'(a)}{g^2(a)}$, dává-li pravá
   strana smysl a $f / g$ je spojitá v $a$.
 \end{enumerate}
\end{theorem}
\begin{thmproof}
 Podobně jako tomu bylo i v případě \hyperref[thm:aritmetika-limit]{věty o
 aritmetice limit}, je důkaz tohoto tvrzení zdlouhavý a výpočetní. Důkaz bodu
 (1) je triviální a bodu (2) snadný; jsou pročež přenechány čtenáři. Dokážeme
 bod (3).

 Protože $f / g$ je z předpokladu spojitá v $a$, nalezneme $\delta>0$ takové, že
 $g$ je nenulová na $B(a,\delta)$. Pro $x \in B(a,\delta)$ počítáme
 \begin{align*}
  \frac{f(x)}{g(x)} - \frac{f(a)}{g(a)} &= \frac{f(x)g(a) -
  f(a)g(x)}{g(x)g(a)}\\
                                        &= \frac{f(x)g(a) - f(a)g(a) + f(a)g(a)
                                        - f(a)g(x)}{g(x)g(a)}\\
                                        &= \frac{1}{g(x)g(a)}\left( (f(x) -
                                        f(a))g(a) - f(a)(g(x) - g(a)) \right).
 \end{align*}
 Pak
 \begin{align*}
  \left( \frac{f}{g} \right)'(a) &= \lim_{x \to a} \frac{1}{x-a}\left(
  \frac{f(x)}{g(x)} - \frac{f(a)}{g(a)} \right)\\
                                 &= \lim_{x \to a} \frac{1}{g(x)g(a)}\left(
                                 \frac{f(x) - f(a)}{x-a}g(a) - f(a) \frac{g(x) -
                                g(a)}{x-a} \right)\\
                                 &= \lim_{x \to a} \frac{1}{g(x)g(a)}\left(
                                 \lim_{x \to a} \frac{f(x) - f(a)}{x-a}g(a) -
                                \lim_{x \to a} f(a) \frac{g(x) - g(a)}{x-a}
                               \right)\\
                                 &= \frac{1}{g^2(a)}\left( f'(a)g(a) - f(a)g'(a)
                                 \right),
 \end{align*}
 čímž je důkaz hotov.
\end{thmproof}

\begin{exercise}{}{aritmetika-derivaci}
 Dokažte body (1) a (2) ve \myref{větě}{thm:aritmetika-derivaci}.
\end{exercise}

\begin{theorem}{Derivace složené funkce}{derivace-slozene-funkce}
 Ať $g$ je spojitá v bodě $a \in \R$ a má v tomto bodě derivaci. Nechť $f$ má
 derivaci v bodě $g(a)$. Potom
 \[
  (f \circ g)'(a) = f'(g(a)) \cdot g'(a).
 \]
\end{theorem}
\begin{thmproof}
 Zdlouhavý a výpočetní. Přeskočíme.
\end{thmproof}

\begin{theorem}{Derivace inverzní funkce}{derivace-inverzni-funkce}
 Ať $I \subseteq \R$ je interval a $f:I \to \R$ je spojitá a rostoucí či
 klesající na $I$. Pak pro bod $a$ ve vnitřku $I$ platí:
 \begin{enumerate}
  \item Je-li $f'(a) \neq 0$, potom $(f^{-1})'(f(a)) = 1 / f'(a)$;
  \item je-li $f'(a) = 0$, potom $(f^{-1})'(f(a)) = \infty$, když $f$ je
   rostoucí, a $f^{-1}(f'(a)) = -\infty$, když $f$ je klesající.
 \end{enumerate}
\end{theorem}

\begin{thmproof}
 Předpokládejme, že $f$ je rostoucí. Pro klesající funkci lze důkaz vést
 obdobně.

 Protože $f$ je spojitá, je z \hyperref[thm:bolzanova]{Bolzanovy věty} $J
 \coloneqq f(I)$ interval. Dále, ježto $a$ leží ve vnitřku $I$, leží rovněž
 $f(a)$ ve vnitřku $J$. Existuje pročež $\varepsilon>0$ takové, že
 $B(f(a),\varepsilon) \subseteq J$. Dále, $f$ je rostoucí, tedy speciálně
 prostá, takže existuje $f^{-1}:J \to I$, která je (ze spojitosti $f$) rovněž
 spojitá.

 Volme nyní $\delta>0$ tak, aby $f(B(a,\delta)) \subseteq B(f(a),\varepsilon)$ a
 pro $x \in B(a,\delta)$ definujme
 \[
  \varphi(x) \coloneqq \frac{f(x) - f(a)}{x - a}.
 \]
 Pak přirozeně $\lim_{x \to a} \varphi(x) = f'(a)$. Díky prostotě $f^{-1}$ na
 $B(f(a),\varepsilon)$ lze díky \hyperref[thm:limita-slozene-funkce]{větě o
 limitě složené funkce} počítat
 \begin{equation}
  \label{eq:derivace-inverzu}
  \tag{$\heartsuit$}
  \begin{split}
   f'(a) &= \lim_{x \to a} \varphi(x) = \lim_{y \to f(a)} (\varphi \circ
   f^{-1})(y)\\
         &= \lim_{y \to f(a)} \frac{f(f^{-1}(y)) - f(a)}{f^{-1}(y) - a} = \lim_{y
         \to f(a)} \frac{y - f(a)}{f^{-1}(y) - a}.
  \end{split}
 \end{equation}
 Předpokládejme nejprve, že $f'(a) \neq 0$. Pak z
 \hyperref[thm:aritmetika-limit-funkci]{věty o aritmetice limit} platí
 \[
  \frac{1}{f'(a)} = \lim_{y \to f(a)} \frac{f^{-1}(y) - a}{y - f(a)} = \lim_{y
  \to f(a)} \frac{f^{-1}(y) - f^{-1}(f(a))}{y - f(a)} = (f^{-1})'(f(a)).
 \]
 
 Nyní ať $f'(a) = 0$. Pak díky~\eqref{eq:derivace-inverzu} máme
 \[
  \lim_{y \to f(a)} \frac{y-f(a)}{f^{-1}(y) - a} = \lim_{y \to f(a)}
  \frac{y-f(a)}{f^{-1}(y) - f^{-1}(f(a))} = 0.
 \]
 Funkce
 \[
  \psi(y) \coloneqq \frac{y-f(a)}{f^{-1}(y) - f^{-1}(f(a))}
 \]
 je na okolí $R(f(a),\varepsilon)$ kladná, neboť $f^{-1}$ je rostoucí -- a tedy
 $y - f(a) > 0 \Leftrightarrow f^{-1}(y) - f^{-1}(f(a)) > 0$. Podle
 \myref{tvrzení}{prop:limita-a/0} platí
 \[
  (f^{-1})'(f(a)) = \lim_{y \to f(a)} \frac{f^{-1}(y) - f^{-1}(f(a))}{y - f(a)}
  = \lim_{y \to f(a)} \frac{1}{\psi(y)} = \infty,
 \]
 což zakončuje důkaz.
\end{thmproof}

\begin{problem}{}{derivace-odmocniny}
 Dokažte, že derivací funkce $x \mapsto \sqrt[n]{x}$ na intervalu $(0,\infty)$
 je $x \mapsto \sqrt[n]{x} / nx$, kde $\N \ni n \geq 1$.
\end{problem}
\begin{probsol}
 Funkce $f(x) = \sqrt[n]{x}$ je jistě spojitá a rostoucí na $(0,\infty)$. Její
 inverzní funkcí je rovněž rostoucí a spojitá $f^{-1}(y) = y^{n}$, jejíž
 derivací je $(f^{-1})'(y) = ny^{n-1}$. Podle
 \hyperref[thm:derivace-inverzni-funkce]{věty o derivaci inverzní funkce} platí
 pro $x \in (0,\infty)$
 \[
  f'(x) = \frac{1}{(f^{-1})'(f(x))} = \frac{1}{n f^{n-1}(x)} = \frac{1}{n
  (\sqrt[n]{x})^{n-1}} = \frac{\sqrt[n]{x}}{nx},
 \]
 jak jsme chtěli.
\end{probsol}

Sekci zakončíme vztahem derivaci k extrémům původní funkce, který hraje stěžejní
roli mimo jiné v optimalizačních problémech, bo často vedou na hledání
minima/maxima jisté funkce.

\begin{proposition}{Vztah derivace a extrému}{vztah-derivace-a-extremu}
 Ať $f:M \to \R$ má v $a \in M$ lokální extrém. Pak $f'(a)$ buď neexistuje, nebo
 je nulová.
\end{proposition}
\begin{propproof}
 Dokážeme kontrapozitivní formu tvrzení. Budeme předpokládat, že $f'(a)$
 existuje a je různá od nuly. Z toho odvodíme, že $f$ nemá v $a$ lokální extrém.

 Ať nejprve $f'(a) > 0$. Pak existuje okolí $R(a,\delta)$, na němž platí
 \[
  \frac{f(x) - f(a)}{x - a} > 0.
 \]
 Odtud plyne, že $f(x) < f(a)$ pro $x \in (a - \delta,a)$ a $f(x) > f(a)$ pro $x
 \in (a,a+\delta)$, čili $f$ nemá v~$a$ lokální extrém. Případ $f'(a) < 0$ se
 ošetří obdobně.
\end{propproof}

\begin{warning}{}{extremy-funkce}
 Radíme čtenářům, by sobě všimli, že
 \hyperref[prop:vztah-derivace-a-extremu]{předchozí tvrzení} je ve formě
 \emph{implikace}. Tedy, \textbf{má-li} funkce $f$ v bodě $a$ \textbf{extrém,
 pak} $f'(a) = 0$, nebo tato neexistuje. Rovnost $f'(a) = 0$ ani neexistence
 derivace v bodě $a$ ještě nezaručují, že $f$ má v bodě $a$ jakýkoli extrém.

 Jako protipříklad stačí jednoduchá funkce $f(x) = x^3$. Zřejmě $f'(x) = 3x^2$,
 která je rovna $0$ pro $x = 0$, ale $x^3$ nemá v $0$ ani lokální extrém.
\end{warning}

\begin{exercise}{}{extremy-funkce}
 Nalezněte lokální i globální extrémy funkce $f(x) = 2 x^{3} - 3 x^{2} - 12 x -
 4$.
\end{exercise}

\section{Věty o střední hodnotě}
\label{sec:vety-o-stredni-hodnote}

Spojitá funkce, jež ve dvou různých bodech nabývá stejných hodnot, musí mít na
nějakém místě mezi těmito body konstantní růst -- přestat růst a začít klesat či
přestat klesat a začít růst. Spojitá funkce musí někde mezi libovolnými dvěma
body mít tečnu rovnoběžnou k úsečce spojující odpovídající body v grafu této
funkce. Stejný závěr platí i pro spojitou křivku v prostoru.

Tato tvrzení souhrnně slují \emph{věty o střední hodnotě}. Přestože je jejich
geometrický význam lákavý, samy o sobě mnoha účelům neslouží. Jejich důležitost
dlí spíše v rozvoji další teorie derivací. Formulujeme a dokážeme si je.

\begin{theorem}{Rolleova věta o střední hodnotě}{rolleova-veta-o-stredni-hodnote}
 Ať $a < b$, $f:[a,b] \to \R$ je spojitá funkce, $f(a) = f(b)$ a $f$ má derivaci
 v každém bodě $(a,b)$. Pak existuje $c \in (a,b)$ takové, že $f'(c) = 0$.
\end{theorem}
\begin{thmproof}
 Podle \myref{věty}{thm:extremy-spojite-funkce} nabývá $f$ na $[a,b]$ svého
 minima $m$ a maxima $M$. Pak zřejmě
 \begin{equation*}
  \label{eq:rolle}
  \tag{$*$}
  m \leq f(a) \leq f(b) \leq M.
 \end{equation*}
 Pokud $m = M$, pak je $f$ konstantní na $[a,b]$, a tudíž $f'(c) = 0$ dokonce
 pro všechna $c \in (a,b)$.

 Předpokládejme, že $m < M$. Pak musí být aspoň jedna z nerovností
 v~\eqref{eq:rolle} ostrá. Bez újmy na obecnosti, ať $f(b) < M$. Ať $c \in
 (a,b)$ je takové, že $f(c) = M$. Dle předpokladu nabývá $f$ v bodě $c$ maxima
 na $[a,b]$, a tedy podle \myref{tvrzení}{prop:vztah-derivace-a-extremu} platí
 $f'(c) = 0$. V případě, kdy $m < f(a)$ postupujeme obdobně.
\end{thmproof}

\begin{theorem}{Lagrangeova o střední hodnotě}{lagrangeova-o-stredni-hodnote}
 Ať $a < b$ a $f:[a,b] \to \R$ je spojitá na $[a,b]$ majíc derivaci v každém
 bodě $(a,b)$. Potom existuje $c$ takové, že
 \[
  f'(c) = \frac{f(b) - f(a)}{b - a}.
 \]
\end{theorem}
\begin{thmproof}
 Převedeme problém na použití
 \hyperref[thm:rolleova-veta-o-stredni-hodnote]{Rolleovy věty}. Definujme funkci
 \[
  g(x) \coloneqq f(x) - \frac{f(b) - f(a)}{b-a} (x-a).
 \]
 Potom je $g$ spojitá na $[a,b]$, má derivaci v každém bodě $(a,b)$ a platí
 $g(a) = g(b)$. Z \hyperref[thm:rolleova-veta-o-stredni-hodnote]{Rolleovy
 věty} plyne, že existuje $c \in (a,b)$ takové, že $g'(c) = 0$. Tato rovnost
 rozepsána znamená, že
 \[
  f'(c) - \frac{f(b) - f(a)}{b - a} = 0,
 \]
 čili
 \[
  f'(c) = \frac{f(b) - f(a)}{b-a},
 \]
 jak jsme chtěli.
\end{thmproof}

\begin{remark}{}{lagrangeova-veta}
 \hyperref[thm:lagrangeova-o-stredni-hodnote]{Lagrangeova věta} říká, že v
 jistém bodě $c \in (a,b)$ musí být derivace $f$ v bodě $c$ rovna směrnici
 přímky spojující body $(a,f(a))$ a $(b,f(b))$.
\end{remark}

\begin{corollary}{Vztah derivace a monotonie}{vztah-derivace-a-monotonie}
 Ať $I \subseteq \R$ je interval a $f:I \to \R$ je spojitá funkce majíc v každém
 bodě $(a,b)$ kladnou, resp. zápornou, derivaci. Pak je $f$ rostoucí, resp.
 klesající, na $I$.
\end{corollary}
\begin{corproof}
 Volme $[a,b] \subseteq I$ libovolně a předpokládejme, že $f'$ je kladná na $I$.
 Podle \hyperref[thm:lagrangeova-o-stredni-hodnote]{Lagrangeovy věty} existuje
 $c \in (a,b)$ takové, že
 \[
  f'(c) = \frac{f(b) - f(a)}{b - a}.
 \]
 Ježto $f'(c) > 0$, plyne odtud, že $f(b) > f(a)$. Čili $f$ je rostoucí na
 $[a,b]$. Jelikož $a < b \in I$ byla volena libovolně, je $f$ rostoucí na $I$.
 Případ $f' < 0$ na $I$ se ošetří analogicky.
\end{corproof}

\begin{exercise}{}{derivace-nula-konstantni}
 Použijte \myref{důsledek}{cor:vztah-derivace-a-monotonie} k důkazu, že spojitá
 funkce $f:I \to \R$ mající nulovou derivaci na $I$, je konstantní.
\end{exercise}

\begin{theorem}{Cauchyho o střední hodnotě}{cauchyho-o-stredni-hodnote}
 Ať $f,g$ jsou spojité funkce na $[a,b] \subseteq \R$, $f$ má v každém bodě
 $(a,b)$ derivaci a $g$ má v každém bodě $(a,b)$ \textbf{konečnou nenulovou}
 derivaci. Potom $g(a) \neq g(b)$ a existuje $c \in (a,b)$ takové, že
 \[
  \frac{f'(c)}{g'(c)} = \frac{f(b) - f(a)}{g(b) - g(a)}.
 \]
\end{theorem}
\begin{thmproof}
 Z \hyperref[thm:lagrangeova-o-stredni-hodnote]{Lagrangeovy věty} plyne
 existence $d \in (a,b)$ takového, že
 \[
  g'(d) = \frac{g(b) - g(a)}{b - a}.
 \]
 Jelikož z předpokladu $g'(d) \neq 0$, rovněž $g(b) \neq g(a)$.

 Opět převedeme problém na
 \hyperref[thm:rolleova-veta-o-stredni-hodnote]{Rolleovu větu}. Definujme funkci
 \[
  \varphi(x) \coloneqq (f(x) - f(a))(g(b) - g(a)) - (g(x) - g(a))(f(b) - f(a)).
 \]
 Pak je $\varphi$ spojitá na $[a,b]$ (neboť $f$ a $g$ jsou tamže spojité) a má v
 každém bodě $(a,b)$ derivaci -- to plyne z
 \hyperref[thm:aritmetika-derivaci]{věty o aritmetice derivací} a faktu, že $f$
 i $g$ mají na $(a,b)$ derivaci.

 Navíc, $\varphi(a) = \varphi(b) = 0$. Z
 \hyperref[thm:rolleova-veta-o-stredni-hodnote]{Rolleovy věty} existuje $c \in
 (a,b)$ splňující $\varphi'(c) = 0$. Platí
 \[
  0 = \varphi'(c) = f'(c)(g(b) - g(a)) - g'(c)(f(b) - f(a)).
 \]
 Odtud úpravou
 \[
  \frac{f'(c)}{g'(c)} = \frac{f(b) - f(a)}{g(b) - g(a)},
 \]
 což zakončuje důkaz.
\end{thmproof}
\begin{remark}{}{cauchyho-veta}
 \hyperref[thm:cauchyho-o-stredni-hodnote]{Cauchyho věta} říká, že křivka v
 rovině $t \mapsto (f(t),g(t))$ má v jistém bodě $c \in (a,b)$ derivaci rovnou
 směrnici přímky spojující body $(f(a),g(a))$ a $(f(b),g(b))$.
\end{remark}
\begin{remark}{}{vety-o-stredni-hodnote}
 Uvědomme si, že \hyperref[thm:lagrangeova-o-stredni-hodnote]{Lagrangeova věta}
 je speciálním případem \hyperref[thm:cauchyho-o-stredni-hodnote]{Cauchyho věty}
 pro $g(x) = x$ a \hyperref[thm:rolleova-veta-o-stredni-hodnote]{Rolleova věta}
 je zase speciálním případem
 \hyperref[thm:lagrangeova-o-stredni-hodnote]{Lagrangeovy věty}, když $f(a) =
 f(b)$. Ovšem, k důkazu jak
 \hyperref[thm:lagrangeova-o-stredni-hodnote]{Lagrangeovy věty}, tak
 \hyperref[thm:cauchyho-o-stredni-hodnote]{Cauchyho věty}, jsme použili téměř
 výhradně \hyperref[thm:rolleova-veta-o-stredni-hodnote]{Rolleovu větu}. Jedná
 se pročež o vzájemně ekvivalentní tvrzení, ač to tak na první pohled nevypadá.
\end{remark}

\section{l'Hospitalovo pravidlo}
\label{sec:l'hospitalovo-pravidlo}

Výpočet limit podílu funkcí patří k nejčastějším úlohám matematické analýzy.
Limita podílu funkcí totiž v principu porovnává, o kolik je jedna funkce na
okolí tohoto bodu větší než druhá. Je pročež základem aproximace funkcí na okolí
bodu mnohem hezčími (například polynomiálními) funkcemi, jak bude vidno v
kapitole o Taylorově polynomu. Následující věta -- všeobecně známa pod jménem
\emph{l'Hospitalovo pravidlo} -- že limita podílu funkcí je (za jistých volných
podmínek) rovna limitě podílu rychlosti jejich růstu.

\begin{theorem}{l'Hospitalovo pravidlo}{lhospitalovo-pravidlo}
 Ať $a \in \R^{*}$, $f,g:M \to \R$ jsou reálné funkce a existuje $\lim_{x \to a}
 f'(x) / g'(x)$. Jestliže platí buď
 \begin{enumerate}[label=(\alph*)]
  \item $\lim_{x \to a} f(x) = \lim_{x \to a} g(x) = 0$, nebo
  \item $\lim_{x \to a} |g(x)| = \infty$,
 \end{enumerate}
 pak
 \[
  \lim_{x \to a} \frac{f(x)}{g(x)} = \lim_{x \to a} \frac{f'(x)}{g'(x)}.
 \]
\end{theorem}


Položme $L \coloneqq \lim_{x \to a} f'(x) / g'(x)$. Budeme postupovat vcelku
přirozeně -- sevřeme podíl $f(x) / g(x)$ pro vhodná $x$ v nekonečně malém
intervalu $(\alpha,\beta)$ okolo $L$. Ačkoli idea je tato přímočará, její
realizace je mírně technická. Pro přehlednost povedeme důkaz přes následující
pomocné lemma.

\begin{lemma*}{pomocné}
 \begin{enumerate}
  \item Pro každé $\alpha > L$ existuje $\delta>0$ takové, že
  \[
   \forall x \in R(a,\delta): \frac{f(x)}{g(x)} < \alpha.
  \]
 \item Pro každé $\beta < L$ existuje $\delta>0$ takové, že
 \[
  \forall x \in R(a,\delta): \frac{f(x)}{g(x)} > \beta.
 \]
 \end{enumerate}
\end{lemma*}
\begin{lemproof}
 Dokážeme část (1), důkaz (2) je analogický.

 Ať je $\alpha > L$ dáno. Volme $r \in (L,\alpha)$. Díky předpokladu existence
 limity $\lim_{x \to a} f'(x) / g'(x)$ existuje okolí $R(a,\delta_1)$ takové,
 že $f$ i $g$ jsou definovány na $R(a,\delta_1)$, $f$ má tamže konečnou
 derivaci a $g$ konečnou nenulovou derivaci. Navíc lze $\delta_1$ volit
 dostatečně malé, aby
 \[
  \frac{f'(x)}{g'(x)} < r \quad \forall x \in R(a,\delta_1).
 \]
 
 Volme libovolná $x < y \in R(a,\delta_1)$. Podle
 \hyperref[thm:cauchyho-o-stredni-hodnote]{Cauchyho věty} existuje $c \in
 (x,y)$ splňující
 \[
  \frac{f'(c)}{g'(c)} = \frac{f(x) - f(y)}{g(x) - g(y)}.
 \]
 Potom tedy pro každý pár $x<y \in R(a,\delta_1)$ platí
 \[
  \frac{f(x) - f(y)}{g(x) - g(y)} < r.
 \]
 
 Předpokládejme nyní, že platí podmínka (a) ve znění
 \hyperref[thm:lhospitalovo-pravidlo]{věty}, tj. předpokládejme rovnosti
 \[
  \lim_{x \to a} f(x) = \lim_{x \to a} g(x) = 0.
 \]
 Pak pro fixní $y \in (a,a+\delta_1)$ dostaneme
 \[
  \lim_{x \to a} \frac{f(x) - f(y)}{g(x) - g(y)} = \frac{f(y)}{g(y)} \leq r <
  \alpha
 \]
 a pro $\tilde{y} \in (a-\delta_1,a)$ zase
 \[
  \lim_{x \to a} \frac{f(\tilde{y}) - f(x)}{g(\tilde{y}) - g(x)} =
  \frac{f(\tilde{y})}{g(\tilde{y})} \leq r <
  \alpha.
 \]
 Celkově tedy nerovnost $f(y) / g(y) < \alpha$ platí pro každé $y \in
 R(a,\delta_1)$.

 Konečně, ať platí podmínka (b), tj. $\lim_{x \to a} |g(x)| = \infty$. Volme
 pevné $\tilde{y} \in (a,a+\delta_1)$. Pak
 \begin{align*}
  \lim_{x \to a} r \left( 1 - \frac{g(\tilde{y})}{g(x)} \right) +
  \frac{f(\tilde{y})}{g(x)} = r \cdot (1 - 0) + 0 = r < \alpha.
 \end{align*}
 Existuje tudíž $\delta_2 \in (0,\delta_1)$ takové, že pro každé $x \in
 (a,a+\delta_2)$
 \[
  r \left( 1 - \frac{g(\tilde{y})}{g(x)} \right) + \frac{f(\tilde{y})}{g(x)} <
  \alpha.
 \]
 Navíc, díky podmínce (b) lze volit $\delta_2$ dostatečně malé, aby
 \[
  \frac{g(\tilde{y})}{g(x)} < 1 \quad \forall x \in (a,a+\delta_2).
 \]
 Pro $x \in (a,a+\delta_2)$ počítejme
 \begin{align*}
  \frac{f(x)}{g(x)} &= \frac{f(x) - f(\tilde{y})}{g(x)} +
  \frac{f(\tilde{y})}{g(x)}\\
                    &= \frac{f(x) - f(\tilde{y})}{g(x) - g(\tilde{y})} \cdot
                    \frac{g(x) - g(\tilde{y)}}{g(x)} +
                    \frac{f(\tilde{y})}{g(x)}\\
                    &= \frac{f(x) - f(\tilde{y})}{g(x) - g(\tilde{y})} \left( 1
                    - \frac{g(\tilde{y})}{g(x)}\right) +
                    \frac{f(\tilde{y})}{g(x)}.
 \end{align*}
 Můžeme odhadnout
 \[
  \frac{f(x)}{g(x)} = \frac{f(x) - f(\tilde{y})}{g(x) - g(\tilde{y})} \left( 1 -
  \frac{g(\tilde{y})}{g(x)}\right) + \frac{f(\tilde{y})}{g(x)} < r \left( 1 -
 \frac{g(\tilde{y})}{g(x)} \right) + \frac{f(\tilde{y})}{g(x)} < \alpha.
 \]
 Podobně dokážeme i rovnost $f(x) / g(x) < \alpha$ v případě, kdy $x$ je z
 vhodného levého prstencového okolí $a$.

Jelikož důkaz části (b) je zcela symetrický, je pomocné lemma dokázáno. 
\end{lemproof}

\begin{thmproof}[\myref{věty}{thm:lhospitalovo-pravidlo}]
 Je-li $L = -\infty$, resp. $L = \infty$, pak tvrzení plyne ihned z části (1),
 resp. části (2), \emph{pomocného lemmatu}.

 Ať $L \in \R$. Volme $\varepsilon>0$ a pro $\alpha \coloneqq L+\varepsilon$
 nalezněme z \emph{pomocného lemmatu}, části (1), $\delta>0$ takové, že pro $x
 \in R(a,\delta)$ platí $f(x) / g(x) < \alpha$. Podobně nalezněme, z
 \emph{pomocného lemmatu} části (2) pro $\beta \coloneqq L-\varepsilon$, číslo
 $\delta'>0$ takové, že pro $x \in R(a,\delta')$ platí $f(x) / g(x) > \beta$.
 Pak ale pro $x \in R(a,\min(\delta,\delta'))$ máme
 \[
  \frac{f(x)}{g(x)} \in (\beta,\alpha) = (L-\varepsilon,L+\varepsilon),
 \]
 čili $\lim_{x \to a} f(x) / g(x) = L$.
\end{thmproof}

Použití l'Hospitalova pravidla pro výpočet limit ponecháme do kapitoly o
elementárních funkcích.

