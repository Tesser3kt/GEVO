\section{l'Hospitalovo pravidlo}
\label{sec:l'hospitalovo-pravidlo}

Výpočet limit podílu funkcí patří k nejčastějším úlohám matematické analýzy.
Limita podílu funkcí totiž v principu porovnává, o kolik je jedna funkce na
okolí tohoto bodu větší než druhá. Je pročež základem aproximace funkcí na okolí
bodu mnohem hezčími (například polynomiálními) funkcemi, jak bude vidno v
kapitole o Taylorově polynomu. Následující věta -- všeobecně známa pod jménem
\emph{l'Hospitalovo pravidlo} -- že limita podílu funkcí je (za jistých volných
podmínek) rovna limitě podílu rychlosti jejich růstu.

\begin{theorem}{l'Hospitalovo pravidlo}{lhospitalovo-pravidlo}
 Ať $a \in \R^{*}$, $f,g:M \to \R$ jsou reálné funkce a existuje $\lim_{x \to a}
 f'(x) / g'(x)$. Jestliže platí buď
 \begin{enumerate}[label=(\alph*)]
  \item $\lim_{x \to a} f(x) = \lim_{x \to a} g(x) = 0$, nebo
  \item $\lim_{x \to a} |g(x)| = \infty$,
 \end{enumerate}
 pak
 \[
  \lim_{x \to a} \frac{f(x)}{g(x)} = \lim_{x \to a} \frac{f'(x)}{g'(x)}.
 \]
\end{theorem}


Položme $L \coloneqq \lim_{x \to a} f'(x) / g'(x)$. Budeme postupovat vcelku
přirozeně -- sevřeme podíl $f(x) / g(x)$ pro vhodná $x$ v nekonečně malém
intervalu $(\alpha,\beta)$ okolo $L$. Ačkoli idea je tato přímočará, její
realizace je mírně technická. Pro přehlednost povedeme důkaz přes následující
pomocné lemma.

\begin{lemma*}{pomocné}
 \begin{enumerate}
  \item Pro každé $\alpha > L$ existuje $\delta>0$ takové, že
  \[
   \forall x \in R(a,\delta): \frac{f(x)}{g(x)} < \alpha.
  \]
 \item Pro každé $\beta < L$ existuje $\delta>0$ takové, že
 \[
  \forall x \in R(a,\delta): \frac{f(x)}{g(x)} > \beta.
 \]
 \end{enumerate}
\end{lemma*}
\begin{lemproof}
 Dokážeme část (1), důkaz (2) je analogický.

 Ať je $\alpha > L$ dáno. Volme $r \in (L,\alpha)$. Díky předpokladu existence
 limity $\lim_{x \to a} f'(x) / g'(x)$ existuje okolí $R(a,\delta_1)$ takové,
 že $f$ i $g$ jsou definovány na $R(a,\delta_1)$, $f$ má tamže konečnou
 derivaci a $g$ konečnou nenulovou derivaci. Navíc lze $\delta_1$ volit
 dostatečně malé, aby
 \[
  \frac{f'(x)}{g'(x)} < r \quad \forall x \in R(a,\delta_1).
 \]
 
 Volme libovolná $x < y \in R(a,\delta_1)$. Podle
 \hyperref[thm:cauchyho-o-stredni-hodnote]{Cauchyho věty} existuje $c \in
 (x,y)$ splňující
 \[
  \frac{f'(c)}{g'(c)} = \frac{f(x) - f(y)}{g(x) - g(y)}.
 \]
 Potom tedy pro každý pár $x<y \in R(a,\delta_1)$ platí
 \[
  \frac{f(x) - f(y)}{g(x) - g(y)} < r.
 \]
 
 Předpokládejme nyní, že platí podmínka (a) ve znění
 \hyperref[thm:lhospitalovo-pravidlo]{věty}, tj. předpokládejme rovnosti
 \[
  \lim_{x \to a} f(x) = \lim_{x \to a} g(x) = 0.
 \]
 Pak pro fixní $y \in (a,a+\delta_1)$ dostaneme
 \[
  \lim_{x \to a} \frac{f(x) - f(y)}{g(x) - g(y)} = \frac{f(y)}{g(y)} \leq r <
  \alpha
 \]
 a pro $\tilde{y} \in (a-\delta_1,a)$ zase
 \[
  \lim_{x \to a} \frac{f(\tilde{y}) - f(x)}{g(\tilde{y}) - g(x)} =
  \frac{f(\tilde{y})}{g(\tilde{y})} \leq r <
  \alpha.
 \]
 Celkově tedy nerovnost $f(y) / g(y) < \alpha$ platí pro každé $y \in
 R(a,\delta_1)$.

 Konečně, ať platí podmínka (b), tj. $\lim_{x \to a} |g(x)| = \infty$. Volme
 pevné $\tilde{y} \in (a,a+\delta_1)$. Pak
 \begin{align*}
  \lim_{x \to a} r \left( 1 - \frac{g(\tilde{y})}{g(x)} \right) +
  \frac{f(\tilde{y})}{g(x)} = r \cdot (1 - 0) + 0 = r < \alpha.
 \end{align*}
 Existuje tudíž $\delta_2 \in (0,\delta_1)$ takové, že pro každé $x \in
 (a,a+\delta_2)$
 \[
  r \left( 1 - \frac{g(\tilde{y})}{g(x)} \right) + \frac{f(\tilde{y})}{g(x)} <
  \alpha.
 \]
 Navíc, díky podmínce (b) lze volit $\delta_2$ dostatečně malé, aby
 \[
  \frac{g(\tilde{y})}{g(x)} < 1 \quad \forall x \in (a,a+\delta_2).
 \]
 Pro $x \in (a,a+\delta_2)$ počítejme
 \begin{align*}
  \frac{f(x)}{g(x)} &= \frac{f(x) - f(\tilde{y})}{g(x)} +
  \frac{f(\tilde{y})}{g(x)}\\
                    &= \frac{f(x) - f(\tilde{y})}{g(x) - g(\tilde{y})} \cdot
                    \frac{g(x) - g(\tilde{y)}}{g(x)} +
                    \frac{f(\tilde{y})}{g(x)}\\
                    &= \frac{f(x) - f(\tilde{y})}{g(x) - g(\tilde{y})} \left( 1
                    - \frac{g(\tilde{y})}{g(x)}\right) +
                    \frac{f(\tilde{y})}{g(x)}.
 \end{align*}
 Můžeme odhadnout
 \[
  \frac{f(x)}{g(x)} = \frac{f(x) - f(\tilde{y})}{g(x) - g(\tilde{y})} \left( 1 -
  \frac{g(\tilde{y})}{g(x)}\right) + \frac{f(\tilde{y})}{g(x)} < r \left( 1 -
 \frac{g(\tilde{y})}{g(x)} \right) + \frac{f(\tilde{y})}{g(x)} < \alpha.
 \]
 Podobně dokážeme i rovnost $f(x) / g(x) < \alpha$ v případě, kdy $x$ je z
 vhodného levého prstencového okolí $a$.

Jelikož důkaz části (b) je zcela symetrický, je pomocné lemma dokázáno. 
\end{lemproof}

\begin{thmproof}[\myref{věty}{thm:lhospitalovo-pravidlo}]
 Je-li $L = -\infty$, resp. $L = \infty$, pak tvrzení plyne ihned z části (1),
 resp. části (2), \emph{pomocného lemmatu}.

 Ať $L \in \R$. Volme $\varepsilon>0$ a pro $\alpha \coloneqq L+\varepsilon$
 nalezněme z \emph{pomocného lemmatu}, části (1), $\delta>0$ takové, že pro $x
 \in R(a,\delta)$ platí $f(x) / g(x) < \alpha$. Podobně nalezněme, z
 \emph{pomocného lemmatu} části (2) pro $\beta \coloneqq L-\varepsilon$, číslo
 $\delta'>0$ takové, že pro $x \in R(a,\delta')$ platí $f(x) / g(x) > \beta$.
 Pak ale pro $x \in R(a,\min(\delta,\delta'))$ máme
 \[
  \frac{f(x)}{g(x)} \in (\beta,\alpha) = (L-\varepsilon,L+\varepsilon),
 \]
 čili $\lim_{x \to a} f(x) / g(x) = L$.
\end{thmproof}

Použití l'Hospitalova pravidla pro výpočet limit ponecháme do kapitoly o
elementárních funkcích.
