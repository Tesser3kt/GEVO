\section{Limity elementárních funkcí}
\label{sec:limity-elementarnich-funkci}

Tato sekce je věnována výpočtu limit, ve kterých figurují elementární funkce.
Nejtěžšími (ale zároveň nejužitečnějšími) úlohami na vyřešení jsou limity
racionálních kombinací (tj. součtů, násobků a především podílů) elementárních
funkcí. Samotné řešení pak obvykle zahrnuje převod výrazu do tvaru, v němž lze
naň uplatnit jisté \uv{známé} limity, případně použít
\hyperref[thm:lhospital]{l'Hospitalova pravidla}.

\begin{proposition}{Běžné limity}{bezne-limity}
 Platí
 \begingroup
 \addtolength{\jot}{3mm}
 \begin{align}
  \lim_{x \to 0} \frac{\sin x}{x} &= 1,\tag{a}\\
  \lim_{x \to 0} \frac{1-\cos x}{x^2} &= \frac{1}{2},\tag{b}\\
  \lim_{x \to 0} \frac{\exp x - 1}{x} &= 1,\tag{c}\\
  \lim_{x \to 0} \frac{\log(1+x)}{x} &= 1.\tag{d}
 \end{align}
 \endgroup
\end{proposition}
\begin{propproof}
 K výpočtu všech limit lze použít \hyperref[thm:lhospital]{l'Hospitalova
 pravidla}. Máme
 \begingroup
 \addtolength{\jot}{3mm}
 \begin{align}
  \lim_{x \to 0} \frac{\sin x}{x} &= \lim_{x \to 0} \frac{\cos x}{1} = \cos(0) =
  1,\tag{a}\\
  \lim_{x \to 0} \frac{1-\cos x}{x^2} &= \lim_{x \to 0} \frac{\sin x}{2x} =
  \frac{1}{2} \lim_{x \to 0} \frac{\sin x}{x} \overset{\text{(a)}}{=}
  \frac{1}{2},\tag{b}\\
  \lim_{x \to 0} \frac{\exp x - 1}{x} &= \lim_{x \to 0} \frac{\exp x}{1} =
  \exp(0) = 1,\tag{c}\\
  \lim_{x \to 0} \frac{\log(1+x)}{x} &= \lim_{x \to 0} \frac{\frac{1}{1+x}}{1} =
  1.\tag{d}
 \end{align}
 \endgroup
 Tím je důkaz hotov.
\end{propproof}

Aplikujeme nyní \myref{tvrzení}{prop:bezne-limity} k výpočtu některých limit
kombinací elementárních funkcí.

\begin{problem}{}{elementarni-limity-1}
 Spočtěte
 \[
  \lim_{x \to 0} \frac{\cos(\sin x) - 1}{\log(\sqrt{1 + x^2})}.
 \]
\end{problem}
\begin{probsol}
 Zřejmě platí
 \[
  \lim_{x \to 0} \cos(\sin x) - 1 = \lim_{x \to 0} \log(\sqrt{1 + x^2}) = 0,
 \]
 je tedy třeba výraz nejprve upravit. \hyperref[thm:lhospital]{l'Hospitalovo
 pravidlo} zde pravděpodobně není vhodným prostředkem, neboť derivace obou
 funkcí jsou značně komplikované. Učiníme nejprve úpravu
 \[
  \frac{\cos(\sin x) - 1}{\log(\sqrt{1 + x^2})} = \frac{\cos(\sin x) -
  1}{\sin^2 x} \cdot \frac{\sin^2 x}{\log(\sqrt{1 + x^2})}.
 \]
 Protože $\lim_{x \to 0} \sin x / x = 1$, platí z
 \hyperref[thm:limita-slozene-funkce]{věty o limitě složené funkce}
 \[
  \lim_{x \to 0} \frac{\cos(\sin x) - 1}{\sin^2 x} = \lim_{y \to 0} \frac{\cos y
  - 1}{y^2} = -\frac{1}{2}.
 \]
 Dále máme (z \hyperref[prop:vlastnosti-logaritmu]{vlastností logaritmu})
 \[
  \frac{\sin^2 x}{\log(\sqrt{1 + x^2})} = \frac{\sin^2 x}{x^2} \cdot
  \frac{x^2}{\frac{1}{2}\log(1 + x^2)} = 2 \cdot \frac{\sin x}{x} \cdot
  \frac{\sin x}{x} \cdot \frac{x^2}{\log(1 + x^2)}.
 \]
 Opět z \hyperref[thm:limita-slozene-funkce]{věty o limitě složené funkce} jest
 \[
  \lim_{x \to 0} \frac{x^2}{\log(1 + x^2)} = \lim_{y \to 0} \frac{y}{\log(1 +
  y)} = 1.
 \]
 Celkem tedy,
 \[
  \lim_{x \to 0} \frac{\sin^2 x}{\log(\sqrt{1 + x^2})} = 2 \cdot \lim_{x \to 0}
  \frac{\sin x}{x} \cdot \lim_{x \to 0} \frac{\sin x}{x} \cdot \lim_{x \to 0}
  \frac{x^2}{\log(1 + x^2)} = 2 \cdot 1 \cdot 1 \cdot 1 = 2.
 \]
 Spolu s předchozím výpočtem dostaneme
 \[
  \lim_{x \to 0} \frac{\cos(\sin x) - 1}{\log(\sqrt{1 + x^2})} = -\frac{1}{2}
  \cdot 2 = -1.
 \]
\end{probsol}

\begin{problem}{}{elementarni-limity-2}
 Spočtěte
 \[
  \lim_{x \to \infty} \sqrt{\log \left( 1 + \frac{3}{x} \right)} \cdot
  \log^2(1+x^3).
 \]
\end{problem}
