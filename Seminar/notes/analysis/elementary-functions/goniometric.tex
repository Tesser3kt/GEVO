\section{Goniometrické funkce}
\label{sec:goniometricke-funkce}

Název \uv{úhloměrné} funkce je zastaralý a nepřesný. Funkce $\sin$ a $\cos$,
které se jmeme definovati, úspěšně modelují fyzikální jevy jakkoli související s
vibrací či vlněním. Jak si brzy rozmyslíme, jsou to ve skutečnosti funkce v
zásadě exponenciální. To by nemělo být na druhý pohled až tak překvapivé --
vibrace jsou v zásadě jen periodicky se střídající růst a pokles.

\begin{definition}{Goniometrické funkce}{goniometricke-funkce}
 Pro $x \in \R$ definujeme funkce
 \begin{align*}
  \sin x & \coloneqq \sum_{n=0}^{\infty} (-1)^{n} \frac{x^{2n+1}}{(2n+1)!},\\
  \cos x & \coloneqq \sum_{n=0}^{\infty} (-1)^{n} \frac{x^{2n}}{(2n)!}.
 \end{align*}
\end{definition}

\begin{theorem}{Vlastnosti goniometrických
funkcí}{vlastnosti-goniometrickych-funkci}
 Funkce $\sin$ a $\cos$ jsou dobře definované a splňují:
 \begin{enumerate}[label=(G\arabic*)]
  \item $ \forall x,y \in \R:$
  \begin{align*}
   \sin(x+y) &= \sin x \cos y + \sin y \cos x,\\
   \cos(x+y) &= \cos x \cos y - \sin x \sin y;
  \end{align*}
 \item $\sin$ je lichá a $\cos$ je sudá funkce;
 \item $ \exists \pi \in \R$ takové, že $\sin$ je rostoucí na $[0,\pi / 2]$,
  $\sin(0) = 0$ a $\sin(\pi / 2) = 1$.
 \item $\sin'(0) = 1$.
 \end{enumerate}
\end{theorem}

K důkazu použijeme následující pomocné lemma.

\begin{lemma*}{Pomocné}
 Ať $x \in \R$. Pak existuje $C > 0$ takové, že pro každé $n \in \N$ a každé
 $h \in (-1,1)$ platí nerovnost
 \[
  |(x+h)^{n} - x^{n} - nhx^{n-1}| \leq h^2C^{n}.
 \]
\end{lemma*}
\begin{lemproof}
 Položme $C \coloneqq 2(|x| + 1)$. Pro $n = 1$ máme
 \[
  |(x+h)^{1} - x^{1} - hx^{0}| = |x + h - x - h| = 0 \leq 2h^2 (|x| + 1)
 \]
 pro každé $h \in (-1,1)$.
 
 Pro $n \geq 2$ lze použít binomickou větu a počítat
 \begin{equation*}
  \label{eq:sin-derivative}
  \tag{$\triangle$}
  (x+h)^{n} - x^{n} - nhx^{n-1} = \sum_{k=0}^{n} \binom{n}{k} x^{n-k} h^{k}
  \clr{-~x^{n} - nhx^{n-1}} = \sum_{\clr{k=2}}^{n} \binom{n}{k} x^{n-k}h^{k}.
 \end{equation*}
 Protože $|x| + 1 \geq |x|$ pro každé $x \in \R$, platí $(|x| + 1)^{n} \geq
 |x|^{k}$, kdykoli $k \leq n$. Rovněž, z~předpokladu $|h|<1$, a tedy naopak
 platí $|h|^{k} \leq |h|^{n}$ pro $k \leq n$. Užitím \clb{těchto nerovností} a
 rovnosti~\eqref{eq:sin-derivative} můžeme odhadnout
 \begin{align*}
  |(x+h)^{n} - x^{n} - nhx^{n-1}| & \leq \sum_{k=2}^{n}
  \binom{n}{k}|x|^{n-k}|h|^{k} \clb{~\leq} \sum_{k=2}^{n} \binom{n}{k} (|x| +
  1)^{n}h^2\\
                                  & \leq h^2(|x| + 1)^{n} \sum_{k=0}^{n}
                                  \binom{n}{k} = h^2(|x| + 1)^{n} 2^{n} =
                                  h^2C^{n},
 \end{align*}
 čímž je důkaz hotov.
\end{lemproof}

\begin{thmproof}[\myref{Věty}{thm:vlastnosti-goniometrickych-funkci}]
 Je zřejmé, že řady
 \[
  \sum_{n=0}^{\infty} (-1)^{n} \frac{x^{2n+1}}{(2n+1)!} \quad \text{a} \quad
  \sum_{n=0}^{\infty} (-1)^{n} \frac{x^{2n}}{(2n)!}
 \]
 konvergují absolutně (použitím stejného argumentu jako v důkazu korektnosti
 exponenciály ve \myref{větě}{thm:vlastnosti-exponencialy}). Podle
 \myref{lemmatu}{lem:absolutni-konvergence-a-konvergence} jsou obě řady rovněž
 konvergentní pro každé $x \in \R$, což dokazuje dobrou definovanost obou
 funkcí.

 Ukážeme nejprve, že $\sin'x = \cos x$. Volme pevné $x \in \R$. Pro $h \in
 (-1,1)$ platí
 \begin{align*}
  \sin(x+h) - \sin x - h \cos x &= \sum_{n=0}^{\infty} (-1)^{n}\left(
  \frac{(x+h)^{2n+1} - x^{2n+1}}{(2n+1)!} - \frac{hx^{2n}}{(2n)!}\right)\\
  &= \sum_{n=0}^{\infty} \frac{(-1)^{n}}{(2n+1)!} ((x+h)^{2n+1} - x^{2n+1} -
  h(2n+1)x^{2n}).
 \end{align*}
 Z \textbf{pomocného lemmatu} nalezneme $C > 0$ takové, že
 \[
  |(x+h)^{2n+1} - x^{2n+1} - h(2n+1)x^{2n}| \leq C^{2n+1}h^2.
 \]
 Pak
 \[
  |\sin(x+h) - \sin x - h\cos x| \leq \sum_{n=0}^{\infty}
  \frac{C^{2n+1}}{(2n+1)!}h^2 = h^2 \sum_{n=0}^{\infty}
  \frac{C^{2n+1}}{(2n+1)!}.
 \]
 Řada $\sum_{n=0}^{\infty} C^{2n+1} / (2n+1)!$ je konvergentní, a tedy
 \[
  \lim_{h \to 0} \left|\frac{\sin(x+h) - \sin x - h \cos x}{h}\right| = 0,
 \]
 z čehož ihned
 \[
  \sin'x = \lim_{h \to 0} \frac{\sin(x+h)-\sin x}{h} = \cos x.
 \]

 Pro důkaz (G1) volme pevné $a \in \R$ a položme
 \[
  \psi(x) \coloneqq (\sin(x + a) - \sin x \cos a - \sin a \cos x)^2
  + (\cos(x + a) - \cos x \cos a + \sin a \sin x)^2.
 \]
 Snadno spočteme, že $\psi'(x) = 0$ pro každé $x \in \R$, a tedy je díky
 \myref{cvičení}{exer:derivace-nula-konstantni} $\psi$ konstantní na $\R$.
 Dosazením dostaneme, že
 \begin{align*}
  \sin(0) &= \sum_{n=0}^{\infty} (-1)^{n} \frac{0^{2n+1}}{(2n+1)!} = 0,\\
  \cos(0) &= \sum_{n=0}^{\infty} (-1)^{n} \frac{0^{2n}}{(2n)!} =
  (-1)^{0}\frac{0^{0}}{0!} + \sum_{n=1}^{\infty} (-1)^{n}\frac{0^{2n}}{(2n!)} =
  1.
 \end{align*}
 Díky těmto rovnostem spočteme $\psi(0) = 0$. Z toho, že $\psi$ je konstantní,
 plyne, že $\psi(x) = 0$ pro každé $x \in \R$. To dokazuje obě rovnosti v (G1),
 neboť $\psi$ je nulová funkce, jež je zároveň součtem čtverců, které musejí být
 tudíž oba nulové.

 Vlastnost (G2) je vidět ihned z definice, neboť proměnná $x$ se v definici
 $\sin$ objevuje pouze v~liché mocnině a v definici $\cos$ pouze v sudé.

 Vlastnost (G3) dokazovat nebudeme. Je výpočetně náročná a neintuitivní.

 Již víme, že $\sin'(x) = \cos x$ a že $\cos(0) = 1$. Odtud (G4).
\end{thmproof}

\begin{remark}{}{sin-cos-jako-exp}
 V úvodu jsme zmínili, že $\sin$ a $\cos$ jsou vlastně exponenciální funkce.
 Vskutku, když se jeden zadívá na jejich řady, vidí (až na znaménko $(-1)^{n}$
 zařizující právě onen \uv{růst a pokles}) v~zásadě exponenciální funkci.
 Konkrétně, $\sin$ je rozdílem \emph{lichých} částí exponenciály a $\cos$ těch
 \emph{sudých}. Rozdělme $\exp x$ na čtyři části podle zbytku po dělení indexu
 $n$ čtyřmi.
 \[
  \exp x = \clr{\sum_{n \bmod 4 = 0} \frac{x^{n}}{n!}}~+ \clb{\sum_{n \bmod 4 =
  1} \frac{x^{n}}{n!}}~+\clg{\sum_{n \bmod 4 = 2} \frac{x^{n}}{n!}}~+\clm{\sum_{n \bmod 4 =
  3} \frac{x^{n}}{n!}}
 \]
 Označme tyto části $\clr{\exp_0}, \clb{\exp_1}, \clg{\exp_2}$ a $\clm{\exp_3}$.
 Všimněme si, že když $n \bmod 4 = a$, pak $n = 4k + a$ pro nějaké $k \in \N$.
 Čili například $\clg{\exp_2}$ lze zapsat ve tvaru
 \[
  \clg{\exp_2}(x) = \sum_{k=0}^{\infty} \frac{x^{4k + 2}}{(4k + 2)!}
 \]
 Tvrdíme, že $\sin = \clb{\exp_1}~-~\clm{\exp_3}$ a $\cos = \clr{\exp_0}~-~
 \clg{\exp_2}$. Vskutku, když je $n$ liché, pak $2n+1 \bmod 4 = 3$ (protože $4
 \nmid 2n$), a když je $n$ sudé, tak $2n + 1 \bmod 4 = 1$. Čili, pro $n$ lichá
 je $2n + 1$ tvaru $4k+3$ a pro $n$ sudá zase tvaru $4k+1$. Můžeme tedy psát
 \begin{align*}
  \clb{\exp_1}(x) - \clm{\exp_3}(x) &= \sum_{k=0}^{\infty}
  \frac{x^{4k+1}}{(4k+1)!} - \sum_{k=0}^{\infty} \frac{x^{4k+3}}{(4k+3)!}\\
                                    &=\sum_{n \text{ sudé}}
                                    \frac{x^{2n+1}}{(2n+1)!} - \sum_{n \text{
                                    liché}} \frac{x^{2n+1}}{(2n+1)!} =
                                    \sum_{n=0}^{\infty}
                                    (-1)^{n}\frac{x^{2n+1}}{(2n+1)!} = \sin x,
 \end{align*}
 neboť $(-1)^{n}$ je rovno $1$ pro $n$ sudé a $-1$ pro $n$ liché. Podobně
 odvodíme i vztah pro $\cos$.
\end{remark}

\begin{definition}{Tangens a kotangens}{tangens-a-kotangens}
 Definujeme goniometrické funkce $\tan$ a $\cot$ předpisy
 \[
  \tan x = \frac{\sin x}{\cos x}, \quad \cot x = \frac{\cos x}{\sin x}.
 \]
 Funkce $\tan$ je definována pro $x \neq n\pi + \pi / 2$, kde je funkce
 $\cos$ nulová, a $\cot$ je definována pro $x$ různé od násobků $\pi$.
\end{definition}

Zformulujeme si několik vlastností funkcí $\tan$ a $\cot$, ale dokazovat je
nebudeme. Důkazy se významně neliší od již spatřených důkazů vlastností jiných
elementárních funkcí.

\begin{proposition}{Vlastnosti tangenty a kotangenty}{vlastnosti-tangenty-a-kotangenty}
 Platí:
 \begin{enumerate}[label=(G\arabic*)]
  \setcounter{enumi}{4}
  \item $\tan$ i $\cot$ jsou spojité na svých doménách;
  \item $\tan$ i $\cot$ jsou liché;
  \item $\tan'x = 1 / \cos^2 x$ a $\cot'x = - 1 / \sin^2 x$;
  \item $\tan \frac{\pi}{4} = \cot \frac{\pi}{4} = 1$;
  \item $\lim_{x \to \frac{\pi}{2}^{-}} \tan x = \infty$ a $\lim_{x \to
   \frac{\pi}{2}^{+}} \tan x = -\infty$.
  \item $\lim_{x \to 0^{+}} \cot x = \infty$ a $\lim_{x \to \pi^{-}} \cot x =
   -\infty$.
  \item $\tan$ je rostoucí na $(-\pi / 2, \pi / 2)$ a $\cot$ je klesající na
   $(0,\pi)$.
 \end{enumerate}
\end{proposition}
