\chapter{Limity funkcí}
\label{chap:limity-funkci}

Limita funkce je dost možná nejdůležitější ideou matematické analýzy a obecně
matematických disciplín, jež využívá fyzika. Davši vzniknout teorii derivací a
primitivních funkcí, umožnila popsat fyzikální jevy soustavami diferenciálních
rovnic a je základem zatím nejlepších známých modelův světa --
diferencovatelných struktur.

Principiálně se pojem \emph{limity funkce} neliší pramnoho od limity
posloupnosti. Matematici funkcí obyčejně myslíme zobrazení popisující vývoj
systému v čase (tzv. funkce \emph{jedné proměnné}), případně závislé na více
parametrech než jen na čase (tzv. funkce \emph{více proměnných}). Limita funkce
v nějakém určeném okamžiku pak znamená vlastně \uv{očekávanou hodnotu} této
funkce v tomto okamžiku -- hodnotu, ke které je funkce, čím méně času zbývá do
onoho okamžiku, tím blíže.

V tomto textu budeme sebe zaobírati pouze funkcemi závislými na čase tvořícími
systémy, jejichž stav je rovněž vyjádřen jediným číslem. Uvidíme, že i teorie
takto primitivních objektů je veskrze širá.

\begin{definition}{Reálná funkce jedné proměnné}{realna-funkce-jedne-promenne}
 Ať $M \subseteq \R$ je libovolná podmnožina $\R$. Zobrazení $f:M \to \R$
 nazýváme \emph{reálnou funkcí (jedné proměnné)}.
\end{definition}

Ačkolivěk ve světě, jest-li nám známo, proudí čas pouze jedním směrem,
matematiku takovými trivialitami netřeba třísnit. Pojem limity reálné funkce
budeme tedy definovat bez ohledu na \uv{proud času}. Budeme zkoumat jak hodnotu
reálné funkce, když se čas blíží \emph{zleva} (tj. přirozeně) k~danému okamžiku,
tak její očekávanou hodnotu proti toku času.

Ona dva přístupa slujeta limita funkce \emph{zleva} a limita funkce
\emph{zprava}. Před jejich výrokem ovšem učiníme kvapný formální obchvat. Bylo
by totiž nanejvýš neelegantní musiti různými logickými výroky definovat konečné
oproti nekonečným limitám v konečných oproti nekonečným bodům. Následující --
čistě formální avšak se silnou geometrickou intuicí -- pojem tyto případy
skuje v~jeden.

\begin{definition}{Okolí a prstencové okolí bodu}{okoli-a-prstencove-okoli-bodu}
Ať $a \in \R^*$ a $\varepsilon \in (0,\infty)$. \emph{Okolím} bodu $a$ (o
poloměru $\varepsilon$) myslíme množinu
\[
 B(a,\varepsilon) \coloneqq \begin{cases}
  (a-\varepsilon,a+\varepsilon), & \text{pokud } a \in \R,\\
  (1 / \varepsilon, \infty), &\text{pokud } a = \infty,\\
  (-\infty,-1 / \varepsilon), &\text{pokud } a = -\infty.
 \end{cases}
\]
Podobně, \emph{prstencovým okolím} $a$ (o velikosti $\varepsilon$) myslíme jeho
okolí bez samotného bodu $a$. Konkrétně,
\[
 R(a,\varepsilon) \coloneqq \begin{cases}
  (a-\varepsilon,a+\varepsilon) \setminus \{a\}, &\text{pokud }a \in \R,\\
  (1 / \varepsilon,\infty), & \text{pokud } a = \infty,\\
  (-\infty,-1 / \varepsilon), &\text{pokud } a = -\infty.
 \end{cases}
\]
Pro účely definice levých a pravých limit, pojmenujeme rovněž množinu
\[
 B_+(a,\varepsilon) \coloneqq \begin{cases}
  [a,a+\varepsilon), &\text{pokud }a \in \R,\\
  \emptyset, &\text{pokud }a = \infty,\\
  (-\infty,-1 / \varepsilon), &\text{pokud }a = -\infty
 \end{cases}
\]
\emph{pravým okolím} bodu $a$ a množinu
\[
 R_+(a,\varepsilon) \coloneqq \begin{cases}
  (a,a+\varepsilon), &\text{pokud }a \in \R,\\
  \emptyset, &\text{pokud }a = \infty,\\
  (-\infty,-1 / \varepsilon), &\text{pokud }a = -\infty
 \end{cases}
\]
\emph{pravým prstencovým okolím} bodu $a$. \emph{Levé okolí} a \emph{levé
prstencové okolí} bodu $a$ se definují analogicky.
\end{definition}

\begin{remark}{}{okoli-a-prstencove-okoli-bodu}
 Písmena \emph{B} a \emph{R} v definici okolí a prstencového okolí pocházejí z
 angl. slov \textbf{b}all a \textbf{r}ing. Okolí se v~angličtině přezdívá
 \emph{ball} pro to, že okolí bodu $a$ je ve skutečnosti (jednodimenzionální)
 kruh s poloměrem $\varepsilon$ o středu $a$. Znázornění okolí bodu jako kruhu v
 rovině je vysoce účinným vizualizačním aparátem. Naopak, slovo \emph{ring}
 vskutku přirozeně značí kruh s \uv{dírou} (velikosti jednoho bodu) v jeho
 středě. 

 Čtenáře možná zarazilo číslo $1 / \varepsilon$ v definici okolí bodu $\infty$.
 Důvod užití $1 / \varepsilon$ oproti prostému $\varepsilon$ je spíše
 intuitivního rázu. V definici limity a v následných tvrzeních si matematici
 obvykle představujeme pod $\varepsilon$ reálné číslo, které je \uv{nekonečně
 malé}. Chceme-li tedy, aby se \textbf{zmenšujícím se} $\varepsilon$ byla
 hodnota dané funkce stále blíže nekonečnu, musí se tato hodnota
 \textbf{zvětšovat}. Díky užité formulaci tomu tak je, neboť s menším
 $\varepsilon$ je číslo $1 / \varepsilon$ větší.
\end{remark}

\begin{figure}[ht]
 \centering
 \begin{subfigure}[b]{.49\textwidth}
  \centering
  \begin{tikzpicture}
   \tkzInit[xmin=-2,xmax=2]
   \tkzDrawX[label=,>=]
   \tkzDefPoints{0/0/a,-1.7/0/b,1.7/0/c}
   \tkzLabelPoint[below=2mm,color=BrickRed](a){$a$}

   \tkzDrawSegment[color=RoyalBlue,ultra thick](b,c)
   \tkzDrawPoint[size=4,color=BrickRed](a)
   \node[color=RoyalBlue] at (b) {\Large $($};
   \node[color=RoyalBlue] at (c) {\Large $)$};
   \tkzLabelPoint[below=2mm,color=RoyalBlue](b){$\clr{a} - \varepsilon$}
   \tkzLabelPoint[below=2mm,color=RoyalBlue](c){$\clr{a} + \varepsilon$}
  \end{tikzpicture}
  \caption{Okolí $\clb{B(\clr{a},\varepsilon)}$ bodu $\clr{a}$.}
  \label{subfig:okoli-a-prstencove-okoli-bodu-1}
 \end{subfigure}
 \begin{subfigure}[b]{.49\textwidth}
  \centering
  \begin{tikzpicture}
   \tkzInit[xmin=-2,xmax=2]
   \tkzDrawX[label=,>=]
   \tkzDefPoints{0/0/a,-1.7/0/b,1.7/0/c}
   \tkzLabelPoint[below=2mm,color=BrickRed](a){$a$}

   \tkzDrawSegment[color=ForestGreen,ultra thick](b,c)
   \tkzDrawPoint[size=6,draw=BrickRed,fill=white](a)
   \node[color=ForestGreen] at (b) {\Large $($};
   \node[color=ForestGreen] at (c) {\Large $)$};
   \tkzLabelPoint[below=2mm,color=ForestGreen](b){$\clr{a} - \varepsilon$}
   \tkzLabelPoint[below=2mm,color=ForestGreen](c){$\clr{a} + \varepsilon$}
  \end{tikzpicture}
  \caption{Prstencové okolí $\clg{R(\clr{a},\varepsilon)}$ bodu $\clr{a}$.}
  \label{subfig:okoli-a-prstencove-okoli-bodu-2}
 \end{subfigure}
 \caption{Okolí a prstencové okolí bodu $\clr{a} \in \R$.}
 \label{fig:okoli-a-prstencove-okoli-bodu}
\end{figure}

\begin{definition}{Jednostranná limita funkce}{jednostranna-limita-funkce}
 Ať $M \subseteq \R, f:M \to \R$ a $a \in \R^*$. Řekneme, že číslo $L \in \R^*$
 je \emph{limitou zleva} funkce $f$ v bodě $a$, pokud
 \[
 \forall \varepsilon>0 \, \exists \delta>0 \, \forall x \in P_{-}(a,\delta):
 f(x) \in B(L,\varepsilon).
 \]
 Tento fakt zapisujeme jako $L = \lim_{x \to a^{-}} f(x)$.

 Podobně, číslo $K \in \R^{*}$ je \emph{limitou zprava} funkce $f$ v bodě $a$,
 pokud
 \[
 \forall \varepsilon>0 \, \exists \delta>0 \, \forall x \in P_{+}(a,\delta):
 f(x) \in B(K,\varepsilon).
 \]
 Tento fakt zapisujeme jako $K = \lim_{x \to a^{+}} f(x)$.
\end{definition}

\begin{figure}[ht]
 \centering
 \begin{subfigure}[b]{.49\textwidth}
  \centering
  \begin{tikzpicture}[scale=1.25]
   \tkzInit[xmin=-1,xmax=3,ymin=-0.5,ymax=2.5]

   \tkzDefPoints{1/0/a,0/1.84/L,1/1.84/x}
   \tkzDefPoints{3/1.122/max1,3/2.558/max2}
   \tkzDefPoints{0.35/0/ad,0.35/1.122/xd,0/1.122/Le1,0/2.558/Le2}
   \tkzDrawPolygon[fill=Magenta!20!white,draw=none](Le1,max1,max2,Le2)

   \tkzDrawX[label=]
   \tkzDrawY[label=]

   \draw[scale=1,domain=-1:3.14,smooth,thick,variable=\x,RoyalBlue] plot
    ({\x},{sin(\x^2 r) + 1});
   \tkzDrawSegment[dashed,thick,BrickRed](a,x)
   \tkzDrawSegment[dashed,thick,BrickRed](ad,xd)
   \tkzDrawSegment[ultra thick,BrickRed](ad,a)
   \tkzDrawSegment[dashed,thick,ForestGreen](L,x)
   \tkzDrawPoint[size=6,draw=BrickRed,thick,fill=white](a)
   \tkzLabelPoint[below=2mm,color=BrickRed](a){$a$}
   \tkzDrawPoint[size=4,color=ForestGreen](L)
   \tkzLabelPoint[left=1mm,color=ForestGreen](L){$L$}

   \tkzDrawSegments[color=Magenta](Le1,max1 Le2,max2)
   \tkzDrawPoint[size=4,color=RoyalBlue](x)
   \tkzDrawPoint[size=4,draw=BrickRed,thick,fill=white](ad)
   \tkzDrawPoint[size=4,color=RoyalBlue](xd)
   \tkzLabelPoint[below=1mm,color=BrickRed](ad){$a - \delta$}
   \tkzDrawPoints[size=4,color=Magenta](Le1,Le2)
   \tkzLabelPoint[left=1mm,color=Magenta](Le1){$L - \varepsilon$}
   \tkzLabelPoint[left=1mm,color=Magenta](Le2){$L + \varepsilon$}

  \end{tikzpicture}
  \caption{Limita $\clb{f}$ v bodě $\clr{a}$ zleva.}
 \end{subfigure}
 \begin{subfigure}[b]{.49\textwidth}
  \centering
  \begin{tikzpicture}[scale=1.25]
   \tkzInit[xmin=-1,xmax=3,ymin=-0.5,ymax=2.5]

   \tkzDefPoints{1/0/a,0/1.84/L,1/1.84/x}
   \tkzDefPoints{3/1.407/max1,3/2.273/max2}
   \tkzDefPoints{1.65/0/ad,1.65/1.407/xd,0/1.407/Le1,0/2.273/Le2}
   \tkzDrawPolygon[fill=Magenta!20!white,draw=none](Le1,max1,max2,Le2)

   \tkzDrawX[label=]
   \tkzDrawY[label=]

   \draw[scale=1,domain=-1:3.14,smooth,thick,variable=\x,RoyalBlue] plot
    ({\x},{sin(\x^2 r) + 1});
   \tkzDrawSegment[dashed,thick,BrickRed](a,x)
   \tkzDrawSegment[dashed,thick,BrickRed](ad,xd)
   \tkzDrawSegment[ultra thick,BrickRed](ad,a)
   \tkzDrawSegment[dashed,thick,ForestGreen](L,x)
   \tkzDrawPoint[size=6,draw=BrickRed,thick,fill=white](a)
   \tkzLabelPoint[below=2mm,color=BrickRed](a){$a$}
   \tkzDrawPoint[size=4,color=ForestGreen](L)
   \tkzLabelPoint[left=1mm,color=ForestGreen](L){$L$}

   \tkzDrawSegments[color=Magenta](Le1,max1 Le2,max2)
   \tkzDrawPoint[size=4,color=RoyalBlue](x)
   \tkzDrawPoint[size=4,draw=BrickRed,thick,fill=white](ad)
   \tkzDrawPoint[size=4,color=RoyalBlue](xd)
   \tkzLabelPoint[below=1mm,color=BrickRed](ad){$a + \delta$}
   \tkzDrawPoints[size=4,color=Magenta](Le1,Le2)
   \tkzLabelPoint[left=1mm,color=Magenta](Le1){$L - \varepsilon$}
   \tkzLabelPoint[left=1mm,color=Magenta](Le2){$L + \varepsilon$}

  \end{tikzpicture}
  \caption{Limita $\clb{f}$ v bodě $\clr{a}$ zprava.}
 \end{subfigure}
 \caption{Jednostranné limity funkce $\clb{f}$ v bodě $\clr{a}$.}
 \label{fig:jednostranna-limita-funkce}
\end{figure}

\begin{warning}{}{okoli-v-limite}
 Fakt, že $L$ je limita \textbf{zleva} funkce $f$ v bodě $a$, vůbec neznamená,
 že hodnoty $f(x)$ se musejí blížit k $L$ rovněž \textbf{zleva}. Adverbia
 \emph{zleva} a \emph{zprava} značí pouze směr, kterým se k číslu $a$ přibližují
 \textbf{vstupy} funkce $f$, nikoli její \textbf{výstupy} k číslu $L$.
\end{warning}

Pochopitelně, lze též požadovat, aby hodnoty $f$ ležely v daném rozmezí kolem
bodu $L$, jak se její vstupy blíží k $a$ zleva i zprava zároveň. V principu,
blíží-li se $f$ ke stejnému číslu zleva i zprava, stačí vzít $\delta$ v
\myref{definici}{def:jednostranna-limita-funkce} tak malé, aby $f(x)$ leželo v
$B(L,\varepsilon)$ kdykoli je $x$ ve vzdálenosti nejvýše $\delta$ od $a$.

\begin{definition}{Oboustranná limita funkce}{oboustranna-limita-funkce}
 Ať $a,L \in \R^{*}$ a $f$ je reálná funkce. Řekneme, že $L$ je
 \emph{(oboustrannou) limitou} funkce $f$ v bodě $a$, pokud
 \[
 \forall \varepsilon > 0 \, \exists \delta > 0 \, \forall x \in P(a,\delta):
 f(x) \in B(L,\varepsilon).
 \]
 Tento fakt zapisujeme jako $L = \lim_{x \to a} f(x)$.
\end{definition}

Je jistě možné představovat si oboustrannou limitu funkce stejně jako limity
jednostranné na \myref{obrázku}{fig:jednostranna-limita-funkce}. Ovšem, ona
vlastnost \uv{oboustrannosti} umožňuje ještě jiný -- však ne rigorózní --
pohled. Povýšíme-li situaci do roviny, tj. do prostoru druhé dimenze, a na
funkci $f$ budeme nahlížet jako na zobrazení bodů roviny na body roviny, pak $L$
je limitou funkce $f$ v bodě $a$, když zobrazuje všechny body zevnitř kruhu o
poloměru $\delta$ a středu $a$ do kruhu o poloměru $\varepsilon$ a středu $L$.
Jako na \myref{obrázku}{fig:oboustranna-limita-ve-2D}.

\begin{figure}[ht]
 \centering
 \begin{tikzpicture}
  \def\del{1}
  \def\eps{1.5}

  \tkzDefPoints{0/0/a,4/2/L}
  \tkzLabelPoint[color=BrickRed,below=1mm](a){$a$}
  \tkzLabelPoint[color=ForestGreen,below=1mm](L){$L$}

  \tkzDefCircle[R](a,\del) \tkzGetPoint{ad}
  \tkzDrawCircle[thick](a,ad)

  \tkzDefCircle[R](L,\eps) \tkzGetPoint{Le}
  \tkzDrawCircle[color=Magenta,thick](L,Le)

  \tkzDefPoints{0/1.2/s,2.8/3.2/t}
  \draw[bend left=45,color=RoyalBlue,-latex] (s) to node[midway,above left]{$f$}
   (t);

  \tkzDefPointOnCircle[R = center a angle 150 radius \del] \tkzGetPoint{B}
  \tkzDrawPoint[color=BrickRed,size=3](B)
  \tkzDrawSegment[color=BrickRed,dashed,thick](a,B)
  \tkzLabelSegment[color=BrickRed,above right=-1mm](a,B){$\delta$}
  \tkzDrawPoint[draw=BrickRed,thick,size=6,fill=white](a)

  \tkzDefPointOnCircle[R = center L angle 150 radius \eps] \tkzGetPoint{C}
  \tkzDrawPoint[color=Magenta,size=3](C)
  \tkzDrawSegment[color=Magenta,dashed,thick](L,C)
  \tkzLabelSegment[color=Magenta,above right=-1mm](L,C){$\varepsilon$}
  \tkzDrawPoint[color=ForestGreen,size=6](L)

  \tkzDefPoints{0.5/0.5/x,3.3/1.8/fx}
  \tkzDrawPoint[size=4,color=black](x)
  \tkzDrawPoint[size=4,color=RoyalBlue](fx)
  \tkzLabelPoint[below](x){$x$}
  \tkzLabelPoint[below,color=RoyalBlue](fx){$f(x)$}
 \end{tikzpicture}

 \caption{Oboustranná limita funkce \uv{ve 2D}.}
 \label{fig:oboustranna-limita-ve-2D}
\end{figure}

Doporučujeme čtenářům, aby se zamysleli, čím by v této dvoudimenzionální říši
byla \emph{jednostranná} limita funkce. Sen zámysl snad vedl k představě, že by
se vstupy $x$ musely blížit k bodu $a$ po nějaké určené přímce. Existence
\uv{všestranné} limity v $a$ by pak byla ekvivalentní existenci nespočetně mnoha
\uv{jednostranných} limit -- jedné pro každou přímku procházející bodem $a$.
Věříme, že není obtížné nahlédnout, jak zbytečný by takový pojem ve dvou
dimenzích byl. Popsaná situace přímo souvisí s faktem, že první dimenze je z
geometrického pohledu \uv{degenerovaná} -- kružnice je pouze dvoubodovou
množinou.

Oboustranné limity jsou spjaty jednostrannými velmi přirozeným způsobem.
Existence oboustran\-né limity funkce v bodě je ekvivalentní existenci limity
jak zleva, tak zprava, v témže bodě. Oboustrannou limitu vlastně dostaneme tak,
že z levého a pravého prstencového okolí limitního bodu, ve kterém již je
funkční hodnota blízko limitě, vybereme to menší.

\begin{proposition}{Vztah jednostranných a oboustranných
 limit}{vztah-jednostrannych-a-oboustrannych-limit}
 Ať $f$ je reálná funkce a $a \in \R^{*}$. Pak $\lim_{x \to a} f(x)$ existuje
 právě tehdy, když existuje $\lim_{x \to a^{+}} f(x)$ i $\lim_{x \to a^{-}}
 f(x)$ a jsou si rovny.
\end{proposition}

\begin{propproof}
 Implikace $( \Rightarrow )$ je triviální. Pokud existuje $L \coloneqq \lim_{x
 \to a} f(x)$, pak pro dané $\varepsilon>0$ máme nalezeno $\delta>0$ takové, že
 pro $x \in R(a,\delta)$ je $f(x) \in B(L,\varepsilon)$. Ovšem, jistě platí
 $R_+(a,\delta) \subseteq R(a,\delta)$ i $R_-(a,\delta) \subseteq R(a,\delta)$.
 To však znamená, že pro $x \in R_+(a,\delta)$ i pro $x \in R_-(a,\delta)$
 rovněž platí $f(x) \in B(L,\varepsilon)$. To dokazuje, že existuje jak $\lim_{x
 \to a^{+}} f(x)$, tak $\lim_{x \to a^{-}} f(x)$ a
 \[
  \lim_{x \to a^{+}} f(x) = \lim_{x \to a^{-}} f(x) = L.
 \]
 
 Pro důkaz $( \Leftarrow )$ položme $L \coloneqq \lim_{x \to a^{+}} f(x) =
 \lim_{x \to a^{-}} f(x)$ a pro dané $\varepsilon>0$ nalezněme $\delta_+>0$ a
 $\delta_- >0$ splňující výroky
 \begin{align*}
  \forall x \in R_+(a,\delta_+) &: f(x) \in B(L,\varepsilon),\\
  \forall x \in R_-(a,\delta_-) &: f(x) \in B(L,\varepsilon).
 \end{align*}
 Ať $\delta \coloneqq \min(\delta_+,\delta_-)$. Pak $R(a,\delta) \subseteq
 R_+(a,\delta_+) \cup R_-(a,\delta_-)$, a tedy $\delta>0$ splňuje, že
 \[
  \forall x \in R(a,\delta):f(x) \in B(L,\varepsilon),
 \]
 čili $\lim_{x \to a} f(x) = L$.
\end{propproof}

\begin{example}{}{limita-signum-0}
 \myref{Tvrzení}{prop:vztah-jednostrannych-a-oboustrannych-limit} je užitečné
 jak v uvedené, tak v kontrapozitivní formě, tj. při důkazu neexistence
 oboustranné limity za předpokladu nerovnosti (nikoli nutně \emph{neexistence})
 limit jednostranných.

 Uvažme funkce $f(x) \coloneqq \sign(x)$ a $g(x) \coloneqq |\sign(x)|$. Ukážeme,
 že $\lim_{x \to 0} f(x)$ neexistuje, zatímco $\lim_{x \to 0} g(x) = 1$.
 Připomeňme, že funkce $\sign(x)$ je definována předpisem
 \[
  \sign(x) \coloneqq \begin{cases}
   -1, & x < 0;\\
   0, & x = 0;\\
   1, & x > 0.
  \end{cases}
 \]

 Ověříme, že $\lim_{x \to 0^{+}} f(x) = 1$. Mějme dáno $\varepsilon>0$. Volme
 například $\delta \coloneqq 1$. Potom pro $x \in R_+(0,1) = (0,1)$ platí $f(x)
 = 1$, čili zřejmě $f(x) \in B(1,\varepsilon) = (1-\varepsilon,1+\varepsilon)$.
 Podobně se ověří, že $\lim_{x \to 0^{-}} f(x) = -1$. To ovšem znamená, že
 $\lim_{x \to 0^{+}} f(x) \neq \lim_{x \to 0^{-}} f(x)$, tudíž dle
 \myref{tvrzení}{prop:vztah-jednostrannych-a-oboustrannych-limit} $\lim_{x \to
 0} f(x)$ neexistuje.

 Velmi obdobným argumentem ukážeme, že $\lim_{x \to 0^{+}} g(x) = \lim_{x \to
 0^{-}} g(x) = 1$. Nuže, podle
 \hyperref[prop:vztah-jednostrannych-a-oboustrannych-limit]{téhož tvrzení} platí
 $\lim_{x \to 0} g(x) = 1$.
\end{example}

\begin{figure}[ht]
 \centering
 \begin{subfigure}[b]{0.49\textwidth}
  \centering
  \begin{tikzpicture}
   \tkzInit[xmin=-3,xmax=3,ymin=-1,ymax=1]
   \tkzDrawX[color=BrickRed]
   \tkzDrawY[color=RoyalBlue]

   \tkzDefPoints{-3/-1/a,0/-1/b,0/0/c,0/1/d,3/1/e}
   \tkzDrawSegments[color=Fuchsia,thick](a,b d,e)
   \tkzDrawPoints[size=4,draw=Fuchsia,fill=white,thick](b,d)
   \tkzDrawPoint[size=4,color=Fuchsia](c)
  \end{tikzpicture}
  \caption{Graf funkce $\clm{f(x) = \sign(x)}$.}
 \end{subfigure}
 \begin{subfigure}[b]{0.49\textwidth}
  \centering
  \begin{tikzpicture}
   \tkzInit[xmin=-3,xmax=3,ymin=-1,ymax=1]
   \tkzDrawX[color=BrickRed]
   \tkzDrawY[color=RoyalBlue]

   \tkzDefPoints{-3/1/a,0/1/b,0/0/c,0/1/d,3/1/e}
   \tkzDrawSegments[color=Fuchsia,thick](a,b d,e)
   \tkzDrawPoints[size=4,draw=Fuchsia,fill=white,thick](b,d)
   \tkzDrawPoint[size=4,color=Fuchsia](c)
  \end{tikzpicture}
  \caption{Graf funkce $\clm{g(x) = |\sign(x)|}$.}
 \end{subfigure}
 \caption{Obrázek k \myref{příkladu}{exam:limita-signum-0}.}
 \label{fig:limita-signum-0}
\end{figure}

\begin{exercise}{}{limita-absolutni-hodnota}
 Dokažte, že pro reálnou funkci $f$ a $a \in \R^{*}$ platí
 \[
  \lim_{x \to a} f(x) = 0 \Leftrightarrow \lim_{x \to a} |f(x)| = 0.
 \]
\end{exercise}

\section{Základní poznatky o limitě funkce}
\label{sec:zakladni-poznatky-o-limite-funkce}

Počneme nyní shrnovati intuitivně vcelku zřejmé výsledky o limitách reálných
funkcí. Jakž jsme již vícekrát děli, ona \uv{intuitivní zřejmost} pravdivosti
výroků nechce nabodnout k přeskoku či trivializaci jejich důkazů. Vodami
nekonečnými radno broditi se ostražitě, bo tvrzení jako
\hyperref[thm:limita-slozene-funkce]{limita složené funkce} ráda svědčí, že
intuicí bez logiky člověk na břeh nedoplove.

Na první pád není překvapivé, že limita funkce je jednoznačně určena,
pochopitelně za předpokladu její existence. Vyzýváme čtenáře, aby se při čtení
důkazu drželi vizualizace oboustranné limity z
\myref{obrázku}{fig:oboustranna-limita-ve-2D}.

\begin{lemma}{Jednoznačnost limity}{jednoznacnost-limity}
	Limita funkce (ať už jednostranná či oboustranná) je jednoznačně určená, pokud
	existuje.
\end{lemma}

\begin{lemproof}
	Dokážeme lemma pouze pro oboustrannou limitu, důkaz pro limity jednostranné je
	v zásadě totožný.

	Pro spor budeme předpokládat, že $L$ i $L'$ jsou limity $f$ v bodě $a \in
	\R^{*}$. Nejprve ošetříme případ, kdy $L,L' \in \R$. Bez újmy na obecnosti
	smíme předpokládat, že $L > L'$. Volme $\varepsilon \coloneqq (L-L') / 3$ (ve
	skutečnosti stačilo volit libovolné $\varepsilon < (L-L') / 2$). K tomuto
	$\varepsilon$ existují z \hyperref[def:oboustranna-limita-funkce]{definice
	limity} $\delta_1>0, \delta_2>0$ takové, že
	\[
		\forall x \in R(a,\delta_1): f(x) \in B(L,\varepsilon).
	\]
	a rovněž
	\[
		\forall x \in R(a,\delta_2): f(x) \in B(L',\varepsilon).
	\]
	Volíme-li ovšem $\delta \coloneqq \min(\delta_1,\delta_2)$, pak pro $x \in
		R(a,\delta)$ dostaneme
	\[
		f(x) \in B(L,\varepsilon) \cap B(L',\varepsilon).
	\]
	Poslední vztah lze přepsat do tvaru
	\begin{align*}
		L - \varepsilon  & < f(x) < L + \varepsilon,  \\
		L' - \varepsilon & < f(x) < L' + \varepsilon.
	\end{align*}
	Odtud plyne, že
	\[
		L - \varepsilon < L' + \varepsilon,
	\]
	což po dosazení $\varepsilon = (L - L') / 3$ a následné úpravě vede na
	\[
		2L - L' < 2L' - L,
	\]
	z čehož ihned
	\[
		L < L',
	\]
	což je spor.

	Nyní ať například $L = \infty$ a $L' \in \R$. Z
	\hyperref[def:okoli-a-prstencove-okoli-bodu]{definice okolí} $B(L,\varepsilon)$
	pro $L = \infty$ stačí nalézt $\varepsilon > 0$ takové, že
	\[
		\frac{1}{\varepsilon} > L' + \varepsilon,
	\]
	pak se totiž nemůže stát, že
	\[
		f(x) \in B(\infty,\varepsilon) \cap B(L',\varepsilon).
	\]
	Snadným výpočtem zjistíme, že
	\[
		\frac{1}{\varepsilon} > L' + \varepsilon
	\]
	právě tehdy, když $\varepsilon < (\sqrt{L'^2 + 4} - L') / 2$. Pro libovolné
	takové $\varepsilon$ tudíž dostáváme spor stejně jako v předchozím případě.

	Ostatní případy se ošetří obdobně.
\end{lemproof}

\begin{figure}[ht]
 \centering
 \begin{tikzpicture}
	\def\del{1}
	\def\eps{1.5}

	\tkzDefPoints{0/0/a,4/2/L,4/-2/Lp}
	\tkzLabelPoint[color=BrickRed,below=1mm](a){$a$}
	\tkzLabelPoint[color=ForestGreen,below left=1mm](L){$L$}
	\tkzLabelPoint[color=ForestGreen,below left=1mm](Lp){$L'$}

	\tkzDefCircle[R](a,\del) \tkzGetPoint{ad}
	\tkzDrawCircle[thick](a,ad)

	\tkzDefCircle[R](L,\eps) \tkzGetPoint{Le}
	\tkzDrawCircle[color=Magenta,thick](L,Le)

	\tkzDefPoints{0/1.2/s,2.8/3.2/t,0/-1.2/s2,2.8/-3.2/t2}
	\draw[bend left=45,color=RoyalBlue,-latex] (s) to node[midway,above left]{$f$}
	(t);
	\draw[bend right=45,color=RoyalBlue,-latex] (s2) to node[midway,below
		left]{$f$} (t2);

	\tkzDefPointOnCircle[R = center a angle 150 radius \del] \tkzGetPoint{B}
	\tkzDrawPoint[color=BrickRed,size=3](B)
	\tkzDrawSegment[color=BrickRed,dashed,thick](a,B)
	\tkzLabelSegment[color=BrickRed,above right=-1mm](a,B){$\delta$}
	\tkzDrawPoint[draw=BrickRed,thick,size=6,fill=white](a)

	\tkzDrawSegment[decorate,decoration={brace,
				amplitude=10pt},color=Aquamarine,thick](L,Lp)
	\tkzLabelSegment[right=3mm,color=Aquamarine](L,Lp){$|L - L'|$}

	\tkzDefPointOnCircle[R = center L angle 150 radius \eps] \tkzGetPoint{C}
	\tkzDrawPoint[color=Magenta,size=3](C)
	\tkzDrawSegment[color=Magenta,dashed,thick](L,C)
	\tkzLabelSegment[color=Magenta,above right=-1mm](L,C){$\varepsilon$}
	\tkzDrawPoint[color=ForestGreen,size=6](L)

	\tkzDefCircle[R](Lp,\eps) \tkzGetPoint{Lpe}
	\tkzDrawCircle[color=Magenta,thick](Lp,Lpe)

	\tkzDefPointOnCircle[R = center Lp angle 150 radius \eps] \tkzGetPoint{D}
	\tkzDrawPoint[color=Magenta,size=3](D)
	\tkzDrawSegment[color=Magenta,dashed,thick](Lp,D)
	\tkzLabelSegment[color=Magenta,above right=-1mm](Lp,D){$\varepsilon$}
	\tkzDrawPoint[color=ForestGreen,size=6](Lp)

 \end{tikzpicture}
 \caption{Spor v důkazu \myref{lemmatu}{lem:jednoznacnost-limity}.}
 \label{fig:jednoznacnost-limity}
\end{figure}

\begin{lemma}{}{ma-limitu-je-omezena}
 Ať reálná funkce $f$ má \textbf{konečnou} limitu $L \in \R$ v bodě $a \in
 \R^{*}$. Pak existuje prstencové okolí $a$, na němž je $f$ omezená.
\end{lemma}
\begin{lemproof}
 Pro dané $\varepsilon>0$ nalezneme z
 \hyperref[def:oboustranna-limita-funkce]{definice limity} $\delta>0$ takové, že
 pro $x \in R(a,\delta)$ platí $f(x) \in B(L,\varepsilon)$. Protože však
 $B(L,\varepsilon) = (L-\varepsilon,L+\varepsilon)$ platí pro $x \in
 R(a,\delta)$ odhady
 \[
  L-\varepsilon \leq f(x) \leq L+\varepsilon,
 \]
 čili je $f$ na $R(a,\delta)$ omezená.
\end{lemproof}

Vzhledem k základním aritmetickým operacím si limity funkcí počínají vychovaně.
Za předpokladu, že výsledný výraz dává smysl, můžeme spočítat limitu součtu,
součinu či podílu funkcí jako součet, součin či podíl limit těchto funkcí.

\begin{theorem}{Aritmetika limit funkcí}{aritmetika-limit-funkci}
 Ať $f,g$ jsou reálné funkce a $a \in \R^{*}$. Předpokládejme, že $\lim_{x \to
 a} f(x)$ i $\lim_{x \to a} g(x)$ existují a označme je po řadě $L_f$ a $L_g$.
 Potom platí
 \begin{enumerate}[label=(\alph*)]
  \item $\lim_{x \to a} (f + g)(x) = L_f + L_g$, dává-li výraz napravo smysl.
  \item $\lim_{x \to a} (f \cdot g)(x) = L_f \cdot L_g$, dává-li výraz napravo
   smysl.
  \item $\lim_{x \to a} (f / g)(x) = L_f / L_g$, dává-li výraz napravo smysl.
 \end{enumerate}
\end{theorem}

\begin{thmproof}
 Dokážeme pouze část (c), neboť je výpočetně nejnáročnější, ač nepřináší mnoho
 intuice. Část (a) je triviální a (b) je lehká. Vyzýváme čtenáře, aby se je
 pokusili dokázat sami.

 Už jen v důkazu samotné části (c) bychom správně měli rozlišit šest různých
 případů:
 \begin{enumerate}
  \item $L_f \in \R, L_g \in \R \setminus \{0\}$,
  \item $L_f \in \R, L_g \in \{-\infty,\infty\}$,
  \item $L_f = \infty, L_g \in (0,\infty)$,
  \item $L_f = \infty, L_g \in (-\infty,0)$,
  \item $L_f = -\infty, L_g \in (0,\infty)$,
  \item $L_f = -\infty, L_g \in (-\infty,0)$.
 \end{enumerate}
 Jelikož se výpočty limit v oněch případech liší vzájemně pramálo a získaná
 intuice je asymptoticky rovna té ze znalosti metod řešení exponenciálních
 rovnic, soustředíme se pouze na (nejzajímavější) případ (1).

 Ať tedy $L_f \in \R, L_g \in \R \setminus \{0\}$. Je nejprve dobré si uvědomit,
 proč vynecháváme $0$ jako možnou hodnotu $L_g$. Totiž, $L_f / L_g$ není
 definován \textbf{nikdy}, pokud $L_g = 0$, bez ohledu na hodnotu $L_f$. Hodnoty
 $g$ se mohou k $L_g$ limitně blížit zprava, zleva či střídavě z obou směrů.
 Nelze tudíž obecně určit, zda dělíme klesajícím kladným číslem, či rostoucím
 záporným číslem.

 Položme $\varepsilon_g = |L_g| / 2$. K tomuto $\varepsilon_g$ existuje z
 \hyperref[def:oboustranna-limita-funkce]{definice limity} $\delta_g$ takové, že
 pro $x \in R(a,\delta_g)$ platí $g(x) \in B(L_g,\varepsilon_g)$. Poslední vztah
 si přepíšeme na
 \begin{align*}
  L_g - \varepsilon_g & < g(x) < L_g + \varepsilon_g,\\
  L_g - \frac{|L_g|}{2} & < g(x) < L_g + \frac{|L_g|}{2}.
 \end{align*}
 Speciálně tedy pro $x \in R(a,\delta_g)$ máme odhad
 \[
  |g(x)| > \left| L_g - \frac{|L_g|}{2} \right| > \frac{|L_g|}{2}.
 \]
 Jelikož poslední výraz je z předpokladu kladný, má výraz $f(x) / g(x)$ smysl
 pro každé $x \in R(a,\delta_g)$, neboť pro tato $x$ platí $g(x) \neq 0$.

 Pro $x \in R(a,\delta_g)$ odhadujme
 \begin{align*}
  \left| \frac{f(x)}{g(x)} - \frac{L_f}{L_g} \right| &= \frac{|f(x)L_g -
  g(x)L_f|}{|g(x)||L_g|} = \frac{|f(x) L_g - L_f L_g + L_f L_g -
  g(x)L_f|}{|g(x)| |L_g|}\\
  																									 & \leq \frac{|L_g| |f(x) -
  																									 L_f| + |L_f| |L_g -
  																									g(x)|}{|g(x)| |L_g|}\\
  																									 &= \frac{1}{|g(x)|}
  																									 |f(x) - L_f| +
  																									 \frac{|L_f|}{|g(x)|
  																									 |L_g|}|L_g - g(x)|\\
  																									 &< \frac{2}{|L_g|}|f(x) -
  																									 L_f| + \frac{2
  																									 |L_f|}{|L_g|^2}|L_g -
  																									 g(x)|\\
  																									 & \leq c(|f(x) - L_f| +
  																									 |L_g - g(x)|)
 \end{align*}
 pro $c \coloneqq \max(2 / |L_g|, 2|L_f|/|L_g|^2)$.

 Ať je nyní dáno $\varepsilon>0$. K číslu $\varepsilon / 2c$ existují z
 \hyperref[def:oboustranna-limita-funkce]{definice limity} $\delta_1,\delta_2>0$
 taková, že
 \begin{align*}
 	\forall x \in R(a,\delta_1)&: |g(x) - L_g| < \frac{\varepsilon}{2c},\\
 	\forall x \in R(a,\delta_2)&: |f(x) - L_f| < \frac{\varepsilon}{2c}.
 \end{align*}
 Položíme-li nyní $\delta \coloneqq \min(\delta_1,\delta_2,\delta_g)$, pak pro
 $x \in R(a,\delta)$ platí
 \[
  \left| \frac{f(x)}{g(x)} - \frac{L_f}{L_g} \right| < c (|f(x) - L_f| + |L_g -
  g(x)|) < c \left(\frac{\varepsilon}{2c} + \frac{\varepsilon}{2c}\right) =
  \varepsilon,
 \]
 což dokazuje rovnost $\lim_{x \to a} (f / g)(x) = L_f / L_g$.
\end{thmproof}

\begin{warning}{}{definovanost-pravych-stran}
 Předpoklad \emph{definovanosti} výsledného výrazu ve znění
 \hyperref[thm:aritmetika-limit-funkci]{věty o aritmetice limit} je zásadní.

 Uvažme funkce $f(x) = x + c$ pro libovolné $c \in \R$, $g(x) = -x$. Pak platí
 \begin{align*}
  \lim_{x \to \infty} f(x) &= \infty,\\
  \lim_{x \to \infty} g(x) &= -\infty,\\
  \lim_{x \to \infty} (f(x) + g(x)) &= c,
 \end{align*}
 ale $\lim_{x \to \infty} f(x) + \lim_{x \to \infty} g(x)$ není definován.
\end{warning}

\begin{exercise}{}{aritmetika-limit}
 Dokažte tvrzení (b) a (c) ve \myref{větě}{thm:aritmetika-limit-funkci}.
\end{exercise}

\begin{problem}{}{vypocet-pres-aritmetiku-limit}
 Spočtěte
 \[
  \lim_{x \to \infty} \frac{(2x - 3)^{20} (3x + 2)^{30}}{(2x + 1)^{50}}.
 \]
\end{problem}
\begin{probsol}
 Jelikož limitním bodem je $\infty$, neliší se výpočet limity funkce v zásadě
 nijak od výpočtu limity sesterské posloupnosti. Stále je třeba identifikovat a
 vytknout \uv{nejrychleji rostoucí} členy z čitatele a jmenovatele zlomku a poté
 se odkázat na \hyperref[thm:aritmetika-limit-funkci]{aritmetiku limit}.

 Přímým dosazením zjistíme, že bez dalších úprav vychází limitní výraz $\infty /
 \infty$, na jehož základě nelze nic rozhodnout. Upravujeme tudíž následující
 způsobem:
 \[
  \frac{(2x-3)^{20}(3x+2)^{30}}{(2x+1)^{50}} = \frac{x^{20}(2 -
  \frac{3}{x^{20}}) \cdot x^{30}(3 + \frac{2}{x^{30}})}{x^{50}(2 +
 	\frac{1}{x^{50}})} = \frac{x^{50}}{x^{50}} \cdot \frac{(2 - \frac{3}{x^{20}})(3
 	+ \frac{2}{x^{30}})}{2 + \frac{1}{x^{50}}}.
 \]
 Předpokládajíce definovanost výsledného výrazu (již je třeba ověřit až na
 samotném konci výpočtu), smíme z
 \hyperref[thm:aritmetika-limit-funkci]{aritmetiky limit} tvrdit, že
 \[
  \lim_{x \to \infty} \frac{(2x-3)^{20}(3x+2)^{30}}{(2x+1)^{50}} = \lim_{x \to
  \infty} \frac{x^{50}}{x^{50}} \cdot \lim_{x \to \infty} \frac{(2 -
  \frac{3}{x^{20}})(3 + \frac{2}{x^{30}})}{2 + \frac{1}{x^{50}}}.
 \]
 Zřejmě platí
 \[
  \lim_{x \to \infty} \frac{x^{50}}{x^{50}} = 1.
 \]
 Opět použitím \hyperref[thm:aritmetika-limit-funkci]{aritmetiky limit} můžeme
 počítat
 \[
  \lim_{x \to \infty} \frac{(2 - x^{20})(3 + \frac{2}{x^{30}})}{2 +
  \frac{1}{x^{50}}} = \frac{\lim_{x \to \infty} (2 - \frac{1}{x^{20}}) \cdot
 	\lim_{x \to \infty} (3 + \frac{2}{x^{30}})}{\lim_{x \to \infty} (2 +
 	\frac{1}{x^{50}})} = \frac{(2 - 0) \cdot (3 + 0)}{2 + 0} = 3.
 \]
 Celkem tedy
 \[
  \lim_{x \to \infty} \frac{(2x-3)^{20}(3x+2)^{30}}{(2x+1)^{50}} = 1 \cdot 3 =
  3.
 \]
 Protože výsledný výraz je definován, byla
 \hyperref[thm:aritmetika-limit-funkci]{věta o aritmetice limit} použita
 korektně.
\end{probsol}

\begin{problem}{}{vypocet-pres-aritmetiku-limit-2}
 Spočtěte
 \[
  \lim_{x \to 3} \frac{\sqrt{x+13} - 2 \sqrt{x+1}}{x^2 - 9}
 \]
\end{problem}
\begin{probsol}
 Úlohy na výpočet limit funkcí v bodech jiných než $ \pm \infty$ jsou však
 fundamentálně rozdílné od výpočtu limit posloupností. Nelze již rozumně hovořit
 o \uv{rychlosti růstu některého členu} či podobných konceptech. Výpočet se
 pochopitelně stále opírá o \hyperref[thm:aritmetika-limit-funkci]{větu o
 aritmetice limit}, ale často dožaduje jiných algebraických úprav -- včetně
 dělení mnohočlenů.

 Dosazením $x = 3$ do zadaného výrazu získáme $0 / 0$, tedy je třeba pro výpočet
 limity výraz nejprve upravit.

 Zde postupujeme takto:
 \begin{align*}
  \frac{\sqrt{x + 13} - 2 \sqrt{x+1}}{x^2 - 9} &= \frac{\sqrt{x+13} - 2
  \sqrt{x+1}}{(x-3)(x+3)} \cdot \frac{\sqrt{x+13} + 2 \sqrt{x+1}}{\sqrt{x + 13}
 	+ 2 \sqrt{x+1}}\\
 																							 &= \frac{(x+13) -
 																							 4(x+1)}{(x-3)(x+3)(\sqrt{x+13} +
 																							 2 \sqrt{x+1})}.
 \end{align*}
 Nyní,
 \[
  (x + 13) - 4(x + 1) = -3x + 9 = -3(x-3).
 \]
 Pročež,
 \[
 	\frac{(x + 13) - 4(x + 1)}{(x-3)(x+3)(\sqrt{x+13} + 2 \sqrt{x+1})} =
 	\frac{-3}{(x+3)(\sqrt{x+13} + 2 \sqrt{x+1})}.
 \]
 Z \hyperref[thm:aritmetika-limit-funkci]{aritmetiky limit} máme
 \begin{align*}
  \lim_{x \to 3} \frac{\sqrt{x+13} - 2 \sqrt{x+1}}{x^2 - 9} &= \lim_{x \to 3}
  \frac{-3}{(x+3)(\sqrt{x+13} + 2 \sqrt{x+1})}\\
  																													&= \frac{-3}{(3 +
  																													3)(\sqrt{3 + 13} + 2
  																												 \sqrt{3 + 1})} =
  																												 \frac{-3}{48}.
 \end{align*}
 Protože je konečný výraz definovaný, směli jsme použít
 \hyperref[thm:aritmetika-limit-funkci]{větu o aritmetice limit}.
\end{probsol}

V důkazu \hyperref[thm:aritmetika-limit-funkci]{věty o aritmetice limit} jsme
zmínili, že na jejím základě nelze nic rozhodnout v případě, že konečný výraz
vyjde $a / 0$, kde $a \in \R^{*}$. K rozřešení právě těchto situací slouží
následující tvrzení.

\begin{proposition}{}{limita-a/0}
 Ať $f,g$ jsou reálné funkce, $a \in \R^{*}$. Dále ať $\lim_{x \to a} f(x) =
 A \in \R^{*}, A > 0$, $\lim_{x \to a} g(x) = 0$ a existuje prstencové okolí
 bodu $a$, na němž je $g$ kladná.

 Potom $\lim_{x \to a} f(x) / g(x) = \infty$.
\end{proposition}
\begin{propproof}
 Ať je z předpokladu dáno $\eta>0$ takové, že pro $x \in R(a,\eta)$ je $g(x) >
 0$. Rozlišíme dva případy.

 První případ nastává, když $A \in \R$ je číslo. Mějme dáno $\varepsilon>0$.
 Jelikož $\lim_{x \to a} f(x) = A$ a $A > 0$, nalezneme pro $A / 2$ číslo
 $\delta_1 > 0$ takové, že pro $x \in R(a,\delta_1)$ platí
 \[
  f(x) \in B \left( A,\frac{A}{2} \right) = \left( \frac{A}{2}, \frac{3A}{2}
  \right),
 \]
 čili $f(x) > A / 2$. Podobně, za předpokladu $\lim_{x \to a} g(x) = 0$
 nalezneme $\delta_2>0$ takové, že pro $x \in R(a,\delta_2)$ platí
 \[
  g(x) \in B \left( 0,\frac{A}{2\varepsilon} \right) =
  \left(-\frac{A}{2\varepsilon},\frac{A}{2\varepsilon}\right),
 \]
 tedy speciálně $g(x) < A / 2\varepsilon$, z čehož dostáváme $1 / g(x) >
 2\varepsilon / A$. Celkově pro $\delta \coloneqq \min(\delta_1,\delta_2,\eta)$
 a $x \in R(a,\delta)$ můžeme počítat
 \[
  \left| \frac{f(x)}{g(x)} \right| = \frac{f(x)}{g(x)} > \frac{A}{2} \cdot
  \frac{2\varepsilon}{A} = \varepsilon,
 \]
 kde první rovnost plyne z toho, že pro $x \in R(a,\delta)$ platí $f(x) > 0$ i
 $g(x) > 0$. To dokazuje, že $\lim_{x \to a} f(x) / g(x) = \infty$ v případě
 $A \in \R$.

 Ošetřemež případ $A = \infty$. Argumentujíce analogicky předchozímu odstavci
 nalezneme $\delta_1>0$ takové, že pro $R(a,\delta_1)$ platí $f(x) > 1$ a pro
 dané $\varepsilon>0$ nalezneme $\delta_2>0$ takové, že pro $x \in
 R(a,\delta_2)$ platí $g(x) < 1 / \varepsilon$, a tedy $1 / g(x) > \varepsilon$.
 Potom, pro $x \in R(a,\min(\eta,\delta_1,\delta_2))$ platí
 \[
  \left| \frac{f(x)}{g(x)} \right| = \frac{f(x)}{g(x)} > 1 \cdot \varepsilon =
  \varepsilon,
 \]
 což dokazuje opět, že $\lim_{x \to a} f(x) / g(x) =\infty$ i v případě $A =
 \infty$. 

 Tím je důkaz završen.
\end{propproof}

\begin{remark}{}{poznamka-k-a/0}
 \hyperref[prop:limita-a/0]{Předchozí tvrzení} pochopitelně platí i při záměně
 ostrých nerovností v jeho znění. Konkrétně, za předpokladů
 \begin{itemize}
  \item[($< >$)] $\lim_{x \to a} f(x) = A < 0$ a $g(x) > 0$ na $R(a,\eta)$ platí
   $\lim_{x \to a} f(x) / g(x) = -\infty$;
  \item[($> <$)] $\lim_{x \to a} f(x) = A > 0$ a $g(x) < 0$ na $R(a,\eta)$ platí
   $\lim_{x \to a} f(x) / g(x) = -\infty$;
  \item[($< <$)] $\lim_{x \to a} f(x) = A < 0$ a $g(x) < 0$ na $R(a,\eta)$ platí
   $\lim_{x \to a} f(x) / g(x) = \infty$.
 \end{itemize}
 Důkazy všech těchto případů jsou identické důkazu
 \hyperref[prop:limita-a/0]{původního tvrzení}.
\end{remark}

Posledním základním tvrzením o limitách funkcí je vztah limit a uspořádání
reálných čísel, vlastně jakási varianta \myref{lemmatu}{lem:o-dvou-straznicich}
pro posloupnosti.

\begin{theorem}{O srovnání}{o-srovnani}
 Ať $a \in \R^{*}$ a $f,g,h$ jsou reálné funkce.
 \begin{enumerate}[label=(\alph*)]
  \item Pokud
   \[
    \lim_{x \to a} f(x) > \lim_{x \to a} g(x),
   \]
   pak existuje prstencové okolí bodu $a$, na němž $f > g$.
  \item Existuje-li prstencové okolí bodu $a$, na němž platí $f \leq g$, pak
   \[
    \lim_{x \to a} f(x) \leq \lim_{x \to a} g(x).
   \]
  \item Existuje-li prstencové okolí $a$, na němž $f \leq h \leq g$ a $\lim_{x
   \to a} f(x) = \lim_{x \to a} g(x) = A \in \R^{*}$, pak existujíc rovněž
   $\lim_{x \to a} h(x)$ jest rovna $A$.
 \end{enumerate}
\end{theorem}

\begin{thmproof}
 Položme $L_f \coloneqq \lim_{x \to a} f(x)$ a $L_g \coloneqq \lim_{x \to a}
 g(x)$.

 Dokážeme (a). Protože $L_f > L_g$, existuje $\varepsilon > 0$ takové, že $L_f -
 L_g > 2\varepsilon$. K tomuto $\varepsilon$ nalezneme z
 \hyperref[def:oboustranna-limita-funkce]{definice limity} $\delta_f > 0$ a
 $\delta_g > 0$ taková, že
 \begin{align*}
  \forall x \in R(a,\delta_f) &: f(x) \in B(L_f,\varepsilon),\\
  \forall x \in R(a,\delta_g)&: g(x) \in B(L_g,\varepsilon).
 \end{align*}
 To ovšem znamená, že pro $x \in R(a,\min(\delta_f,\delta_g))$ platí jak
 \[
  f(x) > L_f - \varepsilon,
 \]
 tak
 \[
  g(x) < L_g + \varepsilon,
 \]
 čili
 \[
  f(x) - g(x) > L_f - \varepsilon - L_g - \varepsilon = L_f - L_g - 2\varepsilon
  > 0,
 \]
 kterak chtiechom.

 Část (b) dokážeme sporem. Ať $L_f > L_g$. Podle (a) pak existuje prstencové
 okolí $R(a,\delta)$ bodu $a$, na němž $f > g$. Ovšem, podle předpokladu
 existuje rovněž okolí $R(a,\eta)$ bodu $a$, kde zase $f \leq g$. Vezmeme-li
 tudíž $x \in R(a,\min(\delta,\eta))$, pak $f(x) > g(x) \geq f(x)$, což je spor.

 V důkazu (c) rozlišíme dva případy. Položme $L \coloneqq L_f = L_g$ a ať
 nejprve $L \in \R$. Pro dané $\varepsilon>0$ existují $\delta_f,\delta_g>0$
 taková, že pro $x \in R(a,\delta_f)$ platí
 \[
  L - \varepsilon< f(x) < L + \varepsilon
 \]
 a pro $x \in R(a,\delta_g)$ zas
 \[
  L - \varepsilon < g(x) < L + \varepsilon.
 \]
 Z předpokladu existuje prstencové okolí $R(a,\eta)$, na němž $f \leq h \leq g$.
 Pročež, pro $x \in R(a,\min(\delta_f,\delta_g,\eta))$ máme
 \[
  L - \varepsilon < f(x) \leq h(x) \leq g(x) < L + \varepsilon,
 \]
 z čehož plyne $h(x) \in B(L,\varepsilon)$, neboli $\lim_{x \to a} h(x) = L$.

 Pro $L = \infty$ postupujeme jednodušeji, neboť stačí dolní odhad. K danému
 $\varepsilon>0$ nalezneme $\delta>0$ takové, že pro $x \in R(a,\delta)$ platí
 \[
  f(x) > \frac{1}{\varepsilon}.
 \]
 Pak pro $x \in R(a,\min(\delta,\eta))$ máme odhad
 \[
  \frac{1}{\varepsilon} < f(x) \leq h(x),
 \]
 čili $h(x) \in B(\infty,\varepsilon)$, což dokazuje rovnost $\lim_{x \to a}
 h(x) = \infty$. 

 Případ $L = -\infty$ se ošetří horním odhadem funkcí $g$.
\end{thmproof}

\begin{exercise}{}{}
 Spočtěte následující limity funkcí
 \begin{align*}
  &\lim_{x \to 1} \frac{3x^{4} - 4x^3 + 1}{(x-1)^2},\\
  &\lim_{x \to 0} \frac{3x + \frac{2}{x}}{x + \frac{4}{x}},\\
  &\lim_{x \to -\infty} \sqrt{x^2 + x} - x,\\
  &\lim_{x \to \infty} \frac{\sqrt{x-1}-2x}{x-7}.
 \end{align*}
\end{exercise}


\section{Spojité funkce}
\label{sec:spojite-funkce}

Vlastnost spojitosti funkce či zobrazení je zcela jistě tou nejdůležitější
především v topologii (disciplíně zpytující \uv{tvar} prostoru), kde se vlastně
s jinými zobrazeními než spojitými v obec nepracuje.

Intuitivně je zobrazení \emph{spojité} v moment, kdy zobrazuje souvislé části
prostoru na souvislé části prostoru. Souvislostí se zde myslí vlastnost
konkrétní podmnožiny prostoru (třeba $\R^{n}$), kdy z každého bodu do každého
jiného bodu existuje cesta (křivka v prostoru), která tuto podmnožinu neopustí
(\myref{obrázek}{fig:souvisla-podmnozina}).

\begin{figure}[ht]
 \centering
 \begin{tikzpicture}[scale=1.5]
  \begin{scope}
   \clip (5.3,5.2) rectangle (8.5,7.2);
   \coordinate (N) at (8.5,10);
   \coordinate (O) at (5.9,5.5);
   \coordinate (P) at (6.6,5.4);
   \coordinate (Q) at (7.6,6);
   \coordinate (R) at (8.4,6.3);
   \coordinate (S) at (8.1,7);
   \coordinate (T) at (7.4,6.9);
   \coordinate (U) at (6.4,6.6);
   \draw[pattern=dots,pattern color=black!30] (O) to [closed, curve through =
    {(O) (P) (Q) (R) (S) (T) (U)}] (O);

   \tkzDefPoints{6/6/x,8/6.7/y}
   \tkzDrawPoints[size=5,color=BrickRed](x,y)
   \draw[thick,color=BrickRed] (x) to [curve through={(7,5.9) (7.5,6.8)}] (y);
  \end{scope}
 \end{tikzpicture}
 \caption{Souvislá podmnožina $\R^2$.}
 \label{fig:souvisla-podmnozina}
\end{figure}

Spojité zobrazení lze tudíž definovat tím způsobem, že dva obrazy lze vždy
spojit křivkou, která neopouští obraz souvislé podmnožiny obsahující jejich
vzory. Jednodušeji, spojité zobrazení nesmí \uv{roztrhnout} souvislou podmnožinu
prostoru, i když do ní může například \uv{udělat díry}.

\begin{figure}[ht]
 \centering
 \begin{tikzpicture}[scale=1]
  \begin{scope}
   \clip (5.3,5.2) rectangle (8.5,7.2);
   \coordinate (N) at (8.5,10);
   \coordinate (O) at (5.9,5.5);
   \coordinate (P) at (6.6,5.4);
   \coordinate (Q) at (7.6,6);
   \coordinate (R) at (8.4,6.3);
   \coordinate (S) at (8.1,7);
   \coordinate (T) at (7.4,6.9);
   \coordinate (U) at (6.4,6.6);
   \draw[pattern=dots,pattern color=BrickRed!30] (O) to [closed, curve through =
    {(O) (P) (Q) (R) (S) (T) (U)}] (O);

   \tkzDefPoints{6/6/x,8/6.7/y}
   \tkzDrawPoints[size=5,color=BrickRed](x,y)
   \draw[thick,color=BrickRed] (x) to [curve through={(7,5.9) (7.5,6.8)}] (y);
   \tkzLabelPoint[above=1mm](x){$\clr{x}$}
   \tkzLabelPoint[below=1mm](y){$\clr{y}$}
  \end{scope}
  \draw[-latex,color=RoyalBlue,thick] (8.4,7.1) to [bend left=45]
   node[midway,yshift=-3mm,color=RoyalBlue]{$f$} (10,7.1);
  \begin{scope}
   \clip (9.8,4.6) rectangle (15.2,8.2);
   \coordinate (N) at (13,10);
   \coordinate (O) at (13.5,5.5);
   \coordinate (P) at (12,5.4);
   \coordinate (Q) at (11,6);
   \coordinate (R) at (10,6.3);
   \coordinate (S) at (12,7);
   \coordinate (T) at (14,8);
   \coordinate (U) at (15,6.6);
   \draw[pattern=dots,pattern color=RoyalBlue!30] (O) to [closed, curve through =
    {(O) (P) (Q) (R) (S) (T) (U)}] (O);

   \node[fill=white,draw,circle,minimum size=10mm] at (13, 6.5) {};
   
   \tkzDefPoints{11/6.5/x,14/6/y}
   \tkzDrawPoints[size=5,color=RoyalBlue](x,y)
   \draw[thick,color=RoyalBlue] (x) to [curve through={(12,6.6) (13,7.2)}] (y);
   \tkzLabelPoint[left=1mm](x){$\clb{f(x)}$}
   \tkzLabelPoint[below=1mm](y){$\clb{f(y)}$}
  \end{scope}
 \end{tikzpicture}
 \caption{Spojité zobrazení $\clb{f}: \R^2 \to \R^2$.}
 \label{fig:spojite-zobrazeni}
\end{figure}

V první dimenzi je situace pochopitelně výrazně jednodušší. Souvislou
podmnožinou $\R$ je \emph{interval}, a tedy spojité zobrazení je takové, které
zobrazuje interval na interval. Díry v intervalu zřejmě není možné dělat bez
téhož kompletního roztržení. Takto se však, primárně z~důvodův technických,
spojité zobrazení obyčejně nedefinuje a vlastnost zachování intervalu dlužno
dokázat.

Pojem \hyperref[def:oboustranna-limita-funkce]{limity funkce} umožňuje definovat
spojitou funkci jako tu, která se v každém bodě blíží ke své skutečné hodnotě,
tj. nedělá žádné \uv{skoky}.

\begin{definition}{Spojitá funkce}{spojita-funkce}
 Ať $a \in \R$. Řekneme, že reálná funkce $f$ je \emph{spojitá v bodě} $a$,
 pokud
 \[
  \lim_{x \to a} f(x) = f(a).
 \]
\end{definition}

\begin{remark}{Jednostranně spojitá funkce}{jednostranne-spojita-funkce}
 Obdobně \hyperref[def:spojita-funkce]{předchozí definici} tvrdíme, že funkce
 $f$ je \emph{spojitá zleva, resp. zprava, v bodě} $a \in \R$, pokud $\lim_{x
 \to a^{-}} f(x) = f(a)$, resp. $\lim_{x \to a^{+}} f(x) = f(a)$. Ona funkce je
 pak spojitá v bodě $a$, když je v $a$ spojitá zleva i zprava.
\end{remark}

\begin{figure}[ht]
 \centering
 \begin{subfigure}[b]{.49\textwidth}
  \centering
  \begin{tikzpicture}
   \tkzInit[xmin=-1,xmax=5,ymin=-1,ymax=3]
   \tkzDrawX[label=$\clr{x}$]
   \tkzDrawY[label=$\clb{f(x)}$]
   
   \coordinate (f0) at (0,0);
   \coordinate (f1) at (1,1);
   \coordinate (f2) at (2,2);
   \coordinate (f3) at (3,2);
   \tkzDefPoint(3,0){a}
   \tkzDefPoint(3,2){fa}
   \tkzDrawSegment[dashed](a,fa)
   \tkzDrawPoint[size=4,color=BrickRed](a)
   \tkzLabelPoint[below=1mm](a){$\clr{a}$}
   \tkzDrawPoint[size=4,color=RoyalBlue](fa)
   \draw[thick,color=RoyalBlue] (f0) to [curve through={(f1) (f2)}] (f3);

   \coordinate (f4) at (3,2.5);
   \coordinate (f5) at (4.3,1);
   \coordinate (f6) at (4.5,0);
   \coordinate (f7) at (5,-0.5);
   \draw[thick,color=RoyalBlue] (f4) to [curve through={(f5) (f6)}] (f7);
   \tkzDefPoint(3,2.5){fa2}
   \tkzDrawPoint[size=4,draw=RoyalBlue,thick,fill=white](fa2)
  \end{tikzpicture}
  \caption{Funkce $\clb{f}$ spojitá zleva (ale ne zprava) v bodě $\clr{a}$.}
 \end{subfigure}
 \begin{subfigure}[b]{.49\textwidth}
  \centering
  \begin{tikzpicture}
   \tkzInit[xmin=-1,xmax=5,ymin=-1,ymax=3]
   \tkzDrawX[label=$\clr{x}$]
   \tkzDrawY[label=$\clb{f(x)}$]
   
   \coordinate (f0) at (0,0);
   \coordinate (f1) at (1.5,1);
   \coordinate (f2) at (2,1.5);
   \coordinate (f3) at (3,1);
   \tkzDefPoint(3,0){a}
   \tkzDefPoint(3,1){fa}
   \tkzDrawSegment[dashed](a,fa)
   \tkzDrawPoint[size=4,color=BrickRed](a)
   \tkzLabelPoint[below=1mm](a){$\clr{a}$}
   \draw[thick,color=RoyalBlue] (f0) to [curve through={(f1) (f2)}] (f3);
   \tkzDrawPoint[size=4,draw=RoyalBlue,thick,fill=white](fa)

   \coordinate (f4) at (3,1.5);
   \coordinate (f5) at (3.7,2);
   \coordinate (f6) at (4.6,3);
   \coordinate (f7) at (5,3);
   \draw[thick,color=RoyalBlue] (f4) to [curve through={(f5) (f6)}] (f7);
   \tkzDefPoint(3,1.5){fa2}
   \tkzDrawPoint[size=4,color=RoyalBlue](fa2)
  \end{tikzpicture}
  \caption{Funkce $\clb{f}$ spojitá zprava (ale ne zleva) v bodě $\clr{a}$.}
 \end{subfigure}
 \caption{Jednostranná spojitost}
 \label{fig:jednostranna-spojitost}
\end{figure}

\begin{definition}{Funkce spojitá na intervalu}{funkce-spojita-na-intervalu}
 Ať $I \subseteq \R$ je interval. Řekneme, že reálná funkce $f$ je
 \emph{spojitá na} $I$, je-li
 \begin{itemize}
  \item spojitá v každém vnitřním bodě $I$,
  \item spojitá zprava v levém krajním bodě $I$, pokud tento leží v $I$ a
  \item spojitá zleva v pravém krajním bodě $I$, pokud tento leží v $I$.
 \end{itemize}
\end{definition}

Nyní se jmeme dokázat, že spojité funkce na intervalu (souvislé množině) mají
skutečně ony přirozené vlastnosti, jimiž jsme je popsali v úvodu do
\hyperref[sec:spojite-funkce]{této sekce}. Konkrétně dokážeme, že spojité funkce
zobrazují interval na interval. K tomu poslouží ještě jedno pomocné tvrzení,
známé též pod přespříliš honosným názvem \uv{Bolzanova věta o nabývání
mezihodnot}.

\begin{theorem}{Bolzanova}{bolzanova}
 Nechť $f$ je reálná funkce spojitá na $[a,b]$ a $f(a) < f(b)$. Potom pro každé
 $y \in (f(a),f(b))$ existuje $x \in (a,b)$ takové, že $f(x) = y$.
\end{theorem}
\begin{thmproof}
 Ať je $y \in (f(a),f(b))$ dáno. Označme
 \[
  M \coloneqq \{z \in [a,b] \mid f(z) < y\}.
 \]
 Ukážeme, že množina $M$ má konečné supremum. K tomu potřebujeme ověřit, že je
 neprázdná a shora omezená. Protože $f(a) < y$, jistě $a \in M$. Podobně,
 jelikož $y < f(b)$, je $b$ horní závorou $M$. Existuje tedy $\sup M$, které
 označíme $S$. Jistě platí $S \in (a,b)$. Ukážeme, že $f(S) = y$ vyloučením
 možností $f(S) < y$ a $f(S) > y$.

 Ať nejprve $f(S) < y$. Protože $f$ je z předpokladu spojitá (čili $\lim_{c \to
 S} f(c) = f(S) < y$), existuje $\varepsilon>0$ takové, že pro každé $c \in
 (S,S+\varepsilon)$ platí $f(c) < y$. To je ovšem spor s tím, že $S$ je horní
 závorou $M$. Nutně tedy $f(S) \geq y$.

 Ať nyní $f(S) > y$. Opět ze spojitosti $f$ nalezneme $\varepsilon>0$ takové, že
 pro $c \in (S-\varepsilon,S)$ platí $f(c) > y$. Dostáváme spor s tím, že $S$ je
 \textbf{nejmenší} horní závorou $M$.

 Celkem vedly obě ostré nerovnosti ke sporu, tudíž $f(S) = y$ a důkaz je hotov.
\end{thmproof}

\begin{corollary}{}{spojita-interval}
 Ať $f: I \to \R$ je funkce spojitá na intervalu $I \subseteq \R$. Pak $f(I)$ je
 interval.
\end{corollary}
\begin{corproof}
 Volme $y_1,y_2 \in f(I), y_1 < y_2$.

 Protože $y_1,y_2 \in f(I)$, existují $x_1,x_2 \in I$ taková, že $f(x_1) = y_1$
 a $f(x_2) = y_2$. Bez újmy na obecnosti předpokládejme, že $x_1 < x_2$. Potom z
 \hyperref[thm:bolzanova]{Bolzanovy věty} pro každé $y \in (f(x_1),f(x_2))$
 existuje $z \in (x_1,x_2)$ takové, že $f(z) = y$. Potom však z definice $y \in
 f(I)$, a tedy je $f(I)$ interval.
\end{corproof}


\section{Hlubší poznatky o limitě funkce}
\label{sec:hlubsi-poznatky-o-limite-funkce}

Kapitolu o limitách funkcí dovršíme několika -- v dalším textu zásadními --
tvrzeními. Některá z~nich se vížou na pojem
\hyperref[def:spojita-funkce]{spojitosti funkce}, některá nikoliv. Ukážeme si
rovněž souvislost limit funkcí s limitami posloupností; uzříme, že je to v
jistém širém smyslu týž koncept.

Nejprve se pozastavíme nad chováním limit funkcí vzhledem k jejich skládání. Je
vskutku velmi přirozené -- má-li funkce $g$ limitu $A$ v bodě $a$ a funkce $f$
limitu $B$ v bodě $A$, pak $f \circ g$ má limitu $B$ v bodě $a$, jak by jeden
čekal. Toto tvrzení má však své předpoklady; pro libovolné dvě funkce pravdivé
není.

\begin{theorem}{Limita složené funkce}{limita-slozene-funkce}
 Ať $a,A,B \in \R^{*}$ a $f,g$ jsou reálné funkce. Nechť navíc platí
 \[
  \lim_{x \to a} g(x) = A \quad \text{a} \quad \lim_{y \to A} f(y) = B.
 \]
 Je-li splněna \textbf{aspoň jedna} z podmínek:
 \begin{itemize}
  \item [(R)] existuje prstencové okolí $a$, na němž platí $g \neq A$;
  \item [(S)] funkce $f$ je spojitá v $A$,
 \end{itemize}
 pak
 \[
  \lim_{x \to a} (f \circ g)(x) = B.
 \]
\end{theorem}
\begin{thmproof}
 Důkaz tvrzení není příliš obtížný, ovšem poněkud technický a alfabeticky
 tíživý. Potřebujeme ukázat, že pro libovolné $\varepsilon>0$ existuje
 $\delta>0$ takové, že všechna $x \in R(a,\delta)$ splňují $f(g(x)) \in
 B(B,\varepsilon)$. Nechť je tedy $\varepsilon>0$ dáno.

 Předpokládejme, že platí (R) a máme $\eta>0$, pro něž $g(x) \neq A$, kdykoli $x
 \in R(a,\eta)$. Ježto platí $\lim_{y \to A} f(y) = B$, existuje k danému
 $\varepsilon>0$ číslo $\delta_f>0$ takové, že pro $y \in R(A,\delta_f)$ platí
 $f(y) \in B(B,\varepsilon)$. K tomuto $\delta_f>0$ pak existuje $\delta_g>0$
 takové, že pro $x \in R(a,\delta_g)$ platí $g(x) \in B(A,\delta_f)$. Volíme-li
 nyní $\delta \coloneqq \min(\eta,\delta_g)$, pak pro $x \in R(a,\delta)$ platí
 \[
  g(x) \in B(A,\delta_f) \quad \text{i} \quad g(x) \neq A,
 \]
 kteréžtě dvě podmínce dávajíta celkem $g(x) \in B(A,\delta_f) \setminus \{A\} =
 R(a,\delta_f)$. Potom však pro $x \in R(a,\delta)$ platí $g(x) \in
 R(A,\delta_f)$, tudíž $f(g(x)) \in B(B,\varepsilon)$, jak jsme chtěli.
  
 Ať naopak platí (S). Pak smíme pro spojitost $f$ v $A$ ve výroku
 \[
 \forall y \in R(A,\delta_f): f(y) \in B(B,\varepsilon)
 \]
 místo prstencového okolí $R(A,\delta_f)$ brát plné okolí $B(A,\delta_f)$, neboť
 $f(A) = \lim_{y \to A} f(y) = B \in B(B,\varepsilon)$. Protože pro $x \in
 R(a,\delta_g)$ však platí $g(x) \in B(A,\delta_f)$, máme rovněž $f(g(x)) \in
 B(B,\varepsilon)$, čímž je důkaz hotov.  
\end{thmproof}

\begin{warning}{}{}
 Platnost aspoň jedné z podmínek (R),(S) v
 \hyperref[thm:limita-slozene-funkce]{předchozí větě} je nezanedbatelná.

 Vezměme $f(y) = |\sign y|$ a $g(x) = 0$. Potom platí
 \[
  \lim_{y \to 0} f(y) = 1 \quad \text{a} \quad \lim_{x \to 0} g(x) = 0,
 \]
 ale $\lim_{x \to 0} (f \circ g)(x) = 0 \neq 1$. Závěr
 \hyperref[thm:limita-slozene-funkce]{předchozí věty} je v tomto případě
 neplatný pro to, že neexistuje okolí $a$, na němž $g \neq 0$ (tj. neplatí (R)),
 ani není $f$ spojitá v $0$ (tj. neplatí (S)).
\end{warning}

\begin{figure}[ht]
 \centering
 \begin{tikzpicture}[scale=1.75]
  \def\dg{0.8}
  \def\df{0.7}
  \def\eps{1}

  \tkzDefPoints{0/0/a,2/1/A,4/2/B}
  \tkzDrawPoint[size=4,color=BrickRed](a)
  \tkzDrawPoint[size=4,color=RoyalBlue](A)
  \tkzDrawPoint[size=4,color=ForestGreen](B)

  \tkzLabelPoint[below=1mm](a){$\clr{a}$}
  \tkzLabelPoint[below=1mm](A){$\clb{A}$}
  \tkzLabelPoint[below=1mm](B){$\clg{B}$}
  
  \tkzDefCircle[R](a,\dg) \tkzGetPoint{ac}
  \tkzDrawCircle[thick,color=BrickRed](a,ac)
  \tkzDefCircle[R](A,\df) \tkzGetPoint{Ac}
  \tkzDrawCircle[thick,color=RoyalBlue](A,Ac)
  \tkzDefCircle[R](B,\dg) \tkzGetPoint{Bc}
  \tkzDrawCircle[thick,color=ForestGreen](B,Bc)

  \tkzDefPointOnCircle[through = center a angle 120 point ac]
  \tkzGetPoint{ad}
  \tkzDrawPoint[size=4,color=BrickRed](ad)
  \tkzDrawSegment[color=BrickRed,dashed,thick](a,ad)
  \tkzLabelSegment[below left](a,ad){$\clr{\delta_g}$}

  \tkzDefPointOnCircle[through = center A angle 120 point Ac]
  \tkzGetPoint{Ad}
  \tkzDrawPoint[size=4,color=RoyalBlue](Ad)
  \tkzDrawSegment[color=RoyalBlue,dashed,thick](A,Ad)
  \tkzLabelSegment[below left](A,Ad){$\clb{\delta_f}$}

  \tkzDefPointOnCircle[through = center B angle 120 point Bc]
  \tkzGetPoint{Bd}
  \tkzDrawPoint[size=4,color=ForestGreen](Bd)
  \tkzDrawSegment[color=ForestGreen,dashed,thick](B,Bd)
  \tkzLabelSegment[below left](B,Bd){$\clg{\varepsilon}$}

  \tkzDefPointOnCircle[through = center a angle 80 point ac]
  \tkzGetPoint{ae}
  \tkzDefPointOnCircle[through = center A angle 150 point Ac]
  \tkzGetPoint{Ae}
  \tkzDefPointOnCircle[through = center A angle 80 point Ac]
  \tkzGetPoint{Af}
  \tkzDefPointOnCircle[through = center B angle 150 point Bc]
  \tkzGetPoint{Be}

  \draw[-latex,shorten <=5pt,shorten >=5pt] (ae) to [bend left=45]
   node[midway,yshift=2mm,xshift=-2mm] {$g$} (Ae);
  \draw[-latex,shorten <=5pt,shorten >=5pt] (Af) to [bend left=45]
   node[midway,yshift=2mm,xshift=-2mm] {$f$} (Be);

  \tkzDefPoints{0.3/0.2/x,1.8/0.8/gx,4.3/2.5/fgx}
  \tkzDrawPoints[size=4,color=black](x,gx,fgx)
  \tkzLabelPoint[below=1mm](x){$x$}
  \tkzLabelPoint[below=1mm](gx){$g(x)$}
  \tkzLabelPoint[below=1mm](fgx){$f(g(x))$}
 \end{tikzpicture}
 \caption{Důkaz \hyperref[thm:limita-slozene-funkce]{věty o limitě složené
 funkce}.}
 \label{fig:limita-slozene-funkce}
\end{figure}

Pokračujeme vztahem limit posloupností a limit funkcí. Ukazuje se, že limitu
funkce možno v principu nahradit limitou posloupnosti jejích funkčních hodnot.
Toto tvrzení je zvlášť užitečné při důkazu \emph{neexistence} oné limity. Stačí
totiž najít dvě posloupnosti, funkční hodnoty jejichž členů se blíží k rozdílným
číslům či kterákolivěk z nich neexistuje.

\begin{theorem}{Heineho}{heineho}
 Ať $f$ je reálná funkce a $a,L \in \R^{*}$. Pak jsou následující výroky
 ekvivalentní.
 \begin{enumerate}
  \item $\lim_{x \to a} f(x) = L$.
  \item Pro \textbf{každou} posloupnost $x:\N \to \R$ takovou, že
  \begin{itemize}
   \item $\lim_{n \to \infty} x_n = a$;
   \item $x_n \neq a$ pro každé $n \in \N$,
  \end{itemize}
  platí $\lim_{n \to \infty} f(x_n) = L$.
 \end{enumerate}
\end{theorem}
\begin{thmproof}
 Dokážeme nejprve $(1) \Rightarrow (2)$. Ať je dáno $\varepsilon>0$. Potřebujeme
 najít $n_0 \in \N$ takové, že pro $n \geq n_0$ je $|f(x_n) - L|<\varepsilon$.
 Uvědomme si, že poslední výrok je ekvivalentní
 \[
  L -\varepsilon< f(x_n)<L+\varepsilon,
 \]
 neboli $f(x_n) \in B(L,\varepsilon)$. Z toho, že $\lim_{x \to a} f(x) = L$,
 existuje $\delta>0$ takové, že všechna $x \in R(a,\delta)$ splňují $f(x) \in
 B(L,\varepsilon)$. Ježto $\lim_{n \to \infty} x_n = a$, existuje k tomuto
 $\delta>0$ index $n_0 \in \N$ takový, že
 \[
 \forall n \geq n_0: |x_n-a|<\delta,
 \]
 neboli $x_n \in B(a,\delta)$. Předpokládáme ovšem, že $x_n \neq a$ pro všechna
 $n \in \N$, a tedy $x_n \in R(a,\delta)$, kdykoli $n \geq n_0$. Potom ale pro
 $n \geq n_0$ rovněž $f(x_n) \in B(L,\varepsilon)$, čili $\lim_{n \to \infty}
 f(x_n) = L$.

 Místo přímého důkazu $(2) \Rightarrow (1)$ (který je zdlouhavý), dokážeme $\neg
 (1) \Rightarrow \neg (2)$. Předpokládejme tedy, že $\lim_{x \to a} f(x)$ buď
 neexistuje, nebo není rovna $L$. Chceme najít posloupnost $x:\N \to \R$, která
 konverguje k $a$, žádný její člen není roven $a$, ale přesto $\lim_{n \to
 \infty} f(x_n)$ opět buď neexistuje, nebo není rovna $L$.

 Uvědomíme si nejprve přesně, jak zní \textbf{negace} výroku $\lim_{x \to a}
 f(x) = L$. Tvrdíme, že
 \[
 \exists \varepsilon>0 \; \forall \delta>0 \; \exists x \in R(a,\delta): f(x)
 \text{ není definováno} \vee f(x) \notin B(L,\varepsilon).
 \]
 Nalezněme tedy ono $\varepsilon>0$ z výroku výše. Pak pro každé $n \in \N$
 existuje $x_n \in R(a,1 / n)$ pro které $f(x_n)$ není definováno, nebo neleží v
 $B(L,\varepsilon)$. Vlastně jsme ve výroku výše položili $\delta \coloneqq 1 /
 n$ postupně pro každé $n \in \N$. Sestrojili jsme pročež posloupnost $x_n$
 takovou, že $\lim_{n \to \infty} x_n = a$ a $x_n \neq a$ pro všechna $n \in
 \N$. Nicméně, jak jsme psali výše, pro každé $x_n$ buď $f(x_n)$ není
 definováno, nebo neleží v $B(L,\varepsilon)$. To ovšem znamená, že $\lim_{n \to
 \infty} f(x_n)$ buď neexistuje, nebo není rovna $L$. Platí tudíž předpoklady v
 (2), ale nikoli závěr v (2), tj. platí $\neg (2)$.

 Tím je důkaz hotov.
\end{thmproof}

\begin{figure}[ht]
 \centering
 \begin{tikzpicture}
  \tkzInit[xmin=-1,xmax=5,ymin=-1,ymax=3]
  \tkzDrawX[label=$\clr{x}$]
  \tkzDrawY[label=$\clb{f(x)}$]
  
  \coordinate (f0) at (0,0);
  \coordinate (f1) at (1,1);
  \coordinate (f2) at (2,2);
  \coordinate (f3) at (3,2);
  \coordinate (f4) at (4,1);
  \coordinate (f5) at (5,0);
  \tkzDefPoint(3,0){a}
  \tkzDefPoint(3,2){fa}
  \tkzDefPoint(0,2){f}
  \foreach \n/\d in
  {1/1.9,2/2.3,3/2.4,4/2.5,5/2.65,6/2.7,7/2.85,8/2.88,9/2.91,10/2.94,11/2.96,12/2.98}
  {
   \tkzDefPoint(\d,0){x\n}
   \tkzDrawPoint[color=black](x\n)
  }
  \tkzLabelPoint[below](x1){\footnotesize $x_1$}
  \tkzLabelPoint[below](x2){\footnotesize $x_2$}
  \tkzLabelPoint[below](x6){\footnotesize $\dots$}

  \tkzDrawSegment[dashed](a,fa)
  \tkzDrawPoint[size=4,color=BrickRed](a)
  \tkzLabelPoint[below=1mm](a){$\clr{a}$}
  \draw[thick,color=RoyalBlue] (f0) to [curve through={(f1) (f2) (f3) (f4)}]
   (f5);
  \tkzDrawSegment[dashed](f,fa)
  \tkzDrawPoint[size=4,color=RoyalBlue](f)
  \tkzLabelPoint[left=1mm](f){$L$}
  \tkzDrawPoint[size=4,color=RoyalBlue](fa)

  \tkzDefPoint(1.9,1.95){fx1}
  \tkzDrawPoint[color=black](fx1)
  \tkzDefPoint(2.3,2.09){fx2}
  \tkzDrawPoint[color=black](fx2)
  \tkzDefPoint(2.4,2.11){fx3}
  \tkzDrawPoint[color=black](fx3)
  \tkzDefPoint(2.5,2.11){fx4}
  \tkzDrawPoint[color=black](fx4)
  \tkzDefPoint(2.65,2.1){fx5}
  \tkzDrawPoint[color=black](fx5)
  \tkzDefPoint(2.7,2.09){fx6}
  \tkzDrawPoint[color=black](fx6)
  \tkzDefPoint(2.85,2.06){fx7}
  \tkzDrawPoint[color=black](fx7)
  \tkzDefPoint(2.88,2.05){fx8}
  \tkzDrawPoint[color=black](fx8)

  \tkzLabelPoint[above](fx1){\footnotesize $f(x_1)$}
  \tkzLabelPoint[below](fx2){\footnotesize $f(x_2)$}
 \end{tikzpicture}
 \caption{Tvrzení (2) v \hyperref[thm:heineho]{Heineho větě}.}
 \label{fig:heineho-veta}
\end{figure}

\begin{corollary}{Heineho věta pro spojitost}{heineho-veta-pro-spojitost}
 Ať $a \in \R^{*}$ a $f$ je reálná funkce. Následující tvrzení jsou
 ekvivalentní.
 \begin{enumerate}
  \item Funkce $f$ je spojitá v bodě $a$.
  \item Pro každou posloupnost $x:\N \to \R$ s limitou $\lim_{n \to \infty} x_n
   = a$ platí $\lim_{n \to \infty} f(x_n) = f(a)$.
 \end{enumerate}
\end{corollary}
\begin{corproof}
 Zkrátka použijeme \hyperref[thm:heineho]{Heineho} větu pro $L = f(a)$.
 Poznamenejme pouze, že podmínka $x_n \neq a$ pro každé $n \in \N$ je zde
 zbytečná, neboť $f$ je z předpokladu spojitá v $a$.
\end{corproof}

\begin{example}{}{}
 V zájmu nabytí představy, jak se \hyperref[thm:heineho]{Heineho} věta používá
 pro důkaz neexistence limity funkce, budeme drze předpokládat, že čtenáři byli
 již seznámeni s funkcí $\mathrm{sin}:\R \to [-1,1]$. Její formální debut dlí v
 kapitole o elementárních funkcích. %TODO odkaz

 Ukážeme, že limita $\lim_{x \to \infty} \sin x$ neexistuje. Volme posloupnosti
 \[
  x_n \coloneqq 2\pi n \quad \text{a} \quad y_n \coloneqq 2\pi n + \pi / 2.
 \]
 Pak platí $\lim_{n \to \infty} x_n = \lim_{n \to \infty} y_n = \infty$ i zřejmě
 $x_n \neq \infty$ a $y_n \neq \infty$ pro každé $n \in \N$. Jsou proto splněny
 předpoklady tvrzení (2) z \hyperref[thm:heineho]{Heineho věty}. Platí však
 \begin{align*}
  \lim_{n \to \infty} \sin(x_n) &= \lim_{n \to \infty} \sin(2\pi n) = 0,\\
  \lim_{n \to \infty} \sin(y_n) &= \lim_{n \to \infty} \sin(2\pi n + \pi / 2) =
  1,
 \end{align*}
 a tedy jsme našli dvě posloupnosti, funkční hodnoty jejichž členů konvergují k
 různým číslům. Podle \hyperref[thm:heineho]{Heineho věty} $\lim_{x \to \infty}
 \sin x$ neexistuje.
\end{example}

Podobně jako monotónní posloupnosti mají vždy limitu (vizte
\myref{lemma}{lem:limita-monotonni-posloupnosti}), stejně tak mono\-tónní funkce
ji vždy mají. Tento přirozený pojem briskně představíme. Řčeme, že reálná funkce
$f$ je \emph{monotónní na intervalu} $I \subseteq \R$, když
\begin{itemize}
 \item[$(<)$] je $f$ \emph{rostoucí}, tj. $\forall x,y \in I: x<y \Rightarrow
  f(x)<f(y)$,
 \item[$(>)$] je $f$ \emph{klesající}, tj. $ \forall x,y \in I: x < y
  \Rightarrow f(x) > f(y)$,
 \item[$(\leq)$] je $f$ \emph{neklesající}, tj. $ \forall x,y \in I:x<y
  \Rightarrow f(x) \leq f(y)$ nebo
 \item[$(\geq)$] je $f$ \emph{nerostoucí}, tj. $ \forall x,y \in I:x<y
  \Rightarrow f(x) \geq f(y)$.
\end{itemize}
\begin{theorem}{Limita monotónní funkce}{limita-monotonni-funkce}
 Ať $(a,b) \subseteq \R^{*}$ a $f$ je monotónní na $(a,b)$. Potom,
 \begin{enumerate}
  \item jsouc $f$ rostoucí nebo neklesající má limity
  \[
   \lim_{x \to a^{+}} f(x) = \inf f((a,b)) \quad \text{a} \quad \lim_{x \to
   b^{-}} f(x) = \sup f((a,b));
  \]
 \item jsouc $f$ klesající nebo nerostoucí má limity
  \[
   \lim_{x \to a^{+}} f(x) = \sup f((a,b)) \quad \text{a} \quad \lim_{x \to
   b^{-}} f(x) = \inf f((a,b)).
  \]
 \end{enumerate}
\end{theorem}
\begin{thmproof}
 Dokážeme pouze část (1), důkaz (2) je totožný.

 Budeme nejprve předpokládat, že $f$ je zdola omezená a označíme $m \coloneqq
 \inf f((a,b)) \in \R$. Ať je dáno $\varepsilon>0$. Z definice infima nalezneme
 $y \in f((a,b))$ takové, že $y < m + \varepsilon$. Jelikož $y \in f((a,b))$,
 existuje $x \in (a,b)$, že $f(x) = y$. Z toho, že $f$ je neklesající či
 rostoucí plyne, že pro každé $z \in (a,x)$ je $f(z) \leq f(x) = y$.

 Nalezneme $\delta>0$ takové, že $R_+(a,\delta) = (a,a+\delta) \subseteq (a,x)$.
 Potom ale pro $z \in R_+(a,\delta)$ platí $f(z) \leq y < m+\varepsilon$. Ježto
 nerovnost $f(z) > m-\varepsilon$ je zřejmá ($m$ je dolní závora $f$), máme
 celkem pro $z \in R_+(a,\delta)$
 \[
  m -\varepsilon < f(z) < m + \varepsilon,
 \]
 čili $f(z) \in B(m,\varepsilon)$, jak bylo dokázati.

 Ať nyní $f$ není zdola omezená na $(a,b)$. Pak $\inf f(a,b) = -\infty$ a pro
 každé $\varepsilon>0$ nalezneme $z \in (a,b)$, pro nějž $f(z) < -1 /
 \varepsilon$. To ovšem z definice znamená, že $\lim_{x \to a^{+}} f(x) =
 -\infty = \inf f((a,b))$.

 Důkaz faktu, že $\lim_{x \to b^{-}} f(x) = \sup f((a,b))$ lze vést obdobně.
\end{thmproof}

\begin{exercise}{}{}
 Dokažte bod (2) ve \myref{větě}{thm:limita-monotonni-funkce}.
\end{exercise}

\subsection{Extrémy funkce}
\label{ssec:extremy-funkce}

V mnoha matematických i externích disciplínách jeden často hledá při studiu
reálných funkcí body, v nichž je hodnota funkce největší či nejmenší. Obecně
jsou maximalizační a minimalizační problémy jedny z nejčastěji řešených. Tyto
problémy vedou přímo na výpočet tzv. \emph{derivací} reálných funkcí, jsoucích
dychtivým čtenářům představeny v následující kapitole. Zde pouze definujeme
lokální a globální extrémy funkcí a ukážeme, že spojité funkce na uzavřených
intervalech nutně na týchž nabývají svých nejmenších i největších hodnot.
%TODO odkaz

\begin{definition}{Lokální a globální extrém}{lokalni-a-globalni-extrem}
 Ať $f:M \to \R$ je reálná funkce a $X \subseteq M$. Řekneme, že funkce $f$ v
 bodě $x \in X$ nabývá
 \begin{itemize}
  \item \emph{globálního maxima} na $X$, když pro každé $y \in X$ platí
   $f(y) \leq f(x)$;
  \item \emph{globálního minima} na $X$, když pro každé $y \in X$ platí $f(y)
   \geq f(x)$;
  \item \emph{lokálního maxima} na $X$, když existuje okolí bodu $x$, na němž
   platí $f \leq f(x)$;
  \item \emph{lokálního minima} na $X$, když existuje okolí bodu $x$, na němž
   platí $f \geq f(x)$.
 \end{itemize}
 Souhrnně přezdíváme globálnímu minimu a maximu \emph{globální extrém} a
 lokálnímu maximu a minimu \emph{lokální extrém}.
\end{definition}

\begin{figure}[ht]
 \centering
 \begin{tikzpicture}
  \tkzInit[xmin=-3,xmax=3,ymin=-3,ymax=3]
  \tkzDrawX[-latex]
  \tkzDrawY[-latex]

  \draw[thick,domain=-2.3:3,color=Fuchsia] plot [smooth] (\x, {0.5 * (\x*\x*\x
   - \x*\x - 4*\x + 2)}) node[right] {$\clm{f}$};
  \tkzDefPoints{-2.2/0/a,2.8/0/b}
  \tkzDrawPoints[size=4,color=black](a,b)
  \tkzLabelPoint[above](a){$a$}
  \tkzLabelPoint[below](b){$b$}

  \tkzDefPoints{-2.2/-2.344/fa,2.8/2.456/fb}
  \tkzDrawSegments[dashed,thick](a,fa b,fb)
  \tkzDrawPoints[size=6,color=BrickRed](fa,fb)

  \tkzDefPoints{-0.867/2.032/fM,1.535/-1.44/fm}
  \tkzDefPoints{-0.867/0/M,1.535/0/m}
  \tkzDrawPoints[size=4,color=black](M,m)
  \tkzLabelPoint[below](M){$M$}
  \tkzLabelPoint[above](m){$m$}
  \tkzDrawSegments[dashed,thick](M,fM m,fm)
  \tkzDrawPoints[size=6,color=RoyalBlue](fM,fm)
 \end{tikzpicture}
 \caption{Lokální a globální extrémy funkce $\clm{f}$ na $[a,b]$. \clb{Lokálních
  extrémů} nabývá $\clm{f}$ v bodech $m$ a $M$ a \clr{globálních extrémů} v
  bodech $a$ a $b$.}
 \label{fig:extremy}
\end{figure}

Jak jsme již zmínili v textu před definicí, spojité funkce nabývají na
uzavřených intervalech globálních extrémů vždy.

\begin{theorem}{Extrémy spojité funkce}{extremy-spojite-funkce}
 Ať $f:[a,b] \to \R$ je spojitá funkce. Pak $f$ nabývá globálního minima a
 maxima na $[a,b]$.
\end{theorem}
\begin{thmproof}
 Dokážeme, že $f$ nabývá na $[a,b]$ globálního maxima. Pro globální minimum lze
 důkaz vést obdobně.

 Využijeme \myref{důsledku}{cor:heineho-veta-pro-spojitost}. Nalezneme
 posloupnost, která se uvnitř intervalu $f([a,b])$ blíží k~supremu funkce $f$ na
 $[a,b]$ a ukážeme, že vzory členů této posloupnosti posloupnosti z~intervalu
 $[a,b]$ se blíží k bodu, kde $f$ nabývá maxima.

 Položme tedy $S \coloneqq \sup f([a,b])$. Sestrojíme posloupnost $y: \N \to
 f([a,b])$, která konverguje k $S$. Je-li $S=\infty$, stačí položit třeba $y_n =
 n$ pro všechna $n \in \N$. Předpokládejme, že $S \in \R$. Z~definice suprema
 existuje pro každé $n \in \N$ prvek $z \in f([a,b])$ takový, že $S-1 / n < z
 \leq S$. Položíme $y_n \coloneqq z$. Tím máme posloupnost $y_n$ s $\lim_{n \to
 \infty} y_n = S$.

 Z definice $f([a,b])$ nalezneme pro každé $y_n$ číslo $x_n \in [a,b]$, pro něž
 $f(x_n) = y_n$. Posloupnost $x_n$ je omezená (leží uvnitř $[a,b]$), a tedy z
 \hyperref[thm:bolzano-weierstrass]{Bolzanovy-Weierstraßovy} věty existuje její
 konvergentní podposloupnost. Můžeme pročež bez újmy na obecnosti předpokládat,
 že sama $x_n$ konverguje. Položme $M \coloneqq \lim_{n \to \infty} x_n$. Ježto
 $a \leq x_n \leq b$ pro všechna $n \in \N$, z
 \myref{lemmatu}{lem:o-dvou-straznicich} plyne, že $M \in [a,b]$. Konečně, $f$
 je z předpokladu spojitá, a tedy z
 \myref{důsledku}{cor:heineho-veta-pro-spojitost} 
 \[
  f(M) = \lim_{n \to \infty} f(x_n) = \lim_{n \to \infty} y_n = S,
 \]
 čili $f$ nabývá v bodě $M$ maxima na $[a,b]$.
\end{thmproof}

\begin{corollary}{}{}
 Je-li funkce $f$ spojitá na $[a,b]$, pak je tamže omezená.
\end{corollary}
\begin{corproof}
 Z \myref{věty}{thm:extremy-spojite-funkce} plyne, že $f$ nabývá na $[a,b]$
 minima $s$ a maxima $S$. Potom ale pro každé $x \in [a,b]$ platí
 \[
  s \leq f(x) \leq S,
 \]
 čili $f$ je na $[a,b]$ omezená.
\end{corproof}


\section{Pár příkladů na konec}
\label{sec:par-prikladu-na-konec}

Účelem této \uv{přiložené} sekce je ukázat na dvou zajímavých příkladech obvyklé
metody práce s~limitami funkcí. Doufáme, že čtenářům dobře pomůže uchápění
tohoto tématu, snad náročnějšího k vnětí než limity posloupností a součty řad.

\begin{example}{}{periodicka-slozeno-rostouci}
 Ať $g$ je rostoucí a spojitá funkce na $[1,\infty)$ s $\lim_{x \to \infty} g(x)
 = \infty$ a $f$ je \textbf{nekonstantní periodická} funkce na $\R$. Pak
 $\lim_{x \to \infty} (f \circ g)(x)$ neexistuje.

 Pro důkaz neexistence limity máme (pochopitelně kromě samotné
 \hyperref[def:oboustranna-limita-funkce]{definice}) zatím pouze dva nástroje --
 \hyperref[def:jednostranna-limita-funkce]{jednostranné limity} a
 \hyperref[thm:heineho]{Heineho větu}. Protože limitním bodem je $\infty$,
 použití jednostranných limit není možné. Zkusíme tedy
 \hyperref[thm:heineho]{Heineho větu}.

 Nejprve si uvědomíme, že z \myref{důsledku}{cor:spojita-interval} je
 $g([1,\infty))$ je interval. Tento navíc není shora omezen, bo $\lim_{x \to
 \infty} g(x) = \infty$. Na oba tyto fakty se budeme vícekrát odvolávat. Označme
 rovněž písmenem $p > 0$ periodu funkce $f$.

 Položme nyní $x_0 \coloneqq 1$ a označme $y_0 \coloneqq g(x_0)$, $A \coloneqq
 f(y_0)$. Induktivně sestrojíme posloupnosti $x:\N \to [1,\infty)$ a $y:\N \to
 \R$. Předpokládejme, že jsou dány členy $x_0,\ldots,x_k$ a $y_0,\ldots,y_k$,
 kde $y_i = g(x_i)$ pro každé $i \leq k$. Položíme $y_{k+1} \coloneqq y_k + p$.
 Potom $f(y_{k+1}) = f(y_k)$. Protože $g([1,\infty)) = [y_0,\infty)$ a $y_{k+1}
 > y_0$, nalezneme $x_{k+1} \in [1,\infty)$ takové, že $g(x_{k+1}) = y_{k+1}$.
 Jelikož $g$ je rostoucí a
 \[
  g(x_{k+1}) = y_{k+1} > y_k = g(x_k),
 \]
 rovněž $x_{k+1} > x_k$. Celkem máme $x_{n+1} > x_n$ pro každé $n \in \N$, čili
 $\lim_{n \to \infty} x_n = \infty$. Rovněž
 \[
  \lim_{n \to \infty} (f \circ g)(x_n) = \lim_{n \to \infty} f(y_n) = \lim_{n
  \to \infty} A = A,
 \]
 neboť členy posloupnosti $y$ jsou od sebe vzdáleny o periodu $p$ funkce $f$.

 Konečně, nalezneme jinou posloupnost $\tilde{x}:\N \to [1,\infty)$ takovou, že
 $\lim_{n \to \infty} (f \circ g)(\tilde{x}_n) \neq A$, čímž završíme důkaz.
 Volme libovolné $0 < \varepsilon < p$. K tomuto $\varepsilon$ nalezneme
 $\delta>0$ takové, že
 \[
  |g(x_0 + \delta) - g(x_0)| < \varepsilon.
 \]
 Toto $\delta$ vskutku existuje pro to, že $g$ je rostoucí -- tudíž $g(x_0 +
 \delta) > g(x_0)$ -- a spojitá -- tudíž dvě různé funkční hodnoty lze volit
 nekonečně blízké.

 Položíme $\tilde{x}_0 \coloneqq x_0 + \delta$ a $\tilde{y}_0 \coloneqq
 g(\tilde{x}_0)$. Potom $f(\tilde{y}_0) = B \neq A$, ježto $\tilde{y}_0 \in
 (y_0, y_0 + p)$. Podobně jako dříve sestrojíme posloupnost $\tilde{x}_n$
 takovou, že $\lim_{n \to \infty} \tilde{x}_n = \infty$ a $(f \circ
 g)(\tilde{x}_n) = B$ pro každé $n \in \N$.

 Potom ale platí
 \[
  \lim_{n \to \infty} (f \circ g)(\tilde{x}_n) = B \neq A = \lim_{n \to \infty}
  (f \circ g)(x_n),
 \]
 čili z \hyperref[thm:heineho]{Heineho věty} $\lim_{x \to \infty} (f \circ
 g)(x)$ neexistuje.
\end{example}

\begin{figure}[ht]
 \centering
 \begin{tikzpicture}
  \tkzInit[xmin=-1,xmax=8,ymin=-1.5,ymax=1.5]
  \tkzDrawX[label=]
  \tkzDrawY[label=]

  \draw[color=Fuchsia,thick,smooth,domain=0:7.5,samples=1000] plot
   (\x,{abs(sin(deg(\x)))}) node[above] {$\clm{f}$};
  \tkzDefPoints{1.5708/0/y0,4.7124/0/y1}
  \tkzDefPoints{2.6179/0/yt0,5.7596/0/yt1}
  \tkzDefPoints{1.5708/1/A1,4.7124/1/A2}
  \tkzDefPoints{2.6179/0.5/B1,5.7596/0.5/B2}

  \tkzDrawSegments[dashed](y0,A1 y1,A2 yt0,B1 yt1,B2)

  \tkzDrawPoints[size=4,color=BrickRed](y0,y1)
  \tkzDrawPoints[size=4,color=RoyalBlue](yt0,yt1)

  \tkzLabelPoint[below=1mm](y0){$\clr{y_0}$}
  \tkzLabelPoint[below=1mm](y1){$\clr{y_1}$}
  \tkzLabelPoint[below=1mm](yt0){$\clb{\tilde{y}_0}$}
  \tkzLabelPoint[below=1mm](yt1){$\clb{\tilde{y}_1}$}

  \tkzDrawPoints[size=4,color=Fuchsia](A1,A2,B1,B2)
  \tkzLabelPoint[above=1mm](A1){$\clm{A}$}
  \tkzLabelPoint[above=1mm](A2){$\clm{A}$}
  \tkzLabelPoint[above=1mm](B1){$\clm{B}$}
  \tkzLabelPoint[above=1mm](B2){$\clm{B}$}

  \draw[decorate,decoration={brace,raise=6mm,mirror,amplitude=10pt}] (y0) --
   (y1) node[pos=0.5,below=9mm] {$p$};
  \draw[decorate,decoration={brace,raise=1.5mm,amplitude=5pt}] (y0) -- (yt0)
   node[pos=0.5,above=2.5mm] {$\varepsilon$};
 \end{tikzpicture}

 \caption{Posloupnosti $\clr{y}$ a $\clb{\tilde{y}}$ z
 \myref{příkladu}{exam:periodicka-slozeno-rostouci}.}
 \label{fig:periodicka-slozeno-rostouci}
\end{figure}

\begin{example}{Singularity funkce}{singularity-funkce}
 Ať $a \in M$ a $f:M \to \R$ je reálná funkce definovaná aspoň na prstencovém
 okolí bodu $a$. Předpokládejme, že $f$ není spojitá v $a$. Pak řekneme, že $f$
 má v $a$
 \begin{itemize}
  \item \textbf{odstranitelnou singularitu}, když existuje konečná
   $\lim_{x \to a} f(x)$. V tomto případě můžeme dodefinovat $f(a) \coloneqq
   \lim_{x \to a} f(x)$;
  \item \textbf{pól}, když existují $\lim_{x \to a^{+}} f(x)$ a
   $\lim_{x \to a^{-}} f(x)$, ale nejsou si rovny;
  \item \textbf{neodstranitelnou singularitu}, když aspoň jedna z jednostranných
   limit $f$ v bodě $a$ neexistuje.
 \end{itemize}
 Singularity funkcí tvoří důležitou část komplexní analýzy, kde poskytují obraz
 zejména o stabilitě fyzikálních systému popsaných těmito funkcemi.
 \emph{Odstranitelné singularity} jsou velmi stabilní a většinou způsobeny pouze
 chybami v měření. \emph{Póly} jsou stabilní při vhodné aproximaci, avšak v
 grafu komplexních funkcí jedné proměnné vypadají vlastně jako nekonečné stále
 se zúžující tuby. Lze si je představovat například jako
 \href{https://cs.wikipedia.org/wiki/Gabriel%C5%AFv_roh}{Gabrielův roh}. Vhodnou
  aproximací je zde uříznutí tohoto tělesa ve zvolené \uv{výšce}. Konečně,
  \emph{neodstranitelné singularity} jsou vskutku neodstranitelné. Dokonce platí
  věta, že komplexní funkce jedné proměnné s neodstranitelnou singularitou
  nabývají \textbf{úplně všech} hodnot z $\C$ na libovolně malém okolí této
  singularity. Menší stability již dosáhnout nelze. Neodstranitelné singularity
  si, tvrdíme, nelze ani rozumně představit.

 Ukážeme, že aspoň pro funkce jedné \emph{reálné} proměnné, jimiž se v tomto
 textu zabýváme, jsou množiny všech odstranitelných singularit i pólů spočetné.

 K důkazu prvního tvrzení uvažme množiny
 \[
  A \coloneqq \{x \in \R \mid \lim_{t \to x} f(t) < f(x)\} \quad \text{a} \quad
  B \coloneqq \{x \in \R \mid \lim_{t \to x} f(t) > f(x)\}.
 \]
 Platí, že množina odstranitelných singularit $f$ je rovna $A \cup B$. Zjevně
 tedy stačí dokázat, že každá z obou množin je spočetná. Provedeme onen důkaz
 pro množinu $A$, pro $B$ lze vést analogicky.

 Protože jsou \hyperref[prop:hustota-q-v-r]{$\Q$ hustá v $\R$}, nalezneme pro
 každé $x \in A$ racionální číslo $r_x \in \Q$ takové, že
 \[
  \lim_{t \to x} f(t) < r_x < f(x).
 \]
 Pak platí $A = \bigcup_{r \in \Q} A_r$, kde
 \[
  A_r = \{x \in A \mid r_x = r\}.
 \]
 Dokážeme, že každá z množin $A_r$ je spočetná. Volme $r \in \Q$. Protože
 $\lim_{t \to x} f(t) < r$ pro každé $x \in A_r$, nalezneme $\delta_x$ takové,
 že pro $t \in R(x,\delta_x)$ platí $f(t) < r$. Dále nahlédneme, že pro každé
 $x \neq y \in A_r$ platí
 \[
  (x - \frac{1}{2}\delta_x, x + \frac{1}{2}\delta_x) \cap
  (y-\frac{1}{2}\delta_y,y+\frac{1}{2}\delta_y) = \emptyset.
 \]
 Pro spor ať platí opak. Nalezneme $x \neq y \in A_r$ taková, že $|x-y|<\delta_x
 / 2 + \delta_y / 2$. Jest-li $\delta_x \leq \delta_y$, pak $|x-y| \leq
 \delta_y$, čili $f(x) < r_y = r$, neboť $x \in P(y,\delta_y)$. Z definice $r_x$
 však také $f(x) > r_x = r$, což je spor. Případ $\delta_x > \delta_y$ vede
 rovněž ke sporu na základě obdobného argumentu. Odtud plyne, že $A_r$ je
 spočetná, bo každý bod $A_r$ má kolem sebe okolí, v němž neleží žádné jiné body
 $A_r$. To mimo jiné znamená, že bodů $A_r$ je nejvýše tolik, jak racionálních
 čísel, tj. spočetně. Ježto $A = \bigcup_{r \in \Q} A_r$, je i $A$ spočetná.

 Nyní dokážeme, že $f$ má i spočetně pólů. Definujme opět množiny
 \begin{align*}
  U &\coloneqq \{x \in \R \mid \lim_{t \to x^{-}} f(t) < \lim_{t \to x^{+}}
  f(t)\},\\
   V &\coloneqq \{x \in \R \mid \lim_{t \to x^{-}} f(t) > \lim_{t \to x^{+}}
   f(t)\}.
 \end{align*}
 Množina pólů funkce $f$ je rovna $U \cup V$. Je třeba ukázat, že každá z množin
 $U,V$ je spočetná. Důkaz pro $U$ bude zjevně symetrický důkazu pro $V$, budeme
 se tudíž zabývat pouze množinou $U$. Pro každé $x \in U$ nalezneme $r_x \in \Q$
 takové, že
 \[
  \lim_{t \to x^{-}} f(t) < r_x < \lim_{t \to x^{+}} f(t).
 \]
 Pak jistě $U = \bigcup_{r \in \Q} U_r$, kde
 \[
  U_r \coloneqq \{x \in U \mid r_x = r\}.
 \]
 Ukážeme, že každá $U_r,r \in \Q$, je spočetná. Pro každé $x \in U_r$ nalezneme
 $\delta_x>0$ takové, že
 \begin{align*}
  \forall t \in R_-(x,\delta_x)&: f(t) < r,\\
  \forall t \in R_+(x,\delta_x)&:f(t) > r.
 \end{align*}
 Podobně jako v důkazu spočetnosti množiny odstranitelných singularit lze snadno
 ukázat, že
 \[
  (x,x+\delta_x) \cap (y-\delta_y,y) = \emptyset
 \]
 pro každá dvě $x \neq y \in U_r$, z čehož plyne, že $U_r$ -- a posléze i $U$ --
 je spočetná.
\end{example}

