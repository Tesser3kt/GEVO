\chapter{Limity funkcí}
\label{chap:limity-funkci}

Limita funkce je dost možná nejdůležitější ideou matematické analýzy a obecně
matematických disciplín, jež využívá fyzika. Davši vzniknout teorii derivací a
primitivních funkcí, umožnila popsat fyzikální jevy soustavami diferenciálních
rovnic a je základem zatím nejlepších známých modelův světa --
diferencovatelných struktur.

Principiálně se pojem \emph{limity funkce} neliší pramnoho od limity
posloupnosti. Matematici funkcí obyčejně myslíme zobrazení popisující vývoj
systému v čase (tzv. funkce \emph{jedné proměnné}), případně závislé na více
parametrech než jen na čase (tzv. funkce \emph{více proměnných}). Limita funkce
v nějakém určeném okamžiku pak znamená vlastně \uv{očekávanou hodnotu} této
funkce v tomto okamžiku -- hodnotu, ke které je funkce, čím méně času zbývá do
onoho okamžiku, tím blíže.

V tomto textu budeme sebe zaobírati pouze funkcemi závislými na čase tvořícími
systémy, jejichž stav je rovněž vyjádřen jediným číslem. Uvidíme, že i teorie
takto primitivních objektů je veskrze širá.

\begin{definition}{Reálná funkce jedné proměnné}{realna-funkce-jedne-promenne}
 Ať $M \subseteq \R$ je libovolná podmnožina $\R$. Zobrazení $f:M \to \R$
 nazýváme \emph{reálnou funkcí (jedné proměnné)}.
\end{definition}

Ačkolivěk ve světě, jest-li nám známo, proudí čas pouze jedním směrem,
matematiku takovými trivialitami netřeba třísnit. Pojem limity reálné funkce
budeme tedy definovat bez ohledu na \uv{proud času}. Budeme zkoumat jak hodnotu
reálné funkce, když se čas blíží \emph{zleva} (tj. přirozeně) k~danému okamžiku,
tak její očekávanou hodnotu proti toku času.

Ona dva přístupa slujeta limita funkce \emph{zleva} a limita funkce
\emph{zprava}. Před jejich výrokem ovšem učiníme kvapný formální obchvat. Bylo
by totiž nanejvýš neelegantní musiti různými logickými výroky definovat konečné
oproti nekonečným limitám v konečných oproti nekonečným bodům. Následující --
čistě formální avšak se silnou geometrickou intuicí -- pojem tyto případy
skuje v~jeden.

\begin{definition}{Okolí a prstencové okolí bodu}{okoli-a-prstencove-okoli-bodu}
Ať $a \in \R^*$ a $\varepsilon \in (0,\infty)$. \emph{Okolím} bodu $a$ (o
poloměru $\varepsilon$) myslíme množinu
\[
 B(a,\varepsilon) \coloneqq \begin{cases}
  (a-\varepsilon,a+\varepsilon), & \text{pokud } a \in \R,\\
  (1 / \varepsilon, \infty), &\text{pokud } a = \infty,\\
  (-\infty,-1 / \varepsilon), &\text{pokud } a = -\infty.
 \end{cases}
\]
Podobně, \emph{prstencovým okolím} $a$ (o velikosti $\varepsilon$) myslíme jeho
okolí bez samotného bodu $a$. Konkrétně,
\[
 R(a,\varepsilon) \coloneqq \begin{cases}
  (a-\varepsilon,a+\varepsilon) \setminus \{a\}, &\text{pokud }a \in \R,\\
  (1 / \varepsilon,\infty), & \text{pokud } a = \infty,\\
  (-\infty,-1 / \varepsilon), &\text{pokud } a = -\infty.
 \end{cases}
\]
Pro účely definice levých a pravých limit, pojmenujeme rovněž množinu
\[
 B_+(a,\varepsilon) \coloneqq \begin{cases}
  [a,a+\varepsilon), &\text{pokud }a \in \R,\\
  \emptyset, &\text{pokud }a = \infty,\\
  (-\infty,-1 / \varepsilon), &\text{pokud }a = -\infty
 \end{cases}
\]
\emph{pravým okolím} bodu $a$ a množinu
\[
 R_+(a,\varepsilon) \coloneqq \begin{cases}
  (a,a+\varepsilon), &\text{pokud }a \in \R,\\
  \emptyset, &\text{pokud }a = \infty,\\
  (-\infty,-1 / \varepsilon), &\text{pokud }a = -\infty
 \end{cases}
\]
\emph{pravým prstencovým okolím} bodu $a$. \emph{Levé okolí} a \emph{levé
prstencové okolí} bodu $a$ se definují analogicky.
\end{definition}

\begin{remark}{}{okoli-a-prstencove-okoli-bodu}
 Písmena \emph{B} a \emph{R} v definici okolí a prstencového okolí pocházejí z
 angl. slov \textbf{b}all a \textbf{r}ing. Okolí se v~angličtině přezdívá
 \uv{ball} pro to, že okolí bodu $a$ je ve skutečnosti (jednodimenzionální) kruh
 s poloměrem $\varepsilon$ o středu $a$. Znázornění okolí bodu jako kruhu v
 rovině je vysoce účinným vizualizačním aparátem.

 Čtenáře možná zarazilo číslo $1 / \varepsilon$ v definici okolí bodu $\infty$.
 Důvod užití $1 / \varepsilon$ oproti prostému $\varepsilon$ je spíše
 intuitivního rázu. V definici limity a v následných tvrzení si matematici
 obvykle představujeme pod $\varepsilon$ reálné číslo, které je \uv{nekonečně
 malé}. Chceme-li tedy, aby se \textbf{zmenšujícím se} $\varepsilon$ byla
 hodnota dané funkce stále blíže nekonečnu, musí se tato hodnota zvětšovat. Díky
 užité formulaci tomu tak je, neboť s menším $\varepsilon$ je číslo $1 /
 \varepsilon$ větší.
\end{remark}

\begin{figure}[ht]
 \centering
 \begin{subfigure}[b]{.49\textwidth}
  \centering
  \begin{tikzpicture}
   \tkzInit[xmin=-2,xmax=2]
   \tkzDrawX[label=,>=]
   \tkzDefPoints{0/0/a,-1.7/0/b,1.7/0/c}
   \tkzLabelPoint[below=2mm,color=BrickRed](a){$a$}

   \tkzDrawSegment[color=RoyalBlue,ultra thick](b,c)
   \tkzDrawPoint[size=4,color=BrickRed](a)
   \node[color=RoyalBlue] at (b) {\Large $($};
   \node[color=RoyalBlue] at (c) {\Large $)$};
   \tkzLabelPoint[below=2mm,color=RoyalBlue](b){$\clr{a} - \varepsilon$}
   \tkzLabelPoint[below=2mm,color=RoyalBlue](c){$\clr{a} + \varepsilon$}
  \end{tikzpicture}
  \caption{Okolí $\clb{B(\clr{a},\varepsilon)}$ bodu $\clr{a}$.}
  \label{subfig:okoli-a-prstencove-okoli-bodu-1}
 \end{subfigure}
 \begin{subfigure}[b]{.49\textwidth}
  \centering
  \begin{tikzpicture}
   \tkzInit[xmin=-2,xmax=2]
   \tkzDrawX[label=,>=]
   \tkzDefPoints{0/0/a,-1.7/0/b,1.7/0/c}
   \tkzLabelPoint[below=2mm,color=BrickRed](a){$a$}

   \tkzDrawSegment[color=ForestGreen,ultra thick](b,c)
   \tkzDrawPoint[size=6,draw=BrickRed,fill=white](a)
   \node[color=ForestGreen] at (b) {\Large $($};
   \node[color=ForestGreen] at (c) {\Large $)$};
   \tkzLabelPoint[below=2mm,color=ForestGreen](b){$\clr{a} - \varepsilon$}
   \tkzLabelPoint[below=2mm,color=ForestGreen](c){$\clr{a} + \varepsilon$}
  \end{tikzpicture}
  \caption{Prstencové okolí $\clg{R(\clr{a},\varepsilon)}$ bodu $\clr{a}$.}
  \label{subfig:okoli-a-prstencove-okoli-bodu-2}
 \end{subfigure}
 \caption{Okolí a prstencové okolí bodu $\clr{a} \in \R$.}
 \label{fig:okoli-a-prstencove-okoli-bodu}
\end{figure}

\begin{definition}{Jednostranná limita funkce}{jednostranna-limita-funkce}
 Ať $M \subseteq \R, f:M \to \R$ a $a \in \R^*$. Řekneme, že číslo $L \in \R^*$
 je \emph{limitou zleva} funkce $f$ v bodě $a$, pokud
 \[
 \forall \varepsilon>0 \, \exists \delta>0 \, \forall x \in P_{-}(a,\delta):
 f(x) \in B(L,\varepsilon).
 \]
 Tento fakt zapisujeme jako $L = \lim_{x \to a^{-}} f(x)$.

 Podobně, číslo $K \in \R^{*}$ je \emph{limitou zprava} funkce $f$ v bodě $a$,
 pokud
 \[
 \forall \varepsilon>0 \, \exists \delta>0 \, \forall x \in P_{+}(a,\delta):
 f(x) \in B(K,\varepsilon).
 \]
 Tento fakt zapisujeme jako $K = \lim_{x \to a^{+}} f(x)$.
\end{definition}

\begin{figure}[ht]
 \centering
 \begin{subfigure}[b]{.49\textwidth}
  \centering
  \begin{tikzpicture}[scale=1.25]
   \tkzInit[xmin=-1,xmax=3,ymin=-0.5,ymax=2.5]

   \tkzDefPoints{1/0/a,0/1.84/L,1/1.84/x}
   \tkzDefPoints{3/1.122/max1,3/2.558/max2}
   \tkzDefPoints{0.35/0/ad,0.35/1.122/xd,0/1.122/Le1,0/2.558/Le2}
   \tkzDrawPolygon[fill=Magenta!20!white,draw=none](Le1,max1,max2,Le2)

   \tkzDrawX[label=]
   \tkzDrawY[label=]

   \draw[scale=1,domain=-1:3.14,smooth,thick,variable=\x,RoyalBlue] plot
    ({\x},{sin(\x^2 r) + 1});
   \tkzDrawSegment[dashed,thick,BrickRed](a,x)
   \tkzDrawSegment[dashed,thick,BrickRed](ad,xd)
   \tkzDrawSegment[ultra thick,BrickRed](ad,a)
   \tkzDrawSegment[dashed,thick,ForestGreen](L,x)
   \tkzDrawPoint[size=6,draw=BrickRed,thick,fill=white](a)
   \tkzLabelPoint[below=2mm,color=BrickRed](a){$a$}
   \tkzDrawPoint[size=4,color=ForestGreen](L)
   \tkzLabelPoint[left=1mm,color=ForestGreen](L){$L$}

   \tkzDrawSegments[color=Magenta](Le1,max1 Le2,max2)
   \tkzDrawPoint[size=4,color=RoyalBlue](x)
   \tkzDrawPoint[size=4,draw=BrickRed,thick,fill=white](ad)
   \tkzDrawPoint[size=4,color=RoyalBlue](xd)
   \tkzLabelPoint[below=1mm,color=BrickRed](ad){$a - \delta$}
   \tkzDrawPoints[size=4,color=Magenta](Le1,Le2)
   \tkzLabelPoint[left=1mm,color=Magenta](Le1){$L - \varepsilon$}
   \tkzLabelPoint[left=1mm,color=Magenta](Le2){$L + \varepsilon$}

  \end{tikzpicture}
  \caption{Limita $\clb{f}$ v bodě $\clr{a}$ zleva.}
 \end{subfigure}
 \begin{subfigure}[b]{.49\textwidth}
  \centering
  \begin{tikzpicture}[scale=1.25]
   \tkzInit[xmin=-1,xmax=3,ymin=-0.5,ymax=2.5]

   \tkzDefPoints{1/0/a,0/1.84/L,1/1.84/x}
   \tkzDefPoints{3/1.407/max1,3/2.273/max2}
   \tkzDefPoints{1.65/0/ad,1.65/1.407/xd,0/1.407/Le1,0/2.273/Le2}
   \tkzDrawPolygon[fill=Magenta!20!white,draw=none](Le1,max1,max2,Le2)

   \tkzDrawX[label=]
   \tkzDrawY[label=]

   \draw[scale=1,domain=-1:3.14,smooth,thick,variable=\x,RoyalBlue] plot
    ({\x},{sin(\x^2 r) + 1});
   \tkzDrawSegment[dashed,thick,BrickRed](a,x)
   \tkzDrawSegment[dashed,thick,BrickRed](ad,xd)
   \tkzDrawSegment[ultra thick,BrickRed](ad,a)
   \tkzDrawSegment[dashed,thick,ForestGreen](L,x)
   \tkzDrawPoint[size=6,draw=BrickRed,thick,fill=white](a)
   \tkzLabelPoint[below=2mm,color=BrickRed](a){$a$}
   \tkzDrawPoint[size=4,color=ForestGreen](L)
   \tkzLabelPoint[left=1mm,color=ForestGreen](L){$L$}

   \tkzDrawSegments[color=Magenta](Le1,max1 Le2,max2)
   \tkzDrawPoint[size=4,color=RoyalBlue](x)
   \tkzDrawPoint[size=4,draw=BrickRed,thick,fill=white](ad)
   \tkzDrawPoint[size=4,color=RoyalBlue](xd)
   \tkzLabelPoint[below=1mm,color=BrickRed](ad){$a + \delta$}
   \tkzDrawPoints[size=4,color=Magenta](Le1,Le2)
   \tkzLabelPoint[left=1mm,color=Magenta](Le1){$L - \varepsilon$}
   \tkzLabelPoint[left=1mm,color=Magenta](Le2){$L + \varepsilon$}

  \end{tikzpicture}
  \caption{Limita $\clb{f}$ v bodě $\clr{a}$ zprava.}
 \end{subfigure}
 \caption{Jednostranné limity funkce $\clb{f}$ v bodě $\clr{a}$.}
 \label{fig:jednostranna-limita-funkce}
\end{figure}

\begin{warning}{}{okoli-v-limite}
 Fakt, že $L$ je limita \textbf{zleva} funkce $f$ v bodě $a$, vůbec neznamená,
 že hodnoty $f(x)$ se musejí blížit k $L$ rovněž \textbf{zleva}. Adverbia
 \emph{zleva} a \emph{zprava} značí pouze směr, kterým se k číslu $a$ přibližují
 \textbf{vstupy} funkce $f$, nikoli její \textbf{výstupy} k číslu $L$.
\end{warning}

Pochopitelně, lze též požadovat, aby hodnoty $f$ ležely v daném rozmezí kolem
bodu $L$, jak se její vstupy blíží k $a$ zleva i zprava zároveň. V principu,
blíží-li se $f$ ke stejnému číslu zleva i zprava, stačí vzít $\delta$ v
\myref{definici}{def:jednostranna-limita-funkce} tak malé, aby $f(x)$ leželo v
$B(L,\varepsilon)$ kdykoli je $x$ ve vzdálenosti nejvýše $\delta$ od $a$.

\begin{definition}{Oboustranná limita funkce}{oboustranna-limita-funkce}
 Ať $a,L \in \R^{*}$ a $f$ je reálná funkce. Řekneme, že $L$ je
 \emph{(oboustrannou) limitou} funkce $f$ v bodě $a$, pokud
 \[
 \forall \varepsilon > 0 \, \exists \delta > 0 \, \forall x \in P(a,\delta):
 f(x) \in B(L,\varepsilon).
 \]
 Tento fakt zapisujeme jako $L = \lim_{x \to a} f(x)$.
\end{definition}

Je jistě možné představovat si oboustrannou limitu funkce stejně jako limity
jednostranné na \myref{obrázku}{fig:jednostranna-limita-funkce}. Ovšem, ona
vlastnost \uv{oboustrannosti} umožňuje ještě jiný -- však ne rigorózní --
pohled. Povýšíme-li situaci do roviny, tj. do prostoru druhé dimenze, a na
funkci $f$ budeme nahlížet jako na zobrazení bodů roviny na body roviny, pak $L$
je limitou funkce $f$ v bodě $a$, když zobrazuje všechny body zevnitř kruhu o
poloměru $\delta$ a středu $a$ do kruhu o poloměru $\varepsilon$ a středu $L$.
