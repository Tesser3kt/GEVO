\chapter{Limity funkcí}
\label{chap:limity-funkci}

Limita funkce je dost možná nejdůležitější ideou matematické analýzy a obecně
matematických disciplín, jež využívá fyzika. Davši vzniknout teorii derivací a
primitivních funkcí, umožnila popsat fyzikální jevy soustavami diferenciálních
rovnic a je základem zatím nejlepších známých modelův světa --
diferencovatelných struktur.

Principiálně se pojem \emph{limity funkce} neliší pramnoho od limity
posloupnosti. Matematici funkcí obyčejně myslíme zobrazení popisující vývoj
systému v čase (tzv. funkce \emph{jedné proměnné}), případně závislé na více
parametrech než jen na čase (tzv. funkce \emph{více proměnných}). Limita funkce
v nějakém určeném okamžiku pak znamená vlastně \uv{očekávanou hodnotu} této
funkce v tomto okamžiku -- hodnotu, ke které je funkce, čím méně času zbývá do
onoho okamžiku, tím blíže.

V tomto textu budeme sebe zaobírati pouze funkcemi závislými na čase tvořícími
systémy, jejichž stav je rovněž vyjádřen jediným číslem. Uvidíme, že i teorie
takto primitivních objektů je veskrze širá.

\begin{definition}{Reálná funkce jedné proměnné}{realna-funkce-jedne-promenne}
 Ať $M \subseteq \R$ je libovolná podmnožina $\R$. Zobrazení $f:M \to \R$
 nazýváme \emph{reálnou funkcí (jedné proměnné)}.
\end{definition}

Ačkolivěk ve světě, jest-li nám známo, proudí čas pouze jedním směrem,
matematiku takovými trivialitami netřeba třísnit. Pojem limity reálné funkce
budeme tedy definovat bez ohledu na \uv{proud času}. Budeme zkoumat jak hodnotu
reálné funkce, když se čas blíží \emph{zleva} (tj. přirozeně) k~danému okamžiku,
tak její očekávanou hodnotu proti toku času.

Ona dva přístupa slujeta limita funkce \emph{zleva} a limita funkce
\emph{zprava}. Před jejich výrokem ovšem učiníme kvapný formální obchvat. Bylo
by totiž nanejvýš neelegantní musiti různými logickými výroky definovat konečné
oproti nekonečným limitám v konečných oproti nekonečným bodům. Následující --
čistě formální avšak se silnou geometrickou intuicí -- pojem tyto případy
skuje v~jeden.

\begin{definition}{Okolí a prstencové okolí bodu}{okoli-a-prstencove-okoli-bodu}
Ať $a \in \R^*$ a $\varepsilon \in (0,\infty)$. \emph{Okolím} bodu $a$ (o
poloměru $\varepsilon$) myslíme množinu
\[
 B(a,\varepsilon) \coloneqq \begin{cases}
  (a-\varepsilon,a+\varepsilon), & \text{pokud } a \in \R,\\
  (1 / \varepsilon, \infty), &\text{pokud } a = \infty,\\
  (-\infty,-1 / \varepsilon), &\text{pokud } a = -\infty.
 \end{cases}
\]
Podobně, \emph{prstencovým okolím} $a$ (o velikosti $\varepsilon$) myslíme jeho
okolí bez samotného bodu $a$. Konkrétně,
\[
 R(a,\varepsilon) \coloneqq \begin{cases}
  (a-\varepsilon,a+\varepsilon) \setminus \{a\}, &\text{pokud }a \in \R,\\
  (1 / \varepsilon,\infty), & \text{pokud } a = \infty,\\
  (-\infty,-1 / \varepsilon), &\text{pokud } a = -\infty.
 \end{cases}
\]
Pro účely definice levých a pravých limit, pojmenujeme rovněž množinu
\[
 B_+(a,\varepsilon) \coloneqq \begin{cases}
  [a,a+\varepsilon), &\text{pokud }a \in \R,\\
  \emptyset, &\text{pokud }a = \infty,\\
  (-\infty,-1 / \varepsilon), &\text{pokud }a = -\infty
 \end{cases}
\]
\emph{pravým okolím} bodu $a$ a množinu
\[
 R_+(a,\varepsilon) \coloneqq \begin{cases}
  (a,a+\varepsilon), &\text{pokud }a \in \R,\\
  \emptyset, &\text{pokud }a = \infty,\\
  (-\infty,-1 / \varepsilon), &\text{pokud }a = -\infty
 \end{cases}
\]
\emph{pravým prstencovým okolím} bodu $a$. \emph{Levé okolí} a \emph{levé
prstencové okolí} bodu $a$ se definují analogicky.
\end{definition}

\begin{remark}{}{okoli-a-prstencove-okoli-bodu}
 Písmena \emph{B} a \emph{R} v definici okolí a prstencového okolí pocházejí z
 angl. slov \textbf{b}all a \textbf{r}ing. Okolí se v~angličtině přezdívá
 \uv{ball} pro to, že okolí bodu $a$ je ve skutečnosti (jednodimenzionální) kruh
 s poloměrem $\varepsilon$ o středu $a$. Znázornění okolí bodu jako kruhu v
 rovině je vysoce účinným vizualizačním aparátem.

 Čtenáře možná zarazilo číslo $1 / \varepsilon$ v definici okolí bodu $\infty$.
 Důvod užití $1 / \varepsilon$ oproti prostému $\varepsilon$ je spíše
 intuitivního rázu. V definici limity a v následných tvrzení si matematici
 obvykle představujeme pod $\varepsilon$ reálné číslo, které je \uv{nekonečně
 malé}. Chceme-li tedy, aby se \textbf{zmenšujícím se} $\varepsilon$ byla
 hodnota dané funkce stále blíže nekonečnu, musí se tato hodnota zvětšovat. Díky
 užité formulaci tomu tak je, neboť s menším $\varepsilon$ je číslo $1 /
 \varepsilon$ větší.
\end{remark}

\begin{figure}[ht]
 \centering
 \begin{subfigure}[b]{.49\textwidth}
  \centering
  \begin{tikzpicture}
   \tkzInit[xmin=-2,xmax=2]
   \tkzDrawX[label=,>=]
   \tkzDefPoints{0/0/a,-1.7/0/b,1.7/0/c}
   \tkzLabelPoint[below=2mm,color=BrickRed](a){$a$}

   \tkzDrawSegment[color=RoyalBlue,ultra thick](b,c)
   \tkzDrawPoint[size=4,color=BrickRed](a)
   \node[color=RoyalBlue] at (b) {\Large $($};
   \node[color=RoyalBlue] at (c) {\Large $)$};
   \tkzLabelPoint[below=2mm,color=RoyalBlue](b){$\clr{a} - \varepsilon$}
   \tkzLabelPoint[below=2mm,color=RoyalBlue](c){$\clr{a} + \varepsilon$}
  \end{tikzpicture}
  \caption{Okolí $\clb{B(\clr{a},\varepsilon)}$ bodu $\clr{a}$.}
  \label{subfig:okoli-a-prstencove-okoli-bodu-1}
 \end{subfigure}
 \begin{subfigure}[b]{.49\textwidth}
  \centering
  \begin{tikzpicture}
   \tkzInit[xmin=-2,xmax=2]
   \tkzDrawX[label=,>=]
   \tkzDefPoints{0/0/a,-1.7/0/b,1.7/0/c}
   \tkzLabelPoint[below=2mm,color=BrickRed](a){$a$}

   \tkzDrawSegment[color=ForestGreen,ultra thick](b,c)
   \tkzDrawPoint[size=6,draw=BrickRed,fill=white](a)
   \node[color=ForestGreen] at (b) {\Large $($};
   \node[color=ForestGreen] at (c) {\Large $)$};
   \tkzLabelPoint[below=2mm,color=ForestGreen](b){$\clr{a} - \varepsilon$}
   \tkzLabelPoint[below=2mm,color=ForestGreen](c){$\clr{a} + \varepsilon$}
  \end{tikzpicture}
  \caption{Prstencové okolí $\clg{R(\clr{a},\varepsilon)}$ bodu $\clr{a}$.}
  \label{subfig:okoli-a-prstencove-okoli-bodu-2}
 \end{subfigure}
 \caption{Okolí a prstencové okolí bodu $\clr{a} \in \R$.}
 \label{fig:okoli-a-prstencove-okoli-bodu}
\end{figure}

\begin{definition}{Jednostranná limita funkce}{jednostranna-limita-funkce}
 Ať $M \subseteq \R, f:M \to \R$ a $a \in \R^*$. Řekneme, že číslo $L \in \R^*$
 je \emph{limitou zleva} funkce $f$ v bodě $a$, pokud
 \[
 \forall \varepsilon>0 \, \exists \delta>0 \, \forall x \in P_{-}(a,\delta):
 f(x) \in B(L,\varepsilon).
 \]
 Tento fakt zapisujeme jako $L = \lim_{x \to a^{-}} f(x)$.

 Podobně, číslo $K \in \R^{*}$ je \emph{limitou zprava} funkce $f$ v bodě $a$,
 pokud
 \[
 \forall \varepsilon>0 \, \exists \delta>0 \, \forall x \in P_{+}(a,\delta):
 f(x) \in B(K,\varepsilon).
 \]
 Tento fakt zapisujeme jako $K = \lim_{x \to a^{+}} f(x)$.
\end{definition}

\begin{figure}[ht]
 \centering
 \begin{subfigure}[b]{.49\textwidth}
  \centering
  \begin{tikzpicture}[scale=1.25]
   \tkzInit[xmin=-1,xmax=3,ymin=-0.5,ymax=2.5]

   \tkzDefPoints{1/0/a,0/1.84/L,1/1.84/x}
   \tkzDefPoints{3/1.122/max1,3/2.558/max2}
   \tkzDefPoints{0.35/0/ad,0.35/1.122/xd,0/1.122/Le1,0/2.558/Le2}
   \tkzDrawPolygon[fill=Magenta!20!white,draw=none](Le1,max1,max2,Le2)

   \tkzDrawX[label=]
   \tkzDrawY[label=]

   \draw[scale=1,domain=-1:3.14,smooth,thick,variable=\x,RoyalBlue] plot
    ({\x},{sin(\x^2 r) + 1});
   \tkzDrawSegment[dashed,thick,BrickRed](a,x)
   \tkzDrawSegment[dashed,thick,BrickRed](ad,xd)
   \tkzDrawSegment[ultra thick,BrickRed](ad,a)
   \tkzDrawSegment[dashed,thick,ForestGreen](L,x)
   \tkzDrawPoint[size=6,draw=BrickRed,thick,fill=white](a)
   \tkzLabelPoint[below=2mm,color=BrickRed](a){$a$}
   \tkzDrawPoint[size=4,color=ForestGreen](L)
   \tkzLabelPoint[left=1mm,color=ForestGreen](L){$L$}

   \tkzDrawSegments[color=Magenta](Le1,max1 Le2,max2)
   \tkzDrawPoint[size=4,color=RoyalBlue](x)
   \tkzDrawPoint[size=4,draw=BrickRed,thick,fill=white](ad)
   \tkzDrawPoint[size=4,color=RoyalBlue](xd)
   \tkzLabelPoint[below=1mm,color=BrickRed](ad){$a - \delta$}
   \tkzDrawPoints[size=4,color=Magenta](Le1,Le2)
   \tkzLabelPoint[left=1mm,color=Magenta](Le1){$L - \varepsilon$}
   \tkzLabelPoint[left=1mm,color=Magenta](Le2){$L + \varepsilon$}

  \end{tikzpicture}
  \caption{Limita $\clb{f}$ v bodě $\clr{a}$ zleva.}
 \end{subfigure}
 \begin{subfigure}[b]{.49\textwidth}
  \centering
  \begin{tikzpicture}[scale=1.25]
   \tkzInit[xmin=-1,xmax=3,ymin=-0.5,ymax=2.5]

   \tkzDefPoints{1/0/a,0/1.84/L,1/1.84/x}
   \tkzDefPoints{3/1.407/max1,3/2.273/max2}
   \tkzDefPoints{1.65/0/ad,1.65/1.407/xd,0/1.407/Le1,0/2.273/Le2}
   \tkzDrawPolygon[fill=Magenta!20!white,draw=none](Le1,max1,max2,Le2)

   \tkzDrawX[label=]
   \tkzDrawY[label=]

   \draw[scale=1,domain=-1:3.14,smooth,thick,variable=\x,RoyalBlue] plot
    ({\x},{sin(\x^2 r) + 1});
   \tkzDrawSegment[dashed,thick,BrickRed](a,x)
   \tkzDrawSegment[dashed,thick,BrickRed](ad,xd)
   \tkzDrawSegment[ultra thick,BrickRed](ad,a)
   \tkzDrawSegment[dashed,thick,ForestGreen](L,x)
   \tkzDrawPoint[size=6,draw=BrickRed,thick,fill=white](a)
   \tkzLabelPoint[below=2mm,color=BrickRed](a){$a$}
   \tkzDrawPoint[size=4,color=ForestGreen](L)
   \tkzLabelPoint[left=1mm,color=ForestGreen](L){$L$}

   \tkzDrawSegments[color=Magenta](Le1,max1 Le2,max2)
   \tkzDrawPoint[size=4,color=RoyalBlue](x)
   \tkzDrawPoint[size=4,draw=BrickRed,thick,fill=white](ad)
   \tkzDrawPoint[size=4,color=RoyalBlue](xd)
   \tkzLabelPoint[below=1mm,color=BrickRed](ad){$a + \delta$}
   \tkzDrawPoints[size=4,color=Magenta](Le1,Le2)
   \tkzLabelPoint[left=1mm,color=Magenta](Le1){$L - \varepsilon$}
   \tkzLabelPoint[left=1mm,color=Magenta](Le2){$L + \varepsilon$}

  \end{tikzpicture}
  \caption{Limita $\clb{f}$ v bodě $\clr{a}$ zprava.}
 \end{subfigure}
 \caption{Jednostranné limity funkce $\clb{f}$ v bodě $\clr{a}$.}
 \label{fig:jednostranna-limita-funkce}
\end{figure}

\begin{warning}{}{okoli-v-limite}
 Fakt, že $L$ je limita \textbf{zleva} funkce $f$ v bodě $a$, vůbec neznamená,
 že hodnoty $f(x)$ se musejí blížit k $L$ rovněž \textbf{zleva}. Adverbia
 \emph{zleva} a \emph{zprava} značí pouze směr, kterým se k číslu $a$ přibližují
 \textbf{vstupy} funkce $f$, nikoli její \textbf{výstupy} k číslu $L$.
\end{warning}

Pochopitelně, lze též požadovat, aby hodnoty $f$ ležely v daném rozmezí kolem
bodu $L$, jak se její vstupy blíží k $a$ zleva i zprava zároveň. V principu,
blíží-li se $f$ ke stejnému číslu zleva i zprava, stačí vzít $\delta$ v
\myref{definici}{def:jednostranna-limita-funkce} tak malé, aby $f(x)$ leželo v
$B(L,\varepsilon)$ kdykoli je $x$ ve vzdálenosti nejvýše $\delta$ od $a$.

\begin{definition}{Oboustranná limita funkce}{oboustranna-limita-funkce}
 Ať $a,L \in \R^{*}$ a $f$ je reálná funkce. Řekneme, že $L$ je
 \emph{(oboustrannou) limitou} funkce $f$ v bodě $a$, pokud
 \[
 \forall \varepsilon > 0 \, \exists \delta > 0 \, \forall x \in P(a,\delta):
 f(x) \in B(L,\varepsilon).
 \]
 Tento fakt zapisujeme jako $L = \lim_{x \to a} f(x)$.
\end{definition}

Je jistě možné představovat si oboustrannou limitu funkce stejně jako limity
jednostranné na \myref{obrázku}{fig:jednostranna-limita-funkce}. Ovšem, ona
vlastnost \uv{oboustrannosti} umožňuje ještě jiný -- však ne rigorózní --
pohled. Povýšíme-li situaci do roviny, tj. do prostoru druhé dimenze, a na
funkci $f$ budeme nahlížet jako na zobrazení bodů roviny na body roviny, pak $L$
je limitou funkce $f$ v bodě $a$, když zobrazuje všechny body zevnitř kruhu o
poloměru $\delta$ a středu $a$ do kruhu o poloměru $\varepsilon$ a středu $L$.
Jako na \myref{obrázku}{fig:oboustranna-limita-ve-2D}.

\begin{figure}[ht]
 \centering
 \begin{tikzpicture}
  \def\del{1}
  \def\eps{1.5}

  \tkzDefPoints{0/0/a,4/2/L}
  \tkzLabelPoint[color=BrickRed,below=1mm](a){$a$}
  \tkzLabelPoint[color=ForestGreen,below=1mm](L){$L$}

  \tkzDefCircle[R](a,\del) \tkzGetPoint{ad}
  \tkzDrawCircle[thick](a,ad)

  \tkzDefCircle[R](L,\eps) \tkzGetPoint{Le}
  \tkzDrawCircle[color=Magenta,thick](L,Le)

  \tkzDefPoints{0/1.2/s,2.8/3.2/t}
  \draw[bend left=45,color=RoyalBlue,-latex] (s) to node[midway,above left]{$f$}
   (t);

  \tkzDefPointOnCircle[R = center a angle 150 radius \del] \tkzGetPoint{B}
  \tkzDrawPoint[color=BrickRed,size=3](B)
  \tkzDrawSegment[color=BrickRed,dashed,thick](a,B)
  \tkzLabelSegment[color=BrickRed,above right=-1mm](a,B){$\delta$}
  \tkzDrawPoint[draw=BrickRed,thick,size=6,fill=white](a)

  \tkzDefPointOnCircle[R = center L angle 150 radius \eps] \tkzGetPoint{C}
  \tkzDrawPoint[color=Magenta,size=3](C)
  \tkzDrawSegment[color=Magenta,dashed,thick](L,C)
  \tkzLabelSegment[color=Magenta,above right=-1mm](L,C){$\varepsilon$}
  \tkzDrawPoint[color=ForestGreen,size=6](L)

  \tkzDefPoints{0.5/0.5/x,3.3/1.8/fx}
  \tkzDrawPoint[size=4,color=black](x)
  \tkzDrawPoint[size=4,color=RoyalBlue](fx)
  \tkzLabelPoint[below](x){$x$}
  \tkzLabelPoint[below,color=RoyalBlue](fx){$f(x)$}
 \end{tikzpicture}

 \caption{Oboustranná limita funkce \uv{ve 2D}.}
 \label{fig:oboustranna-limita-ve-2D}
\end{figure}

Doporučujeme čtenářům, aby se zamysleli, čím by v této dvoudimenzionální říši
byla \emph{jednostranná} limita funkce. Sen zámysl snad vedl k představě, že by
se vstupy $x$ musely blížit k bodu $a$ po nějaké určené přímce. Existence
\uv{všestranné} limity v $a$ by pak byla ekvivalentní existenci nespočetně mnoha
\uv{jednostranných} limit -- jedné pro každou přímku procházející bodem $a$.
Věříme, že není obtížné nahlédnout, jak zbytečný by takový pojem ve dvou
dimenzích byl. Popsaná situace přímo souvisí s faktem, že první dimenze je z
geometrického pohledu \uv{degenerovaná} -- kružnice je pouze dvoubodovou
množinou.

\section{Základní poznatky o limitě funkce}
\label{sec:zakladni-poznatky-o-limite-funkce}

Počneme nyní shrnovati intuitivně vcelku zřejmé výsledky o limitách reálných
funkcí. Jakž jsme již vícekrát děli, ona \uv{intuitivní zřejmost} pravdivosti
výroků nechce nabodnout k přeskoku či trivializaci jejich důkazů. Vodami
nekonečnými radno broditi se ostražitě, bo tvrzení jako \emph{limita složené
	funkce} % TODO
ráda svědčí, že intuicí bez logiky člověk na břeh nedoplove.

Na první pád není překvapivé, že limita funkce je jednoznačně určena,
pochopitelně za předpokladu její existence. Vyzýváme čtenáře, aby se při čtení
důkazu drželi vizualizace oboustranné limity z
\myref{obrázku}{fig:oboustranna-limita-ve-2D}.

\begin{lemma}{Jednoznačnost limity}{jednoznacnost-limity}
	Limita funkce (ať už jednostranná či oboustranná) je jednoznačně určená, pokud
	existuje.
\end{lemma}

\begin{lemproof}
	Dokážeme lemma pouze pro oboustrannou limitu, důkaz pro limity jednostranné je
	v zásadě totožný.

	Pro spor budeme předpokládat, že $L$ i $L'$ jsou limity $f$ v bodě $a \in
		\R^{*}$. Nejprve ošetříme případ, kdy $L,L' \in \R$. Bez újmy na obecnosti
	smíme předpokládat, že $L > L'$. Volme $\varepsilon \coloneqq (L-L') / 3$. K
	tomuto $\varepsilon$ existují z
	\hyperref[def:oboustranna-limita-funkce]{definice limity} $\delta_1>0,
		\delta_2>0$ takové,
	že
	\[
		\forall x \in R(a,\delta_1): f(x) \in B(L,\varepsilon).
	\]
	a rovněž
	\[
		\forall x \in R(a,\delta_2): f(x) \in B(L',\varepsilon).
	\]
	Volíme-li ovšem $\delta \coloneqq \min(\delta_1,\delta_2)$, pak pro $x \in
		R(a,\delta)$ dostaneme
	\[
		f(x) \in B(L,\varepsilon) \cap B(L',\varepsilon).
	\]
	Poslední vztah lze přepsat do tvaru
	\begin{align*}
		L - \varepsilon  & < f(x) < L + \varepsilon,  \\
		L' - \varepsilon & < f(x) < L' + \varepsilon.
	\end{align*}
	Odtud plyne, že
	\[
		L - \varepsilon < L' + \varepsilon,
	\]
	což po dosazení $\varepsilon = (L - L') / 3$ a následné úpravě vede na
	\[
		2L - L' < 2L' - L,
	\]
	z čehož ihned
	\[
		L < L',
	\]
	což je spor.

	Nyní ať například $L = \infty$ a $L' \in \R$. Z
	\hyperref[def:okoli-a-prstencove-okoli-bodu]{definice okolí} $B(L,\varepsilon)$
	pro $L = \infty$, stačí nalézt $\varepsilon > 0$ takové, že
	\[
		\frac{1}{\varepsilon} > L' + \varepsilon,
	\]
	pak se totiž nemůže stát, že
	\[
		f(x) \in B(\infty,\varepsilon) \cap B(L',\varepsilon).
	\]
	Snadným výpočtem zjistíme, že
	\[
		\frac{1}{\varepsilon} > L' + \varepsilon
	\]
	právě tehdy, když $\varepsilon < (\sqrt{L'^2 + 4} - L') / 2$. Pro libovolné
	takové $\varepsilon$ tudíž dostáváme spor stejně jako v předchozím případě.

	Ostatní případy se ošetří obdobně.
\end{lemproof}

\begin{figure}[ht]
 \centering
 \begin{tikzpicture}
	\def\del{1}
	\def\eps{1.5}

	\tkzDefPoints{0/0/a,4/2/L,4/-2/Lp}
	\tkzLabelPoint[color=BrickRed,below=1mm](a){$a$}
	\tkzLabelPoint[color=ForestGreen,below left=1mm](L){$L$}
	\tkzLabelPoint[color=ForestGreen,below left=1mm](Lp){$L'$}

	\tkzDefCircle[R](a,\del) \tkzGetPoint{ad}
	\tkzDrawCircle[thick](a,ad)

	\tkzDefCircle[R](L,\eps) \tkzGetPoint{Le}
	\tkzDrawCircle[color=Magenta,thick](L,Le)

	\tkzDefPoints{0/1.2/s,2.8/3.2/t,0/-1.2/s2,2.8/-3.2/t2}
	\draw[bend left=45,color=RoyalBlue,-latex] (s) to node[midway,above left]{$f$}
	(t);
	\draw[bend right=45,color=RoyalBlue,-latex] (s2) to node[midway,below
		left]{$f$} (t2);

	\tkzDefPointOnCircle[R = center a angle 150 radius \del] \tkzGetPoint{B}
	\tkzDrawPoint[color=BrickRed,size=3](B)
	\tkzDrawSegment[color=BrickRed,dashed,thick](a,B)
	\tkzLabelSegment[color=BrickRed,above right=-1mm](a,B){$\delta$}
	\tkzDrawPoint[draw=BrickRed,thick,size=6,fill=white](a)

	\tkzDrawSegment[decorate,decoration={brace,
				amplitude=10pt},color=Aquamarine,thick](L,Lp)
	\tkzLabelSegment[right=3mm,color=Aquamarine](L,Lp){$\displaystyle
			\frac{|L-L'|}{2}$}

	\tkzDefPointOnCircle[R = center L angle 150 radius \eps] \tkzGetPoint{C}
	\tkzDrawPoint[color=Magenta,size=3](C)
	\tkzDrawSegment[color=Magenta,dashed,thick](L,C)
	\tkzLabelSegment[color=Magenta,above right=-1mm](L,C){$\varepsilon$}
	\tkzDrawPoint[color=ForestGreen,size=6](L)

	\tkzDefCircle[R](Lp,\eps) \tkzGetPoint{Lpe}
	\tkzDrawCircle[color=Magenta,thick](Lp,Lpe)

	\tkzDefPointOnCircle[R = center Lp angle 150 radius \eps] \tkzGetPoint{D}
	\tkzDrawPoint[color=Magenta,size=3](D)
	\tkzDrawSegment[color=Magenta,dashed,thick](Lp,D)
	\tkzLabelSegment[color=Magenta,above right=-1mm](Lp,D){$\varepsilon$}
	\tkzDrawPoint[color=ForestGreen,size=6](Lp)

 \end{tikzpicture}
 \caption{Spor v důkazu \myref{lemmatu}{lem:jednoznacnost-limity}.}
 \label{fig:jednoznacnost-limity}
\end{figure}

\begin{lemma}{}{ma-limitu-je-omezena}
 Ať reálná funkce $f$ má \textbf{konečnou} limitu $L \in \R$ v bodě $a \in
 \R^{*}$. Pak existuje prstencové okolí $a$, na němž je $f$ omezená.
\end{lemma}
\begin{lemproof}
 Pro dané $\varepsilon>0$ nalezneme z
 \hyperref[def:oboustranna-limita-funkce]{definice limity} $\delta>0$ takové, že
 pro $x \in R(a,\delta)$ platí $f(x) \in B(L,\varepsilon)$. Protože však
 $B(L,\varepsilon) = (L-\varepsilon,L+\varepsilon)$ platí pro $x \in
 R(a,\delta)$ odhady
 \[
  L-\varepsilon \leq f(x) \leq L+\varepsilon,
 \]
 čili je $f$ na $R(a,\delta)$ omezená.
\end{lemproof}

Vzhledem k základním aritmetickým operacím si limity funkcí počínají vychovaně.
Za předpokladu, že výsledný výraz dává smysl, můžeme spočítat limitu součtu,
součinu či podílu funkcí jako součet, součin či podíl limit těchto funkcí.

\begin{theorem}{Aritmetika limit funkcí}{aritmetika-limit-funkci}
 Ať $f,g$ jsou reálné funkce a $a \in \R^{*}$. Předpokládejme, že $\lim_{x \to
 a} f(x)$ i $\lim_{x \to a} g(x)$ existují a označme je po řadě $L_f$ a $L_g$.
 Potom platí
 \begin{enumerate}[label=(\alph*)]
  \item $\lim_{x \to a} (f + g)(x) = L_f + L_g$, dává-li výraz napravo smysl.
  \item $\lim_{x \to a} (f \cdot g)(x) = L_f \cdot L_g$, dává-li výraz napravo
   smysl.
  \item $\lim_{x \to a} (f / g)(x) = L_f / L_g$, dává-li výraz napravo smysl.
 \end{enumerate}
\end{theorem}

\begin{thmproof}
 Dokážeme pouze část (c), neboť je výpočetně nejnáročnější, ač nepřináší mnoho
 intuice. Část (a) je triviální a (b) je lehká. Vyzýváme čtenáře, aby se je
 pokusili dokázat sami.

 Už jen v důkazu samotné části (c) bychom správně měli rozlišit šest různých
 případů:
 \begin{enumerate}
  \item $L_f \in \R, L_g \in \R \setminus \{0\}$,
  \item $L_f \in \R, L_g \in \{-\infty,\infty\}$,
  \item $L_f = \infty, L_g \in (0,\infty)$,
  \item $L_f = \infty, L_g \in (-\infty,0)$,
  \item $L_f = -\infty, L_g \in (0,\infty)$,
  \item $L_f = -\infty, L_g \in (-\infty,0)$.
 \end{enumerate}
 Jelikož se výpočty limit v oněch případech liší vzájemně pramálo a získaná
 intuice je asymptoticky rovna té ze znalosti metod řešení exponenciálních
 rovnic, soustředíme se pouze na (nejzajímavější) případ (1).

 Ať tedy $L_f \in \R, L_g \in \R \setminus \{0\}$. Je nejprve dobré si uvědomit,
 že $L_f / L_g$ není definován \textbf{nikdy}, pokud $L_g = 0$, bez ohledu na
 hodnotu $L_f$. Totiž, hodnoty $g$ se mohou k $L_g$ limitně blížit zprava,
 zleva či střídavě z obou směrů. Nelze tudíž obecně určit, zda dělíme čím dál
 tím menším kladným číslem, či čím dál tím větším záporným číslem.

 Položme $\varepsilon_g = |L_g| / 2$. K tomuto $\varepsilon_g$ existuje z
 \hyperref[def:oboustranna-limita-funkce]{definice limity} $\delta_g$ takové, že
 pro $x \in R(a,\delta_g)$ platí $g(x) \in B(L_g,\varepsilon_g)$. Poslední vztah
 si přepíšeme na
 \begin{align*}
  L_g - \varepsilon_g & < g(x) < L_g + \varepsilon_g,\\
  L_g - \frac{|L_g|}{2} & < g(x) < L_g + \frac{|L_g|}{2}.
 \end{align*}
 Speciálně tedy pro $x \in R(a,\delta_g)$ máme odhad
 \[
  |g(x)| > \left| L_g - \frac{|L_g|}{2} \right| > \frac{|L_g|}{2}.
 \]
 Jelikož poslední výraz je z předpokladu kladný, má výraz $f(x) / g(x)$ smysl
 pro každé $x \in R(a,\delta_g)$, neboť pro tato $x$ platí $g(x) \neq 0$.

 Pro $x \in R(a,\delta_g)$ odhadujme
 \begin{align*}
  \left| \frac{f(x)}{g(x)} - \frac{L_f}{L_g} \right| &= \frac{|f(x)L_g -
  g(x)L_f|}{|g(x)||L_g|} = \frac{|f(x) L_g - L_f L_g + L_f L_g -
  g(x)L_f|}{|g(x)| |L_g|}\\
  																									 & \leq \frac{|L_g| |f(x) -
  																									 L_f| + |L_f| |L_g -
  																									g(x)|}{|g(x)| |L_g|}\\
  																									 &= \frac{1}{|g(x)|}
  																									 |f(x) - L_f| +
  																									 \frac{|L_f|}{|g(x)|
  																									 |L_g|}|L_g - g(x)|\\
  																									 &< \frac{2}{|L_g|}|f(x) -
  																									 L_f| + \frac{2
  																									 |L_f|}{|L_g|^2}|L_g -
  																									 g(x)|\\
  																									 & \leq c(|f(x) - L_f| +
  																									 |L_g - g(x)|)
 \end{align*}
 pro $c \coloneqq \max(2 / |L_g|, 2|L_f|/|L_g|^2)$.

 Ať je nyní dáno $\varepsilon>0$. K číslu $\varepsilon / 2c$ existují z
 \hyperref[def:oboustranna-limita-funkce]{definice limity} $\delta_1,\delta_2>0$
 taková, že
 \begin{align*}
 	\forall x \in R(a,\delta_1)&: |g(x) - L_g| < \frac{\varepsilon}{2c},\\
 	\forall x \in R(a,\delta_2)&: |f(x) - L_f| < \frac{\varepsilon}{2c}.
 \end{align*}
 Položíme-li nyní $\delta \coloneqq \min(\delta_1,\delta_2,\delta_g)$, pak pro
 $x \in R(a,\delta)$ platí
 \[
  \left| \frac{f(x)}{g(x)} - \frac{L_f}{L_g} \right| < c (|f(x) - L_f| + |L_g -
  g(x)|) < c \left(\frac{\varepsilon}{2c} + \frac{\varepsilon}{2c}\right) =
  \varepsilon,
 \]
 což dokazuje rovnost $\lim_{x \to a} (f / g)(x) = L_f / L_g$.
\end{thmproof}

\begin{exercise}{}{aritmetika-limit}
 Dokažte tvrzení (b) a (c) ve \myref{větě}{thm:aritmetika-limit-funkci}.
\end{exercise}

