\section{Taylorova řada}
\label{sec:taylorova-rada}

Když už víme, že Taylorův polynom je nejlepší možnou aproximací stupně $n$
nějaké funkce $f$ na okolí daného bodu polynomem stupně nejvýše $n$, je
přirozené se ptát, co se stane, nahradíme-li tento polynom nekonečnou řadou se
stejnými koeficienty? Rozumná, však naivní, domněnka zní, že taková řada
aproximuje danou funkci \emph{nekonečně dobře}, to jest je jí přímo rovna. Lze
snad každou reálnou funkci zapsat nekonečnou řadou?

Jak je tomu u spousty nadějných domněnek, odpověď zní důrazně \textbf{ne}.
Ovšem, mnoho dostatečně hezkých funkcí lze zapsat nekonečnou řadou aspoň na
nějakém okolí zvolených bodů. Částečně překvapivé snad je, že samotná
konvergence takové řady nestačí k tomu, aby byla rovna aproximované funkci.

Nebudeme do této problematiky zabíhat hlouběji, rozmyslíme si pouze jeden
význačný příklad.

\begin{definition}{Taylorova řada}{taylorova-rada}
 Ať $f: M \to \R$ je reálná funkce mající derivace všech řádů v $a \in M$. Pak
 nekonečnou řadu
 \[
  \sum_{n=0}^{\infty} \frac{f^{(n)}(a)}{n!}(x-a)^{n}
 \]
 nazýváme \emph{Taylorovou řadou} funkce $f$ v bodě $a$.
\end{definition}

\begin{warning}{}{taylorova-rada-neaproximuje}
 Bohužel existují nekonečně diferencovatelné funkce, kterým jejich Taylorova
 řada odpovídá pouze v jediném bodě, ale nikoli na sebemenším okolí tohoto bodu.
 Uvažme například
 \[
  f(x) = \begin{cases}
   \exp \left( -\frac{1}{x^2} \right), & x \neq 0,\\
   0,& x = 0.
  \end{cases}
 \]
 Protože $\lim_{x \to 0} \exp\left(-1 / x^2\right) = 0$, je $f$ spojitá v bodě
 $0$ a zřejmě je tam nekonečně diferencovatelná. Platí
 \[
  \sum_{n=0}^{\infty} \frac{f^{(n)}(0)}{n!}x^{n} = 0,
 \]
 neboť $f^{(n)}(0) = 0$ pro každé $n \in \N$. Avšak $f(x) \neq 0$ pro libovolné
 $x \neq 0$.
\end{warning}

\begin{remark}{}{funkce-danou-radou}
 Taylorova řada funkce definované součtem nekonečné řady je (zřejmě z podstaty
 věci) přesně tato řada. Tento fakt nebudeme dokazovat; pro elementární funkce
 jej ponecháme jako snadné cvičení.
\end{remark}

\begin{exercise}{}{taylorovy-rady-exp-sin-cos}
 Dokažte, že Taylorovy řady elementárních funkcí $\exp$, $\sin$ a $\cos$ jsou
 přesně řady z jejich definic.
\end{exercise}

Zbytek sekce věnujeme studiu Taylorovy řady funkce $\log$. Na rozdíl od svého
inverzu není $\log$ dána nekonečnou řadou. Existuje však relativně malé okolí
bodu $1$, kde je $\log$ shodná se svojí Taylorovou řadou. Pro snadnost výpočtů
budeme však místo bodu $1$ uvažovat bod $0$ a místo $\log x$ funkci $\log(1+x)$.
Tento přístup je zjevně ekvivalentní původnímu.

\begin{theorem}{Taylorova řada funkce $\log$}{taylorova-rada-funkce-log}
 Pro $x \in (-1,1]$ platí
 \[
  \log(1+x) = \sum_{n=1}^{\infty} (-1)^{n-1}\frac{x^{n}}{n}.
 \]
\end{theorem}
\begin{thmproof}
 Nejprve ukážeme, že uvedená řada je vskutku Taylorovou řadou funkce $\log$ v
 bodě $0$.

 Položme $f(x) = \log(1+x)$. Indukcí ověříme, že
 \[
  f^{(n)}(x) = (-1)^{n-1}\frac{(n-1)!}{(1 + x)^{n}}.
 \]
 Jistě
 \[
  f^{(1)}(x) = \frac{1}{1 + x} = (-1)^{0} \cdot \frac{0!}{(1+x)^{1}}.
 \]
 Dále, z indukčního předpokladu
 \begin{align*}
  f^{(n+1)}(0) &= (f^{(n)})^{(1)}(0) = \left((-1)^{n-1} \cdot
  \frac{(n-1)!}{(1+x)^{n}}\right)^{(1)}\\ 
               &= -n \cdot (-1)^{n-1} \cdot
               \frac{(n-1)!}{(1+x)^{n+1}} = (-1)^{n} \cdot \frac{n!}{(1+x)^{n+1}},
 \end{align*}
 tedy závěr platí.

 Odtud ihned plyne, že $f^{(n)}(0) = (-1)^{n-1}(n-1)!$. Čili, Taylorovou řadou
 funkce $\log(1+x)$ v~bodě $0$ je vskutku
 \[
  \sum_{n=1}^{\infty} \frac{f^{(n)}(0)}{n!}x^{n} = \sum_{n=0}^{\infty}
  (-1)^{n-1}\frac{(n-1)!}{n!}x^{n} = \sum_{n=0}^{\infty}
  (-1)^{n-1}\frac{x^{n}}{n}.
 \]

 Pokračujme druhou částí tvrzení. Je zřejmé, že pro $|x| > 1$ tato řada není ani
 konvergentní, neboť neplatí $\lim_{n \to \infty} x^{n} / n = 0$. Pro $x = -1$
 máme
 \[
  (-1)^{n-1}\frac{(-1)^{n}}{n} = (-1)^{2n-1}\frac{1}{n} = -\frac{1}{n}
 \]
 a řada $\sum_{n=1}^{\infty} -1 / n$ je divergentní. Konečně, pro $x \in (-1,1]$
 řada vskutku konvergentní je. Plyne to však z tvrzení o konvergenci obecných
 řad, jež jsme si nedokázali; stavíme tudíž tento fakt na slepé víře.

 Nyní dokážeme, že je na tomto intervalu rovna $\log(1+x)$. K tomu stačí ověřit,
 že chyba aproximace
 \begin{equation*}
  \label{eq:chyba-log}
  \tag{$\clubsuit$}
  |\log(1+x) - T^{\log(1+x),0}_n(x)| = \left| \log(1+x) - \sum_{k=1}^n
  (-1)^{k-1}\frac{x^{k}}{k}\right|
 \end{equation*}
 jde pro $n \to \infty$ k $0$.

 Ať je nejprve $x \in [0,1]$. Podle \myref{důsledku}{cor:lagrangeuv-tvar-zbytku}
 existuje pro každé $n \in N$ číslo $\xi_n \in [0,x)$ takové, že
 \[
  \eqref{eq:chyba-log} = \left| \frac{1}{(n+1)!}f^{(n+1)}(\xi_n)x^{n} \right|,
 \]
 kde opět $f(x) = \log(1 + x)$. Využitím výpočtu $f^{(n+1)}$ výše a odhadu $0
 \leq x / (1 + \xi_n) \leq 1$ spočteme
 \begin{align*}
  \eqref{eq:chyba-log} = \left| \frac{1}{(n+1)!}f^{(n+1)}(\xi_n)x^{n} \right| &=
 \left| \frac{1}{(n+1)!}(-1)^{n}\frac{n!}{(1+\xi_n)^{n+1}}x^{n+1} \right) \\
                                                       &=
  \frac{1}{n+1} \left( \frac{x}{1+\xi_n} \right)^{n+1} \leq \frac{1}{n+1}
  \overset{n \to \infty}{\longrightarrow} 0,
 \end{align*}
 jak jsme chtěli.

 Konečně, ať $x \in (-1,0)$. Nyní naopak použijeme
 \myref{důsledek}{cor:cauchyho-tvar-zbytku} a pro každé $n \in \N$ nalezneme
 $\xi_n \in (x,0)$ splňující
 \[
  \eqref{eq:chyba-log} = \left| \frac{1}{n!}f^{(n+1)}(\xi_n)x(x-\xi_n)^{n}
  \right|.
 \]
 Jelikož $\xi_n \in (-1,0)$, platí pro $x \in (-1,0)$ odhad
 \[
  \frac{1+x}{1+\xi_n} \geq 1 + x,
 \]
 z nějž úpravou
 \[
  1 - \frac{1+x}{1 + \xi_n} \leq -x.
 \]
 Čili,
 \begin{align*}
  \eqref{eq:chyba-log} &= \left| \frac{1}{n!}f^{(n+1)}(\xi_n)x(x-\xi_n)^{n}
  \right| = \left| \frac{1}{n!}(-1)^{n}\frac{n!}{(1+\xi_n)^{n+1}}x(x-\xi_n)^{n}
  \right| \\
  &= |x|\frac{(\xi_n - x)^n}{(1 + \xi_n)^{n+1}} = \frac{|x|}{1 + \xi_n} \left(
  \frac{\xi_n - x}{1 + \xi_n} \right)^{n} = \frac{|x|}{1 + \xi_n} \left( 1 -
 \frac{1 + x}{1 + \xi_n} \right)\\
  & \leq (-x)^{n}\frac{|x|}{1 + \xi_n} = \frac{|x|^{n+1}}{1 + \xi_n}
  \overset{n \to \infty}{\longrightarrow} 0,
 \end{align*}
 což zakončuje důkaz kýžené rovnosti pro $x \in (-1,0)$ a tím i důkaz celé věty.
\end{thmproof}
