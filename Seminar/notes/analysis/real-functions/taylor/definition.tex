\section{Definice Taylorova polynomu}
\label{sec:definice-taylorova-polynomu}

\begin{definition}{Taylorův polynom}{tayloruv-polynom}
 Ať $f:M \to \R$ je reálná funkce, majíc konečné derivace všech řádů do $n \in
 \N$ včetně, a $a \in M$. Pak \emph{Taylorovým polynomem stupně $n$ funkce $f$ v
 bodě $a$} rozumíme polynom
 \[
  T^{f,a}_n(x) = \sum_{k=0}^{n} \frac{f^{(k)}(a)}{k!}(x-a)^{k}.
 \]
\end{definition}

\begin{lemma}{Derivace Taylorova polynomu}{derivace-taylorova-polynomu}
 Platí
 \[
  (T^{f,a}_n)^{(1)} = T^{f^{(1)},a}_{n-1}.
 \]
\end{lemma}
\begin{lemproof}
 Z \hyperref[def:tayloruv-polynom]{definice Taylorova polynomu} počítáme
 \begin{align*}
  (T^{f,a}_n)^{(1)}(x) &= \left( \sum_{k=0}^n \frac{f^{(k)}(a)}{k!}(x-a)^{k}
  \right)^{(1)} = \sum_{k=1}^n \frac{k \cdot f^{(k)}(a)}{k!}(x-a)^{k-1}\\
                       &= \sum_{k=1}^{n} \frac{f^{(k)}(a)}{(k-1)!}(x-a)^{k-1}
                       = \sum_{k=0}^{n-1} \frac{(f^{(1)})^{(k)}}{k!}(x-a)^{k} =
                       T^{f^{(1)},a}_{n-1}(x).
 \end{align*}
\end{lemproof}

\begin{proposition}{Aproximace Taylorovým polynomem}{aproximace-taylorovym-polynomem}
 Ať $f:M \to \R$ je reálná funkce, majíc konečné derivace do řádu $n \in \N$
 včetně, a $a \in M$. Pak je $T^{f,a}_n$ aproximací $f$ stupně $n$ na okolí $a$.
\end{proposition}
\begin{propproof}
 Budeme postupovat indukcí podle stupně $n \in \N$. Již víme, že pro $n = 1$ je
 $T^{f,a}_1(x) = f(a) + f^{(1)}(a)(x-a)$ lineární aproximací $f$ na okolí $a$.

 Pro $n > 1$ máme z \hyperref[thm:lhospitalovo-pravidlo]{l'Hospitalova pravidla}
 a \hyperref[lem:derivace-taylorova-polynomu]{předchozího lemmatu}
 \[
  \lim_{x \to a} \frac{f(x) - T^{f,a}_n(x)}{(x-a)^{n}} = \lim_{x \to a}
  \frac{f^{(1)}(x) - T^{f^{(1)},a}_{n-1}(x)}{n(x-a)^{n-1}}.
 \]
 Protože $f^{(1)}$ je reálná funkce a má konečné derivace do řádu $n-1$ včetně,
 je z indukčního předpokladu $T^{f^{(1)},a}_{n-1}$ aproximací $f^{(1)}$ stupně
 $n-1$ na okolí $a$. Platí pročež
 \[
  \lim_{x \to a} \frac{f^{(1)}(x) - T^{f^{(1)},a}_{n-1}(x)}{n(x-a)^{n-1}} = 0,
 \]
 a tedy i
 \[
  \lim_{x \to a} \frac{f(x) - T^{f,a}_n(x)}{(x-a)^{n}} = 0,
 \]
 jak jsme chtěli.
\end{propproof}

Překvapivé možná je, že Taylorův polynom je \emph{jedinou} aproximací funkce $f$
stupně $n$ polynomem stupně nejvýše $n$. K důkazu tohoto faktu si pomůžeme
jedním technickým lemmatem.

\begin{lemma}{}{nulovy-polynom}
 Ať $n \in \N$ a $Q$ je polynom stupně nejvýše $n$. Platí-li $\lim_{x \to a}
 Q(x) / (x-a)^{n} = 0$, pak $Q = 0$.
\end{lemma}

\begin{lemproof}
 Budeme pro spor předpokládat, že $Q$ není nulový. Bez důkazu využijeme tvrzení,
 že když $a$ je kořenem $Q$, pak $x - a \mid Q$. Protože $\lim_{x \to a} Q(x) /
 (x-a)^{n}$, jistě platí $Q(a) = 0$, čili existuje $k \in \N$ takové, že $Q(x) =
 (x-a)^{k} \cdot R(x)$, kde $R$ je polynom nemaje kořen $a$. Pak ale
 \[
  \lim_{x \to a} \frac{Q(x)}{(x-a)^{n}} = \lim_{x \to a}
  \frac{R(x)}{(x-a)^{n-k}}.
 \]
 Tato limita buď neexistuje (pokud $k < n$), nebo je rovna $R(a) \neq 0$ (pokud
 $k = n$). V obou případech je nenulová, což je spor.
\end{lemproof}

\begin{theorem}{Jednoznačnost Taylorova polynomu}{jednoznacnost-taylorova-polynomu}
 Ať $f:M \to \R$ je funkce, majíc konečné derivace do řádu $n \in \N$ včetně, a
 $a \in M$. Předpokládejme, že $P$ je polynom stupně nejvýše $n$, jenž je rovněž
 aproximací $f$ na okolí $a$ stupně $n$. Pak $P = T^{f,a}_n$.
\end{theorem}
\begin{thmproof}
 Podle \myref{tvrzení}{prop:aproximace-taylorovym-polynomem} platí
 \[
  \lim_{x \to a} \frac{f(x) - T^{f,a}_n(x)}{(x-a)^{n}} = 0.
 \]
 Z předpokladu a \hyperref[thm:aritmetika-limit]{věty o aritmetice limit}
 \[
  \lim_{x \to a} \frac{T^{f,a}_n(x) - P(x)}{(x-a)^{n}} = \lim_{x \to a}
  \frac{T^{f,a}_n(x) - f(x)}{(x-a)^{n}} + \lim_{x \to a} \frac{f(x) -
  P(x)}{(x-a)^{n}} = 0 + 0 = 0,
 \]
 čili podle \myref{lemmatu}{lem:nulovy-polynom} jest $P - T^{f,a}_n = 0$. 
\end{thmproof}

\begin{problem}{}{pocitani-taylorova-polynomu}
 Spočtěte $T^{\tan,\pi / 4}_3$, tj. Taylorův polynom stupně $3$ v bodě $\pi / 4$
 funkce $\tan$.
\end{problem}
\begin{probsol}
 Platí
 \begin{align*}
  \tan^{(1)}(x) &= \frac{1}{\cos^2 x},\\
  \tan^{(2)}(x) &= \left( \frac{1}{\cos ^2x} \right)^{(1)} =
  \frac{2\sin x}{\cos^3 x},\\
  \tan^{(3)}(x) &= \left( \frac{2 \sin x}{\cos^3 x} \right)^{(1)} = \frac{2
  \cdot \left(2 - \cos{\left(2 x \right)}\right)}{\cos^{4}{\left(x \right)}}. 
 \end{align*}
 A tedy,
 \[
  \tan \left( \frac{\pi}{4} \right) = 1, \quad \tan^{(1)} \left( \frac{\pi}{4}
   \right) = 2, \quad \tan^{(2)} \left( \frac{\pi}{4} \right) = 4, \quad
   \tan^{(3)} \left( \frac{\pi}{4} \right) = 16.
 \]
 Z čehož již snadno dopočteme
 \begin{align*}
  T^{\tan,\pi / 4}_3(x) &= \sum_{k=0}^3 \frac{\tan^{(k)}(\pi / 4)}{k!}\left( x -
  \frac{\pi}{4}\right)^{k}\\
                        &= 1 + 2 \left( x - \frac{\pi}{4} \right) + 2 \left( x - \frac{\pi}{4}
  \right)^2 + \frac{8}{3} \left( x - \frac{\pi}{4} \right)^3.
 \end{align*}
\end{probsol}
\begin{exercise}{}{vypocet-taylora-sin-cos}
 Spočtěte $T^{\sin \cdot \cos,\pi / 2}_5$.
\end{exercise}
