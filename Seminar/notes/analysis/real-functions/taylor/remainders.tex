\section{Tvary zbytku}
\label{sec:tvary-zbytku}

V této sekci spočteme tzv. \uv{tvary zbytku} Taylorova polynomu. Jsou to výrazy,
které vyjadřují -- aspoň řádově -- velikost chyby při aproximace funkce
Taylorovým polynomem na okolí daného bodu. Budou se hodit primárně při zpytu
poloměru okolí, na němž můžeme stále tvrdit, že Taylorův polynom aproximuje
funkci \uv{dobře}.

\begin{theorem}{Obecný tvar zbytku}{obecny-tvar-zbytku}
 Ať $a, x \in \R$ a $f$ má na $[a,x]$ konečné derivace do řádu $n+1$ včetně. Ať
 je dále $\varphi$ libovolná spojitá funkce na $[a,x]$ s konečnou první derivací
 na $(a,x)$. Pak existuje $\xi \in (a,x)$ takové, že
 \[
  f(x) - T^{f,a}_n(x) = \frac{1}{n!}\frac{\varphi(x) -
  \varphi(a)}{\varphi^{(1)}(\xi)}f^{(n+1)}(\xi)(x-\xi)^{n}.
 \]
\end{theorem}
\begin{thmproof}
 Definujme funkci $F: [a,x] \to \R$ předpisem
 \[
  F(t) \coloneqq f(x) - (f(t) + f^{(1)}(t)(x-t) + \frac{1}{2}f^{(2)}(t)(x-t)^2 +
  \ldots + \frac{1}{n!}f^{(n)}(t)(x-t)^{n}.
 \]
 Pak je $F$ spojitá na $[a,x]$ a $F^{(1)}$ existuje konečná na $(a,x)$. Podle
 \hyperref[thm:cauchyho-o-stredni-hodnote]{Cauchyho věty o střední hodnotě}
 existuje $\xi \in (a,x)$ takové, že
 \begin{equation}
  \label{eq:obecny-zbytek}
  \tag{$\diamondsuit$}
  \frac{F(x) - F(a)}{\varphi(x) - \varphi(a)} =
  \frac{F^{(1)}(\xi)}{\varphi^{(1)}(\xi)}.
 \end{equation}
 
 Snadno spočteme, že
 \begin{align*}
  F^{(1)}(\xi) &= -f^{(1)}(\xi) + f^{(1)}(\xi) - f^{(2)}(\xi) + f^{(2)}(\xi) -
  \ldots - \frac{1}{n!}f^{(n+1)}(\xi)(x-\xi)^{n}\\
               &= -\frac{1}{n!}f^{(n+1)}(\xi)(x - \xi)^{n}.
 \end{align*}
 Zřejmě $F(x) = 0$ a $F(a) = f(x) - T^{f,a}_n(x)$. Čili z
 rovnosti~\eqref{eq:obecny-zbytek} máme
 \[
  \frac{T^{f,a}_n(x) - f(x)}{\varphi(x) - \varphi(a)} =
  -\frac{1}{\varphi^{(1)}(x)} \left( \frac{1}{n!} f^{(n+1)}(\xi)(x-\xi)^{n}
  \right).
 \]
 Odtud přímočarou úpravou plyne tvrzení.
\end{thmproof}

Uvědomme si, že ve \myref{větě}{thm:obecny-tvar-zbytku} je $\varphi$ \emph{zcela
libovolná funkce} s dodatečnými podmínkami spojitosti a diferencovatelnosti. Tím
máme k dispozici celou třídu vyjádření zbytků Taylorova polynomu pouhým
dosazováním za $\varphi$. Dvě konkrétní dosazení (jež sobě dokonce vysloužila
jména) se nám budou v dalším textu hodit více než jiná.

\begin{corollary}{Lagrangeův tvar zbytku}{lagrangeuv-tvar-zbytku}
 Ať $f$ je spojitá na $[a,x]$ a má konečné derivace na $(a,x)$ do řádu $n+1$
 včetně. Pak existuje $\xi \in (a,x)$ takové, že
 \[
  f(x) - T^{f,a}_n(x) = \frac{1}{(n+1)!}f^{(n+1)}(\xi)(x-a)^{n+1}.
 \]
\end{corollary}
\begin{corproof}
 Plyne z dosazení $\varphi(t) = (x-t)^{n+1}$ ve
 \myref{větě}{thm:obecny-tvar-zbytku}.
\end{corproof}

\begin{corollary}{Cauchyho tvar zbytku}{cauchyho-tvar-zbytku}
 Ať $f$ je spojitá na $[a,x]$ a má konečné derivace na $(a,x)$ do řádu $n + 1$
 včetně. Pak existuje $\xi \in (a,x)$ takové, že
 \[
  f(x) - T^{f,a}_n(x) = \frac{1}{n!}f^{(n+1)}(\xi)(x-\xi)^{n}(x-a).
 \]
\end{corollary}
\begin{corproof}
 Plyne z \myref{věty}{thm:obecny-tvar-zbytku} dosazením $\varphi(t) = t$.
\end{corproof}
