\section{Výpočet limit přes Taylorův polynom}
\label{sec:vypocet-limit-pres-tayloruv-polynom}

Fakt, že funkce lze na okolí bodů aproximovat polynomem je mimo jiné užitečný
při výpočtech rozličných limit. Totiž, počítáme-li například limitu v $0$ podílu
funkce a polynomu stupně $4$, jistě není třeba uvažovat Taylorův polynom této
funkce stupně většího než $4$, neboť $x^{n}$ pro $n > 4$ \uv{jde k $0$ mnohem
rychleji} než $x^{4}$. Nejprve trocha formalizmu.

\begin{definition}{Symbol \uv{malé $o$}}{symbol-male-o}
 Ať $f,g:M \to \R$ jsou reálné funkce a $a \in M$. Řekneme, že funkce $f$ je
 \emph{malé $o$} od $g$ v bodě $a$, pokud
 \[
  \lim_{x \to a} \frac{f(x)}{g(x)} = 0.
 \]
 Tento fakt značíme $f(x) = o(g(x)), x \to a$, přičemž příkmetek $x \to a$
 vynecháváme, je-li limitní bod zřejmý z kontextu, a píšeme pouze $f = o(g)$.
\end{definition}
 
\begin{proposition}{Vlastnosti malého $o$}{vlastnosti-maleho-o}
 Ať $f,f_1,f_2,g,g_1,g_2:M \to \R$ jsou reálné funkce a $a \in M$. V
 následujícím výčtu vždy předpokládáme, že právě $a$ je limitním bodem. Platí
 \begin{enumerate}
  \item Je-li $f_1 = o(g)$ a $f_2 = o(g)$, pak $f_1 + f_2 = o(g)$.
  \item Je-li $f_1 = o(g_1)$ a $f_2 = o(g_2)$, pak $f_1f_2 = o(g_1g_2)$.
  \item Je-li $f_1 = o(g)$ a $f_2$ je nenulová na jistém prstencovém okolí $a$,
   pak $f_1f_2 = o(g_1f_2)$.
  \item Je-li $f = o(g_1)$ a $\lim_{x \to a} g_1(x) / g_2(x)$ je konečná, pak $f
   = o(g_2)$.
  \item Je-li $f = o(g)$ a $h:M \to \R$ je omezená na jistém prstencovém okolí
   $a$, pak $hf = o(g)$.
  \item Jsou-li $m \leq n \in \N$ a $f = o((x-a)^{n})$, pak $f = o((x-a)^{m})$.
 \end{enumerate}
\end{proposition}
\begin{propproof}
 Plyne okamžitě z \hyperref[thm:aritmetika-limit-funkci]{věty o aritmetice
 limit}.
\end{propproof}

Raději než budovat komplikovanou teorii, ukážeme výpočet limit přes Taylorův
polynom na několika příkladech.

\begin{problem}{}{taylor-limita-1}
 Spočtěte limitu
 \[
  \lim_{x \to 0} \frac{\cos x - 1 + \frac{1}{2}x^2}{x^{4}}.
 \]
\end{problem}
\begin{probsol}
 Protože pro $x \in \R$ je z definice
 \[
  \cos x = \sum_{n=0}^{\infty} (-1)^{n} \frac{x^{2n}}{(2n)!},
 \]
 je Taylorova řada funkce $\cos$ v libovolném bodě přesně tato. Rozepíšeme si
 její začátek.
 \[
  \cos x = 1 - \frac{x^2}{2} + \frac{x^{4}}{24} + \clr{\sum_{n=3}^{\infty}
  (-1)^{n}\frac{x^{2n}}{(2n)!}}.
 \]
 V \clr{červené řadě} jsou všechna $x$ v mocnině větší než $6$. Platí pročež
 \[
  \sum_{n=3}^{\infty} (-1)^{n}\frac{x^{2n}}{(2n)!} = o(x^{5}).
 \]
 Celkem tedy
 \begin{align*}
  \lim_{x \to 0} \frac{\cos x - 1 + \frac{1}{2}x^2}{x^{4}} &= \lim_{x \to 0}
  \frac{1 - \frac{1}{2}x^2 + \frac{1}{24}x^{4} - 1 + \frac{1}{2}x^2 +
  o(x^{5})}{x_4}\\
                                                           &= \lim_{x \to 0}
                                                 \frac{\frac{1}{24}x^{4}}{x^{4}}
                                                           +
                                                         \frac{o(x^{5})}{x^{4}}
  = \frac{1}{24} + \lim_{x \to 0} \frac{o(x^{5})}{x^{4}} = \frac{1}{24} + 0 =
  \frac{1}{24}.
 \end{align*}
\end{probsol}

\begin{problem}{}{taylor-limita-2}
 Spočtěte limitu
 \[
  \lim_{x \to 0} \frac{1}{x^2} - \cot^2 x.
 \]
\end{problem}
\begin{probsol}
 Upravíme nejprve výraz do tvaru
 \[
  \frac{1}{x^2} - \cot^2 x = \frac{1}{x^2} - \frac{\cos^2 x}{\sin^2 x} =
  \frac{\sin^2 x - x^2\cos^2 x}{x^2\sin^2 x} = \frac{x^2}{\sin^2 x} \cdot
  \frac{\sin^2 x - x^2 \cos^2 x}{x^{4}}.
 \]
 Podle \myref{tvrzení}{prop:bezne-limity} je
 \[
  \lim_{x \to 0} \frac{x^2}{\sin^2x} = 1.
 \]
 Druhou limitu spočteme užitím Taylorova polynomu.

 Je dobré si nejprve rozmyslet, do kterého stupně je třeba Taylorovy polynomy
 zastoupených funkcí počítat. V Taylorově řadě funkce $\cos$ se objevuje
 proměnná v mocninách $0$, $2$, $4$, ... a v~řadě $\sin$ v mocninách $1$, $3$,
 $5$, ... Obě tyto funkce jsou v čitateli na druhou a vyděleny $x^{4}$. Jejich
 první dva nenulové členy musejí stačit. Spočteme tedy Taylorovy polynomy stupně
 $3$ funkcí $\sin$ a $\cos$ v bodě $0$.
 \begin{align*}
  T^{\sin,0}_3(x) &= x - \frac{1}{6}x^3,\\
  T^{\cos,0}_3(x) &= 1 - \frac{1}{2}x^2.
 \end{align*}
 Tedy,
 \begin{align*}
  \lim_{x \to 0} \frac{\sin^2x - x^2\cos^2 x}{x^{4}} &= \frac{\left( x -
  \frac{1}{6}x^3 + o(x^{4}) \right)^2 - x^2 \left( 1 - \frac{1}{4}x^2 + o(x^3)
  \right)^2}{x^{4}}\\
  &= \lim_{x \to 0} \frac{x^2 - \frac{1}{3}x^{4} + o(x^{4}) - x^2 + x^{4} +
  o(x^{4})}{x^{4}}\\
  &= \lim_{x \to 0} \frac{-\frac{1}{3}x^{4} + x^{4}}{x^{4}} +
  \lim_{x \to 0} \frac{o(x^{4})}{x^{4}} = \frac{-\frac{1}{3} + 1}{1} + 0 =
  \frac{2}{3}.
 \end{align*}
\end{probsol}
\begin{problem}{}{taylor-limita-3}
 Spočtěte limitu
 \[
  \lim_{x \to 0} \frac{\exp(x^3) - 1}{\sin x - x}.
 \]
\end{problem}
\begin{probsol}
 Platí
 \begin{align*}
  \exp(x^3) &= \sum_{n=0}^{\infty} \frac{(x^{n})^3}{n!} = 1 + x^3 + o(x^3),\\
  \sin x &= \sum_{n=0}^{\infty} (-1)^{n}\frac{x^{2n+1}}{(2n+1)!} = x -
  \frac{1}{6}x^3 + o(x^3).
 \end{align*}
 Čili,
 \begin{align*}
  \lim_{x \to 0} \frac{\exp(x^3) - 1}{\sin x - x} &= \lim_{x \to 0} \frac{1 + x^3
  + o(x^3) - 1}{x - \frac{1}{6}x^3 + o(x^3) - x}\\
  &= \lim_{x \to 0} \frac{x^3}{-\frac{1}{6}x^3 + o(x^3)} + \lim_{x \to 0}
  \frac{o(x^3)}{-\frac{1}{6}x^3 + o(x^3)} = -6 + 0 = -6.
 \end{align*}
\end{probsol}

\begin{exercise}{}{taylor-limity}
 Spočtěte následující limity.
 \[
  {\everymath={\displaystyle}
   \arraycolsep=1em
   \begin{array}{cc}
    \lim_{x \to 0} \frac{\cos x - \exp(-x^2 / 2)}{x^{4}} & \lim_{x \to 0}
    \frac{1}{x} - \frac{1}{\sin x}\\[2em]
    \lim_{x \to 0} \frac{\tan x - x}{x - \sin x} & \lim_{x \to 0} \frac{2(\sin x
    - \tan x) + x^3}{(\exp x - 1)(\exp(-x^2)-1)^2}
   \end{array}
  }
 \]
 
\end{exercise}
