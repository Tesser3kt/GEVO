\section{Riemannův integrál}
\label{sec:riemannuv-integral}

V této sekci si ukážeme velmi intuitivní způsob, jak počítat \emph{orientovanou}
plochu sevřenou grafem dané funkce a osou $x$. Spojením \emph{orientovaná
plocha} zde neznačí to, jež bychom přirozeně nazvali \uv{obsahem} útvaru mezi
grafem funkce a osou $x$, nýbrž rozdíl obsahu útvaru takto vzniknuvšího
\textbf{nad} osou $x$ a toho \textbf{pod} osou $x$. Fyzikální využití takového
konceptu jsou výrazně širší.

Je velmi překvapivé, že obsah pod grafem funkce jakkoli souvisí s její
primitivní funkcí. Toto spojení, jeho příčina a intuitivní vysvětlení budou
zveřejněny v sekci o Newtonově integrálu.

Riemannův integrál je způsob výpočtu řečené plochy, který a priori s primitivní
funkcí nesouvisí nijak. Pracuje na přirozeném principu aproximace plochy pod
grafem funkce stále větším množstvím stále užších obdélníku (útvarů, jejichž
obsah jsme schopni triviálně spočíst) jednou seshora, jednou zespoda. Blíží-li
se zvyšováním počtu obdélníků k sobě tyto aproximace, jejich společná limita je
právě plochou pod grafem funkce. K formalizaci tohoto odstavce potřebujeme
několik úvodních definic.

\begin{definition}{Dělení intervalu}{deleni-intervalu}
 Ať $[a,b] \subseteq \R$ je interval. Jeho \emph{dělením} nazveme libovolnou
 konečnou posloupnost $x_0,\ldots,x_n$ pro $n \in \N$, kde $x_0 = a, x_n = b$ a
 platí $x_{i-1} < x_{i}$ pro každé $1 \leq i \leq n$.

 Takovou posloupnost značíme $D = (x_i)_{i=0}^{n}$ a délku nejdelšího intervalu
 tvaru $[x_i,x_{i+1}]$ nazveme \emph{normou} $D$, značenou
 \[
  \|D\| \coloneqq \max_{1 \leq i \leq n} (x_{i} - x_{i-1}).
 \]
 Jsou-li $D',D$ dvě dělení $[a,b]$, pak říkáme, že $D'$ \emph{zjemňuje} $D$
 (značíme $D' \leq D$), patří-li každý dělící bod $D$ rovněž do $D'$.
 Speciálně, $D$ zjemňuje samo sebe.
\end{definition}

\begin{definition}{Horní a dolní součty}{horni-a-dolni-soucty}
 Ať $[a,b] \subseteq \R$ je interval, $f:[a,b] \to \R$ \textbf{omezená} funkce a
 $D$ dělení $[a,b]$. Definujeme hodnotu
 \begin{align*}
  \overline{S}(f,D) &\coloneqq \sum_{i=0}^n \sup_{[x_{i-1},x_i]} f \cdot (x_i -
  x_{i-1}), \text{ resp.}\\
  \underline{S}(f,D) & \coloneqq \sum_{i=0}^n \inf_{[x_{i-1},x_i]} f \cdot (x_i
  - x_{i-1}),
 \end{align*}
 a nazýváme ji \emph{horním}, resp. \emph{dolním}, součtem $f$ při dělení $D$.
 Zde výraz $\sup_{[x_{i-1},x_i]} f$, resp. $\inf_{[x_{i-1},x_i]} f$, značí
 supremum, resp. infimum, funkce $f$ na intervalu $[x_{i-1},x_i]$.
\end{definition}

Užitím \hyperref[def:horni-a-dolni-soucty]{horních a dolních součtů} je již
přímočaré definovat \uv{plochu pod funkcí}. Totiž, horní součty jsou právě
aproximací grafu $f$ seshora posloupností obdélníků o šířkách přesně
odpovídajících délkám dělících intervalů (tj. $x_i - x_{i-1}$) a výškách
$\sup_{[x_{i-1},x_i]} f$, a dolní součty zase aproximací grafu $f$ zespoda
obdélníky o stejných šířkách a výškách $\inf_{[x_{i-1},x_i]} f$.

Nyní vezmeme \emph{nejlepší možné} aproximace plochy pod grafem $f$ seshora i
zespoda. Uděláme to tak, že \uv{vyzkoušíme} úplně všechna dělení $[a,b]$ a z
nich vybereme to, jež dává nejmenší horní součty. Formálně, vezmeme infimum
horních součtů přes všechna možná dělení $[a,b]$. Z druhé strany zase supremum
dolních součtů přes všechna dělení. Je dobré si rozmyslet, že je-li plocha pod
grafem funkce \uv{aproximovatelná} pomocí obdélníků, pak tento postup skutečně
dává kýženou hodnotu orientované plochy.

\begin{definition}{Riemannův integrál}{riemannuv-integral}
 Ať $[a,b] \subseteq \R$ je interval a $f:[a,b] \to \R$ \textbf{omezená} funkce.
 Definujeme hodnotu
 \begin{align*}
  \overline{\int_a^{b}} f &\coloneqq \inf \{\overline{S}(f,D) \mid D \text{
  dělení } [a,b]\}, \text{ resp.},\\
   \underline{\int_{a}^{b}} f &\coloneqq \sup \{\underline{S}(f,D) \mid D \text{
   dělení } [a,b]\},
 \end{align*}
 a nazýváme ji \emph{horním Riemannovým integrálem}, resp. \emph{dolním
 Riemannovým integrálem}, funkce $f$ na $[a,b]$.

 Platí-li
 \[
  \overline{\int_{a}^{b}} f = \underline{\int_{a}^{b}} f,
 \]
 říkáme, že funkce $f$ \emph{má Riemannův integrál} na $[a,b]$, což značíme
 $f \in \mathcal{R}(a,b)$. Společnou hodnotu obou integrálů značíme zkrátka
 $\int_{a}^{b} f$ a říkáme ji \emph{Riemannův integrál} funkce $f$ na $[a,b]$.
\end{definition}

\begin{warning}{}{spojita-funkce-a-uzavreny-interval}
 Riemannův integrál je definován pouze pro \textbf{omezenou} funkci na
 \textbf{uzavřeném intervalu}.

 Důvodem předpokladu omezenosti je potřeba dobré definovanosti horních a dolních
 součtů. U neomezených funkcí mohou být hodnoty $\sup_{[x_{i-1},x_i]} f$ a
 $\inf_{[x_{i-1},x_i]} f$ totiž nekonečné.

 Předpoklad uzavřenosti intervalu existuje v zásadě ze stejného důvodu. Při
 definici dělení otevřeného intervalu a následně součtů dané funkce při tomto
 dělení by byl jeden nucen studovat limitní chování takové funkce v jeho
 krajních bodech.
\end{warning}
