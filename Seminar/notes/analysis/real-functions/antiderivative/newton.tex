\section{Newtonův integrál}
\label{sec:newtonuv-integral}

\hyperref[thm:zakladni-veta-kalkulu]{Základní věta kalkulu} nabízí druhou metodu
výpočtu orientované plochy pod danou funkcí. Jsou-li splněny její předpoklady,
pak
\begin{equation}
 \label{eq:newtonuv-integral}
 \int_{a}^{b} f = F(b) - F(a)
\end{equation}
pro reálnou funkci $F$ definovanou jako $F(x) \coloneqq \int_{c}^{x} f$ s jistou
konstantou $c \in (a,b)$ a všechna $x \in [a,b]$. Přestože historicky je
\uv{definice} integrálu v \eqref{eq:newtonuv-integral} ta původní, my k ní
docházíme teprve prostřednictvím integrálu Riemannova. Lze tvrdit, že tato cesta
je více přirozená, stavějíc intuitivní definici plochy před tu odvozenou z
vlastností primitivních funkcí, jež -- na první pohled zcela jistě -- s
orientovanou plochou nikterak nesouvisí.

Výhodou takto definovaného integrálu je jeho existence i pro jisté neomezené
funkce na otevřených intervalech -- v \eqref{eq:newtonuv-integral} lze totiž
nahradit $F(b)$ a $F(a)$ příslušnými jednostrannými limitami, jak též záhy
učiníme. Na druhou stranu se takto omezujeme na funkce, k nimž existuje funkce
primitivní. Takto definovaný integrál sluje \emph{Newtonův}.

\begin{definition}{Newtonův integrál}{newtonuv-integral}
 Ať $a,b \in \R^{*}$. Řekneme, že funkce $f:(a,b) \to \R$ je \emph{newtonovsky}
 integrovatelná, píšeme $f \in \mathcal{N}(a,b)$, pakližeť
 \begin{enumerate}
  \item má funkce $f$ na $(a,b)$ primitivní funkci $F$,
  \item rozdíl $\lim_{x \to b^{-}} F(x) - \lim_{x \to a^{+}} F(x)$ je definován.
 \end{enumerate}
 Jsou-li splněny obě podmínky, je tento rozdíl nazýván hodnotou \emph{Newtonova
 integrálu} z funkce $f$ na $(a,b)$. Symbolicky
 \[
  (\mathcal{N}) \int_{a}^{b} f \coloneqq \lim_{x \to b^{-}} F(x) - \lim_{x \to
  a^{+}} F(x).
 \]
 Pro stručnost zápisu budeme nadále onen rozdíl značit symbolem $[F]_a^{b}$.
\end{definition}

\begin{remark}{}{newtonuv-integral}
 V dalším textu budeme muset jistou chvíli rozlišovat mezi Newtonovým a
 Riemannovým integrálem -- přinejmenším, dokud neukážeme jejich ekvivalenci pro
 funkce newtonovsky i riemannovsky integrovatelné. Učiníme to uvedením písmene
 $(\mathcal{N})$ či $(\mathcal{R})$ před samotný symbol integrálu, jako v
 \hyperref[def:newtonuv-integral]{předchozí definici}, v případě, že uvažovaný
 typ integrálu není zřejmý z kontextu. V celém zbytku oddílu, není-li uvedeno
 jinak, značí symbol $\int_{a}^{b}$ integrál \emph{Newtonův}.
\end{remark}

\begin{warning}{}{newtonuv-integral-nekonecno}
 Je dobré vzít na mysl, že Newtonův integrál z dané funkce může být i nekonečný.
 Podmínkou jeho existence byla pouze \emph{definovanost} výrazu $[F]_a^{b}$,
 nikoli jeho \emph{konečnost}. V této souvislosti si všimněme, že jeho meze
 mohou z \hyperref[def:newtonuv-integral]{definice} nabývat i hodnot $ \pm
 \infty$. Tím se přímo liší od integrálu Riemannova, neboť omezenost funkce v
 tomto případě konečnost horních i dolních součtů zaručuje.
\end{warning}

\begin{example}{}{newtonuv-integral-polynomu}
 Platí
 \[
  \int_{0}^{1} x^{\alpha} \, \mathrm{d}x = \begin{cases}
   \left[ \frac{1}{\alpha+1}x^{\alpha+1} \right]_0^{1} = \frac{1}{\alpha+1},
   & \quad \text{pro } \alpha \in (-1,\infty); \\
   \left[ \frac{1}{\alpha+1}x^{\alpha+1} \right]_0^{1} = -\infty, & \quad
   \text{pro } \alpha \in (-\infty,-1); \\
   [\log x]_0^{1} = \infty,& \quad \text{pro } \alpha = -1.
  \end{cases}
 \]
 Vskutku, primitivní funkce k $x^{\alpha}$ byla spočtena v
 \myref{příkladě}{exam:integral-polynomu} a výpočet příslušných limit je
 triviální.
\end{example}

\begin{example}{}{newton-vs-riemann}
 \begin{enumerate}
  \item Funkce $f(x) = 1 / x$ je na intervalu $(0,1)$ newtonovsky
   integrovatelná, konkrétně
   \[
    (\mathcal{N}) \int_{0}^{1} \frac{1}{x} \, \mathrm{d}x = \infty,
   \]
   podle \myref{příkladu}{exam:newtonuv-integral-polynomu} pro $\alpha = -1$,
   ale není na $[0,1]$ riemannovsky integrovatelná, protože na tomto intervalu
   není omezená.
  \item Naopak, funkce $\sign x$ je na intervalu $[-1,1]$ riemannovsky
   integrovatelná, díky \myref{tvrzení}{prop:integrovatelnost-monotonni-funkce},
   protože je na $[-1,1]$ monotónní. Avšak, na $(-1,1)$ newtonovsky
   integrovatelná není, neboť nemá na celém tomto intervalu primitivní funkci.
 \end{enumerate}
\end{example}

Newtonův integrál je člověku výpočetně bližší než integrál Riemannův; je k němu
třeba umět hledat primitivní funkci. Naopak, Riemannův integrál je mnohem blíže
způsobu, kterým počítače vyhodnocují integrály, neboť limity konvergentních
posloupností (v tomto případě součtů přes zjemňující se dělení) se aproximují
triviálně. Je rozumné se domnívat, že metody výpočtu primitivních funkcí
představené v \myref{sekci}{sec:vypocet-primitivnich-funkci} lze úspěšně použít
i k výpočtu Newtonových integrálů.

\begin{theorem}{Linearita Newtonova integrálu}{linearita-newtonova-integralu}
 Ať $a,b \in \R^{*}, f,g \in \mathcal{N}(a,b)$ a $c \in \R$. Pak $f + g, cf \in
 \mathcal{N}(a,b)$ a platí
 \begin{align*}
  \int_{a}^{b} f(x) + g(x) \, \mathrm{d}x &= \int_{a}^{b} f(x) \, \mathrm{d}x +
  \int_{a}^{b} g(x) \, \mathrm{d}x.\\
  \int_{a}^{b} cf(x) \, \mathrm{d}x &= c \int_{a}^{b} f(x) \, \mathrm{d}x.
 \end{align*}
\end{theorem}
\begin{thmproof}
 Ať $F,G$ jsou primitivní k $f,g$ na $(a,b)$. Potom jsou $F + G$ a $cF$
 primitivní k $f + g$ a $cf$ na $(a,b)$ a z
 \hyperref[thm:aritmetika-limit-funkci]{věty o aritmetice limit} máme
 \[
  [F + G]_a^{b} = [F]_a^{b} + [G]_a^{b} \quad \text{a} \quad [cF]_a^{b} =
  c[F]_a^{b}.
 \]
 Odtud,
 \begin{align*}
  \int_{a}^{b} f(x) + g(x) \, \mathrm{d}x 
  &= [F + G]_a^{b} = [F]_a^{b} + [G]_a^{b} = \int_{a}^{b} f(x) \, \mathrm{d}x +
  \int_{a}^{b} g(x) \, \mathrm{d}x,\\
  \int_{a}^{b} cf(x) \, \mathrm{d}x 
  &= [cF]_a^{b} = c[F]_a^{b} = c \int_{a}^{b} f(x) \, \mathrm{d}x,
 \end{align*}
 jak bylo dokázati.
\end{thmproof}

\begin{theorem}{Per partes pro Newtonův integrál}{per-partes-pro-newtonuv-integral}
 Ať $a,b \in \R^{*}$, $f,g:(a,b) \to \R$ jsou reálné funkce a $F$, resp. $G$, je
 primitivní k $f$, resp. $g$, na $(a,b)$ Potom platí
 \[
  \int_{a}^{b} f(x)G(x) \, \mathrm{d}x = [FG]_{a}^{b} - \int_{a}^{b} F(x)g(x) \,
  \mathrm{d}x,
 \]
 dává-li pravá strana smysl.
\end{theorem}
\begin{thmproof}
 Protože pravá strana je z předpokladu dobře definována, existuje primitivní
 funkce k $Fg$ na $(a,b)$. Označme ji $H$. Platí
 \[
  (FG - H)' = fG + Fg - Fg = fG,
 \]
 čili $FG - H$ je primitivní k $fG$ na $(a,b)$. Dále
 \begin{align*}
  \int_{a}^{b} f(x)G(x) \, \mathrm{d}x
  &= [FG - H]_a^{b} = \lim_{x \to b^{+}} (F(x)G(x) - H(x)) - \lim_{x \to a^{+}}
  (F(x)G(x) - H(x))\\
  &= \lim_{x \to b^{-}} F(x)G(x) - \lim_{x \to b^{-}} F(x)G(x) - (\lim_{x \to
  b^{-}} H(x) - \lim_{x \to a^{+}} H(x))\\
  &= [FG]_a^{b} - [H]_a^{b} = [FG]_a^{b} - \int_{a}^{b} F(x)g(x) \,
  \mathrm{d}x,
 \end{align*}
 čímž je důkaz hotov.
\end{thmproof}

\begin{theorem}{Substituce pro Newtonův integrál}{substituce-pro-newtonuv-integral}
 Nechť $a,b,\alpha,\beta \in \R^{*}$, $f:(a,b) \to \R$ a $\varphi:(\alpha,\beta)
 \to (a,b)$. Ať má $\varphi$ dále konečnou nenulovou derivaci na
 $(\alpha,\beta)$ a platí $\varphi((\alpha,\beta)) = (a,b)$. Potom
 \[
  \int_{a}^{b} f(x) \, \mathrm{d}x = \int_{\alpha}^{\beta} (f \circ \varphi)(t)
  |\varphi'(t)|\,
  \mathrm{d}t,
 \]
 dává-li aspoň jedna strana smysl.
\end{theorem}
\begin{thmproof}
 Funkce $\varphi$ má na $(\alpha,\beta)$ z předpokladu konečnou derivaci. Z
 \myref{věty}{thm:darbouxova} je $\varphi'((\alpha,\beta))$ interval a opět z
 předpokladu $0 \notin \varphi'((\alpha,\beta))$. Tedy, $\varphi'$ je na
 $(\alpha,\beta)$ buď záporná, nebo kladná. Bez újmy na obecnosti budeme
 předpokládat, že $\varphi' > 0$ na $(\alpha,\beta)$, čili $\varphi$ je na
 $(\alpha,\beta)$ rostoucí. Rozlišíme dvě možnosti: budeme předpokládat, že
 existuje levá strana rovnosti a dokážeme existenci integrálu na pravé straně, a
 naopak.

 Ať nejprve existuje $\int_{a}^{b} f$. Pak má $f$ na $(a,b)$ primitivní funkci
 $F$ a existují limity $\lim_{x \to a^{+}} F(x)$ i $\lim_{x \to b^{-}} F(x)$. Z
 \hyperref[thm:derivace-slozene-funkce]{věty o derivaci složené funkce}
 ($\varphi$ je rostoucí, tedy prostá), má i $(f \circ \varphi) \cdot \varphi'$
 na intervalu $(\alpha,\beta)$ primitivní funkci, konkrétně funkci $F \circ
 \varphi$. Platí
 \[
  \lim_{t \to \beta^{-}} \varphi(t) = b \quad \text{a} \quad \lim_{t \to \alpha
  ^{+}} \varphi(t) = a.
 \]
 Z \hyperref[thm:limita-slozene-funkce]{věty o limitě složené funkce} (opět,
 $\varphi$ je prostá) máme
 \[
  \lim_{t \to \beta^{-}} (F \circ \varphi)(t) = \lim_{x \to b^{-}} F(x) \quad
  \text{a} \quad \lim_{t \to \alpha ^{+}} (F \circ \varphi)(t) = \lim_{x \to
  a^{+}} F(x).
 \]
 Můžeme pročež počítat
 \begin{align*}
  \int_{\alpha}^{\beta} (f \circ \varphi)(t)|\varphi'(t)| \, \mathrm{d}t 
  &= \int_{\alpha}^{\beta} (f \circ \varphi)(t) \varphi'(t) \, \mathrm{d}t =
  [F \circ \varphi]_{\alpha}^{\beta}\\
  &= [F]_a^{b} = \int_{a}^{b} f(x) \, \mathrm{d}x.
 \end{align*}

 Nyní ať existuje $\int_{\alpha}^{\beta} (f \circ \varphi) \cdot |\varphi'|$.
 Označme $G$ primitivní funkci k $(f \circ \varphi) \cdot |\varphi'| = (f \circ
 \varphi) \cdot \varphi'$. Z \hyperref[thm:druha-o-substituci]{druhé věty o
 substituci} plyne, že $G \circ \varphi ^{-1}$ je primitivní k $f$ na $(a,b)$.
 Platí
 \[
  \lim_{x \to a^{+}} \varphi ^{-1}(x) = \alpha \quad \text{a} \quad \lim_{x \to
  b^{-}} \varphi ^{-1}(x) = \beta,
 \]
 a tedy
 \[
  \lim_{x \to a^{+}} (G \circ \varphi ^{-1})(x) = \lim_{t \to \alpha ^{+}}
  G(t) \quad \text{a} \quad \lim_{x \to b^{-}} (G \circ \varphi ^{-1})(x) =
  \lim_{t \to b^{-}} G(t).
 \]
 Dopočteme
 \[
  \int_{a}^{b} f(x) \, \mathrm{d}x = [G \circ \varphi ^{-1}]_a^{b} =
  [G]_{\alpha}^{\beta} = \int_{\alpha}^{\beta} (f \circ \varphi)(t)|\varphi'(t)|
  \, \mathrm{d}t,
 \]
 jak jsme chtěli.
\end{thmproof}

\begin{problem}{}{newtonuv-integral-1}
 Spočtěte
 \[
  \int_{-1}^{1} x^2 \exp(-x) \, \mathrm{d}x.
 \]
\end{problem}
\begin{probsol}
 Položme $f(x) = \exp(-x)$ a $G(x) = x^2$. Pak $F(x) = -\exp(-x)$ a $g(x) = 2x$.
 Podle \myref{věty}{thm:per-partes-pro-newtonuv-integral} platí
 \[
  \int_{-1}^{1} x^2 \exp(-x) \, \mathrm{d}x = [-x^2 \exp(-x)]_{-1}^{1} -
  \int_{-1}^{1} -2x \exp(-x) \, \mathrm{d}x.
 \]
 Jistě
 \[
  [-x^2 \exp(-x)]_{-1}^{1} = -1 \cdot \exp(-1) - (-1)\exp(1) = -\frac{1}{e} + e.
 \]
 Z \hyperref[thm:linearita-newtonova-integralu]{linearity integrálu} platí
 \[
  \int_{-1}^{1} -2x \exp(-x) \, \mathrm{d}x = -2 \int_{-1}^{1} x \exp(-x) \,
  \mathrm{d}x.
 \]
 Opětovným použitím \hyperref[thm:per-partes-pro-newtonuv-integral]{per partes}
 dostaneme
 \[
  \int_{-1}^{1} x \exp(-x) \, \mathrm{d}x = [-x \exp(-x)]_{-1}^{1} -
  \int_{-1}^{1} -\exp(-x) \, \mathrm{d}x.
 \]
 Snadno spočteme, že
 \begin{align*}
  [-x \exp(-x)]_{-1}^{1} &= -1 \cdot \exp(-1) - 1 \cdot \exp(1) = -\frac{1}{e} -
  e,\\
  \int_{-1}^{1} -\exp(-x) \, \mathrm{d}x &= [\exp(-x)]_{-1}^{1} = \exp(-1) -
  \exp(1) = \frac{1}{e} - e.
 \end{align*}
 Celkem tedy,
 \[
  \int_{-1}^{1} x\exp(-x) \, \mathrm{d}x = -\frac{1}{e} - e - \frac{1}{e} + e =
  -\frac{2}{e}.
 \]
 Dopočteme,
 \begin{align*}
  \int_{-1}^{1} x^2\exp(-x) \, \mathrm{d}x 
  &= [-x^2 \exp(-x)]_{-1}^{1} - \int_{-1}^{1} -2x \exp(-x) \, \mathrm{d}x\\
  &= -\frac{1}{e} + e + 2 \left( -\frac{2}{e} \right) = -\frac{5}{e} + e.
 \end{align*}
\end{probsol}
