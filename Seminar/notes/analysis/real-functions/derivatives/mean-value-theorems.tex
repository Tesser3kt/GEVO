\section{Věty o střední hodnotě}
\label{sec:vety-o-stredni-hodnote}

Spojitá funkce, jež ve dvou různých bodech nabývá stejných hodnot, musí mít na
nějakém místě mezi těmito body konstantní růst -- přestat růst a začít klesat či
přestat klesat a začít růst. Spojitá funkce musí někde mezi libovolnými dvěma
body mít tečnu rovnoběžnou k úsečce spojující odpovídající body v grafu této
funkce. Stejný závěr platí i pro spojitou křivku v prostoru.

Tato tvrzení souhrnně slují \emph{věty o střední hodnotě}. Přestože je jejich
geometrický význam lákavý, samy o sobě mnoha účelům neslouží. Jejich důležitost
dlí spíše v rozvoji další teorie derivací. Formulujeme a dokážeme si je.

\begin{theorem}{Rolleova věta o střední hodnotě}{rolleova-veta-o-stredni-hodnote}
 Ať $a < b$, $f:[a,b] \to \R$ je spojitá funkce, $f(a) = f(b)$ a $f$ má derivaci
 v každém bodě $(a,b)$. Pak existuje $c \in (a,b)$ takové, že $f'(c) = 0$.
\end{theorem}
\begin{thmproof}
 Podle \myref{věty}{thm:extremy-spojite-funkce} nabývá $f$ na $[a,b]$ svého
 minima $m$ a maxima $M$. Pak zřejmě
 \begin{equation*}
  \label{eq:rolle}
  \tag{$*$}
  m \leq f(a) \leq f(b) \leq M.
 \end{equation*}
 Pokud $m = M$, pak je $f$ konstantní na $[a,b]$, a tudíž $f'(c) = 0$ dokonce
 pro všechna $c \in (a,b)$.

 Předpokládejme, že $m < M$. Pak musí být aspoň jedna z nerovností
 v~\eqref{eq:rolle} ostrá. Bez újmy na obecnosti, ať $f(b) < M$. Ať $c \in
 (a,b)$ je takové, že $f(c) = M$. Dle předpokladu nabývá $f$ v bodě $c$ maxima
 na $[a,b]$, a tedy podle \myref{tvrzení}{prop:vztah-derivace-a-extremu} platí
 $f'(c) = 0$. V případě, kdy $m < f(a)$ postupujeme obdobně.
\end{thmproof}

\begin{theorem}{Lagrangeova o střední hodnotě}{lagrangeova-o-stredni-hodnote}
 Ať $a < b$ a $f:[a,b] \to \R$ je spojitá na $[a,b]$ majíc derivaci v každém
 bodě $(a,b)$. Potom existuje $c$ takové, že
 \[
  f'(c) = \frac{f(b) - f(a)}{b - a}.
 \]
\end{theorem}
\begin{thmproof}
 Převedeme problém na použití
 \hyperref[thm:rolleova-veta-o-stredni-hodnote]{Rolleovy věty}. Definujme funkci
 \[
  g(x) \coloneqq f(x) - \frac{f(b) - f(a)}{b-a} (x-a).
 \]
 Potom je $g$ spojitá na $[a,b]$, má derivaci v každém bodě $(a,b)$ a platí
 $g(a) = g(b)$. Z \hyperref[thm:rolleova-veta-o-stredni-hodnote]{Rolleovy
 věty} plyne, že existuje $c \in (a,b)$ takové, že $g'(c) = 0$. Tato rovnost
 rozepsána znamená, že
 \[
  f'(c) - \frac{f(b) - f(a)}{b - a} = 0,
 \]
 čili
 \[
  f'(c) = \frac{f(b) - f(a)}{b-a},
 \]
 jak jsme chtěli.
\end{thmproof}

\begin{remark}{}{lagrangeova-veta}
 \hyperref[thm:lagrangeova-o-stredni-hodnote]{Lagrangeova věta} říká, že v
 jistém bodě $c \in (a,b)$ musí být derivace $f$ v bodě $c$ rovna směrnici
 přímky spojující body $(a,f(a))$ a $(b,f(b))$.
\end{remark}

\begin{corollary}{Vztah derivace a monotonie}{vztah-derivace-a-monotonie}
 Ať $I \subseteq \R$ je interval a $f:I \to \R$ je spojitá funkce majíc v každém
 bodě $(a,b)$ kladnou, resp. zápornou, derivaci. Pak je $f$ rostoucí, resp.
 klesající, na $I$.
\end{corollary}
\begin{corproof}
 Volme $[a,b] \subseteq I$ libovolně a předpokládejme, že $f'$ je kladná na $I$.
 Podle \hyperref[thm:lagrangeova-o-stredni-hodnote]{Lagrangeovy věty} existuje
 $c \in (a,b)$ takové, že
 \[
  f'(c) = \frac{f(b) - f(a)}{b - a}.
 \]
 Ježto $f'(c) > 0$, plyne odtud, že $f(b) > f(a)$. Čili $f$ je rostoucí na
 $[a,b]$. Jelikož $a < b \in I$ byla volena libovolně, je $f$ rostoucí na $I$.
 Případ $f' < 0$ na $I$ se ošetří analogicky.
\end{corproof}

\begin{exercise}{}{derivace-nula-konstantni}
 Použijte \myref{důsledek}{cor:vztah-derivace-a-monotonie} k důkazu, že spojitá
 funkce $f:I \to \R$ mající nulovou derivaci na $I$, je konstantní.
\end{exercise}

\begin{theorem}{Cauchyho o střední hodnotě}{cauchyho-o-stredni-hodnote}
 Ať $f,g$ jsou spojité funkce na $[a,b] \subseteq \R$, $f$ má v každém bodě
 $(a,b)$ derivaci a $g$ má v každém bodě $(a,b)$ \textbf{konečnou nenulovou}
 derivaci. Potom $g(a) \neq g(b)$ a existuje $c \in (a,b)$ takové, že
 \[
  \frac{f'(c)}{g'(c)} = \frac{f(b) - f(a)}{g(b) - g(a)}.
 \]
\end{theorem}
\begin{thmproof}
 Z \hyperref[thm:lagrangeova-o-stredni-hodnote]{Lagrangeovy věty} plyne
 existence $d \in (a,b)$ takového, že
 \[
  g'(d) = \frac{g(b) - g(a)}{b - a}.
 \]
 Jelikož z předpokladu $g'(d) \neq 0$, rovněž $g(b) \neq g(a)$.

 Opět převedeme problém na
 \hyperref[thm:rolleova-veta-o-stredni-hodnote]{Rolleovu větu}. Definujme funkci
 \[
  \varphi(x) \coloneqq (f(x) - f(a))(g(b) - g(a)) - (g(x) - g(a))(f(b) - f(a)).
 \]
 Pak je $\varphi$ spojitá na $[a,b]$ (neboť $f$ a $g$ jsou tamže spojité) a má v
 každém bodě $(a,b)$ derivaci -- to plyne z
 \hyperref[thm:aritmetika-derivaci]{věty o aritmetice derivací} a faktu, že $f$
 i $g$ mají na $(a,b)$ derivaci.

 Navíc, $\varphi(a) = \varphi(b) = 0$. Z
 \hyperref[thm:rolleova-veta-o-stredni-hodnote]{Rolleovy věty} existuje $c \in
 (a,b)$ splňující $\varphi'(c) = 0$. Platí
 \[
  0 = \varphi'(c) = f'(c)(g(b) - g(a)) - g'(c)(f(b) - f(a)).
 \]
 Odtud úpravou
 \[
  \frac{f'(c)}{g'(c)} = \frac{f(b) - f(a)}{g(b) - g(a)},
 \]
 což zakončuje důkaz.
\end{thmproof}
\begin{remark}{}{cauchyho-veta}
 \hyperref[thm:cauchyho-o-stredni-hodnote]{Cauchyho věta} říká, že křivka v
 rovině $t \mapsto (f(t),g(t))$ má v jistém bodě $c \in (a,b)$ derivaci rovnou
 směrnici přímky spojující body $(f(a),g(a))$ a $(f(b),g(b))$.
\end{remark}
\begin{remark}{}{vety-o-stredni-hodnote}
 Uvědomme si, že \hyperref[thm:lagrangeova-o-stredni-hodnote]{Lagrangeova věta}
 je speciálním případem \hyperref[thm:cauchyho-o-stredni-hodnote]{Cauchyho věty}
 pro $g(x) = x$ a \hyperref[thm:rolleova-veta-o-stredni-hodnote]{Rolleova věta}
 je zase speciálním případem
 \hyperref[thm:lagrangeova-o-stredni-hodnote]{Lagrangeovy věty}, když $f(a) =
 f(b)$. Ovšem, k důkazu jak
 \hyperref[thm:lagrangeova-o-stredni-hodnote]{Lagrangeovy věty}, tak
 \hyperref[thm:cauchyho-o-stredni-hodnote]{Cauchyho věty}, jsme použili téměř
 výhradně \hyperref[thm:rolleova-veta-o-stredni-hodnote]{Rolleovu větu}. Jedná
 se pročež o vzájemně ekvivalentní tvrzení, ač to tak na první pohled nevypadá.
\end{remark}
