\chapter{Taylorův polynom}
\label{chap:tayloruv-polynom}

% TODO
\clr{\textbf{Tato kapitola se nachází v pracovní verzi. Neočekávejte obrázky,
		naopak očekávejte chyby a podivné formulace.}}

Polynomy jsou hezké funkce. Dají se donekonečna derivovat -- všechny tyto
derivace jsou navíc spojité -- pomocí
\href{https://en.wikipedia.org/wiki/Horner%27s_method}{Hornerova schématu} se
 snadno počítá jejich hodnota v daném bodě a stejně snadno se hledají jejich
 kořeny -- body, kde jsou nulové. Není proto překvapivé, že se matematici již
 dlouho snaží aproximovat hodnoty nepolynomiálních funkcí (jako $\exp$, $\log$
 atd.) hodnotami polynomů. V této kapitole si ujasníme, co vlastně myslíme
 \emph{aproximací}, jak jednu konkrétní sestrojit a (aspoň povrchově), k čemu je
 dobrá.

\begin{definition}{Polynomiální funkce}{polynomialni-funkce}
 Řekneme, že funkce $f:\R \to \R$ je \emph{polynomiální}, když existuje $n \in
 \N$ a koeficienty $a_i \in \R, i \leq n$, takové, že
 \[
  f(x) = \sum_{i=0}^{n} a_i x^{i} \quad \forall x \in \R.
 \]
\end{definition}

\begin{remark}{}{polynom-vs-polynomialni-funkce}
 Striktně vzato je rozdíl mezi \emph{polynomem} a \emph{polynomiální funkcí}.
 Polynom je formální výraz tvaru
 \[
  \sum_{i=0}^{n} a_i x^{i},
 \]
 kde $x$ je pouze symbol a nepředstavuje žádnou hodnotu. Polynomiální funkce je
 pak funkce, která vlastně dosazuje do nějakého polynomu za $x$ číslo.

 My však těchto rozdílů dbát nebudeme a slovy \emph{polynom} i
 \emph{polynomiální funkce} budeme mínit objekt z
 \myref{definice}{def:polynomialni-funkce}.
\end{remark}

Co vlastně znamená \emph{aproximovat} funkci? Funkci $\exp$ můžeme například na
intervalu $[0,1]$ aproximovat číslem $-69$, ale intuice čtenářům, doufáme,
napovídá, že toto není \uv{dobrá} aproximace. Jistě nemůžeme obecně doufat v
aproximaci funkce polynomem na celé její doméně; smysluplným však zdá sebe býti
snažit se aproximovat na okolí zvoleného bodu.

Úspěšnost polynomiální aproximace má dobrý smysl měřit rovněž polynomem. Totiž,
z výpočetních důvodů často potřebujeme omezit stupeň (nejvyšší mocninu)
aproximujícího polynomu. Přejeme si, aby chyba aproximace polynomem stupně $n$
na okolí daného bodu klesala (při blížení se k~tomuto bodu) aspoň tak rychle,
jak nejrychleji může polynom stupně $n$ na okolí nějakého bodu k~$0$ klesat. Je
patrné, že nejrychleji ze všech polynomů stupně $n$ klesá na okolí bodu $a$ k
nule polynom $(x-a)^{n}$, neb má v $a$ $n$-násobný kořen. Ukážeme, že ve
skutečnosti můžeme požadovat, aby chyba aproximace na okolí $a$ klesala k $0$
ještě rychleji.

\begin{definition}{Aproximace stupně $n$}{aproximace-stupne-n}
 Ať $f:M \to \R$ je reálná funkce, $a \in M$ a $n \in \N$. Řekneme, že polynom
 $P$ je \emph{aproximací $f$ na okolí $a$ stupně $n$}, když
 \[
  \lim_{x \to a} \frac{f(x) - P(x)}{(x-a)^{n}} = 0.
 \]
 Vyjádřeno slovy: $P$ je aproximací $f$ na okolí $a$ stupně $n$, když chyba
 aproximace na okolí $a$ klesá k $0$ rychleji, než $(x-a)^{n}$.
\end{definition}

Pojďme si nyní rozmyslet, jak aproximace $f$ hledat. Začněme nejjednodušším
případem -- lineární aproximací (tj. aproximací stupně $1$) polynomem rovněž
stupně nejvýše $1$, tedy \uv{přímkou}. Položme tedy $P(x) \coloneqq \psi x +
\omega$ a počítejme
\[
 \lim_{x \to a} \frac{f(x) - P(x)}{x - a} = \lim_{x \to a} \frac{f(x) - \psi x -
 \omega}{x - a} = 0.
\]
Poslední rovnost bystrým čtenářům připomene \hyperref[def:derivace]{definici
derivace}. Vskutku, přepokládáme-li, že existuje konečná $f'(a)$, pak můžeme
poslední limitu upravit do tvaru
\[
 \lim_{x \to a} \frac{f(x) - \psi x - \omega}{x - a} = \lim_{x \to a} \frac{f(x)
 - f(a)}{x-a} - \lim_{x \to a} \frac{\psi x + \omega - f(a)}{x-a} = f'(a) -
 \lim_{x \to a} \frac{\psi x + \omega - f(a)}{x - a}.
\]
Náš úkol je tímto výrazně zjednodušen. Potřebujeme, aby se poslední limita
rovnala konstantě $f'(a)$. Toho lze docílit více způsoby; ten nejvíce přímočarý
je snad zařídit, aby se čitatel zlomku rovnal $f'(a)(x-a)$, neboť zřejmě
\[
 \lim_{x \to a} \frac{f'(a)(x-a)}{x-a} = f'(a).
\]
Odtud plyne rovnost
\[
 \psi x + \omega - f(a) = f'(a)(x-a),
\]
ze které již snadno
\begin{align*}
 \psi &= f'(a),\\
 \omega &= f(a) - a \cdot f'(a),
\end{align*}
čili
\[
 P(x) = \psi x + \omega = f'(a)(x - a) + f(a)
\]
je lineární aproximací funkce $f$ na okolí $a$. Funkci $P(x)$ se obvykle
přezdívá \emph{tečna ke grafu} funkce $f$ v bodě $a$, neboť je to přímka, která
prochází bodem $(a,f(a))$ a na okolí $a$ roste stejně rychle jako $f$.

\begin{definition}{Derivace vyšších řádů}{derivace-vyssich-radu}
 Ať $f:M \to \R$ je reálná funkce. Induktivně definujeme \emph{$n$-tou derivaci
 funkce $f$ v bodě $a$} předpisem
 \[
  f^{(n)}(a) \coloneqq \lim_{h \to 0} \frac{f^{(n-1)}(a + h) - f^{(n -
  1)}(a)}{h} = \lim_{x \to a} \frac{f^{(n-1)}(x) - f^{(n-1)}(a)}{x-a},
 \]
 kde $f^{(0)} \coloneqq f$.
\end{definition}

\begin{remark}{Značení derivací}{znaceni-derivaci}
 V této kapitole budeme vždy $n$-tou derivaci (vizte
 \myref{definici}{def:derivace-vyssich-radu}) funkce $f$ značit symbolem
 $f^{(n)}$, a to i tehdy, když je tato derivace první. Místo $f'$ tedy dočasně 
 píšeme $f^{(1)}$.
\end{remark}

Podobným postupem je možné hledat aproximace vyšších stupňů. Hledáme-li polynom
$Q(x)$ stupně nejvýše $2$ splňující
\[
 \lim_{x \to a} \frac{f(x) - Q(x)}{(x-a)^2},
\]
upravíme nejprve tuto limitu na
\[
 \lim_{x \to a} \frac{\frac{f(x) - Q(x)}{x-a}}{x-a}.
\]
Již totiž víme, že $P(x) = f'(a)(x-a) + f(a)$ je lineární aproximací funkce $f$
na okolí $a$. Budeme tedy směle předpokládat, že $Q(x) = P(x) + R(x)$ a
spočteme, čemu se rovná polynom $R(x)$. Počítáme
\[
 \lim_{x \to a} \frac{\frac{f(x) - Q(x)}{x-a}}{x-a} = \lim_{x \to a} \frac{f(x)
 - P(x)}{(x-a)^2} - \lim_{x \to a} \frac{R(x)}{(x-a)^2}.
\]
Užitím \hyperref[thm:lhospitalovo-pravidlo]{l'Hospitalova pravidla} spočteme
\[
 \lim_{x \to a} \frac{f(x) - P(x)}{(x-a)^2} = \lim_{x \to a} \frac{f(x) - f(a) -
 f^{(1)}(a)(x-a)}{(x-a)^2} = \lim_{x \to a} \frac{f^{(1)}(x) -
f^{(1)}(a)}{2(x-a)} = \frac{f^{(2)}(a)}{2}.
\]
Chceme tudíž, aby platilo
\[
 \lim_{x \to a} \frac{R(x)}{(x-a)^2} = \frac{f^{(2)}(a)}{2},
\]
z čehož plyne přirozená volba
\[
 R(x) \coloneqq \frac{f^{(2)}(a)}{2}(x-a)^2.
\]

Iterováním tohoto postupu se dostaneme k tzv. \emph{Taylorovu polynomu}.

\section{Definice Taylorova polynomu}
\label{sec:definice-taylorova-polynomu}

\begin{definition}{Taylorův polynom}{tayloruv-polynom}
 Ať $f:M \to \R$ je reálná funkce, majíc konečné derivace všech řádů do $n \in
 \N$ včetně, a $a \in M$. Pak \emph{Taylorovým polynomem stupně $n$ funkce $f$ v
 bodě $a$} rozumíme polynom
 \[
  T^{f,a}_n(x) = \sum_{k=0}^{n} \frac{f^{(k)}(a)}{k!}(x-a)^{k}.
 \]
\end{definition}

\begin{lemma}{Derivace Taylorova polynomu}{derivace-taylorova-polynomu}
 Platí
 \[
  (T^{f,a}_n)^{(1)} = T^{f^{(1)},a}_{n-1}.
 \]
\end{lemma}
\begin{lemproof}
 Z \hyperref[def:tayloruv-polynom]{definice Taylorova polynomu} počítáme
 \begin{align*}
  (T^{f,a}_n)^{(1)}(x) &= \left( \sum_{k=0}^n \frac{f^{(k)}(a)}{k!}(x-a)^{k}
  \right)^{(1)} = \sum_{k=1}^n \frac{k \cdot f^{(k)}(a)}{k!}(x-a)^{k-1}\\
                       &= \sum_{k=1}^{n} \frac{f^{(k)}(a)}{(k-1)!}(x-a)^{k-1}
                       = \sum_{k=0}^{n-1} \frac{(f^{(1)})^{(k)}}{k!}(x-a)^{k} =
                       T^{f^{(1)},a}_{n-1}(x).
 \end{align*}
\end{lemproof}

\begin{proposition}{Aproximace Taylorovým polynomem}{aproximace-taylorovym-polynomem}
 Ať $f:M \to \R$ je reálná funkce, majíc konečné derivace do řádu $n \in \N$
 včetně, a $a \in M$. Pak je $T^{f,a}_n$ aproximací $f$ stupně $n$ na okolí $a$.
\end{proposition}
\begin{propproof}
 Budeme postupovat indukcí podle stupně $n \in \N$. Již víme, že pro $n = 1$ je
 $T^{f,a}_1(x) = f(a) + f^{(1)}(a)(x-a)$ lineární aproximací $f$ na okolí $a$.

 Pro $n > 1$ máme z \hyperref[thm:lhospitalovo-pravidlo]{l'Hospitalova pravidla}
 a \hyperref[lem:derivace-taylorova-polynomu]{předchozího lemmatu}
 \[
  \lim_{x \to a} \frac{f(x) - T^{f,a}_n(x)}{(x-a)^{n}} = \lim_{x \to a}
  \frac{f^{(1)}(x) - T^{f^{(1)},a}_{n-1}(x)}{n(x-a)^{n-1}}.
 \]
 Protože $f^{(1)}$ je reálná funkce a má konečné derivace do řádu $n-1$ včetně,
 je z indukčního předpokladu $T^{f^{(1)},a}_{n-1}$ aproximací $f^{(1)}$ stupně
 $n-1$ na okolí $a$. Platí pročež
 \[
  \lim_{x \to a} \frac{f^{(1)}(x) - T^{f^{(1)},a}_{n-1}(x)}{n(x-a)^{n-1}} = 0,
 \]
 a tedy i
 \[
  \lim_{x \to a} \frac{f(x) - T^{f,a}_n(x)}{(x-a)^{n}} = 0,
 \]
 jak jsme chtěli.
\end{propproof}

Překvapivé možná je, že Taylorův polynom je \emph{jedinou} aproximací funkce $f$
stupně $n$ polynomem stupně nejvýše $n$. K důkazu tohoto faktu si pomůžeme
jedním technickým lemmatem.

\begin{lemma}{}{nulovy-polynom}
 Ať $n \in \N$ a $Q$ je polynom stupně nejvýše $n$. Platí-li $\lim_{x \to a}
 Q(x) / (x-a)^{n} = 0$, pak $Q = 0$.
\end{lemma}

\begin{lemproof}
 Budeme pro spor předpokládat, že $Q$ není nulový. Bez důkazu využijeme tvrzení,
 že když $a$ je kořenem $Q$, pak $x - a \mid Q$. Protože $\lim_{x \to a} Q(x) /
 (x-a)^{n}$, jistě platí $Q(a) = 0$, čili existuje $k \in \N$ takové, že $Q(x) =
 (x-a)^{k} \cdot R(x)$, kde $R$ je polynom nemaje kořen $a$. Pak ale
 \[
  \lim_{x \to a} \frac{Q(x)}{(x-a)^{n}} = \lim_{x \to a}
  \frac{R(x)}{(x-a)^{n-k}}.
 \]
 Tato limita buď neexistuje (pokud $k < n$), nebo je rovna $R(a) \neq 0$ (pokud
 $k = n$). V obou případech je nenulová, což je spor.
\end{lemproof}

\begin{theorem}{Jednoznačnost Taylorova polynomu}{jednoznacnost-taylorova-polynomu}
 Ať $f:M \to \R$ je funkce, majíc konečné derivace do řádu $n \in \N$ včetně, a
 $a \in M$. Předpokládejme, že $P$ je polynom stupně nejvýše $n$, jenž je rovněž
 aproximací $f$ na okolí $a$ stupně $n$. Pak $P = T^{f,a}_n$.
\end{theorem}
\begin{thmproof}
 Podle \myref{tvrzení}{prop:aproximace-taylorovym-polynomem} platí
 \[
  \lim_{x \to a} \frac{f(x) - T^{f,a}_n(x)}{(x-a)^{n}} = 0.
 \]
 Z předpokladu a \hyperref[thm:aritmetika-limit]{věty o aritmetice limit}
 \[
  \lim_{x \to a} \frac{T^{f,a}_n(x) - P(x)}{(x-a)^{n}} = \lim_{x \to a}
  \frac{T^{f,a}_n(x) - f(x)}{(x-a)^{n}} + \lim_{x \to a} \frac{f(x) -
  P(x)}{(x-a)^{n}} = 0 + 0 = 0,
 \]
 čili podle \myref{lemmatu}{lem:nulovy-polynom} jest $P - T^{f,a}_n = 0$. 
\end{thmproof}

\begin{problem}{}{pocitani-taylorova-polynomu}
 Spočtěte $T^{\tan,\pi / 4}_3$, tj. Taylorův polynom stupně $3$ v bodě $\pi / 4$
 funkce $\tan$.
\end{problem}
\begin{probsol}
 Platí
 \begin{align*}
  \tan^{(1)}(x) &= \frac{1}{\cos^2 x},\\
  \tan^{(2)}(x) &= \left( \frac{1}{\cos ^2x} \right)^{(1)} =
  \frac{2\sin x}{\cos^3 x},\\
  \tan^{(3)}(x) &= \left( \frac{2 \sin x}{\cos^3 x} \right)^{(1)} = \frac{2
  \cdot \left(2 - \cos{\left(2 x \right)}\right)}{\cos^{4}{\left(x \right)}}. 
 \end{align*}
 A tedy,
 \[
  \tan \left( \frac{\pi}{4} \right) = 1, \quad \tan^{(1)} \left( \frac{\pi}{4}
   \right) = 2, \quad \tan^{(2)} \left( \frac{\pi}{4} \right) = 4, \quad
   \tan^{(3)} \left( \frac{\pi}{4} \right) = 16.
 \]
 Z čehož již snadno dopočteme
 \begin{align*}
  T^{\tan,\pi / 4}_3(x) &= \sum_{k=0}^3 \frac{\tan^{(k)}(\pi / 4)}{k!}\left( x -
  \frac{\pi}{4}\right)^{k}\\
                        &= 1 + 2 \left( x - \frac{\pi}{4} \right) + 2 \left( x - \frac{\pi}{4}
  \right)^2 + \frac{8}{3} \left( x - \frac{\pi}{4} \right)^3.
 \end{align*}
\end{probsol}
\begin{exercise}{}{vypocet-taylora-sin-cos}
 Spočtěte $T^{\sin \cdot \cos,\pi / 2}_5$.
\end{exercise}

\section{Tvary zbytku}
\label{sec:tvary-zbytku}

V této sekci spočteme tzv. \uv{tvary zbytku} Taylorova polynomu. Jsou to výrazy,
které vyjadřují -- aspoň řádově -- velikost chyby při aproximace funkce
Taylorovým polynomem na okolí daného bodu. Budou se hodit primárně při zpytu
poloměru okolí, na němž můžeme stále tvrdit, že Taylorův polynom aproximuje
funkci \uv{dobře}.

\begin{theorem}{Obecný tvar zbytku}{obecny-tvar-zbytku}
 Ať $a, x \in \R$ a $f$ má na $[a,x]$ konečné derivace do řádu $n+1$ včetně. Ať
 je dále $\varphi$ libovolná spojitá funkce na $[a,x]$ s konečnou první derivací
 na $(a,x)$. Pak existuje $\xi \in (a,x)$ takové, že
 \[
  f(x) - T^{f,a}_n(x) = \frac{1}{n!}\frac{\varphi(x) -
  \varphi(a)}{\varphi^{(1)}(\xi)}f^{(n+1)}(\xi)(x-\xi)^{n}.
 \]
\end{theorem}
\begin{thmproof}
 Definujme funkci $F: [a,x] \to \R$ předpisem
 \[
  F(t) \coloneqq f(x) - (f(t) + f^{(1)}(t)(x-t) + \frac{1}{2}f^{(2)}(t)(x-t)^2 +
  \ldots + \frac{1}{n!}f^{(n)}(t)(x-t)^{n}.
 \]
 Pak je $F$ spojitá na $[a,x]$ a $F^{(1)}$ existuje konečná na $(a,x)$. Podle
 \hyperref[thm:cauchyho-o-stredni-hodnote]{Cauchyho věty o střední hodnotě}
 existuje $\xi \in (a,x)$ takové, že
 \begin{equation}
  \label{eq:obecny-zbytek}
  \tag{$\diamondsuit$}
  \frac{F(x) - F(a)}{\varphi(x) - \varphi(a)} =
  \frac{F^{(1)}(\xi)}{\varphi^{(1)}(\xi)}.
 \end{equation}
 
 Snadno spočteme, že
 \begin{align*}
  F^{(1)}(\xi) &= -f^{(1)}(\xi) + f^{(1)}(\xi) - f^{(2)}(\xi) + f^{(2)}(\xi) -
  \ldots - \frac{1}{n!}f^{(n+1)}(\xi)(x-\xi)^{n}\\
               &= -\frac{1}{n!}f^{(n+1)}(\xi)(x - \xi)^{n}.
 \end{align*}
 Zřejmě $F(x) = 0$ a $F(a) = f(x) - T^{f,a}_n(x)$. Čili z
 rovnosti~\eqref{eq:obecny-zbytek} máme
 \[
  \frac{T^{f,a}_n(x) - f(x)}{\varphi(x) - \varphi(a)} =
  -\frac{1}{\varphi^{(1)}(x)} \left( \frac{1}{n!} f^{(n+1)}(\xi)(x-\xi)^{n}
  \right).
 \]
 Odtud přímočarou úpravou plyne tvrzení.
\end{thmproof}

Uvědomme si, že ve \myref{větě}{thm:obecny-tvar-zbytku} je $\varphi$ \emph{zcela
libovolná funkce} s dodatečnými podmínkami spojitosti a diferencovatelnosti. Tím
máme k dispozici celou třídu vyjádření zbytků Taylorova polynomu pouhým
dosazováním za $\varphi$. Dvě konkrétní dosazení (jež sobě dokonce vysloužila
jména) se nám budou v dalším textu hodit více než jiná.

\begin{corollary}{Lagrangeův tvar zbytku}{lagrangeuv-tvar-zbytku}
 Ať $f$ je spojitá na $[a,x]$ a má konečné derivace na $(a,x)$ do řádu $n+1$
 včetně. Pak existuje $\xi \in (a,x)$ takové, že
 \[
  f(x) - T^{f,a}_n(x) = \frac{1}{(n+1)!}f^{(n+1)}(\xi)(x-a)^{n+1}.
 \]
\end{corollary}
\begin{corproof}
 Plyne z dosazení $\varphi(t) = (x-t)^{n+1}$ ve
 \myref{větě}{thm:obecny-tvar-zbytku}.
\end{corproof}

\begin{corollary}{Cauchyho tvar zbytku}{cauchyho-tvar-zbytku}
 Ať $f$ je spojitá na $[a,x]$ a má konečné derivace na $(a,x)$ do řádu $n + 1$
 včetně. Pak existuje $\xi \in (a,x)$ takové, že
 \[
  f(x) - T^{f,a}_n(x) = \frac{1}{n!}f^{(n+1)}(\xi)(x-\xi)^{n}(x-a).
 \]
\end{corollary}
\begin{corproof}
 Plyne z \myref{věty}{thm:obecny-tvar-zbytku} dosazením $\varphi(t) = t$.
\end{corproof}

