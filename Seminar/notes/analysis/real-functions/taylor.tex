\chapter{Taylorův polynom}
\label{chap:tayloruv-polynom}

% TODO
\clr{\textbf{Tato kapitola se nachází v pracovní verzi. Neočekávejte obrázky,
		naopak očekávejte chyby a podivné formulace.}}

Polynomy jsou hezké funkce. Dají se donekonečna derivovat -- všechny tyto
derivace jsou navíc spojité -- pomocí
\href{https://en.wikipedia.org/wiki/Horner%27s_method}{Hornerova schématu} se
 snadno počítá jejich hodnota v daném bodě a stejně snadno se hledají jejich
 kořeny -- body, kde jsou nulové. Není proto překvapivé, že se matematici již
 dlouho snaží aproximovat hodnoty nepolynomiálních funkcí (jako $\exp$, $\log$
 atd.) hodnotami polynomů. V této kapitole si ujasníme, co vlastně myslíme
 \emph{aproximací}, jak jednu konkrétní sestrojit a (aspoň povrchově), k čemu je
 dobrá.

\begin{definition}{Polynomiální funkce}{polynomialni-funkce}
 Řekneme, že funkce $f:\R \to \R$ je \emph{polynomiální}, když existuje $n \in
 \N$ a koeficienty $a_i \in \R, i \leq n$, takové, že
 \[
  f(x) = \sum_{i=0}^{n} a_i x^{i} \quad \forall x \in \R.
 \]
\end{definition}

\begin{remark}{}{polynom-vs-polynomialni-funkce}
 Striktně vzato je rozdíl mezi \emph{polynomem} a \emph{polynomiální funkcí}.
 Polynom je formální výraz tvaru
 \[
  \sum_{i=0}^{n} a_i x^{i},
 \]
 kde $x$ je pouze symbol a nepředstavuje žádnou hodnotu. Polynomiální funkce je
 pak funkce, která vlastně dosazuje do nějakého polynomu za $x$ číslo.

 My však těchto rozdílů dbát nebudeme a slovy \emph{polynom} i
 \emph{polynomiální funkce} budeme mínit objekt z
 \myref{definice}{def:polynomialni-funkce}.
\end{remark}

Co vlastně znamená \emph{aproximovat} funkci? Funkci $\exp$ můžeme například na
intervalu $[0,1]$ aproximovat číslem $-69$, ale intuice čtenářům, doufáme,
napovídá, že toto není \uv{dobrá} aproximace. Jistě nemůžeme obecně doufat v
aproximaci funkce polynomem na celé její doméně; smysluplným však zdá sebe býti
snažit se aproximovat na okolí zvoleného bodu.
