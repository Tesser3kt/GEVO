\subsection{Extrémy funkce}
\label{ssec:extremy-funkce}

V mnoha matematických i externích disciplínách jeden často hledá při studiu
reálných funkcí body, v nichž je hodnota funkce největší či nejmenší. Obecně
jsou maximalizační a minimalizační problémy jedny z nejčastěji řešených. Tyto
problémy vedou přímo na výpočet tzv. \emph{derivací} reálných funkcí, jsoucích
dychtivým čtenářům představeny v \hyperref[chap:derivace]{následující kapitole}.
Zde pouze definujeme lokální a globální extrémy funkcí a ukážeme, že spojité
funkce na uzavřených intervalech nutně na týchž nabývají svých nejmenších i
největších hodnot.

\begin{definition}{Lokální a globální extrém}{lokalni-a-globalni-extrem}
 Ať $f:M \to \R$ je reálná funkce a $X \subseteq M$. Řekneme, že funkce $f$ v
 bodě $x \in X$ nabývá
 \begin{itemize}
  \item \emph{globálního maxima} na $X$, když pro každé $y \in X$ platí
   $f(y) \leq f(x)$;
  \item \emph{globálního minima} na $X$, když pro každé $y \in X$ platí $f(y)
   \geq f(x)$;
  \item \emph{lokálního maxima}, když existuje okolí bodu $x$, na němž platí $f
   \leq f(x)$;
  \item \emph{lokálního minima}, když existuje okolí bodu $x$, na němž platí $f
   \geq f(x)$.
 \end{itemize}
 Souhrnně přezdíváme globálnímu minimu a maximu \emph{globální extrém} a
 lokálnímu maximu a minimu \emph{lokální extrém}.
\end{definition}

\begin{figure}[ht]
 \centering
 \begin{tikzpicture}
  \tkzInit[xmin=-3,xmax=3,ymin=-3,ymax=3]
  \tkzDrawX[-latex,label=]
  \tkzDrawY[-latex,label=]

  \draw[thick,domain=-2.3:3,color=Fuchsia] plot [smooth] (\x, {0.5 * (\x*\x*\x
   - \x*\x - 4*\x + 2)}) node[right] {$\clm{f}$};
  \tkzDefPoints{-2.2/0/a,2.8/0/b}
  \tkzDrawPoints[size=4,color=black](a,b)
  \tkzLabelPoint[above](a){$a$}
  \tkzLabelPoint[below](b){$b$}

  \tkzDefPoints{-2.2/-2.344/fa,2.8/2.456/fb}
  \tkzDrawSegments[dashed,thick](a,fa b,fb)
  \tkzDrawPoints[size=6,color=BrickRed](fa,fb)

  \tkzDefPoints{-0.867/2.032/fM,1.535/-1.44/fm}
  \tkzDefPoints{-0.867/0/M,1.535/0/m}
  \tkzDrawPoints[size=4,color=black](M,m)
  \tkzLabelPoint[below](M){$M$}
  \tkzLabelPoint[above](m){$m$}
  \tkzDrawSegments[dashed,thick](M,fM m,fm)
  \tkzDrawPoints[size=6,color=RoyalBlue](fM,fm)
 \end{tikzpicture}
 \caption{Lokální a globální extrémy funkce $\clm{f}$ na $[a,b]$. \clb{Lokálních
  extrémů} nabývá $\clm{f}$ v bodech $m$ a $M$ a \clr{globálních extrémů} v
  bodech $a$ a $b$.}
 \label{fig:extremy}
\end{figure}

Jak jsme již zmínili v textu před
\hyperref[def:lokalni-a-globalni-extrem]{definicí}, spojité funkce nabývají na
uzavřených intervalech globálních extrémů vždy.

\begin{theorem}{Extrémy spojité funkce}{extremy-spojite-funkce}
 Ať $f:[a,b] \to \R$ je spojitá funkce. Pak $f$ nabývá globálního minima a
 maxima na $[a,b]$.
\end{theorem}
\begin{thmproof}
 Dokážeme, že $f$ nabývá na $[a,b]$ globálního maxima. Pro globální minimum lze
 důkaz vést obdobně.

 Využijeme \myref{důsledku}{cor:heineho-veta-pro-spojitost}. Nalezneme
 posloupnost, která se uvnitř intervalu $f([a,b])$ blíží k~supremu funkce $f$ na
 $[a,b]$ a ukážeme, že vzory členů této posloupnosti z~intervalu $[a,b]$ se
 blíží k bodu, kde $f$ nabývá maxima.

 Položme tedy $S \coloneqq \sup f([a,b])$. Sestrojíme posloupnost $y: \N \to
 f([a,b])$, která konverguje k $S$. Je-li $S=\infty$, stačí položit třeba $y_n =
 n$ pro všechna $n \in \N$. Předpokládejme, že $S \in \R$. Z~definice suprema
 existuje pro každé $n \in \N$ prvek $z \in f([a,b])$ takový, že $S-1 / n < z
 \leq S$. Položíme $y_n \coloneqq z$. Tím jsme dali vzrůst posloupnosti $y_n$ s
 $\lim_{n \to \infty} y_n = S$.

 Z definice $f([a,b])$ nalezneme pro každé $y_n$ číslo $x_n \in [a,b]$, pro něž
 $f(x_n) = y_n$. Posloupnost $x_n$ je omezená (leží uvnitř $[a,b]$), a tedy z
 \hyperref[thm:bolzanova-weierstrassova]{Bolzanovy-Weierstraßovy} věty existuje
 její konvergentní podposloupnost. Můžeme pročež bez újmy na obecnosti
 předpokládat, že sama $x_n$ konverguje. Položme $M \coloneqq \lim_{n \to
 \infty} x_n$. Ježto $a \leq x_n \leq b$ pro všechna $n \in \N$, z
 \myref{lemmatu}{lem:o-dvou-straznicich} plyne, že $M \in [a,b]$. Konečně, $f$
 je z předpokladu spojitá, a tedy v závěsu
 \myref{důsledku}{cor:heineho-veta-pro-spojitost} 
 \[
  f(M) = \lim_{n \to \infty} f(x_n) = \lim_{n \to \infty} y_n = S,
 \]
 čili $f$ nabývá v bodě $M$ maxima na $[a,b]$.
\end{thmproof}

\begin{corollary}{}{}
 Je-li funkce $f$ spojitá na $[a,b]$, pak je tamže omezená.
\end{corollary}
\begin{corproof}
 Z \myref{věty}{thm:extremy-spojite-funkce} plyne, že $f$ nabývá na $[a,b]$
 minima $s$ a maxima $S$. Potom ale pro každé $x \in [a,b]$ platí
 \[
  s \leq f(x) \leq S,
 \]
 čili $f$ je na $[a,b]$ omezená.
\end{corproof}
