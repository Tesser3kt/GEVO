\chapter{Elementární funkce}
\label{chap:elementarni-funkce}

% TODO
\clr{\textbf{Tato kapitola se nachází v pracovní verzi. Neočekávejte obrázky,
		naopak očekávejte chyby a podivné formulace.}}

Jisté speciální funkce v matematické analýze si vysloužily přízvisko
\emph{elementární}. Původ jejich speciality je ryze fyzikální. Jsou to funkce,
jejich prostřednictvím fyzikové modelují mnoho přírodních jevů a pojmů -- růst,
vlnění, proud, gravitaci, úhel \dots

Ježto fyzikální model světa radno ponechati do textů menší náročnosti,
soustředit se budeme pouze na prezentaci těchto funkcí a důkazy jejich
základních vlastností.

Všechny elementární funkce definujeme jako součty nekonečných řad. V tomto textu
jsme se nezabývali pramnoho konvergencí řad s libovolnými členy. Všechna
tvrzení, která tímto směrem budeme vyžadovat, zformulujeme, ač nedokážeme.

\section{Exponenciála a logaritmus}
\label{sec:exponenciala-a-logaritmus}

První na seznamu je \emph{exponenciála} -- funkce spojitého růstu. Toto
pojmenování ještě níže odůvodníme. Nyní přikročíme k definici. Pro stručnost
zápisu, budeme v následujícím textu používat konvenci, že $0^{0} = 1$.

\begin{definition}{Exponenciála}{exponenciala}
 Pro $x \in \R$ definujeme
 \[
  \exp x \coloneqq \sum_{n=0}^{\infty} \frac{x^{n}}{n!}.
 \]
\end{definition}

Jak jsme čtenáře vystříhali, musíme nyní na krátkou chvíli odbočit k číselným
řadám, abychom uměli v obec dokázat, že právě definovaná
\hyperref[def:exponenciala]{exponenciála} je skutečně reálnou funkcí.

\begin{definition}{Absolutní konvergence řady}{absolutni-konvergence-rady}
 Ať $\sum_{n=0}^{\infty} a_n$ je číselná řada, kde $a_n \in \R$. Řekneme, že
 $\sum_{n=0}^{\infty} a_n$ \emph{absolutně konverguje}, když konverguje řada
 $\sum_{n=0}^{\infty} |a_n|$.
\end{definition}

\begin{lemma}{}{absolutni-konvergence-a-konvergence}
 Každá absolutně konvergentní řada je konvergentní.
\end{lemma}
\begin{lemproof}
 Ať je $\varepsilon>0$ dáno. Předpokládejme, že $\sum_{n=0}^{\infty} |a_n|$
 konverguje. Nalezneme $n_0 \in \N$ takové, že pro $m \geq n \geq n_0$ platí
 \[
  \left|\sum_{k=n}^m |a_k| \right| = \sum_{k=n}^m |a_k|<\varepsilon.
 \]
 Potom ale z \hyperref[lem:trojuhelnikova-nerovnost]{trojúhelníkové nerovnosti}
 platí
 \[
  \left| \sum_{k=n}^m a_k \right| \leq \sum_{k=n}^m |a_k|<\varepsilon,
 \]
 čili $\sum_{n=0}^{\infty} a_n$ konverguje.
\end{lemproof}

\begin{definition}{Cauchyho součin řad}{cauchyho-soucin-rad}
 Ať $\sum_{n=0}^{\infty} a_n$ a $\sum_{n=0}^{\infty} b_n$ jsou číselné řady.
 Jejich \emph{Cauchyho součinem} myslíme číselnou řadu
 \[
  \sum_{n=0}^{\infty} \sum_{k=0}^n a_{n-k}b_k.
 \]
\end{definition}
\begin{theorem}{Mertensova}{mertensova}
 Ať $\sum_{n=0}^{\infty} a_n, \sum_{n=0}^{\infty} b_n$ jsou \emph{konvergentní}
 číselné řady, přičemž $\sum_{n=0}^{\infty} a_n$ je navíc absolutně
 konvergentní. Potom $\sum_{n=0}^{\infty} \sum_{k=0}^n a_{n-k}b_k$ konverguje a
 platí
 \[
  \left( \sum_{n=0}^{\infty} a_n \right) \left( \sum_{n=0}^{\infty} b_n \right)
  = \sum_{n=0}^{\infty} \sum_{k=0}^n a_{n-k}b_k.
 \]
\end{theorem}

\begin{theorem}{Vlastnosti exponenciály}{vlastnosti-exponencialy}
 Funkce $\exp$ je dobře definována a platí
  \begin{enumerate}[label=(E\arabic*)]
   \item $\exp(x + y) = \exp x \cdot \exp y$;
   \item $\lim_{x \to 0} \frac{\exp x - 1}{x} = 0$.
  \end{enumerate}
\end{theorem}
\begin{thmproof}
 \emph{Dobrá definovanost} zde znamená, že řada $\sum_{n=0}^{\infty} x_n / n!$
 konverguje pro každé $x \in \R$. Ukážeme, že konverguje absolutně. Je-li $x =
 0$, pak řada konverguje zřejmě. Volme tedy $x \in \R \setminus \{0\}$. Potom
 \[
  \lim_{n \to \infty} \frac{\left| \frac{x^{n+1}}{(n+1)!} \right|}{\left|
  \frac{x^{n}}{n!} \right|} = \lim_{n \to \infty} \frac{|x|}{n+1} = 0,
 \]
 čili podle \myref{věty}{thm:dalembertovo-podilove-kriterium} řada
 $\sum_{n=0}^{\infty} |x^{n}|/n!$ konverguje, což znamená, že konverguje i
 $\sum_{n=0}^{\infty} x^{n}/n!$.

 Dokážeme vlastnost (E1). Počítáme
 \begin{align*}
  \exp(x+y) &= \sum_{n=0}^{\infty} \frac{(x+y)^{n}}{n!} = \sum_{n=0}^{\infty}
  \sum_{k=0}^n \binom{n}{k} \frac{x^{n-k}y^{k}}{n!} = \sum_{n=0}^{\infty}
  \sum_{k=0}^n \frac{n!}{(n-k)!k!} \frac{x^{n-k}y^{k}}{n!}\\
            &= \sum_{n=0}^{\infty} \sum_{k=0}^{n} \frac{x^{n-k}}{(n-k)!}
            \frac{y^{k}}{k!}.
 \end{align*}
 Všimněme si, že poslední řada je \hyperref[def:cauchyho-soucin-rad]{Cauchyho
 součinem} řad $\sum_{n=0}^{\infty} x^{n} / n!$ a $\sum_{n=0}^{\infty} y^{n} /
 n!$. Protože jsou obě tyto řady (podle výše dokázaného) absolutně konvergentní,
 platí z \hyperref[thm:mertensova]{Mertensovy věty}
 \[
  \exp(x+y) = \sum_{n=0}^{\infty} \sum_{k=0}^n
  \frac{x^{n-k}}{(n-k)!}\frac{y^{k}}{k!} = \left( \sum_{n=0}^{\infty}
  \frac{x^{n}}{n!} \right) \left( \sum_{n=0}^{\infty} \frac{y^{n}}{n!} \right) =
  \exp x \cdot \exp y.
 \]

 Nyní vlastnost (E2). Pro $x \in (-1,1)$ odhadujme
 \begin{align*}
  \left| \frac{\exp x - 1}{x} - 1 \right| &= \left| \frac{\exp x - 1 - x}{x}
  \right| = \frac{1}{|x|} \left| \sum_{n=0}^{\infty}\frac{x^{n}}{n!} \clr{- x -
  1} \right| = \frac{1}{|x|} \left| \sum_{\clr{n=2}}^{\infty} \frac{x^{n}}{n!}
  \right| \\
                                          &= |x| \left| \sum_{n=2}^{\infty}
                                          \frac{x^{n-2}}{n!} \right| \leq |x|
                                          \left| \sum_{n=2}^{\infty}
                                          \frac{1}{n!} \right| = c \cdot |x|,
 \end{align*}
 kde $c > 0$ je hodnota součtu řady $\sum_{n=0}^{\infty} 1 / n!$, která zjevně
 konverguje (například díky nerovnosti $1 / n! \leq 1 / n^2$). Jelikož $\lim_{x
 \to 0} c \cdot |x| = 0$, plyne odtud ihned, že
 \[
  \lim_{x \to 0} \left| \frac{\exp x - 1}{x} -1\right| = 0,
 \]
 z čehož zase
 \[
  \lim_{x \to 0} \frac{\exp x - 1}{x} = 1.
 \]
 Tím je důkaz završen.
\end{thmproof}

Ihned si odvodíme další vlastnosti exponenciály plynoucí z (E1) a (E2). Postupně
dokážeme, že pro každé $x \in \R$ platí následující.
\begin{enumerate}[label=(E\arabic*)]
 \setcounter{enumi}{2}
 \item $\exp 0 = 1$;
 \item $\exp' x = \exp x$;
 \item $\exp(-x) = 1 / \exp(x)$;
 \item $\exp x > 0$;
 \item $\exp$ je spojitá na $\R$;
 \item $\exp$ je rostoucí na $\R$;
 \item $\lim_{x \to \infty} \exp x = \infty$ a $\lim_{x \to -\infty} \exp x =
  0$;
 \item $\img \exp = (0,\infty)$.
\end{enumerate}

Z (E1) platí $\exp(0 + x) = \exp 0 \cdot \exp x$. Protože zřejmě existuje $x \in
\R$, pro něž $\exp x \neq 0$, plyne odtud $\exp 0 = 1$, tj. vlastnost (E3).

Pro důkaz (E4) počítáme
\begin{align*}
 \lim_{h \to 0} \frac{\exp (x + h) - \exp h}{h} &\clr{~=~} \lim_{h \to 0}
 \frac{\exp h \cdot \exp x - \exp h}{h} = \lim_{h \to 0} \frac{(\exp h - 1)\exp
 x}{h}\\
                                                &= \exp x \cdot \lim_{h \to 0}
                                                \frac{\exp h - 1}{h} \clb{~=~}
                                                \exp x \cdot 1 = \exp x,
\end{align*}
kde jsme v \clr{červené} rovnosti použili vlastnost (E1) a v \clb{modré} zas
vlastnost (E2).

Pokračujeme vlastností (E5). Z (E1) máme
\[
 \exp(x + (-x)) = \exp x \cdot \exp(-x).
\]
Protože z (E3) je $\exp(x + (-x)) = \exp 0 = 1$, dostáváme
\[
 1 = \exp x \cdot \exp(-x),
\]
čili
\[
 \exp(-x) = \frac{1}{\exp x}.
\]

Ježto má řada $\sum_{n=0}^{\infty} x^{n} / n!$ zjevně kladný součet pro $x > 0$,
plyne (E6) přímo z právě dokázané (E5).

Vlastnost (E7) je okamžitým důsledkem vlastnosti (E4), díky níž má $\exp$
konečnou derivaci na $\R$, a tudíž je podle
\myref{lemmatu}{lem:vztah-derivace-a-spojitosti} tamže spojitá.

Vlastnost (E8) je důsledkem vlastností (E4) a (E6), neboť funkce majíc na
intervalu (v tomto případě celém $\R$) kladnou derivaci, je na tomto intervalu
-- podle \myref{důsledku}{cor:vztah-derivace-a-monotonie} -- rostoucí.

Platí $\exp 1 = \sum_{n=0}^{\infty} \frac{1}{n!} = 1 + \sum_{n=1}^{\infty}
\frac{1}{n!} > 1$, čili z vlastnosti (E1) plyne, že $\exp$ není shora omezená,
neboť $\exp(x+1) = \exp x \cdot \exp 1 > \exp x$ pro každé $x \in \R$. To spolu
s vlastnostmi (E7) a (E8) dává $\lim_{x \to \infty} \exp x = \infty$. Dále,
použitím (E5),
\[
 \lim_{x \to -\infty} \exp x = \lim_{x \to \infty} \exp(-x) = \lim_{x \to
 \infty} \frac{1}{\exp x} = 0,
\]
což dokazuje (E9).

Konečně, vlastnost (E10) plyne z (E9) a \hyperref[thm:bolzanova]{Bolzanovy
věty}.

\begin{example}{}{exponenciala-populace}
 V úvodu do této sekce jsme nazvali exponenciálu \uv{funkcí spojitého růstu}.
 Tuto intuici nyní částečně formalizujeme. Dalšího pohledu nabudeme po definici
 obecné mocniny.
 % TODO odkaz

 Uvažme následující přímočarý populační model. V čase $t \in (0,\infty)$ je
 počet jedinců dán funkcí $P(t)$. Množství nově narozených jedinců závisí pouze
 na počtu právě žijících a na konstantě $r \in [0,\infty)$ -- zvané
 \emph{reproduction rate} -- která značí, kolik nových jedinců se narodí za
 jednoho právě živého. Vnímáme-li derivaci $P'(t)$ jako \emph{rychlost růstu}
 populace v čase $t$, pak dostáváme diferenciální rovnici
 \[
  P'(t) = r \cdot P(t),
 \]
 jelikož v čase $t$ se podle našeho modelu narodí $r$ jedinců za každého živého.
 Díky vlastnosti (E4) vidíme, že například funkce $P(t) = \exp(rt)$ řeší rovnici
 výše. Teorii diferenciálních rovnic v tomto textu probírat nebudeme, bez důkazu
 však zmíníme, že řešení takto triviálních rovnic až na konstantu určena
 jednoznačně. V tomto jednoduchém populačním modelu je tudíž počet živých
 jedinců v čase $t$ dán funkcí $t \mapsto \exp(rt).$
\end{example}

Na závěr si dokážeme jeden možná překvapivý fakt, že vlastnosti (E1) a (E2) již
určují funkci $\exp$ jednoznačně.

\begin{theorem}{Jednoznačnost exponenciály}{jednoznacnost-exponencialy}
 Existuje právě jedna funkce definovaná na celém $\R$ splňující (E1) a (E2).
\end{theorem}
\begin{thmproof}
 Existenci jsme dokázali konstruktivně. Dokážeme jednoznačnost. Ať $f:\R \to \R$
 splňuje (E1) a (E2). Ukážeme, že $f = \exp$.

 Z úvah výše plyne, že $f$ splňuje rovněž vlastnosti (E3) - (E10), protože k
 jejich důkazu byly použity pouze (E1) a (E2). Platí tedy $f(0) = 1$ a $f'(x) =
 f(x)$. Jelikož $\exp x \neq 0$ pro všechna $x \in \R$, máme z
 \hyperref[thm:aritmetika-derivaci]{věty o aritmetice derivací}
 \[
  \left( \frac{f}{\exp} \right)'(x) = \frac{f'(x) \exp x - f(x) \exp'x}{\exp^2
  x} = \frac{f(x) \exp x - f(x) \exp x}{\exp^2 x} = 0.
 \]
 Podle \myref{cvičení}{exer:derivace-nula-konstantni} je $f / \exp$ konstatní
 funkce, čili existuje $c \in \R$ takové, že $f(x) / \exp x = c$ pro každé
 $c \in \R$. Dosazením $x = 0$ zjistíme, že $c = f(0) / \exp 0 = 1$, čili $c =
 1$ a $f = \exp$.
\end{thmproof}

\subsection{Logaritmus}
\label{ssec:logaritmus}

Jelikož je funkce $\exp$ spojitá a rostoucí na $\R$, má na celém $\R$ inverzní
funkci, které přezdíváme \emph{logaritmus} a značíme ji $\log$. Z vlastností
$\exp$ ihned plyne, že $\log$ je reálná funkce $(0,\infty) \to \R$. Na rozdíl od
$\exp$ však $\log$ není dána číselnou řadou -- aspoň ne pro všechna $x \in
(0,\infty)$, více v kapitole o Taylorově polynomu.
%TODO odkaz

Vlastnosti exponenciály nám rovnou umožňují do značné míry prozkoumat k ní
inverzní funkci.

\begin{proposition}{Vlastnosti logaritmu}{vlastnosti-logaritmu}
 Pro každá $x,y \in (0,\infty)$ platí
 \begin{enumerate}[label=(L\arabic*)]
  \item $\log$ je spojitá a rostoucí na $(0,\infty)$;
  \item $\log(xy) = \log x + \log y$;
  \item $\log'x = 1 / x$;
  \item $\lim_{x \to 0^{+}} \log x = -\infty$ a $\lim_{x \to \infty} \log x =
   \infty$.
 \end{enumerate}
\end{proposition}
\begin{propproof}
 \hspace*{\fill}
 \begin{enumerate}[label=(L\arabic*)]
  \item Plyne ihned z faktu, že $\exp$ je spojitá a rostoucí.
  \item Užitím vlastností exponenciály spočteme
  \[
   xy = \exp(\log x) \cdot \exp(\log y) = \exp(\log x + \log y),
  \]
  z čehož po aplikace $\log$ na obě strany rovnosti plyne
  \[
   \log(xy) = \log x + \log y.
  \]
 \item Podle \hyperref[thm:derivace-inverzni-funkce]{věty o derivaci inverzní
  funkce} platí
  \[
   \log'x = \frac{1}{\exp'(\log x)} = \frac{1}{\exp(\log x)} = \frac{1}{x}.
  \]
 \item Protože $\img \log = \R$, není $\log$ zdola ani shora omezená. Z (L1)
  plyne kýžený závěr.
 \end{enumerate}
\end{propproof}

\subsection{Obecná mocnina}
\label{ssec:obecna-mocnina}

Užitím funkcí $\log$ a $\exp$ definujeme pro $a \in (0,\infty)$ a $b \in \R$
výraz $a^{b}$ předpisem
\[
 a^{b} \coloneqq \exp(b \cdot \log a).
\]
Stojí za to věnovat krátkou chvíli ověření, že tato funkce odpovídá naší
představě \emph{mocniny} v případě, kdy $b = n \in \N$. Máme
\[
 a^{n} = \exp(n \cdot \log a) = \exp \left( \sum_{k=1}^n \log a \right) =
 \prod_{k=1}^n \exp(\log(a)) = \prod_{k=1}^n a,
\]
tedy $a^{n}$ je vskutku $a$ $n$-krát vynásobené samo sebou.

\begin{warning}{}{obecna-mocnina}
 Uvědomme si, že $a^{b}$ je definováno pouze pro $a \in (0,\infty)$. Pro $a \leq
 0$ není tato funkce nad reálnými čísly rozumně definovatelná. Důvod je mimo
 jiné následující: pro $n$ sudé a $a < 0$ je $a^{n} > 0$, ale $a^{n+1} < 0$.
 Tedy, měla-li by $a^{b}$ být spojitá funkce, pak by pro každé $n \in \N$ a $a <
 0$ existovalo $\xi \in (n,n+1)$ takové, že $a^{\xi} = 0$. Mocninná funkce, jež
 je nulová pro nekonečně mnoho čísel je i pro matematiky zřejmě příliš divoká
 představa.

 Poznamenejme však, že nad komplexními čísly je funkce $\log$ definována i pro
 záporná reálná čísla, tedy $a^{b}$ dává -- se stejnou definicí -- smysl pro
 všechna $a,b \in \C$.
\end{warning}

Z vlastností $\log$ a $\exp$ plynou ihned vlastnosti obecné mocniny. Jelikož
její definice dává vzniknout \textbf{dvěma} reálným funkcím, konkrétně
\[
 f(x) = a^{x} \text{ pro } x \in \R \quad \text{a} \quad g(x) = x^{b} \text{
 pro } x \in (0,\infty),
\]
musíme tyto při zkoumání vlastností obecné mocniny pochopitelně rozlišovat. Aby
nedošlo ke zmatení, budeme tyto funkce značit zkrátka jako $x \mapsto a^{x}$ a
$x \mapsto x^{b}$, kde $a \in (0,\infty)$ a $b \in \R$ jsou fixní.

\begin{proposition}{Vlastnosti obecné mocniny}{vlastnosti-obecne-mocniny}
 Pro všechna $a \in (0,\infty), b \in \R$ platí
 \begin{enumerate}[label=(O\arabic*)]
  \item Funkce $x \mapsto a^{x}$ i $x \mapsto x^{b}$ jsou spojité na svých
   doménách.
  \item Funkce $x \mapsto a^{x}$ je na celém $\R$ 
   \begin{itemize}
    \item rostoucí pro $a > 1$,
    \item konstantní pro $a = 1$,
    \item klesající pro $a < 1$.
   \end{itemize}
  \item Funkce $x \mapsto x^{b}$ je na $(0,\infty)$
   \begin{itemize}
    \item rostoucí pro $b > 0$,
    \item konstantní pro $b = 0$,
    \item klesající pro $b < 0$.
   \end{itemize}
  \item $(x \mapsto a^{x})' = (x \mapsto a^{x} \log a)$.
  \item $(x \mapsto x^{b})' = (x \mapsto bx^{b-1})$.
  \item Je-li
   \begin{itemize}
    \item $a > 1$, pak $\lim_{x \to \infty} a^{x} = \infty$ a $\lim_{x \to
     -\infty} a^{x} = 0$;
    \item $a < 1$, pak $\lim_{x \to \infty} a^{x} = 0$ a $\lim_{x \to -\infty}
     a^{x} = \infty$;
   \end{itemize}
  \item Je-li 
   \begin{itemize}
    \item $b > 0$, pak $\lim_{x \to 0^{+}} x^{b} = 0$ a $\lim_{x \to \infty}
     x^{b} = \infty$.
    \item $b < 0$, pak $\lim_{x \to 0^{+}} x^{b} = \infty$ a $\lim_{x \to
     \infty} x^{b} = 0$.
   \end{itemize}
  \item $\log(a^{b}) = b \cdot \log a$.
 \end{enumerate}
\end{proposition}
\begin{propproof}
 Vlastnosti (O2), (O3), (O6) a (O7) dokážeme pouze v případě, že $a > 1$ a $b >
 0$. Důkaz tvrzení v případech $a < 1$ a $b < 0$ plyne ihned z rovnosti
 $\exp(-x) = 1 / \exp x$ pro $x \in \R$.
  
 \begin{enumerate}[label=(O\arabic*)]
  \item Jelikož $a^{b} = \exp(b \log a)$ plyne spojitost obou funkcí ze
   spojitosti $\exp$ a $\log$.
  \item Platí $a^{x} = \exp(x \log a)$. Protože $a > 1$, je $\log a > 0$. Funkce
   $\exp$ je rostoucí, a tedy je rostoucí rovněž $a \mapsto a^{x}$.
  \item Máme $x^{b} = \exp(b \log x)$. Jelikož je $b$ z předpokladu kladné a
   $\log$ je rostoucí, je $x \mapsto x^{b}$ též rostoucí.
  \item Počítáme
   \[
    (a^{x})' = (\exp(x \log a))' = \exp'(x \log a) \cdot (x \log a)' = \exp(x
    \log a) \cdot \log a = a^{x} \log a.
   \]
  \item Opět počítáme
   \[
    (x^{b})' = (\exp(b\log x))' = \exp'(b \log x) \cdot (b\log x)' = \exp(b \log
    x) \cdot \frac{b}{x} = \frac{bx^{b}}{x} = bx^{b-1}.
   \]
  \item Podobně jako v důkazu (O2) plyne z $a > 1$, že $\log a > 0$. Potom jsou
   ale limity v $ \pm \infty$ funkce $a^{x} = \exp(x \log a)$ stejné jako tytéž
   limity funkce $\exp x$. Odtud tvrzení.
  \item Jelikož je $b > 0$, máme z \hyperref[thm:limita-slozene-funkce]{věty o
   limitě složené funkce} 
   \begin{align*}
    \lim_{x \to 0^{+}} x^{b} &= \lim_{x \to 0^{+}} \exp(b \cdot \log x) = \lim_{y
    \to -\infty} \exp(b \cdot y) = 0,\\
     \lim_{x \to \infty} x^{b} &= \lim_{x \to \infty} \exp(b \cdot \log x) =
     \lim_{y \to \infty} \exp(b \cdot y) = \infty.
   \end{align*}
  \item Jest
   \[
    \log(a^{b}) = \log(\exp(b \cdot \log a)) = b \cdot \log a. \qedhere
   \]
 \end{enumerate}
\end{propproof}

\begin{example}{Exponenciála jako funkce růstu podruhé}{exp-rust-podruhe}
 V \myref{příkladě}{exam:exponenciala-populace} jsme slíbili čtenářům ještě
 jeden pohled na exponenciálu jako na funkci \uv{spojitého} růstu. Uvažme
 následující obvyklý finanční model.

 Naším prvotním vkladem na spořící účet je částka $P > 0$. Ve smlouvě k účtu je
 uvedeno, že z něj nesmíme vybírat po dobu pěti let za roční úrokové sazby 5 \%.
 Tedy, každý rok se právě uložená částka na účtu zvýší o přesně 5 \%. Když
 chceme vypočítat, kolik budeme mít na účtu za oněch pět let, stačí provést
 snadný výpočet
 \[
  P \cdot 1.05 \cdot 1.05 \cdot 1.05 \cdot 1.05 \cdot 1.05 = P \cdot 1.05^{5}.
 \]
 V zájmu rozšíření tohoto příkladu si zapíšeme tentýž výsledek jako
 \[
  P \cdot \left( 1 + 0.05 \right)^{5}.
 \]
 
 Tento model však předpokládá úročení uložené částky jednou ročně. Když bude
 však úročení probíhat například měsíčně, pak se roční úroková sazba samozřejmě
 rozdělí dvanácti, avšak výpočet úročení je rovněž třeba provést dvanáctkrát do
 roka. Finální částkou po pěti letech bude tudíž
 \[
  P \cdot \left(\left( 1 + \frac{0.05}{12} \right)^{12}\right)^{5} = P \cdot
  \left( 1 + \frac{0.05}{12} \right)^{60}.
 \]
 
 S postupným zkracováním úrokového období se nabízí otázka, jaká by byla finální
 částka, kdyby se původní vklad úročil \uv{nekonečněkrát} do roka, tj. uložená
 částka by se zvyšovala doslova v každém okamžiku (tedy \uv{spojitě}). Odpovědí
 je výraz
 \[
  \lim_{n \to \infty} P \cdot \left( 1 + \frac{0.05}{n} \right)^{n}.
 \]
 
 Ukážeme, že touto limitou je hodnota funkce $\exp$ v bodě $0.05$. Obecněji,
 spočteme, že
 \begin{equation*}
  \label{eq:exp-as-lim}
  \tag{$\spadesuit$}
  \lim_{n \to \infty} \left( 1 + \frac{x}{n} \right)^{n} = \exp x.
 \end{equation*}

 Výpočet je to v celku triviální. Z definice obecné mocniny a
 \hyperref[cor:heineho-veta-pro-spojitost]{Heineho věty} platí
 \[
  \lim_{n \to \infty} \left( 1 + \frac{x}{n} \right)^{n} = \lim_{n \to \infty}
  \exp \left( n \log \left( 1 + \frac{x}{n} \right) \right) = \exp \left(
  \lim_{n \to \infty} n \log \left(1 + \frac{x}{n}\right) \right).
 \]
 Stačí tedy ukázat, že
 \[
  \lim_{n \to \infty} n \log \left( 1+\frac{x}{n} \right) = x.
 \]
 Užitím \hyperref[thm:lhospital]{l'Hospitalova pravidla} dostaneme (derivujeme
 podle $n$ -- to smíme opět díky \hyperref[thm:heineho]{Heineho větě})
 \[
  \lim_{n \to \infty} n \log \left( 1+\frac{x}{n} \right) = \lim_{n \to \infty}
  \frac{\log \left( 1 + \frac{x}{n} \right)}{\frac{1}{n}} = \lim_{n \to \infty}
  \frac{1}{1+\frac{x}{n}} \cdot \frac{-x}{n^2} \cdot (-n^2) = \lim_{n \to
  \infty} \frac{nx}{n + x} = x.
 \]
 Tím je rovnost~\eqref{eq:exp-as-lim} dokázána.
\end{example}

\section{Goniometrické funkce}
\label{sec:goniometricke-funkce}

Název \uv{úhloměrné} funkce je zastaralý a nepřesný. Funkce $\sin$ a $\cos$,
které se jmeme definovati, úspěšně modelují fyzikální jevy jakkoli související s
vibrací či vlněním. Jak si brzy rozmyslíme, jsou to ve skutečnosti funkce v
zásadě exponenciální. To by nemělo být na druhý pohled až tak překvapivé --
vibrace jsou v zásadě jen periodicky se střídající růst a pokles.

\begin{definition}{Goniometrické funkce}{goniometricke-funkce}
 Pro $x \in \R$ definujeme funkce
 \begin{align*}
  \sin x & \coloneqq \sum_{n=0}^{\infty} (-1)^{n} \frac{x^{2n+1}}{(2n+1)!},\\
  \cos x & \coloneqq \sum_{n=0}^{\infty} (-1)^{n} \frac{x^{2n}}{(2n)!}.
 \end{align*}
\end{definition}

\begin{theorem}{Vlastnosti goniometrických
funkcí}{vlastnosti-goniometrickych-funkci}
 Funkce $\sin$ a $\cos$ jsou dobře definované a splňují:
 \begin{enumerate}[label=(G\arabic*)]
  \item $ \forall x,y \in \R:$
  \begin{align*}
   \sin(x+y) &= \sin x \cos y + \sin y \cos x,\\
   \cos(x+y) &= \cos x \cos y - \sin x \sin y;
  \end{align*}
 \item $\sin$ je lichá a $\cos$ je sudá funkce;
 \item $ \exists \pi \in \R$ takové, že $\sin$ je rostoucí na $[0,\pi / 2]$,
  $\sin(0) = 0$ a $\sin(\pi / 2) = 1$.
 \item $\sin'(0) = 1$.
 \end{enumerate}
\end{theorem}

K důkazu použijeme následující pomocné lemma.

\begin{lemma*}{Pomocné}
 Ať $x \in \R$. Pak existuje $C > 0$ takové, že pro každé $n \in \N$ a každé
 $h \in (-1,1)$ platí nerovnost
 \[
  |(x+h)^{n} - x^{n} - nhx^{n-1}| \leq h^2C^{n}.
 \]
\end{lemma*}
\begin{lemproof}
 Položme $C \coloneqq 2(|x| + 1)$. Pro $n = 1$ máme
 \[
  |(x+h)^{1} - x^{1} - hx^{0}| = |x + h - x - h| = 0 \leq 2h^2 (|x| + 1)
 \]
 pro každé $h \in (-1,1)$.
 
 Pro $n \geq 2$ lze použít binomickou větu a počítat
 \begin{equation*}
  \label{eq:sin-derivative}
  \tag{$\triangle$}
  (x+h)^{n} - x^{n} - nhx^{n-1} = \sum_{k=0}^{n} \binom{n}{k} x^{n-k} h^{k}
  \clr{-~x^{n} - nhx^{n-1}} = \sum_{\clr{k=2}}^{n} \binom{n}{k} x^{n-k}h^{k}.
 \end{equation*}
 Protože $|x| + 1 \geq |x|$ pro každé $x \in \R$, platí $(|x| + 1)^{n} \geq
 |x|^{k}$, kdykoli $k \leq n$. Rovněž, z~předpokladu $|h|<1$, a tedy naopak
 platí $|h|^{k} \leq |h|^{n}$ pro $k \leq n$. Užitím \clb{těchto nerovností} a
 rovnosti~\eqref{eq:sin-derivative} můžeme odhadnout
 \begin{align*}
  |(x+h)^{n} - x^{n} - nhx^{n-1}| & \leq \sum_{k=2}^{n}
  \binom{n}{k}|x|^{n-k}|h|^{k} \clb{~\leq} \sum_{k=2}^{n} \binom{n}{k} (|x| +
  1)^{n}h^2\\
                                  & \leq h^2(|x| + 1)^{n} \sum_{k=0}^{n}
                                  \binom{n}{k} = h^2(|x| + 1)^{n} 2^{n} =
                                  h^2C^{n},
 \end{align*}
 čímž je důkaz hotov.
\end{lemproof}

\begin{thmproof}[\myref{Věty}{thm:vlastnosti-goniometrickych-funkci}]
 Je zřejmé, že řady
 \[
  \sum_{n=0}^{\infty} (-1)^{n} \frac{x^{2n+1}}{(2n+1)!} \quad \text{a} \quad
  \sum_{n=0}^{\infty} (-1)^{n} \frac{x^{2n}}{(2n)!}
 \]
 konvergují absolutně (použitím stejného argumentu jako v důkazu korektnosti
 exponenciály ve \myref{větě}{thm:vlastnosti-exponencialy}). Podle
 \myref{lemmatu}{lem:absolutni-konvergence-a-konvergence} jsou obě řady rovněž
 konvergentní pro každé $x \in \R$, což dokazuje dobrou definovanost obou
 funkcí.

 Ukážeme nejprve, že $\sin'x = \cos x$. Volme pevné $x \in \R$. Pro $h \in
 (-1,1)$ platí
 \begin{align*}
  \sin(x+h) - \sin x - h \cos x &= \sum_{n=0}^{\infty} (-1)^{n}\left(
  \frac{(x+h)^{2n+1} - x^{2n+1}}{(2n+1)!} - \frac{hx^{2n}}{(2n)!}\right)\\
  &= \sum_{n=0}^{\infty} \frac{(-1)^{n}}{(2n+1)!} ((x+h)^{2n+1} - x^{2n+1} -
  h(2n+1)x^{2n}).
 \end{align*}
 Z \textbf{pomocného lemmatu} nalezneme $C > 0$ takové, že
 \[
  |(x+h)^{2n+1} - x^{2n+1} - h(2n+1)x^{2n}| \leq C^{2n+1}h^2.
 \]
 Pak
 \[
  |\sin(x+h) - \sin x - h\cos x| \leq \sum_{n=0}^{\infty}
  \frac{C^{2n+1}}{(2n+1)!}h^2 = h^2 \sum_{n=0}^{\infty}
  \frac{C^{2n+1}}{(2n+1)!}.
 \]
 Řada $\sum_{n=0}^{\infty} C^{2n+1} / (2n+1)!$ je konvergentní, a tedy
 \[
  \lim_{h \to 0} \left|\frac{\sin(x+h) - \sin x - h \cos x}{h}\right| = 0,
 \]
 z čehož ihned
 \[
  \sin'x = \lim_{h \to 0} \frac{\sin(x+h)-\sin x}{h} = \cos x.
 \]

 Pro důkaz (G1) volme pevné $a \in \R$ a položme
 \[
  \psi(x) \coloneqq (\sin(x + a) - \sin x \cos a - \sin a \cos x)^2
  + (\cos(x + a) - \cos x \cos a + \sin a \sin x)^2.
 \]
 Snadno spočteme, že $\psi'(x) = 0$ pro každé $x \in \R$, a tedy je díky
 \myref{cvičení}{exer:derivace-nula-konstantni} $\psi$ konstantní na $\R$.
 Dosazením dostaneme, že
 \begin{align*}
  \sin(0) &= \sum_{n=0}^{\infty} (-1)^{n} \frac{0^{2n+1}}{(2n+1)!} = 0,\\
  \cos(0) &= \sum_{n=0}^{\infty} (-1)^{n} \frac{0^{2n}}{(2n)!} =
  (-1)^{0}\frac{0^{0}}{0!} + \sum_{n=1}^{\infty} (-1)^{n}\frac{0^{2n}}{(2n!)} =
  1.
 \end{align*}
 Díky těmto rovnostem spočteme $\psi(0) = 0$. Z toho, že $\psi$ je konstantní,
 plyne, že $\psi(x) = 0$ pro každé $x \in \R$. To dokazuje obě rovnosti v (G1),
 neboť $\psi$ je nulová funkce, jež je zároveň součtem čtverců, které musejí být
 tudíž oba nulové.

 Vlastnost (G2) je vidět ihned z definice, neboť proměnná $x$ se v definici
 $\sin$ objevuje pouze v~liché mocnině a v definici $\cos$ pouze v sudé.

 Vlastnost (G3) dokazovat nebudeme. Je výpočetně náročná a neintuitivní.

 Již víme, že $\sin'(x) = \cos x$ a že $\cos(0) = 1$. Odtud (G4).
\end{thmproof}

\begin{remark}{}{sin-cos-jako-exp}
 V úvodu jsme zmínili, že $\sin$ a $\cos$ jsou vlastně exponenciální funkce.
 Vskutku, když se jeden zadívá na jejich řady, vidí (až na znaménko $(-1)^{n}$
 zařizující právě onen \uv{růst a pokles}) v~zásadě exponenciální funkci.
 Konkrétně, $\sin$ je rozdílem \emph{lichých} částí exponenciály a $\cos$ těch
 \emph{sudých}. Rozdělme $\exp x$ na čtyři části podle zbytku po dělení indexu
 $n$ čtyřmi.
 \[
  \exp x = \clr{\sum_{n \bmod 4 = 0} \frac{x^{n}}{n!}}~+ \clb{\sum_{n \bmod 4 =
  1} \frac{x^{n}}{n!}}~+\clg{\sum_{n \bmod 4 = 2} \frac{x^{n}}{n!}}~+\clm{\sum_{n \bmod 4 =
  3} \frac{x^{n}}{n!}}
 \]
 Označme tyto části $\clr{\exp_0}, \clb{\exp_1}, \clg{\exp_2}$ a $\clm{\exp_3}$.
 Všimněme si, že když $n \bmod 4 = a$, pak $n = 4k + a$ pro nějaké $k \in \N$.
 Čili například $\clg{\exp_2}$ lze zapsat ve tvaru
 \[
  \clg{\exp_2}(x) = \sum_{k=0}^{\infty} \frac{x^{4k + 2}}{(4k + 2)!}
 \]
 Tvrdíme, že $\sin = \clb{\exp_1}~-~\clm{\exp_3}$ a $\cos = \clr{\exp_0}~-~
 \clg{\exp_2}$. Vskutku, když je $n$ liché, pak $2n+1 \bmod 4 = 3$ (protože $4
 \nmid 2n$), a když je $n$ sudé, tak $2n + 1 \bmod 4 = 1$. Čili, pro $n$ lichá
 je $2n + 1$ tvaru $4k+3$ a pro $n$ sudá zase tvaru $4k+1$. Můžeme tedy psát
 \begin{align*}
  \clb{\exp_1}(x) - \clm{\exp_3}(x) &= \sum_{k=0}^{\infty}
  \frac{x^{4k+1}}{(4k+1)!} - \sum_{k=0}^{\infty} \frac{x^{4k+3}}{(4k+3)!}\\
                                    &=\sum_{n \text{ sudé}}
                                    \frac{x^{2n+1}}{(2n+1)!} - \sum_{n \text{
                                    liché}} \frac{x^{2n+1}}{(2n+1)!} =
                                    \sum_{n=0}^{\infty}
                                    (-1)^{n}\frac{x^{2n+1}}{(2n+1)!} = \sin x,
 \end{align*}
 neboť $(-1)^{n}$ je rovno $1$ pro $n$ sudé a $-1$ pro $n$ liché. Podobně
 odvodíme i vztah pro $\cos$.
\end{remark}

\begin{definition}{Tangens a kotangens}{tangens-a-kotangens}
 Definujeme goniometrické funkce $\tan$ a $\cot$ předpisy
 \[
  \tan x = \frac{\sin x}{\cos x}, \quad \cot x = \frac{\cos x}{\sin x}.
 \]
 Funkce $\tan$ je definována pro $x \neq n\pi + \pi / 2$, kde je funkce
 $\cos$ nulová, a $\cot$ je definována pro $x$ různé od násobků $\pi$.
\end{definition}

Zformulujeme si několik vlastností funkcí $\tan$ a $\cot$, ale dokazovat je
nebudeme. Důkazy se významně neliší od již spatřených důkazů vlastností jiných
elementárních funkcí.

\begin{proposition}{Vlastnosti tangenty a kotangenty}{vlastnosti-tangenty-a-kotangenty}
 Platí:
 \begin{enumerate}[label=(G\arabic*)]
  \setcounter{enumi}{4}
  \item $\tan$ i $\cot$ jsou spojité na svých doménách;
  \item $\tan$ i $\cot$ jsou liché;
  \item $\tan'x = 1 / \cos^2 x$ a $\cot'x = - 1 / \sin^2 x$;
  \item $\tan \frac{\pi}{4} = \cot \frac{\pi}{4} = 1$;
  \item $\lim_{x \to \frac{\pi}{2}^{-}} \tan x = \infty$ a $\lim_{x \to
   \frac{\pi}{2}^{+}} \tan x = -\infty$.
  \item $\lim_{x \to 0^{+}} \cot x = \infty$ a $\lim_{x \to \pi^{-}} \cot x =
   -\infty$.
  \item $\tan$ je rostoucí na $(-\pi / 2, \pi / 2)$ a $\cot$ je klesající na
   $(0,\pi)$.
 \end{enumerate}
\end{proposition}

\section{Limity elementárních funkcí}
\label{sec:limity-elementarnich-funkci}

Tato sekce je věnována výpočtu limit, ve kterých figurují elementární funkce.
Nejtěžšími (ale zároveň nejužitečnějšími) úlohami na vyřešení jsou limity
racionálních kombinací (tj. součtů, násobků a především podílů) elementárních
funkcí. Samotné řešení pak obvykle zahrnuje převod výrazu do tvaru, v němž lze
naň uplatnit jisté \uv{známé} limity, případně použít
\hyperref[thm:lhospitalovo-pravidlo]{l'Hospitalova pravidla}.

\begin{proposition}{Běžné limity}{bezne-limity}
 Platí
 \begingroup
 \addtolength{\jot}{3mm}
 \begin{align}
  \lim_{x \to 0} \frac{\sin x}{x} &= 1,\tag{a}\\
  \lim_{x \to 0} \frac{1-\cos x}{x^2} &= \frac{1}{2},\tag{b}\\
  \lim_{x \to 0} \frac{\exp x - 1}{x} &= 1,\tag{c}\\
  \lim_{x \to 0} \frac{\log(1+x)}{x} &= 1.\tag{d}
 \end{align}
 \endgroup
\end{proposition}
\begin{propproof}
 K výpočtu všech limit lze použít
 \hyperref[thm:lhospitalovo-pravidlo]{l'Hospitalova pravidla}. Máme
 \begingroup
 \addtolength{\jot}{3mm}
 \begin{align}
  \lim_{x \to 0} \frac{\sin x}{x} &= \lim_{x \to 0} \frac{\cos x}{1} = \cos(0) =
  1,\tag{a}\\
  \lim_{x \to 0} \frac{1-\cos x}{x^2} &= \lim_{x \to 0} \frac{\sin x}{2x} =
  \frac{1}{2} \lim_{x \to 0} \frac{\sin x}{x} \overset{\text{(a)}}{=}
  \frac{1}{2},\tag{b}\\
  \lim_{x \to 0} \frac{\exp x - 1}{x} &= \lim_{x \to 0} \frac{\exp x}{1} =
  \exp(0) = 1,\tag{c}\\
  \lim_{x \to 0} \frac{\log(1+x)}{x} &= \lim_{x \to 0} \frac{\frac{1}{1+x}}{1} =
  1.\tag{d}
 \end{align}
 \endgroup
 Tím je důkaz hotov.
\end{propproof}

Aplikujeme nyní \myref{tvrzení}{prop:bezne-limity} k výpočtu některých limit
kombinací elementárních funkcí.

\begin{problem}{}{elementarni-limity-1}
 Spočtěte
 \[
  \lim_{x \to 0} \frac{\cos(\sin x) - 1}{\log(\sqrt{1 + x^2})}.
 \]
\end{problem}
\begin{probsol}
 Zřejmě platí
 \[
  \lim_{x \to 0} \cos(\sin x) - 1 = \lim_{x \to 0} \log(\sqrt{1 + x^2}) = 0,
 \]
 je tedy třeba výraz nejprve upravit.
 \hyperref[thm:lhospitalovo-pravidlo]{l'Hospitalovo pravidlo} zde pravděpodobně
 není vhodným prostředkem, neboť derivace obou funkcí jsou značně komplikované.
 Učiníme nejprve úpravu
 \[
  \frac{\cos(\sin x) - 1}{\log(\sqrt{1 + x^2})} = \frac{\cos(\sin x) -
  1}{\sin^2 x} \cdot \frac{\sin^2 x}{\log(\sqrt{1 + x^2})}.
 \]
 Protože $\lim_{x \to 0} \sin x / x = 1$, platí z
 \hyperref[thm:limita-slozene-funkce]{věty o limitě složené funkce}
 \[
  \lim_{x \to 0} \frac{\cos(\sin x) - 1}{\sin^2 x} = \lim_{y \to 0} \frac{\cos y
  - 1}{y^2} = -\frac{1}{2}.
 \]
 Dále máme (z \hyperref[prop:vlastnosti-logaritmu]{vlastností logaritmu})
 \[
  \frac{\sin^2 x}{\log(\sqrt{1 + x^2})} = \frac{\sin^2 x}{x^2} \cdot
  \frac{x^2}{\frac{1}{2}\log(1 + x^2)} = 2 \cdot \frac{\sin x}{x} \cdot
  \frac{\sin x}{x} \cdot \frac{x^2}{\log(1 + x^2)}.
 \]
 Opět z \hyperref[thm:limita-slozene-funkce]{věty o limitě složené funkce} jest
 \[
  \lim_{x \to 0} \frac{x^2}{\log(1 + x^2)} = \lim_{y \to 0} \frac{y}{\log(1 +
  y)} = 1.
 \]
 Celkem tedy,
 \[
  \lim_{x \to 0} \frac{\sin^2 x}{\log(\sqrt{1 + x^2})} = 2 \cdot \lim_{x \to 0}
  \frac{\sin x}{x} \cdot \lim_{x \to 0} \frac{\sin x}{x} \cdot \lim_{x \to 0}
  \frac{x^2}{\log(1 + x^2)} = 2 \cdot 1 \cdot 1 \cdot 1 = 2.
 \]
 Spolu s předchozím výpočtem dostaneme
 \[
  \lim_{x \to 0} \frac{\cos(\sin x) - 1}{\log(\sqrt{1 + x^2})} = -\frac{1}{2}
  \cdot 2 = -1.
 \]
\end{probsol}

Před další úlohou spočteme další obecně užitečnou limitu.
\begin{lemma}{}{log-podle-x}
 Pro každá $\alpha>0,\beta>0$ platí
 \[
  \lim_{x \to \infty} \frac{\log^{\alpha}x}{x^{\beta}} = 0.
 \]
\end{lemma}
\begin{lemproof}
 Protože $\lim_{x \to \infty} x^{\beta} = \infty$, použijeme
 \hyperref[thm:lhospitalovo-pravidlo]{l'Hospitalovo pravidlo}. Dostaneme
 \[
  \lim_{x \to \infty} \frac{\log^{\alpha} x}{x^{\beta}} = \frac{\alpha}{\beta}
  \cdot \frac{\log^{\alpha - 1}x \cdot \frac{1}{x}}{x^{\beta - 1}} =
  \frac{\alpha}{\beta} \cdot \frac{\log^{\alpha-1} x}{x^{\beta}}.
 \]
 Snadno nahlédneme, že se každým použitím
 \hyperref[thm:lhospitalovo-pravidlo]{l'Hospitalova pravidla} mocnina u $\log$
 sníží o $1$ (dokud je větší než $0$) a mocnina u $x$ zůstane stejná. Nalezneme
 tedy $n \in \N$ takové, že $\alpha - n \in (-1,0]$. Iterovaným použitím
 \hyperref[thm:lhospitalovo-pravidlo]{l'Hospitalova pravidla} upravíme původní
 limitu na
 \[
  \lim_{x \to \infty} \frac{\prod_{i=0}^{n-1} \alpha - i}{\beta^{n}} \cdot
  \frac{\log^{\alpha - n} x}{x^{\beta}} = \frac{\prod_{i=0}^{n-1} \alpha -
  i}{\beta^{n}} \cdot \lim_{x \to \infty} \frac{1}{\log^{n-\alpha}x \cdot
  x^{\beta}}.
 \]
 Jelikož $n-\alpha \geq 0$ a $\beta>0$, dostáváme, že
 \[
  \lim_{x \to \infty} \log^{n-\alpha}x \cdot x^{\beta} = \lim_{x \to \infty}
  \log^{n-\alpha}x \cdot \lim_{x \to \infty} x^{\beta} = \infty,
 \]
 z čehož již plyne dokazovaná rovnost.
\end{lemproof}

\begin{problem}{}{elementarni-limity-2}
 Spočtěte
 \[
  \lim_{x \to \infty} \sqrt{\log \left( 1 + \frac{3}{x} \right)} \cdot
  \log^2(1+x^3).
 \]
\end{problem}
\begin{probsol}
 Upravíme
 \[
  \sqrt{\log \left( 1 + \frac{3}{x} \right)} \cdot \log^2(1+x^3) =
  \frac{\sqrt{\log \left( 1 + \frac{3}{x} \right)}}{\sqrt{\frac{3}{x}}} \cdot
  \sqrt{\frac{3}{x}} \cdot \log^2(1 + x^3).
 \]
 Jelikož
 \[
  \frac{\sqrt{\log \left( 1 + \frac{3}{x} \right)}}{\sqrt{\frac{3}{x}}} =
  \sqrt{\frac{\log \left(1 + \frac{3}{x}\right)}{\frac{3}{x}}}
 \]
 a $\lim_{x \to \infty} 3 / x = 0$, plyne z \hyperref[prop:bezne-limity]{běžných
 limit} a z \hyperref[thm:limita-slozene-funkce]{věty o limitě složené funkce},
 že
 \[
  \lim_{x \to \infty} \frac{\sqrt{\log \left( 1 + \frac{3}{x}
  \right)}}{\sqrt{\frac{3}{x}}} = \sqrt{1} = 1.
 \]
 Dále,
 \[
  \sqrt{\frac{3}{x}} \cdot \log^2(1 + x^3) = \frac{\log^2(1 +
  x^3)}{\sqrt{\frac{x}{3}}} = \frac{\log^2(1 + x^3)}{\sqrt[6]{1+x^3}} \cdot
  \frac{\sqrt[6]{1+x^3}}{\sqrt{\frac{x}{3}}}.
 \]
 Podle \hyperref[lem:log-podle-x]{předchozího lemmatu} (a
 \hyperref[thm:limita-slozene-funkce]{věty o limitě složené funkce}) platí
 \[
  \lim_{x \to \infty} \frac{\log^2(1 + x^3)}{\sqrt[6]{1+x^3}} = 0.
 \]
 Jelikož
 \[
  \lim_{x \to \infty} \frac{\sqrt[6]{1+x^3}}{\sqrt{\frac{x}{3}}} = \lim_{x \to
  \infty} \frac{\sqrt{x} \sqrt[6]{\frac{1}{x^3} +
  1}}{\sqrt{x}\sqrt{\frac{1}{3}}} = \sqrt{3},
 \]
 dostáváme celkem, že
 \[
  \lim_{x \to \infty} \sqrt{\log \left( 1+\frac{3}{x} \right)} \cdot \log^2(1 +
  x^3) = 1 \cdot 0 \cdot \sqrt{3} = 0.
 \]
\end{probsol}

\begin{problem}{}{elementarni-limity-3}
 Spočtěte
 \[
  \lim_{x \to 0^{+}} \left( \frac{4^{x} + 5^{x} + 6^{x}}{3}
  \right)^{\frac{1}{x}}.
 \]
\end{problem}
\begin{probsol}
 Z definice obecné mocniny máme
 \[
  \left( \frac{4^{x} + 5^{x} + 6^{x}}{3} \right)^{\frac{1}{x}} = \exp \left(
  \frac{1}{x} \cdot \log \left( \frac{4^{x} + 5^{x} + 6^{x}}{3} \right)
 \right).
 \]
 Spočteme
 \[
  \lim_{x \to 0^{+}} \frac{1}{x} \cdot \log \left( \frac{4^{x} + 5^{x} +
  6^{x}}{3} \right)
 \]
 a výslednou limitu dostaneme přes \hyperref[thm:limita-slozene-funkce]{větu o
 limitě složené funkce}.

 Jelikož
 \[
  \lim_{x \to 0^{+}} \log \left( \frac{4^{x} + 5^{x} + 6^{x}}{3} \right) =
  \lim_{x \to 0^{+}} x = 0,
 \]
 lze použít \hyperref[thm:lhospitalovo-pravidlo]{l'Hospitalovo pravidlo}. Podle
 něj a \hyperref[prop:vlastnosti-obecne-mocniny]{vlastností obecné mocniny}
 platí
 \begin{align*}
  \lim_{x \to 0^{+}} \frac{\log \left( \frac{4^{x} + 5^{x} + 6^{x}}{3}
  \right)}{x} &= \lim_{x \to 0^{+}} \frac{3}{4^{x} + 5^{x} + 6^{x}} \cdot
  \frac{4^{x} \log 4 + 5^{x} \log 5 + 6^{x} \log 6}{3}\\
              &= \frac{3}{1 + 1 + 1} \cdot \frac{\log 4 + \log 5 + \log 6}{3} =
              \frac{\log 120}{3}.
 \end{align*}
 Nyní můžeme dopočíst
 \begin{align*}
  \lim_{x \to 0^{+}} \exp \left( \frac{1}{x} \cdot \log \left( \frac{4^{x} +
   5^{x} + 6^{x}}{3} \right) \right) &= \exp \left( \lim_{x \to 0^{+}}
 \frac{1}{x} \cdot \log \left( \frac{4^{x} + 5^{x} + 6^{x}}{3} \right) \right)\\
                                     &= \exp \left( \frac{\log 120}{3} \right) =
                                     120 - \exp 3.
 \end{align*}
\end{probsol}

\begin{exercise}{Pár limit elementárních funkcí}{par-limit-elementarnich-funkci}
 Spočtěte následující limity.
 {\everymath={\displaystyle}
  \arraycolsep=1em
  \[
   \begin{array}{lll}
    \lim_{x \to 3 \pi / 2} (4x^2 - 9\pi^2) \frac{\cos x}{1 + \sin x} & \lim_{x
    \to 0} \frac{1 - \cos(\arctan x)}{x^2} & \lim_{x \to \infty} \arcsin \left(
    \frac{1-x}{1+x}\right)\\[2em]
    \lim_{x \to 0^{+}} \frac{\exp \left( \frac{\sin x}{2} \right) - \cos
    (\sqrt{x})}{\log^2(1 + \sqrt{x})} & \lim_{x \to 1^{-}} \frac{\sqrt{\exp 2 -
   \exp 2x}}{\arccos x} & \lim_{x \to 0} \frac{\sqrt{x \sin x}}{\exp(x^2) -
   1}\\[2em]
   \lim_{x \to 0^{+}} \left( 1 - \sqrt{\arcsin x} \right)^{\frac{1}{\sqrt[4]{1 -
 \cos x}}} & \lim_{x \to \infty} \left( \frac{x+2}{2x-1} \right)^{x^2} & \lim_{x
 \to 0} (1 + x^2)^{\cot^2 x}
   \end{array}
  \]
 }
\end{exercise}

