\chapter{Primitivní funkce}
\label{chap:primitivni-funkce}

% TODO
\clr{\textbf{Tato kapitola se nachází v pracovní verzi. Neočekávejte obrázky,
		naopak očekávejte chyby a podivné formulace.}}

% TODO odkaz
Příčina vzniku \emph{primitivní funkce} (též \emph{integrálu}) reálné funkce je
veskrze fyzikální, jak bude objasněno v příští kapitole. My, majíce k dispozici
již úplnou teorii integrálu, půjdeme směrem opačným jejímu původnímu vývoji.
Začneme tedy definicí a základními vlastnostmi primitivních funkcí (vlastně
\emph{antiderivací}) a teprve poté si vysvětlíme jejich fyzikálně-geometrický
význam.

\begin{definition}{Primitivní funkce}{primitivni-funkce}
 Ať $(a,b) \subseteq \R$ je interval a $f:(a,b) \to \R$ reálná funkce. Řekneme,
 že $F:(a,b) \to \R$ je \emph{primitivní} k $f$ na $(a,b)$, když platí $F'(x) =
 f(x)$ pro všechna $x \in (a,b)$. Je-li interval zřejmý z~kontextu, píšeme též
 $F = \int f$ nebo \uv{úplněji} $F(x) = \int f(x) \, \mathrm{d}x$ pro $x \in
 (a,b)$.
\end{definition}

\begin{remark}{Proč otevřený interval?}{proc-otevreny-interval}
 Primitivní funkci definujeme pouze na \textbf{otevřeném intervalu}. Důvod tomu
 není nikterak hluboký, spíše technický. Při definici na uzavřeném intervalu
 bychom totiž museli v jeho krajních bodech uvažovat jednostranné derivace a tím
 zkomplikovat zápis i budoucí diskuse vlastností primitivních funkcí.
\end{remark}

Je zajímavé, že \uv{akce derivace} funkce způsobuje ztrátu pouze
\emph{konstantního} množství dat o této funkci. To se projevuje tak, že dvě
primitivní funkce na témže intervalu k téže funkci se liší pouze o konstantu.

\begin{lemma}{Jednoznačnost primitivní funkce}{jednoznacnost-primitivni-funkce}
 Ať $F,G:(a,b) \to \R$ jsou primitivní k $f:(a,b) \to \R$ na $(a,b)$. Pak
 existuje $c \in \R$ takové, že $F(x) = G(x) + c$ pro $x \in (a,b)$.
\end{lemma}
\begin{lemproof}
 Položme $H(x) \coloneqq F(x) - G(x)$ pro $x \in (a,b)$. Potom
 \[
  H'(x) = (F(x) - G(x))' = f(x) - f(x) = 0,
 \]
 tedy je podle \myref{cvičení}{exer:derivace-nula-konstantni} $H$ konstantní na
 $(a,b)$. Odtud tvrzení.
\end{lemproof}

\begin{remark}{}{seru-na-konstanty}
 V závěsu \hyperref[lem:jednoznacnost-primitivni-funkce]{předchozího lemmatu}
 budeme symbolem $\int f$ označovat \emph{kteroukoli} primitivní funkci k $f$ na
 daném intervalu a rozlišující konstantu zanedbáme. Formalizmem nakažený čtenář
 může přemýšlet o symbolu $\int f$ jako značícím \emph{třídu ekvivalence} všech
 funkcí jsoucích primitivních k $f$, kde relací je zde pochopitelně rovnost až
 na konstantu.
\end{remark}

\begin{example}{}{integral-polynomu}
 Platí
 \[
  \int x^{\alpha} \, \mathrm{d}x = 
  \begin{cases}
   \frac{1}{\alpha+1}x^{\alpha+1},& \alpha \neq -1,\\
   \log x,& \alpha = -1.
  \end{cases}
 \]
 Vskutku, máme $\log'x = 1 / x$ a
 \[
  \left( \frac{1}{\alpha+1}x^{\alpha+1} \right)' =
  \frac{\alpha+1}{\alpha+1}x^{\alpha+1-1} = x^{\alpha}
 \]
 pro $\alpha \neq -1$.
\end{example}

\begin{theorem}{Aritmetika primitivních funkcí}{aritmetika-primitivnich-funkci}
 Ať $f,g:(a,b) \to \R$ a $F:(a,b) \to \R$, resp. $G:(a,b) \to \R$, je primitivní
 k $f$, resp. ke $g$, na $(a,b)$. Ať dále $c,d \in \R$. Pak je $cF + dG$
 primitivní k $cf + dg$ na $(a,b)$.
\end{theorem}
\begin{thmproof}
 Zřejmý, z \hyperref[thm:aritmetika-derivaci]{věty o aritmetice derivací}.
 Ponechán jako snadné cvičení.
\end{thmproof}
\begin{exercise}{}{aritmetika-primitivnich-funkci}
 Dokažte \hyperref[thm:aritmetika-primitivnich-funkci]{větu o aritmetice
 primitivních funkcí}.
\end{exercise}

% TODO odkaz
Kompletní charakteristika funkcí, k nímž na jistém intervalu existuje
primitivní, je stále otevřeným problémem matematické analýzy. My si pouze
uvedeme, a v sekci o Riemannově integrálu, dokážeme, že spojité funkce
primitivní funkci mají vždy. Ovšem, tato podmínka je pouze postačující, nikoli
nutná. Existuje mnoho nespojitých funkcí, k nímž primitivní funkce existuje
rovněž.

\begin{theorem}{Spojitost a existence primitivní funkce}{spojitost-a-existence-primitivni-funkce}
 Ať $f:(a,b) \to \R$ je spojitá na $(a,b)$. Pak má na tomto intervalu primitivní
 funkci.
\end{theorem}
\begin{thmproof}
 % TODO odkaz
 Vyplyne ze základní věty kalkulu.
\end{thmproof}

Je ovšem známo mnoho vlastností, které funkce, majíc na jistém intervalu
primitivní funkci, splňovat musejí. Jedna z nich je tzv. \emph{Darbouxova
vlastnost} -- vlastnost funkce zobrazovat interval na interval, která je slabší
než spojitost, ale přesto do značné míry omezující.

\begin{theorem}{Darbouxova}{darbouxova}
 Ať má $f$ na $(a,b)$ primitivní funkci $F$. Pak pro každý interval $I \subseteq
 (a,b)$ platí, že $f(I)$ je rovněž interval.
\end{theorem}
\begin{thmproof}
 Volme interval $I \subseteq (a,b)$ libovolně a nechť $y_1,y_2 \in f(I)$. Je
 třeba ukázat, že je-li $y_1 < z < y_2$, pak $z \in f(I)$.

 Položme $H(x) \coloneqq F(x) - zx$, $x \in (a,b)$. Pak platí $H'(x) = f(x) -
 z$. Nalezněme z definice $f(I)$ čísla $x_1,x_2 \in I$ taková, že $f(x_1) = y_1$
 a $f(x_2) = y_2$. Protože má $H$ na $(a,b)$ konečnou derivaci, je na tomto
 intervalu podle \myref{lemmatu}{lem:vztah-derivace-a-spojitosti} spojitá.
 Jelikož $[x_1,x_2] \subseteq (a,b)$, nabývá podle
 \myref{věty}{thm:extremy-spojite-funkce} funkce $H$ na $[x_1,x_2]$ minima. Ať
 je tomu tak v bodě $x_0$.

 Ukážeme, že $x_0$ není ani jeden z bodů $x_1$, $x_2$. Máme $H'(x_1) = f(x_1) -
 z = y_1 - z < 0$, tj.
 \[
  \lim_{x \to x_1} \frac{H(x) - H(x_1)}{x - x_1} < 0,
 \]
 a tedy existuje okolí bodu $x_1$ na němž platí $H(x) < H(x_1)$. Ergo, $x_1$
 není bodem minima $H$ na $[x_1,x_2]$. Podobně odvodíme, že ani $x_2$ není bodem
 minima $H$. Pak je ale $x_0 \in (x_1,x_2)$ a podle
 \myref{tvrzení}{prop:vztah-derivace-a-extremu} platí $H'(x_0) = 0$, neboli
 $f(x_0) = z$, což dokazuje výrok $z \in f(I)$.
\end{thmproof}

\section{Výpočet primitivních funkcí}
\label{sec:vypocet-primitivnich-funkci}

Na rozdíl od výpočtu derivace funkce, je hledání funkce primitivní téměř vždy
silně nedeterministický úkon. Existují pomůcky k jejich výpočtu, jimž je tato
sekce ovšem věnována, avšak mnohdy musí jeden pracovat v jistém smyslu
\uv{zpětně}, tj. očekávat jistý výsledek a zpytovanou funkci upravit do tvaru,
která co nejblíže odpovídá derivaci právě tohoto očekávaného výsledku. Situaci
nezjednodušuje to, že mnoho (i snadno vyjádřitelných funkcí) nemá primitivní
funkci zapsatelnou použitím pouze funkcí elementárních. Jako příklad vezměme
funkci $\exp(x^2)$, která je na $\R$ spojitá, a tedy má, podle
\myref{věty}{thm:spojitost-a-existence-primitivni-funkce} na $\R$ primitivní
funkci. Avšak, a to bylo jest dokázáno, tato funkce nemá vyjádření pomocí
elementárních funkcí.

Základními nástroji pro výpočet primitivních funkcí jsou \emph{integrace per
partes} a \emph{substituční věty}. Uvedeme a dokážeme všechny.

\begin{theorem}{Integrace per partes}{integrace-per-partes}
 Ať $f,g:(a,b) \to \R$ jsou reálné funkce, $F$ je primitivní k $f$ a $G$
 primitivní ke $g$ na $(a,b)$. Pak platí
 \[
  \int (f \cdot G) = F \cdot G - \int (F \cdot g).
 \]
\end{theorem}
\begin{thmproof}
 Funkce $F$ i $G$ jsou spojité, protože mají podle předpokladu konečné derivace
 na $(a,b)$. Funkce $f$ i $g$ mají z předpokladu primitivní funkce, takže i
 funkce $f \cdot G$ a $F \cdot g$ mají primitivní funkce na $(a,b)$.

 Položme $H \coloneqq \int (F \cdot g)$. Pak z
 \hyperref[thm:aritmetika-derivaci]{aritmetiky derivací}
 \[
  (F \cdot G - H)' = f \cdot G + F \cdot g - F \cdot g = f \cdot G,
 \]
 což dokazuje kýženou rovnost.
\end{thmproof}

\begin{remark}{}{integral-derivace-lomeno-funkce}
 Pro další výpočty bude dobré si povšimnout, že pro libovolnou reálnou funkci
 $f$ platí
 \[
  \int \frac{f'(x)}{f(x)} \, \mathrm{d}x = \log(f(x)).
 \]
 Vskutku, plyne to z přímočarého výpočtu využívajícího
 \hyperref[thm:derivace-slozene-funkce]{větu o derivaci složené funkce}:
 \[
  (\log(f(x)))' = f'(x) \cdot \log'f(x) = f'(x) \cdot \frac{1}{f(x)}.
 \]
\end{remark}

\begin{example}{}{integral-arctan}
 Použitím \myref{věty}{thm:integrace-per-partes} spočteme
 \[
  \int \arctan x \, \mathrm{d}x
 \]
 pro $x \in (-\pi / 2, \pi / 2)$.

 Čtenáři použitím \hyperref[thm:derivace-inverzni-funkce]{věty o derivaci
 inverzní funkce} sobě snadno ověří, že
 \[
  \arctan'x = \frac{1}{1+x^2}.
 \]
 Označme $f(x) \coloneqq 1$ a $G(x) \coloneqq \arctan x$. Pak ve značení
 \myref{věty}{thm:integrace-per-partes} jest $F(x) = x$ a $g(x) = 1 / (1 +
 x^2)$. Počítáme
 \begin{align*}
  \int f(x)G(x) \, \mathrm{d}x &= F(x)G(x) - \int F(x)g(x) \, \mathrm{d}x\\
                               &= x \cdot \arctan x - \int \frac{x}{1+x^2} \,
                               \mathrm{d}x.
 \end{align*}
 Protože $(1+x^2)' = 2x$, poslední integrál snadno spočteme úpravou
 \[
  \int \frac{x}{1+x^2} \, \mathrm{d}x = \frac{1}{2} \int \frac{2x}{1+x^2} \,
  \mathrm{d}x = \frac{1}{2}\log(1+x^2).
 \]
 Celkem tedy máme
 \[
  \int \arctan x \, \mathrm{d}x = x \cdot \arctan x - \frac{1}{2}\log(1+x^2).
 \]
\end{example}

\begin{problem}{}{integral-xex}
 Pro $x \in \R$ spočtěte
 \[
  \int x\exp x \, \mathrm{d}x.
 \]
\end{problem}
\begin{probsol}
 Protože funkce $\exp x$ se nemění derivací či integrací, je dobré si před
 použitím \hyperref[thm:integrace-per-partes]{integrace per partes} rozmyslet,
 zda je druhou funkci jednodušší integrovat či derivovat. Zde platí $x' = 1$,
 tedy budeme $x$ derivovat a $\exp x$ integrovat. Ve značení
 \myref{věty}{thm:integrace-per-partes} máme $f(x) = \exp x$ a $G(x) = x$. Pak
 $F(x) = \exp x$ a $g(x) = 1$. Odtud,
 \begin{align*}
  \int f(x) G(x) \, \mathrm{d}x &= F(x)G(x) - \int F(x)g(x) \, \mathrm{d}x\\
                                &= \exp x \cdot x - \int \exp x \cdot 1 \,
                                \mathrm{d}x = x \exp x - \exp x,
 \end{align*}
 čímž je výpočet završen.
\end{probsol}

\begin{exercise}{}{per-partes}
 Spočtěte
 \[
  \int \frac{\log^2 x}{x^2} \, \mathrm{d}x.
 \]
\end{exercise}

\begin{theorem}{první o substituci}{prvni-o-substituci}
 Ať $a < b,\alpha < \beta \in \R$, $f:(a,b) \to \R$ je reálná funkce, $F$ je
 primitivní k $f$ na $(a,b)$ a $\varphi:(\alpha,\beta) \to (a,b)$ je reálná
 funkce s~konečnou $\varphi'$ na $(\alpha,\beta)$. Potom
 \[
  \int (f \circ \varphi)(t) \cdot \varphi'(t) \, \mathrm{d}t = (F \circ
  \varphi)(t).
 \]
\end{theorem}
\begin{thmproof}
 Podle \hyperref[thm:derivace-slozene-funkce]{věty o derivaci složené funkce}
 platí
 \[
  (F \circ \varphi)'(t) = F'(\varphi(t)) \cdot \varphi'(t) = f(\varphi(t)) \cdot
  \varphi'(t),
 \]
 což dokazuje tvrzení.
\end{thmproof}
\begin{problem}{}{substituce-1}
 Spočtěte
 \[
  \int \sin^{4}t \cos t \, \mathrm{d}t.
 \]
\end{problem}
\begin{probsol}
 Položme $f(x) = x^{4}$ a $\varphi(t) = \sin t$. Pak $F(x) = x^{5} / 5$,
 $\varphi'(t) = \cos t$ a
 \[
  f(\varphi(t)) \cdot \varphi'(t) = \sin^{4}t \cos t.
 \]
 Podle \hyperref[thm:prvni-o-substituci]{první věty o substituci} platí
 \[
  \int \sin^{4}t \cos t \, \mathrm{d}t = \int f(\varphi(t)) \cdot \varphi'(t) \,
  \mathrm{d}t = F(\varphi(t)) = \frac{\sin^{5} t}{5}.
 \]
\end{probsol}
\begin{exercise}{}{substituce-1}
 Spočtěte
 \[
  \int \frac{x}{\sqrt{2 + 5x^2}} \, \mathrm{d}x.
 \]
\end{exercise}
\begin{exercise}{}{substituce-2}
 Spočtěte
 \[
  \int \frac{1}{\sqrt{8 + 6x - 9x^2}} \, \mathrm{d}x.
 \]
 \textbf{Hint}: platí $\arcsin'x = 1 / \sqrt{1 - x^2}$.
\end{exercise}

\begin{theorem}{druhá o substituci}{druha-o-substituci}
 Ať $a < b, \alpha < \beta \in \R$, $f:(a,b) \to \R$, $\varphi:(\alpha,\beta)
 \to (a,b)$ je na a má na $(\alpha,\beta)$ konečnou nenulovou derivaci. Platí-li
 pro každé $t \in (\alpha,\beta)$ rovnost
 \[
  \int f(\varphi(t)) \cdot \varphi'(t) \, \mathrm{d}t = G(t), 
 \]
 pak
 \[
  \int f(x) \, \mathrm{d}x = (G \circ \varphi^{-1})(x)
 \]
 pro $x \in (a,b)$.
\end{theorem}
\begin{thmproof}
 Protože $\varphi'$ má na $(\alpha,\beta)$ primitivní funkci, zobrazuje podle
 \hyperref[thm:darbouxova]{Darbouxovy věty} interval na interval. Z předpokladu
 je $\varphi'$ nenulová, což spolu s předchozí větou implikuje, že buď
 $\varphi'>0$ nebo $\varphi'<0$ na celém $(\alpha,\beta)$. V prvním případě je
 $\varphi$ podle \myref{důsledku}{cor:vztah-derivace-a-monotonie} buď rostoucí,
 nebo klesající, na $(\alpha,\beta)$. V každém případě je prostá, a tedy
 existuje $\varphi^{-1}:(a,b) \to (\alpha,\beta)$.

 Podle \hyperref[thm:derivace-inverzni-funkce]{věty o derivaci inverzní funkce}
 a též \hyperref[thm:derivace-slozene-funkce]{věty o derivaci složené funkce}
 platí pro $x \in (a,b)$
 \begin{align*}
  (G \circ \varphi ^{-1})'(x) &= G'(\varphi ^{-1}(x)) \cdot (\varphi ^{-1})'(x)\\
                              &= f(\varphi(\varphi ^{-1}(x))) \cdot
                              \varphi'(\varphi ^{-1}(x)) \cdot (\varphi
                              ^{-1})'(x)\\
                              &= f(x) \cdot \varphi'(\varphi ^{-1}(x)) \cdot
                              \frac{1}{\varphi'(\varphi ^{-1}(x))} = f(x),
 \end{align*}
 jak bylo jest dokázati.
\end{thmproof}
\begin{example}{}{substituce-3}
 Pro $x \in (-1,1)$ platí
 \[
  \int \sqrt{1-x^2} \, \mathrm{d}x = \frac{1}{2}\arcsin x + \frac{1}{4}\sin(2
  \arcsin x).
 \]

 Vskutku, položme $f(x) = \sqrt{1-x^2}$, $(\alpha,\beta) = (-\pi / 2,\pi / 2)$ a
 $\varphi(t) = \sin t$. Pak $\varphi:(-\pi / 2,\pi / 2) \to (-1,1)$ je na a
 platí $\varphi'(t) = \cos t \neq 0$ pro $t \in (-\pi / 2, \pi / 2)$. Dále
 $\varphi ^{-1}(x) = \arcsin x$ a
 \[
  \int f(\varphi(t)) \cdot \varphi'(t) \, \mathrm{d}t = \int \sqrt{1 - \sin^2
  t} \cdot \cos t \, \mathrm{d}t = \int \cos^2 t \, \mathrm{d}t.
 \]
 Užitím \hyperref[thm:vlastnosti-goniometrickych-funkci]{vlastností
 goniometrických funkcí} se snadno ověří, že
 \[
  \cos^2 t = \frac{1}{2}(1 + \cos 2t).
 \]
 Odtud
 \[
  \int \cos^2 t \, \mathrm{d}t = \int \frac{1}{2} \, \mathrm{d}t + \int
  \frac{1}{2}\cos 2t \, \mathrm{d}t = \frac{1}{2}t + \frac{1}{4}\sin 2t.
 \]
 Čili, ve značení \hyperref[thm:druha-o-substituci]{druhé věty o substituci}
 jest
 \[
  G(t) = \frac{1}{2}t + \frac{1}{4}\sin 2t.
 \]
 Podle téže věty máme 
 \[
  \int \sqrt{1-x^2} \, \mathrm{d}x = \int f(x) \, \mathrm{d}x = G(\varphi
  ^{-1}(x)) = \frac{1}{2}\arcsin x + \frac{1}{4}\sin(2\arcsin x).
 \]
\end{example}

\subsection{Integrace racionálních funkcí}
\label{ssec:integrace-racionalnich-funkci}

Mezi funkce, jež umíme \uv{algoritmicky} integrovat, patří (dle
\myref{příkladu}{exam:integral-polynomu}) jistě funkce polynomiální. Zvlášť
užitečné v závěsu vět o substituci je rovněž umět algoritmicky integrovat funkce
\emph{racionální}, tedy reálné funkce tvaru $R(x) = p(x) / q(x($, kde $p,q$ jsou
polynomy.

Rádi bychom uměli libovolnou racionální funkci rozložit na součet výrazně
jednodušších racionálních funkcí, jejichž integrály umíme spočítat. To lze,
metodou slující \emph{rozklad na parciální zlomky}.

\begin{remark}{}{rozklad-realneho-polynomu}
 V následující větě použijeme sebou nedokázané tvrzení z algebry, že každý
 reálný polynom se rozkládá na součin lineárních a kvadratických činitelů. Tedy,
 je-li $p(x)$ polynom, pak existuje konstanta $c \in \R$ a pro vhodná $k,l \in
 \N$ čísla $a_1,\ldots,a_k,\alpha_1,\ldots,\alpha_l,\beta_1,\ldots,\beta_l \in
 \R$ a přirozená čísla $n_1,\ldots,n_k,m_1,\ldots,m_l \in \N$ splňující
 \[
  p(x) = c(x-a_1)^{n_1}(x-a_2)^{n_2} \cdots (x-a_k)^{n_k}(x^2 + \alpha_1 x +
  \beta_1)^{m_1} \cdots (x^2 + \alpha_l x + \beta_l)^{m_l},
 \]
 kde všechny $x^2 + \alpha_i x + \beta_i$ nemají reálný kořen.
\end{remark}

\begin{theorem}{Rozklad na parciální zlomky}{rozklad-na-parcialni-zlomky}
 Ať $p(x),q(x)$ jsou polynomy a $q$ je rozložený jako v
 \hyperref[rmrk:rozklad-realneho-polynomu]{poznámce výše}. Pak existují
 \emph{jednoznačně určená} čísla
 \begin{align*}
  A_1^{1},A_2^{1},\ldots,A_{n_1}^{1},A_1^{2},A_2^{2},\ldots,A_{n_2}^{2},A_1^{3},A_2^{3},\ldots,A_{1}^{k},\ldots,A_{n_k}^{k}
  & \in \R,\\
  B_1^{1},B_2^{1},\ldots,B_{m_1}^{1},B_1^{2},B_2^{2},\ldots,B_{m_2}^{2},B_1^{3},B_2^{3},\ldots,B_{1}^l,\ldots,B_{m_l}^{l}
  & \in \R,\\
  C_1^{1},C_2^{1},\ldots,C_{m_1}^{1},C_1^{2},C_2^{2},\ldots,C_{m_2}^{2},C_1^{3},C_2^{3},\ldots,C_{1}^l,\ldots,C_{m_l}^{l}
  & \in \R,
 \end{align*}
 taková, že
 \begin{align*}
  \frac{p(x)}{q(x)} &= \frac{A^{1}_1}{x - a_1} + \frac{A_2^{1}}{(x - a_1)^2} +
  \cdots + \frac{A_{n_1}^{1}}{(x-a_1)^{n_1}} + \frac{A_1^{2}}{x-a_2} + \cdots +
  \frac{A_{n_2}^{2}}{(x-a_2)^{n_2}} + \cdots +
  \frac{A_{n_k}^{k}}{(x-a_k)^{n_k}}\\
  &+ \frac{B_1^{1}x + C_1^{1}}{x^2 + \alpha_1x + \beta_1} + \frac{B_2^{1}x +
  C_2^{1}}{(x^2 + \alpha_1x + \beta_1)^2} + \cdots + \frac{B_{m_1}^{1}x +
  C^{1}_{m_1}}{(x^2 + \alpha_1x + \beta_1)^{m_1}} + \frac{B_1^{2}x +
  C_1^{2}}{x^2 + \alpha_2x + \beta_2}\\
  &+ \cdots + \frac{B_{m_2}^{2}x + C_{m_2}^{2}}{(x^2 + \alpha_2 x +
  \beta_2)^{m_2}} + \cdots + \frac{B_{m_l}^{l}x + C_{m_l}^{l}}{(x^2 + \alpha_l x
  + \beta_l)^{m_l}}\\
  &= \sum_{i=1}^k \sum_{j=1}^{n_i} \frac{A_j^{i}}{(x - a_i)^{j}} + \sum_{i=1}^l
  \sum_{j=1}^{m_i} \frac{B_j^{i}x + C_j^{i}}{(x^2 + \alpha_i x + \beta_i)^{j}}.
 \end{align*}
\end{theorem}
\begin{thmproof}
 Triviální, indukcí podle stupně $q$. Ale moc písmenek.
\end{thmproof}

\begin{example}{}{rozklad-na-parcialni-zlomky-1}
 Rozložíme
 \[
  \frac{p(x)}{q(x)} = \frac{3x^3 - 3x^2 + 4x + 10}{(x - 2)^{2}(x^2 + 2x + 2)}
 \]
 na parciální zlomky. Podle \myref{věty}{thm:rozklad-na-parcialni-zlomky}
 existují čísla $A_1,A_2,B,C \in \R$ taková, že
 \begin{equation*}
  \label{eq:parcialni-zlomky}
  \tag{$\heartsuit$}
  \frac{3x^3 - 3x^2 + 4x + 10}{(x-2)^2(x^2 + 2x + 2)} = \frac{A_1}{x - 2} + \frac{A_2}{(x-2)^2}
  + \frac{Bx + C}{x^2 + 2x + 2}.
 \end{equation*}
 Tato čísla nalezneme porovnáním koeficientů obou stran po roznásobení
 jmenovatelem a vyřešením získané soustavy rovnic.

 Roznásobením obou stran~\eqref{eq:parcialni-zlomky} polynomem $q(x)$ dostaneme
 \[
  3x^3 - 3x^2 + 4x + 10 = A_1(x-2)(x^2 + 2x + 2) + A_2(x^2 + 2x + 2) + (Bx +
  C)(x-2)^2.
 \]
 Tato rovnost musí platit pro všechna $x \in \R$, speciálně tedy pro $x = 2$.
 Dosazením dostaneme první rovnici
 \[
  3 \cdot 2^3 - 3 \cdot 2^2 + 4 \cdot 2 + 10 = A_2(2^2 + 2 \cdot 2 + 2),
 \]
 z níž ihned $A_2 = 3$. Nyní dosaďme například $x = 0$. Dosazení dá
 \[
  4 = -4A_1 + 4C.
 \]
 Volbami $x = 1$ a $x = 3$ dostaneme další dvě rovnice
 \begin{align*}
  -1 &= -5A_1 + B + C,\\
  25 &= 17A_1 + 3B + C.
 \end{align*}
 Snadno spočteme, že řešením soustavy těchto tří rovnic je $A_1 = 1, B = 2$ a $C
 = 2$. Platí tedy
 \[
  \frac{3x^3 - 3x^2 + 4x + 10}{(x-2)^2(x^2 + 2x + 2)} = \frac{1}{x - 2} +
  \frac{3}{(x-2)^2} + \frac{2x + 2}{x^2 + 2x + 2}.
 \]
\end{example}

Pochopitelně, rozklad na parciální zlomky nám není mnoho užitečný, neumíme-li
spočítat integrály typu
\[
 \int \frac{A}{(x - a)^{k}} \, \mathrm{d}x \quad \text{a} \quad \int \frac{Bx +
 C}{(x^2 +\alpha x + \beta)^{l}} \, \mathrm{d}x, 
\]
pro $a,\alpha,\beta,A,B,C \in \R$ a $k,l \in \N$. První z těchto integrálů už
jsme v zásadě spočetli v \myref{příkladě}{exam:integral-polynomu}. Druhý je na
výpočet výrazně těžší a hledaná primitivní funkce se nejsnadněji zapisuje
rekurzivně.

Čtenáři snadno ověří, že
\[
 \int \frac{A}{(x - a)^{k}} \, \mathrm{d}x = \begin{cases}
  \frac{A}{(1 - k)(x - a)^{k-1}},& k > 1,\\
  A\log|x - a|, & k = 1.
 \end{cases}
\]
Integrál druhého typu nejprve rozložíme
\[
 \int \frac{Bx + C}{(x^2 + \alpha x + \beta)^{l}} \, \mathrm{d}x =
 \frac{B}{2}\int \frac{2x + \alpha}{(x^2 + \alpha x + \beta)^{l}} \, \mathrm{d}x
 + \left( C - \frac{B\alpha}{2} \right) \int \frac{1}{(x^2 + \alpha x +
 \beta)^{l}} \, \mathrm{d}x.
\]
Protože $(x^2 + \alpha x + \beta)' = 2x + \alpha$, je první z těchto integrálů
rovněž spočten přímočaře. Vskutku,
\[
 \int \frac{2x + \alpha}{(x^2 + \alpha x + \beta)^{l}} \, \mathrm{d}x = \begin{cases}
  \frac{1}{1 - l} \cdot \frac{1}{(x^2 + \alpha x + \beta)^{l-1}}, & l > 1,\\
  \log(x^2 + \alpha x + \beta), & l = 1.
 \end{cases}
\]

Výraz v druhém integrálu nejprve upravíme
\[
 \frac{1}{(x^2 + \alpha x + \beta)^{l}} = \frac{1}{\left( \left( x +
 \frac{\alpha}{2} \right)^2 + \beta - \frac{\alpha^2}{4} \right)^{l}} =
 \frac{1}{\left( \beta - \frac{\alpha^2}{4} \right)^{l}} \cdot \frac{1}{\left( 1
 + \left( \frac{x + \frac{\alpha}{2}}{\sqrt{\beta - \frac{\alpha^2}{4}}}
 \right)^2\right)^{l}}.
\]

Položíme-li nyní $\varphi(x) \coloneqq \frac{x + \alpha / 2}{\sqrt{\beta -
\alpha^2 / 4}}$ a pro strohost zápisu označíme $t = \varphi(x)$, pak z
\hyperref[thm:prvni-o-substituci]{první věty o substituci} platí
\[
 \int \frac{1}{(x^2 + \alpha x + \beta)^{l}} \, \mathrm{d}x = \frac{1}{\left(
 \beta - \frac{\alpha^2}{4} \right)^{l}} \int \frac{1}{(1+t^2)^{l}} \,
 \mathrm{d}t. 
\]
Označme
\[
 I_l \coloneqq \int \frac{1}{(1+t^2)^{l}} \, \mathrm{d}t.
\]

\begin{lemma}{}{vypocet-integralu-Il}
 Platí
 \[
  I_{l+1} = \frac{t}{2l(1 + t^2)^{l}} + \frac{2l - 1}{2l}I_l
 \]
 pro $t \in \R$, přičemž $I_1 = \arctan t$.
\end{lemma}
\begin{lemproof}
 Jistě
 \[
  I_1 = \int \frac{1}{1+t^2} \, \mathrm{d}t = \arctan t.
 \]
 pro $t \in \R$.

 Dále, pro každé $l \in \N$ je $t \mapsto 1 / (1+t^2)^{l}$ spojitá funkce na
 $\R$, tudíž má podle \myref{věty}{thm:spojitost-a-existence-primitivni-funkce}
 primitivní funkci na celém $\R$.
 
 Z \hyperref[thm:integrace-per-partes]{věty o integraci per partes} platí
 \[
  \int 1 \cdot \frac{1}{(1 + t^2)^{l}} \, \mathrm{d}t = \frac{t}{(1 + t^2)^{l}}
  - \int \frac{-2lt^2}{(1+t^2)^{l+1}} \, \mathrm{d}t = \frac{t}{(1 + t^2)^{l}} +
  2l \int \frac{t^2}{(1 + t^2)^{l+1}} \, \mathrm{d}t. 
 \]
 Dále spočteme
 \[
  \int \frac{t^2}{(1 + t^2)^{l+1}} \, \mathrm{d}t = \int \frac{1}{(1 + t^2)^{l}}
  \, \mathrm{d}t - \int \frac{1}{(1 + t^2)^{l+1}} \, \mathrm{d}t = I_l -
  I_{l+1}.  
 \]
 To nám dává rovnost
 \[
  I_l = \int \frac{1}{(1+t^2)^{l}} \, \mathrm{d}t = \frac{t}{(1+t^2)^{l}} + 2l
  \int \frac{t^2}{(1+t^2)^{l+1}} \, \mathrm{d}t = \frac{t}{(1 + t^2)^{l}} +
  2l(I_l - I_{l+1}),
 \]
 z níž již snadnou úpravou plyne tvrzení.
\end{lemproof}

\begin{problem}{}{rozklad-na-parcialni-zlomky-2}
 Spočtěte
 \[
  \int \frac{1}{1 + x^{4}} \, \mathrm{d}x.
 \]
\end{problem}
\begin{probsol}
 Nejprve rozložíme
 \[
  x^{4} + 1 = (x^2 + 1)^2 - 2x^2 = (x^2 + \sqrt{2}x + 1)(x^2 - \sqrt{2}x + 1).
 \]
 Podle \hyperref[thm:rozklad-na-parcialni-zlomky]{věty o rozkladu na parciální
 zlomky} existují $B_1,C_1,B_2,C_2 \in \R$ taková, že
 \[
  \frac{1}{x^{4} + 1} = \frac{B_1x + C_1}{x^2 + \sqrt{2}x + 1} + \frac{B_2x +
  C_2}{x^2 - \sqrt{2}x + 1}.
 \]
 Roznásobením této rovnosti polynomem $x^{4} + 1$ získáme
 \[
  1 = (B_1x + C_1)(x^2 + \sqrt{2}x + 1) + (B_2x + C_2)(x^2 - \sqrt{2}x + 1).
 \]
 Postupným dosazením čtyř libovolných hodnot za $x$ dostaneme soustavu čtyř
 lineárních rovnic s řešením
 \[
  B_1 = \frac{1}{2 \sqrt{2}}, \; C_1 = \frac{1}{2}, \; B_2 = -\frac{1}{2
  \sqrt{2}}, \; C_2 = \frac{1}{2}.
 \]
 Čili,
 \[
  \int \frac{1}{1+x^{4}} \, \mathrm{d}x = \int \frac{\frac{1}{2 \sqrt{2}}x +
  \frac{1}{2}}{x^2 + \sqrt{2}x + 1} \, \mathrm{d}x + \int \frac{-\frac{1}{2
  \sqrt{2}}x + \frac{1}{2}}{x^2 - \sqrt{2}x + 1} \, \mathrm{d}x.
 \]
 První integrál rozložíme jako
 \[
  \int \frac{\frac{1}{2 \sqrt{2}}x + \frac{1}{2}}{x^2 + \sqrt{2}x + 1} \,
  \mathrm{d}x = \int \frac{\frac{1}{4 \sqrt{2}}(2x + \sqrt{2})}{x^2 + \sqrt{2}x
  + 1} \, \mathrm{d}x + \int \frac{\frac{1}{4}}{x^2 + \sqrt{2}x + 1} \,
  \mathrm{d}x.
 \]
 Snadno spočteme, že
 \[
  \int \frac{\frac{1}{4 \sqrt{2}}(2x + \sqrt{2})}{x^2 + \sqrt{2}x + 1} \,
  \mathrm{d}x = \frac{1}{4 \sqrt{2}}\log(x^2 + \sqrt{2}x + 1).
 \]
 Dále,
 \[
  \int \frac{\frac{1}{4}}{x^2 + \sqrt{2}x + 1} \, \mathrm{d}x = \frac{1}{2}\int
  \frac{1}{(\sqrt{2}x + 1)^2 + 1}\, \mathrm{d}x = \frac{1}{2
  \sqrt{2}}\arctan(\sqrt{2}x + 1).
 \]

 Obdobným postupem dostaneme i primitivní funkci
 \[
  \int \frac{-\frac{1}{2 \sqrt{2}}x + \frac{1}{2}}{x^2 - \sqrt{2}x + 1} \,
  \mathrm{d}x = -\log(x^2 - \sqrt{2}x + 1) + \frac{1}{2
  \sqrt{2}}\arctan(\sqrt{2}x - 1).
 \]
 Celkem tedy
 \begin{align*}
  \int \frac{1}{x^{4} + 1} \, \mathrm{d}x &= \frac{1}{4 \sqrt{2}}(\log(x^2 +
  \sqrt{2}x + 1) - \log(x^2 - \sqrt{2}x + 1))\\
  &+ \frac{1}{2 \sqrt{2}}(\arctan(\sqrt{2}x + 1) + \arctan(\sqrt{2}x - 1)).
 \end{align*}
\end{probsol}

\begin{exercise}{}{rozklad-na-parcialni-zlomky-3}
 Spočtěte
 \[
  \int \frac{1}{x(1+x)(1 + x + x^2)} \, \mathrm{d}x. 
 \]
\end{exercise}

