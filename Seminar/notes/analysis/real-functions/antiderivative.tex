\chapter{Primitivní funkce}
\label{chap:primitivni-funkce}

% TODO
\clr{\textbf{Tato kapitola se nachází v pracovní verzi. Neočekávejte obrázky,
		naopak očekávejte chyby a podivné formulace.}}

% TODO odkaz
Příčina vzniku \emph{primitivní funkce} (též \emph{integrálu}) reálné funkce je
veskrze fyzikální, jak bude objasněno v příští kapitole. My, majíce k dispozici
již úplnou teorii integrálu, půjdeme směrem opačným jejímu původnímu vývoji.
Začneme tedy definicí a základními vlastnostmi primitivních funkcí (vlastně
\emph{antiderivací}) a teprve poté si vysvětlíme jejich fyzikálně-geometrický
význam.

\begin{definition}{Primitivní funkce}{primitivni-funkce}
 Ať $(a,b) \subseteq \R$ je interval a $f:(a,b) \to \R$ reálná funkce. Řekneme,
 že $F:(a,b) \to \R$ je \emph{primitivní} k $f$ na $(a,b)$, když platí $F'(x) =
 f(x)$ pro všechna $x \in (a,b)$. Je-li interval zřejmý z~kontextu, píšeme též
 $F = \int f$ nebo \uv{úplněji} $F(x) = \int f(x) \, \mathrm{d}x$ pro $x \in
 (a,b)$.
\end{definition}

\begin{remark}{Proč otevřený interval?}{proc-otevreny-interval}
 Primitivní funkci definujeme pouze na \textbf{otevřeném intervalu}. Důvod tomu
 není nikterak hluboký, spíše technický. Při definici na uzavřeném intervalu
 bychom totiž museli v jeho krajních bodech uvažovat jednostranné derivace a tím
 zkomplikovat zápis i budoucí diskuse vlastností primitivních funkcí.
\end{remark}

Je zajímavé, že \uv{akce derivace} funkce způsobuje ztrátu pouze
\emph{konstantního} množství dat o této funkci. To se projevuje tak, že dvě
primitivní funkce na témže intervalu k téže funkci se liší pouze o konstantu.

\begin{lemma}{Jednoznačnost primitivní funkce}{jednoznacnost-primitivni-funkce}
 Ať $F,G:(a,b) \to \R$ jsou primitivní k $f:(a,b) \to \R$ na $(a,b)$. Pak
 existuje $c \in \R$ takové, že $F(x) = G(x) + c$ pro $x \in (a,b)$.
\end{lemma}
\begin{lemproof}
 Položme $H(x) \coloneqq F(x) - G(x)$ pro $x \in (a,b)$. Potom
 \[
  H'(x) = (F(x) - G(x))' = f(x) - f(x) = 0,
 \]
 tedy je podle \myref{cvičení}{exer:derivace-nula-konstantni} $H$ konstantní na
 $(a,b)$. Odtud tvrzení.
\end{lemproof}

\begin{remark}{}{seru-na-konstanty}
 V závěsu \hyperref[lem:jednoznacnost-primitivni-funkce]{předchozího lemmatu}
 budeme symbolem $\int f$ označovat \emph{kteroukoli} primitivní funkci k $f$ na
 daném intervalu a rozlišující konstantu zanedbáme. Formalizmem nakažený čtenář
 může přemýšlet o symbolu $\int f$ jako značícím \emph{třídu ekvivalence} všech
 funkcí jsoucích primitivních k $f$, kde relací je zde pochopitelně rovnost až
 na konstantu.
\end{remark}

\begin{example}{}{integral-polynomu}
 Platí
 \[
  \int x^{\alpha} \, \mathrm{d}x = 
  \begin{cases}
   \frac{1}{\alpha+1}x^{\alpha+1},& \alpha \neq -1,\\
   \log x,& \alpha = -1.
  \end{cases}
 \]
 Vskutku, máme $\log'x = 1 / x$ a
 \[
  \left( \frac{1}{\alpha+1}x^{\alpha+1} \right)' =
  \frac{\alpha+1}{\alpha+1}x^{\alpha+1-1} = x^{\alpha}
 \]
 pro $\alpha \neq -1$.
\end{example}

\begin{theorem}{Aritmetika primitivních funkcí}{aritmetika-primitivnich-funkci}
 Ať $f,g:(a,b) \to \R$ a $F:(a,b) \to \R$, resp. $G:(a,b) \to \R$, je primitivní
 k $f$, resp. ke $g$, na $(a,b)$. Ať dále $c,d \in \R$. Pak je $cF + dG$
 primitivní k $cf + dg$ na $(a,b)$.
\end{theorem}
\begin{thmproof}
 Zřejmý, z \hyperref[thm:aritmetika-derivaci]{věty o aritmetice derivací}.
 Ponechán jako snadné cvičení.
\end{thmproof}
\begin{exercise}{}{aritmetika-primitivnich-funkci}
 Dokažte \hyperref[thm:aritmetika-primitivnich-funkci]{větu o aritmetice
 primitivních funkcí}.
\end{exercise}

% TODO odkaz
Kompletní charakteristika funkcí, k nímž na jistém intervalu existuje
primitivní, je stále otevřeným problémem matematické analýzy. My si pouze
uvedeme, a v sekci o Riemannově integrálu, dokážeme, že spojité funkce
primitivní funkci mají vždy. Ovšem, tato podmínka je pouze postačující, nikoli
nutná. Existuje mnoho nespojitých funkcí, k nímž primitivní funkce existuje
rovněž.

\begin{theorem}{Spojitost a existence primitivní funkce}{spojitost-a-existence-primitivni-funkce}
 Ať $f:(a,b) \to \R$ je spojitá na $(a,b)$. Pak má na tomto intervalu primitivní
 funkci.
\end{theorem}
\begin{thmproof}
 % TODO odkaz
 Vyplyne ze základní věty kalkulu.
\end{thmproof}

Je ovšem známo mnoho vlastností, které funkce, majíc na jistém intervalu
primitivní funkci, splňovat musejí. Jedna z nich je tzv. \emph{Darbouxova
vlastnost} -- vlastnost funkce zobrazovat interval na interval, která je slabší
než spojitost, ale přesto do značné míry omezující.

\begin{theorem}{Darbouxova}{darbouxova}
 Ať má $f$ na $(a,b)$ primitivní funkci $F$. Pak pro každý interval $I \subseteq
 (a,b)$ platí, že $f(I)$ je rovněž interval.
\end{theorem}
\begin{thmproof}
 Volme interval $I \subseteq (a,b)$ libovolně a nechť $y_1,y_2 \in f(I)$. Je
 třeba ukázat, že je-li $y_1 < z < y_2$, pak $z \in f(I)$.

 Položme $H(x) \coloneqq F(x) - zx$, $x \in (a,b)$. Pak platí $H'(x) = f(x) -
 z$. Nalezněme z definice $f(I)$ čísla $x_1,x_2 \in I$ taková, že $f(x_1) = y_1$
 a $f(x_2) = y_2$. Protože má $H$ na $(a,b)$ konečnou derivaci, je na tomto
 intervalu podle \myref{lemmatu}{lem:vztah-derivace-a-spojitosti} spojitá.
 Jelikož $[x_1,x_2] \subseteq (a,b)$, nabývá podle
 \myref{věty}{thm:extremy-spojite-funkce} funkce $H$ na $[x_1,x_2]$ minima. Ať
 je tomu tak v bodě $x_0$.

 Ukážeme, že $x_0$ není ani jeden z bodů $x_1$, $x_2$. Máme $H'(x_1) = f(x_1) -
 z = y_1 - z < 0$, tj.
 \[
  \lim_{x \to x_1} \frac{H(x) - H(x_1)}{x - x_1} < 0,
 \]
 a tedy existuje okolí bodu $x_1$ na němž platí $H(x) < H(x_1)$. Ergo, $x_1$
 není bodem minima $H$ na $[x_1,x_2]$. Podobně odvodíme, že ani $x_2$ není bodem
 minima $H$. Pak je ale $x_0 \in (x_1,x_2)$ a podle
 \myref{tvrzení}{prop:vztah-derivace-a-extremu} platí $H'(x_0) = 0$, neboli
 $f(x_0) = z$, což dokazuje výrok $z \in f(I)$.
\end{thmproof}

\section{Výpočet primitivních funkcí}
\label{sec:vypocet-primitivnich-funkci}

Na rozdíl od výpočtu derivace funkce, je hledání funkce primitivní téměř vždy
silně nedeterministický úkon. Existují pomůcky k jejich výpočtu, jimž je tato
sekce ovšem věnována, avšak mnohdy musí jeden pracovat v jistém smyslu
\uv{zpětně}, tj. očekávat jistý výsledek a zpytovanou funkci upravit do tvaru,
která co nejblíže odpovídá derivaci právě tohoto očekávaného výsledku. Situaci
nezjednodušuje to, že mnoho (i snadno vyjádřitelných funkcí) nemá primitivní
funkci zapsatelnou použitím pouze funkcí elementárních. Jako příklad vezměme
funkci $\exp(x^2)$, která je na $\R$ spojitá, a tedy má, podle
\myref{věty}{thm:spojitost-a-existence-primitivni-funkce} na $\R$ primitivní
funkci. Avšak, a to bylo jest dokázáno, tato funkce nemá vyjádření pomocí
elementárních funkcí.

Základními nástroji pro výpočet primitivních funkcí jsou \emph{integrace per
partes} a \emph{substituční věty}. Uvedeme a dokážeme všechny.

\begin{theorem}{Integrace per partes}{integrace-per-partes}
 Ať $f,g:(a,b) \to \R$ jsou reálné funkce, $F$ je primitivní k $f$ a $G$
 primitivní ke $g$ na $(a,b)$. Pak platí
 \[
  \int (f \cdot G) = F \cdot G - \int (F \cdot g).
 \]
\end{theorem}
\begin{thmproof}
 Funkce $F$ i $G$ jsou spojité, protože mají podle předpokladu konečné derivace
 na $(a,b)$. Funkce $f$ i $g$ mají z předpokladu primitivní funkce, takže i
 funkce $f \cdot G$ a $F \cdot g$ mají primitivní funkce na $(a,b)$.

 Položme $H \coloneqq \int (F \cdot g)$. Pak z
 \hyperref[thm:aritmetika-derivaci]{aritmetiky derivací}
 \[
  (F \cdot G - H)' = f \cdot G + F \cdot g - F \cdot g = f \cdot G,
 \]
 což dokazuje kýženou rovnost.
\end{thmproof}

\begin{remark}{}{integral-derivace-lomeno-funkce}
 Pro další výpočty bude dobré si povšimnout, že pro libovolnou reálnou funkci
 $f$ platí
 \[
  \int \frac{f'(x)}{f(x)} \, \mathrm{d}x = \log(f(x)).
 \]
 Vskutku, plyne to z přímočarého výpočtu využívajícího
 \hyperref[thm:derivace-slozene-funkce]{větu o derivaci složené funkce}:
 \[
  (\log(f(x)))' = f'(x) \cdot \log'f(x) = f'(x) \cdot \frac{1}{f(x)}.
 \]
\end{remark}

\begin{example}{}{integral-arctan}
 Použitím \myref{věty}{thm:integrace-per-partes} spočteme
 \[
  \int \arctan x \, \mathrm{d}x
 \]
 pro $x \in (-\pi / 2, \pi / 2)$.

 Čtenáři použitím \hyperref[thm:derivace-inverzni-funkce]{věty o derivaci
 inverzní funkce} sobě snadno ověří, že
 \[
  \arctan'x = \frac{1}{1+x^2}.
 \]
 Označme $f(x) \coloneqq 1$ a $G(x) \coloneqq \arctan x$. Pak ve značení
 \myref{věty}{thm:integrace-per-partes} jest $F(x) = x$ a $g(x) = 1 / (1 +
 x^2)$. Počítáme
 \begin{align*}
  \int f(x)G(x) \, \mathrm{d}x &= F(x)G(x) - \int F(x)g(x) \, \mathrm{d}x\\
                               &= x \cdot \arctan x - \int \frac{x}{1+x^2} \,
                               \mathrm{d}x.
 \end{align*}
 Protože $(1+x^2)' = 2x$, poslední integrál snadno spočteme úpravou
 \[
  \int \frac{x}{1+x^2} \, \mathrm{d}x = \frac{1}{2} \int \frac{2x}{1+x^2} \,
  \mathrm{d}x = \frac{1}{2}\log(1+x^2).
 \]
 Celkem tedy máme
 \[
  \int \arctan x \, \mathrm{d}x = x \cdot \arctan x - \frac{1}{2}\log(1+x^2).
 \]
\end{example}

\begin{problem}{}{integral-xex}
 Pro $x \in \R$ spočtěte
 \[
  \int x\exp x \, \mathrm{d}x.
 \]
\end{problem}
\begin{probsol}
 Protože funkce $\exp x$ se nemění derivací či integrací, je dobré si před
 použitím \hyperref[thm:integrace-per-partes]{integrace per partes} rozmyslet,
 zda je druhou funkci jednodušší integrovat či derivovat. Zde platí $x' = 1$,
 tedy budeme $x$ derivovat a $\exp x$ integrovat. Ve značení
 \myref{věty}{thm:integrace-per-partes} máme $f(x) = \exp x$ a $G(x) = x$. Pak
 $F(x) = \exp x$ a $g(x) = 1$. Odtud,
 \begin{align*}
  \int f(x) G(x) \, \mathrm{d}x &= F(x)G(x) - \int F(x)g(x) \, \mathrm{d}x\\
                                &= \exp x \cdot x - \int \exp x \cdot 1 \,
                                \mathrm{d}x = x \exp x - \exp x,
 \end{align*}
 čímž je výpočet završen.
\end{probsol}

\begin{exercise}{}{per-partes}
 Spočtěte
 \[
  \int \frac{\log^2 x}{x^2} \, \mathrm{d}x.
 \]
\end{exercise}

\begin{theorem}{první o substituci}{prvni-o-substituci}
 Ať $a < b,\alpha < \beta \in \R$, $f:(a,b) \to \R$ je reálná funkce, $F$ je
 primitivní k $f$ na $(a,b)$ a $\varphi:(\alpha,\beta) \to (a,b)$ je reálná
 funkce s~konečnou $\varphi'$ na $(\alpha,\beta)$. Potom
 \[
  \int (f \circ \varphi)(t) \cdot \varphi'(t) \, \mathrm{d}t = (F \circ
  \varphi)(t).
 \]
\end{theorem}
\begin{thmproof}
 Podle \hyperref[thm:derivace-slozene-funkce]{věty o derivaci složené funkce}
 platí
 \[
  (F \circ \varphi)'(t) = F'(\varphi(t)) \cdot \varphi'(t) = f(\varphi(t)) \cdot
  \varphi'(t),
 \]
 což dokazuje tvrzení.
\end{thmproof}
\begin{problem}{}{substituce-1}
 Spočtěte
 \[
  \int \sin^{4}t \cos t \, \mathrm{d}t.
 \]
\end{problem}
\begin{probsol}
 Položme $f(x) = x^{4}$ a $\varphi(t) = \sin t$. Pak $F(x) = x^{5} / 5$,
 $\varphi'(t) = \cos t$ a
 \[
  f(\varphi(t)) \cdot \varphi'(t) = \sin^{4}t \cos t.
 \]
 Podle \hyperref[thm:prvni-o-substituci]{první věty o substituci} platí
 \[
  \int \sin^{4}t \cos t \, \mathrm{d}t = \int f(\varphi(t)) \cdot \varphi'(t) \,
  \mathrm{d}t = F(\varphi(t)) = \frac{\sin^{5} t}{5}.
 \]
\end{probsol}
\begin{exercise}{}{substituce-1}
 Spočtěte
 \[
  \int \frac{x}{\sqrt{2 + 5x^2}} \, \mathrm{d}x.
 \]
\end{exercise}
\begin{exercise}{}{substituce-2}
 Spočtěte
 \[
  \int \frac{1}{\sqrt{8 + 6x - 9x^2}} \, \mathrm{d}x.
 \]
 \textbf{Hint}: platí $\arcsin'x = 1 / \sqrt{1 - x^2}$.
\end{exercise}

\begin{theorem}{druhá o substituci}{druha-o-substituci}
 Ať $a < b, \alpha < \beta \in \R$, $f:(a,b) \to \R$, $\varphi:(\alpha,\beta)
 \to (a,b)$ je na a má na $(\alpha,\beta)$ konečnou nenulovou derivaci. Platí-li
 pro každé $t \in (\alpha,\beta)$ rovnost
 \[
  \int f(\varphi(t)) \cdot \varphi'(t) \, \mathrm{d}t = G(t), 
 \]
 pak
 \[
  \int f(x) \, \mathrm{d}x = (G \circ \varphi^{-1})(x)
 \]
 pro $x \in (a,b)$.
\end{theorem}
\begin{thmproof}
 Protože $\varphi'$ má na $(\alpha,\beta)$ primitivní funkci, zobrazuje podle
 \hyperref[thm:darbouxova]{Darbouxovy věty} interval na interval. Z předpokladu
 je $\varphi'$ nenulová, což spolu s předchozí větou implikuje, že buď
 $\varphi'>0$ nebo $\varphi'<0$ na celém $(\alpha,\beta)$. V prvním případě je
 $\varphi$ podle \myref{důsledku}{cor:vztah-derivace-a-monotonie} buď rostoucí,
 nebo klesající, na $(\alpha,\beta)$. V každém případě je prostá, a tedy
 existuje $\varphi^{-1}:(a,b) \to (\alpha,\beta)$.

 Podle \hyperref[thm:derivace-inverzni-funkce]{věty o derivaci inverzní funkce}
 a též \hyperref[thm:derivace-slozene-funkce]{věty o derivaci složené funkce}
 platí pro $x \in (a,b)$
 \begin{align*}
  (G \circ \varphi ^{-1})'(x) &= G'(\varphi ^{-1}(x)) \cdot (\varphi ^{-1})'(x)\\
                              &= f(\varphi(\varphi ^{-1}(x))) \cdot
                              \varphi'(\varphi ^{-1}(x)) \cdot (\varphi
                              ^{-1})'(x)\\
                              &= f(x) \cdot \varphi'(\varphi ^{-1}(x)) \cdot
                              \frac{1}{\varphi'(\varphi ^{-1}(x))} = f(x),
 \end{align*}
 jak bylo jest dokázati.
\end{thmproof}
\begin{example}{}{substituce-3}
 Pro $x \in (-1,1)$ platí
 \[
  \int \sqrt{1-x^2} \, \mathrm{d}x = \frac{1}{2}\arcsin x + \frac{1}{4}\sin(2
  \arcsin x).
 \]

 Vskutku, položme $f(x) = \sqrt{1-x^2}$, $(\alpha,\beta) = (-\pi / 2,\pi / 2)$ a
 $\varphi(t) = \sin t$. Pak $\varphi:(-\pi / 2,\pi / 2) \to (-1,1)$ je na a
 platí $\varphi'(t) = \cos t \neq 0$ pro $t \in (-\pi / 2, \pi / 2)$. Dále
 $\varphi ^{-1}(x) = \arcsin x$ a
 \[
  \int f(\varphi(t)) \cdot \varphi'(t) \, \mathrm{d}t = \int \sqrt{1 - \sin^2
  t} \cdot \cos t \, \mathrm{d}t = \int \cos^2 t \, \mathrm{d}t.
 \]
 Užitím \hyperref[thm:vlastnosti-goniometrickych-funkci]{vlastností
 goniometrických funkcí} se snadno ověří, že
 \[
  \cos^2 t = \frac{1}{2}(1 + \cos 2t).
 \]
 Odtud
 \[
  \int \cos^2 t \, \mathrm{d}t = \int \frac{1}{2} \, \mathrm{d}t + \int
  \frac{1}{2}\cos 2t \, \mathrm{d}t = \frac{1}{2}t + \frac{1}{4}\sin 2t.
 \]
 Čili, ve značení \hyperref[thm:druha-o-substituci]{druhé věty o substituci}
 jest
 \[
  G(t) = \frac{1}{2}t + \frac{1}{4}\sin 2t.
 \]
 Podle téže věty máme 
 \[
  \int \sqrt{1-x^2} \, \mathrm{d}x = \int f(x) \, \mathrm{d}x = G(\varphi
  ^{-1}(x)) = \frac{1}{2}\arcsin x + \frac{1}{4}\sin(2\arcsin x).
 \]
\end{example}

\subsection{Integrace racionálních funkcí}
\label{ssec:integrace-racionalnich-funkci}

Mezi funkce, jež umíme \uv{algoritmicky} integrovat, patří (dle
\myref{příkladu}{exam:integral-polynomu}) jistě funkce polynomiální. Zvlášť
užitečné v závěsu vět o substituci je rovněž umět algoritmicky integrovat funkce
\emph{racionální}, tedy reálné funkce tvaru $R(x) = p(x) / q(x($, kde $p,q$ jsou
polynomy.

Rádi bychom uměli libovolnou racionální funkci rozložit na součet výrazně
jednodušších racionálních funkcí, jejichž integrály umíme spočítat. To lze,
metodou slující \emph{rozklad na parciální zlomky}.

\begin{remark}{}{rozklad-realneho-polynomu}
 V následující větě použijeme sebou nedokázané tvrzení z algebry, že každý
 reálný polynom se rozkládá na součin lineárních a kvadratických činitelů. Tedy,
 je-li $p(x)$ polynom, pak existuje konstanta $c \in \R$ a pro vhodná $k,l \in
 \N$ čísla $a_1,\ldots,a_k,\alpha_1,\ldots,\alpha_l,\beta_1,\ldots,\beta_l \in
 \R$ a přirozená čísla $n_1,\ldots,n_k,m_1,\ldots,m_l \in \N$ splňující
 \[
  p(x) = c(x-a_1)^{n_1}(x-a_2)^{n_2} \cdots (x-a_k)^{n_k}(x^2 + \alpha_1 x +
  \beta_1)^{m_1} \cdots (x^2 + \alpha_l x + \beta_l)^{m_l},
 \]
 kde všechny $x^2 + \alpha_i x + \beta_i$ nemají reálný kořen.
\end{remark}

\begin{theorem}{Rozklad na parciální zlomky}{rozklad-na-parcialni-zlomky}
 Ať $p(x),q(x)$ jsou polynomy a $q$ je rozložený jako v
 \hyperref[rmrk:rozklad-realneho-polynomu]{poznámce výše}. Pak existují
 \emph{jednoznačně určená} čísla
 \begin{align*}
  A_1^{1},A_2^{1},\ldots,A_{n_1}^{1},A_1^{2},A_2^{2},\ldots,A_{n_2}^{2},A_1^{3},A_2^{3},\ldots,A_{1}^{k},\ldots,A_{n_k}^{k}
  & \in \R,\\
  B_1^{1},B_2^{1},\ldots,B_{m_1}^{1},B_1^{2},B_2^{2},\ldots,B_{m_2}^{2},B_1^{3},B_2^{3},\ldots,B_{1}^l,\ldots,B_{m_l}^{l}
  & \in \R,\\
  C_1^{1},C_2^{1},\ldots,C_{m_1}^{1},C_1^{2},C_2^{2},\ldots,C_{m_2}^{2},C_1^{3},C_2^{3},\ldots,C_{1}^l,\ldots,C_{m_l}^{l}
  & \in \R,
 \end{align*}
 taková, že
 \begin{align*}
  \frac{p(x)}{q(x)} &= \frac{A^{1}_1}{x - a_1} + \frac{A_2^{1}}{(x - a_1)^2} +
  \cdots + \frac{A_{n_1}^{1}}{(x-a_1)^{n_1}} + \frac{A_1^{2}}{x-a_2} + \cdots +
  \frac{A_{n_2}^{2}}{(x-a_2)^{n_2}} + \cdots +
  \frac{A_{n_k}^{k}}{(x-a_k)^{n_k}}\\
  &+ \frac{B_1^{1}x + C_1^{1}}{x^2 + \alpha_1x + \beta_1} + \frac{B_2^{1}x +
  C_2^{1}}{(x^2 + \alpha_1x + \beta_1)^2} + \cdots + \frac{B_{m_1}^{1}x +
  C^{1}_{m_1}}{(x^2 + \alpha_1x + \beta_1)^{m_1}} + \frac{B_1^{2}x +
  C_1^{2}}{x^2 + \alpha_2x + \beta_2}\\
  &+ \cdots + \frac{B_{m_2}^{2}x + C_{m_2}^{2}}{(x^2 + \alpha_2 x +
  \beta_2)^{m_2}} + \cdots + \frac{B_{m_l}^{l}x + C_{m_l}^{l}}{(x^2 + \alpha_l x
  + \beta_l)^{m_l}}\\
  &= \sum_{i=1}^k \sum_{j=1}^{n_i} \frac{A_j^{i}}{(x - a_i)^{j}} + \sum_{i=1}^l
  \sum_{j=1}^{m_i} \frac{B_j^{i}x + C_j^{i}}{(x^2 + \alpha_i x + \beta_i)^{j}}.
 \end{align*}
\end{theorem}
\begin{thmproof}
 Triviální, indukcí podle stupně $q$. Ale moc písmenek.
\end{thmproof}

\begin{example}{}{rozklad-na-parcialni-zlomky-1}
 Rozložíme
 \[
  \frac{p(x)}{q(x)} = \frac{3x^3 - 3x^2 + 4x + 10}{(x - 2)^{2}(x^2 + 2x + 2)}
 \]
 na parciální zlomky. Podle \myref{věty}{thm:rozklad-na-parcialni-zlomky}
 existují čísla $A_1,A_2,B,C \in \R$ taková, že
 \begin{equation*}
  \label{eq:parcialni-zlomky}
  \tag{$\heartsuit$}
  \frac{3x^3 - 3x^2 + 4x + 10}{(x-2)^2(x^2 + 2x + 2)} = \frac{A_1}{x - 2} + \frac{A_2}{(x-2)^2}
  + \frac{Bx + C}{x^2 + 2x + 2}.
 \end{equation*}
 Tato čísla nalezneme porovnáním koeficientů obou stran po roznásobení
 jmenovatelem a vyřešením získané soustavy rovnic.

 Roznásobením obou stran~\eqref{eq:parcialni-zlomky} polynomem $q(x)$ dostaneme
 \[
  3x^3 - 3x^2 + 4x + 10 = A_1(x-2)(x^2 + 2x + 2) + A_2(x^2 + 2x + 2) + (Bx +
  C)(x-2)^2.
 \]
 Tato rovnost musí platit pro všechna $x \in \R$, speciálně tedy pro $x = 2$.
 Dosazením dostaneme první rovnici
 \[
  3 \cdot 2^3 - 3 \cdot 2^2 + 4 \cdot 2 + 10 = A_2(2^2 + 2 \cdot 2 + 2),
 \]
 z níž ihned $A_2 = 3$. Nyní dosaďme například $x = 0$. Dosazení dá
 \[
  4 = -4A_1 + 4C.
 \]
 Volbami $x = 1$ a $x = 3$ dostaneme další dvě rovnice
 \begin{align*}
  -1 &= -5A_1 + B + C,\\
  25 &= 17A_1 + 3B + C.
 \end{align*}
 Snadno spočteme, že řešením soustavy těchto tří rovnic je $A_1 = 1, B = 2$ a $C
 = 2$. Platí tedy
 \[
  \frac{3x^3 - 3x^2 + 4x + 10}{(x-2)^2(x^2 + 2x + 2)} = \frac{1}{x - 2} +
  \frac{3}{(x-2)^2} + \frac{2x + 2}{x^2 + 2x + 2}.
 \]
\end{example}

Pochopitelně, rozklad na parciální zlomky nám není mnoho užitečný, neumíme-li
spočítat integrály typu
\[
 \int \frac{A}{(x - a)^{k}} \, \mathrm{d}x \quad \text{a} \quad \int \frac{Bx +
 C}{(x^2 +\alpha x + \beta)^{l}} \, \mathrm{d}x, 
\]
pro $a,\alpha,\beta,A,B,C \in \R$ a $k,l \in \N$. První z těchto integrálů už
jsme v zásadě spočetli v \myref{příkladě}{exam:integral-polynomu}. Druhý je na
výpočet výrazně těžší a hledaná primitivní funkce se nejsnadněji zapisuje
rekurzivně.

Čtenáři snadno ověří, že
\[
 \int \frac{A}{(x - a)^{k}} \, \mathrm{d}x = \begin{cases}
  \frac{A}{(1 - k)(x - a)^{k-1}},& k > 1,\\
  A\log|x - a|, & k = 1.
 \end{cases}
\]
Integrál druhého typu nejprve rozložíme
\[
 \int \frac{Bx + C}{(x^2 + \alpha x + \beta)^{l}} \, \mathrm{d}x =
 \frac{B}{2}\int \frac{2x + \alpha}{(x^2 + \alpha x + \beta)^{l}} \, \mathrm{d}x
 + \left( C - \frac{B\alpha}{2} \right) \int \frac{1}{(x^2 + \alpha x +
 \beta)^{l}} \, \mathrm{d}x.
\]
Protože $(x^2 + \alpha x + \beta)' = 2x + \alpha$, je první z těchto integrálů
rovněž spočten přímočaře. Vskutku,
\[
 \int \frac{2x + \alpha}{(x^2 + \alpha x + \beta)^{l}} \, \mathrm{d}x = \begin{cases}
  \frac{1}{1 - l} \cdot \frac{1}{(x^2 + \alpha x + \beta)^{l-1}}, & l > 1,\\
  \log(x^2 + \alpha x + \beta), & l = 1.
 \end{cases}
\]

Výraz v druhém integrálu nejprve upravíme
\[
 \frac{1}{(x^2 + \alpha x + \beta)^{l}} = \frac{1}{\left( \left( x +
 \frac{\alpha}{2} \right)^2 + \beta - \frac{\alpha^2}{4} \right)^{l}} =
 \frac{1}{\left( \beta - \frac{\alpha^2}{4} \right)^{l}} \cdot \frac{1}{\left( 1
 + \left( \frac{x + \frac{\alpha}{2}}{\sqrt{\beta - \frac{\alpha^2}{4}}}
 \right)^2\right)^{l}}.
\]

Položíme-li nyní $\varphi(x) \coloneqq \frac{x + \alpha / 2}{\sqrt{\beta -
\alpha^2 / 4}}$ a pro strohost zápisu označíme $t = \varphi(x)$, pak z
\hyperref[thm:prvni-o-substituci]{první věty o substituci} platí
\[
 \int \frac{1}{(x^2 + \alpha x + \beta)^{l}} \, \mathrm{d}x = \frac{1}{\left(
 \beta - \frac{\alpha^2}{4} \right)^{l}} \int \frac{1}{(1+t^2)^{l}} \,
 \mathrm{d}t. 
\]
Označme
\[
 I_l \coloneqq \int \frac{1}{(1+t^2)^{l}} \, \mathrm{d}t.
\]

\begin{lemma}{}{vypocet-integralu-Il}
 Platí
 \[
  I_{l+1} = \frac{t}{2l(1 + t^2)^{l}} + \frac{2l - 1}{2l}I_l
 \]
 pro $t \in \R$, přičemž $I_1 = \arctan t$.
\end{lemma}
\begin{lemproof}
 Jistě
 \[
  I_1 = \int \frac{1}{1+t^2} \, \mathrm{d}t = \arctan t.
 \]
 pro $t \in \R$.

 Dále, pro každé $l \in \N$ je $t \mapsto 1 / (1+t^2)^{l}$ spojitá funkce na
 $\R$, tudíž má podle \myref{věty}{thm:spojitost-a-existence-primitivni-funkce}
 primitivní funkci na celém $\R$.
 
 Z \hyperref[thm:integrace-per-partes]{věty o integraci per partes} platí
 \[
  \int 1 \cdot \frac{1}{(1 + t^2)^{l}} \, \mathrm{d}t = \frac{t}{(1 + t^2)^{l}}
  - \int \frac{-2lt^2}{(1+t^2)^{l+1}} \, \mathrm{d}t = \frac{t}{(1 + t^2)^{l}} +
  2l \int \frac{t^2}{(1 + t^2)^{l+1}} \, \mathrm{d}t. 
 \]
 Dále spočteme
 \[
  \int \frac{t^2}{(1 + t^2)^{l+1}} \, \mathrm{d}t = \int \frac{1}{(1 + t^2)^{l}}
  \, \mathrm{d}t - \int \frac{1}{(1 + t^2)^{l+1}} \, \mathrm{d}t = I_l -
  I_{l+1}.  
 \]
 To nám dává rovnost
 \[
  I_l = \int \frac{1}{(1+t^2)^{l}} \, \mathrm{d}t = \frac{t}{(1+t^2)^{l}} + 2l
  \int \frac{t^2}{(1+t^2)^{l+1}} \, \mathrm{d}t = \frac{t}{(1 + t^2)^{l}} +
  2l(I_l - I_{l+1}),
 \]
 z níž již snadnou úpravou plyne tvrzení.
\end{lemproof}

\begin{problem}{}{rozklad-na-parcialni-zlomky-2}
 Spočtěte
 \[
  \int \frac{1}{1 + x^{4}} \, \mathrm{d}x.
 \]
\end{problem}
\begin{probsol}
 Nejprve rozložíme
 \[
  x^{4} + 1 = (x^2 + 1)^2 - 2x^2 = (x^2 + \sqrt{2}x + 1)(x^2 - \sqrt{2}x + 1).
 \]
 Podle \hyperref[thm:rozklad-na-parcialni-zlomky]{věty o rozkladu na parciální
 zlomky} existují $B_1,C_1,B_2,C_2 \in \R$ taková, že
 \[
  \frac{1}{x^{4} + 1} = \frac{B_1x + C_1}{x^2 + \sqrt{2}x + 1} + \frac{B_2x +
  C_2}{x^2 - \sqrt{2}x + 1}.
 \]
 Roznásobením této rovnosti polynomem $x^{4} + 1$ získáme
 \[
  1 = (B_1x + C_1)(x^2 + \sqrt{2}x + 1) + (B_2x + C_2)(x^2 - \sqrt{2}x + 1).
 \]
 Postupným dosazením čtyř libovolných hodnot za $x$ dostaneme soustavu čtyř
 lineárních rovnic s řešením
 \[
  B_1 = \frac{1}{2 \sqrt{2}}, \; C_1 = \frac{1}{2}, \; B_2 = -\frac{1}{2
  \sqrt{2}}, \; C_2 = \frac{1}{2}.
 \]
 Čili,
 \[
  \int \frac{1}{1+x^{4}} \, \mathrm{d}x = \int \frac{\frac{1}{2 \sqrt{2}}x +
  \frac{1}{2}}{x^2 + \sqrt{2}x + 1} \, \mathrm{d}x + \int \frac{-\frac{1}{2
  \sqrt{2}}x + \frac{1}{2}}{x^2 - \sqrt{2}x + 1} \, \mathrm{d}x.
 \]
 První integrál rozložíme jako
 \[
  \int \frac{\frac{1}{2 \sqrt{2}}x + \frac{1}{2}}{x^2 + \sqrt{2}x + 1} \,
  \mathrm{d}x = \int \frac{\frac{1}{4 \sqrt{2}}(2x + \sqrt{2})}{x^2 + \sqrt{2}x
  + 1} \, \mathrm{d}x + \int \frac{\frac{1}{4}}{x^2 + \sqrt{2}x + 1} \,
  \mathrm{d}x.
 \]
 Snadno spočteme, že
 \[
  \int \frac{\frac{1}{4 \sqrt{2}}(2x + \sqrt{2})}{x^2 + \sqrt{2}x + 1} \,
  \mathrm{d}x = \frac{1}{4 \sqrt{2}}\log(x^2 + \sqrt{2}x + 1).
 \]
 Dále,
 \[
  \int \frac{\frac{1}{4}}{x^2 + \sqrt{2}x + 1} \, \mathrm{d}x = \frac{1}{2}\int
  \frac{1}{(\sqrt{2}x + 1)^2 + 1}\, \mathrm{d}x = \frac{1}{2
  \sqrt{2}}\arctan(\sqrt{2}x + 1).
 \]

 Obdobným postupem dostaneme i primitivní funkci
 \[
  \int \frac{-\frac{1}{2 \sqrt{2}}x + \frac{1}{2}}{x^2 - \sqrt{2}x + 1} \,
  \mathrm{d}x = -\log(x^2 - \sqrt{2}x + 1) + \frac{1}{2
  \sqrt{2}}\arctan(\sqrt{2}x - 1).
 \]
 Celkem tedy
 \begin{align*}
  \int \frac{1}{x^{4} + 1} \, \mathrm{d}x &= \frac{1}{4 \sqrt{2}}(\log(x^2 +
  \sqrt{2}x + 1) - \log(x^2 - \sqrt{2}x + 1))\\
  &+ \frac{1}{2 \sqrt{2}}(\arctan(\sqrt{2}x + 1) + \arctan(\sqrt{2}x - 1)).
 \end{align*}
\end{probsol}

\begin{exercise}{}{rozklad-na-parcialni-zlomky-3}
 Spočtěte
 \[
  \int \frac{1}{x(1+x)(1 + x + x^2)} \, \mathrm{d}x. 
 \]
\end{exercise}

\section{Riemannův integrál}
\label{sec:riemannuv-integral}

V této sekci si ukážeme velmi intuitivní způsob, jak počítat \emph{orientovanou}
plochu sevřenou grafem dané funkce a osou $x$. Spojením \emph{orientovaná
	plocha} zde neznačí to, jež bychom přirozeně nazvali \uv{obsahem} útvaru mezi
grafem funkce a osou $x$, nýbrž rozdíl obsahu útvaru takto vzniknuvšího
\textbf{nad} osou $x$ a toho \textbf{pod} osou $x$. Fyzikální využití takového
konceptu jsou výrazně širší.

Je velmi překvapivé, že obsah pod grafem funkce jakkoli souvisí s její
primitivní funkcí. Toto spojení, jeho příčina a intuitivní vysvětlení budou
zveřejněny v sekci o Newtonově integrálu.

Riemannův integrál je způsob výpočtu řečené plochy, který a priori s primitivní
funkcí nesouvisí nijak. Pracuje na přirozeném principu aproximace plochy pod
grafem funkce stále větším množstvím stále užších obdélníku (útvarů, jejichž
obsah jsme schopni triviálně spočíst) jednou seshora, jednou zespoda. Blíží-li
se zvyšováním počtu obdélníků k sobě tyto aproximace, jejich společná limita je
právě plochou pod grafem funkce. K formalizaci tohoto odstavce potřebujeme
několik úvodních definic.

\begin{definition}{Dělení intervalu}{deleni-intervalu}
	Ať $[a,b] \subseteq \R$ je interval. Jeho \emph{dělením} nazveme libovolnou
	konečnou posloupnost $x_0,\ldots,x_n$ pro $n \in \N$, kde $x_0 = a, x_n = b$ a
	platí $x_{i-1} < x_{i}$ pro každé $1 \leq i \leq n$.

	Takovou posloupnost značíme $D = (x_i)_{i=0}^{n}$ a délku nejdelšího intervalu
	tvaru $[x_i,x_{i+1}]$ nazveme \emph{normou} $D$, značenou
	\[
		\|D\| \coloneqq \max_{1 \leq i \leq n} (x_{i} - x_{i-1}).
	\]
	Jsou-li $D',D$ dvě dělení $[a,b]$, pak říkáme, že $D'$ \emph{zjemňuje} $D$
	(značíme $D' \leq D$), patří-li každý dělící bod $D$ rovněž do $D'$.
	Speciálně, $D$ zjemňuje samo sebe.
\end{definition}

\begin{definition}{Horní a dolní součty}{horni-a-dolni-soucty}
	Ať $[a,b] \subseteq \R$ je interval, $f:[a,b] \to \R$ \textbf{omezená} funkce a
	$D$ dělení $[a,b]$. Definujeme hodnotu
	\begin{align*}
		\overline{S}(f,D)  & \coloneqq \sum_{i=0}^n \sup_{[x_{i-1},x_i]} f \cdot (x_i -
		x_{i-1}), \text{ resp.}                                                         \\
		\underline{S}(f,D) & \coloneqq \sum_{i=0}^n \inf_{[x_{i-1},x_i]} f \cdot (x_i
		- x_{i-1}),
	\end{align*}
	a nazýváme ji \emph{horním}, resp. \emph{dolním}, součtem $f$ při dělení $D$.
	Zde výraz $\sup_{[x_{i-1},x_i]} f$, resp. $\inf_{[x_{i-1},x_i]} f$, značí
	supremum, resp. infimum, funkce $f$ na intervalu $[x_{i-1},x_i]$.
\end{definition}

Užitím \hyperref[def:horni-a-dolni-soucty]{horních a dolních součtů} je již
přímočaré definovat \uv{plochu pod funkcí}. Totiž, horní součty jsou právě
aproximací grafu $f$ seshora posloupností obdélníků o šířkách přesně
odpovídajících délkám dělících intervalů (tj. $x_i - x_{i-1}$) a výškách
$\sup_{[x_{i-1},x_i]} f$, a dolní součty zase aproximací grafu $f$ zespoda
obdélníky o stejných šířkách a výškách $\inf_{[x_{i-1},x_i]} f$.

Nyní vezmeme \emph{nejlepší možné} aproximace plochy pod grafem $f$ seshora i
zespoda. Uděláme to tak, že \uv{vyzkoušíme} úplně všechna dělení $[a,b]$ a z
nich vybereme to, jež dává nejmenší horní součty. Formálně, vezmeme infimum
horních součtů přes všechna možná dělení $[a,b]$. Z druhé strany zase supremum
dolních součtů přes všechna dělení $[a,b]$. Je dobré si rozmyslet, že je-li
plocha pod grafem funkce \uv{aproximovatelná} pomocí obdélníků, pak tento postup
skutečně dává kýženou hodnotu orientované plochy.

\begin{definition}{Riemannův integrál}{riemannuv-integral}
	Ať $[a,b] \subseteq \R$ je interval a $f:[a,b] \to \R$ \textbf{omezená} funkce.
	Definujeme hodnotu
	\begin{align*}
		\overline{\int_a^{b}} f    & \coloneqq \inf \{\overline{S}(f,D) \mid D \text{
		dělení } [a,b]\}, \text{ resp.},                                               \\
		\underline{\int_{a}^{b}} f & \coloneqq \sup \{\underline{S}(f,D) \mid D \text{
			dělení } [a,b]\},
	\end{align*}
	a nazýváme ji \emph{horním Riemannovým integrálem}, resp. \emph{dolním
		Riemannovým integrálem}, funkce $f$ na $[a,b]$.

	Platí-li
	\[
		\overline{\int_{a}^{b}} f = \underline{\int_{a}^{b}} f,
	\]
	říkáme, že funkce $f$ \emph{má Riemannův integrál} na $[a,b]$, což symbolicky
	značíme $f \in \mathcal{R}(a,b)$. Společnou hodnotu obou integrálů poté
	zkrátka $\int_{a}^{b} f$ a říkáme jí \emph{Riemannův integrál} funkce $f$ na
	$[a,b]$.
\end{definition}

\begin{warning}{}{spojita-funkce-a-uzavreny-interval}
	Riemannův integrál je definován pouze pro \textbf{omezenou} funkci na
	\textbf{uzavřeném} intervalu.

	Důvodem předpokladu omezenosti je potřeba dobré definovanosti horních a dolních
	součtů. U neomezených funkcí mohou být hodnoty $\sup_{[x_{i-1},x_i]} f$ a
	$\inf_{[x_{i-1},x_i]} f$ totiž nekonečné.

	Předpoklad uzavřenosti intervalu existuje v zásadě ze stejného důvodu. Při
	definici dělení otevřeného intervalu a následně součtů dané funkce při tomto
	dělení by byl jeden nucen studovat limitní chování takové funkce v jeho
	krajních bodech.
\end{warning}

Následuje několik tvrzení o vlastnostech dělení uzavřených intervalů a
Riemannova integrálu, která později pomohou sloučit tento s konceptem primitivní
funkce.

\begin{lemma}{Vlastnosti dělení}{vlastnosti-deleni}
 Nechť $[a,b] \subseteq \R$ je interval a $f:[a,b] \to \R$ omezená funkce.
 \begin{enumerate}[label=(\alph*),topsep=0pt]
	\item Ať $D,D'$ jsou dělení $[a,b]$ a $D' \leq D$. Potom
	\[
	 \underline{S}(f,D) \leq \underline{S}(f,D') \leq \overline{S}(f,D') \leq
	 \overline{S}(f,D).
	\]
 	\item Ať $D_1,D_2$ jsou libovolná dělení $[a,b]$. Pak
 	\[
   \underline{S}(f,D_1) \leq \overline{S}(f,D_2)
 	\]
 	\item Platí
 	\[
	 \underline{\int_{a}^{b}} f \leq \overline{\int_{a}^{b}} f.
 	\]
 \end{enumerate}
\end{lemma}
\begin{lemproof}
 Bod (a) dokážeme v případě, kdy $D'$ má oproti $D$ přesně jeden dělící bod
 navíc. Zbytek důkazu je triviální využití indukce. Poznamenejme, že prostřední
 nerovnost (tj. $\underline{S}(f,D') \leq \overline{S}(f,D')$) plyne přímo z
 \hyperref[def:horni-a-dolni-soucty]{definice}. Označme $D = (x_i)_{i=0}^{n}$ a
 ať $D'$ má navíc dělící bod ${y \in (x_{i-1},x_i)}$ pro jisté $i \in
 \{1,\ldots,n\}$. Potom
 \begin{align*}
  \underline{S}(f,D') - \underline{S}(f,D) &= \inf_{[x_{i-1},y]} f \cdot (y -
  x_{i-1}) + \inf_{[y,x_i]} f \cdot (x_i - y) - \inf_{[x_{i-1},x_i]} f \cdot
  (x_i - x_{i-1}),
 \end{align*}
 neboť $\underline{S}(f,D)$ sdílí s $\underline{S}(f,D')$ všechny sčítance typu
 $\inf_{[x_{j-1},x_j]} f \cdot (x_j - x_{j-1})$ pro $j \neq i$. Jistě platí
 \[
  \inf_{[x_{i-1},x_i]} f \leq \inf_{[x_{i-1},y]} f \quad \text{a} \quad
  \inf_{[x_{i-1},x_i]} f \leq \inf_{[y,x_i]} f,
 \]
 čili lze odhadnout
 \begin{align*}
  \underline{S}(f,D') - \underline{S}(f,D) &= \inf_{[x_{i-1},y]} f \cdot (y -
  x_{i-1}) + \inf_{[y,x_i]} f \cdot (x_i - y) - \inf_{[x_{i-1},x_i]} f \cdot
  (x_i - x_{i-1})\\
  																				 & \geq \inf_{[x_{i-1},x_i]} f \cdot
  																				 (y - x_{i-1} + x_i - y - x_i +
  																				 x_{i-1}) = 0,
 \end{align*}
 což po přeuspořádání dá ihned
 \[
  \underline{S}(f,D') \geq \underline{S}(f,D).
 \]
 Třetí nerovnost, $\overline{S}(f,D') \leq \overline{S}(f,D)$, se dokáže
 obdobně.

 Pro důkaz (b) uvažme dělení $D$ zjemňující jak $D_1$, tak $D_2$. Potom podle
 bodu (a) platí
 \[
  \underline{S}(f,D_1) \leq \underline{S}(f,D) \leq \overline{S}(f,D) \leq
  \overline{S}(f,D_2),
 \]
 což je přímo dokazovaná nerovnost.

 Bod (c) plyne okamžitě z (b) a \hyperref[def:riemannuv-integral]{definice
 Riemannova integrálu}.
\end{lemproof}

Znění \hyperref[lem:vlastnosti-deleni]{předchozího lemmatu} je vyjádření
geometricky intuitivního faktu, že libovolný odhad plochy pod funkcí
posloupností obdélníků seshora je vždy vyšší než odhad téhož posloupností
obdélníku zespoda. Jeden přímočarý důsledek je následující.

\begin{corollary}{}{}
 Ať $[a,b] \subseteq \R$ je interval a $f:[a,b] \to \R$ omezená funkce. Položme
 $m \coloneqq \inf_{[a,b]}f$ a $M \coloneqq \sup_{[a,b]}f$. Pak pro libovolné
 $D$ dělení $[a,b]$ platí
 \[
  m(b-a) \overset{(1)}{ \leq } \underline{S}(f,D) \overset{(2)}{ \leq }
  \underline{\int_{a}^{b}} f \overset{(3)}{ \leq }  \overline{\int_{a}^{b}} f
  \overset{(4)}{ \leq } \overline{S}(f,D) \overset{(5)}{ \leq } M(b-a).
 \]
\end{corollary}
\begin{corproof}
 Nerovnosti (2), (3) a (4) plynou ihned z definic.

 Dále pak (1), resp. (5), plyne okamžitě z
 \hyperref[lem:vlastnosti-deleni]{předchozího lemmatu} dosazením za $D_1$, resp.
 za $D_2$, triviální dělení $D' \coloneqq (x_0,x_1) = (a,b)$. Platí totiž
 \[
  \underline{S}(f,D') = \inf_{[a,b]}f \cdot (b-a) = m(b-a) \quad \text{a} \quad
  \overline{S}(f,D') = \sup_{[a,b]}f \cdot (b-a) = M(b-a).
 \]
\end{corproof}

Nyní dokážeme technické tvrzení, které nám posléze umožní Riemannův integrál z
definice počítat. Je vyjádřením idey, že dolní, resp. horní, Riemannův integrál
lze libovolně dobře aproximovat dolním, resp. horním, součtem přes vhodné dělení
příslušného intervalu. Vzhledem k definici dolního, resp. horního, Riemannova
integrálu jako suprema dolních, resp. infima horních, součtů snad není tento
fakt příliš překvapivý. Dlužno podotknout, že platnost podobných tvrzení se
obyčejně stanovuje důkazovou technikou \emph{mávnutí ruky}, ne však při
bakalářském studiu.

\begin{theorem}{Aproximace Riemannova integrálu}{aproximace-riemannova-integralu}
 Ať $[a,b] \subseteq \R$ je interval a $f:[a,b] \to \R$ omezená funkce. Pak pro
 $\varepsilon>0$ libovolně malé existuje $\delta>0$ takové, že kdykoli je $D =
 (x_i)_{i=0}^{n}$ dělení $[a,b]$ s normou $\|D\|<\delta$, pak
 \begin{align}
  \overline{\int_{a}^{b}} f & \leq \overline{S}(f,D) \leq
  \overline{\int_{a}^{b}} f + \varepsilon, \label{eq:riemann-approx-low}\\
  \underline{\int_{a}^{b}} f & \geq \underline{S}(f,D) \geq
  \underline{\int_{a}^{b}} f - \varepsilon. \label{eq:riemann-approx-high}
 \end{align}
\end{theorem}
\begin{thmproof}
 Poznamenejme, že omezenost $f$ na $[a,b]$ dává existenci čísla $K>0$ takového,
 že $|f(x)| < K$ pro $x \in [a,b]$. Speciálně tedy $\sup_{[a,b]} f \leq K$ a
 $\inf_{[a,b]} f \geq -K$.

 Dokážeme například pár nerovností v \eqref{eq:riemann-approx-low}. Nerovnosti v
 \eqref{eq:riemann-approx-high} se dokazují analogicky. Všimněme si nejprve, že
 první nerovnost v $\eqref{eq:riemann-approx-low}$ plyne ihned z
 \hyperref[def:riemannuv-integral]{definice Riemannova integrálu}. Důkaz té
 druhé je poněkud komplikovaný. Totiž, připomeňme zmíněnou definici:
 \[
  \overline{\int_{a}^{b}} f = \inf \{\overline{S}(f,D) \mid D \text{ dělení }
  [a,b]\}.
 \]
 Z definice infima existuje nějaký prvek množiny napravo, tedy nějaké dělení
 $D_0$ intervalu $[a,b]$, takové, že
 \[
  \overline{S}(f,D_0) < \overline{\int_{a}^{b}} f + \varepsilon,
 \]
 protože $\overline{\int_{a}^{b}} f$ je \emph{nejmenší} dolní závorou množiny
 horních součtů přes všechna možná dělení intervalu $[a,b]$. Proč toto není
 konec důkazu? Protože o normě $D_0$ nemůžeme říct vůbec nic. My nepotřebujeme
 nalézt dělení, které dobře (závisle na $\varepsilon$) aproximuje
 $\overline{\int_{a}^{b}} f$, ale ukázat, že \emph{úplně každé} dělení s
 dostatečně malou normou jej aproximuje dobře. Ukážeme, že (až na konstantu)
 stačí, aby tato norma byla maximálně tak velká jako délka nejmenšího z dělících
 intervalů v~$D_0$.

 Označme tedy $D_0 = (x_i)_{i=0}^{n}$, nechť $\mu(D_0) \coloneqq \min \{x_i -
 x_{i-1} \mid 1 \leq i \leq n\}$ a položme $\delta_1 \coloneqq \min
 \{\mu(D_0),\varepsilon\}$. Ať je nyní $D$ libovolné dělení $[a,b]$ s
 $\|D\|<\delta_1$. Definujme nové dělení $P$ intervalu $[a,b]$ sestávající ze
 všech dělících bodů $D$ i $D_0$. Neformálně můžeme psát $P = D \cup D_0$. Pro
 pohodlí označme písmenem $\mathcal{D}$ množinu všech intervalů v dělení $D$ a
 $\mathcal{P}$ množinu všech intervalů v dělení $P$. Z~definice
 \begin{align*}
  \overline{S}(f,D) &= \sum_{I \in \mathcal{D}} \sup_I f \cdot \ell(I),\\
  \overline{S}(f,P) &= \sum_{I \in \mathcal{P}} \sup_I f \cdot \ell(I),
 \end{align*}
 kde, připomeňme, $\ell(I)$ značí délku intervalu $I$.

 Vezměme libovolný $I \in \mathcal{D}$. Mohou nastat dva případy:
 \begin{enumerate}
  \item $I$ leží rovněž v $\mathcal{P}$. Potom je sčítanec $\sup_I f \cdot
   \ell(I)$ přítomen jak v součtu $\overline{S}(f,D)$, tak
   v~$\overline{S}(f,P)$.
  \item $I$ neleží v $\mathcal{P}$. Pak vnitřek tohoto intervalu protíná nějaký
   dělící interval v $D_0$. To jest, existuje index $j \in \{0,\ldots,n\}$
   takový, že $x_j$ leží ve vnitřku (není krajním bodem) intervalu $I$. Jelikož
   bylo však $D$ zvoleno tak, že $\|D\| < \mu(D_0)$ -- slovy, nejdelší interval
   v $D$ je kratší než nejkratší interval v $D_0$ -- existuje takový index $j$
   \emph{právě jeden}.
 \end{enumerate}
 Nadále předpokládejme, že nastal případ (2) a označme $I = [\alpha,\beta]$. Pak
 $x_j \in (\alpha,\beta)$. V $\overline{S}(f,D)$ se nachází sčítanec $\sup_I f
 \cdot (\beta - \alpha)$, zatímco v $\overline{S}(f,P)$ na místě tohoto sčítance
 máme
 \[
  \sup_{[\alpha,x_j]} f \cdot (x_j - \alpha) + \sup_{[x_j,\beta]} f \cdot
  (\beta - x_j).
 \]
 Jejich rozdíl odhadneme
 \begin{align*}
  |\sup_{[\alpha,\beta]} f \cdot (\beta - \alpha) - (\sup_{[\alpha,x_j]} f \cdot
  (x_j - \alpha) + \sup_{[x_j,\beta]} f \cdot (\beta - x_j))| \\
  \leq K(\beta-\alpha) + K(x_j - \alpha) + K(\beta - x_j) = 2K(\beta-\alpha)
  \leq 2K \cdot \|D\|.
 \end{align*}
 Protože $D_0$ má $n+1$ dělících bodů, může případ (2) nastat pro maximálně $n$
 intervalů z $\mathcal{D}$. Celkem tedy
 \[
  |\overline{S}(f,D) - \overline{S}(f,P)| \leq 2K \cdot n \cdot \|D\|
 \]
 a speciálně $\overline{S}(f,D) \leq \overline{S}(f,P) + 2Kn \|D\|$. Odtud již
 přímočaře spočteme
 \[
  \overline{\int_{a}^{b}} f \leq \overline{S}(f,D) \leq \overline{S}(f,P) + 2Kn
  \|D\| = \overline{S}(f,D_0) + 2Kn\varepsilon < \overline{\int_{a}^{b}} f +
  (2Kn + 1)\varepsilon,
 \]
 kde předposlední nerovnost plyne z toho, že $P \leq D_0$ a z volby $\|D\| <
 \varepsilon$ a poslední nerovnost z volby $D_0$ na začátku důkazu. Protože $2Kn
 + 1$ je konstanta nezávislá na $\varepsilon$, je tímto důkaz
 \eqref{eq:riemann-approx-low} dokončen.

 Podobně bychom nalezli $\delta_2>0$, pro něž platí
 \eqref{eq:riemann-approx-high}. Volba $\delta \coloneqq
 \min(\delta_1,\delta_2)$ zakončuje důkaz.
\end{thmproof}

Nyní si zformulujeme dva důsledky právě dokázané věty, které nám umožní hodnotu
Riemannova integrálu spočíst jako limitu posloupnosti horních (či dolních)
součtů přes dělení se stále menší normou.

\begin{corollary}{}{riemann-vypocet}
 Ať $[a,b] \subseteq \R$ je interval, $f:[a,b] \to \R$ omezená funkce a
 $\{D_n\}_{n=0}^{\infty}$ posloupnost dělení $[a,b]$ splňující $\lim_{n \to
 \infty} \|D_n\| = 0$. Potom
 \[
  \overline{\int_{a}^{b}} f = \lim_{n \to \infty} \overline{S}(f,D_n) \quad
  \text{a} \quad \underline{\int_{a}^{b}} f = \lim_{n \to \infty}
  \underline{S}(f,D_n).
 \]
\end{corollary}
\begin{corproof}
 Ať je dáno $\varepsilon>0$. Podle
 \myref{věty}{thm:aproximace-riemannova-integralu} k němu existuje $\delta>0$,
 že pro každé dělení $D$ s~$\|D\|<\delta$ platí
 \[
  \overline{S}(f,D) < \overline{\int_{a}^{b}} f + \varepsilon.
 \]
 Nalezněme index $n_0 \in \N$ takový, že pro $n \geq n_0$ je $\|D_n\| < \delta$.
 Pak ale rovněž pro každé $n \geq n_0$ máme odhady
 \[
  \overline{\int_{a}^{b}} f \leq \overline{S}(f,D_n) < \overline{\int_{a}^{b}} f
  + \varepsilon,
 \]
 což dokazuje, že
 \[
  \lim_{n \to \infty} \overline{S}(f,D_n) = \overline{\int_{a}^{b}} f.
 \]
 Druhá rovnost se dokáže obdobně.
\end{corproof}

\begin{corollary}{}{riemann-existence}
 Ať $[a,b] \subseteq \R$ je interval, $f:[a,b] \to \R$ omezená funkce a
 $\{D_n\}_{n=0}^{\infty}$ posloupnost dělení taková, že
 \[
  \lim_{n \to \infty} \overline{S}(f,D_n) = \lim_{n \to \infty}
  \underline{S}(f,D_n).
 \]
 Potom $f \in \mathcal{R}(a,b)$ a platí
 \[
  \int_{a}^{b} f = \lim_{n \to \infty} \overline{S}(f,D_n) = \lim_{n \to \infty}
  \underline{S}(f,D_n).
 \]
\end{corollary}
\begin{corproof}
 Pro každé $n \in \N$ platí
 \[
  \underline{S}(f,D_n) \leq \underline{\int_{a}^{b}} f \leq
  \overline{\int_{a}^{b}} f \leq \overline{S}(f,D_n),
 \]
 což podle \myref{lemmatu}{lem:o-dvou-straznicich} znamená, že
 \[
  \underline{\int_{a}^{b}} f = \overline{\int_{a}^{b}} f = \lim_{n \to \infty}
  \overline{S}(f,D_n) = \lim_{n \to \infty} \underline{S}(f,D_n),
 \]
 jak jsme chtěli.
\end{corproof}

\begin{example}{}{}
 Poslední dva důsledky dávají vcelku přímočarý, ač výpočetně obvykle nesnadný
 postup výpočtu Riemannova integrálu. Vypadá následovně.
 \begin{enumerate}
  \item Nalezneme posloupnost dělení $[a,b]$ s klesající normou. Volba
   \uv{vhodné} posloupnosti dělení závisí velmi na zadané funkci. Možností je
   nespočetně.
  \item Pomocí \myref{důsledku}{cor:riemann-existence} dokážeme, že Riemannův
   integrál z $f$ na $[a,b]$ existuje.
  \item \myref{Důsledek}{cor:riemann-vypocet} pak říká, že jeho hodnota je rovna
   limitě horních, nebo dolních, součtů přes zvolenou posloupnost dělení.
 \end{enumerate}
 Body (2) a (3) všedně splývají v jeden, neboť bod (2) káže ukázat, že
 \[
  \lim_{n \to \infty} \overline{S}(f,D_n) = \lim_{n \to \infty}
  \underline{S}(f,D_n),
 \]
 což obvykle zahrnuje explicitní výpočet obou limit jsoucí obsahem bodu (3).

 Navržený postup vykreslíme výpočtem integrálu
 \[
  \int_{0}^{1} x^2 \, \mathrm{d}x.
 \]
 Volme například posloupnost rovnoměrných dělení intervalu $[0,1]$ danou
 předpisem
 \[
  D_n \coloneqq \left\{\frac{i}{n}\right\}_{i=0}^n.
 \]
 Protože $f(x) = x^2$ je rostoucí spojitá funkce na $[0,1]$ je její infimum na
 každém dělícím intervalu přesně její hodnota v levém krajním bodě a její
 supremum zase ta v pravém. Délka každého intervalu v $D_n$ je přesně $1 / n$,
 takže pro $n \in \N$ dostáváme
 \begin{align*}
  \underline{S}(f,D_n) &= \sum_{i=1}^{n} f \left( \frac{i-1}{n} \right) \cdot
  \frac{1}{n} = \sum_{i=1}^{n} \left( \frac{i-1}{n} \right)^2 \cdot
  \frac{1}{n};\\
  \overline{S}(f,D_n) &= \sum_{i=1}^{n} f \left( \frac{i}{n} \right)^2 \cdot
  \frac{1}{n} = \sum_{i=1}^{n} \left( \frac{i}{n} \right)^2 \cdot \frac{1}{n}.
 \end{align*}
 Upravíme
 \[
  \underline{S}(f,D_n) = \sum_{i = 1}^{n} \left( \frac{i-1}{n} \right)^2 \cdot
  \frac{1}{n} = \frac{1}{n^3} \cdot \sum_{i=0}^{n-1} i^2.
 \]
 Snadná indukce dá
 \[
  \sum_{i=0}^{n-1} i^2 = n(n-1)(2n-1)
 \]
 pro každé $n \geq 1$. Čili,
 \[
  \lim_{n \to \infty} \underline{S}(f,D_n) = \lim_{n \to \infty}
  \frac{n(n-1)(2n-1)}{n^3} = \frac{1}{3}.
 \]
 Podobně spočteme, že rovněž
 \[
  \lim_{n \to \infty} \overline{S}(f,D_n) = \lim_{n \to \infty}
  \frac{n(n+1)(2n+1)}{6n^3} = \frac{1}{3}.
 \]
 
 Podle \myref{důsledku}{cor:riemann-existence} existuje Riemannův integrál z
 funkce $f$ na $[0,1]$. \myref{Důsledek}{cor:riemann-vypocet} pak dává, že
 \[
  \int_{0}^{1} x^2 \, \mathrm{d}x = \lim_{n \to \infty} \overline{S}(f,D_n) =
  \lim_{n \to \infty} \underline{S}(f,D_n) = \frac{1}{3}.
 \]
\end{example}

\begin{exercise}{}{}
 Spočtěte následující Riemannovy integrály:
 \begin{enumerate}
  \item z funkce $f(x) = x^3$ na intervalu $[0,1]$;
  \item z funkce $f(x) = 7x + 2$ na intervalu $[-2,3]$.
 \end{enumerate}
\end{exercise}

\subsection{Integrovatelné funkce}
\label{ssec:integrovatelne-funkce}

Chvíli se budeme zabývat nalezením jistých obecných kritérií pro funkce, jejichž
splnění již zaručuje existenci jejich Riemannova integrálu. Snad překvapivě,
kompletní charakterizace integrovatelných funkcí je otevřený problém a žádný
jejich přímočarý popis není znám. My si ukážeme, že dvě důležité třídy funkcí --
spojité a monotónní -- integrovatelné vždy jsou.

Za tímto účelem vyžadujeme jedno technické lemma.

\begin{lemma}{Kritérium existence integrálu}{kriterium-existence-integralu}
 Ať $[a,b] \subseteq \R$ je interval a $f:[a,b] \to \R$ je omezená funkce. Pak
 jsou následující tvrzení ekvivalentní.
 \begin{enumerate}
  \item $f \in \mathcal{R}(a,b)$;
  \item Pro každé $\varepsilon>0$ existuje $D$, dělení $[a,b]$, takové, že
  \[
   \overline{S}(f,D) - \underline{S}(f,D) < \varepsilon.
  \]
 \end{enumerate}
\end{lemma}
\begin{lemproof}
 Začneme implikací $(1) \Rightarrow (2)$. Ať tedy $f \in \mathcal{R}(a,b)$ a
 $\varepsilon>0$ je dáno. Z definic suprema a infima nalezneme dělení $D_1$ a
 $D_2$ taková, že
 \[
  \overline{S}(f,D_1) < \overline{\int_{a}^{b}} f + \varepsilon = \int_{a}^{b} f
  + \varepsilon \quad \text{a} \quad \underline{S}(f,D_2) >
  \underline{\int_{a}^{b}} f - \varepsilon = \int_{a}^{b} f - \varepsilon.
 \]
 Nechť $D$ je dělení zjemňující $D_1$ i $D_2$. Potom podle
 \myref{lemmatu}{lem:vlastnosti-deleni} platí
 \begin{align*}
  \overline{S}(f,D) - \underline{S}(f,D) &\leq \overline{S}(f,D_1) -
  \underline{S}(f,D_2)\\
                                         & \leq \int_{a}^{b} f + \varepsilon -
                                         \int_{a}^{b} f + \varepsilon =
                                         2\varepsilon,
 \end{align*}
 jak jsme chtěli.

 Pro důkaz $(2) \Rightarrow (1)$ mějme rovněž dáno $\varepsilon>0$ a nalezněme
 dělení $D$ splňující
 \[
  \overline{S}(f,D) - \underline{S}(f,D) < \varepsilon.
 \]
 Potom ale
 \[
  0 \leq \overline{\int_{a}^{b}} f - \underline{\int_{a}^{b}} f \leq
  \overline{S}(f,D) - \underline{S}(f,D) < \varepsilon,
 \]
 čili
 \[
  \overline{\int_{a}^{b}} f = \underline{\int_{a}^{b}} f,
 \]
 což znamená, že $f \in \mathcal{R}(a,b)$.
\end{lemproof}

Užitím \hyperref[lem:kriterium-existence-integralu]{tohoto lemmatu} můžeme ihned
dokázat, že monotónní funkce jsou vždy integrovatelné.

\begin{proposition}{Integrovatelnost monotónní funkce}{integrovatelnost-monotonni-funkce}
 Ať $f$ je monotónní funkce na intervalu $[a,b]$. Pak $f \in \mathcal{R}(a,b)$.
\end{proposition}
\begin{propproof}
 Předpokládejme například, že $f$ je neklesající. Zbylé tři případy se dokazují
 v zásadě stejně. Funkce $f$ je v tomto případě jistě omezená, neboť $f(a) \leq
 f(x) \leq f(b)$ pro každé $x \in [a,b]$. Nyní využijeme
 \myref{lemma}{lem:kriterium-existence-integralu}. Ať je tedy $\varepsilon>0$
 dáno. Nalezneme $n \in \N$ dostatečně velké, aby
 \[
  \frac{(b-a)(f(b) - f(a))}{n} < \varepsilon.
 \]
 Zde $(b-a)(f(b) - f(a))$ je obsah obdélníku obklopujícího celou plochu pod $f$
 na $[a,b]$, neboť $f$ je neklesající. Volba řečeného $n \in \N$ představuje
 jeho rozdělení na dostatečně mnoho proužků, aby každý jeden proužek měl obsah
 nižší než dané číslo $\varepsilon$. Volme dělení $D = (x_i)_{i=0}^{n}$ tak, aby
 dolní rohy těchto proužků byly přesně dělícími body; symbolicky
 \[
  x_i \coloneqq a + \frac{b-a}{n}i, \; \text{pro } i \in \{0,\ldots,n\}.
 \]
 Nahlédněme, že z monotonie $f$ plynou rovnosti
 \[
  \sup_{[x_{i-1},x_i]} f = f(x_i) \quad \text{a} \quad \inf_{[x_{i-1},x_i]} f =
  f(x_{i-1})
 \]
 pro každé $i \in \{1,\ldots,n\}$. Pročež můžeme odhadnout
 \begin{align*}
  \overline{S}(f,D) - \underline{S}(f,D) &= \sum_{i = 1}^{n}
  \sup_{[x_{i-1},x_i]}f \cdot (x_i - x_{i-1}) - \sum_{i = 1}^{n}
  \inf_{[x_{i-1},x_i]}f \cdot (x_i - x_{i-1})\\
                                         &= \sum_{i = 1}^{n}
                                         (\sup_{[x_{i-1},x_i]} f-
                                         \inf_{[x_{i-1},x_i]}f)(x_i - x_{i-1})\\
                                         &= \sum_{i = 1}^{n} (f(x_i) -
                                         f(x_{i-1}))(x_i - x_{i-1}) = \sum_{i =
                                         1}^{n} (f(x_i) - f(x_{i-1}))
                                         \frac{b-a}{n}\\
                                         &= (f(b) - f(a))
                                         \frac{b-a}{n}<\varepsilon,
 \end{align*}
 což podle \myref{lemmatu}{lem:kriterium-existence-integralu} znamená, že
 $f \in \mathcal{R}(a,b)$.
\end{propproof}

Abychom rovněž dokázali, že spojité funkce jsou integrovatelné, odbočíme na
krátkou chvíli k~pojmu \emph{stejnosměrně spojité} funkce. Jde o pojem důležitý
primárně při studiu funkcí daných nekonečným součtem (jakými jsou například
funkce elementární) a je ten vlastně zesílením vlastnosti spojitosti o
požadavek, že daná funkce \uv{roste všude stejně rychle}. Z toho důvodu nedává
pojem stejnosměrné spojitosti \emph{v bodě} žádný smysl, nýbrž je potřeba určit
interval, na němž má být funkce stejnosměrně spojitá. Uvedeme nyní definici a
srovnáme ji s již známou limitní definicí spojitosti.

\begin{definition}{Stejnoměrná spojitost}{stejnomerna-spojitost}
 Ať $I \subseteq \R$ je interval. Řekneme, že funkce $f:I \to \R$ je
 \emph{stejnoměrně spojitá} na $I$, pokud
 \begin{equation}
  \label{eq:stejnomerna-spojitost}
  \forall \varepsilon>0 \; \exists \delta>0 \; \forall x,y \in I :|x-y|<\delta
  \Rightarrow |f(x) - f(y)|<\varepsilon.
 \end{equation}
\end{definition}

Onen zásadní rozdíl mezi spojitostí a stejnoměrnou spojitostí funkce dlí ve
faktu, že spojitost je vlastnost \emph{lokální}, ovšem stejnoměrná spojitost je
vlastnost \emph{globální}. Abychom toto nahlédli, přepíšeme definici spojitosti
funkce na intervalu, aby byla co nejblíže definici spojitosti stejnoměrné.

Funkce $f:I \to \R$ je spojitá na $I$, když je spojitá v každém jeho bodě (zde
se projevuje ona \uv{lokálnost}), tedy, když pro každé $y \in I$ platí, že
\[
 \lim_{x \to y} f(x) = f(y).
\]

Přepsáno přes \hyperref[def:oboustranna-limita-funkce]{definici limity}, tento
výrok zní
\[
 \forall y \in I \; \forall \varepsilon>0 \; \exists \delta>0 \; \forall x \in
 R(y,\delta) : f(x) \in B(f(y),\varepsilon).
\]
Můžeme výrok dále přepsat v absolutních hodnotách s užitím implikace.
\[
 \forall y \in I \; \forall \varepsilon>0 \; \exists \delta>0 \; \forall x \in I
 : |x - y| < \delta \Rightarrow |f(x) - f(y)| < \varepsilon.
\]
Srovnejte výrok výše s \eqref{eq:stejnomerna-spojitost}. Liší se pouze pozicí
termu $ \forall y \in I$. Tento zdánlivě bezvýznamný rozdíl je však zásadní.
Značí totiž, že čísla $\varepsilon$ i $\delta$ v definici spojitosti závisejí na
volbě bodu $y \in I$, kolem nějž spojitost funkce \uv{měříme}, zatímco v
definici \emph{stejnoměrné} spojitosti nesmí rozdíl hodnot funkce $f$ v
libovolných dvou bodech vzdálených maximálně $\delta$ překročit $\varepsilon$.
Druhá vlastnost je tudíž závislá na volbě vzdálenosti $\varepsilon$ a naopak
zcela oddělena od volby konkrétního páru bodů v intervalu $I$.

\begin{example}{}{stejnomerna-spojitost}
 Stejnoměrná spojitost je zřejmě (aspoň \emph{intuitivně} zřejmě) silnější
 vlastnost než pouhá spojitost. Rigorózní důkaz následuje. Například funkce
 $f(x) = 1 / x$ je \emph{spojitá} na intervalu $(0,1)$, ale není tamže
 \emph{stejnoměrně spojitá}.

 Vskutku, platí, že
 \[
  \lim_{x \to y} \frac{1}{x} = \frac{1}{y}
 \]
 pro každé $y \in (0,1)$. Naopak, volme třeba $\varepsilon \coloneqq 1$. Pak
 žádná volba $\delta>0$ nezaručí, že $|1 / x - 1 / y| < 1$, kdykoli
 $|x-y|<\delta$. Problém, je totiž v tom, že čím blíže jsme bodu $0$, tím
 rychleji se od sebe funkční hodnoty v různých bodech vzdalují. Formálně, ať je
 dána libovolná vzdálenost $\delta>0$. Nalezneme $n \in \N$ takové, že $1 / n <
 \delta$. Potom pro $x = 1 / n$ a $y = 1 / (n + 1)$ platí
 \[
  |x - y| = \left| \frac{1}{n} - \frac{1}{n+1} \right| = \left| \frac{1}{n(n+1)}
  \right| < \delta,
 \]
 ale přesto
 \[
  \left| \frac{1}{x} - \frac{1}{y} \right| = \left| n - (n + 1) \right| = 1
  \not< \varepsilon.
 \]
 Tedy, $1 / x$ není stejnoměrně spojitá na $(0,1)$.
\end{example}

Stejnoměrně spojité funkce jsou vždy spojité, ale, jak jsme právě viděli,
opačná implikace neplatí. Uvedli jsme je však z toho důvodu, že jest-li $I$
\emph{uzavřený} interval, pak opačná implikace pravdivosti nabývá.

\begin{proposition}{}{stejno-implikuje-spojita}
 Každá stejnoměrně spojitá funkce je spojitá.
\end{proposition}
\begin{propproof}
 Ať $I \subseteq \R$ je interval a $f:I \to \R$ je stejnoměrně spojitá na $I$.
 Ukážeme, že $f$ je spojitá v každém bodě $a \in I$. Ať je vzdálenost
 $\varepsilon>0$ dána. Pro toto $\varepsilon$ existuje z
 \hyperref[def:stejnomerna-spojitost]{definice stejnoměrné spojitosti}
 $\delta>0$ takové, že pro každá dvě $x,y \in I$ s $|x-y|<\delta$ je $|f(x) -
 f(y)|<\varepsilon$. Speciálně, volbou $y \coloneqq a$ dostaneme, že
 $|x-a|<\delta$ implikuje $|f(x) - f(a)|<\varepsilon$. To je ale pouze přepis
 výroku, že pro všechna $x \in B(a,\delta)$ je $f(x) \in B(f(a),\varepsilon)$,
 čili $\lim_{x \to a} f(x) = f(a)$ a $f$ je spojitá v $a$.
\end{propproof}

\begin{theorem}{}{stejno-spojita-na-uzavrenem}
 Je-li $I = [a,b]$ uzavřený interval, pak $f$ je stejnoměrně spojitá na $I$,
 právě když je na $I$ spojitá.
\end{theorem}
\begin{thmproof}
 Z \myref{tvrzení}{prop:stejno-implikuje-spojita} víme, že je-li $f$ na $I$
 stejnoměrně spojitá, pak je tamže spojitá. Dokážeme opačnou implikaci tak, že z
 stejnoměrné nespojitosti $f$ odvodíme nespojitost.

 Ať tedy $f$ není stejnoměrně spojitá na $[a,b]$, to jest,
 \begin{equation}
  \label{eq:not-stejnomerne-spojita}
  \exists \varepsilon>0 \; \forall \delta>0 \; \exists x,y \in [a,b] : |x-y|
  <\delta \wedge |f(x) - f(y)| \geq \varepsilon.
 \end{equation}
 Dokážeme, že existuje bod $x \in [a,b]$ takový, že $\lim_{t \to x} f(t) \neq
 f(x)$ použitím \hyperref[cor:heineho-veta-pro-spojitost]{Heineho věty}. Mějme
 $\varepsilon>0$ z výroku \eqref{eq:not-stejnomerne-spojita}. Pak pro každé $n
 \in \N$ dostaneme volbou $\delta \coloneqq 1 / n$ existenci bodů $x_n,y_n \in
 [a,b]$ takových, že
 \[
  |x_n - y_n|<\frac{1}{n} \wedge |f(x_n) - f(y_n)| \geq \varepsilon.
 \]
 Protože $a \leq x_n \leq b$ pro každé $n \in \N$ je posloupnost $x_n$ omezená.
 Díky \hyperref[thm:bolzanova-weierstrassova]{Bolzanově-Weierstraßově větě}
 můžeme předpokládat, že je konvergentní. Z
 \myref{lemmatu}{lem:o-dvou-straznicich} platí $x \coloneqq \lim_{n \to \infty}
 x_n \in [a,b]$. Jelikož
 \[
  |x-y_n| \leq |x-x_n| + |x_n-y_n| \leq |x-x_n| + \frac{1}{n},
 \]
 neboli $\lim_{n \to \infty} |x-y_n| = 0$, rovněž $\lim_{n \to \infty} y_n = x$.
 Pro spor předpokládejme, že $f$ je spojitá v~$x$. Podle
 \hyperref[cor:heineho-veta-pro-spojitost]{Heineho věty} platí
 \[
  f(x) = \lim_{n \to \infty} f(x_n) = \lim_{n \to \infty} f(y_n).
 \]
 Pro každé $n \in \N$ máme
 \[
  0 \leq |f(x_n) - f(y_n)| \leq |f(x_n) - f(x)| + |f(x) - f(y_n)|,
 \]
 ale také
 \[
  \lim_{n \to \infty} (|f(x_n) - f(x)| + |f(x) - f(y_n)|) = 0,
 \]
 odkud
 \[
  \lim_{n \to \infty} |f(x_n) - f(y_n)| = 0,
 \]
 což je ve sporu s tím, že $|f(x_n) - f(y_n)| \geq \varepsilon$ pro každé $n \in
 \N$. Tedy, $f$ není spojitá v $x$, což zakončuje důkaz.
\end{thmproof}

Závěrem pododdílu konečně dokážeme, že spojité funkce jsou integrovatelné.

\begin{theorem}{Integrovatelnost spojité funkce}{integrovatelnost-spojite-funkce}
 Ať je $f$ spojitá na $[a,b]$. Pak $f \in \mathcal{R}(a,b)$.
\end{theorem}
\begin{thmproof}
 Podle \myref{věty}{thm:extremy-spojite-funkce} nabývá $f$ na $[a,b]$ maxima i
 minima, tudíž je na $[a,b]$ omezená. Ať je dáno $\varepsilon>0$. Nalezneme
 dělení $D$ takové, že
 \[
  \overline{S}(f,D) - \underline{S}(f,D) < \varepsilon.
 \]
 Tím bude díky \myref{lemmatu}{lem:kriterium-existence-integralu} důkaz hotov.

 Podle \myref{věty}{thm:stejno-spojita-na-uzavrenem} je $f$ stejnosměrně spojitá
 na $[a,b]$. Čili k danému $\varepsilon$ existuje $\delta>0$ takové, že pro $x,y
 \in [a,b]$ platí $|x-y|<\delta \Rightarrow |f(x) - f(y)|<\varepsilon$. Volme
 libovolné dělení $D = (x_i)_{i=0}^n$ intervalu $[a,b]$ s $\|D\|<\delta$.
 Protože $f$ nabývá na každém z dělících intervalů $[x_{i-1},x_i]$ minima i
 maxima (speciálně tedy \textbf{konečného} infima i suprema), platí díky
 stejnoměrně spojitosti
 \[
  \sup_{[x_{i-1},x_i]}f - \inf_{[x_{i-1},x_i]}f < \varepsilon
 \]
 pro každé $i \in \{1,\ldots,n\}$. Počítáme
 \begin{align*}
  \overline{S}(f,D) - \underline{S}(f,D) &= \sum_{i = 1}^{n}
  (\sup_{[x_{i-1},x_i]}f - \inf_{[x_{i-1},x_i]} f)(x_i - x_{i-1})\\
                                         &< \sum_{i = 1}^{n} \varepsilon(x_{i-1}
                                         - x_i) = \varepsilon(b-a),
 \end{align*}
 což bylo dokázati.
\end{thmproof}

\subsection{Základní věta kalkulu}
\label{ssec:zakladni-veta-kalkulu}

Zbytek oddílu o Riemannově integrálu věnován jest důkazu \emph{základní věty
kalkulu} -- tvrzení o souvislosti mezi primitivní funkcí a Riemannovým integrál
její derivace. Vlastně říká, že hodnota primitivní funkce v bodě je orientovanou
plochou pod její derivací od nějakého konstantního bodu do tohoto. K jeho důkazu
potřebujeme jen pár základní aritmetických vlastností Riemannova integrálu.

\begin{theorem}{Linearita Riemannova integrálu}{linearita-riemannova-integralu}
 Ať $f,g:[a,b] \to \R$ jsou integrovatelné funkce na $[a,b]$ a $\alpha \in \R$.
 Pak jsou $f + g$ i $\alpha f$ integrovatelné na $[a,b]$ a platí
 \begin{enumerate}
  \item $\int_{a}^{b} (f + g) = \int_{a}^{b} f + \int_{a}^{b} g$;
  \item $\int_{a}^{b} \alpha f = \alpha \int_{a}^{b} f$.
 \end{enumerate}
\end{theorem}
\begin{thmproof}
 Dokážeme vzorec (1). Ježto $f$ i $g$ jsou funkce omezené na $[a,b]$, je taktéž
 $f+g$ omezená na $[a,b]$. Snadno nahlédneme, že pro libovolný interval $I
 \subseteq \R$ platí nerovnosti
 \begin{align*}
  \inf_I f + \inf_I g & \leq \inf_I (f + g),\\
  \sup_I f + \sup_I g & \geq \sup_I (f + g).
 \end{align*}
 Odtud plyne, že pro libovolné dělení $D$ intervalu $[a,b]$ máme
 \begin{equation}
  \label{eq:linearita-odhady}
  \underline{S}(f,D) + \underline{S}(g,D) \leq \underline{S}(f+g,D) \leq
  \overline{S}(f+g,D) \leq \overline{S}(f,D) + \overline{S}(g,D).
 \end{equation}
 Volme posloupnost dělení $\{D_n\}_{n=0}^{\infty}$ s $\lim_{n \to \infty}
 \|D_n\| = 0$. Podle \myref{důsledku}{cor:riemann-vypocet} a
 \hyperref[thm:aritmetika-limit-funkci]{věty o aritmetice limit} jest
 \[
  \lim_{n \to \infty} \underline{S}(f,D_n) + \underline{S}(g,D_n) = \int_{a}^{b}
  f + \int_{a}^{b} g = \lim_{n \to \infty} \overline{S}(f,D_n) +
  \overline{S}(g,D_n),
 \]
 čili z nerovností \eqref{eq:linearita-odhady} užitím
 \myref{lemmatu}{lem:o-dvou-straznicich} rovněž
 \[
  \lim_{n \to \infty} \overline{S}(f + g,D_n) = \lim_{n \to \infty}
  \underline{S}(f+g,D_n) = \int_{a}^{b} f + \int_{a}^{b} g.
 \]
 Podle \myref{důsledku}{cor:riemann-existence} nyní platí $f + g \in
 \mathcal{R}(a,b)$ a
 \[
  \int_{a}^{b} (f + g) = \lim_{n \to \infty} \overline{S}(f+g,D_n) = \lim_{n \to
  \infty} \underline{S}(f+g,D_n) = \int_{a}^{b} f + \int_{a}^{b} g,
 \]
 jak jsme chtěli.

 Platnost vzorce (2) se ověří podobně a její důkaz je přenechán čtenáři.
\end{thmproof}

\begin{exercise}{}{linearita-riemannova-integralu}
 Dokažte vzorec (2) ve \myref{větě}{thm:linearita-riemannova-integralu}.
\end{exercise}

\begin{theorem}{Pár vlastností Riemannova integrálu}{par-vlastnosti-riemannova-integralu}
 Ať $f,g:[a,b] \to \R$ jsou integrovatelné na $[a,b]$ a $c \in (a,b)$. Potom
 platí následující.
 \begin{enumerate}
  \item Je-li $f \leq g$ na $[a,b]$, pak $\int_{a}^{b} f \leq \int_{a}^{b} g$;
  \item $\int_{a}^{b} f = \int_{a}^{c} f + \int_{c}^{b} f$;
  \item $\left| \int_{a}^{b} f \right| \leq \int_{a}^{b} |f|$.
 \end{enumerate}
\end{theorem}
\begin{thmproof}
 Tvrzení (1) je triviální. Z $f \leq g$ na $[a,b]$ plyne, že pro každý interval
 $I \subseteq [a,b]$ jest $\sup_I f \leq \sup_I g$ i $\inf_I f \leq \inf_I g$, a
 tedy pro každé dělení $D$ platí odhady
 \[
  \underline{S}(f,D) \leq \underline{S}(g,D) \quad \text{a} \quad
  \overline{S}(f,D) \leq \overline{S}(g,D),
 \]
 odkud již plyne tvrzení.

 Pro důkaz (2) volme posloupnost $\{D_{n}^{1}\}_{n=0}^{\infty}$ dělení intervalu
 $[a,c]$ a posloupnost $\{D^2_n\}_{n=0}^{\infty}$ intervalu $[c,b]$ s $\lim_{n
 \to \infty} \|D^1_n\| = \lim_{n \to \infty} \|D^2_n\| = 0$. Potom posloupnost
 dělení $D_n$ intervalu $[a,b]$ sestávajíc z~dělících bodů $D^{1}_n$ a $D^2_n$
 splňuje $\lim_{n \to \infty} \|D_n\| = 0$.

 Předpokládejme nejprve, že $f \in \mathcal{R}(a,b)$. Ukážeme, že $f \in
 \mathcal{R}(a,c) \cap \mathcal{R}(c,b)$ přes
 \myref{lemma}{lem:kriterium-existence-integralu}. Ať je $\varepsilon>0$ dáno.
 Nalezneme $\delta>0$ takové, že pro každé dělení $D$ intervalu $[a,b]$ s
 $\|D\|<\delta$ jest
 \[
  \overline{S}(f,D) - \underline{S}(f,D) < \varepsilon.
 \]
 Tedy, existuje rovněž $n_0 \in \N$ takové, že pro $n \geq n_0$ máme
 \[
  \overline{S}(f,D_n) - \underline{S}(f,D_n) < \varepsilon.
 \]
 Z definice $D_n$ jest
 \[
  \overline{S}(f,D_n) = \overline{S}(f,D_n^1) + \overline{S}(f,D_n^2) \quad
  \text{a} \quad \underline{S}(f,D_n) = \underline{S}(f,D_n^{1}) +
  \underline{S}(f,D_n^2).
 \]
 Pak
 \[
  \overline{S}(f,D_n) - \underline{S}(f,D_n) = (\overline{S}(f,D_n^{1}) -
  \underline{S}(f,D_n^{1})) + (\overline{S}(f,D_n^2) - \underline{S}(f,D_n^2)) <
  \varepsilon.
 \]
 Protože $\overline{S}(f,D_n^{i}) - \underline{S}(f,D_n^{i})$, kde $i = 1,2$, je
 kladné číslo, plyne z předchozího, že
 \[
  \overline{S}(f,D_n^{1}) - \underline{S}(f,D_n^{1}) < \varepsilon \quad
  \text{i} \quad \overline{S}(f,D_n^2) - \underline{S}(f,D_n^2)<\varepsilon,
 \]
 čili $f \in \mathcal{R}(a,c)$ i $f \in \mathcal{R}(c,b)$. Z
 \myref{důsledku}{cor:riemann-vypocet} a \hyperref[thm:aritmetika-limit]{věty o
 aritmetice limit} dopočteme
 \[
  \int_{a}^{b} f = \lim_{n \to \infty} \overline{S}(f,D_n) = \lim_{n \to \infty}
  \overline{S}(f,D_n^{1}) + \lim_{n \to \infty} \overline{S}(f,D_n^2) =
  \int_{a}^{c} f + \int_{c}^{b} f.
 \]
 Je-li naopak $f \in \mathcal{R}(a,c) \cap \mathcal{R}(c,b)$, pak zcela
 obráceným argumentem ukážeme, že $f \in \mathcal{R}(a,b)$, a stejný výpočet pak
 dokončuje důkaz vzorce (2).

 Vzorec (3) plyne z triviálního pozorování, že
 \[
  \sup_I |f| - \inf_I |f| \leq \sup_I f - \inf_I f,
 \]
 který přenecháme k ověření čtenáři. Odtud
 \[
  \overline{S}(|f|,D) - \underline{S}(|f|,D) \leq \overline{S}(f,D) -
  \underline{S}(f,D)
 \]
 pro libovolné dělení $D$. Je-li tedy $f \in \mathcal{R}(a,b)$ pak z
 \myref{lemmatu}{lem:kriterium-existence-integralu} rovněž $|f| \in
 \mathcal{R}(a,b)$, neboť pro dané $\varepsilon>0$, $\delta>0$ a dělení $D$ s
 $\|D\|< \delta$ máme
 \[
  \overline{S}(|f|,D) - \underline{S}(|f|,D) \leq \overline{S}(f,D) -
  \underline{S}(f,D) < \varepsilon.
 \]
 Dále, jistě platí
 \[
  \int_{a}^{b} f \leq \overline{S}(f,D) \leq \overline{S}(|f|,D),
 \]
 čili
 \[
  \int_{a}^{b} f \leq \int_{a}^{b} |f|.
 \]
 Podle \myref{věty}{thm:linearita-riemannova-integralu} je funkce $-f$
 integrovatelná a jejím dosazením do rovnosti výše obdržíme
 \[
  - \int_{a}^{b} f = \int_{a}^{b} -f \leq \int_{a}^{b} |-f| = \int_{a}^{b} |f|.
 \]
 Sloučení obou nerovností dává konečně
 \[
  \left| \int_{a}^{b} f \right| \leq \int_{a}^{b} |f|,
 \]
 čímž je důkaz hotov.
\end{thmproof}

\begin{exercise}{}{}
 Dokažte, že pro každý interval $I \subseteq \R$ a omezenou funkci $f:I \to \R$
 platí
 \[
  \sup_I |f| - \inf_I |f| \leq \sup_I f - \inf_I f.
 \]
\end{exercise}

\begin{theorem}{Základní věta kalkulu}{zakladni-veta-kalkulu}
 Ať $I \subseteq \R$ je interval a $f:I \to \R$ je funkce integrovatelná na
 každém podintervalu $[a,b] \subseteq I$. Pak pro libovolné $c \in I$ je funkce
 \[
  F(x) \coloneqq \int_{c}^{x} f(t) \, \mathrm{d}t
 \]
 spojitá na $I$ a pro všechna $x_0 \in I$, v nichž je $f$ spojitá, platí
 $F'(x_0) = f(x_0)$.
\end{theorem}

\begin{thmproof}
 Nejprve ukážeme, že $F$ je spojitá na $I$. Ať $y_0 \in I$ leží ve vnitřku $I$
 (pro krajní body bychom postupovali obdobně, akorát bychom nahradili příslušné
 limity jednostrannými variantami). Díky tomuto předpokladu lze nalézt
 $\delta>0$ takové, že $[y_0-\delta,y_0+\delta] \subseteq I$. Podle předpokladu
 je $f$ integrovatelná na $[y_0-\delta,y_0+\delta]$, tedy je tamže omezená
 například číslem $K>0$. Pro $y \in [y_0-\delta,y_0+\delta]$ odhadujme
 \[
  |F(y) - F(y_0)| = \left| \int_{c}^{y} f - \int_{c}^{y_0} f \right|
  \overset{(\alpha)}{=} \left| \int_{y}^{y_0} f \right| \overset{(\beta)}{ \leq
  } \int_{y}^{y_0} |f| \leq K|y-y_0|,
 \]
 kde rovnost $(\alpha)$ plyne z
 \myref{věty}{thm:par-vlastnosti-riemannova-integralu}, části (2), a nerovnost
 $(\beta)$ z téže věty, části (3). Odtud
 \[
  \lim_{y \to y_0} |F(y) - F(y_0)| = 0,
 \]
 čili $\lim_{y \to y_0} F(y) = F(y_0)$ a $F$ je spojitá v $y_0$.

 Pokračujme důkazem, že v bodech spojitosti $f$ platí $F' = f$. Ať je tedy
 $x_0 \in I$ takovým bodem. Volme $\varepsilon>0$. K němu z
 \hyperref[def:oboustranna-limita-funkce]{definice limity} existuje $\delta>0$
 takové, že pro $x \in B(x_0,\delta)$ jest $|f(x) - f(x_0)|<\varepsilon$.
 Počítejme (opět za pomoci
 \myref{věty}{thm:par-vlastnosti-riemannova-integralu})
 \begin{align*}
  \left| \frac{F(x) - F(x_0)}{x - x_0} - f(x_0) \right| 
  &= \left|\frac{1}{x-x_0} \int_{x_0}^{x} f(t) \, \mathrm{d}t - f(x_0) \right|
  \\
  &= \frac{1}{|x-x_0|} \left| \int_{x_0}^{x} f(t) \, \mathrm{d}t -
  \int_{x_0}^{x} f(x_0) \, \mathrm{d}t \right| \\
  & \leq \frac{1}{|x-x_0|} \left| \int_{x_0}^{x} |f(t) - f(x_0)| \, \mathrm{d}t
  \right| \\
  &< \frac{1}{|x-x_0|} \left| \int_{x_0}^{x} \varepsilon \, \mathrm{d}t \right|
  = \varepsilon.
 \end{align*}
 Z tohoto výpočtu plyne, že
 \[
  \lim_{x \to x_0} \left| \frac{F(x) - F(x_0)}{x - x_0} - f(x_0) \right| = 0,
 \]
 neboli
 \[
  F'(x_0) = \lim_{x \to x_0} \frac{F(x) - F(x_0)}{x - x_0} = f(x_0),
 \]
 jak jsme chtěli.
\end{thmproof}

\begin{remark}{}{zakladni-veta-kalkulu}
 \hyperref[thm:zakladni-veta-kalkulu]{Základní věta kalkulu} dává mimo jiné
 geometrický pohled na fakt, že se dvě různé primitivní funkce liší o konstantu.
 Vyjadřují totiž orientovanou plochu pod svojí derivací od dvou různých
 počátečních bodů.
\end{remark}

% TODO odkaz
Koncem oddílu uvedeme technické lemma, které později použijeme k důkazu
ekvivalence Newtonova a Riemannova integrálu.

\begin{lemma}{}{charakterizace-integrovatelnosti}
 Ať $f:[a,b] \to \R$ je funkce (ne nutně omezená). Pak jsou následující výroky
 ekvivalentní.
 \begin{enumerate}
  \item $f \in \mathcal{R}(a,b)$;
  \item Existuje číslo $I \in \R$ takové, že pro libovolné $\varepsilon>0$
   existuje $\delta>0$ splňující: kdykoli je $D = (x_i)_{i=0}^{n}$ dělení
   $[a,b]$ s $\|D\|<\delta$ a $t_i \in [x_{i-1},x_i], i \in \{1,\ldots,n\},$
   libovolné body ležící v~dělících intervalech, tak
   \begin{equation}
    \label{eq:charakterizace-integrovatelnosti}
    \left| \sum_{i = 1}^{n} f(t_i)(x_i - x_{i-1}) - I \right| < \varepsilon.
   \end{equation}
 \end{enumerate}
\end{lemma}
\begin{lemproof}
 Dokážeme implikaci $(1) \Rightarrow (2)$. Ať $f \in \mathcal{R}(a,b)$ a položme
 $I \coloneqq \int_{a}^{b} f$. Nechť je $\varepsilon>0$ dáno. Podle
 \myref{lemmatu}{lem:kriterium-existence-integralu} existuje k tomuto
 $\varepsilon>0$ číslo $\delta>0$ takové, že pro každé dělení $D =
 (x_i)_{i=0}^{n}$ s
 $\|D\|<\delta$ platí
 \[
  I - \varepsilon = \int_{a}^{b} f - \varepsilon < \underline{S}(f,D) \leq
  \overline{S}(f,D) < \int_{a}^{b} f + \varepsilon = I +\varepsilon.
 \]
 Pro každé $i \in \{1,\ldots,n\}$ a $t_i \in [x_{i-1},x_i]$ zřejmě platí
 \[
  \inf_{[x_{i-1},x_i]} f \leq f(t_i) \leq \sup_{[x_{i-1},x_i]}f,
 \]
 čili
 \[
  I - \varepsilon < \underline{S}(f,D) \leq \sum_{i = 1}^{n} f(t_i)(x_i -
  x_{i-1}) \leq \overline{S}(f,D) < I + \varepsilon,
 \]
 takže \eqref{eq:charakterizace-integrovatelnosti} platí.

 Ad $(2) \Rightarrow (1)$. Nejprve je třeba ukázat, že $f$ je omezená. Pro
 $\varepsilon \coloneqq 1$ nalezneme z platnosti výroku (2) číslo $\delta>0$
 takové, že pro jakoukoli volbu $t_i \in [x_{i-1},x_i]$ jest
 \[
  \left| \sum_{i = 1}^{n} f(t_i)(x_i - x_{i-1}) \right|<\varepsilon.
 \]
 Položme
 \begin{align*}
  \mu(D) &\coloneqq \min \{x_i - x_{i-1} \mid \in \{1,\ldots,n\}\},\\
  K & \coloneqq \max \{|f(x_i) \mid i \in \{0,\ldots,n\}\}.
 \end{align*}
 Ať $t \in [a,b]$. Nalezneme index $j \in \{1,\ldots,n\}$ takový, že $t \in
 [x_{j-1},x_j]$. Volme body
 \[
  t_i \coloneqq \begin{cases}
   t,& i = j,\\
   x_i,& i \neq j.
  \end{cases}
 \]
 Potom můžeme odhadnout
 \begin{align*}
  |f(t)(x_j - x_{j-1})| &= \left| \sum_{i = 1}^{n} f(t_i)(x_i - x_{i-1}) - I + I
  - \sum_{i \neq j} f(x_i)(x_i - x_{i-1}) \right| \\
  & \leq \left| \sum_{i = 1}^{n} f(t_i)(x_i - x_{i-1}) - I \right| + |I| +
  \sum_{i = 1}^{n} |f(x_i)(x_i - x_{i-1})| \\
  & \leq 1 + |I| + \sum_{i=1}^{n} K(x_i - x_{i-1}) = 1 + |I| + K(b-a).
 \end{align*}
 Odtud,
 \[
  |f(t)| \leq \frac{1}{\mu(D)}(1 + |I| + K(b-a)),
 \]
 čili $f$ je omezená, neboť $t \in [a,b]$ bylo voleno libovolně.

 Pro důkaz integrovatelnosti $f$ použijeme opět
 \myref{lemma}{lem:kriterium-existence-integralu}. Ať je nyní $\varepsilon>0$
 dáno libovolné a nalezněme z (2) číslo $\delta>0$ a mějme určeno dělení $D =
 (x_i)_{i=0}^{n}$ splňující $\|D\|<\delta$. Z definice suprema nalezněme body
 $t_i \in [x_{i-1},x_i]$ takové, že
 \[
  \sup_{[x_{i-1},x_i]} f < f(t_i) + \varepsilon
 \]
 pro každé $i \in \{1,\ldots,n\}$. Pak
 \begin{align*}
  \overline{S}(f,D) &= \sum_{i = 1}^{n} \sup_{[x_{i-1},x_i]} f \cdot (x_i -
  x_{i-1}) < \sum_{i = 1}^{n} (f(t_i) + \varepsilon)(x_i - x_{i-1})\\
                    &= \sum_{i = 1}^{n} f(t_i)(x_i - x_{i-1}) + \varepsilon(b-a)
                    < I + \varepsilon + \varepsilon(b-a).
 \end{align*}
 Podobně odhadneme i
 \[
  \underline{S}(f,D) > I - \varepsilon - \varepsilon(b-a).
 \]
 Celkem dostaneme, že
 \[
  \overline{S}(f,D) - \underline{S}(f,D) < 2(1 + b - a)\varepsilon,
 \]
 takže $f \in \mathcal{R}(a,b)$ z
 \myref{lemmatu}{lem:kriterium-existence-integralu}.
\end{lemproof}


\section{Newtonův integrál}
\label{sec:newtonuv-integral}

\hyperref[thm:zakladni-veta-kalkulu]{Základní věta kalkulu} nabízí druhou metodu
výpočtu orientované plochy pod danou funkcí. Jsou-li splněny její předpoklady,
pak
\begin{equation}
 \label{eq:newtonuv-integral}
 \int_{a}^{b} f = F(b) - F(a)
\end{equation}
pro reálnou funkci $F$ definovanou jako $F(x) \coloneqq \int_{c}^{x} f$ s jistou
konstantou $c \in (a,b)$ a všechna $x \in [a,b]$. Přestože historicky je
\uv{definice} integrálu v \eqref{eq:newtonuv-integral} ta původní, my k ní
docházíme teprve prostřednictvím integrálu Riemannova. Lze tvrdit, že tato cesta
je více přirozená, stavějíc intuitivní definici plochy před tu odvozenou z
vlastností primitivních funkcí, jež -- na první pohled zcela jistě -- s
orientovanou plochou nikterak nesouvisí.

Výhodou takto definovaného integrálu je jeho existence i pro jisté neomezené
funkce na otevřených intervalech -- v \eqref{eq:newtonuv-integral} lze totiž
nahradit $F(b)$ a $F(a)$ příslušnými jednostrannými limitami, jak též záhy
učiníme. Na druhou stranu se takto omezujeme na funkce, k nimž existuje funkce
primitivní. Takto definovaný integrál sluje \emph{Newtonův}.

\begin{definition}{Newtonův integrál}{newtonuv-integral}
 Ať $a,b \in \R^{*}$. Řekneme, že funkce $f:(a,b) \to \R$ je \emph{newtonovsky}
 integrovatelná, píšeme $f \in \mathcal{N}(a,b)$, pakližeť
 \begin{enumerate}
  \item má funkce $f$ na $(a,b)$ primitivní funkci $F$,
  \item rozdíl $\lim_{x \to b^{-}} F(x) - \lim_{x \to a^{+}} F(x)$ je definován.
 \end{enumerate}
 Jsou-li splněny obě podmínky, je tento rozdíl nazýván hodnotou \emph{Newtonova
 integrálu} z funkce $f$ na $(a,b)$. Symbolicky
 \[
  (\mathcal{N}) \int_{a}^{b} f \coloneqq \lim_{x \to b^{-}} F(x) - \lim_{x \to
  a^{+}} F(x).
 \]
 Pro stručnost zápisu budeme nadále onen rozdíl značit symbolem $[F]_a^{b}$.
\end{definition}

\begin{remark}{}{newtonuv-integral}
 V dalším textu budeme muset jistou chvíli rozlišovat mezi Newtonovým a
 Riemannovým integrálem -- přinejmenším, dokud neukážeme jejich ekvivalenci pro
 funkce newtonovsky i riemannovsky integrovatelné. Učiníme to uvedením písmene
 $(\mathcal{N})$ či $(\mathcal{R})$ před samotný symbol integrálu, jako v
 \hyperref[def:newtonuv-integral]{předchozí definici}, v případě, že uvažovaný
 typ integrálu není zřejmý z kontextu. V celém zbytku oddílu, není-li uvedeno
 jinak, značí symbol $\int_{a}^{b}$ integrál \emph{Newtonův}.
\end{remark}

\begin{warning}{}{newtonuv-integral-nekonecno}
 Je dobré vzít na mysl, že Newtonův integrál z dané funkce může být i nekonečný.
 Podmínkou jeho existence byla pouze \emph{definovanost} výrazu $[F]_a^{b}$,
 nikoli jeho \emph{konečnost}. V této souvislosti si všimněme, že jeho meze
 mohou z \hyperref[def:newtonuv-integral]{definice} nabývat i hodnot $ \pm
 \infty$. Tím se přímo liší od integrálu Riemannova, neboť omezenost funkce v
 tomto případě konečnost horních i dolních součtů zaručuje.
\end{warning}

\begin{example}{}{newtonuv-integral-polynomu}
 Platí
 \[
  \int_{0}^{1} x^{\alpha} \, \mathrm{d}x = \begin{cases}
   \left[ \frac{1}{\alpha+1}x^{\alpha+1} \right]_0^{1} = \frac{1}{\alpha+1},
   & \quad \text{pro } \alpha \in (-1,\infty); \\
   \left[ \frac{1}{\alpha+1}x^{\alpha+1} \right]_0^{1} = -\infty, & \quad
   \text{pro } \alpha \in (-\infty,-1); \\
   [\log x]_0^{1} = \infty,& \quad \text{pro } \alpha = -1.
  \end{cases}
 \]
 Vskutku, primitivní funkce k $x^{\alpha}$ byla spočtena v
 \myref{příkladě}{exam:integral-polynomu} a výpočet příslušných limit je
 triviální.
\end{example}

\begin{example}{}{newton-vs-riemann}
 \begin{enumerate}
  \item Funkce $f(x) = 1 / x$ je na intervalu $(0,1)$ newtonovsky
   integrovatelná, konkrétně
   \[
    (\mathcal{N}) \int_{0}^{1} \frac{1}{x} \, \mathrm{d}x = \infty,
   \]
   podle \myref{příkladu}{exam:newtonuv-integral-polynomu} pro $\alpha = -1$,
   ale není na $[0,1]$ riemannovsky integrovatelná, protože na tomto intervalu
   není omezená.
  \item Naopak, funkce $\sign x$ je na intervalu $[-1,1]$ riemannovsky
   integrovatelná, díky \myref{tvrzení}{prop:integrovatelnost-monotonni-funkce},
   protože je na $[-1,1]$ monotónní. Avšak, na $(-1,1)$ newtonovsky
   integrovatelná není, neboť nemá na celém tomto intervalu primitivní funkci.
 \end{enumerate}
\end{example}

Newtonův integrál je člověku výpočetně bližší než integrál Riemannův; je k němu
třeba umět hledat primitivní funkci. Naopak, Riemannův integrál je mnohem blíže
způsobu, kterým počítače vyhodnocují integrály, neboť limity konvergentních
posloupností (v tomto případě součtů přes zjemňující se dělení) se aproximují
triviálně. Je rozumné se domnívat, že metody výpočtu primitivních funkcí
představené v \myref{sekci}{sec:vypocet-primitivnich-funkci} lze úspěšně použít
i k výpočtu Newtonových integrálů.

\begin{theorem}{Linearita Newtonova integrálu}{linearita-newtonova-integralu}
 Ať $a,b \in \R^{*}, f,g \in \mathcal{N}(a,b)$ a $c \in \R$. Pak $f + g, cf \in
 \mathcal{N}(a,b)$ a platí
 \begin{align*}
  \int_{a}^{b} f(x) + g(x) \, \mathrm{d}x &= \int_{a}^{b} f(x) \, \mathrm{d}x +
  \int_{a}^{b} g(x) \, \mathrm{d}x.\\
  \int_{a}^{b} cf(x) \, \mathrm{d}x &= c \int_{a}^{b} f(x) \, \mathrm{d}x.
 \end{align*}
\end{theorem}
\begin{thmproof}
 Ať $F,G$ jsou primitivní k $f,g$ na $(a,b)$. Potom jsou $F + G$ a $cF$
 primitivní k $f + g$ a $cf$ na $(a,b)$ a z
 \hyperref[thm:aritmetika-limit-funkci]{věty o aritmetice limit} máme
 \[
  [F + G]_a^{b} = [F]_a^{b} + [G]_a^{b} \quad \text{a} \quad [cF]_a^{b} =
  c[F]_a^{b}.
 \]
 Odtud,
 \begin{align*}
  \int_{a}^{b} f(x) + g(x) \, \mathrm{d}x 
  &= [F + G]_a^{b} = [F]_a^{b} + [G]_a^{b} = \int_{a}^{b} f(x) \, \mathrm{d}x +
  \int_{a}^{b} g(x) \, \mathrm{d}x,\\
  \int_{a}^{b} cf(x) \, \mathrm{d}x 
  &= [cF]_a^{b} = c[F]_a^{b} = c \int_{a}^{b} f(x) \, \mathrm{d}x,
 \end{align*}
 jak bylo dokázati.
\end{thmproof}

\begin{theorem}{Per partes pro Newtonův integrál}{per-partes-pro-newtonuv-integral}
 Ať $a,b \in \R^{*}$, $f,g:(a,b) \to \R$ jsou reálné funkce a $F$, resp. $G$, je
 primitivní k $f$, resp. $g$, na $(a,b)$ Potom platí
 \[
  \int_{a}^{b} f(x)G(x) \, \mathrm{d}x = [FG]_{a}^{b} - \int_{a}^{b} F(x)g(x) \,
  \mathrm{d}x,
 \]
 dává-li pravá strana smysl.
\end{theorem}
\begin{thmproof}
 Protože pravá strana je z předpokladu dobře definována, existuje primitivní
 funkce k $Fg$ na $(a,b)$. Označme ji $H$. Platí
 \[
  (FG - H)' = fG + Fg - Fg = fG,
 \]
 čili $FG - H$ je primitivní k $fG$ na $(a,b)$. Dále
 \begin{align*}
  \int_{a}^{b} f(x)G(x) \, \mathrm{d}x
  &= [FG - H]_a^{b} = \lim_{x \to b^{+}} (F(x)G(x) - H(x)) - \lim_{x \to a^{+}}
  (F(x)G(x) - H(x))\\
  &= \lim_{x \to b^{-}} F(x)G(x) - \lim_{x \to b^{-}} F(x)G(x) - (\lim_{x \to
  b^{-}} H(x) - \lim_{x \to a^{+}} H(x))\\
  &= [FG]_a^{b} - [H]_a^{b} = [FG]_a^{b} - \int_{a}^{b} F(x)g(x) \,
  \mathrm{d}x,
 \end{align*}
 čímž je důkaz hotov.
\end{thmproof}

\begin{theorem}{Substituce pro Newtonův integrál}{substituce-pro-newtonuv-integral}
 Nechť $a,b,\alpha,\beta \in \R^{*}$, $f:(a,b) \to \R$ a $\varphi:(\alpha,\beta)
 \to (a,b)$. Ať má $\varphi$ dále konečnou nenulovou derivaci na
 $(\alpha,\beta)$ a platí $\varphi((\alpha,\beta)) = (a,b)$. Potom
 \[
  \int_{a}^{b} f(x) \, \mathrm{d}x = \int_{\alpha}^{\beta} (f \circ \varphi)(t)
  |\varphi'(t)|\,
  \mathrm{d}t,
 \]
 dává-li aspoň jedna strana smysl.
\end{theorem}
\begin{thmproof}
 Funkce $\varphi$ má na $(\alpha,\beta)$ z předpokladu konečnou derivaci. Z
 \myref{věty}{thm:darbouxova} je $\varphi'((\alpha,\beta))$ interval a opět z
 předpokladu $0 \notin \varphi'((\alpha,\beta))$. Tedy, $\varphi'$ je na
 $(\alpha,\beta)$ buď záporná, nebo kladná. Bez újmy na obecnosti budeme
 předpokládat, že $\varphi' > 0$ na $(\alpha,\beta)$, čili $\varphi$ je na
 $(\alpha,\beta)$ rostoucí. Rozlišíme dvě možnosti: budeme předpokládat, že
 existuje levá strana rovnosti a dokážeme existenci integrálu na pravé straně, a
 naopak.

 Ať nejprve existuje $\int_{a}^{b} f$. Pak má $f$ na $(a,b)$ primitivní funkci
 $F$ a existují limity $\lim_{x \to a^{+}} F(x)$ i $\lim_{x \to b^{-}} F(x)$. Z
 \hyperref[thm:derivace-slozene-funkce]{věty o derivaci složené funkce}
 ($\varphi$ je rostoucí, tedy prostá), má i $(f \circ \varphi) \cdot \varphi'$
 na intervalu $(\alpha,\beta)$ primitivní funkci, konkrétně funkci $F \circ
 \varphi$. Platí
 \[
  \lim_{t \to \beta^{-}} \varphi(t) = b \quad \text{a} \quad \lim_{t \to \alpha
  ^{+}} \varphi(t) = a.
 \]
 Z \hyperref[thm:limita-slozene-funkce]{věty o limitě složené funkce} (opět,
 $\varphi$ je prostá) máme
 \[
  \lim_{t \to \beta^{-}} (F \circ \varphi)(t) = \lim_{x \to b^{-}} F(x) \quad
  \text{a} \quad \lim_{t \to \alpha ^{+}} (F \circ \varphi)(t) = \lim_{x \to
  a^{+}} F(x).
 \]
 Můžeme pročež počítat
 \begin{align*}
  \int_{\alpha}^{\beta} (f \circ \varphi)(t)|\varphi'(t)| \, \mathrm{d}t 
  &= \int_{\alpha}^{\beta} (f \circ \varphi)(t) \varphi'(t) \, \mathrm{d}t =
  [F \circ \varphi]_{\alpha}^{\beta}\\
  &= [F]_a^{b} = \int_{a}^{b} f(x) \, \mathrm{d}x.
 \end{align*}

 Nyní ať existuje $\int_{\alpha}^{\beta} (f \circ \varphi) \cdot |\varphi'|$.
 Označme $G$ primitivní funkci k $(f \circ \varphi) \cdot |\varphi'| = (f \circ
 \varphi) \cdot \varphi'$. Z \hyperref[thm:druha-o-substituci]{druhé věty o
 substituci} plyne, že $G \circ \varphi ^{-1}$ je primitivní k $f$ na $(a,b)$.
 Platí
 \[
  \lim_{x \to a^{+}} \varphi ^{-1}(x) = \alpha \quad \text{a} \quad \lim_{x \to
  b^{-}} \varphi ^{-1}(x) = \beta,
 \]
 a tedy
 \[
  \lim_{x \to a^{+}} (G \circ \varphi ^{-1})(x) = \lim_{t \to \alpha ^{+}}
  G(t) \quad \text{a} \quad \lim_{x \to b^{-}} (G \circ \varphi ^{-1})(x) =
  \lim_{t \to b^{-}} G(t).
 \]
 Dopočteme
 \[
  \int_{a}^{b} f(x) \, \mathrm{d}x = [G \circ \varphi ^{-1}]_a^{b} =
  [G]_{\alpha}^{\beta} = \int_{\alpha}^{\beta} (f \circ \varphi)(t)|\varphi'(t)|
  \, \mathrm{d}t,
 \]
 jak jsme chtěli.
\end{thmproof}

\begin{problem}{}{newtonuv-integral-1}
 Spočtěte
 \[
  \int_{-1}^{1} x^2 \exp(-x) \, \mathrm{d}x.
 \]
\end{problem}
\begin{probsol}
 Položme $f(x) = \exp(-x)$ a $G(x) = x^2$. Pak $F(x) = -\exp(-x)$ a $g(x) = 2x$.
 Podle \myref{věty}{thm:per-partes-pro-newtonuv-integral} platí
 \[
  \int_{-1}^{1} x^2 \exp(-x) \, \mathrm{d}x = [-x^2 \exp(-x)]_{-1}^{1} -
  \int_{-1}^{1} -2x \exp(-x) \, \mathrm{d}x.
 \]
 Jistě
 \[
  [-x^2 \exp(-x)]_{-1}^{1} = -1 \cdot \exp(-1) - (-1)\exp(1) = -\frac{1}{e} + e.
 \]
 Z \hyperref[thm:linearita-newtonova-integralu]{linearity integrálu} platí
 \[
  \int_{-1}^{1} -2x \exp(-x) \, \mathrm{d}x = -2 \int_{-1}^{1} x \exp(-x) \,
  \mathrm{d}x.
 \]
 Opětovným použitím \hyperref[thm:per-partes-pro-newtonuv-integral]{per partes}
 dostaneme
 \[
  \int_{-1}^{1} x \exp(-x) \, \mathrm{d}x = [-x \exp(-x)]_{-1}^{1} -
  \int_{-1}^{1} -\exp(-x) \, \mathrm{d}x.
 \]
 Snadno spočteme, že
 \begin{align*}
  [-x \exp(-x)]_{-1}^{1} &= -1 \cdot \exp(-1) - 1 \cdot \exp(1) = -\frac{1}{e} -
  e,\\
  \int_{-1}^{1} -\exp(-x) \, \mathrm{d}x &= [\exp(-x)]_{-1}^{1} = \exp(-1) -
  \exp(1) = \frac{1}{e} - e.
 \end{align*}
 Celkem tedy,
 \[
  \int_{-1}^{1} x\exp(-x) \, \mathrm{d}x = -\frac{1}{e} - e - \frac{1}{e} + e =
  -\frac{2}{e}.
 \]
 Dopočteme,
 \begin{align*}
  \int_{-1}^{1} x^2\exp(-x) \, \mathrm{d}x 
  &= [-x^2 \exp(-x)]_{-1}^{1} - \int_{-1}^{1} -2x \exp(-x) \, \mathrm{d}x\\
  &= -\frac{1}{e} + e + 2 \left( -\frac{2}{e} \right) = -\frac{5}{e} + e.
 \end{align*}
\end{probsol}

\begin{problem}{}{newtonuv-integral-2}
 Spočtěte
 \[
  \int_{1}^{2} \frac{3x^2}{x^3 - 1} \, \mathrm{d}x.
 \]
\end{problem}
\begin{probsol}
 Rozložíme $3x^2 / (x^3 - 1)$ na parciální zlomky. Podle
 \myref{věty}{thm:rozklad-na-parcialni-zlomky} existují čísla $A,B,C \in \R$
 taková, že
 \[
  \frac{3x^2}{x^3 - 1} = \frac{A}{x+1} + \frac{Bx + C}{x^2 - x + 1}.
 \]
 Roznásobení dá
 \[
  3x^2 = A(x^2 - x + 1) + (Bx + C)(x + 1).
 \]
 Dosazením $x = -1$ dostaneme $A = 1$. Potom
 \[
  2x^2 = Bx^2 + (B + C - 1)x + (C + 1). 
 \]
 Srovnáním koeficientů zjistíme, že musí platit $B = 2$ a $C = -1$. Tím
 dostáváme rozklad
 \[
  \frac{3x^2}{x^3 - 1} = \frac{1}{x+1} + \frac{2x - 1}{x^2 - x + 1}.
 \]
 Z \hyperref[thm:linearita-newtonova-integralu]{linearity integrálu} tudíž
 \[
  \int_{1}^{2} \frac{3x^2}{x^3 - 1} \, \mathrm{d}x = \int_{1}^{2} \frac{1}{x+1}
  \, \mathrm{d}x + \int_{1}^{2} \frac{2x - 1}{x^2 - x + 1} \, \mathrm{d}x.
 \]
 Podle \myref{příkladu}{exam:newtonuv-integral-polynomu} jest
 \[
  \int_{1}^{2} \frac{1}{x+1} \, \mathrm{d}x = \left[ \log|x+1|\right]_1^{2} =
  \log 3 - \log 2.
 \]
 Dále si všimneme, že $(x^2 - x + 1)' = 2x - 1$, a navíc je $x^2 - x + 1$ na
 $(1,2)$ kladná. Odtud plyne, že $\log(x^2 - x + 1)$ je primitivní k $(2x - 1) /
 (x^2 - x + 1)$ na $(1,2)$, čili
 \[
  \int_{1}^{2} \frac{2x - 1}{x^2 - x + 1} \, \mathrm{d}x = \left[ \log(x^2 - x +
  1)\right]_1^{2} = \log 3 - \log 1 = \log 3.
 \]
 Celkem tedy
 \[
  \int_{1}^{2} \frac{3x^2}{x^3 - 1} \, \mathrm{d}x = 2 \log 3 - \log 2. 
 \]
\end{probsol}

\begin{exercise}{}{newtonovy-integraly}
 Spočtěte následující Newtonovy integrály.
 {\everymath={\displaystyle}
  \arraycolsep=1em
  \[
   \begin{array}{ll}
    \int_{0}^{\frac{\pi}{4}} \sqrt{\cos x - \cos^3 x} \, \mathrm{d}x &
    \int_{0}^{1} x^2 \sqrt{1 - x^2} \, \mathrm{d}x\\[2em]
    \int_{1}^{2} \frac{1}{x\log x} \, \mathrm{d}x & \int_{1}^{\infty}
    \frac{\exp(-\sqrt{x})}{\sqrt{x}} \, \mathrm{d}x
   \end{array}
  \]
 }
\end{exercise}

Naším posledním počinem v kapitole o primitivních funkcích a vlastně i v celém
tomto skromném úvodu do disciplíny matematické analýzy bude důkaz ekvivalence
Newtonova a Riemannova integrálu na intervalu, kde je daná funkce riemannovsky i
newtonovsky integrovatelná. Zde přichází.

\begin{theorem}{Ekvivalence Newtonova a Riemannova integrálu}{ekvivalence-newtonova-a-riemannova-integralu}
 Ať $f:[a,b] \to \R$ je reálná funkce a $f \in \mathcal{R}(a,b) \cap
 \mathcal{N}(a,b)$. Potom
 \[
  (\mathcal{R}) \int_{a}^{b} f = (\mathcal{N})\int_{a}^{b} f.
 \]
\end{theorem}
\begin{thmproof}
 Použijeme \myref{lemma}{lem:charakterizace-integrovatelnosti}. Z předpokladu je
 $f \in \mathcal{R}(a,b)$, tedy pro dané $\varepsilon>0$ existuje $\delta>0$
 takové, že kdykoli je $D = (x_i)_{i=0}^{n}$ dělení $[a,b]$ s $\|D\|<\delta$,
 pak pro libovolné body $t_i \in [x_{i-1},x_i]$, $i \in \{1,\ldots,n\}$, platí
 odhad
 \[
  \left| \sum_{i = 1}^{n} f(t_i)(x_i - x_{i-1}) - (\mathcal{R})\int_{a}^{b} f
  \right| < \varepsilon.
 \]
 Vezměme pročež nějaké takové dělení $D$ s $\|D\|<\delta$. Ať je $F$ primitivní
 k $f$ na $(a,b)$ -- ta existuje, protože $f \in \mathcal{N}(a,b)$. Z
 newtonovské integrovatelnosti $f$ rovněž plyne, že $\lim_{x \to b^{-}} F(x)$ a
 $\lim_{x \to a^{+}} F(x)$ existují konečné a $F$ je spojitá a omezená na
 $(a,b)$. Definujme novou funkci $H:[a,b] \to \R$ předpisem
 \[
  H(x) \coloneqq \begin{cases}
   \lim_{x \to a^{+}} F(x),& x = a,\\
   F(x), & x \in (a,b),\\
   \lim_{x \to b^{-}} F(x), & x = b.
  \end{cases}
 \]
 Funkce $H$ je vlastně \uv{dodefinováním} funkce $F$ jejími limitami v krajních
 bodech $[a,b]$. Zřejmě pro každé $c \in (a,b)$ platí $H'(c) = F'(c) = f(c)$.
 Nyní využijeme \hyperref[thm:lagrangeova-o-stredni-hodnote]{Lagrangeovy věty o
 střední hodnotě}. Podle ní existuje pro každé $i \in \{1,\ldots,n\}$ bod
 $t_i \in (x_{i-1},x_i)$ takový, že
 \[
  H(x_i) - H(x_{i-1}) = H'(t_i)(x_i - x_{i-1}) = f(t_i)(x_i - x_{i-1}).
 \]
 Pak ale můžeme počítat
 \begin{align*}
  (\mathcal{N})\int_{a}^{b} f 
  &= \lim_{x \to b^{-}} F(x) - \lim_{x \to a^{+}} F(x) = H(b) - H(a)\\
  &= \sum_{i = 1}^{n} H(x_i) - H(x_{i-1}) = \sum_{i = 1}^{n} f(t_i)(x_i -
  x_{i-1}).
 \end{align*}
 Odtud ihned plyne odhad
 \[
  \left| (\mathcal{N})\int_{a}^{b} f - (\mathcal{R})\int_{a}^{b} f\right| =
  \left| \sum_{i = 1}^{n} f(t_i)(x_i - x_{i-1}) - (\mathcal{R}) \int_{a}^{b} f
  \right| < \varepsilon.
 \]
 Jelikož $\varepsilon>0$ bylo voleno libovolně, znamená nerovnost výše, že
 \[
  (\mathcal{N}) \int_{a}^{b} f = (\mathcal{R}) \int_{a}^{b} f,
 \]
 jak bylo jest dokázati.
\end{thmproof}

