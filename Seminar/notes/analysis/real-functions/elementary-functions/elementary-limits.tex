\section{Limity elementárních funkcí}
\label{sec:limity-elementarnich-funkci}

Tato sekce je věnována výpočtu limit, ve kterých figurují elementární funkce.
Nejtěžšími (ale zároveň nejužitečnějšími) úlohami na vyřešení jsou limity
racionálních kombinací (tj. součtů, násobků a především podílů) elementárních
funkcí. Samotné řešení pak obvykle zahrnuje převod výrazu do tvaru, v němž lze
naň uplatnit jisté \uv{známé} limity, případně použít
\hyperref[thm:lhospitalovo-pravidlo]{l'Hospitalova pravidla}.

\begin{proposition}{Běžné limity}{bezne-limity}
 Platí
 \begingroup
 \addtolength{\jot}{3mm}
 \begin{align}
  \lim_{x \to 0} \frac{\sin x}{x} &= 1,\tag{a}\\
  \lim_{x \to 0} \frac{1-\cos x}{x^2} &= \frac{1}{2},\tag{b}\\
  \lim_{x \to 0} \frac{\exp x - 1}{x} &= 1,\tag{c}\\
  \lim_{x \to 0} \frac{\log(1+x)}{x} &= 1.\tag{d}
 \end{align}
 \endgroup
\end{proposition}
\begin{propproof}
 K výpočtu všech limit lze použít
 \hyperref[thm:lhospitalovo-pravidlo]{l'Hospitalova pravidla}. Máme
 \begingroup
 \addtolength{\jot}{3mm}
 \begin{align}
  \lim_{x \to 0} \frac{\sin x}{x} &= \lim_{x \to 0} \frac{\cos x}{1} = \cos(0) =
  1,\tag{a}\\
  \lim_{x \to 0} \frac{1-\cos x}{x^2} &= \lim_{x \to 0} \frac{\sin x}{2x} =
  \frac{1}{2} \lim_{x \to 0} \frac{\sin x}{x} \overset{\text{(a)}}{=}
  \frac{1}{2},\tag{b}\\
  \lim_{x \to 0} \frac{\exp x - 1}{x} &= \lim_{x \to 0} \frac{\exp x}{1} =
  \exp(0) = 1,\tag{c}\\
  \lim_{x \to 0} \frac{\log(1+x)}{x} &= \lim_{x \to 0} \frac{\frac{1}{1+x}}{1} =
  1.\tag{d}
 \end{align}
 \endgroup
 Tím je důkaz hotov.
\end{propproof}

Aplikujeme nyní \myref{tvrzení}{prop:bezne-limity} k výpočtu některých limit
kombinací elementárních funkcí.

\begin{problem}{}{elementarni-limity-1}
 Spočtěte
 \[
  \lim_{x \to 0} \frac{\cos(\sin x) - 1}{\log(\sqrt{1 + x^2})}.
 \]
\end{problem}
\begin{probsol}
 Zřejmě platí
 \[
  \lim_{x \to 0} \cos(\sin x) - 1 = \lim_{x \to 0} \log(\sqrt{1 + x^2}) = 0,
 \]
 je tedy třeba výraz nejprve upravit.
 \hyperref[thm:lhospitalovo-pravidlo]{l'Hospitalovo pravidlo} zde pravděpodobně
 není vhodným prostředkem, neboť derivace obou funkcí jsou značně komplikované.
 Učiníme nejprve úpravu
 \[
  \frac{\cos(\sin x) - 1}{\log(\sqrt{1 + x^2})} = \frac{\cos(\sin x) -
  1}{\sin^2 x} \cdot \frac{\sin^2 x}{\log(\sqrt{1 + x^2})}.
 \]
 Protože $\lim_{x \to 0} \sin x / x = 1$, platí z
 \hyperref[thm:limita-slozene-funkce]{věty o limitě složené funkce}
 \[
  \lim_{x \to 0} \frac{\cos(\sin x) - 1}{\sin^2 x} = \lim_{y \to 0} \frac{\cos y
  - 1}{y^2} = -\frac{1}{2}.
 \]
 Dále máme (z \hyperref[prop:vlastnosti-logaritmu]{vlastností logaritmu})
 \[
  \frac{\sin^2 x}{\log(\sqrt{1 + x^2})} = \frac{\sin^2 x}{x^2} \cdot
  \frac{x^2}{\frac{1}{2}\log(1 + x^2)} = 2 \cdot \frac{\sin x}{x} \cdot
  \frac{\sin x}{x} \cdot \frac{x^2}{\log(1 + x^2)}.
 \]
 Opět z \hyperref[thm:limita-slozene-funkce]{věty o limitě složené funkce} jest
 \[
  \lim_{x \to 0} \frac{x^2}{\log(1 + x^2)} = \lim_{y \to 0} \frac{y}{\log(1 +
  y)} = 1.
 \]
 Celkem tedy,
 \[
  \lim_{x \to 0} \frac{\sin^2 x}{\log(\sqrt{1 + x^2})} = 2 \cdot \lim_{x \to 0}
  \frac{\sin x}{x} \cdot \lim_{x \to 0} \frac{\sin x}{x} \cdot \lim_{x \to 0}
  \frac{x^2}{\log(1 + x^2)} = 2 \cdot 1 \cdot 1 \cdot 1 = 2.
 \]
 Spolu s předchozím výpočtem dostaneme
 \[
  \lim_{x \to 0} \frac{\cos(\sin x) - 1}{\log(\sqrt{1 + x^2})} = -\frac{1}{2}
  \cdot 2 = -1.
 \]
\end{probsol}

Před další úlohou spočteme další obecně užitečnou limitu.
\begin{lemma}{}{log-podle-x}
 Pro každá $\alpha>0,\beta>0$ platí
 \[
  \lim_{x \to \infty} \frac{\log^{\alpha}x}{x^{\beta}} = 0.
 \]
\end{lemma}
\begin{lemproof}
 Protože $\lim_{x \to \infty} x^{\beta} = \infty$, použijeme
 \hyperref[thm:lhospitalovo-pravidlo]{l'Hospitalovo pravidlo}. Dostaneme
 \[
  \lim_{x \to \infty} \frac{\log^{\alpha} x}{x^{\beta}} = \frac{\alpha}{\beta}
  \cdot \frac{\log^{\alpha - 1}x \cdot \frac{1}{x}}{x^{\beta - 1}} =
  \frac{\alpha}{\beta} \cdot \frac{\log^{\alpha-1} x}{x^{\beta}}.
 \]
 Snadno nahlédneme, že se každým použitím
 \hyperref[thm:lhospitalovo-pravidlo]{l'Hospitalova pravidla} mocnina u $\log$
 sníží o $1$ (dokud je větší než $0$) a mocnina u $x$ zůstane stejná. Nalezneme
 tedy $n \in \N$ takové, že $\alpha - n \in (-1,0]$. Iterovaným použitím
 \hyperref[thm:lhospitalovo-pravidlo]{l'Hospitalova pravidla} upravíme původní
 limitu na
 \[
  \lim_{x \to \infty} \frac{\prod_{i=0}^{n-1} \alpha - i}{\beta^{n}} \cdot
  \frac{\log^{\alpha - n} x}{x^{\beta}} = \frac{\prod_{i=0}^{n-1} \alpha -
  i}{\beta^{n}} \cdot \lim_{x \to \infty} \frac{1}{\log^{n-\alpha}x \cdot
  x^{\beta}}.
 \]
 Jelikož $n-\alpha \geq 0$ a $\beta>0$, dostáváme, že
 \[
  \lim_{x \to \infty} \log^{n-\alpha}x \cdot x^{\beta} = \lim_{x \to \infty}
  \log^{n-\alpha}x \cdot \lim_{x \to \infty} x^{\beta} = \infty,
 \]
 z čehož již plyne dokazovaná rovnost.
\end{lemproof}

\begin{problem}{}{elementarni-limity-2}
 Spočtěte
 \[
  \lim_{x \to \infty} \sqrt{\log \left( 1 + \frac{3}{x} \right)} \cdot
  \log^2(1+x^3).
 \]
\end{problem}
\begin{probsol}
 Upravíme
 \[
  \sqrt{\log \left( 1 + \frac{3}{x} \right)} \cdot \log^2(1+x^3) =
  \frac{\sqrt{\log \left( 1 + \frac{3}{x} \right)}}{\sqrt{\frac{3}{x}}} \cdot
  \sqrt{\frac{3}{x}} \cdot \log^2(1 + x^3).
 \]
 Jelikož
 \[
  \frac{\sqrt{\log \left( 1 + \frac{3}{x} \right)}}{\sqrt{\frac{3}{x}}} =
  \sqrt{\frac{\log \left(1 + \frac{3}{x}\right)}{\frac{3}{x}}}
 \]
 a $\lim_{x \to \infty} 3 / x = 0$, plyne z \hyperref[prop:bezne-limity]{běžných
 limit} a z \hyperref[thm:limita-slozene-funkce]{věty o limitě složené funkce},
 že
 \[
  \lim_{x \to \infty} \frac{\sqrt{\log \left( 1 + \frac{3}{x}
  \right)}}{\sqrt{\frac{3}{x}}} = \sqrt{1} = 1.
 \]
 Dále,
 \[
  \sqrt{\frac{3}{x}} \cdot \log^2(1 + x^3) = \frac{\log^2(1 +
  x^3)}{\sqrt{\frac{x}{3}}} = \frac{\log^2(1 + x^3)}{\sqrt[6]{1+x^3}} \cdot
  \frac{\sqrt[6]{1+x^3}}{\sqrt{\frac{x}{3}}}.
 \]
 Podle \hyperref[lem:log-podle-x]{předchozího lemmatu} (a
 \hyperref[thm:limita-slozene-funkce]{věty o limitě složené funkce}) platí
 \[
  \lim_{x \to \infty} \frac{\log^2(1 + x^3)}{\sqrt[6]{1+x^3}} = 0.
 \]
 Jelikož
 \[
  \lim_{x \to \infty} \frac{\sqrt[6]{1+x^3}}{\sqrt{\frac{x}{3}}} = \lim_{x \to
  \infty} \frac{\sqrt{x} \sqrt[6]{\frac{1}{x^3} +
  1}}{\sqrt{x}\sqrt{\frac{1}{3}}} = \sqrt{3},
 \]
 dostáváme celkem, že
 \[
  \lim_{x \to \infty} \sqrt{\log \left( 1+\frac{3}{x} \right)} \cdot \log^2(1 +
  x^3) = 1 \cdot 0 \cdot \sqrt{3} = 0.
 \]
\end{probsol}

\begin{problem}{}{elementarni-limity-3}
 Spočtěte
 \[
  \lim_{x \to 0^{+}} \left( \frac{4^{x} + 5^{x} + 6^{x}}{3}
  \right)^{\frac{1}{x}}.
 \]
\end{problem}
\begin{probsol}
 Z definice obecné mocniny máme
 \[
  \left( \frac{4^{x} + 5^{x} + 6^{x}}{3} \right)^{\frac{1}{x}} = \exp \left(
  \frac{1}{x} \cdot \log \left( \frac{4^{x} + 5^{x} + 6^{x}}{3} \right)
 \right).
 \]
 Spočteme
 \[
  \lim_{x \to 0^{+}} \frac{1}{x} \cdot \log \left( \frac{4^{x} + 5^{x} +
  6^{x}}{3} \right)
 \]
 a výslednou limitu dostaneme přes \hyperref[thm:limita-slozene-funkce]{větu o
 limitě složené funkce}.

 Jelikož
 \[
  \lim_{x \to 0^{+}} \log \left( \frac{4^{x} + 5^{x} + 6^{x}}{3} \right) =
  \lim_{x \to 0^{+}} x = 0,
 \]
 lze použít \hyperref[thm:lhospitalovo-pravidlo]{l'Hospitalovo pravidlo}. Podle
 něj a \hyperref[prop:vlastnosti-obecne-mocniny]{vlastností obecné mocniny}
 platí
 \begin{align*}
  \lim_{x \to 0^{+}} \frac{\log \left( \frac{4^{x} + 5^{x} + 6^{x}}{3}
  \right)}{x} &= \lim_{x \to 0^{+}} \frac{3}{4^{x} + 5^{x} + 6^{x}} \cdot
  \frac{4^{x} \log 4 + 5^{x} \log 5 + 6^{x} \log 6}{3}\\
              &= \frac{3}{1 + 1 + 1} \cdot \frac{\log 4 + \log 5 + \log 6}{3} =
              \frac{\log 120}{3}.
 \end{align*}
 Nyní můžeme dopočíst
 \begin{align*}
  \lim_{x \to 0^{+}} \exp \left( \frac{1}{x} \cdot \log \left( \frac{4^{x} +
   5^{x} + 6^{x}}{3} \right) \right) &= \exp \left( \lim_{x \to 0^{+}}
 \frac{1}{x} \cdot \log \left( \frac{4^{x} + 5^{x} + 6^{x}}{3} \right) \right)\\
                                     &= \exp \left( \frac{\log 120}{3} \right) =
                                     120 - \exp 3.
 \end{align*}
\end{probsol}

\begin{exercise}{Pár limit elementárních funkcí}{par-limit-elementarnich-funkci}
 Spočtěte následující limity.
 {\everymath={\displaystyle}
  \arraycolsep=1em
  \[
   \begin{array}{lll}
    \lim_{x \to 3 \pi / 2} (4x^2 - 9\pi^2) \frac{\cos x}{1 + \sin x} & \lim_{x
    \to 0} \frac{1 - \cos(\arctan x)}{x^2} & \lim_{x \to \infty} \arcsin \left(
    \frac{1-x}{1+x}\right)\\[2em]
    \lim_{x \to 0^{+}} \frac{\exp \left( \frac{\sin x}{2} \right) - \cos
    (\sqrt{x})}{\log^2(1 + \sqrt{x})} & \lim_{x \to 1^{-}} \frac{\sqrt{\exp 2 -
   \exp 2x}}{\arccos x} & \lim_{x \to 0} \frac{\sqrt{x \sin x}}{\exp(x^2) -
   1}\\[2em]
   \lim_{x \to 0^{+}} \left( 1 - \sqrt{\arcsin x} \right)^{\frac{1}{\sqrt[4]{1 -
 \cos x}}} & \lim_{x \to \infty} \left( \frac{x+2}{2x-1} \right)^{x^2} & \lim_{x
 \to 0} (1 + x^2)^{\cot^2 x}
   \end{array}
  \]
 }
\end{exercise}
