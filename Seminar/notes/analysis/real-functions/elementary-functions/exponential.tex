\section{Exponenciála a logaritmus}
\label{sec:exponenciala-a-logaritmus}

První na seznamu je \emph{exponenciála} -- funkce spojitého růstu. Toto
pojmenování ještě níže odůvodníme. Nyní přikročíme k definici. Pro stručnost
zápisu, budeme v následujícím textu používat konvenci, že $0^{0} = 1$.

\begin{definition}{Exponenciála}{exponenciala}
 Pro $x \in \R$ definujeme
 \[
  \exp x \coloneqq \sum_{n=0}^{\infty} \frac{x^{n}}{n!}.
 \]
\end{definition}

Jak jsme čtenáře vystříhali, musíme nyní na krátkou chvíli odbočit k číselným
řadám, abychom uměli v obec dokázat, že právě definovaná
\hyperref[def:exponenciala]{exponenciála} je skutečně reálnou funkcí.

\begin{definition}{Absolutní konvergence řady}{absolutni-konvergence-rady}
 Ať $\sum_{n=0}^{\infty} a_n$ je číselná řada, kde $a_n \in \R$. Řekneme, že
 $\sum_{n=0}^{\infty} a_n$ \emph{absolutně konverguje}, když konverguje řada
 $\sum_{n=0}^{\infty} |a_n|$.
\end{definition}

\begin{lemma}{}{absolutni-konvergence-a-konvergence}
 Každá absolutně konvergentní řada je konvergentní.
\end{lemma}
\begin{lemproof}
 Ať je $\varepsilon>0$ dáno. Předpokládejme, že $\sum_{n=0}^{\infty} |a_n|$
 konverguje. Nalezneme $n_0 \in \N$ takové, že pro $m \geq n \geq n_0$ platí
 \[
  \left|\sum_{k=n}^m |a_k| \right| = \sum_{k=n}^m |a_k|<\varepsilon.
 \]
 Potom ale z \hyperref[lem:trojuhelnikova-nerovnost]{trojúhelníkové nerovnosti}
 platí
 \[
  \left| \sum_{k=n}^m a_k \right| \leq \sum_{k=n}^m |a_k|<\varepsilon,
 \]
 čili $\sum_{n=0}^{\infty} a_n$ konverguje.
\end{lemproof}

\begin{definition}{Cauchyho součin řad}{cauchyho-soucin-rad}
 Ať $\sum_{n=0}^{\infty} a_n$ a $\sum_{n=0}^{\infty} b_n$ jsou číselné řady.
 Jejich \emph{Cauchyho součinem} myslíme číselnou řadu
 \[
  \sum_{n=0}^{\infty} \sum_{k=0}^n a_{n-k}b_k.
 \]
\end{definition}
\begin{theorem}{Mertensova}{mertensova}
 Ať $\sum_{n=0}^{\infty} a_n, \sum_{n=0}^{\infty} b_n$ jsou \emph{konvergentní}
 číselné řady, přičemž $\sum_{n=0}^{\infty} a_n$ je navíc absolutně
 konvergentní. Potom $\sum_{n=0}^{\infty} \sum_{k=0}^n a_{n-k}b_k$ konverguje a
 platí
 \[
  \left( \sum_{n=0}^{\infty} a_n \right) \left( \sum_{n=0}^{\infty} b_n \right)
  = \sum_{n=0}^{\infty} \sum_{k=0}^n a_{n-k}b_k.
 \]
\end{theorem}

\begin{theorem}{Vlastnosti exponenciály}{vlastnosti-exponencialy}
 Funkce $\exp$ je dobře definována a platí
  \begin{enumerate}[label=(E\arabic*)]
   \item $\exp(x + y) = \exp x \cdot \exp y$;
   \item $\lim_{x \to 0} \frac{\exp x - 1}{x} = 0$.
  \end{enumerate}
\end{theorem}
\begin{thmproof}
 \emph{Dobrá definovanost} zde znamená, že řada $\sum_{n=0}^{\infty} x_n / n!$
 konverguje pro každé $x \in \R$. Ukážeme, že konverguje absolutně. Je-li $x =
 0$, pak řada konverguje zřejmě. Volme tedy $x \in \R \setminus \{0\}$. Potom
 \[
  \lim_{n \to \infty} \frac{\left| \frac{x^{n+1}}{(n+1)!} \right|}{\left|
  \frac{x^{n}}{n!} \right|} = \lim_{n \to \infty} \frac{|x|}{n+1} = 0,
 \]
 čili podle \myref{věty}{thm:dalembertovo-podilove-kriterium} řada
 $\sum_{n=0}^{\infty} |x^{n}|/n!$ konverguje, což znamená, že konverguje i
 $\sum_{n=0}^{\infty} x^{n}/n!$.

 Dokážeme vlastnost (E1). Počítáme
 \begin{align*}
  \exp(x+y) &= \sum_{n=0}^{\infty} \frac{(x+y)^{n}}{n!} = \sum_{n=0}^{\infty}
  \sum_{k=0}^n \binom{n}{k} \frac{x^{n-k}y^{k}}{n!} = \sum_{n=0}^{\infty}
  \sum_{k=0}^n \frac{n!}{(n-k)!k!} \frac{x^{n-k}y^{k}}{n!}\\
            &= \sum_{n=0}^{\infty} \sum_{k=0}^{n} \frac{x^{n-k}}{(n-k)!}
            \frac{y^{k}}{k!}.
 \end{align*}
 Všimněme si, že poslední řada je \hyperref[def:cauchyho-soucin-rad]{Cauchyho
 součinem} řad $\sum_{n=0}^{\infty} x^{n} / n!$ a $\sum_{n=0}^{\infty} y^{n} /
 n!$. Protože jsou obě tyto řady (podle výše dokázaného) absolutně konvergentní,
 platí z \hyperref[thm:mertensova]{Mertensovy věty}
 \[
  \exp(x+y) = \sum_{n=0}^{\infty} \sum_{k=0}^n
  \frac{x^{n-k}}{(n-k)!}\frac{y^{k}}{k!} = \left( \sum_{n=0}^{\infty}
  \frac{x^{n}}{n!} \right) \left( \sum_{n=0}^{\infty} \frac{y^{n}}{n!} \right) =
  \exp x \cdot \exp y.
 \]

 Nyní vlastnost (E2). Pro $x \in (-1,1)$ odhadujme
 \begin{align*}
  \left| \frac{\exp x - 1}{x} - 1 \right| &= \left| \frac{\exp x - 1 - x}{x}
  \right| = \frac{1}{|x|} \left| \sum_{n=0}^{\infty}\frac{x^{n}}{n!} \clr{- x -
  1} \right| = \frac{1}{|x|} \left| \sum_{\clr{n=2}}^{\infty} \frac{x^{n}}{n!}
  \right| \\
                                          &= |x| \left| \sum_{n=2}^{\infty}
                                          \frac{x^{n-2}}{n!} \right| \leq |x|
                                          \left| \sum_{n=2}^{\infty}
                                          \frac{1}{n!} \right| = c \cdot |x|,
 \end{align*}
 kde $c > 0$ je hodnota součtu řady $\sum_{n=0}^{\infty} 1 / n!$, která zjevně
 konverguje (například díky nerovnosti $1 / n! \leq 1 / n^2$). Jelikož $\lim_{x
 \to 0} c \cdot |x| = 0$, plyne odtud ihned, že
 \[
  \lim_{x \to 0} \left| \frac{\exp x - 1}{x} -1\right| = 0,
 \]
 z čehož zase
 \[
  \lim_{x \to 0} \frac{\exp x - 1}{x} = 1.
 \]
 Tím je důkaz završen.
\end{thmproof}

Ihned si odvodíme další vlastnosti exponenciály plynoucí z (E1) a (E2). Postupně
dokážeme, že pro každé $x \in \R$ platí následující.
\begin{enumerate}[label=(E\arabic*)]
 \setcounter{enumi}{2}
 \item $\exp 0 = 1$;
 \item $\exp' x = \exp x$;
 \item $\exp(-x) = 1 / \exp(x)$;
 \item $\exp x > 0$;
 \item $\exp$ je spojitá na $\R$;
 \item $\exp$ je rostoucí na $\R$;
 \item $\lim_{x \to \infty} \exp x = \infty$ a $\lim_{x \to -\infty} \exp x =
  0$;
 \item $\img \exp = (0,\infty)$.
\end{enumerate}

Z (E1) platí $\exp(0 + x) = \exp 0 \cdot \exp x$. Protože zřejmě existuje $x \in
\R$, pro něž $\exp x \neq 0$, plyne odtud $\exp 0 = 1$, tj. vlastnost (E3).

Pro důkaz (E4) počítáme
\begin{align*}
 \lim_{h \to 0} \frac{\exp (x + h) - \exp h}{h} &\clr{~=~} \lim_{h \to 0}
 \frac{\exp h \cdot \exp x - \exp h}{h} = \lim_{h \to 0} \frac{(\exp h - 1)\exp
 x}{h}\\
                                                &= \exp x \cdot \lim_{h \to 0}
                                                \frac{\exp h - 1}{h} \clb{~=~}
                                                \exp x \cdot 1 = \exp x,
\end{align*}
kde jsme v \clr{červené} rovnosti použili vlastnost (E1) a v \clb{modré} zas
vlastnost (E2).

Pokračujeme vlastností (E5). Z (E1) máme
\[
 \exp(x + (-x)) = \exp x \cdot \exp(-x).
\]
Protože z (E3) je $\exp(x + (-x)) = \exp 0 = 1$, dostáváme
\[
 1 = \exp x \cdot \exp(-x),
\]
čili
\[
 \exp(-x) = \frac{1}{\exp x}.
\]

Ježto má řada $\sum_{n=0}^{\infty} x^{n} / n!$ zjevně kladný součet pro $x > 0$,
plyne (E6) přímo z právě dokázané (E5).

Vlastnost (E7) je okamžitým důsledkem vlastnosti (E4), díky níž má $\exp$
konečnou derivaci na $\R$, a tudíž je podle
\myref{lemmatu}{lem:vztah-derivace-a-spojitosti} tamže spojitá.

Vlastnost (E8) je důsledkem vlastností (E4) a (E6), neboť funkce majíc na
intervalu (v tomto případě celém $\R$) kladnou derivaci, je na tomto intervalu
-- podle \myref{důsledku}{cor:vztah-derivace-a-monotonie} -- rostoucí.

Platí $\exp 1 = \sum_{n=0}^{\infty} \frac{1}{n!} = 1 + \sum_{n=1}^{\infty}
\frac{1}{n!} > 1$, čili z vlastnosti (E1) plyne, že $\exp$ není shora omezená,
neboť $\exp(x+1) = \exp x \cdot \exp 1 > \exp x$ pro každé $x \in \R$. To spolu
s vlastnostmi (E7) a (E8) dává $\lim_{x \to \infty} \exp x = \infty$. Dále,
použitím (E5),
\[
 \lim_{x \to -\infty} \exp x = \lim_{x \to \infty} \exp(-x) = \lim_{x \to
 \infty} \frac{1}{\exp x} = 0,
\]
což dokazuje (E9).

Konečně, vlastnost (E10) plyne z (E9) a \hyperref[thm:bolzanova]{Bolzanovy
věty}.

\begin{example}{}{exponenciala-populace}
 V úvodu do této sekce jsme nazvali exponenciálu \uv{funkcí spojitého růstu}.
 Tuto intuici nyní částečně formalizujeme. Dalšího pohledu nabudeme po definici
 obecné mocniny.
 % TODO odkaz

 Uvažme následující přímočarý populační model. V čase $t \in (0,\infty)$ je
 počet jedinců dán funkcí $P(t)$. Množství nově narozených jedinců závisí pouze
 na počtu právě žijících a na konstantě $r \in [0,\infty)$ -- zvané
 \emph{reproduction rate} -- která značí, kolik nových jedinců se narodí za
 jednoho právě živého. Vnímáme-li derivaci $P'(t)$ jako \emph{rychlost růstu}
 populace v čase $t$, pak dostáváme diferenciální rovnici
 \[
  P'(t) = r \cdot P(t),
 \]
 jelikož v čase $t$ se podle našeho modelu narodí $r$ jedinců za každého živého.
 Díky vlastnosti (E4) vidíme, že například funkce $P(t) = \exp(rt)$ řeší rovnici
 výše. Teorii diferenciálních rovnic v tomto textu probírat nebudeme, bez důkazu
 však zmíníme, že řešení takto triviálních rovnic až na konstantu určena
 jednoznačně. V tomto jednoduchém populačním modelu je tudíž počet živých
 jedinců v čase $t$ dán funkcí $t \mapsto \exp(rt).$
\end{example}

Na závěr si dokážeme jeden možná překvapivý fakt, že vlastnosti (E1) a (E2) již
určují funkci $\exp$ jednoznačně.

\begin{theorem}{Jednoznačnost exponenciály}{jednoznacnost-exponencialy}
 Existuje právě jedna funkce definovaná na celém $\R$ splňující (E1) a (E2).
\end{theorem}
\begin{thmproof}
 Existenci jsme dokázali konstruktivně. Dokážeme jednoznačnost. Ať $f:\R \to \R$
 splňuje (E1) a (E2). Ukážeme, že $f = \exp$.

 Z úvah výše plyne, že $f$ splňuje rovněž vlastnosti (E3) - (E10), protože k
 jejich důkazu byly použity pouze (E1) a (E2). Platí tedy $f(0) = 1$ a $f'(x) =
 f(x)$. Jelikož $\exp x \neq 0$ pro všechna $x \in \R$, máme z
 \hyperref[thm:aritmetika-derivaci]{věty o aritmetice derivací}
 \[
  \left( \frac{f}{\exp} \right)'(x) = \frac{f'(x) \exp x - f(x) \exp'x}{\exp^2
  x} = \frac{f(x) \exp x - f(x) \exp x}{\exp^2 x} = 0.
 \]
 Podle \myref{cvičení}{exer:derivace-nula-konstantni} je $f / \exp$ konstatní
 funkce, čili existuje $c \in \R$ takové, že $f(x) / \exp x = c$ pro každé
 $c \in \R$. Dosazením $x = 0$ zjistíme, že $c = f(0) / \exp 0 = 1$, čili $c =
 1$ a $f = \exp$.
\end{thmproof}

\subsection{Logaritmus}
\label{ssec:logaritmus}

Jelikož je funkce $\exp$ spojitá a rostoucí na $\R$, má na celém $\R$ inverzní
funkci, které přezdíváme \emph{logaritmus} a značíme ji $\log$. Z vlastností
$\exp$ ihned plyne, že $\log$ je reálná funkce $(0,\infty) \to \R$. Na rozdíl od
$\exp$ však $\log$ není dána číselnou řadou -- aspoň ne pro všechna $x \in
(0,\infty)$, více v kapitole o Taylorově polynomu.
%TODO odkaz

Vlastnosti exponenciály nám rovnou umožňují do značné míry prozkoumat k ní
inverzní funkci.

\begin{proposition}{Vlastnosti logaritmu}{vlastnosti-logaritmu}
 Pro každá $x,y \in (0,\infty)$ platí
 \begin{enumerate}[label=(L\arabic*)]
  \item $\log$ je spojitá a rostoucí na $(0,\infty)$;
  \item $\log(xy) = \log x + \log y$;
  \item $\log'x = 1 / x$;
  \item $\lim_{x \to 0^{+}} \log x = -\infty$ a $\lim_{x \to \infty} \log x =
   \infty$.
 \end{enumerate}
\end{proposition}
\begin{propproof}
 \hspace*{\fill}
 \begin{enumerate}[label=(L\arabic*)]
  \item Plyne ihned z faktu, že $\exp$ je spojitá a rostoucí.
  \item Užitím vlastností exponenciály spočteme
  \[
   xy = \exp(\log x) \cdot \exp(\log y) = \exp(\log x + \log y),
  \]
  z čehož po aplikace $\log$ na obě strany rovnosti plyne
  \[
   \log(xy) = \log x + \log y.
  \]
 \item Podle \hyperref[thm:derivace-inverzni-funkce]{věty o derivaci inverzní
  funkce} platí
  \[
   \log'x = \frac{1}{\exp'(\log x)} = \frac{1}{\exp(\log x)} = \frac{1}{x}.
  \]
 \item Protože $\img \log = \R$, není $\log$ zdola ani shora omezená. Z (L1)
  plyne kýžený závěr.
 \end{enumerate}
\end{propproof}

\subsection{Obecná mocnina}
\label{ssec:obecna-mocnina}

Užitím funkcí $\log$ a $\exp$ definujeme pro $a \in (0,\infty)$ a $b \in \R$
výraz $a^{b}$ předpisem
\[
 a^{b} \coloneqq \exp(b \cdot \log a).
\]
Stojí za to věnovat krátkou chvíli ověření, že tato funkce odpovídá naší
představě \emph{mocniny} v případě, kdy $b = n \in \N$. Máme
\[
 a^{n} = \exp(n \cdot \log a) = \exp \left( \sum_{k=1}^n \log a \right) =
 \prod_{k=1}^n \exp(\log(a)) = \prod_{k=1}^n a,
\]
tedy $a^{n}$ je vskutku $a$ $n$-krát vynásobené samo sebou.

\begin{warning}{}{obecna-mocnina}
 Uvědomme si, že $a^{b}$ je definováno pouze pro $a \in (0,\infty)$. Pro $a \leq
 0$ není tato funkce nad reálnými čísly rozumně definovatelná. Důvod je mimo
 jiné následující: pro $n$ sudé a $a < 0$ je $a^{n} > 0$, ale $a^{n+1} < 0$.
 Tedy, měla-li by $a^{b}$ být spojitá funkce, pak by pro každé $n \in \N$ a $a <
 0$ existovalo $\xi \in (n,n+1)$ takové, že $a^{\xi} = 0$. Mocninná funkce, jež
 je nulová pro nekonečně mnoho čísel je i pro matematiky zřejmě příliš divoká
 představa.

 Poznamenejme však, že nad komplexními čísly je funkce $\log$ definována i pro
 záporná reálná čísla, tedy $a^{b}$ dává -- se stejnou definicí -- smysl pro
 všechna $a,b \in \C$.
\end{warning}

Z vlastností $\log$ a $\exp$ plynou ihned vlastnosti obecné mocniny. Jelikož
její definice dává vzniknout \textbf{dvěma} reálným funkcím, konkrétně
\[
 f(x) = a^{x} \text{ pro } x \in \R \quad \text{a} \quad g(x) = x^{b} \text{
 pro } x \in (0,\infty),
\]
musíme tyto při zkoumání vlastností obecné mocniny pochopitelně rozlišovat. Aby
nedošlo ke zmatení, budeme tyto funkce značit zkrátka jako $x \mapsto a^{x}$ a
$x \mapsto x^{b}$, kde $a \in (0,\infty)$ a $b \in \R$ jsou fixní.

\begin{proposition}{Vlastnosti obecné mocniny}{vlastnosti-obecne-mocniny}
 Pro všechna $a \in (0,\infty), b \in \R$ platí
 \begin{enumerate}[label=(O\arabic*)]
  \item Funkce $x \mapsto a^{x}$ i $x \mapsto x^{b}$ jsou spojité na svých
   doménách.
  \item Funkce $x \mapsto a^{x}$ je na celém $\R$ 
   \begin{itemize}
    \item rostoucí pro $a > 1$,
    \item konstantní pro $a = 1$,
    \item klesající pro $a < 1$.
   \end{itemize}
  \item Funkce $x \mapsto x^{b}$ je na $(0,\infty)$
   \begin{itemize}
    \item rostoucí pro $b > 0$,
    \item konstantní pro $b = 0$,
    \item klesající pro $b < 0$.
   \end{itemize}
  \item $(x \mapsto a^{x})' = (x \mapsto a^{x} \log a)$.
  \item $(x \mapsto x^{b})' = (x \mapsto bx^{b-1})$.
  \item Je-li
   \begin{itemize}
    \item $a > 1$, pak $\lim_{x \to \infty} a^{x} = \infty$ a $\lim_{x \to
     -\infty} a^{x} = 0$;
    \item $a < 1$, pak $\lim_{x \to \infty} a^{x} = 0$ a $\lim_{x \to -\infty}
     a^{x} = \infty$;
   \end{itemize}
  \item Je-li 
   \begin{itemize}
    \item $b > 0$, pak $\lim_{x \to 0^{+}} x^{b} = 0$ a $\lim_{x \to \infty}
     x^{b} = \infty$.
    \item $b < 0$, pak $\lim_{x \to 0^{+}} x^{b} = \infty$ a $\lim_{x \to
     \infty} x^{b} = 0$.
   \end{itemize}
  \item $\log(a^{b}) = b \cdot \log a$.
 \end{enumerate}
\end{proposition}
\begin{propproof}
 Vlastnosti (O2), (O3), (O6) a (O7) dokážeme pouze v případě, že $a > 1$ a $b >
 0$. Důkaz tvrzení v případech $a < 1$ a $b < 0$ plyne ihned z rovnosti
 $\exp(-x) = 1 / \exp x$ pro $x \in \R$.
  
 \begin{enumerate}[label=(O\arabic*)]
  \item Jelikož $a^{b} = \exp(b \log a)$ plyne spojitost obou funkcí ze
   spojitosti $\exp$ a $\log$.
  \item Platí $a^{x} = \exp(x \log a)$. Protože $a > 1$, je $\log a > 0$. Funkce
   $\exp$ je rostoucí, a tedy je rostoucí rovněž $a \mapsto a^{x}$.
  \item Máme $x^{b} = \exp(b \log x)$. Jelikož je $b$ z předpokladu kladné a
   $\log$ je rostoucí, je $x \mapsto x^{b}$ též rostoucí.
  \item Počítáme
   \[
    (a^{x})' = (\exp(x \log a))' = \exp'(x \log a) \cdot (x \log a)' = \exp(x
    \log a) \cdot \log a = a^{x} \log a.
   \]
  \item Opět počítáme
   \[
    (x^{b})' = (\exp(b\log x))' = \exp'(b \log x) \cdot (b\log x)' = \exp(b \log
    x) \cdot \frac{b}{x} = \frac{bx^{b}}{x} = bx^{b-1}.
   \]
  \item Podobně jako v důkazu (O2) plyne z $a > 1$, že $\log a > 0$. Potom jsou
   ale limity v $ \pm \infty$ funkce $a^{x} = \exp(x \log a)$ stejné jako tytéž
   limity funkce $\exp x$. Odtud tvrzení.
  \item Jelikož je $b > 0$, máme z \hyperref[thm:limita-slozene-funkce]{věty o
   limitě složené funkce} 
   \begin{align*}
    \lim_{x \to 0^{+}} x^{b} &= \lim_{x \to 0^{+}} \exp(b \cdot \log x) =
    \lim_{y \to -\infty} \exp(b \cdot y) = 0,\\
     \lim_{x \to \infty} x^{b} &= \lim_{x \to \infty} \exp(b \cdot \log x) =
     \lim_{y \to \infty} \exp(b \cdot y) = \infty.
   \end{align*}
  \item Jest
   \[
    \log(a^{b}) = \log(\exp(b \cdot \log a)) = b \cdot \log a.
   \]
 \end{enumerate}
\end{propproof}

\begin{example}{Exponenciála jako funkce růstu podruhé}{exp-rust-podruhe}
 V \myref{příkladě}{exam:exponenciala-populace} jsme slíbili čtenářům ještě
 jeden pohled na exponenciálu jako na funkci \uv{spojitého} růstu. Uvažme
 následující obvyklý finanční model.

 Naším prvotním vkladem na spořící účet je částka $P > 0$. Ve smlouvě k účtu je
 uvedeno, že z něj nesmíme vybírat po dobu pěti let za roční úrokové sazby 5 \%.
 Tedy, každý rok se právě uložená částka na účtu zvýší o přesně 5 \%. Když
 chceme vypočítat, kolik budeme mít na účtu za oněch pět let, stačí provést
 snadný výpočet
 \[
  P \cdot 1.05 \cdot 1.05 \cdot 1.05 \cdot 1.05 \cdot 1.05 = P \cdot 1.05^{5}.
 \]
 V zájmu rozšíření tohoto příkladu si zapíšeme tentýž výsledek jako
 \[
  P \cdot \left( 1 + 0.05 \right)^{5}.
 \]
 
 Tento model však předpokládá úročení uložené částky jednou ročně. Když bude
 však úročení probíhat například měsíčně, pak se roční úroková sazba samozřejmě
 rozdělí dvanácti, avšak výpočet úročení je rovněž třeba provést dvanáctkrát do
 roka. Finální částkou po pěti letech bude tudíž
 \[
  P \cdot \left(\left( 1 + \frac{0.05}{12} \right)^{12}\right)^{5} = P \cdot
  \left( 1 + \frac{0.05}{12} \right)^{60}.
 \]
 
 S postupným zkracováním úrokového období se nabízí otázka, jaká by byla finální
 částka, kdyby se původní vklad úročil \uv{nekonečněkrát} do roka, tj. uložená
 částka by se zvyšovala doslova v každém okamžiku (tedy \uv{spojitě}). Odpovědí
 je výraz
 \[
  \lim_{n \to \infty} P \cdot \left( 1 + \frac{0.05}{n} \right)^{n}.
 \]
 
 Ukážeme, že touto limitou je hodnota funkce $\exp$ v bodě $0.05$. Obecněji,
 spočteme, že
 \begin{equation*}
  \label{eq:exp-as-lim}
  \tag{$\spadesuit$}
  \lim_{n \to \infty} \left( 1 + \frac{x}{n} \right)^{n} = \exp x.
 \end{equation*}

 Výpočet je to v celku triviální. Z definice obecné mocniny a
 \hyperref[cor:heineho-veta-pro-spojitost]{Heineho věty} platí
 \[
  \lim_{n \to \infty} \left( 1 + \frac{x}{n} \right)^{n} = \lim_{n \to \infty}
  \exp \left( n \log \left( 1 + \frac{x}{n} \right) \right) = \exp \left(
  \lim_{n \to \infty} n \log \left(1 + \frac{x}{n}\right) \right).
 \]
 Stačí tedy ukázat, že
 \[
  \lim_{n \to \infty} n \log \left( 1+\frac{x}{n} \right) = x.
 \]
 Užitím \hyperref[thm:lhospitalovo-pravidlo]{l'Hospitalova pravidla} dostaneme
 (derivujeme podle $n$ -- to smíme opět díky \hyperref[thm:heineho]{Heineho
 větě})
 \[
  \lim_{n \to \infty} n \log \left( 1+\frac{x}{n} \right) = \lim_{n \to \infty}
  \frac{\log \left( 1 + \frac{x}{n} \right)}{\frac{1}{n}} = \lim_{n \to \infty}
  \frac{1}{1+\frac{x}{n}} \cdot \frac{-x}{n^2} \cdot (-n^2) = \lim_{n \to
  \infty} \frac{nx}{n + x} = x.
 \]
 Tím je rovnost~\eqref{eq:exp-as-lim} dokázána.
\end{example}
