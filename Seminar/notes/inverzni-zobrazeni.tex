\documentclass[a4paper,11pt]{article}

\usepackage[czech]{babel}

% Colors %
\usepackage[dvipsnames]{xcolor}

% Page Layout %
\usepackage[margin=1.5in]{geometry}

% Fancy Headers %
\usepackage{fancyhdr}
\fancyhf{}
\cfoot{\thepage}
\rhead{}
\renewcommand{\headrulewidth}{0pt}
\setlength{\headheight}{16pt}

% Math
\usepackage{mathtools}
\usepackage{amssymb}
\usepackage{faktor}
\usepackage{import}
\usepackage{caption}
\usepackage{subcaption}
\usepackage{wrapfig}

% TikZ
\usepackage{tikz}
\usetikzlibrary{cd,babel,positioning,arrows,calc}

\usepackage{dsfont}

% Theorems
\usepackage{amsthm}
\usepackage{thmtools}

\theoremstyle{plain}
\newtheorem*{rmrk}{Poznámka}

\theoremstyle{definition}
\newtheorem*{defin}{Definice}

\theoremstyle{theorem}
\newtheorem*{theorem}{Věta}

% Title %
\title{\Huge\textsf{}\\
 \Large\textsf{}
 \author{}
 \date{}
}

% Table of Contents %
\usepackage{hyperref}
\hypersetup{
 colorlinks=true,
 linktoc=all,
 linkcolor=blue
}

% Tables %
\usepackage{booktabs}

% Enumerate %
\usepackage{enumitem}

% Operators %
\DeclareMathOperator{\im}{im}
\DeclareMathOperator{\End}{End}
\DeclareMathOperator{\Aut}{Aut}
\DeclareMathOperator{\Inn}{Inn}
\DeclareMathOperator{\dom}{dom}
\DeclareMathOperator{\codom}{codom}

% Common operators %
\newcommand{\R}{\mathbb{R}}
\newcommand{\N}{\mathbb{N}}
\newcommand{\Z}{\mathbb{Z}}
\newcommand{\Q}{\mathbb{Q}}
\newcommand{\C}{\mathbb{C}}

% American Paragraph Skip %
\setlength{\parindent}{0pt}
\setlength{\parskip}{1em}

% Document %
\pagestyle{fancy}
\begin{document}

Slinty o inverzních zobrazeních. Zobrazení $f:A \to B$ jsou pro nás automaticky
\textbf{zobrazení definovaná všude}, tedy $f(a)$ je prvek $B$ pro každé $a \in
A$.

Když $f:A \to B$ a $g:B \to C$, pak složení zobrazení $g \circ f$, které je
definováno stejně jako složení relací (bo zobrazení jsou relace), je zobrazení
$A \to C$. Většinou budu vynechávat symbol $ \circ $ a místo $g \circ f$ psát
jenom $gf$. Dobře se složení zobrazení představují jako skládání šipek za sebe:
\begin{center}
 \begin{tikzcd}
  A \ar[r, "f"] \ar[rr, "gf", bend right=45] & B \ar[r, "g"] & C
 \end{tikzcd}
\end{center}

Dívat se na složení $gf$ jako na základnu trojúhelníku s rameny $f$ a $g$ asi
taky může pomoct:

\begin{center}
 \begin{tikzcd}[sep=small]
  & & B \arrow[ddrr, "g"] & &\\
  & & & &\\
  A \arrow[uurr, "f"] \arrow[rrrr, "gf"'] & & & & C
 \end{tikzcd}
\end{center}

Zobrazení $f:A \to B$ a $g:C \to D$ lze v skládat pořadí $g \circ f$ jenom
tehdy, když $f$ končí tam, kde $g$ začíná, formálně když $\codom f = \dom g$. V
tomhle případě to znamená $B = C$. V opačném pořadí, tj. $f \circ g$, je lze
skládat, když $\codom g = \dom f$ neboli $D = A$.

Na každé množině $A$ je jedno speciální zobrazení, které každému prvku přiřadí
ten samý. Budu mu říkat \emph{identické zobrazení} a značit je $\mathds{1}_A$.
Tedy, $\mathds{1}_A$ je zobrazení $A \to A$ takové, že $\mathds{1}_A(a) = a$ pro
každé $a \in A$.

\begin{defin}[Inverzní zobrazení]
 Ať $f:A \to B$. \emph{Inverzním zobrazením} k $f$ nazveme zobrazení $g:B \to A$
 splňující
 \[
  gf = \mathds{1}_A \quad \text{a} \quad fg = \mathds{1}_B.
 \]
 Inverzní zobrazení samozřejmě nemusí existovat. Pokud existuje, značíme ho,
 pravdaže dost nesmyslně, $f^{-1}$. Čili $ff^{-1} = \mathds{1}_{B}$ a $f^{-1}f =
 \mathds{1}_{A}$.
\end{defin}

\begin{rmrk}
 Všimněte si, že $ff^{-1}$ je zobrazení $B \to B$ a $f^{-1}f$ je zobrazení $A
 \to A$! V obrázcích
 \begin{center}
  \begin{tikzcd}
   A \arrow[r, "f",] \arrow[rr, "f^{-1}f = \mathds{1}_A"', bend right=45] & B
   \arrow[r, "f^{-1}"] & A & & B \arrow[r, "f^{-1}"] \arrow[rr, "ff^{-1} =
   \mathds{1}_B"', bend right=45] & A \arrow[r, "f"] & B.
  \end{tikzcd}
 \end{center}
\end{rmrk}

Možná vám někdo někdy řekl, že k zobrazení (asi jim říkali \uv{funkce}) existuje
zobrazení inverzní právě tehdy, když je prosté. To nám nestačí. My budeme
uvažovat inverzní zobrazení pouze k bijekcím (tj. k zobrazením, která jsou
prostá a na). Má to následující důvod.

Prosté zobrazení $f:A \to B$ je totiž \uv{to samé}, co bijekce $f:A \to \im f$,
kde $\im f$ je množina všech obrazů prvků z $A$ při zobrazení $f$. Symbolicky,
\[
 \im f = \{f(a) \mid a \in A\} \subseteq B.
\]
Když zobrazení $f$ není na, pak $\im f$ je pouze podmnožina $B$, a ne celé $B$.
Když ale vynechám z $B$ ty prvky, na které se nic z $A$ nezobrazuje, tak přece
dostanu úplně to samé zobrazení. V obrázcích si to můžete představovat tak, že
pokud $f$ je třeba následující zobrazení:
\begin{center}
 \begin{tikzpicture}
  \node[circle,fill,inner sep=1.5pt] (A1) at (-1.5,2) {};
  \node[circle,fill,inner sep=1.5pt] (A2) at (-1.5,1) {};
  \node[circle,fill,inner sep=1.5pt] (A3) at (-1.5,0) {};
  \node[left=2mm of A1] {$1$};
  \node[left=2mm of A2] {$2$};
  \node[left=2mm of A3] {$3$};
  \node (A) [above=5mm of A1] {$A$};


  \node[circle,fill,inner sep=1.5pt] (B1) at (1.5,2) {};
  \node[circle,fill,inner sep=1.5pt] (B2) at (1.5,1) {};
  \node[circle,fill,inner sep=1.5pt] (B3) at (1.5,0) {};
  \node[circle,fill,inner sep=1.5pt] (B4) at (1.5,-1) {};
  \node[circle,fill,inner sep=1.5pt] (B5) at (1.5,-2) {};
  \node[right=2mm of B1] {$1$};
  \node[right=2mm of B2] {$2$};
  \node[right=2mm of B3] {$3$};
  \node[right=2mm of B4] {$4$};
  \node[right=2mm of B5] {$5$,};
  \node (B) [above=5mm of B1] {$B$};

  \draw[blue, ->, shorten >=2pt, shorten <=2pt] (A1) -- (B2);
  \draw[blue, ->, shorten >=2pt, shorten <=2pt] (A2) -- (B3);
  \draw[blue, ->, shorten >=2pt, shorten <=2pt] (A3) -- (B5);
  \node[blue] at (0, 2) {$f$};

  \draw[red,thick] (1.3, 2.2) -- (1.7, 1.8);
  \draw[red,thick] (1.7, 2.2) -- (1.3, 1.8);

  \draw[red,thick] (1.7, -0.8) -- (1.3, -1.2);
  \draw[red,thick] (1.7, -1.2) -- (1.3, -0.8);
 \end{tikzpicture}
\end{center}
pak když vynechám z $B$ prvky $1$ a $4$, které nejsou v $\im f$, pak dostanu
opravdu to samé zobrazení. Konkrétně,
\begin{center}
 \begin{tikzpicture}
  \node[circle,fill,inner sep=1.5pt] (A1) at (-1.5,2) {};
  \node[circle,fill,inner sep=1.5pt] (A2) at (-1.5,1) {};
  \node[circle,fill,inner sep=1.5pt] (A3) at (-1.5,0) {};
  \node[left=2mm of A1] {$1$};
  \node[left=2mm of A2] {$2$};
  \node[left=2mm of A3] {$3$};
  \node (A) [above=5mm of A1] {$A$};


  \node[circle,fill,inner sep=1.5pt] (B2) at (1.5,2) {};
  \node[circle,fill,inner sep=1.5pt] (B3) at (1.5,1) {};
  \node[circle,fill,inner sep=1.5pt] (B5) at (1.5,0) {};
  \node[right=2mm of B2] {$2$};
  \node[right=2mm of B3] {$3$};
  \node[right=2mm of B5] {$5$.};
  \node (B) [above=5mm of B1] {$\im f \subsetneq B$};

  \draw[blue, ->, shorten >=2pt, shorten <=2pt] (A1) -- (B2);
  \draw[blue, ->, shorten >=2pt, shorten <=2pt] (A2) -- (B3);
  \draw[blue, ->, shorten >=2pt, shorten <=2pt] (A3) -- (B5);
  \node[blue] at (0, 2.5) {$f$};
 \end{tikzpicture}
\end{center}

Skončíme následující větou, která potvrzuje, že přemýšlíme správným směrem.

\begin{theorem}[Bijekce $\iff$ existuje inverzní zobrazení]
 Ať $f:A \to B$ je zobrazení. Pak $f$ je bijekce (prosté a na) právě tehdy, když
 k němu existuje inverzní zobrazení.
\end{theorem}
\begin{proof}
 Tvrzení je ekvivalence, takže budeme dokazovat dvě implikace.

 Nejdřív dokážeme implikaci \uv{zleva doprava}, tj. že k bijekci vždycky
 existuje inverzní zobrazení. Ať $f$ je tedy bijekce, tedy prosté a na.
 Potřebujeme definovat zobrazení $g:B \to A$ takové, aby $fg = \mathds{1}_{B}$ a
 $gf = \mathds{1}_{A}$.

 Uděláme to prostě prvek po prvku. Zvolme si náhodně nějaké $b \in B$. Protože
 $f$ je na, existuje $a \in A$, že $f(a) = b$. Navíc, protože $f$ je prosté,
 tohle $a$ je právě jedno, tj. žádný jiný prvek z $A$ se na $b$ nezobrazuje.
 Definujme $g(b) = a$. Pak máme $fg(b) = f(a) = b$ (tady využíváme toho, že  $f$ 
 je na, tedy máme prvek $a$, který se zobrazuje na $b$) a taky $gf(a) = g(b) =
 a$ (tady využíváme toho, že $f$ je prosté, tedy že opravdu jenom $a$ se zobrazí
 na $b$). Čili, $g = f^{-1}$.

 Implikaci zprava doleva uděláme trochu jinak. Pamatujte z logiky, že implikace
 $p \Rightarrow q$ je to samé, jako implikace $\neg q \Rightarrow \neg p$. Takže
 budeme předpokládat, že $f$ \textbf{není} bijekce (tedy není prosté nebo není
 na) a chceme dokázat, že $f$ \textbf{nemá} k~sobě inverzní funkci. Pro spor
 tedy budeme předpokládat, že $f^{-1}$ existuje a ukážeme, že to vede na
 nesmysl.

 Máme celkem dvě možnosti:
 \begin{enumerate}[label=(\arabic*),topsep=0pt]
  \item Zobrazení $f$ není prosté. Pak existují dva prvky $a_1,a_2 \in A$
   takové, že $f(a_1) = f(a_2)$. Označíme jejich obraz $b$. Pak ale $f^{-1}f$
   nemůže být rovno $\mathds{1}_{A}$, protože buď
   \begin{enumerate}[label=(\alph*)]
    \item $f^{-1}(b) = a_1$ a pak $f^{-1}f(a_2) = f^{-1}(b) = a_1$, nebo
    \item $f^{-1}(b) = a_2$ a pak $f^{-1}f(a_1) = f^{-1}(b) = a_2$.
   \end{enumerate}
   V obou případech jsme se dostali z jednoho prvku pomocí zobrazení $f^{-1}f$ 
   do jiného, tedy to nemůže být identické zobrazení. Pomocný obrázek ukazuje
   ten problém -- $f^{-1}$ totiž může $b$ zobrazovat jen na jeden prvek, což je
   ale dost problém, když $f$ na $b$ zobrazuje prvky \textbf{dva}.
   \begin{center}
    \begin{tikzpicture}
     \node at (-1.5,1.75) {\vdots};
     \node[circle,fill,inner sep=1.5pt] (a1) at (-1.5,1) {};
     \node[circle,fill,inner sep=1.5pt] (a2) at (-1.5,0) {};
     \node at (-1.5,-0.5) {\vdots};
     \node[left=2mm of a1] {$a_1$};
     \node[left=2mm of a2] {$a_2$};
     \node (A) at (-1.5, 2.25) {$A$};

     \node at (1.5,1.25) {\vdots};
     \node[circle,fill,inner sep=1.5pt] (b) at (1.5,0.5) {};
     \node at (1.5,-0.5) {\vdots};
     \node[below=1mm of b] {$b$};
     \node (B) at (1.5, 2.25) {$B$};

     \node at (4.5,1.75) {\vdots};
     \node[circle,fill,inner sep=1.5pt] (a11) at (4.5,1) {};
     \node[circle,fill,inner sep=1.5pt] (a22) at (4.5,0) {};
     \node at (4.5,-0.5) {\vdots};
     \node[right=2mm of a11] {$a_1$};
     \node[right=2mm of a22] {$a_2$};
     \node (A) at (4.5, 2.25) {$A$};

     \draw[blue, ->, shorten >= 2pt, shorten <= 2pt] (a1) -- (b);
     \draw[blue, ->, shorten >= 2pt, shorten <= 2pt] (a2) -- (b);
     \node[blue] (f) at (0, 1) {$f$};

     \draw[red, ->, shorten >= 2pt, shorten <= 2pt, dashed] (b) -- (a11);
     \draw[red, ->, shorten >= 2pt, shorten <= 2pt, dashed] (b) -- (a22);
     \node[red] (f-1) at (3, 1) {$f^{-1}$};
    \end{tikzpicture}
   \end{center}
  \item Zobrazení $f$ není na. Pak existuje prvek $b \in B$, na který se žádné
   $a \in A$ nezobrazuje. To je ovšem taky dost problém, protože potom se $b$ 
   pomocí $ff^{-1}$ nemůže zobrazit zpátky na $b$. Vskutku, ať $f^{-1}(b)$ je
   nějaký prvek $a$. Pak ale $f(a) \neq b$, protože $f$ nezobrazuje nic na $b$.
   tedy $ff^{-1}(b) \neq b$, takže $ff^{-1} \neq \mathds{1}_{B}$. Problém opět
   vidíte na obrázku.
   \begin{center}
    \begin{tikzpicture}
     \node at (-1.5,1.75) {\vdots};
     \node[circle,fill,inner sep=1.5pt] (b) at (-1.5,1) {};
     \node[circle,fill,inner sep=1.5pt] (c) at (-1.5,0) {};
     \node at (-1.5,-0.5) {\vdots};
     \node[left=2mm of b] {$b$};
     \node[left=2mm of c] {$c$};
     \node (B) at (-1.5, 2.25) {$B$};

     \node at (1.5,1.25) {\vdots};
     \node[circle,fill,inner sep=1.5pt] (a) at (1.5,0.5) {};
     \node at (1.5,-0.5) {\vdots};
     \node[below=1mm of a] {$a$};
     \node (A) at (1.5, 2.25) {$A$};
     
     \node at (4.5,1.75) {\vdots};
     \node[circle,fill,inner sep=1.5pt] (b') at (4.5,1) {};
     \node[circle,fill,inner sep=1.5pt] (c') at (4.5,0) {};
     \node at (4.5,-0.5) {\vdots};
     \node[right=2mm of a11] {$b$};
     \node[right=2mm of a22] {$c$};
     \node (B) at (4.5, 2.25) {$B$};

     \draw[red, ->, shorten >= 2pt, shorten <= 2pt] (b) -- (a);
     \node[red] (f-1) at (0, 1.25) {$f^{-1}$};

     \draw[blue, ->, shorten >= 2pt, shorten <= 2pt] (a) -- (c');
     \node[blue] (f) at (3, 0.75) {$f$};
    \end{tikzpicture}
   \end{center}
   Shrnuto, když $f$ není prosté, pak nemůže platit $f^{-1}f = \mathds{1}_{A}$,
   a když $f$ není na, pak nemůže platit $ff^{-1} = \mathds{1}_{B}$. Celkově,
   zobrazení, které není bijektivní, k sobě nemůže mít inverzní
   zobrazení.\qedhere
 \end{enumerate}
\end{proof}

\end{document}
