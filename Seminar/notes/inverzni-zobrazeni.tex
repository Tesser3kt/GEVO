\documentclass[a4paper,11pt]{article}

\usepackage[czech]{babel}
\usepackage[T1]{fontenc}

\usepackage{etoolbox}
\preto\tabular{\shorthandoff{-}}

% Colors %
\usepackage[dvipsnames]{xcolor}

% Page Layout %
\usepackage[margin=1.5in]{geometry}

% Fancy Headers %
\usepackage{fancyhdr}
\fancyhf{}
\cfoot{\thepage}
\rhead{}
\renewcommand{\headrulewidth}{0pt}
\setlength{\headheight}{16pt}

% Math
\usepackage{mathtools}
\usepackage{amssymb}
\usepackage{faktor}
\usepackage{import}
\usepackage{caption}
\usepackage{subcaption}
\usepackage{wrapfig}
\usepackage{tikz-cd}
\usepackage{dsfont}

% Theorems
\usepackage{amsthm}
\usepackage{thmtools}

\theoremstyle{definition}
\newtheorem*{defin}{Definice}

% Title %
\title{\Huge\textsf{}\\
 \Large\textsf{}
 \author{}
 \date{}
}

% Table of Contents %
\usepackage{hyperref}
\hypersetup{
 colorlinks=true,
 linktoc=all,
 linkcolor=blue
}

% Tables %
\usepackage{booktabs}

% Enumerate %
\usepackage{enumitem}

% Operators %
\DeclareMathOperator{\Ker}{Ker}
\DeclareMathOperator{\Img}{Im}
\DeclareMathOperator{\End}{End}
\DeclareMathOperator{\Aut}{Aut}
\DeclareMathOperator{\Inn}{Inn}

% Common operators %
\newcommand{\R}{\mathbb{R}}
\newcommand{\N}{\mathbb{N}}
\newcommand{\Z}{\mathbb{Z}}
\newcommand{\Q}{\mathbb{Q}}
\newcommand{\C}{\mathbb{C}}

% American Paragraph Skip %
\setlength{\parindent}{0pt}
\setlength{\parskip}{1em}

% Document %
\pagestyle{fancy}
\begin{document}

Slinty o inverzních zobrazeních. Zobrazení $f:A \to B$ jsou pro nás automaticky
\textbf{zobrazení definovaná všude}, tedy $f(a)$ je prvek $B$ pro každé $a \in
A$.

Když $f:A \to B$ a $g:B \to C$, pak složení zobrazení $g \circ f$, které je
definováno stejně jako složení relací (bo zobrazení jsou relace), je zobrazení
$A \to C$. Většinou budu vynechávat symbol $ \circ $ a místo $g \circ f$ psát
jenom $gf$. Dobře se složení zobrazení představují jako skládání šipek za sebe:
\begin{center}
 \begin{tikzcd}
  A \ar[r, "f"] \ar[rr, "gf", bend right=45] & B \ar[r, "g"] & C
 \end{tikzcd}
\end{center}

Na každé množině $A$ je jedno speciální zobrazení, které každému prvku přiřadí
ten samý. Budu mu říkat \emph{identické zobrazení} a značit je $\mathds{1}_A$.
Tedy, $\mathds{1}_A$ je zobrazení $A \to A$ takové, že $\mathds{1}_A(a) = a$ pro
každé $a \in A$.

\begin{defin}[Inverzní zobrazení]
 Ať $f:A \to B$. \emph{Inverzním zobrazením} k $f$ nazveme zobrazení $g:B \to A$
 splňující
 \[
  gf = 
 \]
 
\end{defin}

\end{document}
