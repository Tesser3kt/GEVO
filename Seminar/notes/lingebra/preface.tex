\chapter*{Preface}

This text covers selected topics from the curriculum of a typical undergraduate
linear algebra course. Almost no pre-existing knowledge is strictly required
save a superficial understanding of propositional logic and set theory. A
reasonably good ability to manipulate algebraic expressions should prove
advantageous, too.

Mathematics is an exact and rigorous language. Words and symbols have singular,
precisely defined, meaning. Many students fail to grasp that intuition and
imagination are paramount, but they serve as a \emph{starting point}, with
formal logical expression being the end. For example, an intuitive understanding
of a \emph{line} as an infinite flat 1D object is pretty much correct but not
\emph{formal}. It is indeed the formality of mathematics which puts many
students off. Whereas high school mathematics is mostly algorithmic and
non-argumentative, higher level maths tends to be the exact opposite -- full of
concepts and relations between those, which one is expected to be capable of
grasping and formally describing. Owing to this, I wish this text would be a
kind of synthesis of the formal and the conceptual. On one hand, rigorous
definitions and proofs are given; on the other, illustrations, examples and
applications serve as hopefully efficient conveyors of the former's geometric
nature.

Linear algebra is a mathematical discipline which studies -- as its name rightly
suggests -- the \emph{linear}. Nevertheless, the word \emph{linear} (as in
`line-like') is slightly misplaced. The correct term would perhaps be
\emph{flat} or, nigh equivalently, \emph{not curved}. It isn't hard to imagine
why curved objects (as in \emph{geometric} objects, say) are more difficult to
describe and manipulate than objects flat. For instance, the formula for the
volume of a cube is just the product of the lengths of its sides. Contrast this
with the volume of a still `simple', yet curved, object -- the ball. Its volume
cannot even be \emph{precisely} determined; its calculation involves
approximating an irrational constant and the derivation of its formula is
starkly unintuitive without basic knowledge of measure theory.

As such, linear algebra is a highly `geometric' discipline and opportunities for
visual interpretations abound. This is also a drawback in a certain sense. One
should not dwell on visualisations alone as they tend to lead astray where
imagination falls short. Symbolic representation of the geometry at hand is
key.

% TODO reference
The word \emph{linear} however dons a broader sense in modern mathematics. It
can be rephrased as reading, `related by addition and multiplication by a
scalar'. We trust kind readers have been acquainted with the notion of a
\emph{linear function}. A linear function is (rightly) called \emph{linear} for
it receives a number as input and outputs its \emph{constant} multiple plus
another \emph{constant} number. Therefore, the output is in a \emph{linear}
relation to the input -- it is multiplied by some fixed number and added to
another. This understanding of the word is going to prove crucial already in the
first chapter, where we study \emph{linear systems}. Following are \emph{vector
spaces} and \emph{linear maps}, concepts whose depth shall occupy the span of
this text. Each chapter is further endowed with an \emph{applications} section
where I try to draw a simile between mathematics and common sense.
