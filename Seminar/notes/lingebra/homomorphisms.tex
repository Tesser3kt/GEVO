\chapter{Homomorphisms (chybějí obrázky)}
\label{chap:homomorphisms}

In this chapter, our aim is to study and understand maps between vector spaces.
Not just any kind of maps, however, but maps that \emph{preserve structure}.

Most of modern mathematics is dedicated to the study of \emph{structures} --
basically prescribed rules of interaction between elements of a set. We call
these rules, \emph{operations}, and when moving from a set with structure to a
set with structure by a map, we tend to require that said map somehow respects
the structures of both sets. Such maps are often called \emph{homomorphisms},
from Greek ὁμός (\uv{same}) and μορφή (\uv{form, shape}).

The only structure we consider in this book is that of a vector space given by
two operations: scalar multiplication and vector addition. A \emph{homomorphism
between vector spaces} $V$ and $W$ (also called a \emph{linear map}) is thus a
map which respects both operations; in practice, this means that the image of a
scalar multiple should be the same scalar multiple of the image and that the
image of a sum of vectors should be the sum of the images.

One last note: we ought to be careful when comparing two structures. We labelled
the operations on a vector space by symbols $ \cdot $ and $+$ but these two
symbols \textbf{mean different things in different vector spaces}! To keep the
text tidy, we shan't resort to using yet another distinct pair of symbols.
However, we \emph{are} going to distinguish the structure in a small number of
ensuing lemmata and definitions, to drive the point home.

\begin{definition}{Homomorphism}{homomorphism}
 Let $V$ and $W$ be vector spaces over the field $\F$. We denote the operations
 of scalar multiplication and vector addition on $V$ by $\clr{ \cdot _V}$ and
 $\clr{ +_V}$ and those on $W$ by $\clb{ \cdot _W}$ and $\clb{+_W}$. A map $f:V
 \to W$ is a \emph{homomorphism} (or a \emph{linear map}) if
 \begin{enumerate}
  \item $f(\mathbf{v}_1~\clr{+_V}~\mathbf{v}_2) =
   f(\mathbf{v}_1)~\clb{+_W}~f(\mathbf{v}_2)$ for every two vectors
   $\mathbf{v}_1,\mathbf{v}_2 \in V$.
  \item $f(t~\clr{ \cdot _V}~\mathbf{v}) = t~\clb{ \cdot_W}~f(\mathbf{v})$ for
   every $t \in \F$ and $\mathbf{v} \in V$.
 \end{enumerate}
\end{definition}

\begin{example}{}{homs}
 The following maps are homomorphisms:
 \begin{enumerate}[label=(\alph*)]
  \item the map $f:\R^2 \to \R^2$ given by $f(\mathbf{v}) = 2 \cdot \mathbf{v}$;
  \item the map $f:\mathcal{P}_3(\F) \to \F^{4}$ given by
   \[
    f(a_0 + a_1x + a_2x^2 + a_3x^3) = 
    \begin{pmatrix}
     a_0\\
     a_1\\
     a_2\\
     a_3
    \end{pmatrix},
   \]
   where $\mathcal{P}_3(\F)$ denotes the space of polynomials of degree $3$ with
   coefficients in the field $\F$;
  \item the map $\pi: \R^3 \to \R^2$ given by
   \[
    \pi \left( 
    \begin{pmatrix}
     x\\
     y\\
     z
    \end{pmatrix}
    \right) = 
    \begin{pmatrix}
     x\\
     y
    \end{pmatrix};
   \]
   maps that `forget coordinates' are often called \emph{projections}.
 \end{enumerate}
 The following maps are \textbf{not} homomorphisms:
 \begin{enumerate}[label=(\alph*)]
  \item the map $f:\R^3 \to \R^3$ given by
   \[
    f \left( 
    \begin{pmatrix}
     x\\
     y\\
     z
    \end{pmatrix}
    \right) = 
    \begin{pmatrix}
     x\\
     y\\
     z
    \end{pmatrix}
    +
    \begin{pmatrix}
     1\\
     2\\
     -3
    \end{pmatrix};
   \]
  \item the map $f: \R^2 \to \R$ given by
   \[
    f \left( 
    \begin{pmatrix}
     x\\
     y
    \end{pmatrix}
    \right) = x^2 + y^3 - 6.
   \]
  \item the map $f:\R^{2 \times 2} \to \R^2$ given by
   \[
    f \left( 
    \begin{pmatrix}
     a & b\\
     c & d
    \end{pmatrix}
    \right) = 
    \begin{pmatrix}
     a \cdot b + c \cdot d\\
     a \cdot d - b \cdot c
    \end{pmatrix}.
   \]
 \end{enumerate}
\end{example}

In the previous example, we claimed that certain maps were homomorphisms without
giving a proof. We did so because we first want to provide a characterisation of
homomorphisms which makes checking whether a given map is a homomorphism
somewhat easier. Hence, we now collect two qualities only homomorphisms possess.

\begin{lemma}{Zero to zero}{zero-to-zero}
 Let $f:V \to W$ be a homomorphism and label the zero vector of $V$ by
 $\clr{\mathbf{0}_V}$ and the zero vector of $W$ by $\clb{\mathbf{0}_W}$. Then,
 \[
  f(\clr{\mathbf{0}_V}) = \clb{\mathbf{0}_W}.
 \]
\end{lemma}
\begin{lemproof}
 Exploiting axiom (2) in the \hyperref[def:homomorphism]{definition of
 homomorphism}, we get
 \[
  f(\clr{\mathbf{0}_V}) = f(0~\clr{ \cdot _V}~\clr{\mathbf{0}_V})
  \overset{(2)}{=} 0~\clb{ \cdot _W}~f(\clr{\mathbf{0}_V}) = \clb{\mathbf{0}_W}
 \]
 as required.
\end{lemproof}

\begin{lemma}{}{hom-linear-combination}
 For two vector spaces $V,W$ over $\F$ and a map $f:V \to W$, the following
 statements are equivalent.
 \begin{enumerate}[label=(\alph*)]
  \item The map $f$ is a homomorphism.
  \item For any two vectors $\mathbf{v}_1,\mathbf{v}_2 \in V$ and two numbers
   $t_1,t_2 \in \F$, we have
   \[
    f(t_1~\clr{ \cdot _V}~\mathbf{v}_1~\clr{+_V}~t_2~\clr{ \cdot
    _V}~\mathbf{v}_2) = t_1~\clb{ \cdot
    _W}~f(\mathbf{v}_1)~\clb{+_W}~t_2~\clb{ \cdot _W}~f(\mathbf{v}_2).
   \]
  \item For any vectors $\mathbf{v}_1,\ldots,\mathbf{v}_n$ and numbers
   $t_1,\ldots,t_n \in \F$, we have
   \[
    f(t_1~\clr{ \cdot_V}~\mathbf{v_1}~\clr{ + _V}~\ldots~\clr{ +_V}~t_n~\clr{
    \cdot _V}~\mathbf{v}_n) = t_1~\clb{ \cdot
    _W}~f(\mathbf{v}_1)~\clb{+_W}~\ldots~\clb{+_W}~t_n~\clb{ \cdot
    _W}~f(\mathbf{v}_n).
   \]
 \end{enumerate}
\end{lemma}
\begin{lemproof}
 In the proof (as well as the following text), we stop distinguishing between
 $\clr{ \cdot _V}$, $\clb{ \cdot _W}$ and $\clr{+_W},\clb{+_W}$ for the sake of
 clarity. The readers should do well to keep in mind that $V$ and $W$ host
 different structures, though.

 We prove $(a) \Leftrightarrow (b)$ and $(b) \Leftrightarrow (c)$.

 As for $(a) \Rightarrow (b)$, we shall, naturally, invoke the axioms (1) and
 (2) of the \hyperref[def:homomorphism]{definition of homomorphism}. We compute
 \[
  f(t_1 \cdot \mathbf{v}_1 + t_2 \cdot \mathbf{v}_2) \overset{(1)}{=} f(t_1
  \cdot \mathbf{v}_1) + f(t_2 \cdot \mathbf{v}_2) \overset{(2)}{=} t_1 \cdot
  f(\mathbf{v}_1) + t_2 \cdot f(\mathbf{v}_2).
 \]
 The implication $(b) \Rightarrow (a)$ we simply substitute $t_1 = t_2 = 1$ for
 axiom (1) and $\mathbf{v}_2 = \mathbf{0}$ for axiom (2).

 Similarly, the implication $(c) \Rightarrow (b)$ follows trivially by $n = 2$.
 We prove the last implication $(b) \Rightarrow (c)$ by induction on $n$. The
 base case $n = 2$ is covered completely by statement $(b)$. For the induction
 step, label $\mathbf{w} = t_1 \cdot \mathbf{v}_1 + \ldots + t_n \cdot
 \mathbf{v}_n$. Then,
 \[
  f(t_1 \cdot \mathbf{v}_1 + \ldots + t_n \cdot \mathbf{v}_n + t_{n+1} \cdot
  \mathbf{v}_{n+1}) = f(\mathbf{w} + t_{n+1} \cdot \mathbf{v}_{n+1}).
 \]
 Using statement $(b)$ again, we get
 \[
  f(\mathbf{w} + t_{n+1} \cdot \mathbf{v}_{n+1}) = f(\mathbf{w}) + t_{n+1} \cdot
  f(\mathbf{v}_{n+1}).
 \]
 By the induction hypothesis,
 \[
  f(\mathbf{w}) = f(t_1 \cdot \mathbf{v}_1 + \ldots + t_n \cdot \mathbf{v}_n) =
  t_1 \cdot f(\mathbf{v}_1) + \ldots + t_n \cdot f(\mathbf{v}_n).
 \]
 And thus,
 \begin{align*}
  f(\mathbf{w}) + t_{n+1} \cdot f(\mathbf{v}_{n+1}) &= f(t_1 \cdot \mathbf{v}_1 +
  \ldots + t_n \cdot \mathbf{v}_n) + t_{n+1} \cdot \mathbf{v}_{n+1}\\
                                                    &= t_1 \cdot
  f(\mathbf{v}_{1}) + \ldots + t_n \cdot f(\mathbf{v}_n) + t_{n+1} \cdot
  f(\mathbf{v}_{n+1})
 \end{align*}
 and we're done.
\end{lemproof}

\begin{remark}{}{hom-parallelepiped}
 The statement $(b)$ in \myref{lemma}{lem:hom-linear-combination} can be
 geometrically interpreted as saying that a homomorphism `transforms
 parallelepipeds into parallelepipeds'. Let's see this on an example.

 Any two linearly independent vectors $\mathbf{u},\mathbf{v} \in \R^2$ define a
 parallelogram as the set of all linear combinations of $\mathbf{u}$ and
 $\mathbf{v}$ with coefficients between $0$ and $1$, i.e.
 \[
  \mathbf{P}(\mathbf{u},\mathbf{v}) \coloneqq \{a \cdot \mathbf{u} + b \cdot
  \mathbf{v} \mid a,b \in [0,1]\}.
 \]
 \begin{center}
  \clr{TODO obrazek}
 \end{center}
 Then, the mentioned statement $(b)$ says the following for a homomorphism
 $f:\R^2 \to \R^2$
 \[
  f(a \cdot \mathbf{u} + b \cdot \mathbf{v}) = a \cdot f(\mathbf{u}) + b \cdot
  f(\mathbf{v}).
 \]
 However, this can be read to say that the image of every point in the
 parallelogram determined by $\mathbf{u}$ and $\mathbf{v}$ is a point in the
 parallelogram determined by $f(\mathbf{u})$ and $f(\mathbf{v})$. Symbolically,
 \[
  f(\mathbf{P}(\mathbf{u},\mathbf{v})) =
  \mathbf{P}(f(\mathbf{u}),f(\mathbf{v})).
 \]
\end{remark}

Let us return to \myref{example}{exam:homs}. There, we claimed that certain maps
were homomorphisms without proof. For some of them, we're providing the proof
now.

It is easily checked that the projection
\[
 \pi \left( 
  \begin{pmatrix}
   x\\
   y\\
   z
  \end{pmatrix}
 \right) = 
 \begin{pmatrix}
  x\\
  y
 \end{pmatrix}
\]
is a homomorphism. Indeed, we can calculate
\begin{align*}
 \pi \left( a \cdot 
 \begin{pmatrix}
  x_1\\
  y_1\\
  z_1
 \end{pmatrix} + b \cdot 
 \begin{pmatrix}
  x_2\\
  y_2\\
  z_2
 \end{pmatrix}
\right) &= 
 \pi \left( 
 \begin{pmatrix}
  ax_1 + bx_2\\
  ay_1 + by_2\\
  az_1 + bz_2
 \end{pmatrix}
 \right) = 
 \begin{pmatrix}
  ax_1 + bx_2\\
  ay_1 + by_2
 \end{pmatrix}
 \\
        &=a \cdot 
 \begin{pmatrix}
  x_1\\
  y_1
 \end{pmatrix}
 + b \cdot 
 \begin{pmatrix}
  x_2\\
  y_2
 \end{pmatrix} = 
 a \cdot \pi
 \left( 
 \begin{pmatrix}
  x_1\\
  y_1\\
  z_1
 \end{pmatrix}
 \right) + b \cdot \pi \left( 
 \begin{pmatrix}
  x_2\\
  y_2\\
  z_2
 \end{pmatrix}
 \right).
\end{align*}
The rest is up to \myref{lemma}{lem:hom-linear-combination}.

In a very similar vein, the map $f(\mathbf{v}) = 2 \cdot \mathbf{v}$ is a
homomorphism for any vector space $V$ over $\R$. Indeed, we may compute
\[
 f(a \cdot \mathbf{u} + b \cdot \mathbf{v}) = 2 \cdot (a \cdot \mathbf{u} + b
 \cdot \mathbf{v}) = a \cdot (2 \cdot \mathbf{u}) + b \cdot (2 \cdot \mathbf{v})
 = a \cdot f(\mathbf{u}) = b \cdot f(\mathbf{v}).
\]
This last homomorphism is an example of an \emph{automorphism} -- a bijective
homomorphism from a space to itself. Automorphisms are just one interesting
class of homomorphisms we shall present shortly.

The map
\[
 f \left( 
 \begin{pmatrix}
  x\\
  y\\
  z
 \end{pmatrix}
 \right) = 
 \begin{pmatrix}
  x\\
  y\\
  z
 \end{pmatrix}
 +
 \begin{pmatrix}
  1\\
  2\\
  -3
 \end{pmatrix}
\]
is not a homomorphism because it doesn't send $\mathbf{0}$ to $\mathbf{0}$ (and
thus contradicts \myref{lemma}{lem:zero-to-zero}. It serves as a good example of
a more general notion -- homomorphisms \emph{cannot} translate vectors. This is
quite a stark restriction, yet it allows the images of homomorphisms to be
vector spaces.

Also, the map
\[
 f \left( 
  \begin{pmatrix}
   x\\
   y
  \end{pmatrix}
 \right) = x^2 + y^2 - 6
\]
is equally \textbf{not} a homomorphism as it `curves' the space. We may easily
check that it breaks both axioms (1) and (2) in the
\hyperref[def:homomorphism]{definition}. As we've observed by virtue of
\myref{remark}{rmrk:hom-parallelepiped}, homomorphisms send `flat objects' to
`flat objects', not to misshapen ellipses.

The quality of maps we haven't fully commented on is left for kind readers to
determine.

\begin{exercise}{}{}
 Prove that the map $f:\mathcal{P}_3(\F) \to \F^{4}$ from
 \myref{example}{exam:homs} is a homomorphism.
\end{exercise}

\begin{exercise}{}{}
 Prove that the map $f:\R^{2 \times 2} \to \R^2$ from \myref{example}{exam:homs}
 is \textbf{not} an isomorphism.
\end{exercise}

Now, onto learning some new words, kids, shall we?

\begin{definition}{Classes of homomorphisms}{classes-of-homomorphisms}
 A homomorphism $f:V \to W$ is called
 \begin{enumerate}
  \item an \emph{isomorphism}, if it is bijective,
  \item an \emph{endomorphism}, if $W = V$,
  \item an \emph{automorphism}, if it is both an \emph{isomorphism} and an
   \emph{endomorphism}.
 \end{enumerate}
\end{definition}

If there exists an \emph{isomorphism} between two vector spaces $V$ and $W$, we
call these spaces \emph{isomorphic}. Intuitively, this means that the two spaces
behave exactly the same, we have only chosen to represent the vectors of one a
little differently than the other.

One immediate example is the correspondence between polynomials of degree $n$
and vectors with $n + 1$ entries we have mentioned many times throughout the
text.

The map
\begin{align*}
 f: \mathcal{P}_3(\F) &\to \F^{4}\\
 a_0 + a_1x + a_2x^2 + a_3x^3 & \mapsto 
 \begin{pmatrix}
  a_0\\
  a_1\\
  a_2\\
  a_3
 \end{pmatrix}
\end{align*}
from \myref{example}{exam:homs} is an isomorphism between $\mathcal{P}_3(\F)$
and $\F^{4}$. The fact that it's both \emph{injective} and \emph{surjective} is
almost obvious since it just sends a polynomial to its vector of coefficients.

We shan't spend more time discussing different classes of homomorphisms now but
we certainly shall later. The readers are encouraged to make up examples
themselves.

\begin{exercise}{}{}
 Find an example of a homomorphism $f:V \to W$ which is
 \begin{enumerate}[label=(\alph*)]
  \item an \emph{isomorphism} but \textbf{not} an automorphism,
  \item an \emph{endomorphism} but \textbf{not} an automorphism,
  \item an \emph{automorphism}.
 \end{enumerate}
\end{exercise}

Now, there are quite a few special vector spaces tied to a homomorphism $f:V \to
W$. As we've mentioned earlier, its image is a subspace of $W$, the preimage via
$f$ of a vector subspace of $W$ is a subspace of $V$, and, finally, the set of
homomorphisms themselves is a subspace of the vector space of all maps from $V
\to W$. We are proving all these assertions now.

Before that however, we just briefly recall the important definitions. Given a
map $f:V \to W$, its \emph{image} is the set
\[
 \img f = f(V) = \{f(v) \mid v \in V\} \subseteq W.
\]
The \emph{preimage} of a subset $S \subseteq W$ via $f$ is the set
\[
 f^{-1}(S) \coloneqq \{v \in V \mid f(v) \in S\} \subseteq V.
\]
Finally, the set of all homomorphisms from $V$ to $W$ is denoted $\Hom(V,W)$.

\begin{lemma}{}{}
 Let $f:V \to W$ be a homomorphism. Then,
 \begin{enumerate}[label=(\alph*)]
  \item $f(V)$ is a subspace of $W$.
  \item $f^{-1}(U)$ is a subspace of $V$ whenever $U$ is a subspace of $W$.
  \item $Hom(V,W)$ is a subspace of the vector space of all maps $V \to W$.
 \end{enumerate}
\end{lemma}
\begin{lemproof}
 To prove $(a)$, pick two vectors $\mathbf{w}_1,\mathbf{w}_2 \in f(V)$. We shall
 prove that $t_1 \cdot \mathbf{w}_1 + t_2 \cdot \mathbf{w}_2 \in f(V)$. This
 will mean that $f(V)$ is a subspace by
 \myref{lemma}{lem:characterisation-of-subspaces}. Since
 $\mathbf{w}_1,\mathbf{w}_2 \in f(V)$, there exist by definition vectors
 $\mathbf{v}_1,\mathbf{v}_2 \in V$ such that $f(\mathbf{v}_1) = \mathbf{w}_1$
 and $f(\mathbf{v}_2) = \mathbf{w}_2$. We thus rewrite
 \[
  t_1 \cdot \mathbf{w}_1 + t_2 \cdot \mathbf{w}_2 = t_1 \cdot f(\mathbf{v}_1) +
  t_2 \cdot f(\mathbf{v}_2).
 \]
 Since $f$ is a homomorphism by assumption, the right side of the above equality
 can by virtue of \myref{lemma}{lem:hom-linear-combination} be reshaped as such:
 \[
  t_1 \cdot f(\mathbf{v}_1) + t_2 \cdot f(\mathbf{v}_2) = f(t_1 \cdot
  \mathbf{v}_1 + t_2 \cdot \mathbf{v}_2)
 \]
 and thus $f$ sends the vector $t_1 \cdot \mathbf{v}_1 + t_2 \cdot \mathbf{v}_2$
 to $t_1 \cdot \mathbf{w}_1 + t_2 \cdot \mathbf{w}_2$. This, in particular,
 ascertains that $t_1 \cdot \mathbf{w}_1 + t_2 \cdot \mathbf{w}_2 \in f(V)$, as
 desired.

 We continue with statement $(b)$. Pick $\mathbf{v}_1,\mathbf{v}_2 \in
 f^{-1}(U)$. By definition of $f^{-1}(U)$, there exist
 $\mathbf{u}_1,\mathbf{u}_2 \in U$ such that $f(\mathbf{v}_1) = \mathbf{u}_1$
 and $f(\mathbf{v}_2) = \mathbf{u}_2$. We thus compute
 \[
  f(t_1 \cdot \mathbf{v}_1 + t_2 \cdot \mathbf{v}_2) = t_1 \cdot f(\mathbf{v}_1)
  + t_2 \cdot f(\mathbf{v}_2) = t_1 \cdot \mathbf{u}_1 + t_2 \cdot \mathbf{u}_2.
 \]
 The last linear combination lies in $U$ as it is a subspace by assumption. It
 follows that $t_1 \cdot \mathbf{v}_1 + t_2 \cdot \mathbf{v}_2 \in f^{-1}(U)$
 since $f(t_1 \cdot \mathbf{v}_1 + t_2 \cdot \mathbf{v}_2) \in U$.

 As for $(c)$, we are given two homomorphisms $f,g \in \Hom(V,W)$. We must prove
 that $a \cdot f + b \cdot g$ is also a homomorphism for any $a,b \in \F$. For a
 change, we prove the axioms (1) and (2) in the
 \hyperref[def:homomorphism]{definition of homomorphism}. Taking
 $\mathbf{v}_1,\mathbf{v}_2 \in V$, we compute
 \begin{align*}
  (a \cdot f + b \cdot g)(\mathbf{v}_1 + \mathbf{v}_2) 
  &= (a \cdot f)(\mathbf{v}_1 + \mathbf{v}_2) + (b \cdot g)(\mathbf{v}_1 +
  \mathbf{v}_2)\\
  &= f(a \cdot \mathbf{v}_1 + a \cdot \mathbf{v}_2) + g(b \cdot \mathbf{v}_1 +
  b \cdot \mathbf{v}_2)\\
  &= a \cdot f(\mathbf{v}_1) + a \cdot f(\mathbf{v}_2) + b \cdot g(\mathbf{v}_1)
  + b \cdot g(\mathbf{v}_2)\\
  &= (a \cdot f(\mathbf{v}_1) + b \cdot g(\mathbf{v}_1)) + (a \cdot
  f(\mathbf{v}_2) + b \cdot g(\mathbf{v}_2))\\
  &= (a \cdot f + b \cdot g)(\mathbf{v}_1) + (a \cdot f + b \cdot
  g)(\mathbf{v}_2),
 \end{align*}
 hence (1) holds. The proof of (2) is left as an exercise. The validity of both
 axioms ascertains that $\Hom(V,W)$ is really a vector space and thus a subspace
 of the space of all maps $V \to W$.
\end{lemproof}

\begin{exercise}{}{}
 Prove that for two homomorphisms $f,g \in \Hom(V,W)$, $a,b \in \F$, $\mathbf{v}
 \in V$ and $t \in \F$, we have
 \[
  (a \cdot f + b \cdot g)(t \cdot \mathbf{v}) = t \cdot (a \cdot f + b \cdot
  g)(\mathbf{v}).
 \]
\end{exercise}

The fact that images and preimages of homomorphisms are vector spaces have
geometric consequences. We now illustrate those on a few examples.

\begin{example}{}{}
 Consider the projection $\pi:\R^3 \to \R^2$ prescribed as
 \[
  \pi \left( 
  \begin{pmatrix}
   x\\
   y\\
   z
  \end{pmatrix}
  \right) = 
  \begin{pmatrix}
   x\\
   y
  \end{pmatrix}.
 \]
 The image of $\pi(\R^3)$ is of course the entirety of $\R^2$. One may think of
 this projection as the `squishing' of an entire room onto its floor. If we're
 allowed to move only along one of the walls, we're confined to just one edge of
 the floor. This is formalised by the fact that the image of the set of vectors
 with $y = 0$ is the subspace of $\R^2$ of vectors with $y = 0$, and similarly
 for $x$.

 \begin{center}
  \clr{TODO obrazky}
 \end{center}

 The preimage of a given vector $\mathbf{w} \in \R^2$ via $\pi$ is the set of
 all vectors whose tips sit on the vertical line rooted at the tip of
 $\mathbf{w}$. This is because any vector
 \[
  \begin{pmatrix}
   w_1\\
   w_2\\
   z
  \end{pmatrix}
 \]
 gets mapped onto
 \[
  \mathbf{w} = 
  \begin{pmatrix}
   w_1\\
   w_2
  \end{pmatrix}.
 \]
 Of course, a set consisting of a single non-zero vector is not a subspace, just
 as the vertical line rooted at the tip of such vector is not. 

 On a similar note, the preimage of a line $\{t \cdot \mathbf{v} \mid t \in
 \R\}$ for a given $\mathbf{v} \in \R^2$ is the entire vertical plane containing
 said line in $\R^3$.
\end{example}

\begin{example}{}{}
 The image of the homomorphism
 \begin{align*}
  h: \R^2 &\to \R\\
  \begin{pmatrix}
   x\\
   y
  \end{pmatrix}
          & \mapsto 
          x + y
 \end{align*}
 is the entirety of $\R$. Fixing a number $r \in \R$, its preimage via $h$ is
 the set of vectors whose coordinates add up to $r$. Their tips form a line in
 $\R^2$ given by the equation $x + y = r$.

 The fact that $h$ is a homomorphism can be expressed in a neat geometric manner
 -- fix now two numbers $r_1,r_2 \in \R$. Think of the vectors whose tips lie on
 the line $x + y = r_1$ as `$r_1$' vectors. Analogously, vectors whose tips form
 the line $x + y = r_2$ are regarded as `$r_2$' vectors. Axiom (1) in the
 \hyperref[def:homomorphism]{definition of homomorphism} can be, in this
 particular case, restated as `$r_1$' vectors plus `$r_2$' vectors equal `$r_1 +
 r_2$' vectors. That is, $\mathbf{v}_1$ is an `$r_1$' vector if $h(\mathbf{v}_1)
 = r_1$ and $\mathbf{v}_2$ is an `$r_2$' vector if $h(\mathbf{v}_2)$; the
 previous sentence thus signifies exactly that $h(\mathbf{v}_1 + \mathbf{v}_2) =
 h(\mathbf{v}_1) + h(\mathbf{v}_2)$ since $\mathbf{v}_1 + \mathbf{v}_2$ is
 clearly a `$r_1 + r_2$' vector.
 \begin{center}
  \clr{TODO obrazek}
 \end{center}
\end{example}

\begin{example}{}{der-int}
 Define the `derivative' homomorphism
 \begin{align*}
  \frac{\partial }{\partial x}: \mathcal{P}_3 &\to \mathcal{P}_3\\
  a_0 + a_1x + a_2x^2 + a_3x^3 & \mapsto a_1 + 2a_2x + 3a_3x^2.
 \end{align*}
 Its image is the subspace $\mathcal{P}_2$ (polynomials of degree $2$ over $\R$)
 of $\mathcal{P}_3$. The inverse image of set of polynomials of degree $2$ is
 the `integral' homomorphism (with constant $0$). Said formally, there exists a
 homomorphism
 \begin{align*}
  \int: \mathcal{P}_2 & \to \mathcal{P}_3\\
  a_0 + a_1x + a_2x^2 & \mapsto a_0x + \frac{a_1}{2}x^2 + \frac{a_2}{3}x^3
 \end{align*}
 such that $\frac{\partial }{\partial x} \circ \int $ is the identity map on
 $\mathcal{P}_2$. Notice however that it is \textbf{not} the case that $\int
 \circ \frac{\partial }{\partial x}$ is the identity map on $\mathcal{P}_3$. The
 derivative $\frac{\partial }{\partial x}$ is \textbf{not} an injective map and
 as such it doesn't have a `two-sided' inverse, it only has a `right' inverse in
 the form of $\int $.
\end{example}

The last examples provokes a question: `When does a homomorphism have an inverse
which is also a homomorphism?' We remind dear readers that an inverse to a map
$f:V \to W$ is a map $f^{-1}:W \to V$ such that $(f \circ f^{-1})(\mathbf{w}) =
\mathbf{w}$ for every $\mathbf{w} \in W$ and $(f^{-1} \circ f)(\mathbf{v}) =
\mathbf{v}$ for every $\mathbf{v} \in V$. As seen in the mentioned
\myref{example}{exam:der-int} \textbf{just one} of these equalities \textbf{is
not enough}. A homomorphism may be invertible from just one side.

The rest of the introductory section to homomorphisms is dedicated to answering
this question. First, we must establish a strong connection between
homomorphisms and bases of their domains. We illustrate this connection first
before conjuring a proof.

Consider a homomorphism $f$ with domain $\R^3$. By
\myref{lemma}{lem:hom-linear-combination}, the image of a vector
\[
 \begin{pmatrix}
  x\\
  y\\
  z
 \end{pmatrix}
\]
via this homomorphism can be broken into the linear combination
\[
 f \left( 
 \begin{pmatrix}
  x\\
  y\\
  z
 \end{pmatrix}
 \right) = f \left( 
 \begin{pmatrix}
  x\\
  0\\
  0
 \end{pmatrix}
 \right) + f \left( 
 \begin{pmatrix}
  0\\
  y\\
  0
 \end{pmatrix}
 \right) + f \left( 
 \begin{pmatrix}
  0\\
  0\\
  z
 \end{pmatrix}
 \right) = x \cdot 
 f \left( 
 \begin{pmatrix}
  1\\
  0\\
  0
 \end{pmatrix}
 \right) + y \cdot f \left(
 \begin{pmatrix}
  0\\
  1\\
  0
 \end{pmatrix}
 \right) + z \cdot f \left( 
 \begin{pmatrix}
  0\\
  0\\
  1
 \end{pmatrix}
 \right).
\]
This immediately suggests that the image of any vector via a homomorphism is
determined purely by the images of \hyperref[def:standard-basis]{standard basis}
vectors. Indeed, this statement is true in general, for any homomorphism and any
basis.

\begin{theorem}{}{}
 Let $V,W$ be vector spaces over $\F$, $B = (\mathbf{b}_1,\ldots,\mathbf{b}_n)$
 be a basis of $V$. Given (not necessarily distinct) vectors
 $\mathbf{w}_1,\mathbf{w}_2,\ldots,\mathbf{w}_n \in W$, there exists
 \textbf{just a single} homomorphism $f \in \Hom(V,W)$ such that
 $f(\mathbf{b}_1) = \mathbf{w}_1, f(\mathbf{b}_2) =
 \mathbf{w}_2,\ldots,f(\mathbf{b}_n) = \mathbf{w}_n$.
\end{theorem}
\begin{thmproof}
 First, given $\mathbf{v} \in V$, it can be written uniquely as a linear
 combination of vectors from $B$, that is, there exist unique coefficients
 $t_1,\ldots,t_n \in \F$ such that
 \[
  \mathbf{v} = t_1 \cdot \mathbf{b}_1 + \ldots + t_n \cdot \mathbf{b}_n.
 \]
 We define $f(\mathbf{v})$ by
 \[
  f(\mathbf{v}) = t_1 \cdot \mathbf{w}_1 + \ldots + t_n \cdot \mathbf{w}_n.
 \]
 We must prove three statements:
 \begin{enumerate}
  \item $f$ is well-defined (i.e. it's \emph{actually} a map $V \to W$);
  \item $f$ is a homomorphism;
  \item $f$ is unique.
 \end{enumerate}
 The statement (1) follows from the uniqueness of representation of a vector
 with respect to a basis -- content of
 \myref{theorem}{thm:characterisation-of-a-basis}. Consequently, as the vector
 $\mathbf{v}$ can only ever be represented by the same $n$-tuple of
 coefficients, the image of $\mathbf{v}$ under $f$ is always the same.

 Ad (2), choose $\mathbf{v}_1,\mathbf{v}_2 \in V$ and $t_1,t_2 \in \F$. We
 calculate
 \begin{align*}
  
 \end{align*}
\end{thmproof}
