\chapter{Homomorphisms}
\label{chap:homomorphisms}

In this chapter, our aim is to study and understand maps between vector spaces.
Not just any kind of maps, however, but maps that \emph{preserve structure}.

Most of modern mathematics is dedicated to the study of \emph{structures} --
basically prescribed rules of interaction between elements of a set. We call
these rules, \emph{operations}, and when moving from a set with structure to a
set with structure by a map, we tend to require that said map somehow respects
the structures of both sets. Such maps are often called \emph{homomorphisms},
from Greek ὁμός (\uv{same}) and μορφή (\uv{form, shape}).

The only structure we consider in this book is that of a vector space given by
two operations: scalar multiplication and vector addition. A \emph{homomorphism
between vector spaces} $V$ and $W$ (also called a \emph{linear map}) is thus a
map which respects both operations; in practice, this means that the image of a
scalar multiple should be the same scalar multiple of the image and that the
image of a sum of vectors should be the sum of the images.

One last note: we ought to be careful when comparing two structures. We labelled
the operations on a vector space by symbols $ \cdot $ and $+$ but these two
symbols \textbf{mean different things in different vector spaces}! To keep the
text tidy, we shan't resort to using yet another distinct pair of symbols.
However, we \emph{are} going to distinguish the structure in a small number of
ensuing lemmata and definitions, to drive the point home.

\begin{definition}{Homomorphism}{homomorphism}
 Let $V$ and $W$ be vector spaces over the field $\F$. We denote the operations
 of scalar multiplication and vector addition on $V$ by $\clr{ \cdot _V}$ and
 $\clr{ +_V}$ and those on $W$ by $\clb{ \cdot _W}$ and $\clb{+_W}$. A map $f:V
 \to W$ is a \emph{homomorphism} (or a \emph{linear map}) if
 \begin{enumerate}
  \item $f(\mathbf{v}_1~\clr{+_V}~\mathbf{v}_2) =
   f(\mathbf{v}_1)~\clb{+_W}~f(\mathbf{v}_2)$ for every two vectors
   $\mathbf{v}_1,\mathbf{v}_2 \in V$.
  \item $f(t~\clr{ \cdot _V}~\mathbf{v}) = t~\clb{ \cdot_W}~f(\mathbf{v})$ for
   every $t \in \F$ and $\mathbf{v} \in V$.
 \end{enumerate}
\end{definition}

\begin{example}{}{}
 The following maps are homomorphisms:
 \begin{enumerate}[label=(\alph*)]
  \item the map $f:\R^2 \to \R^2$ given by $f(\mathbf{v}) = 2 \cdot \mathbf{v}$;
  \item the map $f:\mathcal{P}_3(\F) \to \F^{4}$ given by
   \[
    f(a_0 + a_1x + a_2x^2 + a_3x^3) = 
    \begin{pmatrix}
     a_0\\
     a_1\\
     a_2\\
     a_3
    \end{pmatrix},
   \]
   where $\mathcal{P}_3(\F)$ denotes the space of polynomials of degree $3$ with
   coefficients in the field $\F$;
  \item the map $\pi: \R^3 \to \R^2$ given by
   \[
    \pi \left( 
    \begin{pmatrix}
     x\\
     y\\
     z
    \end{pmatrix}
    \right) = 
    \begin{pmatrix}
     x\\
     y
    \end{pmatrix};
   \]
   maps that `forget coordinates' are often called \emph{projections}.
 \end{enumerate}
 The following maps are \textbf{not} homomorphisms:
 \begin{enumerate}[label=(\alph*)]
  \item the map $f:\R^3 \to \R^3$ given by
   \[
    f \left( 
    \begin{pmatrix}
     x\\
     y\\
     z
    \end{pmatrix}
    \right) = 
    \begin{pmatrix}
     x\\
     y\\
     z
    \end{pmatrix}
    +
    \begin{pmatrix}
     1\\
     2\\
     -3
    \end{pmatrix};
   \]
  \item the map $f: \R^2 \to \R$ given by
   \[
    f \left( 
    \begin{pmatrix}
     x\\
     y
    \end{pmatrix}
    \right) = x^2 + y^3 - 6.
   \]
  \item the map $f:\R^{2 \times 2} \to \R^2$ given by
   \[
    f \left( 
    \begin{pmatrix}
     a & b\\
     c & d
    \end{pmatrix}
    \right) = 
    \begin{pmatrix}
     a \cdot b + c \cdot d\\
     a \cdot d - b \cdot c
    \end{pmatrix}.
   \]
 \end{enumerate}
\end{example}

In the previous example, we claimed that certain maps were homomorphisms without
giving a proof. We did so because we first want to provide a characterisation of
homomorphisms which makes checking whether a given map is a homomorphism
somewhat easier. Hence, we now collect two qualities only homomorphisms possess.

\begin{lemma}{Zero to zero}{zero-to-zero}
 Let $f:V \to W$ be a homomorphism and label the zero vector of $V$ by
 $\clr{\mathbf{0}_V}$ and the zero vector of $W$ by $\clb{\mathbf{0}_W}$. Then,
 \[
  f(\clr{\mathbf{0}}_V) = \clb{\mathbf{0}_W}.
 \]
\end{lemma}
\begin{lemproof}
 Exploiting axiom (2) in the \hyperref[def:homomorphism]{definition of
 homomorphism}, we get
 \[
  f(\clr{\mathbf{0}_V}) = f(0~\clr{ \cdot _V}~\clr{\mathbf{0}_V})
  \overset{(2)}{=} 0~\clb{ \cdot _W}~f(\clr{\mathbf{0}_V}) = \clb{\mathbf{0}_W}
 \]
 as required.
\end{lemproof}
