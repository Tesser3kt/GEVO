\subsection{Gauss-Jordan Elimination Revisited}
\label{ssec:gauss-jordan-elimination-revisited}

Here, we provide a fully algorithmic description of the Gauss-Jordan elimination
algorithm discussed in \myref{section}{sec:gauss-jordan-elimination} and also
one possible way of encoding it in SageMath. Here goes nothing.
\begin{algorithm}[H]
 \SetAlgoVlined
 \SetKwInOut{Input}{input}
 \SetKwInOut{Output}{output}

 \Input{An $n \times m$ matrix $A = (a_{i,j})_{i=1,j=1}^{n,m}$ with real
 entries.}
 \Output{The matrix $A$ in echelon form.}
 \BlankLine
 \tcc{Row to be used for elimination of other rows.}
 $r \leftarrow 1$;\\

 \BlankLine
 \tcc{Traverse the columns.}
 \For {$c \in \{1,\ldots,m\}$}{
  \tcc{Find the row (below $r$) with maximal value in column $c$. Denote by $b$
  the row with the maximal currently known value.}
  $b \leftarrow r$;\\
  \tcc{Traverse the rows below $r$.}
  \For {$i \in \{r+1,\ldots,n\}$} {
   \If {$|a_{i,c}| > |a_{b,c}|$} {
    \tcc{Found a row with higher value in column $c$. Replace $b$ with $i$.}
    $b \leftarrow i$;
   }
  }
  \tcc{If $a_{b,c} = 0$, then move to next column since this column is full of
  zeroes.}
  \If {$a_{b,c} = 0$} {
   continue;
  }
  \BlankLine
  swap rows with indices $r$ and $b$;\\
  \tcc{Eliminate variables in column $c$ in all rows below $r$.}
  \For {$i \in \{r+1,\ldots,n\}$} {
   \For {$j \in \{c,\ldots,m\}$} {
    $a_{i,j} \leftarrow a_{i,j} - \frac{a_{i,c}}{a_{r,c}} a_{r,j}$;
   }
  }
  \BlankLine
  \tcc{Row $r$ now contains the pivot in column $c$ so it will remain the same
  for the rest of the algorithm. Move to the next row.}
  $r \leftarrow r + 1$;
 }
 \BlankLine
 \Return{$A$};
 \caption{Gauss-Jordan Elimination.}
 \label{algo:gauss-jordan-elimination}
\end{algorithm}

\begin{example}{}{gauss-jordan-elimination}
 Let's put the matrix
 \[
  A = \left( 
   \begin{matrix*}[r]
    2 & 0 & 1\\
    -1 & 1 & 1\\
    4 & 2 & 2
   \end{matrix*}
  \right)
 \]
 into echelon form using \myref{algorithm}{algo:gauss-jordan-elimination}. At
 first, we have $r = 1$ and $c = 1$. Going through rows $2$ and $3$, we see that
 the number with the highest value in column $1$ lies in row $3$. Hence, we
 first swap row $1$ with row $3$.
 \[
  A = \left( 
   \begin{matrix*}[r]
    4 & 2 & 2\\
    -1 & 1 & 1\\
    2 & 0 & 1
   \end{matrix*}
  \right)
 \]
 Now begins the process of elimination. Since $r = 1$, the index $i$ exhausts
 the set $\{2,3\}$. For $i = 2$, we calculate $a_{i,c} / a_{r,c}$ = $a_{2,1} /
 a_{1,1} = -1 / 4$. We thus subtract $(-1 / 4)$-multiple of row $1$ from row
 $2$. 
 \[
  A = \left( 
   \begin{matrix*}[r]
    4 & 2 & 2\\
    0 & \frac{3}{2} & \frac{3}{2}\\
    2 & 0 & 1
   \end{matrix*}
  \right)
 \]
 Next, we set $i \coloneqq 3$ and calculate $a_{i,c} / a_{r,c} = 1 / 2$; we then
 subtract $(1 / 2)$-multiple of row $1$ from row $3$.
 \[
  A = \left( 
   \begin{matrix*}[r]
    4 & 2 & 2\\
    0 & \frac{3}{2} & \frac{3}{2}\\
    0 & -1 & 0
   \end{matrix*}
  \right)
 \]
 Since all rows in column $c$ below $r$ have been eliminated, we move to the
 next row by setting $r \coloneqq 2$. We also move to the next column via $c
 \coloneqq 2$.

 Now, the number with the largest absolute value in column $c$ and all rows
 below (and including) $r$ already lies in row $r$, so no swap is needed. We
 perform the elimination of row $3$ by calculating $a_{3,c} / a_{r,c} = a_{3,2}
 / a_{2,2} = - 2 / 3$ and subtracting the $(-2 / 3)$-multiple of row $2$ from
 row $3$.
 \[
  A = \left( 
   \begin{matrix*}[r]
    4 & 2 & 2\\
    0 & \frac{3}{2} & \frac{3}{2}\\
    0 & 0 & 1
   \end{matrix*}
  \right)
 \]
 We move to the next row via $r \coloneqq 3$ and the next column via $c
 \coloneqq 3$. No further elimination takes place because there are no rows
 below row $3$. The matrix $A$ has been put into echelon form.
\end{example}

Before finally marching on, we present a way of implementing
\myref{algorithm}{algo:gauss-jordan-elimination} in SageMath. This
implementation almost aligns with the implementation of the algorithm in Python.
Let us be responsible adults and break it into functions.

The function for finding the row with the highest pivot coefficient in the
current row might look like this:
\begin{Verbatim}
sage: \clg{def} \clr{find_best_row}(A: \clb{Matrix}, cur_row: \clb{int}, cur_col: \clb{int}):
....:     best_row = cur_row
....:
....:     \clg{for} row_below \clg{in} \clr{range}(cur_row + 1, \clr{len}(A.\clr{rows}())):
....:         if \clr{abs}(A[row_below][cur_col]) > \clr{abs}(A[best_row][cur_col]):
....:             best_row = row_below
....:
....:     \clg{return} best_row
\end{Verbatim}
We also implement the function to eliminate the first non-zero element in all
rows below a given row.
\begin{Verbatim}
sage: \clg{def} \clr{eliminate_rows}(A: \clb{Matrix}, cur_row: \clb{int}, cur_col: \clb{int}):
....:     \clg{for} row_below \clg{in} \clr{range}(cur_row + 1, \clr{len}(A.\clr{rows}())):
....:         scalar = A[row_below][cur_col] / A[cur_row][cur_col]
....:         A[row_below] = A[row_below] - scalar * A[cur_row]
\end{Verbatim}
These are all the functions we need to cleanly implement Gauss-Jordan
elimination in SageMath.
\begin{Verbatim}
sage: \clg{def} \clr{eliminate}(A: \clb{Matrix}):
....:     cur_row = \clm{0}
....:
....:     \clg{for} cur_col \clg{in} \clr{range}(\clr{len}(A.\clr{columns}())):
....:         best_row = \clr{find_best_row}(A, cur_row, cur_col)
....:         A[cur_row], A[best_row] = A[best_row], A[cur_row]
....:
....:         \clg{if} A[cur_row][cur_col] == \clm{0}:
....:             \clg{continue}
....:
....:         \clr{eliminate_rows}(A, cur_row, cur_col)
....:         cur_row += \clm{1}
\end{Verbatim}
And, to wrap things up, a short application on the matrix from
\myref{example}{exam:gauss-jordan-elimination}.
\begin{Verbatim}
sage: A = \clb{Matrix}(QQ, [
....:     [\clr{2}, \clr{0}, \clr{1}],
....:     [\clr{-1}, \clr{1}, \clr{1}],
....:     [\clr{4}, \clr{2}, \clr{2}],
....: ])
sage: \clr{eliminate}(A)
sage: A
[  \clr{4}   \clr{2}   \clr{2}]
[  \clr{0} \clr{3/2} \clr{3/2}]
[  \clr{0}   \clr{0}   \clr{1}]
\end{Verbatim}
