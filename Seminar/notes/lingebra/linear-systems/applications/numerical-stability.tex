\subsection{Numerical Stability}
\label{ssec:numerical-stability}

Numerical stability (of a linear system) refers to one of its computational
qualities -- the quality described best as `small change in input causes a small
change in output'. As real numbers are represented in computer memory with a
given precision (more or less the number of decimal places), deviations in input
data small enough to go unnoticed may cause issues. We shall highlight two of
said `issues' (and possible countermeasures) in this subsection.

Consider the system
\begin{equation}
 \label{eq:same-eq-twice}
 \begin{array}{r c r c r}
  2x & + & y & = & 3\\
  2x & + & y & = & 3
 \end{array}
\end{equation}
with infinitely many solutions of the form $((3-y) / 2, y)$. Now, altering the
system slightly
\[
 \begin{array}{r c r c r}
  2x & + & y & = & 3\\
  2.000000002x & + & 1.000000001y & = & 3.000000003
 \end{array}
\]
yields a system with exactly one solution -- $(1,1)$. We see that immediately
but a computer with limited precision might regard this altered system exactly
the same way as the previous one. Should we draw the system, we would basically
see just one line given that the size of the angle between the lines
corresponding to the two equations is negligible.

Systems where two or more equations are indistinguishable with low enough
precision are typically called \emph{ill-conditioned}. In this case, there is
not much that can be done to alleviate the problem. See for yourself.
\begin{Verbatim}
sage: A = Matrix(RR, [
....:     [\clr{2}, \clr{1}],
....:     [\clr{2} + \clb{2*10**-18}, \clr{1} + \clb{10**-18}],
....: ])
sage: b = vector(RR, [\clr{3}, \clr{3} + \clb{3*10**-18}])
sage: A.solve_right(b)
(\clr{1.50000000000000}, \clr{0.000000000000000})
\end{Verbatim}
The solution given by SageMath is clearly wrong because of the \clb{tiny
deviation} in input data. It instead computed the solution to the
system~\eqref{eq:same-eq-twice} and substituted $y = 0$, which is default
behaviour.

Next, we take a look at the system
\[
 \begin{array}{r c r c r}
  \frac{1}{1000}x & + & y & = & 1\\
  x & - & y & = & 0
 \end{array}
\]
with unique solution $(1000 / 1001, 1000 / 1001)$. Here, depending on the order
of the equations, computers can arrive at a wrong solution. In the first step of
Gauss-Jordan elimination, we subtract a $1000$-multiple of row \texttt{I} from
row \texttt{II}, obtaining
\begin{equation}
 \label{eq:wrong-order}
 \begin{array}{r c r c r}
  \frac{1}{1000}x & + & y & = & 1\\
  & & -1001 y & = & -1000.
 \end{array}
\end{equation}
Even if we are working with enough precision to represent thousandths of
integers, the result of the computation
\[
 y = \frac{-1000}{-1001}
\]
may easily be rounded to $1$ due to how computers perform division. As three
decimal places are hardly enough to push modern computers to their limits, see
the following example instead.
\begin{Verbatim}
sage: a = \clr{-1 * 10**18}
sage: b = \clr{-1 * 10**18 - 1}
sage: \clb{numerical_approx}(a / b)
\clr{1.00000000000000}
\end{Verbatim}
The \texttt{numerical\_approx} function tells SageMath to represent $a / b$ as a
real number, otherwise it would store it as a fraction.

Should we know begin the process of back-substitution in the
system~\eqref{eq:wrong-order}, we would inevitably get a wrong solution. If the
second equation yields (with low precision) that $y = 1$, then from the first
equation, we get $x = 0$. This is a \emph{completely} different solution from
the original. The difference between $(0,1)$ and $(1000 / 1001, 1000 / 1001)$
might not seem too high but imagine $x$ and $y$ represented \emph{percentages}
for example. Then, instead of both $x$ and $y$ being nearly $100\%$, $x$ gets
smashed down all the way to $0 \%$.

Perhaps a little surprisingly, this problem can be \emph{thoroughly} solved by
simply changing the order of the equations. If we instead used Gauss-Jordan
elimination to solve the system
\[
 \begin{array}{r c r c r}
  x & - & y & = & 0\\
  \frac{1}{1000}x & + & y & = & 1,
 \end{array}
\]
we would not run into any issues. Indeed, the first step here entails
subtracting $(1 / 1000)$-multiple of row \texttt{I} from row \texttt{II}. This
yields
\[
 \begin{array}{r c r c r}
  x & - & y & = & 0\\
  & & \frac{1001}{1000}y & = & 1.
 \end{array}
\]
