\subsection{Input-Output Analysis}
\label{ssec:input-output-analysis}

A place where linear systems naturally flourish is \emph{economics}. Put
briefly, economy is a network of mutually influenced industries. An important
observation is that this `influence' is mostly of \emph{linear} nature. We take
as an example the \emph{steel} and \emph{automobile} industries. Both of these
industries use its own output and the other industry's output to optimize
production. The steel industry might use steel to produce factories, and use
cars for the transport of goods between them. Similarly, the automobile industry
uses its own cars to transports its other cars and uses steel to produce them in
the first place. In economics, we're typically interested in predicting the
future value of an industry. However, in cases like these, it isn't intuitively
evident how the total value of steel used by external actors (meaning not the
steel or automobile industries) would influence the system, for example.

Suppose we accumulated the following data:
\begin{table}[ht]
 \centering
 \begin{tabular}{r | r | r | r | r}
  & \textbf{used by steel} & \textbf{used by auto} & \textbf{used by others} &
  \textbf{total}\\
  \toprule
  value of steel (in billions of \$) & 6.90 & 1.28 & 10.51 & 18.69 \\
  value of auto (in billions of \$) & 2.24 & 4.40 & 7.63 & 14.27
 \end{tabular}

 \caption{The annual summary of the value of steel and automobile industries.}
 \label{table:steel-and-automobile}
\end{table}

Based on this data, how ought we to attempt to predict the total values of steel
and automobile industries based on shifting external demand? First and foremost,
why do we care primarily about external demand? The answer is simple. As long as
external demand stays stable, it is improbable that the automobile industry
would suddenly produce more cars or that the steel industry more steel. It is
indeed mostly individual customers and other affiliated industries which cause a
change in production.

Suppose that the value of steel and automobile industries used externally in the
next year shifts by $d_s$ and $d_a$, respectively. How does this affect their
total value? To answer this, we need observe that the steel and automobile
industries form a linear system. Under the premise that the steel industry uses
the same \emph{fraction} of its own output and the automobile industry also uses
the same fraction of the steel industry output as this year, we can predict its
value next year (which we denote $s$) to equal
\[
 s = (6.90 / 18.69)s + (1.28 / 14.27)a + (10.51 + d_s).
\]
This formula essentially says the obvious:
\begin{align*}
 \text{next year's value of steel} &= \text{next year's value of steel used by
 steel}\\
                                   &+ \text{next year's value of steel used by
                                   auto}\\
                                   &+ \text{next year's value of steel used by
                                   others}.
\end{align*}
We are just predicting the next year values based on this year's ones while
keeping the ratios of output distribution stable.

Similarly, the equation for the predicted next year's total automobile industry
value (denoted $a$) is
\[
 a = (2.24 / 18.69)s + (4.40 / 14.27)a + (7.63 + d_a).
\]
Both of these linear equations put together form the linear system
\[
 \begin{array}{r c r c r c r}
  s & = & (6.90 / 18.69)s & + & (1.28 / 14.27)a & + & (10.51 + d_s)\\
  a & = & (2.24 / 18.69)s & + & (4.40 / 14.27)a & + & (7.63 + d_a)
 \end{array}
\]
An easy computation and rearrangement gives
\[
 \begin{array}{r c r c r}
  (11.79 / 18.69)s & - & (1.28 / 14.27)a & = & (10.51 + d_s)\\
  -(2.24 / 18.69)s & + & (9.87 / 14.27)a & = & (7.63 + d_a)
 \end{array}
\]
As we did many a time already, we collect the equations into a matrix $A$ and a
vector $b$ like so:
\[
 A = \left( 
  \begin{matrix*}[r]
   11.79 / 18.69 & -1.28 / 14.27\\
   -2.24 / 18.69 & 9.87 / 14.27
  \end{matrix*}
 \right), \quad \mathbf{b} = \left( 
  \begin{matrix*}[r]
   10.51 + d_s\\
   7.63 + d_a
  \end{matrix*}
 \right).
\]
Fortunately, SageMath has in-built support for variables. We can thus let it
solve the system for us and represent the solution in terms of variables $d_s$
and $d_a$.
\begin{Verbatim}
sage: \clr{var}(\clbr{'ds da'})
(\clbr{ds}, \clbr{da})
sage: A = \clb{Matrix}([
....:     [\clr{11.79/18.69}, \clr{-1.28/14.27}],
....:     [\clr{-2.24/18.69}, \clr{9.87/14.27}],
....: ])
sage: b = \clb{vector}([\clr{10.51} + \clbr{ds}, \clr{7.63} + \clbr{da}])
sage: sol = A.\clr{solve_right}(b)
(\clr{0.210776906804487}*\clbr{da} + \clr{1.62528755481273}*\clbr{ds} + \clr{18.6900000000000},
\clr{1.48231851778104}*\clbr{da} + \clr{0.281627945702251}*\clbr{ds} + \clr{14.2700000000000})
\end{Verbatim}
We can now easily get a solution for \emph{concrete} values of $d_s$ and $d_a$
by using SageMath's symbolic substitution capabilities. For example, if we
expect the external output value of automobile industry will rise by $d_a =
0.05$ and the external output value of steel will fall by $0.10$, that is $d_s =
-0.10$, we can calculate the predicted future total values of the industries by
setting
\begin{Verbatim}
sage: sol(\clbr{da}=\clr{0.05},\clbr{ds}=\clr{-0.10})
(\clr{18.5380100898590}, \clr{14.3159531313188})
\end{Verbatim}
In this case, we predict the total value of the steel industry to fall by about
$\$ 0.15$ billion and the value of the automobile industry to rise by roughly
$\$ 0.045$ billion.

\begin{exercise}{}{input-output}
 Predict next year's total productions of each of the three sectors of the
 hypothetical economy shown in \myref{table}{table:hypothetical-economy}.
 \begin{table}[H]
  \centering
  \begin{tabular}{l | r | r | r | r | r}
   value of / \textbf{used by} & \textbf{farm} & \textbf{rail} &
   \textbf{shipping} & \textbf{others} & \textbf{total}\\
   \toprule
   farm & 25 & 50 & 100 & & 800\\
   rail & 25 & 50 & 50 & & 300\\
   shipping & 15 & 10 & 0 & & 500
  \end{tabular}
  \caption{The output data of a hypothetical economy.}
  \label{table:hypothetical-economy}
 \end{table}
 if next year's external demands are as stated.
 \begin{enumerate}[label=(\alph*)]
  \item 625 for farm, 200 for rail, 475 for shipping,
  \item 650 for farm, 150 for rail, 450 for shipping.
 \end{enumerate}
 Can you solve the system with data presented in (a) and (b) simultaneously by
 making the given external demands into parameters?
\end{exercise}
