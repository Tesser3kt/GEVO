\subsection{Two-dimensional Linear Systems}
\label{ssec:two-dimensional-linear-systems}

We dedicate a section to the visualization of linear systems in two variables
and their solutions. As already established, a linear equation in two variables
represents a \hyperref[def:straight-line]{straight line}. A solution to a linear
system in two variables is a pair of real numbers (equivalently, a point in the
real plane) which lies on every straight line determined by the equations of the
system. Simply put, the solution of a linear system in two variables is the
\emph{intersection} of all objects described by its equations.

An `ideal' linear system in two variables contains two linear equations
describing distinct lines. One such system is
\[
 \begin{array}{r c r c r}
  2x & - & y & = & 1\\
  x & + & y & = & 2
 \end{array}
\]
with solution $(1,1)$ and whose visual depiction is provided in
\myref{figure}{fig:well-determined-system}.

\begin{figure}[ht]
 \centering
 \begin{tikzpicture}
  \tkzInit[xmin=-1,xmax=5,ymin=-1,ymax=2]
  \tkzDrawX
  \tkzDrawY
  \tkzDefPoints{0/0/o,1/1/i,0.5/0/a,2/0/b}
  \tkzDrawLine[color=RoyalBlue,thick,add=0.8 and 1.3](a,i)
  \tkzDrawLine[color=ForestGreen,thick,add=0.8 and 1.3](b,i)

  \tkzDefPoints{1/0/x,0/1/y}
  \tkzDrawPoint[size=6,color=BrickRed](i)
  \tkzLabelPoint[below](x){$1$}
  \tkzLabelPoint[left](y){$1$}
  \tkzLabelPoint[right=2mm,color=BrickRed](i){$(1,1)$}
  \tkzDrawSegments[dashed,thick,color=BrickRed](i,x i,y)
  \tkzDrawPoints[size=4](x,y)
 \end{tikzpicture}

 \caption{Well-determined linear system in two variables with solution
 $\clr{(1,1)}$.}
 \label{fig:well-determined-system}
\end{figure}

An easily proven fact (which we shall eventually prove in greater generality)
that follows immediately from the geometric view reads that a linear system in
two variables with two \emph{distinct} linear equations always has a solution --
the intersection point of the corresponding lines.

A linear system in two variables can only be underdetermined should it feature
just one non-trivial linear equation (or, equivalently, many identical linear
equations). In this case, assuming the system consists of the single linear
equation
\[
 ax + by = c,
\]
its solution set is spanned by the points $(x, (c - ax) / b)$, for $x \in \R$,
or $(c / a, y)$, for $y \in \R$, should $b = 0$. Geometrically, all points lying
on the line determined by its sole equation solve the underdetermined linear
system.

Overdetermined linear systems in two variables are considerably more
interesting. There are four possible arrangements of three lines in the plane,
they're depicted in \myref{figure}{fig:arrangement-of-lines}.

\begin{figure}[ht]
 \centering
 \begin{subfigure}[b]{.45\textwidth}
  \centering
  \begin{tikzpicture}[scale=0.75]
   \tkzInit[xmin=-1,xmax=5,ymin=-1,ymax=3]
   \tkzDrawX
   \tkzDrawY

   \tkzDefPoints{0/0.5/a1,1/0.5/a2,0/1/b1,1/1/b2,0/2/c1,1/2/c2}
   \tkzDrawLines[thick,add=1 and 3,color=RoyalBlue](a1,a2 b1,b2 c1,c2)
  \end{tikzpicture}
  \caption{All three lines parallel.}
  \vspace*{1em}
 \end{subfigure}
 \hfill
 \begin{subfigure}[b]{.45\textwidth}
  \centering
  \begin{tikzpicture}[scale=0.75]
   \tkzInit[xmin=-1,xmax=5,ymin=-1,ymax=3]
   \tkzDrawX
   \tkzDrawY

   \tkzDefPoints{1/3/a1,2/1/a2,0/1/b1,1/1/b2,0/2/c1,1/2/c2}
   \tkzDrawLines[thick,add=1 and 3,color=RoyalBlue](b1,b2 c1,c2)
   \tkzDrawLine[thick,add=0.3 and 0.8,color=BrickRed](a1,a2)
  \end{tikzpicture}
  \caption{Two of the three lines parallel.}
  \vspace*{1em}
 \end{subfigure}
 \begin{subfigure}[t]{.45\textwidth}
  \centering
  \begin{tikzpicture}[scale=0.75]
   \tkzInit[xmin=-1,xmax=5,ymin=-1,ymax=3]
   \tkzDrawX
   \tkzDrawY

   \tkzDefPoints{1/3/a1,2/1/a2,0/0/b1,1/3/b2,0/2/c1,1/2/c2}
   \tkzDrawLines[thick,add=1 and 3,color=RoyalBlue](c1,c2)
   \tkzDrawLine[thick,add=0.3 and 0.8,color=BrickRed](a1,a2)
   \tkzDrawLine[thick,add=0.2 and 0.2,color=ForestGreen](b1,b2)
  \end{tikzpicture}
  \caption{No lines parallel without a common intersection.}
 \end{subfigure}
 \hfill
 \begin{subfigure}[t]{.45\textwidth}
  \centering
  \begin{tikzpicture}[scale=0.75]
   \tkzInit[xmin=-1,xmax=5,ymin=-1,ymax=3]
   \tkzDrawX
   \tkzDrawY

   \tkzDefPoints{1/3/a1,2/1/a2,0/0/b1,1.5/2/b2,0/2/c1,1/2/c2}
   \tkzDefPoint(1.5,2){i}
   \tkzDrawLines[thick,add=1 and 3,color=RoyalBlue](c1,c2)
   \tkzDrawLine[thick,add=0.3 and 0.8,color=BrickRed](a1,a2)
   \tkzDrawLine[thick,add=0.3 and 0.8,color=ForestGreen](b1,b2)
   \tkzDrawPoint[size=6](i)
  \end{tikzpicture}
  \caption{No lines parallel with a common intersection.}
 \end{subfigure}
 \caption{All the possible arrangements of three lines in the real plane.}
 \label{fig:arrangement-of-lines}
\end{figure}

It is clear that in cases (a), (b) and (c) in
\myref{figure}{fig:arrangement-of-lines}, the linear system has no solution. In
case (d), the system does have a solution but one of the lines is redundant --
it can in fact (as we've claimed before) be written as a linear combination of
the other two lines. By putting the linear system in question into
\hyperref[def:echelon-form]{echelon form}, we can easily deduce which of the
depicted cases emerged true.

Indeed, consider the system
\[
 \begin{array}{r c r c r}
  x & + & y & = & 2\\
  2x & + & 2y & = & 3\\
  -x & - & y & = & 1
 \end{array}.
\]
By subtracting $\mathtt{2I}$ from $\mathtt{II}$ and adding $\mathtt{I}$ to
$\mathtt{III}$, we put it into the following echelon form:
\[
 \begin{array}{r c r c r}
  x & + & y & = & 2\\
    & & 0 & = & -1\\
    & & 0 & = & 3
 \end{array}.
\]
Since two of the three equations have no solutions, case (a) arises -- the three
lines are all parallel to one another.

As yet another example, we present in all its glory the system
\[
 \begin{array}{r c r c r}
  2x & + & y & = & 5\\
  x & & & = & 2\\
  3x & - & y & = & 0
 \end{array}.
\]
By swapping $\mathtt{I}$ with $\mathtt{II}$, then subtracting $\mathtt{II} -
2\mathtt{I}$ and $\mathtt{III} - 3\mathtt{I}$, we get
\[
 \begin{array}{r c r c r}
  x & & & = & 2\\
  2x & + & y & = & 5\\
  3x & - & y & = & 0
 \end{array}.
\]

