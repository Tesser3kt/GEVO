\section{Visualizing Linear Systems}
\label{sec:visualizing-linear-systems}

In this, rather informal, section, we present a way to visualize linear systems
in two and three variables and their solutions. Why two and three, you ask? The
number of variables in a linear equation determines the \emph{dimension} of the
\emph{geometric object} described by this equation. We shall soon provide the
necessary definitions to make rigorous sense of the sentence previous.
Intuitively, each variable represents a new `direction' we're allowed to move
in. Therefore, linear equations in two variables live in two-dimensional spaces
and linear equations in three variables occupy three dimensions.

Nonetheless, the equations themselves (if non-trivial) never describe objects of
the maximal possible dimension but of the dimension lower by one. This is
because they establish a relationship between the variables -- a relationship
where one variable grows entirely dependent on the rest, essentially `locking' a
single direction of movement. Think of it like this: a linear equation is a sort
of order, telling you that for every step forward you must also make two steps
to the right and so rendering you unable to ever walk straight ahead again.

We proceed to show that the objects described by linear equations in two
variables are \emph{straight lines}. Said `objects described' are formally the
sets of points satisfying given equation. For instance, the object described by
the equation $3x + 2y = 4$ is the set
\[
 L \coloneqq \{(x,y) \in \R^2 \mid 3x + 2y = 4\}.
\]
Before we move on, we need establish an important fact. What is a \emph{straight
line} \textbf{exactly}? Wishing not to cheat and define straight line as the
object described by a linear equation, we employ a more geometric approach to
the definition. As we hope dear readers agree, a (one-dimensional) object is
\emph{straight} if moving along it requires `keeping the initial direction',
that is, always moving the same number of steps upward for a given number of
steps rightward, or vice versa. In other words, the \emph{ratio} between the
number of steps upward and rightward must remain constant. We encourage kind
readers to absorb that this particular property is what distinguishes
\emph{curved} objects from \emph{straight} ones.

\begin{figure}[ht]
 \centering
 \begin{tikzpicture}
  \tkzInit[xmin=-1,xmax=5,ymin=-1,ymax=2]
  \tkzDrawX
  \tkzDrawY

  \tkzDefPoints{0/0/A,2/1/B,1.5/0.75/C,3.5/1.75/D,3.5/0.75/E}
  \tkzDrawLine[color=BrickRed,thick,add=.5 and 1](A,B)
  \tkzDrawPoint[size=4,color=RoyalBlue](E)
  \tkzDrawSegments[dashed,thick,color=RoyalBlue](C,E D,E)
  \tkzDrawPoints[size=4,color=BrickRed](C,D)
  \tkzLabelPoint[above,color=BrickRed,xshift=-3mm,yshift=1mm](C){$(x_1,y_1)$}
  \tkzLabelPoint[above,color=BrickRed,xshift=-3mm,yshift=1mm](D){$(x_2,y_2)$}
  \tkzLabelSegment[right,color=RoyalBlue](D,E){$\Delta y = y_2 - y_1$}
  \tkzLabelSegment[below,color=RoyalBlue](C,E){$\Delta x = x_2 - x_1$}
 \end{tikzpicture}
 \caption{The `definition' of straightness. The ratio $\clb{\Delta y / \Delta
  x}$ must remain \textbf{constant}. It is habitually referred to as the
  \emph{slope} of the line.}
 \label{fig:straight-line}
\end{figure}

\myref{Figure}{fig:straight-line} inspires the following definition.

\begin{definition}{Straight line}{straight-line}
 An \textbf{infinite} subset $L \subseteq \R^2$ is called a \emph{straight line}
 if for all triples of points $(x_1,y_1)$, $(x_2,y_2)$, $(x_3,y_3) \in L$ it
 holds true that either
 \begin{equation}
  \label{eq:straight-line}
  \frac{y_2 - y_1}{x_2 - x_1} = \frac{y_3 - y_2}{x_3 - x_2},
 \end{equation}
 or $x_1 = x_2 = x_3$ (a vertical line).
\end{definition}

We proceed to show that the all the points in the plane satisfying a linear
equation form a \hyperref[def:straight-line]{straight line}. This is exceedingly
easy. Suppose we have three solutions $(x_1,y_1),(x_2,y_2)$ and $(x_3,y_3)$
satisfying the equation $ax + by = c$, where $a,b,c \in \R$ and at least one of
$a$, $b$ is not zero. In other words, we have $ax_i + by_i = c$ for $i \in
\{1,2,3\}$.

We've had to exclude the case $a = b = 0$ because the set of solutions of the
linear equation $0 = c$ is never a straight line. If $c \neq 0$, it is empty,
and if $c = 0$, it equals $\R^2$.

Assume first that $b = 0$. Then, $x_i = c / a$ and so $x_1 = x_2 = x_3$. Hence,
in this case, the set of solutions is indeed a straight line.

In case $b \neq 0$, we may rearrange
\[
 y_i = \frac{c - ax_i}{b}.
\]
Plugging this into~\eqref{eq:straight-line} gives
\begin{equation}
 \label{eq:straight-line-sub}
 \frac{(c - ax_2) - (c - ax_1)}{b(x_2 - x_1)} = \frac{(c - ax_3) - (c -
 ax_1)}{b(x_3 - x_1)}.
\end{equation}
Simple calculation yields
\[
 \frac{(c - ax_2) - (c - ax_1)}{b(x_2 - x_1)} = \frac{a(x_1 - x_2)}{b(x_2 -
 x_1)} = - \frac{a}{b}
\]
and similarly for $(y_3 - y_1) / (x_3 - x_1)$. Hence, both sides
of~\eqref{eq:straight-line-sub} equal $-a / b$ and the proof is complete.

\subsection{Two-dimensional Linear Systems}
\label{ssec:two-dimensional-linear-systems}

We dedicate this section to the visualization of linear systems in two variables
and their solutions. As already established, a linear equation in two variables
represents a \hyperref[def:straight-line]{straight line}. A solution to a linear
system in two variables is a pair of real numbers (equivalently, a point in the
real plane) which lies on every straight line determined by the equations of the
system. Simply put, the solution of a linear system in two variables is the
\emph{intersection} of all objects described by its equations.

An `ideal' linear system in two variables contains two linear equations
describing distinct lines. One such system is
\[
 \begin{array}{r c r c r}
  2x & - & y & = & 1\\
  x & + & y & = & 2
 \end{array}
\]
with solution $(1,1)$ and whose visual depiction is provided in
\myref{figure}{fig:well-determined-system}.

\begin{figure}[ht]
 \centering
 \begin{tikzpicture}
  \tkzInit[xmin=-1,xmax=5,ymin=-1,ymax=2]
  \tkzDrawX
  \tkzDrawY
  \tkzDefPoints{0/0/o,1/1/i,0.5/0/a,2/0/b}
  \tkzDrawLine[color=RoyalBlue,thick,add=0.8 and 1.3](a,i)
  \tkzDrawLine[color=ForestGreen,thick,add=0.8 and 1.3](b,i)

  \tkzDefPoints{1/0/x,0/1/y}
  \tkzDrawPoint[size=6,color=BrickRed](i)
  \tkzLabelPoint[below](x){$1$}
  \tkzLabelPoint[left](y){$1$}
  \tkzLabelPoint[right=2mm,color=BrickRed](i){$(1,1)$}
  \tkzDrawSegments[dashed,thick,color=BrickRed](i,x i,y)
  \tkzDrawPoints[size=4](x,y)
 \end{tikzpicture}

 \caption{Well-determined linear system in two variables with solution
 $\clr{(1,1)}$.}
 \label{fig:well-determined-system}
\end{figure}

An easily proven fact (which we shall eventually prove in greater generality)
that follows immediately from the geometric view reads that a linear system in
two variables with two \emph{distinct} linear equations always has a solution --
the intersection point of the corresponding lines.

A linear system in two variables can only be underdetermined should it feature
just one non-trivial linear equation (or, equivalently, many identical linear
equations). 

