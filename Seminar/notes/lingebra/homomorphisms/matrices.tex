\section{Homomorphisms As Matrices}
\label{sec:homomorphisms-as-matrices}

Homomorphisms of vector spaces are even more special (as maps between sets) than
the preceding text might have grown to reflect. They are one of those very few
maps that, even though infinite in nature (in the sense that they have
well-defined image of every vector in a usually infinite vector space), can be
represented by finite sets of numbers. Let us derive this `representation' using
an example.

Assume we're given a homomorphism $f:\R^3 \to R^3$ and wish to determine its
image on any vector $\begin{psmallmatrix} x \\ y \\ z \end{psmallmatrix} \in
R^3$. The fact that $f$ is a homomorphism enables the following calculation.
\begin{align*}
 \left( 
 \begin{pmatrix}
  x\\
  y\\
  z
 \end{pmatrix}
\right) &= f \left( 
 \begin{pmatrix}
  x\\
  0\\
  0
 \end{pmatrix} + 
 \begin{pmatrix}
  0\\
  y\\
  0
 \end{pmatrix}
 +
 \begin{pmatrix}
  0\\
  0\\
  z
 \end{pmatrix}
 \right) = f \left( x \cdot 
 \begin{pmatrix}
  1\\
  0\\
  0
 \end{pmatrix} + y \cdot
 \begin{pmatrix}
  0\\
  1\\
  0
 \end{pmatrix} + z \cdot 
 \begin{pmatrix}
  0\\
  0\\
  1
 \end{pmatrix}
 \right)\\
 &= x \cdot f \left( 
 \begin{pmatrix}
  1\\
  0\\
  0
 \end{pmatrix}
 \right) + y \cdot f \left( 
 \begin{pmatrix}
  0\\
  1\\
  0
 \end{pmatrix}
 \right) + z \cdot f \left( 
 \begin{pmatrix}
  0\\
  0\\
  1
 \end{pmatrix}
 \right).
\end{align*}
This calculation shows that it is in fact enough to know the image of the
\hyperref[def:standard-basis]{standard basis} vectors to know the image of
\emph{every} vector via the homomorphism $f$. Perhaps even more importantly, the
final expression reeks of \hyperref[def:dot-product]{dot product} of two
vectors. Imagine for a second that $f$ is actually a homomorphism $f:\R^3 \to
\R$ so that each of the vectors
\[
 f \left( 
 \begin{pmatrix}
  1\\
  0\\
  0
 \end{pmatrix}
 \right), f \left( 
 \begin{pmatrix}
 0\\
 1\\
 0
 \end{pmatrix}
 \right), f \left( 
 \begin{pmatrix}
  0\\
  0\\
  1
 \end{pmatrix}
 \right)
\]
is just a real number, say $1$, $2$ and $3$ in order for the sake of
concreteness.
