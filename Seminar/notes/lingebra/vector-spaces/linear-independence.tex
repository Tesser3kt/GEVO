\section{Linear Independence, Basis And Dimension}
\label{sec:linear-independence-basis-and-dimension}

Given a set $S \subseteq V$ of vectors, we first answer the question of
\emph{which vectors can be removed from $S$ while not altering its span}. That
is, given a vector $\mathbf{s} \in S$, how do we find out whether $\spn (S
\setminus \{\mathbf{s}\}) = \spn S$? Vaguely speaking, provided that $\spn S$ is
a set of linear combinations of vectors from $S$, should some vector in $S$
already \emph{be a linear combination} of the other vectors in $S$, it wouldn't
be needed. Turns out this statement is not so vague after all, as we proceed to
demonstrate.

\begin{lemma}{}{span-lemma}
 Let $V$ be a vector space, $S \subseteq V$ and $\mathbf{v} \in V$. Then, $\spn
 (S \cup \{\mathbf{v}\}) = \spn S$ if and only if $\mathbf{v} \in \spn S$.
\end{lemma}
\begin{lemproof}
 We must prove two implications.

 As for the implication $( \Rightarrow )$, it is simpler to prove it in
 contrapositive form. If $\mathbf{v} \notin \spn S$, then clearly $\spn S \neq
 \spn (S \cup \{\mathbf{v}\})$ simply because the latter contains the vector
 $\mathbf{v}$ while the former does not.

 In proving $( \Leftarrow )$, assume that $\mathbf{v} \in \spn S$. Clearly,
 $\spn S \subseteq \spn (S \cup \{\mathbf{v}\})$ as the latter set contains
 every linear combination of the vectors in $S$. We must show that also $\spn(S
 \cup \{\mathbf{v})\} \subseteq \spn S$. To this end, choose a vector
 $\mathbf{w} \in \spn(S \cup \{\mathbf{v}\})$. This vector $\mathbf{w}$ is a
 linear combination of vectors from $S \cup \{\mathbf{v}\}$, i.e.
 \[
  \mathbf{w} = \sum_{\mathbf{s} \in S \cup \{\mathbf{v}\}} r_{\mathbf{s}} \cdot
  \mathbf{s}
 \]
 for some $r_{\mathbf{s}} \in \R$. We can break this linear combination into two
 parts like so:
 \begin{equation}
  \label{eq:span-lemma}
  \mathbf{w} = \sum_{\mathbf{s} \in S \cup \{\mathbf{v}\}} r_{\mathbf{s}} \cdot
  \mathbf{s} = \sum_{\mathbf{s} \in S} r_{\mathbf{s}} \cdot \mathbf{s} +
  r_{\mathbf{v}} \cdot \mathbf{v}.
 \end{equation}
 Now, $\mathbf{v} \in \spn S$ by assumption so there also exist numbers
 $t_{\mathbf{s}} \in \R$ such that
 \[
  \mathbf{v} = \sum_{\mathbf{s} \in S} t_{\mathbf{s}} \cdot \mathbf{s}.
 \]
 Substituting this into the equation~\eqref{eq:span-lemma} gives
 \[
  \mathbf{w} = \sum_{\mathbf{s} \in S} r_{\mathbf{s}} \cdot \mathbf{s} +
  r_{\mathbf{v}} \cdot \left( \sum_{\mathbf{s} \in S} t_{\mathbf{s}} \cdot
  \mathbf{s} \right) = \sum_{\mathbf{s} \in S} r_{\mathbf{s}} \cdot \mathbf{s} +
  \sum_{\mathbf{s} \in S} (r_{\mathbf{v}}t_{\mathbf{s}}) \cdot \mathbf{s} =
  \sum_{\mathbf{s} \in S} (r_{\mathbf{s}} + r_{\mathbf{v}}t_{\mathbf{s}}) \cdot
  \mathbf{s}.
 \]
 The last sum is a linear combination of vectors from $S$ and thus $\mathbf{w}
 \in \spn S$, as desired.
\end{lemproof}

As a corollary, we get a formalisation of the idea from the introductory
paragraph.

\begin{corollary}{}{span-corollary}
 Given $\mathbf{s} \in S$, it holds that $\spn S = \spn (S \setminus
 \{\mathbf{s}\})$ if and only if $\mathbf{s} \in \spn (S \setminus
 \{\mathbf{s}\})$.
\end{corollary}
\begin{corproof}
 Follows immediately from \myref{lemma}{lem:span-lemma}. Simply substitute
 $\mathbf{v} \coloneqq \mathbf{s}$ and $S \coloneqq S \setminus \{\mathbf{s}\}$.
\end{corproof}

The \hyperref[cor:span-corollary]{just uttered corollary} has algorithmic vibes.
Can't we just keep removing vectors from $S$ which are linearly dependent on
other vectors until there are no longer any? Indeed, we can. First however, we
should devise a computationally sound way to determine which vectors we may
omit. As we now stand, the best we can do is guess at random. Pick a vector
$\mathbf{s} \in S$ and check if it lies in $\spn (S \setminus \{\mathbf{s}\})$
(as we learnt to do in the \hyperref[sec:subspaces-and-spans]{the previous
section}). If it doesn't, tough luck, try again. We might potentially have to go
through \emph{every} vector in $S$ before we find one that can be left out, if
there even were one to begin with. This is about as algorithmic as cooking a
soup by mixing random ingredients until we stumble upon a combination which is
reasonably non-lethal.

Fortunately, there is an algorithmic approach to the problem and we are
unveiling it promptly. Before that however, we should label sets with no
`unnecessary' vectors somehow.

\begin{definition}{Linear independence}{linear-independence}
 Let $V$ be a vector space and $S \subseteq V$. If no vector $\mathbf{s} \in S$
 can be written as a linear combination of vectors from $S \setminus
 \{\mathbf{s}\}$, we call $S$ \emph{linearly independent}. If such is not the
 case, it is called \emph{linearly dependent}.
\end{definition}

There lies just a simple observation between us and a feasible algorithm for
determining linear independence of a given set of vectors. Suppose $S =
\{\mathbf{s}_1,\ldots,\mathbf{s}_n\}$ and may the vector $\mathbf{s}_i$ be a
linear combination of the other vectors, that is to say, there are numbers
$r_1,\ldots,r_{i-1},r_{i+1},\ldots,r_n \in \R$ satisfying the equation
\[
 \mathbf{s}_i = r_1 \cdot \mathbf{s}_1 + r_2 \cdot \mathbf{s}_2 + \ldots +
 r_{i-1} \cdot \mathbf{s}_{i-1} + r_{i+1} \cdot \mathbf{s}_{i-1} + \ldots + r_n
 \cdot \mathbf{s}_n.
\]
We can naturally put $\mathbf{s}_i$ to the right hand side and set $r_i
\coloneqq -1$ to arrive at the equality
\[
 \mathbf{0} = r_1 \cdot \mathbf{s}_1 + \ldots + r_{i-1} \cdot \mathbf{s}_{i-1} +
 r_i \cdot \mathbf{s}_i + r_{i+1} \cdot \mathbf{s}_{i+1} + \ldots + r_n
 \mathbf{s}_n.
\]
To express this equality in words: we have found a linear combination (with
non-zero coefficients) of vectors from $S$ that gives the zero vector. Could
this happen were $S$ linearly independent? Of course it couldn't! If it did,
then we could just rearrange the last equality to the first one and get
$\mathbf{s}_i$ as a linear combination of the other vectors from $S$, proving
thus that $S$, in fact, had \emph{not} been linearly independent. Let us dock
this train of thought in the following, computationally indispensable,
proposition.

\begin{proposition}{}{linear-independence-zero-vector}
 Let $V$ be a vector space and $S \subseteq V$. Then $S$ is linearly independent
 if and only if the equality
 \[
  \sum_{\mathbf{s} \in S} r_{\mathbf{s}} \cdot s = \mathbf{0}
 \]
 enforces $r_{\mathbf{s}} = 0$ for every $\mathbf{s} \in S$. In other words, the
 only linear combination that gives the zero vector has all coefficients equal
 to $0$.
\end{proposition}
\begin{propproof}
 The paragraph preceding this proposition already illustrates the idea of the
 proof.

 To prove the implication $( \Leftarrow )$, suppose that $S$ is linearly
 dependent. That is, there exists $\mathbf{v} \in S$ such that $\mathbf{v}
 \neq \mathbf{0}$ and
 \[
  \mathbf{v} = \sum_{\mathbf{s} \in S \setminus \{\mathbf{v}\}} r_{\mathbf{s}}
  \cdot \mathbf{s}.
 \]
 If we put $\mathbf{v}$ to the right hand side and set $r_{\mathbf{v}} \coloneqq
 -1$, we get
 \[
  \mathbf{0} = \sum_{\mathbf{s} \in S \setminus \{\mathbf{v}\}} r_{\mathbf{s}}
  \cdot \mathbf{s} + (-1) \cdot \mathbf{v} = \sum_{\mathbf{s} \in S \setminus
  \{\mathbf{v}\}} r_{\mathbf{s}} \cdot \mathbf{s} + r_{\mathbf{v}} \cdot
  \mathbf{v} = \sum_{\mathbf{s} \in S} r_{\mathbf{s}} \cdot \mathbf{s}.
 \]
 We've thus wrought a linear combination of vectors from $S$ which has non-zero
 coefficients but equals the zero vector.

 As for $( \Rightarrow )$, assume that there exists a linear combination
 \[
  \sum_{\mathbf{s} \in S} r_{\mathbf{s}} \cdot \mathbf{s} = \mathbf{0}
 \]
 with at least one $r_{\mathbf{v}} \neq 0$. This means that we can rearrange
 \begin{align*}
  \sum_{\mathbf{s} \in S} r_{\mathbf{s}} \cdot \mathbf{s} &= \mathbf{0}\\
  \sum_{\mathbf{s} \in S \setminus \{\mathbf{v}\}} r_{\mathbf{s}} \cdot
  \mathbf{s} + r_{\mathbf{v}} \cdot \mathbf{v} &= \mathbf{0}\\
  \sum_{\mathbf{s} \in S \setminus \{\mathbf{v}\}} r_{\mathbf{s}} \cdot
  \mathbf{s} &= -r_{\mathbf{v}} \cdot \mathbf{v}\\
  -\frac{1}{r_{\mathbf{v}}} \cdot  \left( \sum_{\mathbf{s} \in S} r_{\mathbf{s}}
 \cdot \mathbf{s}\right) = \sum_{\mathbf{s} \in S \setminus \{\mathbf{v}\}}
  -\frac{r_{\mathbf{s}}}{r_{\mathbf{v}}} \cdot \mathbf{s} &= \mathbf{v}
 \end{align*}
 and thus $\mathbf{v} \in \spn(S \setminus \{\mathbf{v}\})$ which shows that $S$
 is linearly dependent.
\end{propproof}

\begin{corollary}{Computing linear independence}{computing-linear-independence}
 Let $V \leq \R^{n}$ be a subspace of $\R^{n}$ for some $n \in \N$ and $S =
 \{\mathbf{s}_1,\ldots,\mathbf{s}_k\} \subseteq V$. Then, $S$ is linearly
 independent if and only if the linear system
 \[
  \left(
   \begin{matrix*}[c]
    \mathbf{s}_1 & \mathbf{s}_2 & \cdots & \mathbf{s}_k
   \end{matrix*}
   \hspace{1mm}
  \right|
  \left.
   \begin{matrix*}[c]
    \mathbf{0}
   \end{matrix*}
  \right)
 \]
 has the unique solution $\mathbf{0}$.
\end{corollary}
\begin{corproof}
 The proof just amounts to rewriting the equality
 \[
  \sum_{\mathbf{s} \in S} r_{\mathbf{s}} \cdot \mathbf{s} = \sum_{i=1}^{k}
  r_{\mathbf{s}_i} \cdot \mathbf{s}_i = \mathbf{0}
 \]
 into a linear system and applying
 \myref{proposition}{prop:linear-independence-zero-vector}.
\end{corproof}

\begin{example}{}{}
 The set
 \[
  S \coloneqq \left\{ 
   \begin{pmatrix}
    1\\
    2
   \end{pmatrix},
   \begin{pmatrix}
    1\\
    -1
   \end{pmatrix},
   \begin{pmatrix}
    0\\
    3
   \end{pmatrix}
  \right\}
 \]
 is linearly dependent in $\R^2$. Indeed, the system
 \[
  \left(
   \begin{matrix*}[r]
    1 & 1 & 0\\
    2 & -1 & 3
   \end{matrix*}
   \hspace{1mm}
  \right|
  \left.
   \begin{matrix*}[r]
    0\\
    0
   \end{matrix*}
  \right)
 \]
 is solved by $(z, -z, -z)$ for any $z \in \R$.
 \myref{Corollary}{cor:computing-linear-independence} states that $S$ is
 linearly dependent.
\end{example}

\begin{example}{}{}
 The set $S \coloneqq \{1-x,1+x\}$ is linearly independent in the vector space
 of quadratic polynomials. To see this, consider a linear combination
 \begin{align*}
  r_1 \cdot (1-x) + r_2 \cdot (1+x) &= 0 + 0x + 0x^2\\
  (r_1 + r_2) + (-r_1 + r_2)x + 0x^2 &= 0 + 0x + 0x^2.
 \end{align*}
 Comparing coefficients gives
 \[
  \begin{array}{r c r c l}
   r_1 & + & r_2 & = & 0\\
   -r_1 & + & r_2 & = & 0\\
        & & 0 & = & 0.
  \end{array}
 \]
 This system has the unique solution $(0,0)$, hence the only way to linearly
 combine the polynomials $1-x$ and $1+x$ into the zero polynomial requires
 multiplying them both by $0$.
 \myref{Proposition}{prop:linear-independence-zero-vector} takes the reins.
\end{example}

Now that we have an algorithmic way of determining whether a given set is
linearly independent or not, we should tackle the problem of which vectors can
be removed from the set without shrinking its span. Before we do that, let us
first ascertain that indeed every (at least \textbf{finite}) linearly dependent
set can be made linearly independent by successive removal of redundant vectors.

\begin{lemma}{Linearly independent subset}{linearly-independent-subset}
 Given a vector space $V$ and a \textbf{finite subset} $S \subseteq V$, there
 exists a set $T \subseteq S$ that is linearly independent and $\spn S = \spn
 T$.
\end{lemma}
\begin{lemproof}
 Label $S = \{\mathbf{s}_1,\ldots,\mathbf{s}_k\}$. If $S$ is linearly
 independent, we're done. Otherwise, set $S_0 \coloneqq S$ and find $i \in
 \{1,\ldots,k\}$ such that
 \[
  \mathbf{s}_i = \sum_{j \neq i} r_j \cdot \mathbf{s}_j
 \]
 for some $r_j \in \R$. Set $S_1 \coloneqq S_0 \setminus \{\mathbf{s}_i\}$. By
 \myref{corollary}{cor:span-corollary}, $\spn S_1 = \spn S_0$.

 Repeat this process until $S_m$ is linearly independent for some $m \in \N$.
 Such $m$ necessarily exists because $S$ has a finite number of elements and a
 one-vector set is always linearly independent. Again, by
 \myref{corollary}{cor:span-corollary} (applied $m$ times), we have $\spn S_m =
 \spn S$ and thus we have found a linearly independent subset of $S$ with the
 same span as $S$.
\end{lemproof}

Recall from \myref{section}{sec:describing-solution-sets-of-linear-systems} that
some variables of linear systems are pivots and some are free. Pivots have their
value expressed as a linear combination of free variables. If we put vectors
from a given finite set $S \subseteq V$ into columns of a matrix (as in
\myref{corollary}{cor:span-corollary}), we claim that columns hosting free
variables mark vectors that can be removed without shrinking the span of $S$.
Why is it so? The answer is actually quite easy. We show it on an example.

Consider the set
\[
 S = \left\{ 
  \begin{pmatrix}
   1\\
   3\\
   2
  \end{pmatrix},
  \begin{pmatrix}
   1\\
   -1\\
   0
  \end{pmatrix},
  \begin{pmatrix}
   2\\
   0\\
   1
  \end{pmatrix},
  \begin{pmatrix}
   -1\\
   1\\
   1
  \end{pmatrix},
  \begin{pmatrix}
   1\\
   1\\
   3
  \end{pmatrix}
 \right\} \subseteq \R^3.
\]
Upon organizing the vectors of $S$ into the matrix
\[
 \left( 
  \begin{matrix*}[r]
   1 & 1 & 2 & -1 & 1\\
   3 & -1 & 0 & 1 & 1\\
   2 & 0 & 1 & 1 & 3
  \end{matrix*}
 \right)
\]
and performing Gauss-Jordan elimination, we get
\[
 \left( 
  \begin{matrix*}[r]
   1 & 1 & 2 & -1 & 1\\
   0 & -4 & -6 & 4 & -2\\
   0 & 0 & 0 & 1 & 2
  \end{matrix*}
 \right).
\]
It follows that the variables $x_3$ and $x_5$ are free. Why does it mean that
the third and fifth vector of $S$ are redundant? Well, the solution of the
homogeneous linear system with this matrix is
\begin{equation}
 \label{eq:li-if-0}
 \left\{ x_3 \cdot 
  \begin{pmatrix}
   1\\
   0\\
   -5\\
   -6\\
   3
  \end{pmatrix}
  + x_5 \cdot
  \begin{pmatrix}
   0\\
   1\\
   1\\
   2\\
   -1
  \end{pmatrix}
  \mid x_3,x_5 \in \R
 \right\}.
\end{equation}
By \myref{corollary}{cor:span-corollary}, the set $S$ is linearly independent if
and only if the set above contains only the vector $\mathbf{0}$. However, that
happens if and only if we force $x_3 = x_5 = 0$. Next, every linear combination
of vectors from $S$ is of the form
\[
 x_1 \cdot 
 \begin{pmatrix}
  1\\
  3\\
  2
 \end{pmatrix}
 + x_2 \cdot 
 \begin{pmatrix}
  1\\
  -1\\
  0
 \end{pmatrix}
 + x_3 \cdot 
 \begin{pmatrix}
  2\\
  0\\
  1
 \end{pmatrix}
 + x_4 \cdot 
 \begin{pmatrix}
  -1\\
  1\\
  1
 \end{pmatrix}
 + x_5 \cdot 
 \begin{pmatrix}
  1\\
  1\\
  3
 \end{pmatrix}.
\]
If, to ensure linear independence, we must require that $x_3$ and $x_5$ both be
always equal to $0$, it is completely pointless that we include the vectors
$\begin{psmallmatrix} 2 \\ 0 \\ 1 \end{psmallmatrix}$ and $\begin{psmallmatrix}
1 \\ 1 \\ 3 \end{psmallmatrix}$ in the linear combination in the first place.

Furthermore, observe that the vectors in the set~\eqref{eq:li-if-0} also hint at
how we can express the third and fifth vectors as linear combinations of the
other three. Indeed, the vector
\[
 \begin{pmatrix}
  1\\
  0\\
  -5\\
  -6\\
  3
 \end{pmatrix}
\]
in fact contains the coefficients of the linear combination of vectors in $S$
that gives the zero vector (as it is the solution of the corresponding
homogeneous system). This means that
\[
 1 \cdot
 \begin{pmatrix}
  1\\
  3\\
  2
 \end{pmatrix}
 + 0 \cdot 
 \begin{pmatrix}
  1\\
  -1\\
  0
 \end{pmatrix}
 + (-5) \cdot 
 \begin{pmatrix}
  2\\
  0\\
  1
 \end{pmatrix}
 + (-6) \cdot 
 \begin{pmatrix}
  -1\\
  1\\
  1
 \end{pmatrix}
 + 3 \cdot 
 \begin{pmatrix}
  1\\
  1\\
  3
 \end{pmatrix}
 =
 \begin{pmatrix}
  0\\
  0\\
  0
 \end{pmatrix}
\]
Rearranging (and dividing by $-3$) gives
\[
 -\frac{1}{3} \cdot 
 \begin{pmatrix}
  1\\
  3\\
  2
 \end{pmatrix}
 + \frac{5}{3} \cdot 
 \begin{pmatrix}
  2\\
  0\\
  1
 \end{pmatrix}
 + 2 \cdot 
 \begin{pmatrix}
  -1\\
  1\\
  1
 \end{pmatrix}
 = 
 \begin{pmatrix}
  1\\
  1\\
  3
 \end{pmatrix}
\]
and thus we have expressed the vector $\begin{psmallmatrix} 1\\1\\3
 \end{psmallmatrix}$ as a linear combination of the other four vectors. We could
 do the same for the vector $\begin{psmallmatrix} 2\\0\\1 \end{psmallmatrix}$
 which we also know to be redundant. Note, however, that we would have to in
 addition substitute the vector $\begin{psmallmatrix} 1 \\ 1 \\ 3
 \end{psmallmatrix}$ in the resulting linear combination as it should have been
 already removed from $S$.

To breathe some clarity into the concluded discussion, we shall show the
described procedure in a more algorithmic way.

\begin{problem}{}{reduce-to-li}
 Prove that the set
 \[
  S \coloneqq 
  \left\{ 
   \begin{pmatrix}
    1\\
    3
   \end{pmatrix},
   \begin{pmatrix}
    -2\\
    1
   \end{pmatrix},
   \begin{pmatrix}
    0\\
    -3
   \end{pmatrix},
   \begin{pmatrix}
    3\\
    0
   \end{pmatrix}
  \right\} \subseteq \R^2
 \]
 is linearly dependent and reduce it to a linearly independent set $S' \subseteq
 S$ with $\spn S' = \spn S$. In addition, express the removed vectors as linear
 combinations of the remaining ones.
\end{problem}
\begin{probsol}
 We compute the solution of the homogeneous linear system
 \[
  \left(
   \begin{matrix*}[r]
    1 & -2 & 0 & 3\\
    3 & 1 & -3 & 0
   \end{matrix*}
   \hspace{1mm}
  \right|
  \left.
   \begin{matrix*}[r]
    0\\
    0
   \end{matrix*}
  \right).
 \]
 After Gauss-Jordan elimination, we're left with
 \[
  \left(
   \begin{matrix*}[r]
    1 & -2 & 0 & 3\\
    0 & 7 & -3 & -9
   \end{matrix*}
   \hspace{1mm}
  \right|
  \left.
   \begin{matrix*}[r]
    0\\
    0
   \end{matrix*}
  \right).
 \]
 This means that columns $1$ and $2$ host pivots and columns $3$ and $4$ the
 free variables. We shall thus remove the third and the fourth vector from $S$.
 To finish the calculation, back-substitute and arrive at the set
 \[
  \left\{ x_3 \cdot 
   \begin{pmatrix}
    3\\
    0\\
    3\\
    -1
   \end{pmatrix}
   + x_4 \cdot 
   \begin{pmatrix}
    0\\
    3\\
    1\\
    2
   \end{pmatrix} \mid x_3,x_4 \in \R
  \right\}.
 \]
 We get rid of the fourth vector first. From the shape of the just computed
 solution, we infer that
 \[
  0 \cdot 
  \begin{pmatrix}
   1\\
   3
  \end{pmatrix}
  + 3 \cdot 
  \begin{pmatrix}
   -2\\
   1
  \end{pmatrix}
  + 1 \cdot 
  \begin{pmatrix}
   0\\
   -3
  \end{pmatrix}
  + 2 \cdot 
  \begin{pmatrix}
   3\\
   0
  \end{pmatrix} =
  \begin{pmatrix}
   0\\
   0
  \end{pmatrix}
 \]
 and so
 \begin{equation}
  \label{eq:fourth-as-lc}
  \begin{pmatrix}
   3\\
   0
  \end{pmatrix}
  = -\frac{3}{2} \cdot 
  \begin{pmatrix}
   -2\\
   1
  \end{pmatrix}
  -\frac{1}{2} \cdot 
  \begin{pmatrix}
   0\\
   -3
  \end{pmatrix},
 \end{equation}
 which by \myref{corollary}{cor:span-corollary} proves that
 \[
  \spn \left( 
   \begin{pmatrix}
    1\\
    3
   \end{pmatrix},
   \begin{pmatrix}
    -2\\
    1
   \end{pmatrix},
   \begin{pmatrix}
    0\\
    -3
   \end{pmatrix}
  \right) = \spn S.
 \]
 We now proceed to further remove $\begin{psmallmatrix} 0 \\-3
 \end{psmallmatrix}$ and express it as a linear combination of the remaining two
 vectors. The second vector in the computed solution gives the equality
 \[
  3 \cdot 
  \begin{pmatrix}
   1\\
   3
  \end{pmatrix}
  +0 \cdot 
  \begin{pmatrix}
   -2\\
   1
  \end{pmatrix}
  +3 \cdot 
  \begin{pmatrix}
   0\\
   -3
  \end{pmatrix}
  -1 \cdot 
  \begin{pmatrix}
   3\\
   0
  \end{pmatrix}
  = 
  \begin{pmatrix}
   0\\
   0
  \end{pmatrix},
 \]
 hence
 \[
  \begin{pmatrix}
   0\\
   -3
  \end{pmatrix}
  = -
  \begin{pmatrix}
   1\\
   3
  \end{pmatrix}
  +\frac{1}{3} \cdot 
  \begin{pmatrix}
   3\\
   0
  \end{pmatrix}.
 \]
 Substituting for $\begin{psmallmatrix} 3\\0 \end{psmallmatrix}$ the linear
 combination in~\eqref{eq:fourth-as-lc} and merging the coefficients yields
 \[
  \begin{pmatrix}
   0\\
   -3
  \end{pmatrix}
  = -\frac{6}{7} \cdot 
  \begin{pmatrix}
   1\\
   3
  \end{pmatrix}
  -\frac{3}{7} \cdot 
  \begin{pmatrix}
   -2\\
   1
  \end{pmatrix}.
 \]
 Finally, the desired linearly independent set $S'$ with $\spn S' = \spn S$ is
 \[
  S' = \left\{ 
   \begin{pmatrix}
    1\\
    3
   \end{pmatrix},
   \begin{pmatrix}
    -2\\
    1
   \end{pmatrix}
  \right\}.
 \]
\end{probsol}

\begin{remark}{}{}
 Observe that the solution of \myref{problem}{prob:reduce-to-li} basically
 mimics the proof of \myref{lemma}{lem:linearly-independent-subset} with an
 algorithmic approach to the selection of redundant vectors.
\end{remark}

\begin{warning}{}{}
 The indices of columns with pivots vs. free variables only point at the vectors
 of the original set which \textbf{are sure not to} shrink the span but they
 maken't the choice of vectors unique in any way. As a matter of fact, in many
 cases any of the present vectors can be removed without altering the span of
 the original set.

 To give one trivial example, consider the set
 \[
  S \coloneqq \left\{
   \begin{pmatrix}
    1\\0
   \end{pmatrix},
   \begin{pmatrix}
    -1\\
    0
   \end{pmatrix},
   \begin{pmatrix}
    2\\
    0
   \end{pmatrix}
  \right\}.
 \]
 Any one of the vectors in $S$ is linearly dependent on the other two (on either
 of them actually); all the vectors in $S$ are redundant.

 On the other hand, in the set
 \[
  S \coloneqq \left\{ 
   \begin{pmatrix}
    1\\
    0
   \end{pmatrix},
   \begin{pmatrix}
    0\\
    2
   \end{pmatrix},
   \begin{pmatrix}
    0\\
    -2
   \end{pmatrix}
  \right\},
 \]
 the first vector \textbf{is not} redundant. Only either of the second and third
 vectors may be mercilessly cut down without shrinking the span of $S$. It is in
 cases like these that the procedure outlined in
 \myref{problem}{prob:reduce-to-li} has its merit.
\end{warning}

We conclude the introduction to the concept of linear independence with one last
brief little inconsequential unimportant and just barely appealing discussion.
We've proven that removing a vector from a (finite) linearly dependent set can
make it independent. Adding a vector to a linearly dependent set on the other
hand cannot fix linear dependence. We shall summarise the link between subsets
and linear independence of the original set in
\myref{table}{table:linear-dependence-subsets}.
\begin{table}[ht]
 \centering
 \begin{tabular}{c | c  c}
  & $\hat{S} \subseteq S$ & $\tilde{S} \supseteq S$ \\
  \toprule
  $S$ linearly independent & $\hat{S}$ also linearly independent & $\tilde{S}$
  can be either\\
  $S$ linearly dependent & $\hat{S}$ can be either & $\tilde{S}$ also linearly
  dependent 
 \end{tabular}
 \caption{Linear dependence/independence of subsets.}
 \label{table:linear-dependence-subsets}
\end{table}

\subsection{Basis Of A Vector Space}
\label{ssec:basis-of-a-vector-space}

The study of linearly independent sets in
\myref{section}{sec:linear-independence-basis-and-dimension} carries on its back
yet another question: `Can \emph{every} vector space be expressed as the span of
a linearly independent set?' The answer this time is \emph{almost}. As we've
made customary, before we proceed to elucidate the given answer, we establish
some nomenclature to achieve a manageable level of brevity.

\begin{definition}{Basis}{basis}
 Let $V$ be a vector space. An ordered $n$-tuple
 $(\mathbf{v}_1,\ldots,\mathbf{v}_n)$, where $\mathbf{v}_1,\ldots,\mathbf{v}_n
 \in V$, which is both linearly independent and spans $V$ is called the
 \emph{basis} of $V$.
\end{definition}

\begin{warning}{}{basis-is-a-tuple}
 We've defined the basis of a vector space specifically to be an \textbf{ordered
 tuple} and \textbf{not just a set}. The reason for this will be given later in
 the chapter when we discuss representation of vectors with respect to distinct
 bases. Practically, this means that a basis, for example,
 \[
  \left(
   \begin{pmatrix}
    69\\
    0
   \end{pmatrix},
   \begin{pmatrix}
    0\\
    420
   \end{pmatrix}
  \right)
 \]
 is \textbf{different} from
 \[
  \left( 
   \begin{pmatrix}
    0\\
    420
   \end{pmatrix},
   \begin{pmatrix}
    69\\
    0
   \end{pmatrix}
  \right).
 \]
\end{warning}

\begin{example}{}{}
 The pair
 \[
  B \coloneqq \left( 
   \begin{pmatrix}
    2\\
    4
   \end{pmatrix},
   \begin{pmatrix}
    1\\
    1
   \end{pmatrix}
  \right)
 \]
 is a basis of $\R^2$. Verification of this statement entails making sure that
 $B$ is linearly independent and that the linear system
 \[
  \left(
   \begin{matrix*}[r]
    2 & 1 \\
    4 & 1
   \end{matrix*}
   \hspace{1mm}
  \right|
  \left.
   \begin{matrix*}[r]
    v_1\\
    v_2
   \end{matrix*}
  \right).
 \]
 has a solution for every $v_1,v_2 \in \R$.
\end{example}

Every $n$-dimensional space has a basis -- many of them in fact. One particular
basis is considered the `most natural', for chiefly geometric reasons. It is the
basis whose vectors have directions of the coordinate axes; it bears many names,
e.g. \emph{standard}, \emph{canonical} or \emph{natural}.

\begin{definition}{Standard basis}{standard-basis}
 The $n$-tuple
 \[
  \mathcal{E}_n \coloneqq \left( 
   \begin{pmatrix}
    1\\
    0\\
    0\\
    \vdots\\
    0
   \end{pmatrix},
   \begin{pmatrix}
    0\\
    1\\
    0\\
    \vdots\\
    0
   \end{pmatrix},
   \begin{pmatrix}
    0\\
    0\\
    1\\
    \vdots\\
    0
   \end{pmatrix},\ldots,
   \begin{pmatrix}
    0\\
    0\\
    0\\
    \vdots\\
    1
   \end{pmatrix}
  \right)
 \]
 is a basis of $\R^{n}$ and is called the \emph{standard} (or \emph{canonical}
 or \emph{natural}) basis. We denote the vectors of $\mathcal{E}_n$ (in order)
 $\mathbf{e}_1$ up to $\mathbf{e}_n$.
\end{definition}

\begin{example}{}{}
 The natural basis of the vector space of cubic polynomials is $(1,x,x^2,x^3)$.
 Other bases of the same space include $(x^3,3x^2,6x,6)$ or
 $(1,1+x,1+x+x^2,1+x+x^2+x^3)$.
\end{example}

\begin{example}{}{}
 The trivial space $\{\mathbf{0}\}$ has only one basis -- the empty set
 $\emptyset$.
\end{example}

\begin{example}{}{}
 The vector spaces of functions $f:\N \to \R$ and of functions $f:\R \to \R$
 \textbf{do not} have bases because there is no reasonable way to enumerate
 functions with outputs in the real numbers.
\end{example}

\begin{example}{}{}
 We have met bases before when studying sets of solutions of homogeneous
 systems; we only wouldn't call them such. The solution set of the linear system
 \[
  \begin{array}{r c r c r c r c l}
   x_1 & + & x_2 & & & - & x_4 & = & 0\\
       & & & & x_3 & + & x_4 & = &0
  \end{array}
 \]
 is
 \[
  \left\{
   x_2 \cdot 
   \begin{pmatrix}
    -1\\
    1\\
    0\\
    0
   \end{pmatrix}
   + x_4 \cdot 
   \begin{pmatrix}
    1\\
    0\\
    -1\\
    1
   \end{pmatrix}
   \mid x_2,x_4 \in \R
  \right\}.
 \]
 Notice that the set is written as a span of two linearly independent vectors.
 In other words, its basis is the pair
 \[
  \left( 
   \begin{pmatrix}
    -1\\
    1\\
    0\\
    0
   \end{pmatrix},
   \begin{pmatrix}
    1\\
    0\\
    -1\\
    1
   \end{pmatrix}
  \right).
 \]
\end{example}

Before we return to the original question of \emph{existence} of a basis, we
merge our current knowledge into a very important theorem which has both
theoretical and computational consequences.

\begin{theorem}{Characterisation of a basis}{characterisation-of-a-basis}
 Given a vector space $V$, an $n$-tuple $B = (\mathbf{b}_1,\ldots,\mathbf{b}_n)$
 is a basis of $V$ if and only if every vector in $V$ can be written as a linear
 combination of vectors in $B$ in a \textbf{unique} way.
\end{theorem}
\begin{thmproof}
 By \hyperref[def:basis]{definition of a basis}, $\spn B = V$ so indeed every
 vector in $V$ must be expressible as a linear combination of vectors in $B$.

 We now prove that this expression need be unique. For contradiction, assume
 that there exists a vector $\mathbf{v} \in V$ such that
 \[
  \mathbf{v} = \sum_{i=1}^{n} r_i \cdot \mathbf{b}_i \quad \text{and also} \quad
  \mathbf{v} = \sum_{i=1}^{n} t_i \cdot \mathbf{b}_i
 \]
 with $r_j \neq t_j$ for at least one index $j \leq n$. We may rearrange
 \begin{align*}
  \sum_{i = 1}^{n} r_i \cdot \mathbf{b}_i &= \sum_{i = 1}^{n} t_i \cdot
  \mathbf{b}_i\\
  \sum_{i = 1}^{n} r_i \cdot \mathbf{b}_i - \sum_{i = 1}^{n} t_i \cdot
  \mathbf{b}_i &= \mathbf{0}\\
  \sum_{i = 1}^{n} (r_i - t_i) \cdot \mathbf{b}_i &= \mathbf{0}.
 \end{align*}
 Since $r_j \neq t_j$ and thus $r_j - t_j \neq 0$, the linear combination on the
 left hand side has non-zero coefficients. By
 \myref{proposition}{prop:linear-independence-zero-vector}, this means that $B$
 is linearly dependent. That's a contradiction with the assumption that it is a
 basis, hence such a vector $\mathbf{v}$ can't exist and the theorem is proven.
\end{thmproof}

Unfortunately, we lack the theoretical background to fully answer the question
of which vector spaces have bases and which don't. We can only define a class of
vector spaces that \textbf{always} do have bases. Nevertheless, we can't prove
that vector spaces outside of this class do not have bases -- perhaps because it
is not true \dots

What we can say is that vector spaces which can be written as spans of vectors
have a basis. This is actually a trivial consequence of
\myref{lemma}{lem:linearly-independent-subset}. Suppose a vector space $V$ is
spanned by a finite set of vectors $S \subseteq V$. We can keep removing vectors
from $S$ until we reach a set $S' \subseteq S$ which is linearly independent and
$\spn S' = \spn S$. Any ordering of the set $S'$ is now a basis of $V$. Indeed,
it spans $V$ and every vector in $V$ can be written as a linear combination of
vectors from $S'$ in a unique way. The former statement is clear (by
\myref{corollary}{cor:span-corollary}) and the latter follows from the proof of
\myref{lemma}{lem:characterization-of-a-basis}. Should a vector $\mathbf{v}
\in V$ have two different expressions in terms of vectors of $S'$, we could
subtract one from the other and get a non-trivial linear combination giving the
zero vector -- a contradiction with the linear independence of $S'$ by
\myref{proposition}{prop:linear-independence-zero-vector}.

There is a point relevant to bases we should address. Intuitively, a basis of a
space is the set of all possible \emph{unique} directions of movement in that
space. Wouldn't it be weird were we able to move in $n$ possible ways in
$\R^{n}$ when using the \hyperref[def:standard-basis]{standard basis} and, say,
$n + 2$ ways when using some different basis? We tend to think of the dimension
(or the total number of distinct directions of travel) of a space as something
\emph{fixed}, something inherent to the space itself, unrelated to any specific
choice of vectors representing the directions.

As is thankfully often the case in linear algebra, our geometric intuition is
correct. The formal way to express it is to say that all bases of a space should
have the same number of elements. This is indeed the case and the number of
elements in a basis is then called the \emph{dimension} of said vector space.

First, we classify the vector spaces whereof we know they have a basis.

\begin{definition}{Finitely generated vector space}{finitely-generated-vector-space}
 A vector space $V$ is called \emph{finitely generated} if it has a basis with
 finite number of vectors.
\end{definition}

\begin{remark}{}{}
 In the \hyperref[def:finitely-generated-vector-space]{definition above}, we
 specifically said `has \textbf{a} basis' because we have not yet proven that
 all bases of a vector space have the same number of elements. Once we do so,
 finitely generated vector spaces can be seen as vector spaces of finite
 dimension.
\end{remark}

\begin{example}{}{}
 Every $n$-dimensional real space is finitely generated (take its
 \hyperref[def:standard-basis]{standard basis} for example) while the space of
 all functions $f:\R \to \R$ is not.
\end{example}

We prove the statement of equal number of elements across all bases in a
somewhat roundabout way. This has the advantage of introducing a method of --
somewhat algorithmically -- transforming one basis of a space into another
vector by vector.

\begin{lemma}{Exchange lemma}{exchange-lemma}
 Assume $V$ is a finitely generated vector space with basis $B =
 (\mathbf{b}_1,\ldots,\mathbf{b}_n)$ and pick a vector $\mathbf{v} \in V$ given
 by the linear combination
 \[
  \mathbf{v} = r_1 \cdot \mathbf{b}_1 + r_2 \cdot \mathbf{b}_2 + \ldots + r_n
  \cdot \mathbf{b}_n
 \]
 with $r_i \neq 0$ for some $i \leq n$. Then, $\hat{B} \coloneqq
 \{\mathbf{b}_1,\mathbf{b}_2,\ldots,\mathbf{b}_{i-1},\mathbf{v},\mathbf{b}_{i+1},
 \ldots,\mathbf{b}_n\}$ is also a basis of $V$.
\end{lemma}
\begin{lemproof}
 We need to show that
 \begin{enumerate}[label=(\alph*)]
  \item $\hat{B}$ is linearly independent.
  \item $\hat{B}$ spans $V$.
 \end{enumerate}
 As for (a), assume we have a linear combination
 \begin{equation}
  \label{eq:exchange-lemma}
  t_1 \cdot \mathbf{b}_1 + \ldots + t_{i-1} \cdot \mathbf{b}_{i-1} + t_i \cdot
  \mathbf{v} + t_{i+1} \cdot \mathbf{b}_{i+1} + \ldots + t_n \cdot \mathbf{b}_n
  = \mathbf{0}
 \end{equation}
 for some $t_1,\ldots,t_n \in \R$. Substituting for $\mathbf{v}$ gives
 \[
  t_1 \cdot \mathbf{b}_1 + \ldots + t_{i-1} \cdot \mathbf{b}_{i-1} + t_i \cdot
  (r_1 \mathbf{b}_1 + \ldots + r_n \mathbf{b}_n) + t_{i+1} \cdot
  \mathbf{b}_{i+1} + \ldots + t_n \cdot \mathbf{b}_n = \mathbf{0}.
 \]
 Rearranging then
 \begin{equation}
  \label{eq:exchange-lemma-2}
  \begin{split}
   (t_1 + t_i r_1) \cdot \mathbf{b_1} &+ \ldots + (t_{i-1} + t_ir_{i-1}) \cdot
   \mathbf{b}_{i-1} + \clr{t_i r_i \cdot \mathbf{b}_i}\\
   &+ (t_{i+1} + t_i r_{i+1}) \cdot \mathbf{b}_{i+1} + \ldots + (t_n + t_i r_n)
   \cdot \mathbf{b}_n = \mathbf{0}.
  \end{split}
 \end{equation}
 This is a linear combination of vectors from the linearly independent basis $B$
 and thus by \myref{proposition}{prop:linear-independence-zero-vector}, every
 coefficient of this combination is equal to $0$. In particular, this means that
 $t_i r_i = 0$ and, since we've assumed $r_i \neq 0$, necessarily $t_i = 0$.
 However, in the wake of this, the combination~\eqref{eq:exchange-lemma} becomes
 \[
  t_1 \cdot \mathbf{b}_1 + \ldots + t_{i-1} \cdot \mathbf{b}_{i-1} + t_{i+1}
  \cdot \mathbf{b}_{i+1} + \ldots + t_n \mathbf{b}_n = \mathbf{0},
 \]
 i.e. a linear combination of vectors from $B$. Using
 \myref{proposition}{prop:linear-independence-zero-vector} again gives $t_j = 0$
 for all $j \leq n$ since we already knew that $t_i = 0$. It follows that also
 $t_j + t_i r_j = 0$ for every $j \leq n$ and the linear combination
 in~\eqref{eq:exchange-lemma-2} has all coefficients equal to $0$. This proves
 that $\hat{B}$ is linearly independent.

 To prove (b), we check that $\spn \hat{B} \subseteq \spn B$ and $\spn B
 \subseteq \spn \hat{B}$. The inclusion $\spn \hat{B} \subseteq \spn B$ is
 obvious as $\mathbf{v}$ lies in $\spn B$ (and so do all the vectors
 $\mathbf{b}_i$ of course). For the reverse inclusion to hold, it is enough to
 represent the exchanged vector $\mathbf{b}_i$ as linear combination of vectors
 from $\hat{B}$ because $B$ and $\hat{B}$ share all the other vectors besides
 $\mathbf{b}_i$. In the linear combination
 \[
  \mathbf{v} = r_1 \cdot \mathbf{b_1} + \ldots + \clr{r_i \cdot \mathbf{b}_i} +
  \ldots + r_n \cdot \mathbf{b}_n,
 \]
 we assumed that $r_i \neq 0$. We can thus rearrange
 \begin{align*}
  \mathbf{v} &= r_1 \cdot \mathbf{b}_1 + \ldots + r_i \cdot \mathbf{b}_i +
  \ldots + r_n \cdot \mathbf{b}_n\\
  -r_i \cdot \mathbf{b}_i &= r_1 \cdot \mathbf{b}_1 + \ldots + r_{i-1} \cdot
  \mathbf{b}_{i-1} + \clr{(-1) \cdot \mathbf{v}} + r_{i+1} \cdot
  \mathbf{b}_{i+1} + \ldots + r_n \cdot \mathbf{b}_n\\
  \mathbf{b}_i &= -\frac{r_1}{r_i} \cdot \mathbf{b}_1 + \ldots +
  \left( -\frac{r_{i-1}}{r_i} \right) \cdot \mathbf{b}_{i-1} + \clr{\left(
  -\frac{1}{r_i} \right) \cdot \mathbf{v}} + \left( -\frac{r_{i+1}}{r_i}
 \right) \cdot \mathbf{b}_{i+1} + \ldots + \left( -\frac{r_n}{r_i} \right) \cdot
 \mathbf{b}_n
 \end{align*}
 which proves that $\mathbf{b}_i \in \spn \hat{B}$ and with it, the lemma.
\end{lemproof}

We intend to use the \hyperref[lem:exchange-lemma]{exchange lemma} to prove that
all bases of a finitely generated vector space have the same number of vectors
by inductively exchanging the vectors of one basis for the vectors of another.

\begin{theorem}{The dimension theorem}{the-dimension-theorem}
 All bases of a finitely generated vector space have the same number of elements.
\end{theorem}
\begin{thmproof}
 Fix a vector space $V$ and its basis $B \coloneqq
 (\mathbf{b}_1,\ldots,\mathbf{b}_n)$ with minimal number of elements. Given
 another basis $D = (\mathbf{d}_1,\ldots,\mathbf{d}_m)$, necessarily $n \leq m$
 because the number of elements of $B$ is assumed to be minimal. We shall prove
 that $m \leq n$.

 The idea of the proof is to exchange all vectors in $B$ for vectors in $D$
 until we get a basis of $V$ consisting of only $n$ vectors of $D$.

 We proceed by induction on the number of exchanged vectors. Set $B_0 \coloneqq
 B$. So far no vectors have been exchanged. Since $B$ spans $V$ and
 $\mathbf{d}_1 \in V$, there exists a linear combination
 \[
  \mathbf{d}_1 = d_{1,1} \cdot \mathbf{b}_1 + d_{1,2} \cdot \mathbf{b}_2 +
  \ldots + d_{1,n} \cdot \mathbf{b}_n
 \]
 with at least one $d_{1,i}$ non-zero (as the zero vector is always linearly
 dependent on others, thus $\mathbf{d}_1 \neq \mathbf{0}$). By the
 \hyperref[lem:exchange-lemma]{exchange lemma}, we may exchange $\mathbf{d}_1$
 for $\mathbf{b}_i$ and get the basis
 \[
  B_1 \coloneqq
  (\mathbf{b}_1,\ldots,\mathbf{b}_{i-1},\mathbf{d}_1,\mathbf{b}_{i+1}, \ldots,
  \mathbf{b}_n)
 \]
 of $V$.

 For the induction step, suppose the basis $B_k$ has been formed by exchanging
 the vectors $\mathbf{d}_1,\ldots,\mathbf{d}_k \in D$ for exactly $k$ vectors
 from $B$. Let us denote the set of indices of the remaining original vectors as
 $I \subseteq \{1,\ldots,n\}$. That is, $\mathbf{b}_i \in B_k$ if and only if $i
 \in I$. Pick $\mathbf{d}_{k+1} \in D$ and write
 \[
  \mathbf{d}_{k+1} = \sum_{i=1}^{k} d_{k+1,i} \cdot \mathbf{d}_i + \sum_{i \in
  I} d_{k+1,i} \cdot \mathbf{b}_i
 \]
 as a linear combination of vectors from $B_k$. The important observation to
 make is that at least one of the coefficients $d_{k+1,i}, i \in I$, must be
 non-zero. To see why, assume we have $d_{k+1,i} = 0$ for all $i \in I$. Then,
 the linear combination above assumes the form
 \[
  \mathbf{d}_{k+1} = \sum_{i=1}^{k} d_{k+1,i} \cdot \mathbf{d}_i.
 \]
 But, this means that $\mathbf{d}_{k+1}$ is a linear combination of other
 vectors from $D$. This contradicts the assumption that $D$ is linearly
 independent and so this situation cannot arise.

 Now that we know that there exists an index $i \in I$ such that $d_{k+1,i} \neq
 0$, we may (again by the \hyperref[lem:exchange-lemma]{exchange lemma})
 exchange the vector $\mathbf{b}_i$ for $\mathbf{d}_{k+1}$ and form the basis
 $B_{k+1}$.

 Upon having exchanged the last remaining vector $\mathbf{b}_i$ for
 $\mathbf{d}_n$, we have constructed the basis
 \[
  B_n = (\mathbf{d}_1,\ldots,\mathbf{d}_n)
 \]
 of the space $V$. Since $B_n$ is linearly independent and spans $V$, it follows
 that $\mathbf{d}_{n+1},\ldots,\mathbf{d}_m \in \spn B_n$ which is a
 contradiction because $B_n$ is a subset of $D$ and $D$ is assumed to be
 linearly independent. Thus, there must be no more vectors in $D$ after
 $\mathbf{d}_n$ which proves that $m \leq n$ and with that also that $m = n$, as
 desired.
\end{thmproof}

The \hyperref[thm:the-dimension-theorem]{dimension theorem} has a few immediate
consequences. For instance, we can finally define the dimension of any finitely
generated vector space.

\begin{definition}{Dimension}{dimension}
 Given a finitely generated vector space $V$, its \emph{dimension} is the number
 of elements of any of its bases. We label it $\dim V$.
\end{definition}

\begin{example}{}{}
 The $n$-dimensional real space has dimension $n$. The testifying basis is
 $\mathcal{E}_n$, for example.
\end{example}

\begin{example}{}{}
 The space of polynomials of degree at most $n$ has dimension $n + 1$. As we've
 partially observed, its standard basis is $(1,x,x^2,\ldots,x^{n})$ which has $n
 + 1$ elements.
\end{example}

\begin{corollary}{}{}
 In a finitely generated vector space $V$, no linearly independent set $S
 \subseteq V$ can have more elements than the dimension of $V$.
\end{corollary}
\begin{corproof}
 Follows from the proof of the \hyperref[thm:the-dimension-theorem]{dimension
 theorem}. Observe that in the proof we have never used the assumption that $D$
 spans $V$, only that it is linearly independent.
\end{corproof}

\begin{corollary}{}{expand-to-basis}
 Any linearly independent set $S \subseteq V$ in a finitely generated vector
 space $V$ can be expanded to a basis of $V$.
\end{corollary}
\begin{corproof}
 If $\spn S \neq V$, then there exists a vector $\mathbf{v} \in V$ such that
 $\mathbf{v} \notin \spn S$. By \myref{lemma}{lem:span-lemma}, $S \subsetneq S
 \cup \{\mathbf{v}\}$ and $S \cup \{\mathbf{v}\}$ is linearly independent
 because $\mathbf{v} \notin \spn S$. Hence, we simply keep adding vectors to $S$
 until $\spn S = V$ and $\# S = \dim V$.
\end{corproof}

\begin{corollary}{}{shrink-to-basis}
 Any set $S \subseteq V$ with $\spn S = V$ can be shrunk to a basis of the
 finitely generated vector space $V$.
\end{corollary}
\begin{corproof}
 If $S$ is empty, then it spans the space $\{\mathbf{0}\}$ and is already a
 basis of it. If $S = \{\mathbf{0}\}$, then it also spans just the space
 $\{\mathbf{0}\}$ and we can remove the vector $\mathbf{0}$ from it, keeping its
 span.

 Otherwise, $S$ contains a vector $\mathbf{s}_1 \neq \mathbf{0}$. We form a
 basis $B_1 \coloneqq (\mathbf{s}_1)$. If $\spn B_1 = \spn S$, we're done.
 Otherwise, there exists a vector $\mathbf{s}_2 \in S$ such that
 $\mathbf{s}_2 \notin \spn B_1$. Form $B_2 \coloneqq
 (\mathbf{s}_1,\mathbf{s}_2)$. This pair is linearly independent by the same
 argument as in the proof of \myref{corollary}{cor:expand-to-basis}. We repeat
 this process until $\spn B_n = \spn S$ which takes exactly $\dim V$ steps.
\end{corproof}

\begin{corollary}{}{}
 In a vector space $V$ with $\dim V = n$, an $n$-element set is linearly
 independent if and only if it spans $V$.
\end{corollary}
\begin{corproof}
 As for $( \Rightarrow )$, any linearly independent set $S$ can be expanded to a
 basis of $V$ by \myref{corollary}{cor:expand-to-basis}. Since a basis of $V$
 has $n$ elements and so does $S$, there is no expansion to be done and any
 ordering of $S$ is already a basis of $V$; in particular $\spn S = V$.

 The implication $( \Leftarrow )$ is also immediate. If $\spn S = V$, then by
 \myref{corollary}{cor:shrink-to-basis}, it can be shrunk to a basis of $V$,
 which has $n$ elements. Since $S$ also has $n$ elements, no shrinking takes
 place and any ordering of $S$ is again a basis of $V$ and is thus linearly
 independent.
\end{corproof}


\subsection{Representation With Respect To A Basis}
\label{ssec:representation-with-respect-to-a-basis}

\myref{Theorem}{thm:characterisation-of-a-basis} leads to a corollary of mainly
computational importance: \textbf{every} vector in a vector space $V$ with basis
$B$ corresponds to \textbf{exactly one} sequence of real coefficients of the
linear combination of vectors from $B$ that equals this vector.

To put this symbolically, denote $B =
(\mathbf{b}_1,\mathbf{b}_2,\ldots,\mathbf{b}_n)$ and consider a vector
$\mathbf{v} \in V$. By the mentioned
\myref{theorem}{thm:characterisation-of-a-basis}, there exists exactly one
$n$-tuple $(r_1,\ldots,r_n) \in \R^{n}$ such that
\[
 \mathbf{v} = r_1 \cdot \mathbf{b_1} + r_2 \cdot \mathbf{b}_2 + \ldots + r_n
 \cdot \mathbf{b}_n.
\]
However, in \myref{chapter}{chap:linear-systems}, we observed that elements of
$\R^{n}$ are really just $n$-dimensional vectors with entries in $\R$. These two
facts brought together beget an important idea we shall formalise in due time --
\emph{vector spaces of dimension $n$ are `equivalent' to $\R^{n}$}. The last
sentence should be read as such: in every vector space $V$, we can choose a
basis $B$ and write every vector in $V$ as a linear combination of vectors from
$B$. The coefficients of this linear combination (that are unique for every
vector) can be assembled into a vector in $\R^{n}$. This forges a two-way
relationship (a correspondence, if you will) between vectors in $V$ and vectors
in $\R^{n}$. We call this relationship a \emph{representation} of the vector
$v \in V$ for the reason that it gives a concrete form to an abstract vector.

\begin{definition}{Representation of a vector}{representation-of-a-vector}
 Let $V$ be a vector space with basis $B = (\mathbf{b}_1,\ldots,\mathbf{b}_n)$
 and $\mathbf{v} \in V$. We call the vector
 \[
  [\mathbf{v}]_B \coloneqq
  \begin{pmatrix}
   r_1\\
   r_2\\
   \vdots\\
   r_n
  \end{pmatrix}
 \in \R^{n}
 \]
 a \emph{representation of $\mathbf{v}$ with respect to $B$} if $\mathbf{v} =
 r_1 \cdot \mathbf{b}_1 + r_2 \cdot \mathbf{b}_2 + \ldots + r_n \cdot
 \mathbf{b}_n$.
\end{definition}

\begin{remark}{}{}
 The \hyperref[def:representation-of-a-vector]{preceding definition} underlines
 the necessity of defining a basis as a \textbf{sequence}, not just a set. A
 permutation of the elements of a basis changes the representation of many
 vectors with respect to it.
\end{remark}

The notion of \emph{representation} formalises the approach we've taken many
times ere of `writing' polynomials or matrices as vectors of coefficients.
Confront the following example. 

\begin{example}{}{}
 In the space of cubic polynomials, the representation of the polynomial $x +
 x^2$ with respect to the basis $B = (1, 2x, 2x^2, 2x^3)$ is given by
 \[
  [x + x^2]_B = 
  \begin{pmatrix}
   0\\
   1 / 2\\
   1 / 2\\
   0  
  \end{pmatrix}.
 \]
 With respect to a different basis $C = (1 + x, 1 - x, x + x^2, x + x^3)$, it
 instead looks like this:
 \[
  [x + x^2]_C = 
  \begin{pmatrix}
   0\\
   0\\
   1\\
   0
  \end{pmatrix}.
 \]
\end{example}

\begin{problem}{}{}
 Find the representation of the vector
 \[
  \mathbf{v} = 
  \begin{pmatrix}
   3\\
   2
  \end{pmatrix}
 \]
 with respect to
 \[
  B = \left( 
   \begin{pmatrix}
    1\\
    1
   \end{pmatrix},
   \begin{pmatrix}
    0\\
    2
   \end{pmatrix}
  \right).
 \]
\end{problem}
\begin{probsol}
 We need to find real scalars $r_1,r_2 \in \R$ such that
 \[
  r_1 \cdot
  \begin{pmatrix}
   1\\
   1 
  \end{pmatrix} + r_2 \cdot
  \begin{pmatrix}
   0\\
   2
  \end{pmatrix}
  =
  \begin{pmatrix}
   3\\
   2
  \end{pmatrix}.
 \]
 This is tantamount to solving the linear system
 \[
  \begin{array}{r c r c l}
   r_1 & & & = & 3\\
   r_1 & + & 2r_2 & = & 2
  \end{array}
 \]
 with obvious solution $r_1 = 3$ and $r_2 = -1 / 2$. With this, we've affirmed
 the equality
 \[
  \left[ 
  \begin{pmatrix}
   3\\
   2
  \end{pmatrix}
  \right]_B =
  \begin{pmatrix}
   3\\
   -1 / 2
  \end{pmatrix}.
 \]
\end{probsol}
\begin{example}{Representation with respect to canonical basis}{representation-with-respect-to-canonical-basis}
 Since every vector $\mathbf{v} \in \R^{n}$ can be trivially broken into a
 linear combination of \hyperref[def:standard-basis]{canonical basis} vectors,
 its representation with respect to this basis are exactly its coordinates.

 Expressed symbolically,
 \[
  [\mathbf{v}]_{\mathcal{E}_n} = 
  \left[ 
  \begin{pmatrix}
   v_1\\
   v_2\\
   \vdots\\
   v_n
  \end{pmatrix}
  \right]_{\mathcal{E}_n} = 
  \begin{pmatrix}
   v_1\\
   v_2\\
   \vdots\\
   v_n
  \end{pmatrix}
 \]
 for every $\mathbf{v} \in \R^{n}$ because
 \[
  \mathbf{v} = v_1 \cdot \mathbf{e}_1 + v_2 \cdot \mathbf{e}_2 + \ldots + v_n
  \cdot \mathbf{e}_n.
 \]
\end{example}

We intend not to dwell on the idea of representation any longer for now. It
shall emerge again when we discuss linear transformations known as \emph{changes
of basis}. We close with a result concerning a link between linear independence
and vector representation. In fact, linear independence of vectors is equivalent
to the linear independence of their representations with respect to any basis.

\begin{lemma}{}{}
 Let $V$ be a vector space of dimension $n \in \N$ with basis $B$,
 $\mathbf{v}_1,\ldots,\mathbf{v}_k \in V$ and $r_1,\ldots,r_k \in \R$. Then,
 \[
  r_1 \cdot \mathbf{v}_1 + r_2 \cdot \mathbf{v}_2 + \ldots + r_k \cdot
  \mathbf{v}_k = \mathbf{0}_V
 \]
 if and only if
 \[
  r_1 \cdot [\mathbf{v}_1]_B + r_2 \cdot [\mathbf{v}_2]_B + \ldots + r_k \cdot
  [\mathbf{v}]_k = \mathbf{0}_{\R^{n}}
 \]
 where $\mathbf{0}_V$ is the zero vector of the space $V$ and
 $\mathbf{0}_{\R^{n}}$ that of $\R^{n}$.
\end{lemma}
\begin{lemproof}
 Write $B = (\mathbf{b}_1,\mathbf{b}_2,\ldots,\mathbf{b}_n)$ and also denote
 \[
  [\mathbf{v}_1]_B = 
  \begin{pmatrix}
   a_{1,1}\\
   a_{2,1}\\
   \vdots\\
   a_{n,1}
  \end{pmatrix},
  [\mathbf{v}_2]_B =
  \begin{pmatrix}
   a_{1,2}\\
   a_{2,2}\\
   \vdots\\
   a_{n,2}
  \end{pmatrix},\ldots,[\mathbf{v}_k]_B = 
  \begin{pmatrix}
   a_{1,k}\\
   a_{2,k}\\
   \vdots\\
   a_{n,k}
  \end{pmatrix}.
 \]
 Then, the condition
 \[
  r_1 \cdot \mathbf{v}_1 + r_2 \cdot \mathbf{v}_2 + \ldots + r_k \cdot
  \mathbf{v}_k = \mathbf{0}_V
 \]
 is equivalent to
 \begin{align*}
  &r_1 \cdot (a_{1,1} \cdot \mathbf{b}_1 + \ldots + a_{n,1} \cdot \mathbf{b}_n)
  +\\
  &r_2 \cdot (a_{1,2} \cdot \mathbf{b}_1 + \ldots + a_{n,2} \cdot \mathbf{b}_n)
  +\\
  &\ldots +\\
  &r_k \cdot (a_{1,k} \cdot \mathbf{b}_1 + \ldots + a_{n,k} \cdot \mathbf{b}_n) =
  \mathbf{0}_V.
 \end{align*}
 Grouping together coefficients of the basis vectors
 $\mathbf{b}_1,\ldots,\mathbf{b}_n$ gives
 \begin{align*}
  &(r_1 \cdot a_{1,1} + r_2 \cdot a_{1,2} + \ldots + r_k \cdot a_{1,k}) \cdot
  \mathbf{b}_1 +\\
  &(r_1 \cdot a_{2,1} + r_2 \cdot a_{2,2} + \ldots + r_k \cdot a_{2,k}) \cdot
  \mathbf{b}_2+\\
  &\ldots+\\
  &(r_1 \cdot a_{n,1} + r_2 \cdot a_{n,2} + \ldots + r_k \cdot a_{n,k}) \cdot
  \mathbf{b}_n = \mathbf{0}_V.
 \end{align*}
 By \myref{proposition}{prop:linear-independence-zero-vector}, this equality is
 satisfied if and only if each of the coefficients is equal to $0$. On the
 horizon there glitters the homogeneous linear system
 \[
  \begin{array}{r c r c r c r c l}
   r_1 \cdot a_{1,1} & + & r_2 \cdot a_{1,2} & + & \ldots & + & r_k \cdot
   a_{1,k} & = & 0\\
   r_1 \cdot a_{2,1} & + & r_2 \cdot a_{2,2} & + & \ldots & + & r_k \cdot
   a_{2,k} & = & 0\\
    & & & & & & & \vdots &\\
   r_1 \cdot a_{n,1} & + & r_2 \cdot a_{n,2} & + & \ldots & + & r_k \cdot
   a_{n,k} & = & 0,
  \end{array}
 \]
 which can be rewritten (as we've done many times before) into vector form as
 \[
  r_1 \cdot 
  \begin{pmatrix}
   a_{1,1}\\
   a_{2,1}\\
   \vdots\\
   a_{n,1}
  \end{pmatrix}
  + r_2 \cdot 
  \begin{pmatrix}
   a_{1,2}\\
   a_{2,2}\\
   \vdots\\
   a_{n,2}
  \end{pmatrix}
  + \ldots + r_k \cdot 
  \begin{pmatrix}
   a_{1,k}\\
   a_{2,k}\\
   \vdots\\
   a_{n,k}
  \end{pmatrix} = 
  \begin{pmatrix}
   0\\
   0\\
   \vdots\\
   0
  \end{pmatrix}.
 \]
 Shown vectors are of course just representations of the vectors $\mathbf{v}_1$
 up to $\mathbf{v}_n$ with respect to $B$ and so the result is proven.
\end{lemproof}


\begin{exercise}{}{}
 Decide which of the following sets are linearly independent.
 \begin{enumerate}[label=(\alph*)]
  \item $\left\{ 
   \begin{pmatrix}
    1\\
    -3\\
    5
   \end{pmatrix},
   \begin{pmatrix}
    2\\
    2\\
    4
   \end{pmatrix},
   \begin{pmatrix}
    4\\
    -4\\
    14
   \end{pmatrix}
   \right\}$;
  \item $\left\{ 
   \begin{pmatrix}
    1\\
    7\\
    7
   \end{pmatrix},
   \begin{pmatrix}
    2\\
    7\\
    7
   \end{pmatrix},
   \begin{pmatrix}
    3\\
    7\\
    7
   \end{pmatrix}
   \right\}$;
  \item $\left\{ 
   \begin{pmatrix}
    0\\
    0\\
    -1
   \end{pmatrix},
   \begin{pmatrix}
    1\\
    0\\
    4
   \end{pmatrix}
   \right\}$;
  \item $\left\{ 
   \begin{pmatrix}
    9\\
    9\\
    0
   \end{pmatrix},
   \begin{pmatrix}
    2\\
    0\\
    1
   \end{pmatrix},
   \begin{pmatrix}
    3\\
    5\\
    -4
   \end{pmatrix},
   \begin{pmatrix}
    12\\
    12\\
    -1
   \end{pmatrix}
   \right\}$.
 \end{enumerate}
\end{exercise}
\begin{exercise}{}{}
 Determine which of the sets are linearly independent in the space of quadratic 
 polynomials.
 \begin{enumerate}[label=(\alph*)]
  \item $\{3-x+9x^2,5-6x+3x^2,1+1x-5x^2\}$;
  \item $\{-x^2,1+4x^2\}$;
  \item $\{2+x+7x^2,3-x+2x^2,4-3x^2\}$.
 \end{enumerate}
\end{exercise}
\begin{exercise}{}{}
 Prove that each of the following sets is linearly independent in the vector
 space of all functions $f:(0,\infty) \to \R$.
 \begin{enumerate}[label=(\alph*)]
  \item $\{x \mapsto x,x \mapsto \frac{1}{x}\}$;
  \item $\{x \mapsto \cos x,x \mapsto \sin x\}$;
  \item $\{x \mapsto \exp x,x \mapsto \log x\}$.
 \end{enumerate}
\end{exercise}
\begin{exercise}{}{}
 Prove that the rows of a real-valued matrix in echelon form are a linearly
 independent set.
\end{exercise}
\begin{exercise}{}{}
 Prove that if $\{\mathbf{x},\mathbf{y},\mathbf{z}\}$ is a linearly independent
 set, then so are all its proper subsets: $\{\mathbf{x},\mathbf{y}\}$,
 $\{\mathbf{x},\mathbf{z}\}$, $\{\mathbf{y},\mathbf{z}\}$, $\{\mathbf{x}\}$,
 $\{\mathbf{y}\}$, $\{\mathbf{z}\}$ and $\emptyset$. Is the converse also true?
\end{exercise}
\begin{exercise}{}{}
 Is there a set of four vectors in $\R^3$ such that any three of them form a
 linearly independent set?
\end{exercise}
\begin{exercise}{}{}
 Prove that a set of two perpendicular non-zero vectors in $\R^{n}$ is always
 linearly independent as long as $n > 1$. Generalise the result to more than two
 vectors.
\end{exercise}
\begin{exercise}{}{}
 Decide whether $\{x^2 - x + 1, 2x + 1,2x - 1\}$ and $\{x + x^2,x - x^2\}$ are bases
 of the space of quadratic polynomials.
\end{exercise}
\begin{exercise}{}{}
 Find a basis for the solution set of the linear system
 \[
  \begin{array}{r c r c r c r c l}
   x_1 & - & 4x_2 & + & 3x_3 & - & x_4 & = & 0\\
   2x_1 & - & 8x_2 & + & 6x_3 & - & 2x_4 & = & 0.
  \end{array}
 \]
\end{exercise}
\begin{exercise}{}{}
 Find a basis for $\R^{2 \times 2}$, the space of $2 \times 2$ real matrices.
\end{exercise}
\begin{exercise}{}{}
 Let $(\mathbf{b}_1,\mathbf{b}_2,\mathbf{b}_3)$ be a basis.
 \begin{enumerate}[label=(\alph*)]
  \item Show that $(r_1 \cdot \mathbf{b}_1, r_2 \cdot \mathbf{b}_2, r_3 \cdot
   \mathbf{b}_3)$ is also a basis as long $r_1,r_2,r_3 \neq 0$. What happens if
   at least one of $r_i$ \textbf{is} zero?
  \item Prove that $(\mathbf{a}_1,\mathbf{a}_2,\mathbf{a}_3)$ is also a basis
   where $\mathbf{a}_i = \mathbf{b}_1 + \mathbf{b}_i$, $i \in \{1,2,3\}$.
 \end{enumerate}
\end{exercise}
\begin{exercise}{}{}
 \myref{Theorem}{thm:characterization-of-a-basis} shows that, with respect to a
 basis, every linear combination is unique. If a subset is not a basis, can
 linear combinations be not unique? If so, must they be?
\end{exercise}
\begin{exercise}{}{}
 Represent the polynomials
 \[
  \begin{array}{r r r}
   \text{a) } 2 + 4x^2 \hspace{2em}& \text{b) } 1 + 3x^2 \hspace{2em}& \text{c)
   } 1 + 5x^2
  \end{array}
 \]
 with respect to the basis $B = (1 - x, 1 + x, x^2)$ of the space of quadratic
 polynomials. Use these representations to show that the three featured
 polynomials are linearly dependent.
\end{exercise}
\begin{exercise}{}{}
 Represent the vector
 \[
  \begin{pmatrix}
   1\\
   2
  \end{pmatrix}
 \]
 with respect to the basis
 \[
  B = \left( 
  \begin{pmatrix}
   1\\
   1
  \end{pmatrix},
  \begin{pmatrix}
   -1\\
   1
  \end{pmatrix}
  \right)
 \]
 of $\R^2$.
\end{exercise}
