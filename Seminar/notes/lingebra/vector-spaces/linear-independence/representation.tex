\subsection{Representation With Respect To A Basis}
\label{ssec:representation-with-respect-to-a-basis}

\myref{Theorem}{thm:characterisation-of-a-basis} leads to a corollary of mainly
computational importance: \textbf{every} vector in a vector space $V$ with basis
$B$ corresponds to \textbf{exactly one} sequence of real coefficients of the
linear combination of vectors from $B$ that equals this vector.

To put this symbolically, denote $B =
(\mathbf{b}_1,\mathbf{b}_2,\ldots,\mathbf{b}_n)$ and consider a vector
$\mathbf{v} \in V$. By the mentioned
\myref{theorem}{thm:characterisation-of-a-basis}, there exist exactly one
$n$-tuple $(r_1,\ldots,r_n) \in \R^{n}$ such that
\[
 \mathbf{v} = r_1 \cdot \mathbf{b_1} + r_2 \cdot \mathbf{b}_2 + \ldots + r_n
 \cdot \mathbf{b}_n.
\]
However, in \myref{chapter}{chap:linear-systems}, we observed that elements of
$\R^{n}$ are really just $n$-dimensional vectors with entries in $\R$. These two
facts brought together beget an important idea we shall formalise in due time --
\emph{vector spaces of dimension $n$ are `equivalent' to $\R^{n}$}. The last
sentence should be read as such: in every vector space $V$, we can choose a
basis $B$ and write every vector in $V$ as a linear combination of vectors from
$B$. The coefficients of this linear combination (that are unique for every
vector) can be assembled into a vector in $\R^{n}$. This forges a two-way
relationship (a correspondence, if you will) between vectors in $V$ and vectors
in $\R^{n}$. We call this relationship a \emph{representation} of the vector
$v \in V$ for the reason that it gives a concrete form to an abstract vector.

\begin{definition}{Representation of a vector}{representation-of-a-vector}
 Let $V$ be a vector space with basis $B = (\mathbf{b}_1,\ldots,\mathbf{b}_n)$
 and $\mathbf{v} \in V$. We call the vector
 \[
  [\mathbf{v}]_B \coloneqq
  \begin{pmatrix}
   r_1\\
   r_2\\
   \vdots\\
   r_n
  \end{pmatrix}
 \in \R^{n}
 \]
 a \emph{representation of $\mathbf{v}$ with respect to $B$} if $\mathbf{v} =
 r_1 \cdot \mathbf{b}_1 + r_2 \cdot \mathbf{b}_2 + \ldots + r_n \cdot
 \mathbf{b}_n$.
\end{definition}

\begin{remark}{}{}
 The \hyperref[def:representation-of-a-vector]{preceding definition} underlines
 the necessity of defining a basis as a \textbf{sequence}, not just a set. A
 permutation of the elements of a basis changes the representation of many
 vectors with respect to it.
\end{remark}

The notion of \emph{representation} formalises the approach we've taken many
times ere of `writing' polynomials or matrices as vectors of coefficients.
Confront the following example. 

\begin{example}{}{}
 In the space of cubic polynomials, the representation of the polynomial $x +
 x^2$ with respect to the basis $B = (1, 2x, 2x^2, 2x^3)$ is given by
 \[
  [x + x^2]_B = 
  \begin{pmatrix}
   0\\
   1 / 2\\
   1 / 2\\
   0  
  \end{pmatrix}.
 \]
 With respect to a different basis $C = (1 + x, 1 - x, x + x^2, x + x^3)$
 instead looks like this:
 \[
  [x + x^2]_C = 
  \begin{pmatrix}
   0\\
   0\\
   1\\
   0
  \end{pmatrix}
 \]
\end{example}

\begin{problem}{}{}
 Find the representation of the vector
 \[
  \mathbf{v} = 
  \begin{pmatrix}
   3\\
   2
  \end{pmatrix}
 \]
 with respect to
 \[
  B = \left( 
   \begin{pmatrix}
    1\\
    1
   \end{pmatrix},
   \begin{pmatrix}
    0\\
    2
   \end{pmatrix}
  \right).
 \]
\end{problem}
\begin{probsol}
 We need to find 
\end{probsol}
