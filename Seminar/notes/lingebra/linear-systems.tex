\chapter{Linear Systems}
\label{chap:linear-systems}

Linear systems are by definition sets of linear equations, that is, of equations
which relate present variables in a \emph{linear} way. It is important to
understand what this means. Spelled out, an expression on either side of any of
the equations is formed \emph{solely} by
\begin{enumerate}
 \item multiplying the variables by a given number (\textbf{not another
  variable}),
 \item adding these multiples together.
\end{enumerate}
Any such combination where variables are only allowed to be multiplied by a
constant and added is called a \emph{linear combination}. This term is extremely
important and ubiquitous throughout the text; hence, it warrants an isolated
definition. 

\begin{definition}{Linear combination}{linear-combination}
 Let $x_1,\ldots,x_n$ with $n \in \N$ be variables. Their \emph{linear
 combination} is any expression of the form
 \[
  a_1x_1 + a_2x_2 + \cdots + a_n x_n
 \]
 where $a_1,\ldots,a_n$ are numbers.
\end{definition}

\begin{remark}{}{linear-combination}
 In the \hyperref[def:linear-combination]{definition above}, we have
 deliberately not specified what type of \emph{numbers} we mean. In the future,
 we shall work extensively with real and complex numbers as well as elements of
 other fields, which dear readers might not have even recognised as `numbers'
 thus far. The only important concept in this regard is the clear distinction
 between a \emph{number} (later \emph{scalar}) and a \emph{variable} (later
 \emph{vector}).
\end{remark}

\begin{example}{}{linear-combination}
 Consider the variables $x,y$ and $z$. The expression
 \[
  3x + 2y - 0.5z
 \]
 \textbf{is} their linear combination whereas
 \[
  5x + 3y - yz + 7z^2
 \]
 \textbf{is not}.
\end{example}

To reiterate, a \emph{linear system} is any set of equations featuring only
linear combinations of variables; these equations are consequently called
\emph{linear} as well. A \emph{solution} of a linear system is the set of all
possible substitutions of numbers (in place of variables) which make the
equations true.

It is clear that every linear equation can be rearranged to
\[
 a_1 x_1 + \cdots + a_n x_n = c
\]
for some variables $x_1,\ldots,x_n$ and numbers $a_1,\ldots,a_n,c$, by simple
subtraction. This is how we shall define it, for simplicity.

\begin{definition}{Linear equation}{linear-equation}
 Any equation of the form
 \begin{equation}
  \label{eq:linear-equation}
  a_1x_1 + a_2x_2 + \cdots + a_n x_n = c
 \end{equation}
 where $x_1,\ldots,x_n$ are variables and $a_1,\ldots,a_n,c$ are numbers, is
 called \emph{linear}. A \emph{solution} of a linear equation is an $n$-tuple
 $(b_1,\ldots,b_n)$ of numbers such that under the substitutions $x_i \coloneqq
 b_i$, for $i \in \{1,\ldots,n\}$, the equation~\eqref{eq:linear-equation} is
 satisfied.
\end{definition}

\begin{example}{}{linear-equation}
 The equation
 \[
  3x_1 - 2x_2 + 4x_3 + x_4 = 5
 \]
 is \hyperref[def:linear-equation]{linear} in variables $x_1,x_2,x_3$ and $x_4$.
 On the contrary,
 \[
  3x_1x_2 - 4x_3^2 = 10
 \]
 is \textbf{not} linear.
\end{example}

\begin{definition}{Linear system}{linear-system}
 Any set of linear equations in the given variables $x_1,\ldots,x_n$ is called a
 \emph{linear system}. A \emph{solution} of a linear system is an $n$-tuple
 $(b_1,\ldots,b_n)$ which solves every linear equation in the set.
\end{definition}

\begin{example}{}{linear-system}
 The set of equations
 \[
  \begin{array}{rcrcrcr}
   3x_1 & - & x_2 & + & 2x_3 &= &1\\
   x_1 & & & - & x_3 &= &-1 \\
   2x_1 & - & 3x_2 & + & 3x_3 & = & 0
  \end{array}
 \]
 is a \hyperref[def:linear-system]{linear system} whose solution is the triple
 $(0,1,1)$.
\end{example}

%TODO reference
We proceed to discuss two trivial examples, which readers might have discussed
in high school, that naturally lead to linear systems. More sophisticated
examples are presented in the applications section.

\begin{example}{Static equations}{static-equations}
 Suppose we have three objects -- one with a mass of $2$ and the other two with
 masses unknown. Experimentation produces these two balances.
 \begin{figure}[H]
  \centering
  \begin{subfigure}[b]{.45\textwidth}
   \centering
   \begin{tikzpicture}
    \node[isosceles triangle, draw, fill=black!30, anchor=left corner, shape
     border rotate=90, minimum height=5mm, minimum width=1cm, isosceles
     triangle stretches] (base) at (0,0) {};
    \coordinate[left=3 of base.north] (left);
    \coordinate[right=3 of base.north] (right);
    \draw (left) -- (right);

    \node[circle,draw,left=0.75cm of base.north,minimum height=6mm,inner
     sep=0,yshift=3mm] (x) {$x$};
    \node[circle,draw,left=2cm of base.north,minimum height=4mm,inner
     sep=0,yshift=2mm] (y) {$y$};
    \node[circle,draw,right=2.5cm of base.north,minimum height=5mm,inner
     sep=0,yshift=2.5mm] (two) {$2$};

    \coordinate[above=2cm of base.north] (above);
    \draw[dashed] (base.north) -- (above);
    \draw[dashed] (x.north) -- ($(above) - (1.05,0)$);
    \draw[dashed] (y.north) -- ($(above) - (2.2,0)$);
    \draw[dashed] (two.north) -- ($(above) + (2.75,0)$);

    \draw[<->,thick] ($(x.north) + (0,0.5)$) to
     node[midway,circle,fill=white,inner sep=1pt] {$15$} ($(x.north) +
     (1.05,0.5)$);
    \draw[<->,thick] ($(y.north) + (0,1.25)$) to
     node[midway,circle,fill=white,inner sep=1pt] {$40$} ($(y.north) +
     (2.2,1.25)$);
    \draw[<->,thick] ($(two.north) + (0,1.15)$) to
     node[midway,circle,fill=white,inner sep=1pt] {$50$} ($(two.north) +
     (-2.75,1.15)$);
   \end{tikzpicture}
  \end{subfigure}
  \hspace*{\fill}
  \begin{subfigure}[b]{.45\textwidth}
   \centering
   \begin{tikzpicture}
    \node[isosceles triangle, draw, fill=black!30, anchor=left corner, shape
     border rotate=90, minimum height=5mm, minimum width=1cm, isosceles
     triangle stretches] (base) at (0,0) {};
    \coordinate[left=3 of base.north] (left);
    \coordinate[right=3 of base.north] (right);
    \draw (left) -- (right);

    \node[circle,draw,left=1.25cm of base.north,minimum height=6mm,inner
     sep=0,yshift=3mm] (x) {$x$};
    \node[circle,draw,right=2.5cm of base.north,minimum height=4mm,inner
     sep=0,yshift=2mm] (y) {$y$};
    \node[circle,draw,right=1.25cm of base.north,minimum height=5mm,inner
     sep=0,yshift=2.5mm] (two) {$2$};

    \coordinate[above=2cm of base.north] (above);
    \draw[dashed] (base.north) -- (above);
    \draw[dashed] (x.north) -- ($(above) - (1.55,0)$);
    \draw[dashed] (y.north) -- ($(above) + (2.7,0)$);
    \draw[dashed] (two.north) -- ($(above) + (1.5,0)$);

    \draw[<->,thick] ($(x.north) + (0,1.05)$) to
     node[midway,circle,fill=white,inner sep=1pt] {$25$} ($(x.north) +
     (1.55,1.05)$);
    \draw[<->,thick] ($(y.north) + (0,1.25)$) to
     node[midway,circle,fill=white,inner sep=1pt] {$50$} ($(y.north) +
     (-2.7,1.25)$);
    \draw[<->,thick] ($(two.north) + (0,0.6)$) to
     node[midway,circle,fill=white,inner sep=1pt] {$25$} ($(two.north) +
     (-1.5,0.6)$);
   \end{tikzpicture}
  \end{subfigure}
 \end{figure}

 For the weights to be in balance, the sum of \emph{moments} on either side of
 the scales must be identical. A \emph{moment} of an object is its distance from
 the centre of the scales times its mass. This condition yields a system of two
 linear equations
 \[
  \begin{array}{ccccccc}
   15x & + & 40y & = & 50 \cdot 2, & &\\
       &   & 25x & = & 25 \cdot 2 & + & 50y.
  \end{array}
 \]
 Or, after rearrangement (to stay true to \hyperref[def:linear-equation]{our
 definition of linear equation}),
 \[
  \begin{array}{ccccc}
   15x & + & 40y & = & 50 \cdot 2,\\
   25x & - & 50y & = & 25 \cdot 2.
  \end{array}
 \]
\end{example}

\begin{example}{Chemical reactions}{chemical-reactions}
 Toluene, $\mathtt{C_7 H_8}$, mixes (under right conditions) with nitric acid,
 $\mathtt{HNO_3}$, to produce trinitrotoluene (widely known as TNT),
 $\mathtt{C_7H_5O_6N_3}$, along with dihydrogen monoxide, $\mathtt{H_2O}$. If we
 want this chemical reaction to occur successfully, we must (among other things)
 ascertain we mix the constituents in the right proportion. In pseudo-chemical
 notation, the reaction to take place can be written as
 \[
  x \cdot \mathtt{C_7H_8} + y \cdot \mathtt{HNO_3} \longrightarrow z \cdot
  \mathtt{C_7H_5O_6N_3} + w \cdot \mathtt{H_2O}.
 \]
 Comparing the number of atoms of each element before the reaction and
 afterwards (which must remain identical owing to the conservation of energy)
 yields the system
 \[
  \begin{array}{cccccccc}
   \mathtt{H}: & 8x & + & 1y & = & 5z & + & 2w,\\
   \mathtt{C}: & 7x &   &    & = & 7z,&   &\\
   \mathtt{N}: &    &   & 1y & = & 3z,&   &\\
   \mathtt{O}: &    &   & 3y & = & 6z & + & 1w.
  \end{array}
 \]
\end{example}

In the next section, we devise an algorithm to solve any system of linear
equations.

\section{Gauss-Jordan Elimination}
\label{sec:gauss-jordan-elimination}

Probably the most well-known algorithm for solving a
\hyperref[def:linear-system]{linear system} is the \emph{Gauss-Jordan
Elimination}. As its name partially implies, its heart lies in the
\emph{elimination} of variables one by one, until only a single linear equation
in one variable stands unsolved. This is done by applying different
\emph{transformations} to the initial system that are guaranteed not to alter
the solution. We're going to solve a linear system first and describe the
general method second.

\begin{problem}{}{gauss-jordan-elimination}
 Solve the linear system
 \[
  \begin{array}{rcrcrcr}
   & & & & 3x_3 & = & 9\\
   x_1 & + & 5x_2 & - & 2x_3 & = & 2\\
   \frac{1}{3}x_1 & + & 2x_2 & & & = & 3
  \end{array}.
 \]
\end{problem}
\begin{probsol}
 We aim to transform the system step by step to a form which allows us to
 (successively) eliminate all variables.

 The first transformation entails a simple exchange of the first and third row.
 \begin{center}
  \begin{tikzpicture}
   \node (eq) at (0,0) {$
     \begin{array}{rcrcrcr}
      \frac{1}{3}x_1 & + & 2x_2 & & & = & 3\\
      x_1 & + & 5x_2 & - & 2x_3 & = & 2\\
      & & & & 3x_3 & = & 9
     \end{array}.
   $};
  \coordinate (eqsw) at ($(eq.south west) + (-0.2,0.36)$);
  \coordinate (eqnw) at ($(eq.north west) - (0.2,0.36)$);
  \draw[<->,thick] (eqsw) to[bend left=90,looseness=2] (eqnw);
  \node at (-4.9,0) {\footnotesize \texttt{Swapped first and third row.}};
  \end{tikzpicture}
 \end{center}
 Next, we shall scale the first row by a factor of $3$.
 \begin{center}
  \begin{tikzpicture}
   \node (eq) at (0,0) {$
     \begin{array}{rcrcrcr}
      x_1 & + & 6x_2 & & & = & 9\\
      x_1 & + & 5x_2 & - & 2x_3 & = & 2\\
      & & & & 3x_3 & = & 9
     \end{array}.
   $};
  \coordinate (eqnw) at ($(eq.north west) - (0.2,0.36)$);
  \coordinate (start) at ($(eqnw) - (1,0)$);
  \draw[->,thick] (start) to node[midway,yshift=2mm] {\footnotesize
   $\mathtt{\cdot 3}$} (eqnw);
  \node at ($(start) - (2,0)$) {\footnotesize \texttt{Scaled the first row by
   3.}};
  \end{tikzpicture}
 \end{center}
 Finally, we subtract the first row from the second row. Said in a more
 foreshadowing manner, we add the $(-1)$-multiple of the first row to the second
 row.
 \begin{center}
  \begin{tikzpicture}
   \node (eq) at (0,0) {$
     \begin{array}{rcrcrcr}
      x_1 & + & 6x_2 & & & = & 9\\
       & - & x_2 & - & 2x_3 & = & -7\\
      & & & & 3x_3 & = & 9
     \end{array}.
   $};
  \coordinate (row1) at ($(eq.north west) - (0.2,0.36)$);
  \coordinate (row2) at ($(eq.west) - (0.2,0)$);
  \draw[->,thick] (row1) to[bend right=90,looseness=2] node[midway,xshift=-4mm]
   (mid) {\footnotesize $\mathtt{\cdot (-1)}$} (row2);
  \node at ($(mid) - (3.5,0)$) {\footnotesize \texttt{Subtracted the first row
   from the second.}};
  \end{tikzpicture}
 \end{center}
 These transformations have wrought the system into a state where it can be
 easily solved.

 Indeed, we immediately see that the third equation implies $x_3 = 3$.
 Substituting into the second equation gives 
 \[
  -x_2 - 2 \cdot 3 = -7
 \]
 whose solution is $x_2 = 1$. Finally, knowing the value of $x_2$, we can solve
 the first equation by another substitution. We get
 \[
  x_1 + 6 \cdot 1 = 9,
 \]
 thus $x_1 = 3$ and the triple $(3,1,3)$ is the \emph{unique} solution of the
 system.
\end{probsol}

Observant readers might have already identified the `kinds' of transformations
that were used in solving the \hyperref[prob:gauss-jordan-elimination]{linear
system above}. Nonetheless, we're about to spell them out.

The transformations that do not change the solution of a
\hyperref[def:linear-system]{linear system} include
\begin{enumerate}
 \item swapping two equations;
 \item scaling an equation by a non-zero constant;
 \item adding a multiple of an equation to \emph{another} equation.
\end{enumerate}
Note that transformations (2) and (3) come with sensible restrictions. Scaling
an equation by $0$ clearly changes the set of solutions of the system as it
basically removes the equation entirely. Adding a multiple of an equation to
\emph{itself} suffers from the same problem; it might result in `invalidating'
the equation should the scaling factor be $-1$.

We know proceed to prove that transformations (1) - (3) truly do not alter the
solutions of the initial system.

\begin{theorem}{Gauss-Jordan}{gauss-jordan}
 The transformations (1) - (3) of a linear system outlined above do not change
 its solution set.
\end{theorem}
\begin{thmproof}
 We will cover transformation (3) here. The proofs for transformations (1) and
 (2) are similar and thus left as an exercise.

 Consider the linear system
 \[
  \begin{array}{rcrcccrcr}
   a_{1,1}x_1 & + & a_{1,2}x_2 & + & \cdots & + & a_{1,n}x_n & = & c_1\\
   a_{2,1}x_1 & + & a_{2,2}x_2 & + & \cdots & + & a_{2,n}x_n & = & c_2\\
              &   &            &   &        &   &            & \vdots &\\
   a_{m,1}x_1 & + & a_{m,2}x_2 & + & \cdots & + & a_{m,n}x_n & = & c_m
  \end{array}
 \]
 of $m$ equations in variables $x_1,\ldots,x_n$ and let $(b_1,\ldots,b_n)$ be
 one of its solutions. Choose a constant $k$ and add the $k$-multiple of the
 $i$-th equation to the $j$-th equations for some $i,j \in \{1,\ldots,m\}$.
 Hence, the $j$-th equation of the system gets replaced by
 \[
  (a_{j,1} + k \cdot a_{i,1})x_1 + (a_{j,2} + k \cdot a_{i,2})x_2 + \cdots +
  (a_{j,n} + k \cdot a_{i,n})x_n = c_j + k \cdot c_i,
 \]
 which can be rearranged to
 \begin{equation}
  \label{eq:gauss-jordan}
  a_{j,1}x_1 + a_{j,2}x_2 + \cdots + a_{j,n}x_n + k \cdot (a_{i,1}x_1 +
  a_{i,2}x_2 + \cdots + a_{i,n}x_n) = c_j + k \cdot c_i.
 \end{equation}
 Since $(b_1,\ldots,b_n)$ is a solution of the original system, we know that
 \[
  \begin{array}{rcrcccrcr}
   a_{i,1}b_1 & + & a_{i,2}b_2 & + & \cdots & + & a_{i,n}b_n & = & c_i\\
   a_{j,1}b_1 & + & a_{j,2}b_2 & + & \cdots & + & a_{j,n}x_n & = & c_j
  \end{array}.
 \]
 Substituting this into equation~\eqref{eq:gauss-jordan} gives
 \[
  c_j + k \cdot c_i = c_j + k \cdot c_i,
 \]
 hence $(b_1,\ldots,b_n)$ is also the solution of the transformed system, as
 required.
\end{thmproof}

\begin{exercise}{}{gauss-jordan}
 Show that transformations (1) and (2) also don't change the set of solutions of
 the transformed linear system.
\end{exercise}

\begin{definition}{Elementary operations}{elementary-operations}
 The transformations (1) - (3) outlined above are called \emph{elementary
 operations} or \emph{row operations}.
\end{definition}

As we've seen in \myref{problem}{prob:gauss-jordan-elimination}, the application
of transformations (1) - (3) has its purpose in preparing the system for a final
back-substitution, where the values of all variables in a row save the first one
are known beforehand. A system which is `ready' to be solved by
back-substitution is said to be in \emph{echelon form}.

\begin{definition}{Echelon form}{echelon-form}
 In each row of a \hyperref[def:linear-system]{linear system}, the first
 variable with a non-zero coefficient is called the row's \emph{leading
 variable}.

 A linear system is in \emph{echelon form} (or \emph{upper triangular form}) if
 the leading variable in each row is at least one column to the right of the
 leading variable in the row above and all rows filled with zeroes are at the
 bottom.
\end{definition}

\begin{example}{}{echelon-form}
 The system
 \[
  \begin{array}{rcrcrcr}
   x_1 & + & 6x_2 & & & = & 9\\
    & - & x_2 & - & 2x_3 & = & -7\\
   & & & & 3x_3 & = & 9
  \end{array}
 \]
 \textbf{is} in echelon form whereas
 \[
  \begin{array}{rcrcrcr}
   2x_1 & + & 3x_2 & - & x_3 & = & 9\\
    & & 3x_2 & - & 2x_3 & = & 2\\
    x_1 & & & - & x_3 & = & 0
  \end{array}
 \]
 is \textbf{not}.
\end{example}

For now, we shall employ intuition and a nibble of foresight to guide our
transformation of a \hyperref[def:linear-system]{linear system} into its
\hyperref[def:echelon-form]{echelon form}. Later, we intend to present a precise
algorithm (that computers also use) that achieves this.

\begin{example}{}{echelon-form-2}
 We're going to put the system
 \[
  \begin{array}{rcrcrcr}
    x_1 & + & x_2 & & & = & 0\\
    2x_1 & - & x_2 & + & 3x_3 & = & 3\\
    x_1 & - & 2x_2 & - & x_3 & = & 3
  \end{array}
 \]
 into echelon form and solve it using back-substitution. We'll label the rows of
 the system by Roman letters and denote transformations accordingly. For
 example, adding a $3$-multiple of row one to row three would be written
 symbolically as $\mathtt{3 \cdot I + III}$.

 First, we need to get rid of the variable $x_1$ in rows \texttt{II} and
 \texttt{III}. This can be done by subtracting adequate multiples of row
 \texttt{I}.
 \begin{center}
  \begin{tikzpicture}
   \node at (-6,0) {$
    \begin{array}{rcrcrcr}
      x_1 & + & x_2 & & & = & 0\\
      2x_1 & - & x_2 & + & 3x_3 & = & 3\\
      x_1 & - & 2x_2 & - & x_3 & = & 3
    \end{array}
    $};
   \node (eq) at (0,0) {$
    \begin{array}{rcrcrcr}
      x_1 & + & x_2 & & & = & 0\\
      & - & 3x_2 & + & 3x_3 & = & 3\\
      & - & 3x_2 & - & x_3 & = & 3
    \end{array}
   $};
  \draw[|->,thick,shorten <=5pt, shorten >=5pt] ($(eq.west) - (2.5,0)$) to
   node[midway,yshift=2mm] {\footnotesize $\mathtt{-2I + II}$}
   node[midway,yshift=-2mm] {\footnotesize $\mathtt{-I + III}$} (eq.west);
  \end{tikzpicture}
 \end{center}
 We continue by subtracting row \texttt{II} from row \texttt{III}.
 \begin{center}
  \begin{tikzpicture}
   \node at (-6,0) {$
    \begin{array}{rcrcrcr}
      x_1 & + & x_2 & & & = & 0\\
      & - & 3x_2 & + & 3x_3 & = & 3\\
      & - & 3x_2 & - & x_3 & = & 3
    \end{array}
    $};
   \node (eq) at (0,0) {$
    \begin{array}{rcrcrcr}
      x_1 & + & x_2 & & & = & 0\\
      & - & 3x_2 & + & 3x_3 & = & 3\\
      & & & - & 4x_3 & = & 0
    \end{array}
   $};
  \draw[|->,thick,shorten <=5pt, shorten >=5pt] ($(eq.west) - (2.5,0)$) to
   node[midway,yshift=2mm] {\footnotesize $\mathtt{-II + III}$} (eq.west);
  \end{tikzpicture}
 \end{center}
 The system is now in \hyperref[def:echelon-form]{echelon form}. The equation in
 row \texttt{III} forces $x_3 = 0$. Substitution into row \texttt{II}
 immediately gives $x_2 = -1$ and one final substitution into row \texttt{I}
 yields $x_1 = 1$.

 Hence, the solution of the system is the triple $(1, -1, 0)$.
\end{example}

\begin{exercise}{}{echelon-form}
 Using \emph{Gauss-Jordan elimination} solve the systems from
 examples~\ref{exam:static-equations} and~\ref{exam:chemical-reactions}.
\end{exercise}

All the systems we've studied so far have had the same number of equations as
variables. This of course need not be the case in general. Thankfully,
Gauss-Jordan elimination can \emph{always} be used to determine the solution set
of a \hyperref[def:linear-system]{linear system}. However, this set can also be
empty or infinite in cases where the number of variables doesn't match the
number of equations. The following two examples illustrate this.

\begin{example}{}{echelon-form-3}
 The following system has more equations than variables.
 \begin{equation}
  \label{eq:overdetermined-system}
  \begin{array}{rcrcr}
    x_1 & + & 3x_2 & = & 1\\
    2x_1 & + & x_2 & = & -3\\
    2x_1 & + & 2x_2 & = & -2
  \end{array}
 \end{equation}
 Before we put it into \hyperref[def:echelon-form]{echelon form} and solve it,
 let us ponder what the solution set may look like. Intuitively, a linear
 equation is basically a `restraint' or `condition' on the range of possible
 values the present variables may attain. If there are three equations
 restraining only two variables, then this restraint may be too harsh and lead
 to the system having no solution at all. The only case where solution
 \emph{does} exist involves one of the equations being \emph{redundant} --
 providing no additional condition. Algebraically, this happens if said equation
 is a \hyperref[def:linear-combination]{linear combination} of the other two.

 To draw a `real-life' simile, imagine the price of an apple being \$5/kg and
 that of bananas \$1.5/kg. Saying that 3 kg of apples and 4 kg of bananas cost,
 say, \$30 is simply false because we can calculate that this amount actually
 costs \$21. The third condition on the price of apples and bananas contradicted
 the previous two; just as a third equation in a
 \hyperref[def:linear-system]{linear system} in two variables can contradict the
 first two equations. We tend to call such systems \emph{overdetermined} and
 will in time dedicate a section to finding a `good' approximation of their
 solution.

 To solve the system~\eqref{eq:overdetermined-system}, we transform it into
 echelon form. First, we subtract twice the first row from the other two.
 \begin{center}
  \begin{tikzpicture}
   \node at (-5.5,0) {$
    \begin{array}{rcrcr}
      x_1 & + & 3x_2 & = & 1\\
      2x_1 & + & x_2 & = & -3\\
      2x_1 & + & 2x_2 & = & -2
    \end{array}
   $};
   \node (eq) at (0,0) {$
    \begin{array}{rcrcr}
      x_1 & + & 3x_2 & = & 1\\
      & & -5x_2 & = & -5\\
      & & -4x_2 & = & -4
    \end{array}
   $};
  \draw[|->,thick,shorten <=5pt, shorten >=5pt] ($(eq.west) - (2.5,0)$) to
   node[midway,yshift=2mm] {\footnotesize $\mathtt{-2I + II}$}
   node[midway,yshift=-2mm] {\footnotesize $\mathtt{-2I + III}$}(eq.west);
  \end{tikzpicture}
 \end{center}
 Finally, we add $(-4 / 5)$-times row \texttt{II} to row \texttt{III}.
 \begin{center}
  \begin{tikzpicture}
   \node at (-6,0) {$
    \begin{array}{rcrcr}
      x_1 & + & 3x_2 & = & 1\\
      & & -5x_2 & = & -5\\
      & & -4x_2 & = & -4
    \end{array}
   $};
   \node (eq) at (0,0) {$
    \begin{array}{rcrcr}
      x_1 & + & 3x_2 & = & 1\\
      & & -5x_2 & = & -5\\
      & & 0 & = & 0
    \end{array}
   $};
  \draw[|->,thick,shorten <=5pt, shorten >=5pt] ($(eq.west) - (3,0)$) to
   node[midway,yshift=2mm] {\footnotesize $\mathtt{-(4 / 5)II + III}$}
   (eq.west);
  \end{tikzpicture}
 \end{center}
 Clearly, the third equation is \emph{redundant} because it provides no
 condition on the values of the variables. Back-substitution yields $x_2 = 1$
 and $x_1 = -2$. As we've claimed (but not yet proven), row \texttt{III} is
 indeed a linear combination of rows \texttt{I} and \texttt{II}. In this
 particular case, it holds that $\mathtt{(2 / 5)I + (4 / 5)II = III}$.
\end{example}

