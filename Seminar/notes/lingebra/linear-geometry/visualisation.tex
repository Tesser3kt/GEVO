\section{Visualisation of Linear Systems Revisited}
\label{sec:visualisation-of-linear-systems-revisited}

In \myref{section}{sec:visualizing-linear-systems}, we discussed geometric
properties of the sets of solutions of linear systems. In
\myref{section}{sec:describing-solution-sets-of-linear-systems}, we described
them as sets of vectors. Finally, now that we have revealed the geometric side
of vectors as well, the two different ways of looking at sets of solutions of
linear systems should align. The conception of this alignment is the content of
this, rather brief, section.

The solution of the linear equation
\[
 x + 3y = 4
\]
is the set $\{(4-3y,y) \mid y \in \R\}$ and also the set
\[
 \left\{ 
  \begin{pmatrix}
   4\\
   0
  \end{pmatrix}
  + y 
  \begin{pmatrix}
   -3\\
   1
  \end{pmatrix} \mid y \in \R
 \right\}.
\]
We've already proven that the first set describes a line. Under our current
geometric interpretation of vectors, does the second set describe the same line?
As you may expect, the answer is \emph{yes}, but only if we identify (as we
already have multiple times) the targets of vectors with the vectors themselves.
You see, the second set is a set of \emph{vectors} while the first one is a set
of \emph{points}. The idea here is that these two sets are the same as long as
we consider the second set as a line formed by the ends or targets of the
vectors within.

Now, any vector of the form $\begin{psmallmatrix} 4\\0 \end{psmallmatrix} + y
\begin{psmallmatrix} -3 \\ 1 \end{psmallmatrix}$ is a vector which is rooted at
the end of $\begin{psmallmatrix} 4\\0 \end{psmallmatrix}$ and then extended by
an arbitrary length in the direction of $\begin{psmallmatrix} -3 \\ 1
\end{psmallmatrix}$ (see \myref{figure}{fig:line-set-of-vectors}). This means
that in order to reach any point on the line, we must travel $4$ steps to the
right (in the direction of $\begin{psmallmatrix} 4\\0 \end{psmallmatrix}$) and
then some distance in the direction of $\begin{psmallmatrix} -3 \\ 1
\end{psmallmatrix}$. Clearly, if we separate the directions, we see that we
move by $4 + y \cdot (-3)$ in the horizontal direction and by $y \cdot 1$ in
the vertical direction. The point we reach this way has coordinates $(4 - 3y,
y)$ for some choice of $y \in \R$. This shows that we are indeed moving along
the same line; as well we should.

\begin{figure}[ht]
 \centering
 \begin{tikzpicture}[scale=1.5]
  \tkzInit[xmin=-1,xmax=3,ymin=-1,ymax=2]
  \tkzDrawX[arrows={-Latex[width=4pt,length=6pt]},label=,black!30]
  \tkzDrawY[arrows={-Latex[width=4pt,length=6pt]},label=,black!30]
  \tkzDefPoints{0/0/O,1/1/a,-2/2/b,2.5/0.5/c}
  \tkzDrawSegment[-Latex,thick,BrickRed,shorten <=2pt,shorten >=2pt](O,a)
  \tkzDrawSegment[-Latex,thick,RoyalBlue,shorten <=2pt,shorten >=2pt](a,b)
  \tkzDrawSegment[-Latex,thick,RoyalBlue,shorten <=2pt,shorten >=2pt](a,c)
  \tkzDrawSegments[-Latex,thick,dashed,ForestGreen,shorten <=2pt,shorten
  >=2pt](O,b O,c)
  \tkzDrawLine[dashed,add=.5 and .5](b,c)
  \tkzDrawPoints[size=3,fill=black](a,b,c)

  \tkzLabelSegment[above left](O,a){$\clr{\mathbf{v}}$}
  \tkzLabelSegment[below left](O,b){$\clr{\mathbf{v}} + 1\clb{\mathbf{w}}$}
  \tkzLabelSegment[below right](O,c){$\clr{\mathbf{v}} -
  \frac{1}{2}\clb{\mathbf{w}}$}
  \tkzLabelSegment[above=1mm](a,b){$1\clb{\mathbf{w}}$}
  \tkzLabelSegment[above=1mm](a,c){$-\frac{1}{2}\clb{\mathbf{w}}$}
 \end{tikzpicture}

 \caption{Line as a set of vectors. Every \clg{vector} whose end lies on the
 line is of the form $\clr{\mathbf{v}} + y \clb{\mathbf{w}}$ for $y \in \R$.}
 \label{fig:line-set-of-vectors}
\end{figure}

Since we already proved in
\myref{section}{sec:describing-solution-sets-of-linear-systems} that the two
descriptions (using points vs. using vectors) of the sets of solutions of linear
systems are equivalent, we shan't dwell on this matter much longer. Let us close
this section with two examples from $\R^3$, the kind we studied and visualised
in \myref{subsection}{ssec:three-dimensional-linear-systems}.

The linear equation
\[
 x - y + z = 4
\]
defines a plane in $\R^3$. Its solution set can be represented as the set of
points $\{(4 + y - z, y, z) \mid y,z \in \R\}$ or the set of (ends of) vectors
\[
 \left\{ 
  \begin{pmatrix}
   4\\
   0\\
   0
  \end{pmatrix} + y
  \begin{pmatrix}
   1\\
   1\\
   0
  \end{pmatrix} + z
  \begin{pmatrix}
   -1\\
   0\\
   1
  \end{pmatrix}
 \right\}.
\]
This second representation reveals that we're dealing with a plane created by
moving freely in the directions of $\begin{psmallmatrix} 1\\1\\0
\end{psmallmatrix}$ and $\begin{psmallmatrix} -1\\0\\1 \end{psmallmatrix}$,
shifted $4$ steps to the right from the origin (in the direction of
$\begin{psmallmatrix} 4\\0\\0 \end{psmallmatrix}$). As a matter of fact, the
\hyperref[thm:triangle-inequality]{triangle inequality} assures that the
geometric object defined as the set of all vectors of the form $\mathbf{u} + y
\mathbf{v} + z \mathbf{w}$, for $\mathbf{u},\mathbf{v},\mathbf{w} \in \R^{n}$ and
$y,z \in \R$, is always a plane (that is a `two-dimensional flat object')
because the shortest distance between two points on such an object is always the
straight segment connecting them. Kind readers would do well to realize this is
the very definition of `flatness'.

\begin{figure}[ht]
 \centering
 \begin{tikzpicture}
  \tkzInit[xmin=-1,xmax=4,ymin=-1,ymax=5]
  \tkzDrawX[arrows={-Latex[width=4pt,length=6pt]},label=,black!20]
  \tkzDrawY[arrows={-Latex[width=4pt,length=6pt]},label=,black!20]
  \tkzDefPoints{-3/1/x1,1/5/x2,5/5/x3,1/1/x4}
  \tkzDrawPolygon[black!20,fill=black!20](x1,x2,x3,x4)
  \tkzDefPoints{-1/-1/z1,3/3/z2}
  \tkzDrawLine[arrows={-Latex[width=4pt,length=6pt]},black!20,add=0 and
  0.4](z1,z2)
  \tkzDefPoints{0/0/O,-1/2/a,1/4/b,1/2/c,3/4/d}
  \tkzDrawSegment[-Latex,thick,BrickRed,shorten <=2pt,shorten >=2pt](O,a)
  \tkzLabelSegment[right,BrickRed](O,a){$\mathbf{u}$}
  \tkzDrawSegment[-Latex,thick,RoyalBlue,shorten <=2pt,shorten >=2pt](a,b)
  \tkzLabelSegment[above left,RoyalBlue](a,b){$2\mathbf{v}$}
  \tkzDrawSegment[-Latex,thick,Fuchsia,shorten <=2pt,shorten >=2pt](a,c)
  \tkzLabelSegment[below,Fuchsia](a,c){$1\mathbf{w}$}
  \tkzDrawSegment[-Latex,thick,dashed,ForestGreen,shorten <=2pt,shorten
  >=2pt](O,d)
  \tkzDrawSegment[-Latex,dashed,RoyalBlue!80,shorten <=2pt,shorten >=2pt](b,d)
  \tkzDrawSegment[-Latex,dashed,Fuchsia!80,shorten <=2pt,shorten >=2pt](c,d)
  \tkzDrawPoints[size=3,fill=black](a,b,c,d)
  \tkzLabelSegment[below right](O,d){$\clr{\mathbf{u}} + 2\clb{\mathbf{v}} +
  1\clm{\mathbf{w}}$}
  \tkzDrawLine[dashed,add=.75 and .75](a,b)
  \tkzDrawLine[dashed,add=1 and 1](a,c)
 \end{tikzpicture}
 \caption{Plane as a set of vectors. Every \clg{vector} lying on the plane is of
 the form $\clr{\mathbf{u}} + y\clb{\mathbf{v}} + z \clm{\mathbf{w}}$ for some
 $y,z \in \R$.}
 \label{fig:plane-set-of-vectors}
\end{figure}

Adding another linear equation creates an intersection of two planes -- a line
in $\R^3$. This is actually best seen from its vector representation. The system
\[
 \begin{array}{r c r c r c r}
  x & - & y & + & z & = & 4\\
  -x & + & 3y & - & 3z & = & 0
 \end{array}
\]
has the solution set $\{(6, z + 2, z) \mid z \in \R\}$ or
\[
 \left\{
  \begin{pmatrix}
   6\\
   2\\
   0
  \end{pmatrix} + z
  \begin{pmatrix}
   0\\
   1\\
   1
  \end{pmatrix}
 \right\}.
\]
The latter description immediately suggests that the geometric object in
question is indeed a line -- we reach every solution by first moving along
$\begin{psmallmatrix} 6 \\ 2 \\ 0\end{psmallmatrix}$ and then any distance
whatsoever in the direction of $\begin{psmallmatrix} 0 \\ 1 \\
1\end{psmallmatrix}$.

