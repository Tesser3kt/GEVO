% Language %
\usepackage[czech,english]{babel}

% Encoding %
\usepackage[T1]{fontenc}

% Turn off Czech hyphens %
\AtBeginDocument{\shorthandoff{-}}

% Make chapter standalone
\usepackage{bookmark}

% Layout %
\usepackage[%
 bmargin=1.25in,
 tmargin=1.25in,
 inner margin=1.25in,
 outer margin=1in
]{geometry}

% Header & Footer %
\renewcommand{\chaptermark}[1]{\markboth{#1}{}}
\renewcommand{\sectionmark}[1]{\markright{\thesection\ #1}}
\usepackage{fancyhdr}
\pagestyle{fancy}
\fancyhf{}
\fancyhead[LE,RO]{\thepage}
\fancyhead[RE]{\textit{\nouppercase{\leftmark}}}
\fancyhead[LO]{\textit{\nouppercase{\rightmark}}}
\fancypagestyle{plain}{%
 \fancyhf{}
 \renewcommand{\headrulewidth}{0pt}
}
\setlength{\headheight}{16pt}

% AMS LaTeX %
\usepackage{amsmath,mathtools}
\usepackage{dsfont}
\usepackage[thmmarks,amsmath]{ntheorem}

% Fonts %
\usepackage[libertine]{newtx}
\usepackage[makeroom]{cancel} % crossing out stuff
\renewcommand*{\ttdefault}{lmvtt}

% Use AMS fonts for mathbb %
\DeclareMathAlphabet{\mathbb}{U}{msb}{m}{n}

% Make Titles sans-serif %
\usepackage{sectsty}
\allsectionsfont{\sffamily}

% Graphics %
\usepackage[dvipsnames]{xcolor}
\usepackage{graphicx}
\usepackage{tikz}
\usepackage{tkz-euclide}
\usetikzlibrary{%
 arrows,
 arrows.meta,
 decorations.pathreplacing,
 positioning,
 intersections,
 perspective,
 patterns.meta,
 calc,
 hobby,
 shapes,
 3d
}
\tikzset{
  vertex/.style = {shape=circle,fill,text=white,minimum size=6pt,inner sep=1pt}
}
\usepackage{pgfplots}
\pgfplotsset{compat=1.18}
\usepgfplotslibrary{fillbetween}
\usepackage{circuitikz}

% Tables %
\usepackage{booktabs}
\usepackage{multirow}

% Hyperlinks %
\usepackage{hyperref}
\hypersetup{
 colorlinks=true,
 linkcolor=RoyalBlue,
 filecolor=Magenta,
 urlcolor=Aquamarine
}

% Figures %
\usepackage{caption}
\usepackage{subcaption}
\usepackage{float} 

% Itemize and Enumerate %
\usepackage[inline]{enumitem}
\setlist{topsep=0pt}
\setlist[enumerate,1]{label=(\arabic*)}
\setlist[enumerate,2]{label=(\alph*)}

% Code insertion %
\usepackage{verbatim}
\usepackage{fancyvrb}
\usepackage{pmboxdraw}
\setmonofont[Scale=0.9]{Cascadia Code}
\fvset{baselinestretch=0.8,vspace=0pt,commandchars=\\\{\}}

% Algorithms %
\usepackage[ruled]{algorithm2e}

% Math Operators %
\DeclareMathOperator{\dom}{dom}
\DeclareMathOperator{\img}{im}
\DeclareMathOperator{\id}{\mathds{1}}
\DeclareMathOperator{\codom}{codom}
\DeclareMathOperator{\sign}{sgn}

% Number Sets %
\newcommand{\N}{\mathbb{N}}
\newcommand{\Z}{\mathbb{Z}}
\newcommand{\Q}{\mathbb{Q}}
\newcommand{\R}{\mathbb{R}}
\newcommand{\C}{\mathbb{C}}
 
% Colors %
\newcommand{\clr}{\textcolor{BrickRed}}
\newcommand{\clb}{\textcolor{RoyalBlue}}
\newcommand{\clg}{\textcolor{ForestGreen}}
\newcommand{\clm}{\textcolor{Fuchsia}}
\newcommand{\clv}{\textcolor{violet}}
\newcommand{\clbr}{\textcolor{Sepia}}
\newcommand{\cly}{\textcolor{Dandelion}}

% American Paragraph Skip %
\setlength{\parindent}{0pt}
\setlength{\parskip}{1em}

% Theorem definitions %
\usepackage[theorems,breakable,most]{tcolorbox}

%% General Theorem Style %%
\tcbset{%
 thmbox/.style = {
  enhanced,
  breakable,
  frame hidden,
  parbox=false,
  colbacktitle=black!5,
  coltitle=#1,
  fonttitle=\bfseries\sffamily,
  colback=white,
  borderline west={0.5mm}{0mm}{#1},
  toprule=0pt,
  bottomrule=0pt,
  rightrule=0pt,
  leftrule=0pt,
  titlerule=0.2mm,
  titlerule style={shorten >=.5\textwidth,#1},
  left=1ex,
  sharp corners,
  toptitle=1mm,
  bottomtitle=1mm,
  bottom=1mm,
  description delimiters parenthesis,
  separator sign none,
  description font=\normalfont\sffamily,
  fontupper=\itshape,
  topsep at break=0pt
 }
}
\tcbset{%
 defbox/.style = {
  thmbox={#1},
  fontupper=\normalfont,
  topsep at break=0pt
 }
}
\tcbset{%
 prfbox/.style = {
  parbox=false,
  top=0pt,
  bottom=0mm,
  left=1ex,
  boxrule=0pt,
  leftrule=0pt,
  toprule=0pt,
  titlerule=0pt,
  rightrule=0pt,
  bottomrule=0pt,
  borderline west={0.5mm}{0mm}{#1},
  frame hidden,
  enhanced,
  colback=white,
  sharp corners,
  breakable,
  topsep at break=0pt
 }
}

% Theorems %
\newtcbtheorem[number within=section]{theorem}{Theorem}{thmbox=Orange}{thm}
\newtcbtheorem[use counter
from=theorem]{proposition}{Proposition}{thmbox=RoyalBlue}{prop}
\newtcbtheorem[use counter
from=theorem]{lemma}{Lemma}{thmbox=ForestGreen}{lem}
\newtcbtheorem[use counter
from=theorem]{observation}{Observation}{thmbox=YellowGreen}{obs}
\newtcbtheorem[use counter
from=theorem]{corollary}{Corollary}{thmbox=RoyalPurple}{cor}
\newtcbtheorem[use counter
from=theorem]{problem}{Problem}{thmbox=Plum}{prob}
\newtcbtheorem[use counter
from=theorem]{definition}{Definition}{defbox=Aquamarine}{def}
\newtcbtheorem[use counter
from=theorem]{example}{Example}{defbox=VioletRed}{exam}
\newtcbtheorem[use counter
from=theorem]{remark}{Remark}{defbox=Mahogany}{rmrk}
\newtcbtheorem[use counter
from=theorem]{warning}{Warning}{defbox=Red}{warn}
\newtcbtheorem[use counter
from=theorem]{exercise}{Exercise}{defbox=darkgray}{exer}

% Proofs and solutions, each for different color %
\theoremstyle{nonumberplain}
\theoremseparator{.\smallskip}
\theorembodyfont{\upshape}
\theoremheaderfont{\scshape}

% Theorem Proof %
\theoremsymbol{\ensuremath{\color{Orange}\blacksquare}}
\newtheorem{thmproof}{Proof}
\tcolorboxenvironment{thmproof}{prfbox=Orange}

% Proposition Proof %
\theoremsymbol{\ensuremath{\color{RoyalBlue}\blacksquare}}
\newtheorem{propproof}{Proof}
\tcolorboxenvironment{propproof}{prfbox=RoyalBlue}

% Lemma Proof %
\theoremsymbol{\ensuremath{\color{ForestGreen}\blacksquare}}
\newtheorem{lemproof}{Proof}
\tcolorboxenvironment{lemproof}{prfbox=ForestGreen}

% Observation Proof %
\theoremsymbol{\ensuremath{\color{YellowGreen}\blacksquare}}
\newtheorem{obsproof}{Proof}
\tcolorboxenvironment{obsproof}{prfbox=YellowGreen}

% Corollary Proof %
\theoremsymbol{\ensuremath{\color{RoyalPurple}\blacksquare}}
\newtheorem{corproof}{Proof}
\tcolorboxenvironment{corproof}{prfbox=RoyalPurple}

% Problem Solution %
\theoremsymbol{\ensuremath{\color{Plum}\spadesuit}}
\newtheorem{probsol}{Solution}
\tcolorboxenvironment{probsol}{prfbox=Plum}

% Extra custom commands %
\newcommand{\myref}[2]{\hyperref[#2]{#1~\ref*{#2}}}
