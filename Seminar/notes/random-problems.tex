\documentclass[a4paper,11pt]{article}

\usepackage[czech]{babel}

% Colors %
\usepackage[dvipsnames]{xcolor}

% Page Layout %
\usepackage[margin=1.5in]{geometry}

% Fancy Headers %
\usepackage{fancyhdr}
\fancyhf{}
\cfoot{\thepage}
\rhead{}
\renewcommand{\headrulewidth}{0pt}
\setlength{\headheight}{16pt}

% Math
\usepackage{mathtools}
\usepackage{amssymb}
\usepackage{faktor}
\usepackage{import}
\usepackage{caption}
\usepackage{subcaption}
\usepackage{wrapfig}

% Theorems
\usepackage{amsthm}
\usepackage{thmtools}

% Title %
\title{\Huge\textsf{}\\
 \Large\textsf{}
 \author{}
 \date{}
}

% Table of Contents %
\usepackage{hyperref}
\hypersetup{
 colorlinks=true,
 linktoc=all,
 linkcolor=blue
}

% Tables %
\usepackage{booktabs}

% Enumerate %
\usepackage{enumitem}

% Operators %
\DeclareMathOperator{\Ker}{Ker}
\DeclareMathOperator{\Img}{Im}
\DeclareMathOperator{\End}{End}
\DeclareMathOperator{\Aut}{Aut}
\DeclareMathOperator{\Inn}{Inn}

% Common operators %
\newcommand{\R}{\mathbb{R}}
\newcommand{\N}{\mathbb{N}}
\newcommand{\Z}{\mathbb{Z}}
\newcommand{\Q}{\mathbb{Q}}
\newcommand{\C}{\mathbb{C}}

% American Paragraph Skip %
\setlength{\parindent}{0pt}
\setlength{\parskip}{1em}

% Document %
\pagestyle{fancy}
\begin{document}

Jen pár náhodných úloh ze semináře. Některé s řešením.

\section*{Něco k relacím}

Připomínám, že relaci $R$ na množině $A$ nazveme \emph{transitivní}, když $xRy
\wedge yRz \Rightarrow xRz$ pro všechna $x,y,z \in A$.

Dál připomínám, že pokud jsou $R,S$ relace na $A$, pak jejich \emph{složení}
$R \circ S$ je relace definovaná jako
\[
 R \circ S \coloneqq \{(x,z) \mid  \exists y \in A: xRy \wedge ySz\},
\]
tedy relace, která je vlastně \uv{slepením} $R$ a $S$ přes nějaký prostřední
prvek $y$.

Ať je $R$ teďka \textbf{libovolná} relace na $A$ (tedy ne nutně transitivní).
Relaci, kterou dostanu, když složím $R$ samu se sebou $k$-krát, čili
$\underbrace{R \circ R \circ R \circ \cdots \circ R}_{\text{$k$-krát}}$, označím
zkráceně jako $R^{k}$. Definuju si relaci na $A$ předpisem
\[
 T \coloneqq \bigcup_{i=1}^{\infty} R^{i} = (R) \cup (R \circ R) \cup (R \circ R
 \circ R) \cup \ldots
\]
Lidsky řečeno, $T$ je relace, která obsahuje $R$ složenou samu se sebou $k$-krát
pro každé přirozené číslo $k \in \N$.
\begin{enumerate}
 \item Dokažte, že $T$ je transitivní.
 \item Dokažte, že každá transitivní relace $S$ na $A$ obsahuje (myšleno \uv{je
  nadmnožinou}) $R$ právě tehdy, když obsahuje $T$.
 \item Dokažte, že pokud $\# X = n$, pak
  \[
   T = \bigcup_{i=1}^{n-1} R^{i},
  \]
  tedy že všechny složené relace $R^{k}$ pro $k \geq n$ už jsou obsaženy v
  nějaké složené relaci $R^{l}$ pro $l \leq n-1$.
\end{enumerate}

\textbf{Částečné řešení:}
\begin{enumerate}
 \item Tady vidím dva možné důkazy. Jeden přímo a jeden užitím cvičení, které
  jsme dokázali na kroužku, že $T$ je transitivní právě tehdy, když $T \circ
  T \subseteq T$. Napíšu tu ten přímý a napovím, jak na ten druhý.

  Z definice transitivity předpokládáme, že máme prvky $x,y,z \in A$, které
  splňují $xTy \wedge yTz$. Chceme odtud odvodit, že $xTz$. No, $T$ je přece
  sjednocením všech složení $R$ se sebou samou, takže tu dvojici $(x,y) \in T$
  musím najít v nějakém složení $R^{k}$, kde $k \in \N$ a taky dvojici $(y,z)$
  musím najít v~nějakém složení $R^{l}$, kde $k,l \in \N$. Pak ale najdu dvojici
  $(x,z)$ ve složení $R^{k} \circ R^{l} = R^{k+l}$ (dosaďte si do definice
  složení relací za $R$ to $R^{k}$ a za $S$ to $R^{l}$, uvidíte to hned), které
  je z definice podmnožinou $T$. Čili $(x,z) \in T$, jak jsem chtěl.

  V tom nepřímém důkazu chce člověk ukázat, že $T \circ T \subseteq T$, pak bude
  vědět, že $T$ je transitivní. Když si to rozepíšeme, dostaneme
  \[
   T \circ T = \left( \bigcup_{i=1}^{\infty} R^{i} \right) \circ
   \left(\bigcup_{i=1}^{\infty} R^{i} \right). 
  \]
  Na to, aby mohl jeden tvrdit, že $T \circ T \subseteq T$, stačí ověřit, že
  \[
   \left( \bigcup_{i=1}^{\infty} R^{i} \right) \circ \left(
   \bigcup_{i=1}^{\infty} R^{i} \right) = \bigcup_{i=1}^{\infty} (R^{i} \circ
   R^{i}), 
  \]
  protože $R^{i} \circ R^{i} = R^{2i}$ leží pro každé $i  \in \N$ uvnitř $T$,
  takže tam leží i celé to sjednocení nahoře. Ověřit tu rovnost si můžete zkusit
  sami.

 \item Všimněte si, že to tvrzení je (logická) ekvivalence. Chci dokázat
  konkrétně, že
  \[
   R \subseteq S \Leftrightarrow T \subseteq S
  \]
  pro danou transitivní relaci $S$. Když dokazuju ekvivalenci, obvykle dokazuju
  dvě implikace, tu zleva doprava a tu zprava doleva. Ta zprava doleva je v
  tomhle případě samozřejmá. Relace $R$ je podmnožinou $T$, takže když $T
  \subseteq S$, pak $R \subseteq T \subseteq S$, čili celkem $R \subseteq S$. Ta
  zleva doprava je těžší, takže je spravedlivě na vás.
\end{enumerate}

\section*{Něco k indukci}

Na kroužku jsem ukazoval důkaz indukcí, že $F_n \leq ((1+\sqrt{5}) / 2)^{n-1}$
pro všechna $n \geq 0$, kde $F_n$ je $n$-tý člen Fibonacciho posloupnosti, která
je definovaná:
\[
 F_0 = 0, F_1 = 1 \text{ a všechny další členy vzorcem } F_n = F_{n-1} +
 F_{n-2}.
\]
Dělal jsem to trochu narychlo, tak to tu radši zopáknu.

\textbf{První indukční krok:} Když $n = 0$, pak máme
\[
 F_0 = 0 \leq \left( \frac{1+\sqrt{5}}{2} \right) ^{0-1} = \left(
 \frac{1+\sqrt{5}}{2} \right) ^{-1} = \frac{2}{1+\sqrt{5}},
\]
což je samozřejmě pravda, už jen pro to, že to číslo napravo je kladné. Takže
všechno v okeju.

\textbf{Druhý indukční krok:} Předpokládám, že platí
\[
 F_k \leq \left( \frac{1+\sqrt{5}}{2} \right)^{k-1} 
\]
pro nějaké $k \in \N$ a všechna menší přirozená čísla. Odtud chci odvodit, že
\[
 F_{k+1} \leq \left( \frac{1+\sqrt{5}}{2} \right) ^{k}.
\]
Ještě si označím $\varphi \coloneqq (1+\sqrt{5}) / 2$, abych se neupsal.
Uvidíte, že na číselné hodnotě $\varphi$ záleží vlastně až úplně na konci
důkazu. Vím, že $F_{k+1} = F_k + F_{k-1}$ a předpokládám, že
\[
 F_k \leq \varphi^{k-1} \quad \text{a} \quad F_{k-1} \leq \varphi^{k-2}.
\]
Když ta čísla sečtu dohromady, dostanu, že
\[
 F_{k+1} \leq F_k + F_{k-1} \leq \varphi^{k-1} + \varphi^{k-2}.
\]
Z toho výrazu napravo můžu vytknout $\varphi^{k-2}$, protože
\[
 \varphi^{k-1} = \varphi^{k-2} \cdot \varphi.
\]
Pak mi zůstane
\[
 \varphi^{k-1} + \varphi^{k-2} = \varphi^{k-2}(\varphi + 1).
\]
Dokazuju, že $F_{k+1} \leq \varphi^k$ a vím (z předpokladu), že $F_k \leq
\varphi^{k-2}(\varphi + 1)$. Jinak řečeno, potřebuju ověřit nerovnost
\[
 \varphi^{k-2}(\varphi + 1) \leq \varphi^{k}.
\]
Ještě předtím, než dosadím zpátky za $\varphi$ to číslo, uvědomím si, že můžu
zkrátit tu nerovnici $\varphi^{k-2}$, jelikož $\varphi^{k} =
\varphi^{k-2}\varphi^{2}$. Teď už to prostě dopočítám.
\begin{align*}
 \varphi^{k-2}(\varphi + 1) & \leq \varphi^{k}\\
 \varphi^{k-2}(\varphi + 1) & \leq \varphi^{k-2}\varphi^{2}\\
 \varphi + 1 & \leq \varphi^2\\
 \frac{1+\sqrt{5}}{2} + 1 & \leq \left( \frac{1+\sqrt{5}}{2} \right)^2\\
 \frac{3+\sqrt{5}}{2} & \leq \frac{1 + 2\sqrt{5} + 5}{4}\\
 \frac{3+\sqrt{5}}{2} & \leq \frac{6+2\sqrt{5}}{4}\\
 \frac{3+\sqrt{5}}{2} & \leq \frac{3+\sqrt{5}}{2},
\end{align*}
čímž mám důkaz hotov.

Všimněte si, jen pro zajímavost, že jsem na konci dokázal, že $\varphi + 1 =
\varphi^2$, čili číslo $\varphi$ je řešením rovnice $x^2 - x - 1 = 0$.

\end{document}
