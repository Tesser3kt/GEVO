\documentclass[a4paper,11pt]{article}

% Babel %
\usepackage[czech]{babel}
\usepackage[utf8]{inputenc} % vstupní kódování je UTF-8
\usepackage[T1]{fontenc} % výstupní kódování

% Colors %
\usepackage[dvipsnames]{xcolor}

% Page Layout %
\usepackage[margin=1.5in]{geometry}

% Fancy Headers %
\usepackage{fancyhdr}
\fancyhf{}
\cfoot{\thepage}
\rhead{}
\renewcommand{\headrulewidth}{0pt}
\setlength{\headheight}{16pt}

% Math
\usepackage{mathtools}
\usepackage{amssymb}
\usepackage{import}
\usepackage{caption}
\usepackage{subcaption}
\usepackage{wrapfig}
\usepackage{graphicx}
\usepackage{tikz}
\newcommand{\tkzcircsimple}{\tikz{\node[circle,inner sep=0,outer sep=0,draw]{$\phantom{o}$}}}

% Theorems
\usepackage{amsthm}
\usepackage{thmtools}

\newtheorem{thm}{Věta}[subsection]

\theoremstyle{definition}
\newtheorem{exm}[thm]{Příklad}
\newtheorem{dfn}[thm]{Definice}

\theoremstyle{plain}
\newtheorem*{exr}{Cvičení}

% Title %
\title{\Huge\textsf{Logika, množiny, důkazy}\\
 \Large\textsf{}
 \author{}
 \date{}
}

% Table of Contents %
\usepackage{hyperref}

% Tables %
\usepackage{booktabs}

% Enumerate %
\usepackage{enumitem}
\setlist[itemize]{topsep=0pt}
\setlist[enumerate]{topsep=0pt}
\setlist[enumerate,1]{label=(\arabic*)}

% Operators %
\DeclareMathOperator{\Ker}{Ker}
\DeclareMathOperator{\Img}{Im}
\DeclareMathOperator{\End}{End}
\DeclareMathOperator{\Aut}{Aut}
\DeclareMathOperator{\Inn}{Inn}

% Common operators %
\newcommand{\R}{\mathbb{R}}
\newcommand{\N}{\mathbb{N}}
\newcommand{\Z}{\mathbb{Z}}
\newcommand{\Q}{\mathbb{Q}}
\newcommand{\C}{\mathbb{C}}

% American Paragraph Skip %
\setlength{\parindent}{0pt}
\setlength{\parskip}{1em}

% Document %
\pagestyle{fancy}
\begin{document}

\maketitle
\thispagestyle{fancy}

\section{Logika a množiny}
\label{sec:logika-a-mnoziny}

Matematická logika poskytuje formální prostředí pro práci s \emph{výroky}. Výrok
je pro nás jakákoli věta, o které lze rozhodnout, zda \textbf{platí}, či
\textbf{neplatí}.

\begin{exm}\hfill
 \begin{itemize}
  \item \uv{Mám žízeň.} a \uv{Nikdo nemá rád matiku.} jsou výroky.
  \item \uv{Jak se máš?} a \uv{Ať už je konec!} nejsou výroky.
  \item \uv{Trump vyhraje příští volby.} je taky výrok, i když neumíme říct,
   jestli platí, nebo ne. Naše schopnost rozhodnout o pravdivosti věty nesouvisí
   s~tím, že věta je nebo není výrokem.
 \end{itemize}
\end{exm}

\subsection{Logika prvního řádu}
\label{ssec:logika-prvniho-radu}

Logikou prvního řádu se v matice myslí jazyk o dvou číslech/konstantách a pěti
(obvykle pěti) operacích na výrocích.

Konstanty $0$ a $1$ (někdy taky $\perp$ a $\top$) se obvykle interpretují jako
\uv{lež} a \uv{pravda}. V tomto kontextu můžeme říct, že výrok je věta, které
můžeme přiřadit hodnotu $0$ nebo $1$.

Výrazy se v logice obvykle reprezentují malými písmeny latinské abecedy
(pokud jste o nich neslyšeli, jsou to $a$, $b$, $c$, atd.).  

\begin{dfn}[Logické operace]
 Řekněme, že $p$ a $q$ jsou výroky \uv{Prší.} a \uv{Mám žízeň.} Pak
 \begin{itemize}
  \item $\neg p$ (\textbf{ne} $p$; \textbf{negace} $p$) je výrok s opačnou
   pravdivostní hodnotou k $p$. V~tom\-to případě
   \[
    \neg p = \text{\uv{Neprší.}}
   \]
   Operaci $\neg$ říkáme \textbf{negace}.
  \item $p \wedge q$ ($p$ \textbf{a (zároveň)} $q$) platí jenom tehdy, když
   platí $p$ i $q$. V našem případě je
   \[
    p \wedge q = \text{\uv{Prší a mám žízeň.}}
   \]
   Operaci $ \wedge $ říkáme \textbf{konjunkce}.
  \item $p \vee q$ ($p$ \textbf{nebo} $q$) platí tehdy, když platí aspoň jeden z
   výroků $p$, $q$. Tedy, \[
    p \vee q = \text{\uv{Prší nebo mám žízeň.}}
   \]
   \textcolor{red}{Pozor!} Tohle \uv{nebo} není výlučné. Výrok $p \vee q$ platí
   i v případě, že platí $p$ i $q$. Operaci $ \vee $ říkáme \textbf{disjunkce}.
  \item $p \Rightarrow q$ ($p$ \textbf{implikuje} $q$; \textbf{z} $p$ 
   \textbf{plyne} $q$; \textbf{když} $p$, \textbf{pak} $q$) platí v jedné ze
   dvou možností:
   \begin{enumerate}[label=(\arabic*)]
    \item Platí $p$ a platí $q$.
    \item Neplatí $p$.
   \end{enumerate}
   V našem případě
   \[
    p \Rightarrow q = \text{\uv{Když prší, mám žízeň.}}
   \]
   Operaci $ \Rightarrow $ říkáme \textbf{implikace}. Implikace může být trochu
   ošemetná na pochopení. Možná pomůže fakt, že výroku $p$ se někdy říká 
   \emph{předpoklad} a výroku $q$ \emph{důsledek}. Jde o to, že když platí
   předpoklad, musí platit i důsledek; když ale předpoklad neplatí, důsledek
   může i nemusí být pravdivý.

   Pro naše zvolené výroky to konkrétně znamená, že \textbf{vždy} když prší, tak
   mám žízeň, ale když neprší, tak můžu mít žízeň, ale taky nemusím. \item $p
   \Leftrightarrow q$ ($p$ \textbf{právě tehdy, když} $q$; $p$ \textbf{je
   ekvivalentní} $q$) platí přesně v situaci, kdy $p$ má stejnou pravdivostní
   hodnotu jako $q$. V našem případě
   \[
    p \Leftrightarrow q = \text{\uv{Prší právě tehdy, když mám žízeň.}}
   \]
   Operaci $ \Leftrightarrow $ se říká \textbf{ekvivalence}.
 \end{itemize}
\end{dfn}

\begin{dfn}[Logická formule]
 Logická formule je pro nás jakákoli \emph{smysluplná} posloupnost výroků,
 logických operací a závorek. Smysluplnost znamená, že za $\neg$ je výrok a na
 obou stranách $ \wedge , \vee , \Rightarrow , \Leftrightarrow $ je vždy výrok.
 Závorky určují prioritu logických operací.
\end{dfn}

Obecně se matematici nedohodli na nějaké přirozené prioritě logických operací,
ale nejčastější úzus je, že nejvyšší prioritu má $\neg$, pod ní jsou  $ \wedge ,
\vee $ a nejnižší prioritu mají $ \Rightarrow , \Leftrightarrow $.

\begin{exm}
 \label{exm:formula}
 Třeba
 \[
  ((\neg p) \vee q) \Leftrightarrow ((p \vee (\neg q)) \Rightarrow r) 
 \]
 je formule, ale
 \[
  p \Leftrightarrow  \Rightarrow q \neg r
 \]
 není.
\end{exm}

V logice člověk často potřebuje zjistit, jakou pravdivostní hodnotu má formule v
závislosti na pravdivostních hodnotách původních výroků. K tomu slouží tzv.
\emph{tabulka pravdivostních hodnot}, kam si člověk napíše všechny možné
pravdivostní hodnoty výroků a prostě počítá. Třeba pro formuli 
\[
 ((\neg p) \vee q) \Rightarrow r
\]
bychom sestrojili tabulku

\begin{center}
 \begin{tabular}{c|c|c|c|c|c|c}
  $p$ & $q$ & $r$ & $\neg p$ & $\neg q$ & $(\neg p) \vee q$ & $((\neg p) \vee q)
  \Rightarrow r$\\
  \midrule
  0 & 0 & 0 & 1 & 1 & 1 & 0\\
  0 & 0 & 1 & 1 & 1 & 1 & 1\\
  0 & 1 & 0 & 1 & 0 & 1 & 0\\
  1 & 0 & 0 & 0 & 1 & 0 & 1\\
  0 & 1 & 1 & 1 & 0 & 1 & 1\\
  1 & 0 & 1 & 0 & 1 & 0 & 1\\
  1 & 1 & 0 & 0 & 0 & 1 & 0\\
  1 & 1 & 1 & 0 & 0 & 1 & 1
 \end{tabular}
\end{center}

Pro představu, kdyby $p$ bylo \uv{Prší.}, $q$ bylo \uv{Mám žízeň.} a $r$ bylo
\uv{Stíhám autobus.}, pak by výrok $((\neg p) \vee q) \Rightarrow r$ znamenal
\begin{center}
 \emph{Když neprší nebo mám žízeň, tak stíhám autobus}.
\end{center}
Hodně štěstí bez pomoci rozluštit, kdy je tahle věta pravdivá.

\subsection{Množiny}
\label{ssec:mnoziny}

Množiny jsou základní stavební prvky moderní matematiky. Oproti tomu, co se vám
někdo možná pokusí tvrdit, matematika neodpovídá na otázku \uv{Co je množina?}.
Matika jenom používá množiny k modelování struktur, které obvykle mají šanci
někoho zajímat. Ta otázka je veskrze filosofická.

Asi každý si představuje množiny jako soubory prvků, které mají něco společného.
Téhle interpretace se budeme taky držet. Fakt, že prvek $x$ patří do množiny $A$
budeme zapisovat $x \in A$, kde symbol $ \in $ je pokroucené $e$ z anglického
\textbf{e}lement. Když chci tvrdit opak, tedy, že prvek $x$ \emph{neleží} v
množině $A$, píšu $x \notin A$.

S množinami se táhne v závěsu pár vztahů a operací, které si teď zadefinujeme
pomocí logických operací.

\begin{dfn}[množinové vztahy]
 Vztahy mezi množinami budeme velmi pravděpodobně potřebovat jen dva, a to
 \emph{inkluzi} a \emph{rovnost}. Mějme dvě množiny $A,B$.
 \begin{itemize}
  \item $A \subseteq B$ ($A$ \textbf{je podmnožina/součást} $B$; $A$ 
   \textbf{leží v/patří do} $B$) značí skutečnost, že každý prvek $A$ je taky
   prvkem $B$. V logice
   \[
    A \subseteq B \Leftrightarrow (x \in A \Rightarrow x \in B).
   \]
   Vztahu $ \subseteq $ se říká \textbf{inkluze}. Samozřejmě můžu psát i $A
   \supseteq B$ pro inkluzi opačným směrem.
  \item $A = B$ ($A$ \textbf{je rovno} $B$) znamená, že $A$ a $B$ sdílejí
   všechny prvky. Logicky buď třeba
   \[
    A = B \Leftrightarrow (A \subseteq B \wedge B \subseteq A)
   \]
   nebo
   \[
    A = B \Leftrightarrow (x \in A \Leftrightarrow x \in B).
   \]
 \end{itemize}
 Když píšu $A \subseteq B$ může se stát i to, že $A = B$. Pokud chci zdůraznit
 fakt, že $A \subseteq B$, ale $A \neq B$, napíšu $A \subsetneq B$.
\end{dfn}

\begin{dfn}[množinové operace]
 Budeme rozeznávat čtyři množinové operace. Podobně jako výsledkem logických
 operací na výrocích je zase výrok, výsledkem množinové operace na množinách je
 množina (to je ale překvápko).
 \begin{itemize}
  \item $A \cup B$ ($A$ \textbf{sjednoceno s} $B$) je množina, která obsahuje
   prvky ležící v $A$ \emph{nebo} v $B$. Logicky
   \[
    x \in A \cup B \Leftrightarrow (x \in A \vee x \in B).
   \]
   \begin{center}
    \begin{figure}[h]
     \centering
     \begin{subfigure}{0.45\textwidth}
      \centering
      \def\svgwidth{0.6\textwidth}
      \import{figs/}{union-before.pdf_tex}
      \caption*{Množiny \textcolor{green}{$A$} a \textcolor{red}{$B$}.}
     \end{subfigure}
     \begin{subfigure}{0.45\textwidth}
      \centering
      \def\svgwidth{0.6\textwidth}
      \import{figs/}{union-after.pdf_tex}
      \caption*{Sjednocení \textcolor{blue}{$A \cup B$}.}
     \end{subfigure}
    \end{figure}
   \end{center}
   \vspace*{-8mm}
   Operaci $ \cup $ říkáme \textbf{sjednocení}. Sjednocení z typografického
   hlediska příliš velkého množství množin $A_1,\ldots,A_n$, kde $n \in \N$,
   píšeme
   \[
    \bigcup_{i=1}^{n} A_i.
   \]
  \item $A \cap B$ ($A$ \textbf{proniknuto s} $B$) je množina, která obsahuje
   jen ty prvky, které leží v obou množinách $A$, $B$.
   \newpage
   \begin{center}
    \begin{figure}[h]
     \centering
     \begin{subfigure}{0.45\textwidth}
      \centering
      \def\svgwidth{0.6\textwidth}
      \import{figs/}{inter-before.pdf_tex}
      \caption*{Množiny \textcolor{green}{$A$} a \textcolor{red}{$B$}.}
     \end{subfigure}
     \begin{subfigure}{0.45\textwidth}
      \centering
      \def\svgwidth{0.6\textwidth}
      \import{figs/}{inter-after.pdf_tex}
      \caption*{Průnik \textcolor{blue}{$A \cap B$}.}
     \end{subfigure}
    \end{figure}
   \end{center}
   \vspace*{-8mm}
   Operaci $ \cap $ říkáme \textbf{průnik}. Průnik spousty množin
   $A_1,\ldots,A_n$ píšeme
   \[
    \bigcap_{i=1}^{n} A_i.
   \]
  \item $A \setminus B$ ($A$ \textbf{bez}/\textbf{méně} $B$) obsahuje právě ty
   prvky, které jsou v $A$, ale nejsou v $B$. Logikou například takto:
   \[
    x \in A \setminus B \Leftrightarrow (x \in A \wedge x \notin B).
   \]
   \begin{center}
    \begin{figure}[h]
     \centering
     \begin{subfigure}{0.45\textwidth}
      \centering
      \def\svgwidth{0.6\textwidth}
      \import{figs/}{diff-before.pdf_tex}
      \caption*{Množiny \textcolor{green}{$A$} a \textcolor{red}{$B$}.}
     \end{subfigure}
     \begin{subfigure}{0.45\textwidth}
      \centering
      \def\svgwidth{0.6\textwidth}
      \import{figs/}{diff-after.pdf_tex}
      \caption*{Rozdíl \textcolor{blue}{$A \setminus B$}.}
     \end{subfigure}
    \end{figure}
   \end{center}
   \vspace*{-8mm}
   Operaci $ \setminus $ říkáme rozdíl. \textcolor{red}{Pozor!} Operace $
   \setminus $ \textbf{není komutativní}, tzn., že $A \setminus B \neq B
   \setminus A$. Z toho důvodu nemá rozdíl více než dvou množin smysl, protože
   není jasné, v jakém pořadí třeba zápis
   \[
    A \setminus B \setminus C
   \]
   chápat.
  \item $A \times B$ ($A$ \textbf{krát} $B$) je množina všech dvojic, kde první
   prvek je z $A$ a druhý z $B$. Logicky
   \[
    (x,y) \in A \times B \Leftrightarrow (x \in A \wedge y \in B).
   \]
   Operaci $ \times $ říkáme \textbf{součin} nebo \textbf{produkt}. Součin mnoha
   množin $A_1,\ldots,A_n$ píšeme
   \[
    \prod_{i=1}^{n} A_i,
   \]
   kde symbol $\Pi$ je velké řecké \uv{pí} pro \textbf{p}rodukt. V případě, že
   násobíme množinu samu se sebou, můžeme taky psát
   \[
    A^{n} \coloneqq \prod_{i=1}^{n} A. 
   \]
 \end{itemize}
\end{dfn}

\begin{exm}
 Uvážíme množiny
 \[
  A = \{\heartsuit, \clubsuit, \diamondsuit \}, \quad B = \{ \clubsuit,
  \spadesuit \}, \quad C = \{\heartsuit, \diamondsuit, \spadesuit, \clubsuit\}.
 \]
 Pak
 \begin{align*}
  A \cup B &= \{\heartsuit, \clubsuit, \diamondsuit, \spadesuit\} = C, \\
  A \cap B &= \{\clubsuit\}, \\
  C \setminus A &= \{\spadesuit\}, \\
  A \times B &= \{(\heartsuit, \clubsuit), (\heartsuit, \spadesuit), (\clubsuit,
  \clubsuit), (\clubsuit, \spadesuit), (\diamondsuit, \clubsuit),
  (\diamondsuit, \spadesuit)\}. \\
 \end{align*}
 \textcolor{red}{Pozor!} Sjednocení $ \cup $ a rozdíl $ \setminus $ se nechovají
 jako $+$ a $-$. Všimněte si, že i když $A \cup B = C$, tak $C \setminus A \neq
 B$.
\end{exm}

\subsection{Kvantifikátory}
\label{ssec:kvantifikatory}

Poslední základní ingrediencí matematického jazyka jsou kvantifikátory, které
nám umožňují charakterizovat prvky množin pomocí logických výrazů.

\begin{dfn}[kvantifikátory]
 Rozlišujeme dva druhy kvantifikátorů -- univerzální a existenční. Řekněme, že
 $p(x)$ značí výrok $p$, který v sobě obsahuje proměnnou $x$. Takový výrok může
 vypadat třeba jako $x \geq 2$ nebo $x + 1 = 5$. Pokud $A$ je nějaká množina,
 pak výrazem
 \begin{itemize}
  \item $ \forall x \in A: p(x)$ (\textbf{pro každé} x \textbf{z} A
   \textbf{platí} $p(x)$) myslíme, že pro každý prvek $x \in A$ platí výrok
   $p(x)$. Jinak řečeno, ať už si vyberu libovolný prvek množiny $A$ a dosadím
   ho za $x$ do $p(x)$, tak $p(x)$ bude pravdivý. Symbol $ \forall $ je obrácené
   \uv{A} z německého \emph{\textbf{A}llgemein} (pro všechny).
  \item $ \exists x \in A: p(x)$ (\textbf{existuje} $x$ \textbf{z} $A$
   \textbf{takové, že platí} $p(x)$) myslíme, že v $A$ se nachází prvek, po
   jehož dosazení za $x$ do $p(x)$ je výrok $p(x)$ pravdivý. Symbol $ \exists $
   je obrácené \uv{E} z anglického \emph{\textbf{E}xists}.
 \end{itemize}
\end{dfn}

Kromě těchto kvantifikátorů budeme ještě používat symbol $ \exists !$, který
znamená \uv{existuje právě jeden}. Logicky, napíšu-li $ \exists !x \in A:p(x)$,
myslím tím
\[
 (\exists x \in A:p(x)) \wedge ( \forall y \in A:p(y) \Rightarrow y = x).
\]
Tedy, existuje prvek $x \in A$ splňující $p(x)$ a kdykoli $y \in A$ je prvek
splňující $p(y)$, pak $y = x$. Konečně, symbolem $\nexists$ myslím
\uv{neexistuje}.

\begin{exm}\hfill
 \begin{enumerate}
  \item Řekněme, že $p(x)$ je výrok $x \geq 2$. Pak výroky $ \forall x \in
   \R:p(x)$ a $ \exists !x \in \R:p(x)$ jsou lživé a $ \exists x \in \R:p(x)$ je
   pravdivý.
   \begin{itemize}
    \item $ \forall x \in \R:p(x)$ je lživý, protože každé reálné číslo určitě
     není větší než $2$. Například $-420$ není.
    \item $ \exists !x \in \R:p(x)$ je taky lživý, protože existuje víc reálných
     čísel větších než $2$. $69$ a $911$ jsou dvě z nich.
    \item $ \exists x \in \R:p(x)$ je pravdivý, protože umím najít reálné číslo
     větší než dva. Například $1337$.
   \end{itemize}
  \item Písmenem $P$ označíme množinu všech lidí. Pro $x,y \in P$ znamená výrok
   $p(x,y)$, že \uv{Člověk $x$ zná jméno člověka $y$.} Pak výrok
   \begin{itemize}
    \item $ \forall x \; \forall y: p(x,y)$ znamená, že všichni lidé znají
     navzájem svá jména.
    \item $ \forall x \; \exists y: p(x,y)$ znamená, že každý člověk zná jméno
     aspoň jednoho člověka.
    \item $ \forall x \; \exists !y:p(x,y)$ znamená, že každý člověk zná jméno
     přesně jednoho člověka.
    \item $  \exists x \; \forall y :p(x,y)$ znamená, že je na světě člověk,
     který zná jména všech lidí.
    \item $ \exists x \; \exists y: p(x,y)$ znamená, že existuje člověk, který
     zná jméno nějakého člověka.
   \end{itemize}
 \end{enumerate}
\end{exm}

\subsection{Relace a zobrazení}
\label{ssec:relace-a-zobrazeni}

Operace součinu na množinách je zvlášť zajímavá pro to, že nám umožňuje
definovat vztahy mezi prvky množin. Hádám, že vás nikdy neučili dívat se třeba
na $ \leq $ nebo na funkci $f(x) = 2x + 3$ jako na množiny...

\begin{dfn}[relace/vztah]
 Mějme množiny $A,B$. Jakékoli podmnožině $R \subseteq A \times B$ budeme říkat
 \textbf{relace/vztah mezi} $A$ \textbf{a} $B$. O prvku $x \in A$ řekneme, že je
 v relaci/vztahu $R$ s prvkem $y \in B$, pokud $(x,y) \in R$. Tohle často
 zapisujeme jako $xRy$.

 Pokud $A = B$, pak $R \subseteq A \times A$ je \textbf{relace na} $A$.
\end{dfn}

\begin{exm}
 Vztah \uv{menší nebo rovno} je relace na reálných číslech. Smysl zápisu $xRy$
 je opodstatněn tím, že důležité relace v matice mají často vlastní symboly.
 Když totiž $R = \; \leq \; \subseteq \R \times \R$, pak nepíšeme třeba $
 \leq\!\!(2,3)$, abychom vyjádřili, že $2$ je menší nebo rovno $3$, ale spíš $2
 \leq 3$.
\end{exm}

Jinak řečeno, relace je doslova výpis všech dvojic prvků, které jsou v tom daném
vztahu. Takto lze vnímat právě relaci $ \leq $ (třeba na $\R$) jako množinu
všech dvojic ${(x,y) \in \R^2}$ pro $x \leq y$.

\begin{exm}
 Zkusím jednu analogii ze života. To nedopadne... Když bude $P$ zase označovat
 množinu všech lidí, tak \uv{svazek manželský} je relace na $P$. Relaci
 intuitivně interpretujeme jako nějaké pouto mezi objekty (v tomto případě
 dvěma), ale v matematice je pouto mezi prvky zase množina.

 Jde o to si uvědomit, že v matice tohle funguje obráceně než v životě. Neříkám,
 že dva lidé jsou či nejsou manželé podle definice svazku manželského. Říkám, že
 definice svazku manželského je právě množina všech manželských párů. V moment,
 co se někdo rozvede nebo ožení/vdá, změní se moje definice manželského svazku.

 Tohle je jedna z těch divných věcí v matice, které sice ze začátku nedávají
 úplně intuitivní smysl, ale příliš snadno se s nimi pracuje, než aby se je
 někdo snažil předělávat.
\end{exm}

\begin{dfn}[zobrazení]
 Pokud $A,B$ jsou množiny, pak relaci $R$ mezi $A$ a $B$ nazveme
 \textbf{zobrazením} (\textbf{mezi} $A$ \textbf{a} $B$), pokud
 \[
  \forall x \in A \; \forall y,z \in B: (xRy \wedge xRz) \Rightarrow (y = z).
 \]
 V případě, že $A = \R^{n}$ pro nějaké přirozené $n$ a $B = \R$, můžeme relaci
 mezi $A$ a $B$ splňující výrok výše říkat \textbf{funkce}. To slovo je ale
 hodně naprd, takže budu spíš vždycky říkat \textbf{zobrazení}. Fakt, že $f$ je
 zobrazení mezi $A$ a $B$ budeme zapisovat buď jako
 \[
  f:A \to B \quad \text{nebo} \quad A \overset{f}{\longrightarrow} B.
 \]
\end{dfn}

\begin{exr}
 Rozluštěte předchozí definici.
\end{exr}

\begin{exm}
 Zobrazení $\R  \overset{f}{\longrightarrow} \R$, $f(x) = 2x + 3$ je množina
 všech dvojic $(x,2x+3) \in \R^2$. Když je zobrazení definováno na prvcích první
 množiny (jako v~tomto případě), často budu psát $x \mapsto 2x + 3$ místo $f(x)
 = 2x + 3$.
\end{exm}

\begin{dfn}[doména, kodoména, vzor, obraz]
 Mějme zobrazení $A \overset{f}{\longrightarrow} B$. Pak se
 \begin{itemize}
  \item množina $A$ nazývá \textbf{doménou} zobrazení $f$.
  \item množina $B$ nazývá \textbf{kodoménou} zobrazení $f$.
  \item pro každé $x \in A$, nazývá $f(x)$ \textbf{obrazem} prvku $x$ při
   zobrazení $f$.
  \item pro každé $y \in B$ nazývá prvek $x \in A$ takový, že $f(x) = y$ 
   \textbf{vzorem} prvku $y$ při zobrazení $f$. Značíme $y =
   f^{-1}(x)$.

   \textcolor{red}{Pozor! Vzor prvku nemusí vždy existovat.} Například při $\R
   \overset{f}{ \to } \R, x \mapsto 1 / x$ nemá $0$ žádný vzor.
 \end{itemize}
\end{dfn}

Tahle sekce ještě není hotová.

\subsection{Metody důkazů}
\label{ssec:metody-dukazu}

Zatím nic. rip

\end{document}
