\subsection{Kombinační čísla}
\label{ssec:kombinacni-cisla}

V této podsekci se budeme zabývat asi poměrně přirozenou otázkou -- kolik má
množina $X$ podmnožin velikosti $k$, kde $k$ může být libovolné číslo od $0$ do
$\# X$. Pro $k=0$ i $k=\# X$ je odpověď jednoduchá: přesně jednu. Pro $k = 1$
člověku hádám taky dojde, že jednoprvková podmnožina je vlastně totéž, co její
jediný prvek, takže takových máme $\# X$. Od $k = 2$ nám ale začína, borcovia,
prituhovať. Nejspíš bychom pořád zvládli počet dvouprvkových množin nějak
zpatlat, ale co třeba $k = \# X / 2$ (když je $\# X$ sudé) a podobné takřka
nekřesťanské výmysly? To už chce nějaké udělátko.

Nejdřív si to ale, jakožto slušní a spořádaní matematikové, definujeme.

\begin{definition}[Počet $k$-prvkových podmnožin]
 Ať $X$ je množina a $0 \leq k \leq \# X$ je přirozené číslo. Definujeme množinu
 \[
  \binom{X}{k} \coloneqq \{A \subseteq X \mid \# A = k\}
 \]
 všech $k$-prvkových podmnožin množiny $X$. Výraz $\binom{X}{k}$ čteme \uv{$X$ 
 nad $k$}.
\end{definition}

Chvilku se budeme bavit přemítáním o způsobu, jak spočítat $\# \binom{X}{k}$ pro
libovolné $k$ mezi $0$ a $\# X$.

Použijeme kombinatorickou metodu důkazu zvanou \emph{počítání dvěma způsoby}.
Jde o užitečný (a podle mého velmi elegantní) přístup ve chvíli, kdy neumím
spočítat rovnou konkrétní množství, ale umím spočítat něco vel\-mi podobného.
Technika počítání dvěma způsoby spočívá v tom, že tu kvantitu, kterou spočítat
\emph{umím}, vyjádřím na jedné straně pomocí kvantity, kterou spočítat
\emph{neumím}, a na druhé straně pomocí vzorečku, který znám. To mi dá rovnici,
ze které pak vyjádřím to číslo, které chci určit.

Takto abstraktně vám asi \emph{počítání dvěma způsoby} nic neřeklo, takže je
raději pojďme aplikovat na zpytovaný problém. Počet $k$-prvkových podmnožin $X$
spočítat neumím; což takhle začít tím, že si nějakou náhodnou podmnožinu
$\{x_1,\ldots,x_k\} \subseteq X$ zvolím. Jeden z důvodů, proč neumím počet
takovýchhle podmnožin spočítat, je, že mi chybí nějaké \emph{uspořádání}.

Zatím všechny věci, které jsme počítali, byly v jistém smyslu \emph{uspořádané}.
Počet všech zobrazení $A \to B$ jsme počítali tak, že jsme prvky $A$
\textbf{jeden po druhém} zobrazovali na prvky $B$. Vlastně nevědomky jsme si tak
nějakým náhodným způsobem \emph{uspořádali} množinu $A$, aby se nám dobře
počítalo. Permutace jsou v podstatě přímo definované tak, že mění
\emph{uspořádání} prvků na množině.

Vybrat si nějaké uspořádání na $X$ a všechny její podmnožiny pak považovat za
uspořádané podle stejného uspořádání je chytrý nápad, který přinese ovoce. Na
konci výpočtu je však potřeba zanedbat všechny možné způsoby, kterými jsem
$k$-prvkové množiny mohl uspořádat. Uvidíte, že nám nakonec opravdu vyjde, že
počet všech $k$-prvkových podmnožin $X$ je vlastně počet všech $k$-prvkových
podmnožin $X$ s nějakým konkrétním uspořádáním dělen počtem způsobů, kolika jsem
takové uspořádání mohl zvolit.

Pojďme tedy místo množiny $\{x_1,\ldots,x_k\}$ uvažovat její \uv{uspořádanou
verzi}, tím míním $k$-tici $(x_1,\ldots,x_k)$. Rozdíl je samozřejmě v tom, že
(třeba pro $k = 3$) je množina $\{x_1,x_2,x_3\}$ ta samá, co $\{x_2,x_1,x_3\}$,
ale trojice $(x_1,x_2,x_3)$ je různá od trojice $(x_2,x_1,x_3)$. Záleží na
pořadí, v jakém prvky za sebe umisťuji, na \emph{uspořádání}.

Na první straně rovnice vzniklé \emph{počítáním dvěma způsoby} si určíme, kolik
různých takových $k$-tic mi z jedné množiny může vzniknout. No přeci tolik,
kolika způsoby mohu mezi sebou proházet (nebo třeba cizeji \uv{pro\-permutovat}
$\leftarrow$ hint btw) její prvky. Každá permutace na $\{x_1,\ldots,x_k\}$ mi
určuje přesně jedno možné uspořádání. Těch je, podle
 \hyperref[prop:pocet-permutaci-na-mnozine]{tvrzení~\ref*{prop:pocet-permutaci-na-mnozine}},
 $k!$. Z~definice máme $\# \binom{X}{k}$ různých $k$-prvkových podmnožin $X$ a
 každá určuje $k!$ uspořádaných $k$-tic. Celkem těchto tedy máme $k! \cdot \#
 \binom{X}{k}$. To činí jednu stranu naší rovnice.

Na druhé straně, vybrat 
