\subsection{Zobrazení}
\label{ssec:zobrazeni}

Druhým ze tří zvláště užitečných typů relace je tzv. \emph{zobrazení}, které
spíš znáte pod pojmem \emph{funkce}. Narozdíl od ekvivalence, zobrazení budeme
uvažovat jak na množině, tak mezi množinami.

Definující vlastností funkce/zobrazení je fakt, že každý prvek nezobrazí buď na
nic (pokud v něm \uv{není definováno}) nebo na jeden jiný prvek. V~jazyce relací
to znamená, že každý prvek z množiny \uv{nalevo} je v relaci s maximálně jedním
prvkem \uv{napravo}.

\begin{definition}[Zobrazení]
 \label{def:zobrazeni}
 Relaci $R$ mezi množinami $A$ a $B$ nazveme \textbf{zobrazením}, pokud pro
 každé $x \in A$ existuje \textbf{nejvýše jedno} $y \in B$ takové, že $xRy$.
\end{definition}

\begin{example}
 Mezi množinami $\clr{A} \coloneqq \clr{\{1,2,3,4\}}$ a $\clb{B} \coloneqq
 \clb{\{a,b,c\}}$ uvažme zobrazení
 \[
  \clg{R} \coloneqq \clg{\{(1, a), (2, a), (3, c), (4, b)\}} \subseteq \clr{A}
  \times \clb{B}.
 \]
 Jeho mříž vypadá následovně.
 \begin{figure}[H]
  \centering
  \begin{tikzpicture}
   \foreach \x in {0, 1, 2, 3} {
    \foreach \y in {0, 1, 2} {
     \node[circle,draw,fill=black,minimum size=2mm,inner sep=0pt,outer
     sep=0pt] at (\x, \y) {};
    }
   }
   \foreach \x in {1, 2, 3, 4} {
    \node at (\x - 1, -0.75) {\clr{\x}};
   }
   \node at (-0.75, 0) {\clb{a}};
   \node at (-0.75, 1) {\clb{b}};
   \node at (-0.75, 2) {\clb{c}};

   \node[draw=mygreen,circle,thick] at (0, 0) {};
   \node[draw=mygreen,circle,thick] at (1, 0) {};
   \node[draw=mygreen,circle,thick] at (2, 2) {};
   \node[draw=mygreen,circle,thick] at (3, 1) {};
  \end{tikzpicture}
  \caption{Mříž zobrazení $\clg{R} \subseteq \clr{A} \times \clb{B}$.}
  \label{fig:mriz-zobrazeni}
 \end{figure}
 Fakt, že relace je zobrazení poznáte z její mříže velmi snadno tak, že (za
 předpokladu, že prvky levé množiny píšete vždy dole) v každém sloupci je
 \textbf{maximálně} jeden zelený kroužek.
\end{example}

Jelikož lidé přemýšleli o zobrazeních dříve než o relacích, je jejich zápis a
názvosloví dost odlišné (a dost zmatené). Budeme je v dalších textu pravidelně
užívat, takže vás s ním chca nechca musíme seznámit.

Pro zápis zobrazení se obvykle používají malá písmena latinské abecedy počínaje
$f$ (pro \textbf{f}unction) nebo malá písmena řecké abecedy počínaje $\varphi$
(čteno \uv{fí}, opět pro \textbf{f}unction). Fakt, že relace $f \subseteq A
\times B$ je zobrazení mezi $A$ a $B$ (též říkáme \uv{z $A$ do $B$}), zapisujeme
obvykle jako
\[
 f:A \to B \quad \text{nebo} \quad A \overset{f}{ \to } B.
\]
Několik dalších názvů:
\begin{itemize}
 \item Fakt, že $xfy$ pro $x \in A$ a $y \in B$, zapisujeme jako $f(x) = y$ nebo
  jako $f:x \mapsto y$. Prvku $y$ říkáme \textbf{obraz} prvku $x$ \textbf{při
  zobrazení} $f$. \textbf{Obrazem zobrazení} $f$ pak myslíme množinu všech
  obrazů prvků z $A$ a značíme ji $\im f$ (z angl. \textbf{im}age). Konkrétně,
  \[
   \im f \coloneqq \{f(x) \mid x \in A\}.
  \]
 \item Pro dané $y \in B$ značíme množinu všech $x \in A$ takových, že $f(x) =
  y$, jako $f^{-1}(y)$ a říkáme jí \textbf{vzor} prvku $y$ \textbf{při
  zobrazení} $f$. Čili, $x \in f^{-1}(y)$ vyjadřuje fakt, že $f(x) = y$.
 \item Pokud $f:A \to B$, množině $A$ říkáme \textbf{doména zobrazení} $f$ a
  množině $B$ \textbf{kodoména zobrazení} $f$.
\end{itemize}

\begin{warning}
 Vzor prvku $y \in B$ při zobrazení $f$ je \textbf{množina}.
 \hyperref[def:zobrazeni]{Definice zobrazení} mi říká jenom, že jedno $x \in A$
 se zobrazí na jedno $y \in B$. To ale nebrání tomu, aby se víc různých prvků z
 $A$ zobrazilo na \textbf{ten samý} prvek z $B$. Naopak, množina $f^{-1}(y)$ 
 může být i prázdná, pokud se na $y$ nezobrazuje žádný prvek z $A$.
\end{warning}

\begin{example}[Kvadratická funkce]
 Kvadratická funkce daná předpisem
 \[
  f(x) \coloneqq x^2 + 4x + 5
 \]
 je zobrazení $\R \to \R$, čili jeho \textbf{doménou} i \textbf{kodoménou} jsou
 reálná čísla. \textbf{Obrazem} prvku $3$ je $f(3) = 26$, ale \textbf{vzorem}
 prvku $26$ je množina $\{-7,3\}$. Dále třeba vzorem prvku $0$ je prázdná
 množina, což je totéž, co říci, že rovnice
 \[
  x^2 + 4x + 5 = 0
 \]
 nemá v $\R$ řešení. Tradiční zápis $f$ jako relace by vypadal
 \[
  f = \{(x, x^2 + 4x + 5) \mid x \in \R\}  \subseteq \R \times \R.
 \]
\end{example}

Bohužel následuje ještě poslední kus názvosloví, protože pro určité
\uv{zají\-mavé} typy zobrazení máme zvláštní názvy.

\begin{definition}
 Zobrazení $f:A \to B$ nazveme
 \begin{itemize}
  \item \textbf{prosté} (nebo \textbf{injektivní}), pokud se každé dva
   \emph{různé }prvky v $A$ zobrazují na dva \emph{různé} prvky v $B$. Formálně,
   zobrazení $f$ je prosté, když
   \[
    f(x) = f(x') \Rightarrow x = x' \quad  \forall x,x' \in A.
   \]
   Ještě jinak řečeno, zobrazení je prosté, když vzorem každého prvku je buď
   prázdná nebo jednoprvková množina. Fakt, že $f$ je prosté, často zapisujeme
   jako $f:A \hookrightarrow B$.
  \item \textbf{na} (nebo \textbf{surjektivní}), když má každý prvek z $B$ 
   nějaký vzor v~$A$. Formálně, zobrazení je na, když
   \[
    \forall y \in B \; \exists x \in A: f(x) = y.
   \]
   Ještě jinak řečeno, zobrazení je na, když je vzor každého prvku neprázdná
   množina. Fakt, že $f$ je na, často symbolicky zapisujeme jako ${f:A
   \twoheadrightarrow B}$.
  \item \textbf{vzájemně jednoznačné} (nebo \textbf{bijektivní}), když je
   \emph{prosté} a \emph{na}, čili vzorem každého prvku je přesně jednoprvková
   množina. Fakt, že $f$ je bijekce, často zapisujeme jako $f:A \leftrightarrow
   B$ nebo $f: A \cong B$.
 \end{itemize}
\end{definition}

\begin{example}
 \hfill
 \vspace*{-.5\parskip}
 \begin{itemize}
  \item Zobrazení $f:\R \leftrightarrow \R, x \mapsto 2x + 3$ je
   \textbf{bijektivní}. Obecně, každá lineární funkce je bijektivní zobrazení.
   Důkaz je ponechán jako cvičení.
  \item Zobrazení $f:\R \hookrightarrow \R, x \mapsto 3 / x$ je \textbf{prosté},
   ale není na. To proto, že $f^{-1}(0) = \emptyset$.
  \item Zobrazení $f:\R \twoheadrightarrow \R, x \mapsto (x-2)(x-3)(x+1)$ je
   \textbf{na}, ale není prosté. Třeba $f^{-1}(0) = \{-1,2,3\}$.
  \item Zobrazení $f:\R \to \R, x \mapsto 1 + 2 / (x^2 - 1)$ není ani prosté,
   ani na. Například $f^{-1}(5 / 3) = \{-2,2\}$ a $f^{-1}(1) = \emptyset$.
 \end{itemize}
\end{example}

Prostá, surjektivní i bijektivní zobrazení mezi konečnými množinami z~jejich
mříží poznáte velmi snadno. Prostá zobrazení mají v řádcích maximálně jeden
prvek; surjektivní zobrazení mají v~každém řádku aspoň jeden prvek; ta
bijektivní mají v každém řádku přesně jeden prvek. Pár obrázků.

\begin{figure}[h]
 \centering
 \begin{tikzpicture}
   \foreach \x in {0, 1, 2, 3} {
    \foreach \y in {0, 1, 2} {
     \node[circle,draw,fill=black,minimum size=2mm,inner sep=0pt,outer
     sep=0pt] at (\x, \y) {};
    }
   }
   \foreach \x in {1, 2, 3, 4} {
    \node at (\x - 1, -0.75) {\clr{\x}};
   }
   \node at (-0.75, 0) {\clb{a}};
   \node at (-0.75, 1) {\clb{b}};
   \node at (-0.75, 2) {\clb{c}};

   \node[draw=mygreen,circle,thick] at (0, 0) {};
   \node[draw=mygreen,circle,thick] at (3, 1) {};
 \end{tikzpicture}
 \caption{Mříž \textbf{prostého} zobrazení $\clg{f} \coloneqq
 \clg{\{(1,a),(4,b)\}}$.}
 \label{fig:mriz-proste}
\end{figure}
\newpage

\begin{figure}[h]
 \centering
 \begin{tikzpicture}
   \foreach \x in {0, 1, 2, 3} {
    \foreach \y in {0, 1, 2} {
     \node[circle,draw,fill=black,minimum size=2mm,inner sep=0pt,outer
     sep=0pt] at (\x, \y) {};
    }
   }
   \foreach \x in {1, 2, 3, 4} {
    \node at (\x - 1, -0.75) {\clr{\x}};
   }
   \node at (-0.75, 0) {\clb{a}};
   \node at (-0.75, 1) {\clb{b}};
   \node at (-0.75, 2) {\clb{c}};

   \node[draw=mygreen,circle,thick] at (0, 2) {};
   \node[draw=mygreen,circle,thick] at (1, 1) {};
   \node[draw=mygreen,circle,thick] at (2, 1) {};
   \node[draw=mygreen,circle,thick] at (3, 0) {};
 \end{tikzpicture}
 \caption{Mříž \textbf{surjektivního} zobrazení $\clg{f} \coloneqq
 \clg{\{(1,c),(2,b),(3,b),(4,a)\}}$.}
 \label{fig:mriz-proste}
\end{figure}

\textbf{Bijekce mezi množinami, které mají různý počet prvků existovat nemůže}.
Důkaz si zkusíte za cvičení. Částečně ho ale dává následující slibovaná definice
velikosti množiny pomocí bijektivních zobrazení.

\begin{definition}[Velikost množiny pořádně]
 \label{def:velikost-mnoziny-poradne}
 Pro přirozené číslo $n \in \N$ označíme symbolem $[n]$ množinu všech
 přirozených čísel od $1$ až do $n$ včetně. Čili,
 \[
  [n] \coloneqq \{1,2,\ldots,n\}.
 \]
 Množinu $A$ nazveme \textbf{konečnou}, pokud existuje přirozené číslo ${n \in
 \N}$ a bijekce $f:[n] \cong A$. V takovém případě číslu $n$ říkáme
 \textbf{velikost} množiny $A$ a značíme $\# A \coloneqq n$.
\end{definition}

Bijekci $f:[n] \cong A$ z \hyperref[def:velikost-mnoziny-poradne]{definice
nahoře} můžeme vnímat jako \uv{očíslování} prvků množiny $A$ čísly od $1$ do
$n$. Takových očíslování je samozřejmě mnoho. Kolik?

\begin{example}
 Množina $B \coloneqq \{a,b,c\}$ má tři prvky. Jedna z možných bijekcí ${f:[3]
 \cong B}$ je
 \[
  f \coloneqq \{(1,c),(2,a),(3,b)\}.
 \]
\end{example}

Posledním důležitým konceptem je pojem \emph{inverzního zobrazení}. Intuitivně,
a vlastně i formálně, inverzní zobrazení je zobrazení, které jde opačným směrem
a obrazy posílá zpátky na vzory. Toto samozřejmě vyžaduje například, aby vzor
byl vždy nejvýše jeden. Detaily si rozmyslíte jako cvičení.

\begin{definition}[Inverzní zobrazení]
 Nechť $f:A \to B$ je zobrazení. \textbf{Inverzním zobrazením} k $f$, značeným
 dost nevhodně $f^{-1}$, nazveme zobrazení $B \to A$ splňující
 \[
  f^{-1}(f(x)) = x \wedge f(f^{-1}(y)) = y \quad  \forall x \in A,y \in \im f.
 \]
 Pozor! Inverzní zobrazení \emph{nemusí existovat}.
\end{definition}

\begin{warning}
 Pokud k $f:A \to B$ existuje inverzní zobrazení, značí $f^{-1}(y)$ jak množinu
 vzorů prvku $y \in B$, tak obraz prvku $y$ při inverzním zobrazení.

 Toto však je problém pouze formální. Pokud totiž existuje inverzní zobrazení,
 pak má množina $f^{-1}(y)$ buď jeden prvek, nebo žádný. V~prvním případě tedy
 akorát ztotožňuji jednoprvkovou množinu s jejím jediným prvkem. To je totéž, co
 považovat třeba množinu $\{2\}$ a číslo $2$ za to samé. Vskutku, problém pouze
 formální, bez praktických důsledků.
\end{warning}

Sekci završíme ku radosti všech párem cvičení.

\begin{exercise}
 Vyřešte následující úlohy rozprostřené po sekci. Konkrétně,
 \begin{itemize}
  \item dokažte, že mezi dvěma konečnými množinami různé velikosti neexistuje
   žádná bijekce.
  \item pro množinu $A$ velikosti $n$ určete počet různých bijektivních
   zobrazení ${f:[n]
   \cong A}$.
  \item určete, jakou podmínku splňují zobrazení $f:A \to B$, ke kterým existuje
   zobrazení inverzní.
 \end{itemize}
\end{exercise}

\begin{exercise}
 Dokažte, že každá lineární funkce $f:\R \to \R$, tedy funkce daná předpisem
 \[
  f(x) = ax + b \quad \text{pro } a,b \in \R,a \neq 0
 \]
 je bijektivní zobrazení.
\end{exercise}

\begin{exercise}
 Nechť $A$ je konečná množina. Zformulujte důkaz, že zobrazení ${f:A \to A}$,
 které je definované pro každé $x \in A$, je \textbf{prosté, právě tehdy když je
 na}.

 Tento fakt se často též používá v teorii množin jako definice konečné množiny.
 Je hezčí než naše v tom, že nespoléhá na množinu přirozených čísel. Tedy,
 množinu $A$ nazvu \emph{konečnou}, když každé zobrazení $f:A \to A$ definované
 všude je prosté, právě tehdy když je na.
\end{exercise}

\begin{exercise}
 Najděte příklad zobrazení $f:\N \to \N$ definovaného na celém $\N$, které je
 \begin{enumerate}
  \item prosté, ale není na.
  \item na, ale není prosté.
 \end{enumerate}
\end{exercise}
