\subsection{Permutace}
\label{ssec:permutace}

Permutace jsou vlastně zobrazení, která prohazují prvky množin. Jejich asi
hlavním účelem je formalizovat koncept, že \uv{nezáleží na pořadí} nebo naopak,
že všechno dělám pro všechna možná přeuspořádání prvků. Člověk by měl dobrý
důvod si myslet, že nejsou dobré k ničemu jinému, než ke zkrášlení zápisu. Opak
je pravdou. Permutace mají velmi překvapivé aplikace v oblastech matematiky, kde
by je jeden nehledal. Zmiňme tři příklady.
\begin{itemize}
 \item Důkaz základní věty algebry -- tvrzení, že každý komplexní polynom má
  komplexní kořen -- silně využívá tzv. rozklad na symetrické polynomy, založený
  na vlastnostech permutací.
 \item Fakt, že kořeny obecných reálných (i komplexních) polynomů nelze zapsat v
  radikálech (tj. odmocninách), když je stupeň polynomu větší nebo roven 5 (tj.
  objevuje se v něm $x^{5}$), se opírá o tzv. \uv{neřešitelnost} permutačních
  grup (množin permutací na dané množině s binární operací skládání).
 \item Důkaz, že na každé Riemannově pseudovarietě dimenze 4 (kterou fyzikové
  používají jako model časoprostoru) existuje nekonečně mno\-ho neisomorfních
  Riemannových metrik (tj. v našem vesmíru mohu měřit vzdálenost nekonečně mnoha
  neekvivalentními způsoby) staví na symetrii tensorů definovaných pomocí
  permutací.
\end{itemize}

Takže, o co tu vlastně jde.
\begin{definition}[Permutace]
 \label{def:permutace}
 Bijekce $\sigma:X \to X$ konečné množiny $X$ na sebe samu se nazývá
 \emph{permutace} množiny $X$.

 Množinu všech permutací na $X$ značíme $S_X$ (jako grupa \textbf{s}ymetrií
 $X$).
\end{definition}
