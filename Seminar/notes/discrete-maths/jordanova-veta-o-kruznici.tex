\subsubsection{Jordanova věta o kružnici}
\label{sssec:jordanova-veta-o-kruznici}

Zásadním nástrojem pro studium rovinných grafů -- grafů, jež lze kreslit bez
křížení hran -- je tzv. \emph{Jordanova věta o kružnici}, dlejíc opět spíše
v~bydle matematické analýzy. Je zdárným příkladem faktu, že matematický jazyk
není vždy v souladu s lidskou intuicí. Jedná se totiž, čtenáři jistě budou
souhlasit, o intuitivně zřejmé tvrzení, které není však zdaleka snadné (avšak
ani příliš náročné) dokázat.

Jordanova věta o kružnici říká zjednodušeně to, že nakreslíme-li v rovině
kružnici a poté ji vyjmeme, rozdělíme rovinu na dvě oblastě -- jednu omezenou
(vnitřek kružnice) a jednu neomezenou (vnějšek kružnice). Jistá potíž skrývá
sebe ve faktu, že zde \emph{kružnicí} nemyslíme onu krásnou buclatou pravidelně
kulatou ... no ... kružnici, ale libovolnou souvislou čáru, která se neprotíná a
je uzavřená, tj. začíná tam, kde končí. Takové kružnici se říká třeba
\emph{topologická}, ale spíš jí nikdo žádné zvláštní jmě obyčejně nepřiřazuje.
Je to vlastně křivka v rovině s tím rozdílem, že není zcela prostá, neboť se
její koncové body shodují. Učinivše za dosti intuici, pokračujeme již formální
definicí.

\begin{definition}[Topologická kružnice]
 \label{def:topologicka-kruznice}
 \emph{Topologickou kružnicí} nazveme libovolné \textbf{spojité} zobrazení
 $\kappa: [0,1] \to \R^2$ takové, že
 \begin{itemize}
  \item $\kappa(0) = \kappa(1)$ a
  \item $\kappa$ je prosté na $(0,1)$.
 \end{itemize}
\end{definition}

Než trpělivé čtenáře seznámíme s formálním zněním Jordanovy věty o kružnici,
musíme zmínit, co znamená, že nějaká podmnožina $\Omega \subseteq \R^2$ je
\emph{oblast}. Toto substantivum v sobě obvykle nese dvě vlastnosti:
\begin{itemize}
 \item otevřenost (každý bod $\Omega$ má kolem sebe okolí, tj. nekonečně mnoho
  bodů nekonečně blízko sebe; též se dá říct, že $\Omega$ \uv{nemá hranici}),
 \item souvislost (z každého bodu se dá po křivce dostat do každého).
\end{itemize}

Není těžké si všimnout (aspoň v případě kulaté kružnice), že hranicí jejího
vnějšku i vnitřku je právě ona. Když ji vyjmeme, obě množiny přijdou o svou
hranici a stanou se oblastmi.

Konečně, \emph{omezenost} podmnožiny $\R^2$ znamená, že existuje horní limit na
vzdálenost (v přirozeném slova smyslu) mezi dvěma jejími body.

\begin{theorem}[Jordanova o kružnici]
 \label{thm:jordanova-o-kruznici}
 Ať $\kappa$ je topologická kružnice. Pak se $\R^2 \setminus \img \kappa$
 rozpadá na dvě disjunktní oblasti -- jednu omezenou a jednu neomezenou.
\end{theorem}

Jak jsme již zmiňovali, důkaz \hyperref[thm:jordanova-o-kruznici]{Jordanovy
věty} je kvítí ve vínku matematické analýzy a my jej zde uvádět nebudeme. Pouze
ji v dalším textu občas použijeme.

