\subsection{Zobrazení a podmnožiny}
\label{ssec:zobrazeni-a-podmnoziny}

Chvíli se budeme bavit počítáním zobrazení a podmnožin obvykle určených nějakou
hezkou podmínkou. Začít právě tady je vhodné z páru důvodů. Zaprvé, není potřeba
vymýšlet žádnou novou teorii a zadruhé -- snad trochu překvapivě -- umět počítat
zobrazení a podmnožiny se hodí do spousty dalších matematických disciplín.
Zmiňme Čínskou větu o zbytcích, v podstatě jeden ze základních stavebních kamenů
teorie čísel, jejíž důkaz je založen právě na tom, že umíme počítat zobrazení
mezi množinami. Dále je tu třeba Burnsideova věta z abstraktní algebry, na jejíž
pravdivosti jsou postaveny různé třídy monitorů a jejíž důkaz vyžaduje
porovnávání velikostí systémů podmnožin. Konečně, patří sem i latinské čtverce
-- struktury, jejichž princip stojí za vznikem Sudoku.

Pojďme začít tím nejjednodušším možným tvrzením, tedy o počtu všech zobrazení
mezi množinami. Ukážeme si dva důkazy: jeden přímý a jeden indukcí.

\begin{warning}
 Pro stručnost budu v celé kapitole slovem zobrazení myslet \textbf{zobrazení
 definované všude}. Diskrétní matematiku totiž pravdať úplně netrápí problémy
 definičních oborů, takže není žádná výhoda v tom uvažovat zobrazení, která
 nejsou definována pro všechny prvky svých domén.
\end{warning}

\begin{claim}
 \label{claim:pocet-zobrazeni}
 Mějme konečné množiny $A$ a $B$. Počet všech zobrazení $A \to B$ je $\# B^{\#
 A}$.
\end{claim}

\begin{enhproof}[tvrzení~\ref{claim:pocet-zobrazeni} přímo]
 Rozmysleme si nejprve, kdy se dvě zobrazení $f,g:A \to B$ liší. To je přeci
 tehdy, když existuje nějaký prvek $a \in A$ takový, že $f(a) \neq g(a)$.

 Jinak řečeno, každé zobrazení $A \to B$ popíšu tak, že určím obrazy všech prvků
 z $A$. Kdykoli mám dvě zobrazení, jejichž obraz byť i jednoho prvku z $A$ se
 neshoduje, pak jsou to různá zobrazení. Pro každý prvek z $A$ mám přesně $\# B$
 prvků, na které ho mohu zobrazit, tedy mám celkem přesně $\# B^{\# A}$ možností,
 jak zobrazit všechny prvky z $A$ na prvky z $B$.
\end{enhproof}

\begin{enhproof}[tvrzení~\ref{claim:pocet-zobrazeni} indukcí]
 Dokážeme předchozí tvrzení užitím indukce podle velikosti množiny $A$.

 Když je $A$ prázdná, čili $\# A = 0$, pak mám právě jedno zobrazení $A \to B$
 -- to, které nezobrazuje nic na nic. Čili mám vskutku $\# B^{\# A} = \# B^{0} =
 1$ různých zobrazení $A \to B$.

 Předpokládejme, že platí, že zobrazení z $A$ do $B$ je právě $\# B^{\# A}$ a
 přidejme do množiny $A$ jeden prvek, třeba $x$. Chceme ukázat, že všech
 zobrazení $A \cup \{x\} \to B$ je $\# B^{\# A + 1}$. Jeden způsob, jak to
 udělat, je podívat se kolika způsoby můžeme zobrazení $A \to B$
 \uv{dodefinovat} v $x$. No, $x$ přeci mohu zobrazit na jakýkoliv prvek z $B$ a
 každá volba obrazu mi dává jiné zobrazení. Čili, z jednoho zobrazení $A \to B$
 mi vznikne právě $\# B$ různých zobrazení $A \cup \{x\} \to B$. To ale znamená,
 že všech zobrazení $A \cup \{x\} \to B$ je $\# B$-krát víc než zobrazení $A \to
 B$. Tedy jich je podle předpokladu
 \[
  \# B^{\# A}\# B = \# B^{\#A + 1}.\qedhere
 \]
\end{enhproof}


