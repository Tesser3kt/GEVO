\subsection{Pohyb v grafu}
\label{ssec:pohyb-v-grafu}

Jak jsme zmínili na začátku kapitoly, mnoho aplikací teorie grafů využívá
interpretace této struktury jako množiny \uv{uzlů} se \uv{spojnicemi}, po
kterých se dá mezi uzly pohybovat. V této podsekci dáme pohybu po spojnicích
formální tvář.

Nejzákladnějším typem pohybu v grafu je libovolná posloupnost vrcholů se zcela
přirozenou podmínkou, že mezi vrcholy v posloupnosti bezprostředně za sebou musí
vést hrana. Takové posloupnosti se říká \emph{sled} (angl. \emph{walk}).

\begin{definition}[Sled poprvé]
\label{def:sled-poprve}
 Ať $G = (V,E)$ je graf. Posloupnost vrcholů $v_1v_2\cdots v_n$ nazveme
 \emph{sledem} v grafu $G$, pokud $v_i v_{i+1} \in E$ pro každý index ${i \in
 \{1,\ldots,n-1\}}$.
\end{definition}

Je však užitečné si uvědomit, že každý sled lze ekvivalentně definovat jako
posloupnost navazujících hran. Tento pohled (jak uvidíme později v kapitole) má
své aplikace -- často nás totiž zajímá, po kterých spojnicích chodíme, spíše než
které uzly procházíme.

Máme-li sled $v_1v_2 \cdots v_n$, tak každá dvojice $v_iv_{i+1}$ musí být hranou
v $G$. To ovšem znamená, že zcela identickou informaci poskytuje i posloupnost
hran $e_1e_2 \cdots e_{n-1}$, kde $e_i = v_iv_{i+1}$.

\begin{remark}
 Tento princip, který prostupuje matematické struktury, je až překvapivě
 zásadního významu. Zobecníme-li trochu předchozí pozorování, uvědomíme si, že
 množina hran vlastně už v sobě obsahuje informaci o všech prvcích množiny $V$.
 Tedy bychom dokonce mohli definovat graf pouze jako množinu hran a množinu
 vrcholů bychom takto získali automaticky. Opak samozřejmě není pravdou.

 Situace, kdy struktura na množině (či něčem složitějším) je dostatečně
 \uv{hustá}, aby obsahovala kompletní informaci o této množině, je obecně velmi
 žádaná, jelikož struktura je často konstruována systematicky (a tedy jí
 rozumíme lépe) ve srovnání s náhodnou volbou její bázové množiny.

 Pro nedostatek představivosti uvedeme příklad z homologické algebry, kde
 injektivní a projektivní rezolventy modulů obsahují již kompletní informaci o
 daném modulu. Tedy člověku pro pochopení teorie modulů z homologického hlediska
 \uv{stačí} studovat injektivní a projektivní moduly, které jsou z definice více
 omezené než moduly obecné, pročež snadněji popsatelné.
\end{remark}

\begin{definition}[Sled podruhé]
\label{def:sled-podruhe}
 Ať $G = (V,E)$ je graf. Posloupnost hran $e_1e_2 \cdots e_n$ nazveme
 \emph{sledem} v grafu $G$, pokud $t(e_i) = s(e_{i+1})$ pro všechna $i \in
 \{1,2,\ldots,n-1\}$.
\end{definition}

\begin{figure}[h]
 \centering
 \begin{tikzpicture}
  \tikzset{vertex/.style = {shape=circle,fill,text=white,minimum size=6pt,inner sep=1pt}}
  \tikzset{->-/.style={decoration={
  markings,
  mark=at position #1 with {\arrow{>[scale=0.8]}}},postaction={decorate}}}
  \node[vertex,myred,text=white] (1) at (-1, 1) {$1$};
  \node[vertex,myblue,text=white] (2) at (1, 1) {$2$};
  \node[vertex,myblue,text=white] (3) at (1, -1) {$3$};
  \node[vertex,myred,text=white] (4) at (-1, -1) {$4$};

  \draw[->-=.55,ultra thick,myblue] (2) -- (1);
  \draw[thick] (1) -- (3);
  \draw[->-=.3,->-=.8,ultra thick,myred] (1) -- (4);
  \draw[thick] (2) -- (3);
  \draw[->-=.55,ultra thick,myblue] (4) -- (2);
  \draw[->-=.55,ultra thick,myblue] (4) -- (3);
 \end{tikzpicture}
 \caption{Příklad sledu $142143$ v grafu. Jednou navštívené hrany a vrcholy jsou
 značené \clb{modře}, dvakrát navštívené \clr{červeně}.}
 \label{fig:priklad-sledu}
\end{figure}

Méně obecným pohybem v grafu je sled, ve kterém se nesmějí opakovat hrany.
Takové sledy lze najít často třeba v běžné situaci, kdy si jako rozvůzce jídla
plánujete cestu městem. Jako uzly si označíte například místa, která musíte
objet, a hrany budou nejkratší trasy mezi nimi. Projet přes některá místa
vícekrát vám příliš vadit nemusí, ale ztrácet čas cestováním po stejné trase
sem a tam byste neradi.

Takovému sledu se říká \emph{tah} (angl. \emph{trail}). Jeho definice je velmi
přirozená.

\begin{definition}[Tah]
\label{def:tah}
 Ať $G = (V,E)$. \emph{Tahem} v grafu $G$ nazveme buď
 \begin{enumerate}
  \item sled (vrcholů) $v_1v_2\cdots v_n$ takový, že $v_i v_{i+1} \neq v_j
   v_{j+1}$ pro všechna $i \neq j$, nebo
  \item sled (hran) $e_1e_2 \cdots e_{n-1}$ takový, že $e_i \neq e_j$ pro
   všechna $i \neq j$.
 \end{enumerate}
\end{definition}

\begin{warning}
 Obě uvedené definice tahu jsou v našem (částečně neformálním) pojetí hran
 skutečně ekvivalentní. Napíšeme-li totiž $v_i v_{i+1} \neq v_j v_{j+1}$ myslíme
 tím vlastně dvě nerovnosti:
 \[
  (v_i,v_{i+1}) \neq (v_j,v_{j+1}) \wedge (v_i,v_{i+1}) \neq (v_{j+1},v_j);
 \]
 nebo zápis můžeme též chápat jako
 \[
  \{v_i,v_{i+1}\} \neq \{v_j,v_{j+1}\},
 \]
 čili každou hranu chceme projít (kterýmkoli směrem) maximálně jednou.
\end{warning}

\begin{figure}[h]
 \centering
 \begin{tikzpicture}
  \tikzset{vertex/.style = {shape=circle,fill,text=white,minimum size=6pt,inner sep=1pt}}
  \tikzset{->-/.style={decoration={
   markings,
   mark=at position #1 with {\arrow{>[scale=0.8]}}},postaction={decorate}}}

  \node[vertex,myred,text=white] (1) at (-1, 1) {$1$};
  \node[vertex,myred,text=white] (2) at (1, 1) {$2$};
  \node[vertex,myblue,text=white] (3) at (1, -1) {$3$};
  \node[vertex,myblue,text=white] (4) at (-1, -1) {$4$};

  \draw[->-=.55,ultra thick,myblue] (1) -- (2);
  \draw[->-=.45,ultra thick,myblue] (3) -- (1);
  \draw[->-=.55,ultra thick,myblue] (1) -- (4);
  \draw[->-=.55,ultra thick,myblue] (2) -- (3);
  \draw[->-=.65,ultra thick,myblue] (4) -- (2);
  \draw[thick] (3) -- (4);
 \end{tikzpicture}
 \caption{Příklad tahu $142312$ v grafu. Jednou navštívené hrany a vrcholy jsou
 značené \clb{modře}, dvakrát navštívené \clr{červeně}.}
 \label{fig:priklad-sledu}
\end{figure}

Posledním, a asi nejčastěji zkoumaným, typem pohybu grafem je \emph{cesta}
(angl. \emph{path}), což je sled, ve kterém se nesmějí opakovat ani hrany ani
vrcholy. Pro jednoduchost je užitečné si uvědomit, že z podmínky neopakování
vrcholů automaticky plyne podmínka neopakování hran. Pokud bychom totiž chtěli
po nějaké hraně přejít dvakrát, tak zároveň dvakrát projdeme její koncové
vrcholy. Definovat cestu pomocí sledu vrcholů je tudíž přímočaré.

U sledu hran je to však horší. Protože každá hrana definuje dva vrcholy sledu,
je třeba zařídit, aby se oba koncové body každé hrany lišily od obou koncových
bodů jiné hrany, ale jenom tehdy, když hrany nejdou bezprostředně za sebou. To
bohužel vede na trochu neintuitivní definici cesty přes hrany. Totiž, cesta
obsahující $n$ hran nutně obsahuje $n+1$ vrcholů, protože po sobě jdoucí hrany
vždy sdílejí jeden vrchol. Rafinovaně se tedy cesta přes hrany dá vyjádřit jako
tah, kde sjednocení přes všechny hrany (vnímané tentokrát jako množiny) má
velikost právě $n + 1$.

Studium cest má aplikace například právě v návrhu elektrických obvodů, kdy zcela
jistě nechcete, aby do připojených zařízení šel proud z více, než jednoho místa.

\begin{definition}[Cesta]
\label{def:cesta}
 Ať $G = (V,E)$ je graf. \emph{Cestou} v grafu $G$ nazveme buď
 \begin{enumerate}
  \item sled (vrcholů) $v_1v_2 \cdots v_n$ takový, že $v_i \neq v_j$ pro všechna
   $i \neq j$, nebo
  \item tah (hran) $e_1e_2 \cdots e_{n-1}$ takový, že
   \[
    \# \bigcup_{i=1}^{n-1} e_i = n.
   \]
 \end{enumerate}
\end{definition}

\begin{figure}[h]
 \centering
 \begin{tikzpicture}
  \tikzset{vertex/.style = {shape=circle,fill,text=white,minimum size=6pt,inner sep=1pt}}
  \tikzset{->-/.style={decoration={
   markings,
   mark=at position #1 with {\arrow{>[scale=0.8]}}},postaction={decorate}}}

  \node[vertex,myblue,text=white] (1) at (-1, 1) {$1$};
  \node[vertex,myblue,text=white] (2) at (1, 1) {$2$};
  \node[vertex,myblue,text=white] (3) at (1, -1) {$3$};
  \node[vertex,myblue,text=white] (4) at (-1, -1) {$4$};

  \draw[->-=.55,ultra thick,myblue] (1) -- (2);
  \draw[thick] (1) -- (3);
  \draw[thick] (1) -- (4);
  \draw[->-=.55,ultra thick,myblue] (2) -- (3);
  \draw[thick] (2) -- (4);
  \draw[->-=.55,ultra thick,myblue] (3) -- (4);
 \end{tikzpicture}
 \caption{Příklad cesty $1234$ v grafu. Navštívené hrany a vrcholy jsou značené
 \clb{modře}.}
 \label{fig:priklad-sledu}
\end{figure}

Posledním důležitým konceptem v grafu je tzv. \emph{cyklus} (též
\emph{kružnice}, angl. \emph{cycle}). Jedná se vlastně o \uv{téměř cestu}, která
končí tam, kde začala. Konkrétně je to tedy cesta prodloužená o svůj první
vrchol (samozřejmě automaticky předpokládáme, že existuje hrana z posledního
vrcholu do prvního).

\begin{definition}[Cyklus]
\label{def:cyklus}
 Ať $G = (V,E)$ je graf. \emph{Cyklem} v grafu nazveme buď
 \begin{enumerate}
  \item sled (vrcholů) $v_1v_2  \cdots v_nv_1$, pokud $v_1v_2 \cdots v_n$ je
   cesta v $G$, nebo
  \item tah (hran) $e_1e_2 \cdots e_{n-1}e_n$, kde $t(e_n) = s(e_1)$ a $e_1e_2
   \cdots e_{n-1}$ je cesta v~$G$.
 \end{enumerate}
\end{definition}

\begin{figure}[h]
 \centering
 \begin{tikzpicture}
  \tikzset{vertex/.style = {shape=circle,fill,text=white,minimum size=6pt,inner sep=1pt}}
  \tikzset{->-/.style={decoration={
   markings,
   mark=at position #1 with {\arrow{>[scale=0.8]}}},postaction={decorate}}}

  \node[vertex,myred,text=white] (1) at (-1, 1) {$1$};
  \node[vertex,myblue,text=white] (2) at (1, 1) {$2$};
  \node[vertex,myblue,text=white] (3) at (1, -1) {$3$};
  \node[vertex,myblue,text=white] (4) at (-1, -1) {$4$};

  \draw[->-=.55,ultra thick,myblue] (1) -- (2);
  \draw[thick] (1) -- (3);
  \draw[->-=.55,ultra thick,myblue] (4) -- (1);
  \draw[->-=.55,ultra thick,myblue] (2) -- (3);
  \draw[thick] (2) -- (4);
  \draw[->-=.55,ultra thick,myblue] (3) -- (4);
 \end{tikzpicture}
 \caption{Příklad cyklu $12341$ v grafu. Jednou navštívené hrany a vrcholy jsou
 značené \clb{modře}, dvakrát navštívené \clr{červeně}.}
 \label{fig:priklad-cyklu}
\end{figure}

Příklad cyklu vidíte na \myref{obrázku}{fig:priklad-cyklu}. Jejich význam zatím
necháme zahalen tajemstvím, jež má být odkryto v nejvíce dramatickou chvíli.
