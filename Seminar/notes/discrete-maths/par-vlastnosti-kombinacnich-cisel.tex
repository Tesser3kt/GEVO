\subsubsection{Pár vlastností kombinačních čísel}
\label{sssec:par-vlastnosti-kombinacnich-cisel}

Tahle sekce si neklade za cíl objevit zatím neznámý kontinent ani čtenáře naučit
životu v afrických pralesích. Baže naopak, jedná se o prostou přílohu k již
známému. Kombinační čísla se objevují, kdykoli člověk počítá s~podobjekty
konečných objektů, tedy v podstatě pořád. Věnujeme chvilku času prozkoumání
způsobů, jak s nimi zacházet.

Nejprve jeden výpočetně užitečný vzoreček.

\begin{lemma}
 \label{lem:vzorec-pro-kombinacni-cislo}
 Ať $k,n \in \N$, $k \leq n$. Platí
 \[
  \binom{n}{k} = \frac{n!}{k!(n-k)!}.
 \]
\end{lemma}
\begin{proof}
 Z \hyperref[def:kombinacni-cislo]{definice kombinačního čísla} máme
 \[
  \binom{n}{k} = \frac{\prod_{i=0}^{k-1} n-i}{k!},
 \]
 stačí tedy ukázat, že
 \[
  \prod_{i=0}^{k-1} n - i = \frac{n!}{(n-k)!}.
 \]
 To je však zřejmé, neboť
 \begin{align*}
  n! &= n(n-1) \cdots (n-k+1)(n-k)(n-k-1) \cdots 1\\
  &= n(n-1)\cdots (n-k+1)(n-k)! = \left( \prod_{i=0}^{k-1} n-i \right)(n-k)!.
 \end{align*}
 Důkaz plyne z vydělení poslední rovnice číslem $(n-k)!$.
\end{proof}

Dále si povíme o tzv. \emph{Pascalově trojúhelníku}. Tím se obvykle míní
následující struktura.

\begin{figure}[h]
 \centering
 \begin{tikzpicture}
  \node at (0,0.5) {$1$};

  \node at (-0.5,0) {$1$};
  \node at (0.5,0) {$1$};

  \node at (-1,-0.5) {$1$};
  \node at (0,-0.5) {$2$};
  \node at (1,-0.5) {$1$};

  \node at (-1.5,-1) {$1$};
  \node at (-0.5,-1) {$3$};
  \node at (0.5,-1) {$3$};
  \node at (1.5,-1) {$1$};

  \node at (-2,-1.5) {$1$};
  \node at (-1,-1.5) {$4$};
  \node at (0,-1.5) {$6$};
  \node at (1,-1.5) {$4$};
  \node at (2,-1.5) {$1$};

  \node at (-2.5,-2) {$1$};
  \node at (-1.5,-2) {$5$};
  \node at (-0.5,-2) {$10$};
  \node at (0.5,-2) {$10$};
  \node at (1.5,-2) {$5$};
  \node at (2.5,-2) {$1$};

  \node at (0, -2.5) {$\vdots$};
 \end{tikzpicture}
 \caption{Pascalův trojúhelník.}
 \label{fig:pascaluv-trojuhelnik}
\end{figure}

Jde vlastně o posloupnost řad, kde na začátku a na konci každé řady je číslo $1$
a prostřední čísla dostanu tak, že sečtu ta dvě čísla z předchozí řady těsně nad
ním.

Formálně můžeme říci, že Pascalův trojúhelník je posloupnost uspořádaných
$n$-tic $(p_1^{n},\ldots,p_n^{n}) \in \N^{n}$ ($n$ jsou \textbf{indexy} řádků,
nikoli mocniny), taková, že $p_i^{n} = p_{i-1}^{n-1} + p_{i}^{n-1}$ pro každé $n
\geq 1$ a každé $1 \leq i \leq n$. Pro začátek položíme $p_1^{1} = 1$ a v každém
řádku dodefinujeme $p_{0}^{n} = p_{n+1}^{n} = 0$.

Skutečně, když $p_1^{1} = 1$, tedy první číslo prvního řádku je $1$, pak
$p_1^{2} = p_0^{1} + p_1^{1} = 0 + 1 = 1$ a $p_2^{2} = p_1^{1} + p_2^{1} = 1 + 0
= 1$, čili druhý řádek je dvojice $(1, 1)$. Zde vidíte důvod, proč jsme
dodefinovali též $0$-tý a $(n+1)$-ní prvek $n$-tého řádku. Museli bychom totiž
jinak psát speciální pravidlo pro určení prvního a posledního prvku každého
řádku. Místo toho předstíráme, že je Pascalův trojúhelník ještě z obou stran
obklopen nulami.

Pro pořádek si ještě v tomto formálním pohledu spočteme třetí řádek, tj. trojici
$(p_1^{3},p_2^{3},p_3^3)$. Máme
\begin{align*}
 p_1^3&= p_0^2+p_1^2 = 0 + 1 = 1, \\
 p_2^3&= p_1^{2}+p_2^2 = 1 + 1 = 2, \\
 p_3^3&= p_2^2 + p_3^2 = 1 + 0 = 1.
\end{align*}
Vše je, jak má být.

Budeme chtít ukázat, že $n$-tý řádek Pascalova trojúhelníku tvoří přesně čísla
$\binom{n-1}{0},\binom{n-1}{1},\ldots,\binom{n-1}{n-1}$. Pro první řádky je to
jistě pravda, neboť $\binom{0}{0} = 1$ (neboť $0!$ se tradičně definuje jako
$1$) a dále $\binom{1}{0} = \binom{1}{1} = 1$.

Rozepíšeme si, co naše tvrzení vlastně znamená z pohledu kombinačních čísel.
Prvky $p_i^{n+1}$ v $(n+1)$-ním řádku Pascalova trojúhelníku jsou definovány
pomocí prvků v předchozím řádku vzorcem $p_{i+1}^{n+1} = p_{i}^{n} +
p_{i+1}^{n}$. A my tvrdíme, že $p_{i}^{n} = \binom{n-1}{i-1}$. (Ověřte si, že to
je \textbf{opravdu} to, co říkáme!) Přepíšeme-li tuto rovnost v kombinačních
číslech, potřebujeme dokázat, že
\begin{equation*}
 \label{eq:pascal-identity}
 \tag{$*$}
 \binom{n}{i} = \binom{n-1}{i-1} + \binom{n-1}{i}
\end{equation*}
pro každé $n \geq 1$ každé $1 \leq i \leq n$.

I když by to jistě nějak šlo upočítat, my zvolíme elegantnější způsob, který
zůstává věrný tomu, co kombinační číslo vlastně \textbf{vyjadřuje}. Nezapomeňte,
že $\binom{n}{i}$ je počet $i$-prvkových podmnožin $n$-prvkové množiny. Je
jisté, že množina $(i-1)$-prvkových podmnožin je disjunktní (má prázdný průnik) s
množinou $i$-prvkových podmnožin. To ovšem znamená, že
\begin{align*}
 \# \left( \binom{\{1,\ldots,n-1\}}{i-1} \cup \binom{\{1,\ldots,n-1\}}{i}
 \right)
 &= \# \binom{\{1,\ldots,n-1\}}{i-1} + \# \binom{\{1,\ldots,n-1\}}{i}\\
 &= \binom{n-1}{i-1} + \binom{n-1}{i}.
\end{align*}
Řečeno selsky, když vezmu množinu obsahující všechny $i$-prvkové i
$(i-1)$-prvkové podmnožiny, pak její velikost je počet všech $i$-prvkových
podmnožin plus počet všech $(i-1)$-prvkových podmnožin. No shit.

Čili, abychom dokázali rovnost \eqref{eq:pascal-identity}, najdeme bijekci mezi
množinou všech $i$-prvkových a $(i-1)$-prvkových podmnožin $(n-1)$-prvkové
množiny a množinou všech  $i$-prvkových podmnožin $n$-prvkové množiny.

\begin{claim}[Pascalova rovnost]
 \label{claim:pascalova-rovnost}
 Ať $1 \leq i,n \in \N$ a $i \leq n$. Pak platí
 \[
  \binom{n}{i} = \binom{n-1}{i-1} + \binom{n-1}{i}.
 \]
\end{claim}
\begin{proof}
 Definujeme bijekci
 \[ 
  f: \binom{\{1,\ldots,n-1\}}{i-1} \cup \binom{\{1,\ldots,n\}}{i} \to
  \binom{\{1,\ldots,n\}}{i}.
 \]
 Ať nejprve $A \in \binom{\{1,\ldots,n-1\}}{i}$, čili $A$ je $i$-prvková
 podmnožina ${\{1,\ldots,n-1\}}$. Pak je $A$ též $i$-prvková podmnožina
 $\{1,\ldots,n\}$, neboli $A \in \binom{\{1,\ldots,n\}}{i}$ a stačí definovat
 $f(A) \coloneqq A$. Stručně řečeno, $f$ je identické zobrazení na $i$-prvkových
 podmnožinách.

 Teď ať $B \in \binom{\{1,\ldots,n-1\}}{i-1}$. Pak $B \cup \{n\}$ je $i$-prvková
 podmnožina $\{1,\ldots,n\}$ a tedy můžeme definovat $f(B) \coloneqq B \cup
 \{n\}$.

 Je zřejmé, že $f$ je bijekce. Když $A \subseteq \{1,\ldots,n\}$ neobsahuje $n$,
 pak je jejím vzorem při $f$ ta samá množina, tj. $A$. Když $A \subseteq
 \{1,\ldots,n\}$ obsahuje $n$, pak je jejím vzorem množina $A \setminus
 \{n\} \subseteq \{1,\ldots,n-1\}$.

 Tím je důkaz dokončen.
\end{proof}

\hyperref[claim:pascalova-rovnost]{Předchozí tvrzení} ukazuje, že pro $n$-tý
řádek Pascalova trojúhelníka oprav\-du platí rovnost
\[
 (p_1^{n},\ldots,p_n^{n}) = \left(
 \binom{n-1}{0},\binom{n-1}{1},\ldots,\binom{n-1}{n-1} \right).
\]

Na závěr celé sekce o kombinačních číslech si ukážeme ještě poslední snadno
dokazatelnou rovnost, která je však výpočetně též užitečná.

\begin{lemma}
 \label{lemma:stejne-doplnku}
 Ať $k,n \in \N$ a $k \leq n$. Pak
 \[
  \binom{n}{k} = \binom{n}{n-k}.
 \]
\end{lemma}
\begin{proof}
 Nalezneme bijekci
 \[
  \binom{\{1,\ldots,n\}}{k} \to \binom{\{1,\ldots,n\}}{n-k}.
 \]
 Uvědomme si, že když $A \subseteq \{1,\ldots,n\}$ a $\# A = k$, pak $\#
 (\{1,\ldots,n\} \setminus A) = n - k$. Kýžená bijekce je tudíž zobrazení $A
 \mapsto \{1,\ldots,n\} \setminus A$.
\end{proof}

Ještě několik úloh pro bystré hlavy.

\begin{exercise}
 Dokažte, že
 \[
  \sum_{i=0}^{n} \binom{n}{i}^2 = \binom{2n}{n}.
 \]
 \textbf{Hint}: použijte
 \hyperref[lemma:stejne-doplnku]{lemma~\ref*{lemma:stejne-doplnku}}.
\end{exercise}

\begin{exercise}
 Dokažte vzorec
 \[
  \sum_{k=r}^{n} \binom{k}{r} = \binom{n+1}{r+1}
 \]
 pro pevné $r \in \N$ indukcí podle $n \in \N$.
\end{exercise}

\begin{exercise}[těžké]
 Kolik existuje podmnožin $\{1,\ldots,n\}$, které neobsahují žádná dvě po sobě
 jdoucí čísla. Formálně, určete velikost množiny
 \[
  \{A \subseteq \{1,\ldots,n\} \mid \{i,j\} \nsubseteq A \text{ kdykoli }
  |i-j|=1\}.
 \]
\end{exercise}

\begin{exercise}[trocha teorie čísel]
 Ať $p$ je prvočíslo a $k,n$ přirozená čísla.
 \begin{enumerate}[label=(\alph*),topsep=0pt]
  \item Dokažte, že pro $k<p$ je $\binom{p}{k}$ dělitelné $p$.
  \item Dokažte, že $\binom{n}{p}$ je dělitelné $p$ právě tehdy, když $\left\lfloor n
   / p \right\rfloor$ je dělitelné $p$, kde $\left\lfloor  \cdot \right\rfloor$ 
   značí \emph{dolní celou část}.
 \end{enumerate}
\end{exercise}

\begin{exercise}
 Budeme vybírat $k$-tice předmětů z $n$ druhů předmětů. Budeme uvažovat různé
 typy výběru podle toho, jestli vybíráme $k$-tice uspořádané, nebo neuspořádané
 (tj. podmnožiny) a též podle toho, zda každého druhu je vždy jen jeden předmět,
 či nikoli. Doplňte následující tabulku:
 \begin{figure}[H]
  \centering
  \begin{tabular}{c|c|c}
   & Jen 1 předmět & Libovolně mnoho předmětů\\
   & každého druhu & každého druhu\\
   \midrule
   Uspořádané& &\\
   $k$-tice& &\\
   \midrule
   Neuspořádané & &\\
   $k$-tice& &
  \end{tabular}
 \end{figure}
\end{exercise}

\begin{exercise}[těžké]
 Kolika způsoby můžeme postavit $7$ čarodějnic a $5$ vodníků do řady tak, aby
 $2$ vodníci nikdy nestáli vedle sebe?
\end{exercise}
