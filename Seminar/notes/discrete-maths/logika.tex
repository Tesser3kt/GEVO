\subsection{Logické spojky a kvantifikátory}
\label{ssec:logicke-spojky-a-kvantifikatory}

\begin{definition}[Výrok]
 Výrokem nazveme jakoukoli větu, o které lze rozhodnout, zda je pravdivá, či
 nikoliv.
\end{definition}

\begin{example}
 Věty \uv{Je mi zle.} a \uv{Sumec je drůbež.} jsou výroky, zatímco \uv{Tvoje
 máma.} a \uv{Cos' dostala z matiky?} nikoliževěk.

 Je též dlužno mít na paměti, že naše znalost pravdivosti věty nemění nic na
 tom, jestli daná věta je, nebo není výrokem. Třeba \uv{Do pěti století
 kolonizujeme celou Sluneční soustavu.} je zcela jistě výrok.
\end{example}

Další text vyžaduje znalost operátorů $\neg, \wedge, \vee, \Rightarrow$ a
$\Leftrightarrow$. Je-li $x$ výrok \uv{Prší.} a $y$ výrok \uv{Vezmu si
deštník.}, pak
\begin{itemize}
 \item výrok $\neg x$ znamená \uv{\textbf{Ne}prší.},
 \item výrok $x \wedge y$ znamená \uv{Prší \textbf{a} vezmu si deštník.},
 \item výrok $x \vee y$ znamená \uv{Prší \textbf{nebo} si vezmu deštník.},
 \item výrok $x \Rightarrow y$ znamená \uv{\textbf{Když} prší, \textbf{tak} si
  vezmu deštník.} a
 \item výrok $x \Leftrightarrow y$ znamená \uv{Prší, \textbf{právě tehdy když}
  si vezmu deštník.}
\end{itemize}

\begin{warning}\hfill
 \vspace*{-\parskip}
 \begin{itemize}
  \item Logická spojka $ \vee $ \textbf{není výlučná}. Tedy $x \vee y$ platí
   v situaci, kdy
   \begin{itemize}
    \item platí pouze $x$,
    \item platí pouze $y$,
    \item platí $x$ i $y$.
   \end{itemize}
  \item Výrok $x \Rightarrow y$ je vždy \textbf{pravdivý}, pokud $x$ je
   \textbf{lživý}. Jinak řečeno, $x \Rightarrow y$ platí za situace, kdy
   \begin{itemize}
    \item platí $x$ i $y$,
    \item neplatí $x$ a platí $y$,
    \item neplatí $x$ a neplatí $y$.
   \end{itemize}
 \end{itemize}
\end{warning}

Jako znalost logických spojek je kritická i znalost kvantifikátorů $ \forall $ 
a $ \exists $, které se čtou \uv{pro všechny} a \uv{existuje}, resp.

Pokud je $p(x)$ výrok závislý na proměnné $x$ (třeba \uv{$x$ je sudé.}), pak
výrok
\begin{itemize}
 \item $ \forall x \in \N: p(x)$ zní \uv{Všechna přirozená čísla jsou sudá.} a
 \item $ \exists x \in \N: p(x)$ zní \uv{Existuje sudé přirozené číslo.}
\end{itemize}
Budeme rovněž užívat kvantifikátory $ \exists!$ a $\nexists$, které znamenají
\uv{existuje přesně jeden} a \uv{neexistuje}.

Podáno intutivně: chci-li tvrdit, že $ \forall x \in \N:p(x)$, musím dokázat,
že ať mi nepřítel dá \textbf{jakéḱoliv} přirozené číslo $x$, tak $p(x)$ platí.
Naopak, dokázat $ \exists x \in \N: p(x)$ je obvykle zásadně jednodušší, neboť
musím pouze najít \textbf{jedno} přirozené číslo $x$, pro které $p(x)$ platí.
