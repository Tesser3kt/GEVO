\subsubsection{Podgrafy, souvislost a metrika}
\label{sssec:podgrafy-souvislost-a-metrika}

V této relativně krátké podsekci dáme formální tvář představě, že
\begin{enumerate}
 \item nějaký graf je \uv{uvnitř} druhého;
 \item graf je souvislý a rozpadá se na tzv. \emph{komponenty souvislosti}, tedy
  maximální souvislé části;
 \item graf je \emph{metrický} prostor, tedy prostor, ve kterém lze měřit
  vzdálenosti.
\end{enumerate}

Začneme bodem (1), vedoucím na pojem \emph{podgrafu}. Jeho definice je opravdu
nejjednodušší možná, požadujeme pouze, aby vrcholy a hrany podgrafu tvořili
podmnožinu vrcholů a hran většího grafu.

\begin{definition}[Podgraf]
 \label{def:podgraf}
 Ať $G = (V,E)$ je graf. Řekneme, že graf $H = (V',E')$ je \emph{podgrafem} $G$,
 pokud
 \[
  V' \subseteq V \quad \text{a} \quad E' \subseteq E.
 \]
\end{definition}

\begin{remark}
 Žádáme čtenáře, aby sobě povšimli, že podgraf \textbf{nemusí zachovat} hranovou
 strukturu svého nadgrafu. Přesněji, \hyperref[def:podgraf]{definice podgrafu}
 neobsahuje podmínku, že mezi vrcholy $H$ musí vést hrana, pokud mezi těmi
 samými vrcholy v $G$ hrana vedla.

 Taková definice je z pohledu algebraika zhola zbytečná, bať odpudivá. Hranová
 struktura je zásadní součástí definice grafu a měla by být dodržena. Z tohoto
 důvodu se nehodí říkat, že by podgraf byl \emph{podstrukturou} svého nadgrafu,
 kterakžekolivěk vágně je ono slovo vyloženo.
\end{remark}

Předchozí poznámka motivuje definici \emph{indukovaného podgrafu}, podgrafu,
jemuž je přikázáno původní strukturu zachovat. Žargonový výraz \uv{indu\-kovaný}
v~tomto kontextu obyčejně znamená přibližně \uv{plynoucí z}. Čili, indukovaný
podgraf je podgraf, jehož struktura \textbf{plyne ze} struktury vyššího grafu.

Tuto podmínku lze formulovat snadno. Díváme-li se na hrany jako na podmnožiny
systému dvouprvkových podmnožin $V$, tedy jako na množinu $E \subseteq
\binom{V}{2}$, pak požadavek, aby nějaký podgraf $H = (V',E')$ grafu $G = (V,E)$
obsahoval spolu s každou dvojicí vrcholů i hranu mezi nimi, pokud je v~$E$, lze
vyjádřit zkrátka tak, že nařídíme, aby $E' = E \cap \binom{V'}{2}$, tedy aby
$E'$ byla vlastně množina $E$, ve které necháme jen ty hrany, které vedou mezi
vrcholy z $V'$.

\begin{definition}[Indukovaný podgraf]
 \label{def:indukovany-podgraf}
 Ať $G = (V,E)$ je graf a $H = (V',E')$ jeho podgraf. Řekneme, že $H$ je
 \emph{indukovaný} (grafem $G$), pokud \textbf{zachovává hranovou strukturu na
 $G$}, to jest,
 \[
  E' = E \cap \binom{V'}{2}.
 \]
\end{definition}

\begin{figure}[h]
 \centering
 \begin{subfigure}{.47\textwidth}
  \centering
  \begin{tikzpicture}
   \node[vertex] (v1) at (30:2) {};
   \node[vertex,myred,minimum size=11pt] (v2) at (90:2) {};
   \node[vertex,myred,minimum size=11pt] (v3) at (150:2) {};
   \node[vertex,myred,minimum size=11pt] (v4) at (210:2) {};
   \node[vertex,myred,minimum size=11pt] (v5) at (270:2) {};
   \node[vertex] (v6) at (330:2) {};

   \draw[thick] (v1) -- (v2);
   \draw[thick] (v2) -- (v3);
   \draw[ultra thick,myred] (v3) -- (v4);
   \draw[thick] (v4) -- (v5);
   \draw[thick] (v5) -- (v6);
   \draw[thick] (v6) -- (v1);
   
   \draw[thick] (v1) -- (v4);
   \draw[thick] (v3) -- (v6);
   \draw[ultra thick,myred] (v2) -- (v5);
  \end{tikzpicture}
  \caption{\clr{Podgraf}, který není indukovaný.}
  \label{subfig:neindukovany-podgraf}
 \end{subfigure}
 \begin{subfigure}{.47\textwidth}
  \centering
  \begin{tikzpicture}
   \node[vertex,myred,minimum size=11pt] (v1) at (30:2) {};
   \node[vertex] (v2) at (90:2) {};
   \node[vertex,myred,minimum size=11pt] (v3) at (150:2) {};
   \node[vertex,myred,minimum size=11pt] (v4) at (210:2) {};
   \node[vertex] (v5) at (270:2) {};
   \node[vertex,myred,minimum size=11pt] (v6) at (330:2) {};
   
   \draw[thick] (v1) -- (v2);
   \draw[thick] (v2) -- (v3);
   \draw[ultra thick,myred] (v3) -- (v4);
   \draw[thick] (v4) -- (v5);
   \draw[thick] (v5) -- (v6);
   \draw[ultra thick,myred] (v6) -- (v1);
   
   \draw[ultra thick,myred] (v1) -- (v4);
   \draw[thick] (v2) -- (v5);
   \draw[ultra thick,myred] (v3) -- (v6);
  \end{tikzpicture}
  \caption{\clr{Indukovaný} podgraf.}
  \label{subfig:indukovany-podgraf}
 \end{subfigure}
 \caption{Rozdíl mezi podgrafem a \emph{indukovaným} podgrafem.}
 \label{fig:indukovany-podgraf}
\end{figure}
