\subsection{Jordanovo centrum}
\label{ssec:jordanovo-centrum}

V návaznosti na \hyperref[sssec:minimalni-kostra]{sekci o minimální kostře} se
rozhovoříme o jednom dalším optimalizačním problému -- konkrétně hledání
\uv{centra} ohodnoceného grafu.

Motivační úlohou je tzv. \emph{facility location problem}, v přibližném překladu
\emph{úloha umístění střediska}. Jde o úlohu, kdy máte dánu dopravní síť sídlišť
(obecně obydlených zón) a význačných uzlů, přes které se chtě nechtě musí jezdit
(například velké křižovatky, Nuselák apod.). Hrany vedou mezi sídlišti či uzly,
když od jednoho k druhému vede bezprostřední cesta (tedy cesta neprocházející
žádným jiným sídlištěm nebo uzlem).

Nejprve si však rozmyslíme, že onen \emph{facility location problem}, odteď
FLP, má v závislosti na typu střediska dvě různá znění. V branži se jim občas
říká \emph{\textbf{emergency} facility location problem} (EFLP), který zahrnuje
určení vhodné polohy například právě nemocnic a poliklinik, a
\emph{\textbf{service} facility location problem} (SFLP), sestávající z úloh
nalezení nejdostupnějšího místa pro školy či obchodní centra, kde není nezbytně
nutné, aby se z \textbf{každé} pokryté oblasti dalo ke středisku dostat
efektivně (bo nejde o život), ale aby naopak průměr délek cest od všech
obydlených oblastí ke středisku byl co nejmenší.

K formulaci obou úloh ale potřebujeme zavést základní pojem \emph{vzdálenosti}
mezi vrcholy v ohodnoceném grafu. Pro podrobnější výklad o cestách, souvislosti
grafu., vzdálenosti a jejich relevanci k pojmu \emph{metriky} na množině vizte
\myref{dodatek}{ssec:souvislost-grafu-a-metrika}.

V zájmu strohosti vyjádření označíme pro libovolné dva vrcholy $u,v \in V$
grafu $(V,E,w)$ symbolem $\mathcal{P}(u,v)$ množinu všech cest mezi $u$ a $v$.
Speciálně, $\mathcal{P}(v,v) = \{v\}$, čili cesta z vrcholu do něj samého
obsahuje pouze tento jeden vrchol, a $\mathcal{P}(u,v) = \emptyset$, pokud mezi
$u$ a $v$ nevede v $G$ cesta.

\begin{definition}[Vzdálenost v grafu]
\label{def:vzdalenost-v-grafu}
 Ať $G = (V,E,w)$ je ohodnocený graf a $v,w \in V$. \emph{Vzdáleností} mezi $u$
 a $v$ v grafu $G$, značenou $d_G(u,v)$ (z angl. \textbf{d}istance) definujeme
 jako
 \[
  d_G(u,v) \coloneqq 
  \begin{cases}
   \min_{\mathcal{P} \in \mathcal{P}(u,v)} w(\mathcal{P}),& \text{pokud }
   \mathcal{P}(u,v) \neq \emptyset;\\
   \infty, &\text{pokud } \mathcal{P}(u,v) = \emptyset.
  \end{cases}
 \]
 Lidsky řečeno, vzdáleností mezi vrcholy je váha nejkratší cesty mezi nimi
 vedoucí, pokud taková existuje.
\end{definition}

Abychom našli pro daný graf $G = (V,E,w)$ řešení EFLP, musíme najít takový
vrchol, který minimalizuje největší možnou vzdálenost od něj ke všem ostatní
vrcholům $G$. Takový vrchol (nebo vrchol\textbf{y}?) nazveme \emph{Jordanovým
centrem} grafu $G$, po žabožroutím počtáři, Marie E. C. Jordanovi.

Formálně, \emph{excentricita} vrcholu $v \in V$ je kvantita $e(v) \coloneqq
max_{u \in V} d_G(v,u)$, tedy maximum přes všechny vzdálenosti od něj k
ostatním vrcholům. Toto číslo vyjadřuje, jak moc je vrchol vzdálen od
\uv{ideálního centra} grafu, tedy od bodu, od kterého by každý vrchol byl
stejně daleko. Samozřejmě, toto ideální centrum málokdy existuje, takže hledáme
pouze vrchol s nejmenším excentricitou, s nejmenší \emph{odchylkou} od centra.

\begin{definition}[Emergency Facility Location Problem]
\label{def:eflp}
 Ať $G = (V,E,w)$ je \textbf{souvislý} ohodnocený graf. Úlohu nalézt vrchol s
 minimální excentricitou nazveme EFLP. Jejím \emph{řešením} je vrchol s touto
 vlastností, tedy vrchol $c \in V$ splňující
 \[
  e(c) = \min_{v \in V} e(v).
 \]
\end{definition}

\begin{warning}
 Řešení EFLP \textbf{není jednoznačně určeno}! Vizte např. graf na
 \myref{obrázku}{fig:minimalni-excentricita}.

 \begin{figure}[H]
 \centering
  \begin{tikzpicture}
   \tikzset{vertex/.style = {shape=circle,fill,text=white,minimum size=6pt,inner
   sep=1pt}}
   \tikzset{->-/.style={decoration={ markings, mark=at position #1 with
   {\arrow{>[scale=1]}}},postaction={decorate}}}

   \draw[thick] (-1,0) to node[above right=-2pt and -2pt] {$\clb{2}$} (-2, 1);
   \draw[thick] (-1,0) to node[below right=-2pt and -2pt] {$\clb{2}$} (-2, -1);
   \draw[thick] (-1,0) to node[above] {$\clb{1}$} (1, 0);
   \draw[thick] (1,0) to node[above left=-2pt and -2pt] {$\clb{2}$} (2, 1);
   \draw[thick] (1,0) to node[below left=-2pt and -2pt] {$\clb{2}$} (2, -1);

   \node[vertex,minimum size=9pt,myred] at (-1, 0) {};
   \node[vertex] at (-2, 1) {};
   \node[vertex] at (-2, -1) {};
   \node[vertex,minimum size=9pt,myred] at (1, 0) {};
   \node[vertex] at (2, 1) {};
   \node[vertex] at (2, -1) {};

  \end{tikzpicture}
  \caption{\clr{Vrcholy s minimální excentricitou} v grafu $G =
  (V,E,\clb{w})$.}
  \label{fig:minimalni-excentricita}
 \end{figure}
\end{warning}

\begin{definition}[Jordanovo centrum]
\label{def:jordanovo-centrum}
 Množinu všech řešení EFLP pro graf $G = (V,E,w)$ nazýváme \emph{Jordanovým
 centrem} grafu $G$.
\end{definition}

\begin{definition}[Poloměr grafu]
\label{def:polomer-grafu}
 Je-li $c$ vrchol v Jordanově centru grafu $G = (V,E,w)$, pak hodnotu $e(c)$
 nazýváme \emph{poloměrem} grafu $G$ a značíme ji $\rho(G)$.
\end{definition}

\begin{remark}
 V \hyperref[def:eflp]{definici EFLP} jsme požadovali, aby byl graf souvislý.
 To z ryze technického hlediska není nutné, protože excentricita vrcholu je
 definována i pro nesouvislý graf. Uvědomme si ale, že pro nesouvislý graf je
 excentricita každého vrcholu rovna $\infty$, tedy Jordanovým centrem je celý
 graf a úloha poněkud pozbývá smyslu.
\end{remark}

Obdobným způsobem si formalizujeme i SFLP a později si ukážeme souvislost obou
úloh s minimální kostrou.


