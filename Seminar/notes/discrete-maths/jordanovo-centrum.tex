\subsection{Jordanovo centrum}
\label{ssec:jordanovo-centrum}

V návaznosti na \hyperref[sssec:minimalni-kostra]{sekci o minimální kostře} se
rozhovoříme o jednom dalším optimalizačním problému -- konkrétně hledání
\uv{centra} ohodnoceného grafu.

Motivační úlohou je tzv. \emph{facility location problem}, v přibližném překladu
\emph{úloha umístění střediska}. Jde o úlohu, kdy máte dánu dopravní síť sídlišť
(obecně obydlených zón) a význačných uzlů, přes které se chtě nechtě musí jezdit
(například velké křižovatky, Nuselák apod.). Hrany vedou mezi sídlišti či uzly,
když od jednoho k druhému vede bezprostřední cesta (tedy cesta neprocházející
žádným jiným sídlištěm nebo uzlem).

Pojďme si úlohu rozmyslet podrobně. Máme u nějakého uzlu v síti dopravních
křižovatek postavit středisko. Bez ohledu na typ střediska nás pravděpodobně
zajímá, aby se všichni z obydlených zón, které má toto středisko pokrýt, dostali
k němu co nejsnáze. Ovšem, výraz \uv{co nejsnáze} je matematicky příliš vágní.
Jistě hledáme optimální řešení, ale v jakém smyslu konkrétně?

Snad bude užitečné kýžený smysl optimality vyzkoumat na příkladě. Řekněme, že
střediskem, jež potřebujeme umístit, je nemocnice. Co znamená, že je vzhledem k
dané síti nemocnice \emph{optimálně} umístěna? První, co by nás mohlo napadnout,
je chtít, aby průměr vzdáleností od obydlených zón k~nemocnici byl co nejmenší.
To je přirozená myšlenka, ale jak si ihned rozmyslíme, vcelku morbidní.

U průměru totiž často nastává situace, že značný počet nízkých hodnot převáží
nad zanedbatelným počtem hodnot vysokých. Uvažte síť uzlů a sídlišť danou grafem
na \myref{obrázku}{fig:spatne-vyvazena-sit}, kde sídliště jsou značena
\clb{modře}, dopravní uzly \clr{červeně} a ohodnocení hran, jak je vlastně
zvykem, značí průměrnou dobu jízdy po těchto cestách.

\begin{figure}[h]
\centering
 \begin{tikzpicture}
  \tikzset{vertex/.style = {shape=circle,fill,text=white,minimum size=6pt,inner
  sep=1pt}}
  \tikzset{->-/.style={decoration={ markings, mark=at position #1 with
  {\arrow{>[scale=1]}}},postaction={decorate}}}

  \foreach \y in {2, 1, ..., -2} {
   \draw[thick] (2, 0) to node[pos=0.7, draw, circle, fill=white, inner
    sep=0.5pt] {\footnotesize $1$} (3.5, \y);
  }
  \foreach \y in {1, -1} {
   \draw[thick] (-2, 0) to node[pos=0.6, draw, circle, fill=white, inner
    sep=0.5pt] {\footnotesize $7$} (-4, \y);
  }
  \draw[thick] (-2, 0) to node[midway, draw, circle, fill=white, inner
   sep=0.5pt] {\footnotesize $10$} (2, 0);

  \node[vertex, myred, minimum size=9pt] (v2) at (2, 0) {};
  \node[above left = -1mm and -1mm of v2, myred] {$v_2$};
  \node[vertex, myred, minimum size=9pt] (v1) at (-2, 0) {};
  \node[above right = -1mm and -1mm of v1, myred] {$v_1$};

  \foreach \y in {2, 1, ..., -2} {
   \node[vertex, myblue] at (3.5, \y) {};
  }
  \foreach \y in {1, -1} {
   \node[vertex, myblue] at (-4, \y) {};
  }
 \end{tikzpicture}
 \caption{Špatně vyvážená síť \clb{sídlišť} a \clr{dopravních uzlů}.}
 \label{fig:spatne-vyvazena-sit}
\end{figure}

V tomto případě by uzel $\clr{v_2}$ jistě nabízel mnohem lepší průměrné řešení,
neboť můžeme snadno spočítat, že průměrná doba jízdy od libovolného sídliště k
němu je asi pět a půl minuty. Naopak, průměrná doba jízdy k~uzlu $\clr{v_1}$
činí těsně pod deset minut.

My se ale přesto rozhodneme postavit nemocnici v uzlu $\clr{v_1}$. Proč? Totiž,
při stavbě nemocnice nám nejde ani tolik o to, aby se do ní dostalo co nejvíce
lidí co nejrychleji, ale \textbf{aby to nikdo neměl příliš daleko}. Kdybychom
učinili opak a postavili nemocnici v uzlu $\clr{v_2}$, pak by sice lidé ze
sídlišť na pravé straně to měli do nejbližší nemocnice pouhou minutu, ale lidé
z~levých sídlišť by cestovali průměrně až sedmnáct minut. Do uzlu $\clr{v_1}$ se
dostane každý člověk nejpozději za jedenáct minut.

Tento způsob měření vhodnosti umístění nemocnice v dopravních sítích je skutečně
v praxi používaný a je dozajista nepěkným příkladem volby menšího ze dvou zel.
Tedy, není prioritou, aby se někomu dostalo pomoci velmi brzy, ale aby se nikomu
nedostalo pomoci příliš pozdě.

Problém, jenž jsme právě zformulovali, je variantou výše zmíněného
\emph{facility location problem} (dále jen FLP), která sluje EFLP, tedy
\emph{\textbf{emergency} facility location problem}. Úlohou je nalézt takový
uzel, jehož \textbf{maximální} vzdálenost ke všem ostatním uzlům je minimální,
čili uzel pro umístění středisek jako jsou právě nemocnice a polikliniky, či
obecněji \uv{pohotovostní} střediska.

Druhou známou variantou je SFLP, čili \emph{\textbf{service} facility location
problem} -- úloha nalézt vhodný uzel pro umístění střediska \uv{služeb}, kde je
naopak žádoucí, aby nastala druhá z výše diskutovaných situací, aby se k němu
přiblížilo co nejvíce lidí co nejrychleji. Ti, již cestují dlouho, budou na
nejméně libý pád kalé nálady sedíce rozčileně ve voze a shrbení klejíce
ustavičně v kolena, však nezhynou.

Formálně tedy hledáme uzel, který minimalizuje průměr všech vzdáleností od něj
ke všem ostatním uzlům. Pro usnadnění výpočtu je dobré si uvědomit, že
minimalizovat \textbf{průměr} všech vzdáleností je totéž, co minimalizovat
\textbf{součet} všech vzdáleností, neboť počet uzlů se nemění
(\textbf{Rozmys\-lete si to!}).

K formulaci obou úloh potřebujeme zavést základní pojem \emph{vzdálenosti}
me\-zi vrcholy v ohodnoceném grafu.

% TODO
% Pro podrobnější výklad o cestách, souvislosti
% grafu, vzdálenosti a jejich relevanci k pojmu \emph{metriky} na množině vizte
% \myref{dodatek}{ssec:souvislost-grafu-a-metrika}.

V zájmu strohosti vyjádření označíme pro libovolné dva vrcholy $u,v \in V$
grafu $(V,E,w)$ symbolem $\mathcal{P}(u,v)$ množinu všech cest mezi $u$ a $v$.
Speciálně, $\mathcal{P}(v,v) = \{v\}$, čili cesta z vrcholu do něj samého
obsahuje pouze tento jeden vrchol, a $\mathcal{P}(u,v) = \emptyset$, pokud mezi
$u$ a $v$ nevede v $G$ cesta.

\begin{definition}[Vzdálenost v grafu]
\label{def:vzdalenost-v-grafu}
 Ať $G = (V,E,w)$ je ohodnocený graf a $v,w \in V$. \emph{Vzdálenost} mezi $u$
 a $v$ v grafu $G$, značenou $d_G(u,v)$ (z angl. \textbf{d}istance), definujeme
 jako
 \[
  d_G(u,v) \coloneqq 
  \begin{cases}
   \min_{\mathcal{P} \in \mathcal{P}(u,v)} w(\mathcal{P}),& \text{pokud }
   \mathcal{P}(u,v) \neq \emptyset;\\
   \infty, &\text{pokud } \mathcal{P}(u,v) = \emptyset.
  \end{cases}
 \]
 Lidsky řečeno, vzdáleností mezi vrcholy je váha nejkratší cesty mezi nimi
 vedoucí, pokud taková existuje.
\end{definition}

\begin{figure}[h]
 \centering
 \begin{tikzpicture}
  \tikzset{vertex/.style = {shape=circle,fill,text=white,minimum size=6pt,inner
  sep=1pt}}
  \tikzset{->-/.style={decoration={ markings, mark=at position #1 with
  {\arrow{>[scale=1]}}},postaction={decorate}}}
  \draw[thick,myblue] (-3, 0) to node[pos=0.3,above=1mm] {$2$} (-2, 1);
  \draw[thick,myblue] (-2, 1) to node[pos=0.45,above=0mm] {$4$} (0, 1.5);
  \draw[thick,myblue] (0, 1.5) to node[pos=0.6,above=0.5mm] {$1$} (1.5, 0.5);
  \draw[thick,myblue] (1.5, 0.5) to node[pos=0.5,above=0mm] {$1$} (3, 0);

  \draw[thick,myred] (-3, 0) to node[pos=0.5,above=0mm] {$4$} (-1,0);
  \draw[thick,myred] (-1, 0) to node[pos=0.5,above=0mm] {$7$} (3,0);

  \draw[thick,mygreen] (-3,0) to node[pos=0.4,below=0mm] {$2$} (-1.5,-1);
  \draw[thick,mygreen] (-1.5,-1) to node[pos=0.45,below=0mm] {$3$} (0.5,-2);
  \draw[thick,mygreen] (0.5,-2) to node[pos=0.3,above=1mm] {$2$} (1.5,-0.5);
  \draw[thick,mygreen] (1.5,-0.5) to node[pos=0.85,above=1.5mm] {$1$} (2,-1.5);
  \draw[thick,mygreen] (2,-1.5) to node[pos=0.7,below=1mm] {$2$} (3,0);

  \node[vertex,mypurple,minimum size=9pt] (u) at (-3, 0) {};
  \node[above left= -1mm and -1mm of u] {\textcolor{mypurple}{$u$}};
  \node[vertex,mypurple,minimum size=9pt] (v) at (3, 0) {};
  \node[above right= -1mm and -1mm of v] {\textcolor{mypurple}{$v$}};

  \node[vertex] at (-2, 1) {};
  \node[vertex] at (0, 1.5) {};
  \node[vertex] at (1.5, 0.5) {};

  \node[vertex] at (-1, 0) {};

  \node[vertex] at (-1.5, -1) {};
  \node[vertex] at (0.5, -2) {};
  \node[vertex] at (1.5, -0.5) {};
  \node[vertex] at (2, -1.5) {};
 \end{tikzpicture}
 \caption{Zde $d_G(\textcolor{mypurple}{u},\textcolor{mypurple}{v})$ je váha 
 nejkratší, to jest \clb{modré}, cesty.}
 \label{fig:vzdalenost-vrcholu}
\end{figure}

Abychom našli pro daný graf $G = (V,E,w)$ řešení EFLP, musíme najít takový
vrchol, který minimalizuje největší možnou vzdálenost od něj ke všem ostatní
vrcholům $G$. Takový vrchol (nebo vrchol\textbf{y}?) nazveme \emph{Jordanovým
centrem} grafu $G$, po žabožroutím počtáři, Marie E. C. Jordanovi.

Samozřejmě, nezajímá-li nás vzdálenost ke \textbf{všem} vrcholům, jako je tomu v
příkladě umístění nemocnice, uvážíme pouze maximum vzdáleností k~relevantním
vrcholům (tedy k sídlištím). Na principu úlohy to nic nemění.

Formálně, \emph{excentricita} vrcholu $v \in V$ je kvantita $e(v) \coloneqq
max_{u \in V} d_G(v,u)$, tedy maximum přes všechny vzdálenosti od něj k
ostatním vrcholům. Toto číslo vyjadřuje, jak moc je vrchol vzdálen od
\uv{ideálního centra} grafu, tedy od bodu, od kterého by každý vrchol byl
stejně daleko. Samozřejmě, toto ideální centrum málokdy existuje, takže hledáme
pouze vrchol s nej\-menší excentricitou, s nejmenší \emph{odchylkou} od centra.

\begin{definition}[Emergency Facility Location Problem]
\label{def:eflp}
 Ať $G = (V,E,w)$ je \textbf{souvislý} ohodnocený graf. Úlohu nalézt vrchol s
 minimální excentricitou nazveme EFLP. Jejím \emph{řešením} je vrchol s touto
 vlastností, tedy vrchol $c \in V$ splňující
 \[
  e(c) = \min_{v \in V} e(v).
 \]
\end{definition}

\begin{warning}
 Řešení EFLP \textbf{není jednoznačně určeno}! Vizte např. graf na
 \myref{obrázku}{fig:minimalni-excentricita}.

 \begin{figure}[H]
 \centering
  \begin{tikzpicture}
   \tikzset{vertex/.style = {shape=circle,fill,text=white,minimum size=6pt,inner
   sep=1pt}}
   \tikzset{->-/.style={decoration={ markings, mark=at position #1 with
   {\arrow{>[scale=1]}}},postaction={decorate}}}

   \draw[thick] (-1,0) to node[above right=-2pt and -2pt] {$\clb{2}$} (-2, 1);
   \draw[thick] (-1,0) to node[below right=-2pt and -2pt] {$\clb{2}$} (-2, -1);
   \draw[thick] (-1,0) to node[above] {$\clb{1}$} (1, 0);
   \draw[thick] (1,0) to node[above left=-2pt and -2pt] {$\clb{2}$} (2, 1);
   \draw[thick] (1,0) to node[below left=-2pt and -2pt] {$\clb{2}$} (2, -1);

   \node[vertex,minimum size=9pt,myred] at (-1, 0) {};
   \node[vertex] at (-2, 1) {};
   \node[vertex] at (-2, -1) {};
   \node[vertex,minimum size=9pt,myred] at (1, 0) {};
   \node[vertex] at (2, 1) {};
   \node[vertex] at (2, -1) {};

  \end{tikzpicture}
  \caption{\clr{Vrcholy s minimální excentricitou} v grafu $G =
  (V,E,\clb{w})$.}
  \label{fig:minimalni-excentricita}
 \end{figure}
\end{warning}

\begin{definition}[Jordanovo centrum]
\label{def:jordanovo-centrum}
 Množinu všech řešení EFLP pro graf $G = (V,E,w)$ nazýváme \emph{Jordanovým
 centrem} grafu $G$.
\end{definition}

\begin{definition}[Poloměr grafu]
\label{def:polomer-grafu}
 Je-li $c$ vrchol v Jordanově centru grafu $G = (V,E,w)$, pak hodnotu $e(c)$
 nazýváme \emph{poloměrem} grafu $G$ a značíme ji $\rho(G)$.
\end{definition}

\begin{remark}
 V \hyperref[def:eflp]{definici EFLP} jsme požadovali, aby byl graf souvislý.
 To z ryze technického hlediska není nutné, protože excentricita vrcholu je
 definována i pro nesouvislý graf. Uvědomme si ale, že pro nesouvislý graf je
 excentricita každého vrcholu rovna $\infty$, tedy Jordanovým centrem je celý
 graf a úloha poněkud pozbývá smyslu.
\end{remark}

Obdobným způsobem si formalizujeme i SFLP.

Nyní chceme nalézt vrchol, který minimalizuje průměr (nebo ekvivalent\-ně
součet) vzdáleností od něj k, buď všem nebo pouze zajímavým, vrcholům.
Názvosloví zde poněkud selhává a tomuto součtu přes vzdálenosti ke všem vrcholům
se rovněž často přezdívá \emph{excentricita}. Abychom pojmy odlišili, slovo
\uv{excentricita} přeložíme a budeme říkat \uv{výstřednost}. Tedy,
\emph{výstředností} vrcholu $v \in V$ v ohodnoceném grafu $G = (V,E,w)$ myslíme
číslo
\[
 e'(v) \coloneqq \sum_{u \in V} d_G(v,u).
\]

\begin{definition}[Service Facility Location Problem]
 \label{def:sflp}
 Ať $G = (V,E,w)$ je souvislý ohodnocený graf. Úlohu nalézt vrchol s minimální
 výstředností nazveme SFLP. Jejím řešením je vrchol $c' \in V$ s touto
 vlastností, tedy takový, že
 \[
  e'(c') = \min_{v \in V} e'(v).
 \]
\end{definition}

Pochopitelně, stejně jako EFLP, i SFLP může mít více řešení. Avšak, podle našeho
nejlepšího vědomí se množině řešení SFLP nijak význačně neříká. Minimální
výstřednost se občas nazývá \emph{status} grafu $G$. Etymologie tohoto
názvosloví je spjata s novodobým využitím SFLP při vyvažování nervových sítí a
její zevrubné objasnění je nad rámec tohoto textu.

\begin{definition}[Status grafu]
 Ať $G = (V,E,w)$ je souvislý ohodnocený graf. Když $c' \in V$ je řešení SFLP,
 pak se hodnota $e'(c)$ nazývá \emph{status} grafu $G$ a značí se $\sigma(G)$.
\end{definition}

\begin{warning}
 Řešení EFLP a řešení SFLP mohou (ale \textbf{nemusí}) být disjunktní.
 Vrátíme-li se ke grafu na \myref{obrázku}{fig:spatne-vyvazena-sit}, pak řešením
 EFLP je pouze vrchol $\clr{v_1}$, zatímco řešením SFLP je pouze vrchol
 $\clr{v_2}$. 
\end{warning}

Následující podsekci dedikujeme obecnému algoritmu, pomocí nějž lze také hledat
řešení jak EFLP, tak SFLP. Poznamenáme však, že existují mnohem efektivnější
algoritmy, jež naleznou řešení těchto úloh; tyto ale vyžadují o poznání hlubší
poznatky teorie grafů.

\subsubsection{Floydův-Warshallův algoritmus}
\label{sssec:floyduv-warshalluv-algoritmus}

Vlastně jedinou výzvou při na cestě vyřešení FLP je umět aspoň rámcově efektivně
najít vzdálenost dvou vrcholů. Bohužel, žádná pěkná věta jako Pythagorova, která
umožňuje okamžitě počítat vzdálenosti bodů v Eukleidovských prostorech, v teorii
grafů neexistuje a existovat nemůže.

Připomeňme, že v ohodnoceném grafu $G = (V,E,w)$ je vzdálenost vrcholů $u,v \in
V$ \hyperref[def:vzdalenost-v-grafu]{definována} jako váha nejkratší cesty.
Poznamenejme, že zde slovo \uv{nejkratší} chápeme ve smyslu \emph{váhy} cesty
(tedy součtu vah všech jejích hran), nikoli ve smyslu \emph{délky} cesty (tedy
počtu jejích hran). Asi bychom měli říkat \uv{nejlehčí} cesta, to je ale poněkud
nepřirozené...

Floydův-Warshallův algoritmus nesouvisí přímo s FLP. Je to algoritmus, který
nalezne vzdálenost (tedy váhu nejkratší cesty) mezi všemi dvojicemi vrcholů. Je
však zřejmé, jak znalost této informace vede okamžitě k řešení EFLP, resp. SFLP.
V moment, kdy známe vzdálenost každého vrcholu od každého, stačí pouze spočítat
excentricitu, resp. výstřednost, každého vrcholu a vybrat pouze ty s minimální.

Jistě není překvapením, že zkoušet z každého vrcholu všechny možné cesty do
všech ostatních vrcholů a z nich vybírat ty nejkratší není dvakrát efektivní.
Floydův-Warshallův algoritmus stojí na dvou principech, jimž se především v
programování říká \emph{rekurze} a \emph{dynamické programování}. Jejich úplné
pochopení a nabytí schopnosti využívat může být časově náročné, ale
Floydův-Warshallův algoritmus jich využívá velmi přímočaře. Postupně si
rozebereme, že ze znalosti váhy nejkratší cesty mezi dvěma vrcholy,
\textbf{která využívá jen nějakou podmnožinu ostatních vrcholů}, lze zvětšováním
této podmnožiny získat nakonec váhu nejkratší cesty mezi těmito vrcholy v celém
grafu (odtud \emph{rekurze}). Dále, ze znalosti vzdálenosti mezi určitými
dvojicemi vrcholů můžeme rychle určit vzdálenost mezi párem, u kterého jsme ji
zatím neznali (odtud \emph{dynamické programování}).

Naší prací v této podsekci je dát předchozímu odstavci formální podobu. Ať $G =
(V,E,w)$ je souvislý ohodnocený graf, kde $V = \{v_1,\ldots,v_n\}$. Definujme
funkci $\delta(i,j,k): \{1,\ldots,n\}^3 \to \R^+$ následovně. Ať $\delta(i,j,k)$
je váha nejkratší cesty mezi $v_i$ a $v_j$ \textbf{využívající pouze vrcholy z
podmnožiny vrcholů} $\{v_1,\ldots,v_k\}$ (samozřejmě, kromě počátečního $v_i$ a
koncového $v_j$). I když je $G$ souvislý, tak taková cesta nemusí vždy
existovat, v takovém případě je $\delta(i,j,k) = \infty$.

Je zřejmé, že $\delta(i,j,k) \leq \delta(i,j,k-1)$, neboť máme jeden vrchol
navíc, a přes ten může vést nějaká kratší cesta. Základní, a vlastně jedinou,
myšlenkou Floydova-Warshallova algoritmu je pozorování, že když nastane situace,
kdy ${\delta(i,j,k) < \delta(i,j,k-1)}$, pak ta kratší cesta musí využívat
vrchol $v_k$. Ovšem, tuto cestu lze v takovém případě rozdělit na dvě -- na
cestu z $v_i$ do $v_k$ a na cestu $v_k$ do $v_j$. Původní cesta $v_i\cdots v_k
\cdots v_j$ využívala pouze vrcholy z $\{v_1,\ldots,v_k\}$, takže obě její
části, $v_1 \cdots v_k$ i $v_k \cdots v_j$ využívají pouze vrcholy
z~$\{v_1,\ldots,v_{k-1}\}$. Zároveň to musejí být právě ty nejkratší cesty mezi
$v_i$ a $v_k$ a $v_k$ a $v_j$, jinak by celková cesta $v_1 \cdots v_k \cdots
v_j$ nebyla ta nejkratší.

