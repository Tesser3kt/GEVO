\section{Teorie grafů}
\label{sec:teorie-grafu}

Velkou část moderní matematiky (zcela jistě topologii, geometrii i algebru)
tvoří studium \uv{struktur}. Toto obecně nedefinované slovo obvykle značí
množinu s nějakou další informací o vztahu mezi jejími prvky -- tím obvykle bývá
operace nebo třeba, jako v případě grafů, relace.

Tato kapitola zároveň značí jakýsi milník ve vývoji matematického myšlení,
především algebraickým směrem. Můžeme se totiž začít bavit o speciálních
zobrazeních, které zachovávají strukturu na množinách, mezi kterými vedou, tzv.
\emph{homomorfismech}; pochopit, že je dobré mít více popisů stejné struktury
ekvivalentních v tom smyslu, že poskytují stejné množství informací, přestože se
o žádné bijekci nedá formálně hovořit; uvidět, že je užitečné dva různé grafy
(či obecně dvě různé struktury) považovat za stejné, když se liší pouze
zanedbatelně.

Jednou, avšak zdaleka ne \emph{jedinou}, motivací pro teorii grafů je schopnost
analyzovat struktury tvořené množinou \uv{uzlů}, mezi některýmiž vedou
\uv{spojnice}. Takováto struktura úspěšně modeluje až neuvěřitelné množství
přírodních i společenských úkazů. Mezi nimi jmenujmež
\begin{itemize}
 \item návrhy elektrických obvodů, kde uzly jsou elektrická zařízení a spojnice
  jsou kabely mezi nimi vedoucí;
 \item lingvistické modely, kde uzly jsou slova a spojnice vede mezi
  těmi syntakticky souvisejícími;
 \item studium molekul, kde uzly jsou atomy a spojnice vazby mezi nimi;
 \item 
\end{itemize}
