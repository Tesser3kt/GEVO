\subsubsection{Rovinné grafy}
\label{sssec:rovinne-grafy}

Přívlastek \uv{rovinný} jsme v kontextu kreslení grafů zmínili již několikrát a
neformálně jsme uvedli, že grafu řekneme rovinný, když se dá nakreslit bez
křížení hran. Formulace \uv{dá nakreslit} je zde zásadního významu. Libovolný
graf, který má aspoň dvě hrany, lze vždy nakreslit tak, aby se tyto křížily. Nás
však zajímá pouze, zda existuje \textbf{nějaké} nakreslení takového grafu, ve
kterém se žádné hrany nekříží.

Obecně, minimální nutný počet křížení hran v nakreslení grafu je bohatě nejvíce
studovaná vlastnost z celé oblasti kreslení grafů. Definice křížení (nebo
absence téhož) v nakreslení grafu není výrazně odlišná od té intuitivní.
Zkrátka, řekneme, že dvě hrany se nekříží, když v průniku jejich obrazů leží
pouze nakreslené vrcholy (tam se samozřejmě hrany křížit mohou a některé
musejí).

\begin{definition}[Rovinný graf]
 \label{def:rovinny-graf}
 Graf $G = (V,E)$ nazveme \emph{rovinným}, když existuje jeho nakreslení $(p,c)$
 takové, že pro každé dvě různé hrany $e,e' \in E$ platí
 \[
  \img \gamma_e \cap \img \gamma_{e'} \subseteq \img p.
 \]
 Každé nakreslení rovinného grafu, které splňuje tuto podmínku, nazveme též
 \emph{rovinným}.
\end{definition}

Ve zbytku kapitoly nás čekají jeden snadný a dva těžké úkoly.
\begin{enumerate}
 \item Definovat několik základních operací na vrcholech a hranách grafu, které
  přijdou vhod v dalším textu.
 \item Zformulovat a dokázat několik základních vlastností rovinných gra\-fů.
 \item Dokázat Kuratowskiho větu, která dokonale klasifikuje rovinné gra\-fy,
  tj. poskytuje konkrétní přímočaré kritérium, podle nějž lze poznat, zda je
  graf rovinný, či nikoliv.
\end{enumerate}

Začněme bodem (1). Budeme potřebovat zavést operace přidání a odebrání vrcholů a
hran a konečně operaci dělení hrany. Význam prvních čtyř je snad jasný z názvu,
pátá operace vyžaduje vlastně přidání vrcholu doprostřed nějaké hrany. Dá se
pomocí prvních čtyř vyjádřit jako
\begin{itemize}
 \item odebrání dlouhé hrany,
 \item přidání vrcholu,
 \item přidání dvou hran vedoucích z tohoto vrcholu do počátečního a koncového
  vrcholu původní odebrané hrany.
\end{itemize}

I když intuitivně se operace pojímají snadno, formální definice není naprosto
přímočará. Spěšně si rozmyslíme, jak se dají formalizovat.

\begin{definition}[Základní grafové operace]
 \label{def:zakladni-grafove-operace}
 Ať $G = (V,E)$ je graf. Definujeme následující operace na množinách $V$ a $E$:
 \begin{itemize}
  \item operace přidání vrcholu, zapisujeme
   \[
    G + v \coloneqq (V \cup \{v\}, E).
   \]
  \item operace odebrání vrcholu (zde je třeba \textbf{odebrat i všechny hrany,
   které do tohoto vrcholu vedou}), pro $v \in V$ zapisujeme
   \[
    G - v \coloneqq (V \setminus \{v\}, E \setminus \{e \in E \mid v \in e\}).
   \]
  \item operace přidání hrany, pro $u,v \in V$ takové, že $uv \notin E$,
   zapisujeme
   \[
    G + uv \coloneqq (V, E \cup \{uv\}).
   \]
  \item operace odebrání hrany, pro $uv \in E$ zapisujeme
   \[
    G - uv \coloneqq (V, E \setminus \{uv\}).
   \]
  \item operace dělení hrany, pro $uv \in E$ zapisujeme
   \begin{equation*}
    \begin{split} 
     G \de uv &\coloneqq (V \cup \{w\}, (E \setminus \{uv\}) \cup \{uw\} \cup
     \{wv\})\\
              &= (((G - uv) + w) + uw) + wv.
    \end{split}
   \end{equation*}
 \end{itemize}
\end{definition}

\begin{figure}[h]
 \centering
 \begin{tikzpicture}[scale=1.5]
  \node[vertex] (u) at (0,0) {};
  \node[vertex] (v) at (2,-1) {};
  \node[left=0mm of u] {$u$};
  \node[right=0mm of v] {$v$};
  \draw[thick] (u) -- (v);
  
  \draw[thick,->] (3,-0.5) to node[midway,yshift=4mm] {$G \de uv$} (4,-0.5);
  
  \node[vertex] (u2) at (5,0) {};
  \node[vertex] (v2) at (7,-1) {};
  \draw[thick,myred] (u2) to node[midway,myred,vertex] (w) {} (v2);
  \node[left=0mm of u2] {$u$};
  \node[right=0mm of v2] {$v$};
  \node[below left=-1mm and -1mm of w,myred] {$w$};
  
 \end{tikzpicture}

 \caption{Operace dělení hrany $uv$.}
 \label{fig:deleni-hrany}
\end{figure}

Pokračujeme bodem (2). Asi není příliš překvapivé, že rovinné grafy nemohou mít
příliš mnoho hran (vzhledem k počtu vrcholů). Čím víc hran do grafu přidám, tím
se snižuje šance, že každou další hranu zvládnu nakreslit bez křížení s
ostatními.

Nejprve objasníme, k čemu nám vlastně slouží
\hyperref[thm:jordanova-o-kruznici]{Jordanova věta o kružnici}. Totiž, kromě
křivek a bodů reprezentujících hrany a vrcholy, získává nakreslení
\textbf{rovinných} grafů ještě jednu strukturu -- stěny. Lidsky řečeno, stěny
daného nakreslení jsou oblasti roviny ohraničené hranami, přesněji cestami.
Nahlédneme, že v důsledku \hyperref[thm:jordanova-o-kruznici]{Jordanovy věty} se
po vynětí všech hran nakreslení daného grafu $G = (V,E)$ rozpadne roviny na
několik omezených oblastí (tzv. \uv{vnitřní stěny}) a jednu neomezenou (tzv.
\uv{vnější stěna}). Každou z těchto oblastí nazveme \emph{stěnou} grafu $G$ a
množinu všech stěn označíme písmenem $F$ (z angl. \textbf{f}ace).

\begin{figure}[h]
 \centering
 \begin{tikzpicture}
  \pgfdeclarelayer{background}
  \pgfdeclarelayer{foreground}
  \pgfsetlayers{background,main,foreground}
  \begin{pgfonlayer}{foreground}
   \node[vertex] (a) at (0,0) {};
   \node[vertex] (b) at (1,2) {};
   \node[vertex] (c) at (3,3) {};
   \node[vertex] (d) at (2,-1) {};
   \node[vertex] (e) at (4,2) {};
   \node[vertex] (f) at (4,0) {};
   
   \node[vertex] (k) at (-0.5,2.5) {};
   \node[vertex] (l) at (-1,1.5) {};
   \node[vertex] (m) at (5,-0.5) {};
   
   \draw[thick] (a) -- (b);
   \draw[thick] (b) -- (c);
   \draw[thick] (c) to node[midway,vertex] {} (d);
   \draw[thick] (d) -- (a);
   \draw[thick] (c) -- (e);
   \draw[thick] (e) -- (f);
   \draw[thick] (f) -- (d);

   \draw[thick] (b) -- (k);
   \draw[thick] (k) -- (l);
   \draw[thick] (f) -- (m);
   
  \end{pgfonlayer}
  \fill[myred,opacity=0.25] (a.center) -- (b.center) -- (c.center) --
   (d.center) -- (a.center);
  \fill[mygreen,opacity=0.25] (c.center) -- (e.center) -- (f.center) -- (d.center)
   -- (c.center);
  \node[myred] (F1) at (1.5,0.8) {$\mathbf{F_1}$};
  \node[mygreen] (F2) at (3.3,0.8) {$\mathbf{F_2}$};

  \begin{pgfonlayer}{background}
   \fill[myblue,opacity=0.25] (-4,4) -- (8,4) -- (8,-2) -- (-4,-2) -- (-4,4);
   \node[myblue] (F3) at (6,3) {$\mathbf{F_3}$};
  \end{pgfonlayer}
  
 \end{tikzpicture}

 \caption{Nakreslení grafu s vnitřními stěnami $\mathbf{\clr{F_1}}$ a
  $\mathbf{\clg{F_2}}$ a vnější stěnou $\mathbf{\clb{F_3}}$.}
 \label{fig:steny-nakresleni}
\end{figure}

\begin{remark}
 Všimněte si, že pojem \emph{stěny} definujeme pouze pro \textbf{rovinné} grafy.
 To má vlastně dva důvody. Zaprvé, stěny, které mají na svých hranicích
 průsečíky nakreslených hran neodpovídající žádným vrcholům grafu, vytvářejí
 strukturu nezávislou na grafu $G = (V,E)$, jejž kreslíme. To je algebraicky
 zcela nepřirozené.

 Druhý důvod je více praktický. Totiž, jak jsme již zmínili, hrany můžeme
 kreslit tak, aby vznikl libovolný počet křížení. To zároveň znamená, že můžeme
 vytvořit libovolný počet stěn v daném nakreslení. Zásadní výsledek pro rovinné
 grafy (který si dokážeme) říká, že počet stěn v každém jeho rovinném nakreslení
 závisí pouze na počtu vrcholů a hran, a \textbf{nikoli na volbě samotného
 rovinného nakreslení}. To je překvapivě velmi hluboký výsledek, neboť ukazuje,
 že v principu geometrická struktura stěn je v případě rovinných grafů naprosto
 kompatibilní s jeho v principu algebraickou strukturou hran.
\end{remark}

Abychom mohli tvrdit, že graf vůbec má nějaké stěny, potřebujeme si rozmyslet,
že nakreslený cyklus je topologická kružnice. To je intuitivně zřejmé, zkrátka
za sebe spojíme nakreslené hrany a ta poslední bude končit tam, kde ta první
začala. Rozmyslet si tento fakt formálně je mírně složitější úloha.

Potřebujeme definovat spojení dvou křivek $\gamma_1,\gamma_2$ takových, že
$\gamma_1(1) = \gamma_2(0)$ a dokázat, že za předpokladu, že se neprotínají
nikde jinde, se jedná rovněž o křivku v rovině. Princip definice je velmi
přímočarý -- nová křivka $\gamma$ bude zkrátka v čase od $0$ do $1 / 2$ sledovat
křivku $\gamma_1$ a od $1 / 2$ do $1$ sledovat křivku $\gamma_2$. Formálně lze
toto nové zobrazení $\gamma:[0,1] \to \R^2$ definovat například následovně.

\begin{definition}[Spojení křivek]
 \label{def:spojeni-krivek}
 Ať $\gamma_1,\gamma_2$ jsou dvě křivky v rovině, které se neprotínají, a platí
 $\gamma_1(1) = \gamma_2(0)$. Potom definujeme jejich \emph{spojení}, zapisované
 často zkrátka $\gamma_1\gamma_2$ předpisem
 \[
  \gamma_1\gamma_2(t) \coloneqq
  \begin{cases}
   \gamma_1(2t), &\text{pokud } t \in [0,\frac{1}{2}]\\
   \gamma_2(2t - 1), &\text{pokud } t \in [\frac{1}{2},1].
  \end{cases}
 \]
\end{definition}

\begin{lemma}
 \label{lem:spojeni-krivek}
 Ať $\gamma_1,\gamma_2$ jsou křivky v rovině, které se neprotínají, a
 $\gamma_1(1) = \gamma_2(0)$. Potom je spojení $\gamma_1\gamma_2$ dobře
 definované a je to křivka v~rovině.
\end{lemma}
\begin{proof}
 \uv{Dobrá definovanost} $\gamma_1\gamma_2$ zkrátka znamená, že jsme nenapsali
 žádný nesmysl, tj. že to je opravdu \emph{zobrazení} (to je zde zřejmé) a že
 jsme při jeho definici například neuvažovali body mimo domény zobrazení
 $\gamma_1$ a $\gamma_2$. To je velmi snadné ověřit, neboť pro $t \in [0,1 / 2]$
 je $2t \in [0,1]$ a pro $t \in [1 / 2,1]$ je $2t-1 \in [0,1]$. V čase spojení
 $t = 1 / 2$ máme
 \[
  \gamma_1(2t) = \gamma_1(1) = \gamma_2(0) = \gamma_2(2t - 1),
 \]
 tedy vše funguje, jak má.

 Fakt, že $\gamma_1\gamma_2$ je křivka v rovině, je též snadno vidět. Totiž,
 zcela jistě je to zobrazení $[0,1] \to \R^2$. Dále, je spojité, ježto
 $\gamma_1$ i $\gamma_2$ jsou spojitá a plynule na sebe navazují. Je též prosté,
 neboť $\gamma_1$ i $\gamma_2$ jsou prostá a z~předpokladu se neprotínají.
\end{proof}

Z \hyperref[lem:spojeni-krivek]{předchozího lemmatu} triviální indukcí vyplývá,
že je-li 
