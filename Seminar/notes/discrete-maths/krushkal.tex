\subsubsection{Kruskalův algoritmus}
\label{sssec:kruskaluv-algoritmus}

Na problém nalezení minimální kostry souvislého ohodnoceného grafu existuje
skoro až zázračně přímočarý algoritmus, pojmenovaný po americkém matematiku,
Josephu B. Kruskalovi. Jeho základní myšlenkou je prostě začít s grafem $K$
obsahujícím všechny vrcholy z $V$ a přidávat hrany od těch s nejnižším
ohodnocením po ty s nejvyšším tak dlouho, dokud nevznikne souvislý graf. Je
potřeba pouze dávat pozor na cykly. K tomu stačí si pamatovat stromy (jako
množiny vrcholů), které přidáváním hran vytvářím, a povolit přidání hrany
jedině v případě, že spojuje dva různé stromy.

Pro zápis v pseudokódu vizte \myref{algoritmus}{alg:kruskal}. Manim s průběhem
algoritmu s náhodně vygenerovaným hodnocením je k dispozici
\href{https://raw.githubusercontent.com/Tesser3kt/GEVO/main/Seminar/animations/media/videos/graphs/2160p60/SpanningTreeExample.mp4}{zde}.

\pagebreak

\begin{algorithm}
 \caption{Kruskalův algoritmus.}
 \label{alg:kruskal}

 \SetKwInOut{Input}{input}
 \SetKwInOut{Output}{output}
 \SetKw{KwReturn}{return}

 \Input{souvislý ohodnocený graf $G = (V,E,w)$, kde $V = \{v_1,\ldots,v_n\}$}
 \Output{množina hran $E'$ minimální kostry grafu $G$}
 \BlankLine
 \emph{Inicializace}\;
 $E' \leftarrow \emptyset$\;
 \emph{Množina hran, které nelze přidat (jinak by vznikl cyklus)}\;
 $X \leftarrow \emptyset$\;
 \For{$i \leftarrow 1$ \KwTo $n$} {
  \emph{Každý strom nejprve obsahuje pouze jediný vrchol}\;
  $T_i \leftarrow \{v_i\}$\;
 }
 \emph{Množina indexů pro pamatování sloučených stromů}\;
 $I \leftarrow \{1,\ldots,n\}$\;
 \BlankLine
 \emph{Přidávám hrany, dokud mám pořád víc než jeden strom}\;
 \While{$\#I > 1$} {
  $e \leftarrow \text{libovolná hrana s nejnižším } w(e) \text{, která není v }
  E' \text{ ani v } X$\;
  $i \leftarrow \text{index v } I \text{ takový, že } s(e) \in T_i$\;
  $j \leftarrow \text{index v } I \text{ takový, že } t(e) \in T_j$\;
  \BlankLine
  \If{$i = j$}{
   \emph{Hrana spojuje vrcholy ve stejném stromě, jejím přidáním by vznikl
   cyklus}\;
   $X \leftarrow X \cup \{e\}$\;
  }
  \Else {
   \emph{Hrana spojuje různé stromy. Přidávám ji do kostry}\;
   $E' \leftarrow E' \cup \{e\}$\;
   $T_i \leftarrow T_i \cup T_j$\;
   $I \leftarrow I \setminus \{j\}$\;
  }
 }
 \KwReturn $E'$\;
\end{algorithm}

\begin{claim}
 \label{claim:kruskal-korektni}
 Kruskalův algoritmus je korektní.
\end{claim}
\begin{proof}
 Potřebujeme ověřit, že
 \begin{enumerate}
  \item algoritmus provede pouze konečný počet kroků;
  \item algoritmus vrátí správnou odpověď.
 \end{enumerate}
 Případ (1) je zřejmý, protože $\# E < \infty$, čili algoritmus přidá do $E'$
 pouze konečně mnoho hran. Navíc, graf $G$ je z předpokladu souvislý, a tedy
 vždy existuje hrana $e$ spojující dva různé stromy $T_i$ a $T_j$.

 V případě (2) uvažme, že $K = (V,E',w)$ není minimální kostra $G$. V~takovém
 případě se mohlo stát, že
 \begin{enumerate}[label=(\alph*)]
  \item $K$ není strom -- tedy buď není souvislý nebo obsahuje cyklus;
  \item existuje podmnožina $E'' \subseteq E$ taková, že $K' = (V,E'',w)$ je strom a
   \[
    \sum_{e \in E''}^{} w(e) < \sum_{e \in E'}^{} w(e).
   \]
 \end{enumerate}
 Případ (a) lze vyloučit snadno, neboť souvislost $K$ plyne ihned ze
 souvislosti $G$, tedy, jak již bylo řečeno v bodě (1), vždy lze nalézt hranu
 spojující do té doby dva různé stromy. Pokud $K$ obsahuje cyklus, tak
 algoritmus musel v jednom kroku spojit hranou dva vrcholy ze stejného stromu,
 což je spor.

 Pokud nastal případ (b), pak musejí existovat hrany $e'' \in E'' \setminus E'$
 a $e' \in E' \setminus E''$ takové, že $w(e'') < w(e')$. Protože algoritmus
 zkouší přidávat hrany vždy počínaje těmi s nejmenší vahou, musel v nějakém
 kroku narazit na hranu $e''$ a zavrhnout ji. To ovšem znamená, že hrana $e''$ 
 spojila dva různé vrcholy téhož stromu a graf $K' = (V,E'',w)$ obsahuje
 cyklus. To je spor s předpokladem, že $K'$ je strom. Tedy, taková podmnožina
 $E'' \subseteq E$ nemůže existovat a $K$ je vskutku minimální kostra $G$.
\end{proof}

\begin{warning}
 Minimální kostra grafu \textbf{není jednoznačně určena}! Všimněte si, že na
 řádku 12 \hyperref[alg:kruskal]{Kruskalova algoritmu} volím
 \textbf{libovolnou} hranu, která má ze všech zatím nepřidaných nejnižší váhu a
 jejímž přidáním nevznikne cyklus.
\end{warning}

\begin{remark}
 V \hyperref[def:minimalni-kostra]{definici minimální kostry} požadujeme, aby
 $G$ byl souvislý graf. Pokud tomu však tak není, je pořád možné zkonstruovat
 minimální kostru pro každou část $G$, která souvislá je (pro každou jeho tzv.
 \emph{komponentu souvislosti}), zkrátka tím, že
 \hyperref[alg:kruskal]{Kruskalův algoritmus} spouštíme opakovaně.
\end{remark}
