\subsubsection{Grafové homomorfismy}
\label{sssec:grafove-homomorfismy}

Trochu času věnujeme přemýšlení o tom, co může znamenat \emph{zobrazení} mezi
grafy. Ať tedy $G = (V_G,E_G)$ a $H = (V_H,E_H)$ jsou dva grafy. Zápis
$\varphi:G \to H$, tj. zobrazení z $G$ do $H$, sice smysl dává, neb $G$ a $H$
jsou (jako všechno v teorii množin) též množiny, ale není pro rozvoj teorie
nijak zajímavý. Takové zobrazení totiž může klidně posílat třeba hrany v $G$ na
vrcholy v $H$ a tvořit jiné podobné zrůdnosti.

V celé sekci se díváme na graf jako na množinu vrcholů se strukturou danou
hranami. Tenhle pohled zůstává užitečným i nyní. Totiž, \emph{homomorfismem} (z
řec. homós, \uv{stejný}, a morphé, \uv{tvar, forma}) mezi obecně množinami se
strukturou se myslí \textbf{zobrazení mezi těmito množinami, které zachovává
strukturu}.

V případě grafů toto znamená, že \emph{homomorfismus} $\varphi:G \to H$ je
zobrazení $\varphi_V:V_G \to V_H$ z vrcholů $G$ do vrcholů $H$, které
\uv{zachovává} hrany. To znamená zkrátka jen to, že když mezi vrcholy v $G$ vede
hrana, tak mezi jejich obrazy při zobrazení $\varphi_V$ vede též hrana.

\begin{definition}[Grafový homomorfismus]
 \label{def:grafovy-homomorfismus}
 Buďtež $G = (V_G,E_G)$ a $H = (V_H,E_H)$ dva grafy. \emph{Homomorfismem} z $G$
 do $H$, značeným běžně $\varphi:G \to H$, myslíme zobrazení
 \[
  \varphi_V:V_G \to V_H
 \]
 takové, že
 \[
  (u,v) \in E_G \Rightarrow (\varphi_V(u),\varphi_V(v)) \in E_H.
 \]
 Obvykle ztotožňujeme homomorfismus $\varphi:G \to H$ se zobrazením
 $\varphi_V:V_G \to V_H$ mezi bázovými množinami a oboje značíme zkrátka
 $\varphi$.
\end{definition}

S \hyperref[def:grafovy-homomorfismus]{touto definicí} je jistý problém. Uvažme
grafy $G$ a $H$ dané \myref{obrázkem}{fig:problem-s-definici-homomorfismu}.

\begin{figure}[h]
 \centering
 \begin{subfigure}{.47\textwidth}
  \centering
  \begin{tikzpicture}[scale=1.5]
   \node[vertex] (a) at (0,0) {};
   \node[vertex] (b) at (0,1) {};
   \node[vertex] (c) at (1,1) {};
   
   \draw[thick] (a) -- (b);
   \draw[thick] (b) -- (c);
   \draw[thick] (a) -- (c);
   
   \node[below left=-1mm and -1mm of a] {$a$};
   \node[above left=-1mm and -1mm of b] {$b$};
   \node[above right=-1mm and -1mm of c] {$c$};
  \end{tikzpicture}
  \caption{Graf $G$.}
  \label{subfig:graf-g}
 \end{subfigure}
 \begin{subfigure}{.47\textwidth}
  \centering
  \begin{tikzpicture}[scale=1.5]
   \node[vertex] (1) at (0,0) {};
   \node[vertex] (2) at (1,0) {};
   \node[vertex] (3) at (1,1) {};
   \node[vertex] (4) at (0,1) {};

   \draw[thick] (1) -- (2);
   \draw[thick] (2) -- (3);
   \draw[thick] (3) -- (4);
   \draw[thick] (4) -- (1);
   \draw[thick] (1) -- (3);
   
   \node[below left=-1mm and -1mm of 1] {$1$};
   \node[below right=-1mm and -1mm of 2] {$2$};
   \node[above right=-1mm and -1mm of 3] {$3$};
   \node[above left=-1mm and -1mm of 4] {$4$};
  \end{tikzpicture}
  \caption{Graf $H$.}
  \label{subfig:graf-h}
 \end{subfigure}
 \caption{Problém s definicí homomorfismu $G \to H$.}
 \label{fig:problem-s-definici-homomorfismu}
\end{figure}

Zde $V_G = \{a,b,c\}$ a $V_H = \{1,2,3,4\}$. Jeden možný homomorfismus $G \to H$
je třeba dán zobrazením $\varphi:V_G \to V_H$ takovým, že
\begin{equation*}
 \begin{split}
  \varphi(a) &= 1, \\
  \varphi(b) &= 4, \\
  \varphi(c) &= 3.
 \end{split}
\end{equation*}
Pak je opravdu splněna podmínka z \myref{definice}{def:grafovy-homomorfismus},
neboť mezi každými dvěma z vrcholů $a,b,c$ vede hrana a mezi každými dvěma z
vrcholů $1,4,3$ rovněž vede hrana.

Uvažme však jen trochu jiný homomorfismus $\psi: G \to H$, který je dán
rovnostmi
\begin{equation*}
 \begin{split}
  \psi(a) &= 1, \\
  \psi(b) &= 3, \\
  \psi(c) &= 3; \\
 \end{split}
\end{equation*}
čili vlastně \uv{slepuje} vrcholy $b$ a $c$ do vrcholu $3$. Všimněte si, že
takovéto zobrazení není podle \myref{definice}{def:grafovy-homomorfismus}
homomorfismem. Hrany $(a,b)$ a $(a,c)$ jsou sice zachovány, ale podmínka z téže
definice pro hranu $(b,c)$ říká, že
\[
 (b,c) \in E_G \Rightarrow (\psi(b),\psi(c)) = (3,3) \in E_H,
\]
což zřejmě nelze, neboť podle naší \hyperref[def:graf-poprve]{první definice
grafu} je $E$ relace na $V$, která je symetrická a \textbf{antireflexivní}, což
znamená, že v grafu nepovolujeme smyčky, kterou by $(3,3)$ jistě byla.

Tato peripetie má několik možných katastrof. V zájmu rozvoje matematického
myšlení drahých čtenářů si některé rozebereme.

\begin{enumerate}
 \item Nic nedělat držet se původní definice a nepovažovat $\psi$ za
  homomorfismus. To je jistě jedno možné řešení a vyniká množstvím potřebné
  práce k jeho dosažení. Lidsky řečeno vlastně znamená, že slepovat vrcholy mohu
  jedině tehdy, když mezi nimi nevede hrana. Toto řešení je zároveň nejvíce
  omezující.
 \item Modifikovat \hyperref[def:grafovy-homomorfismus]{definici homomorfismu}
  tak, aby ignoroval tyhle případy. Tedy, upravit podmínku zachování hran
  následovně:
  \[
   ((u,v) \in E_G \wedge \psi(u) \neq \psi(v)) \Rightarrow (\psi(u),\psi(v)) \in
   E_H.
  \]
  Tohle řešení je \emph{extrémně nepřirozené}. Totiž, když z grafu $H$ na
  \myref{obrázku}{subfig:graf-h} vynecháme hranu $(b,c)$ a homomorfismus $\psi$
  ponecháme beze změny, pak je v obou případech (tedy s hranou $(b,c)$ i bez ní)
  obrazem grafu $H$ v $G$ úsečka $13$. Aby obrazy dvou různých grafů stejným
  homomorfismem byly stejné, je algebraicky nepřijatelné.
 \item Zvolnit definici grafu tak, aby povolovala smyčky. To by znamenalo
  konkrétně odstranit z \hyperref[def:graf-poprve]{první definice grafu}
  podmínku antireflexivity relace $E$. Zůstala by tedy pouze symetrickou a
  dvojice $(v,v)$ pro $v \in V$ by byly smyčkami. Ovšem,
  \hyperref[def:graf-podruhe]{druhá definice grafů} by se stala nepoužitelnou,
  neboť dvouprvková podmnožina $V$ nemůže obsahovat dvakrát tentýž prvek. Z
  \hyperref[def:graf-potreti]{třetí definice grafu} by stačilo rovněž odstranit
  pouze podmínku (a) zakazující shodnost zdroje a cíle téže hrany.

  Při adopci tohoto řešení by graf $G$ musel obsahovat smyčku $(3,3)$, aby mohla
  definice $\psi$ zůstat beze změny. V takovém případě by obrazem grafu $H$
  \textbf{bez} hrany $(b,c)$ byla opět úsečka $13$, ale obrazem grafu $H$
  \textbf{se} hranou $(b,c)$ by byla úsečka $13$ spolu se smyčkou na vrcholu
  $3$.
\end{enumerate}
