\subsection{Stromy}
\label{ssec:stromy}

Chvíli se budeme bavit o stromech -- ano, těch s listy a kořenem. Stromy jsou
speciální typy grafů, které se takto nazývají ne nadarmo. Jsou to totiž grafy, u
kterých si člověk může zvolit jakýsi \uv{počáteční} vrchol (zvaný \emph{kořen}),
z nějž se po cestě (ve smyslu \myref{definice}{def:cesta}) vždy dostane do
jednoho z \uv{koncových} vrcholů, tzv. \emph{listů}. Příklad stromu je na
\myref{obrázku}{fig:priklad-stromu}.

\begin{figure}[h]
 \centering
 \begin{tikzpicture}
  \tikzset{vertex/.style = {shape=circle,fill,text=white,minimum size=6pt,inner
  sep=1pt}}
  \tikzset{->-/.style={decoration={
   markings,
   mark=at position #1 with {\arrow{>[scale=0.8]}}},postaction={decorate}}}

  \node[vertex,myred,text=white] (1) at (0, 0) {};

  \node[vertex] (2) at (-2, -1) {};
  \node[vertex] (3) at (2, -1) {};

  \node[vertex] (4) at (-3, -2) {};
  \node[vertex] (5) at (-1, -2) {};
  \node[vertex] (6) at (1, -2) {};
  \node[vertex] (7) at (3, -2) {};

  \node[vertex,myblue] (8) at (-3.5, -3) {};
  \node[vertex,myblue] (9) at (-2.5, -3) {};
  \node[vertex,myblue] (10) at (-1.5, -3) {};
  \node[vertex,myblue] (11) at (-0.5, -3) {};

  \node[vertex,myblue] (12) at (0.5, -3) {};
  \node[vertex,myblue] (13) at (1.5, -3) {};
  \node[vertex,myblue] (14) at (2.5, -3) {};
  \node[vertex,myblue] (15) at (3.5, -3) {};

  \foreach \i in {2,3} {
   \draw[thick] (1) -- (\i);
  }
  \foreach \i in {4,5} {
   \draw[thick] (2) -- (\i);
  }
  \foreach \i in {6,7} {
   \draw[thick] (3) -- (\i);
  }
  \foreach \i in {8,9} {
   \draw[thick] (4) -- (\i);
  }
  \foreach \i in {10,11} {
   \draw[thick] (5) -- (\i);
  }
  \foreach \i in {12,13} {
   \draw[thick] (6) -- (\i);
  }
  \foreach \i in {14,15} {
   \draw[thick] (7) -- (\i);
  }
 \end{tikzpicture}
 \caption{Příklad stromu. Kořen je značen \clr{červeně} a listy \clb{modře}.}
 \label{fig:priklad-stromu}
\end{figure}

\begin{warning}
 Listy stromu jsou určeny jednoznačně jeho strukturou (jsou to ty jediné
 vrcholy, do nichž cesty od kořene mohou pouze vést a nikoli jimi procházet). Za
 kořen lze však volit libovolný vrchol, klidně i jeden z listů. Strom z
 \myref{obrázku}{fig:priklad-stromu} může proto vypadat i jak ukazuje
 \myref{obrázek}{fig:strom-jiny-koren}.

 Často není nutno o kořenu a listech stromu hovořit, pokud je v konkrétní
 situaci irelevantní rozlišovat jednotlivé vrcholy. Součástí definice stromu
 (kterou si záhy odvodíme) kořen ani listy nejsou.
\end{warning}

\begin{figure}[h]
	\centering
	\begin{tikzpicture}
		\tikzset{vertex/.style = {shape=circle,fill,text=white,minimum size=6pt,inner
					sep=1pt}}
		\tikzset{->-/.style={decoration={
				markings,
				mark=at position #1 with {\arrow{>[scale=0.8]}}},postaction={decorate}}}

		\node[vertex,myred,text=white] (2) at (0, 0) {};

		\node[vertex] (4) at (-3, -1) {};
		\node[vertex] (5) at (0, -1) {};
		\node[vertex] (1) at (3, -1) {};

		\node[vertex,myblue] (8) at (-4, -2) {};
		\node[vertex,myblue] (9) at (-2, -2) {};
		\node[vertex,myblue] (10) at (-1, -2) {};
		\node[vertex,myblue] (11) at (1, -2) {};
		\node[vertex] (3) at (3, -2) {};

		\node[vertex] (6) at (2, -3) {};
		\node[vertex] (7) at (4, -3) {};

		\node[vertex,myblue] (12) at (1.5, -4) {};
		\node[vertex,myblue] (13) at (2.5, -4) {};
		\node[vertex,myblue] (14) at (3.5, -4) {};
		\node[vertex,myblue] (15) at (4.5, -4) {};

		\foreach \i in {2,3} {
				\draw[thick] (1) -- (\i);
			}
		\foreach \i in {4,5} {
				\draw[thick] (2) -- (\i);
			}
		\foreach \i in {6,7} {
				\draw[thick] (3) -- (\i);
			}
		\foreach \i in {8,9} {
				\draw[thick] (4) -- (\i);
			}
		\foreach \i in {10,11} {
				\draw[thick] (5) -- (\i);
			}
		\foreach \i in {12,13} {
				\draw[thick] (6) -- (\i);
			}
		\foreach \i in {14,15} {
				\draw[thick] (7) -- (\i);
			}
	\end{tikzpicture}
	\caption{Strom z \myref{obrázku}{fig:priklad-stromu} s jinou volbou kořene.}
	\label{fig:strom-jiny-koren}
\end{figure}

Nyní si rozmyslíme dvě ekvivalentní definice stromu.

Za první podmínku, abychom mohli graf nazvat stromem, budeme považovat fakt, že
od kořene se dá dostat po hranách do každého z listů. Ekvivalentně, že z každého
vrcholu se dá cestou dostat do každého vrcholu, protože za kořen lze, jak jsme
nahlédli, volit kterýkoli vrchol, a cestu z~kořene do vrcholu můžeme zkrátit
tak, aby končila v nějakém vrcholu, jímž původně procházela. Grafy splňující
tuto podmínku slují \emph{souvislé}.

\begin{definition}[Souvislý graf]
	\label{def:souvisly-graf}
 Graf $G = (V,E)$ nazveme \emph{souvislým}, pokud pro každé dva vrcholy $v,w \in
 V$ existuje cesta z $v$ do $w$, tedy cesta $v_1v_2 \cdots v_n$, kde $v_1 = v$ a
 $v_n = w$.
\end{definition}

Samotný název \uv{strom} plyne z faktu, že se jako graf pouze \uv{větví}, čímž
míníme, že při cestě směrem od (libovolného) kořene se jeden může pouze
přibližovat k listům, ale nikoli se dostat zpět blíže ke kořeni. To lze snadno
zařídit tak, že zakážeme cykly. Totiž, neexistuje-li v grafu cyklus, pak se po
libovolné cestě ze zvoleného vrcholu můžeme od tohoto vrcholu pouze vzdalovat.
Takové grafy nazveme, přirozeně, \emph{acyklické}.

\begin{definition}[Acyklický graf]
	\label{def:acyklicky-graf}
	Graf $G = (V,E)$ nazveme \emph{acyklický}, pokud neobsahuje cyklus o aspoň
	třech vrcholech (samotné vrcholy jsou totiž z \hyperref[def:cyklus]{definice}
	vždy cykly).
\end{definition}

\begin{definition}[Strom]
	\label{def:strom}
	Graf $G = (V,E)$ nazveme \emph{stromem}, je-li souvislý a acyklický.
\end{definition}

Na začátku sekce jsme slíbili ještě ekvivalentní definici stromu; ta činí začnou
část důvodu užitečnosti stromů, především v informatice.

Ukazuje se totiž, že neexistence cyklů spolu se souvislostí způsobují, že mezi
dvěma vrcholy vede vždy \textbf{přesně jedna cesta}. Po chvíli zamyšlení snad
toto nepřichází jako nijak divoké tvrzení. Přeci, pokud by mezi vrcholy vedly
cesty dvě, pak vrchol, kde se rozpojují, a vrchol, kde se opět spojují, by byly
součástí cyklu uvnitř stromu, který jsme výslovně zakázali. Třeba překvapivější
je fakt, že platí i opačná implikace.

\begin{claim}[Ekvivalentní definice stromu]
	\label{claim:ekvivalentni-definice-stromu}
	Graf $G = (V,E)$ je stromem ve smyslu \myref{definice}{def:strom} právě tehdy,
	když mezi každými dvěma vrcholy $G$ vede přesně jedna cesta.
\end{claim}

\begin{proof}
	Dokazujeme dvě implikace. Obě budeme dokazovat v jejich \emph{kontrapozitivní}
	formě, tedy jako obrácenou implikaci mezi negacemi výroků. Lidsky, dokážeme, že
	(1) když existují vrcholy, mezi kterými nevede žádná nebo vede více než jedna
	cesta, pak $G$ není strom, a (2) když $G$ není strom, tak existují vrcholy,
	mezi kterými nevede žádná cesta nebo vede více než jedna.
	\begin{enumerate}
		\item Pokud existují vrcholy, mezi kterými nevede cesta, pak $G$ není
		      souvislý, což odporuje \hyperref[def:strom]{definici stromu}. Budeme tedy
		      předpokládat, že existují vrcholy $v,w$, mezi kterými vedou různé cesty $v_1
			      \cdots v_n$ a $v'_1 \cdots v'_m$, kde $v_1 = v'_1 = v$ a $v_n = v'_m = w$.
		      Myšlenka důkazu je najít cyklus obsahující vrchol, kde se cesty rozdělují, a
		      vrchol, kde se opět spojují. Vizte
		      \myref{obrázek}{fig:cast-1-dukazu-definice-stromu}.

		      Ať $r$ (od \textbf{r}ozpojení) je \textbf{největší} index takový, že $v_i =
			      v'_i$ pro všechna $i \leq r$ (čili $v_r = v'_r$ je vrchol, ve kterém se cesty
		      rozpojují). Ten určitě existuje, protože cesty se v nejhorším případě dělí už
		      ve vrcholu $v = v_1$.

		      Podobně, ať $s$ (od \textbf{s}pojení) je \textbf{nejmenší} index takový, že
		      existuje $k \in \Z$ splňující $v_j = v'_{j+k}$ pro všechna $j \geq s$. Čili,
		      vrchol $v_s = v'_{s+k}$ je vrchol, ve kterém se cesty opět spojily. Ovšem,
		      mohlo se tak stát v okamžiku, kdy jsme po jedné cestě prošli více nebo méně
		      vrcholů než po druhé -- tento počet vyjadřuje ono číslo $k$. Takový vrchol
		      jistě existuje, v nejhorším je to přímo koncový vrchol $w = v_n$.

		      Zřejmě platí $r < s$, jinak by cesty nebyly různé. Potom je ovšem například
		      posloupnost vrcholů
		      \[
			      (v_r,v_{r+1},\ldots,v_s = v'_{s+k},v'_{s+k-1},\ldots,v'_r = v_r)
		      \]
		      cyklem v $G$. Tedy ani v tomto případě $G$ není strom.
		\item Když $G$ není strom, tak není souvislý nebo obsahuje cyklus. Když $G$
		      není souvislý, tak existují vrcholy, mezi nimiž nevede v $G$ cesta, což
		      protiřečí podmínce, aby mezi každým párem vrcholů vedla přesně jedna.

		      Budeme tedy předpokládat, že $G$ obsahuje cyklus $v_1v_2 \cdots v_n$ (tedy
		      $v_n = v_1$ a $n \geq 3$). Pak ovšem pro libovolné indexy $i < j \leq n$ jsou
		      posloupnosti
		      \[
			      (v_i,v_{i+1},\ldots,v_j) \quad \text{a} \quad (v_i,v_{i-1},\ldots,v_1 =
			      v_n,v_{n-1},\ldots,v_j)
		      \]
		      dvě různé cesty mezi $v_i$ a $v_j$. Vizte
		      \myref{obrázek}{fig:cast-2-dukazu-definice-stromu}.
	\end{enumerate}
	Tím je důkaz dokončen.
\end{proof}

\begin{figure}[h]
	\centering
	\begin{subfigure}{.45\textwidth}
		\centering
		\begin{tikzpicture}
			\tikzset{vertex/.style = {shape=circle,fill,text=white,minimum size=6pt,inner
						sep=1pt}}
			\tikzset{->-/.style={decoration={
					markings,
					mark=at position #1 with {\arrow{>[scale=1]}}},postaction={decorate}}}

			\node[vertex] (1) at (0,0) {};
			\node[vertex] (2) at (0.5,-1) {};
			\node[vertex,myred,minimum size=9pt] (3) at (0.5,-2) {};

			\node[vertex] (4) at (-0.5,-3) {};

			\node[vertex] (5) at (1,-2.5) {};
			\node[vertex] (6) at (1,-3.5) {};

			\node[vertex,myblue,minimum size=9pt] (7) at (0.25,-4) {};
			\node[vertex] (8) at (-1,-4.5) {};

			\draw[->-=.55,thick] (1) -- (2);
			\draw[->-=.55,thick] (2) -- (3);
			\draw[->-=.55,thick] (3) -- (4);
			\draw[->-=.55,thick] (3) -- (5);
			\draw[->-=.55,thick] (4) -- (7);
			\draw[->-=.55,thick] (5) -- (6);
			\draw[->-=.55,thick] (6) -- (7);
			\draw[->-=.55,thick] (7) -- (8);

			\node[right = -.5mm of 1] {$v = v_1 = v'_1$};
			\node[right = -.5mm of 2] {$v_2 = v'_2$};
			\node[above left = -3mm and -.5mm of 3,text=myred] {$v_3 = v'_3$};
			\node[left = -.5mm of 4]{$v_4$};
			\node[right = -.5mm of 5]{$v'_4$};
			\node[right = -.5mm of 6]{$v'_5$};
			\node[below right = -3mm and -.5mm of 7,text=myblue]{$v_5 = v'_6$};
			\node[left = -.5mm of 8]{$v'_7 = v_6 = w$};
		\end{tikzpicture}
		\caption{Část (1) důkazu \myref{tvrzení}{claim:ekvivalentni-definice-stromu}.
			Zde $\clr{r = 3}, \clb{s = 5}$ a $k = 1$. Sestrojený cyklus je
			$v_3v_4v_5v'_5v'_4v_3'$.}
		\label{fig:cast-1-dukazu-definice-stromu}
	\end{subfigure}
	\hfill
	\begin{subfigure}{.45\textwidth}
		\centering
		\begin{tikzpicture}
			\tikzset{vertex/.style = {shape=circle,fill,text=white,minimum size=6pt,inner
						sep=1pt}}
			\tikzset{->-/.style={decoration={
					markings,
					mark=at position #1 with {\arrow{>[scale=1]}}},postaction={decorate}}}

			\node[vertex] (1) at (0,0) {};
			\node[vertex] (2) at (0.5,-1) {};
			\node[vertex] (3) at (0.5,-2) {};

			\node[vertex,minimum size=9pt,myred] (4) at (-0.5,-3) {};

			\node[vertex] (5) at (1,-2.5) {};
			\node[vertex,minimum size=9pt,myblue] (6) at (1,-3.5) {};

			\node[vertex] (7) at (0.25,-4) {};
			\node[vertex] (8) at (-1,-4.5) {};

			\draw[thick] (1) -- (2);
			\draw[thick] (2) -- (3);
			\draw[->-=.55,thick] (4) -- (3);
			\draw[->-=.55,thick] (3) -- (5);
			\draw[->-=.55,thick] (4) -- (7);
			\draw[->-=.55,thick] (5) -- (6);
			\draw[->-=.55,thick] (7) -- (6);
			\draw[thick] (7) -- (8);

			\node[above left = -3mm and -.5mm of 3] {$v_6 = v_1$};
			\node[left = -.5mm of 4,text=myred]{$v_2$};
			\node[right = -.5mm of 5]{$v_5$};
			\node[right = -.5mm of 6,text=myblue]{$v_4$};
			\node[below right = -2mm and -.5mm of 7]{$v_3$};
		\end{tikzpicture}
		\caption{Část (2) důkazu \myref{tvrzení}{claim:ekvivalentni-definice-stromu}.
			Zde $\clr{i = 2}, \clb{j = 4}$ a sestrojené cesty jsou $v_2v_3v_4$ a
			$v_2v_1v_5v_4$.}
		\label{fig:cast-2-dukazu-definice-stromu}
	\end{subfigure}
	\caption{Ilustrace k důkazu
		\myref{tvrzení}{claim:ekvivalentni-definice-stromu}.}
	\label{fig:ilustrace-k-ekvivalentni-definici-stromu}
\end{figure}

\begin{exercise}
 Dokažte, že je-li $T = (V,E)$ strom, pak $\# E = \# V - 1$.
\end{exercise}

\begin{exercise}
 Spočtěte, kolik existuje stromů na $n$ vrcholech.
\end{exercise}

\subsubsection{Minimální kostra}
\label{sssec:minimalni-kostra}

Ne všechny hrany jsou si rovny. Kterési se rodí krátké, jiné dlouhé; kterési
štíhlé, jiné otylé; kterési racionální, jiné iracionální.

Často nastávají situace, kdy jeden potřebuje hranám grafu přiřadit nějakou
hodnotu, obvykle číselnou, která charakterizuje klíčovou vlastnost této hrany.
Při reprezentaci dopravní sítě grafem to může být délka silnice či její
vytížení, při reprezentaci elektrických obvodů pak například odpor. V teorii
grafů takové přiřazení hodnoty hranám grafu sluje \emph{ohodnocení}.

\begin{definition}[Ohodnocený graf]
\label{def:ohodnoceny-graf}
 Ať $G = (V,E)$ je graf. Libovolné zobrazení $w: E \to \R^{+} = (0,\infty)$ 
 nazveme \emph{ohodnocením} grafu $G$. Trojici $(V,E,w)$, kde $w$ je ohodnocení
 $G$, nazveme \emph{ohodnoceným grafem}.
\end{definition}

\begin{remark}
 \label{rmrk:ohodnoceni-neohodnoceneho}
 Každý graf $G = (V,E)$ lze triviálně ztotožnit s ohodnoceným grafem $(V,E,w)$,
 kde $w: E \to \R^{+}$ je konstantní zobrazení. Obvykle se volí konkrétně $w
 \equiv 1$, tedy zobrazení $w$ takové, že $w(e) = 1$ pro každou $e \in E$.
\end{remark}

Porozumění struktuře ohodnocených grafů může odpovědět na spoustu zajímavých
(jak prakticky tak teoreticky) otázek. Můžeme se kupříkladu ptát, jak se nejlépe
(vzhledem k danému ohodnocení) dostaneme cestou z~jednoho vrcholu do druhého.
Slovo \uv{nejlépe} zde chápeme pouze intuitivně. V závislosti na zpytovaném
problému můžeme požadovat, aby cesta třeba minimalizovala či maximalizovala
součet hodnot všech svých hran přes všechny možné cesty mezi danými vrcholy.
Jsou však i případy, kdy člověk hledá cestu, která je nejblíže \uv{průměru}.

Abychom pořád neříkali \uv{součet přes všechny hrany cesty}, zavedeme si pro
toto často zkoumané množství název \emph{váha cesty}. Čili, je-li $\mathcal{P}
\coloneqq e_1 \cdots e_n$ cesta v nějakém ohodnoceném grafu $G$, pak její vahou
rozumíme výraz
\[
 w(\mathcal{P}) \coloneqq \sum_{i=1}^{n} w(e_i).
\]
Zápis $w(\mathcal{P})$ můžeme vnímat buď jako zneužití zavedeného značení, nebo
jako fakt, že jsme zobrazení $w$ rozšířili z množiny všech hran na množinu všech
cest v grafu $G$ (kde samotné hrany jsou z \hyperref[def:cesta]{definice} též
cesty).

\begin{remark}
 Záměrně jsme užili slovního spojení \emph{váha} cesty místo snad přirozenějšího
 \emph{délka} cesty. V teorii grafů se totiž délkou cesty myslí obyčejně počet
 hran (nebo vrcholů), které obsahuje. Délka cesty $\mathcal{P} = e_1 \cdots e_n$
 je tudíž $n$ (resp. $n + 1$), bo obsahuje $n$ hran (resp. $n + 1$ vrcholů).

 Tento úzus svědčí účelu \hyperref[rmrk:ohodnoceni-neohodnoceneho]{předchozí
 poznámky}. Pokud totiž každý graf bez ohodnocení vnímáme vlastně jako
 ohodnocený graf, kde každá hrana má váhu přesně $1$, pak váha každé cesty je
 rovna její délce.
\end{remark}

První (a nejjednodušší) problém, kterým se budeme zabývat, je nalezení
\emph{minimální kostry} (angl. \emph{spanning tree}).

\begin{definition}[Minimální kostra]
\label{def:minimalni-kostra}
 Ať $G = (V,E,w)$ je \textbf{souvislý} ohodnocený graf. Ohodnocený graf $K =
 (V',E',w)$ nazveme \emph{minimální kostrou} grafu $G$, pokud je souvislý, $V' =
 V$ (tedy $K$ obsahuje všechny vrcholy $G$), $E' \subseteq E$ a
 \[
  \sum_{e \in E'}^{} w(e)
 \]
 je minimální vzhledem ke všem možným volbám podmnožiny $E' \subseteq E$.
 Lidsky řečeno, graf $K$ spojuje všechny vrcholy $G$ tím \uv{nejlevnějším}
 způsobem vzhledem k ohodnocení $w$.
\end{definition}

\begin{figure}[h]
\centering
 \begin{tikzpicture}[scale=2]
  \tikzset{vertex/.style = {shape=circle,fill,text=white,minimum size=9pt,inner
  sep=1pt}}
  \tikzset{->-/.style={decoration={ markings, mark=at position #1 with
  {\arrow{>[scale=1]}}},postaction={decorate}}}
  \foreach \weightx/\x in {3/0, 4/1, 1/2} {
   \foreach \weighty/\y [evaluate=\weighty as \weight using {int(\weightx +
   \weighty)}] in {1/0, 2/1, 0/2} {
    \draw[thick] (\x,\y) to node[above,color=myblue] {$\weightx$} (\x+1,\y);
    \draw[thick] (\x,\y) to node[right,color=myblue] {$\weight$} (\x,\y+1);
   }
  }
  \foreach \weight/\x in {1/0, 2/1, 4/2} {
   \draw[thick] (\x,3) to node[above,color=myblue] {$\weight$} (\x+1,3);
  }
  \foreach \weight/\y in {5/0, 1/1, 3/2} {
   \draw[thick] (3,\y) to node[right,color=myblue] {$\weight$} (3,\y+1);
  }

  \draw[line width=1mm, color=myred] (0, 3) -- (1, 3);
  \draw[line width=1mm, color=myred] (2, 2) -- (2, 3);
  \draw[line width=1mm, color=myred] (2, 2) -- (3, 2);
  \draw[line width=1mm, color=myred] (2, 0) -- (3, 0);
  \draw[line width=1mm, color=myred] (2, 1) -- (3, 1);
  \draw[line width=1mm, color=myred] (3, 1) -- (3, 2);

  \draw[line width=1mm, color=myred] (2, 0) -- (2, 1);
  \draw[line width=1mm, color=myred] (1, 3) -- (2, 3);

  \draw[line width=1mm, color=myred] (3, 3) -- (3, 2);
  \draw[line width=1mm, color=myred] (0, 3) -- (0, 2);
  \draw[line width=1mm, color=myred] (0, 2) -- (1, 2);
  \draw[line width=1mm, color=myred] (0, 1) -- (1, 1);
  \draw[line width=1mm, color=myred] (0, 0) -- (1, 0);

  \draw[line width=1mm, color=myred] (1, 0) -- (2, 0);
  \draw[line width=1mm, color=myred] (0, 0) -- (0, 1);

  \foreach \x in {0, 1, 2, 3} {
   \foreach \y in {0, 1, 2, 3} {
    \node[vertex,fill=myred] (v\x\y) at (\x, \y) {};
   }
  }
 
 \end{tikzpicture}
 \caption{\clr{Minimální kostra} grafu s ohodnocením \clb{$w$}.}
 \label{fig:minimalni-kostra}
\end{figure}

\begin{observation}
 Minimální kostra ohodnoceného grafu je strom.
\end{observation}
\begin{proof}
 Kdyby minimální kostra nebyla strom, pak buď není souvislá, což jsme výslovně
 zakázali, nebo obsahuje cyklus. Tudíž se mezi nějakými dvěma vrcholy dá jít po
 více než jedné cestě, a proto můžeme přinejmenším jednu hranu z kostry
 odebrat. Protože každá hrana má kladné ohodnocení, snížili jsme tím součet
 hodnot všech hran. To je spor.
\end{proof}

\begin{corollary}
 Minimální kostra ohodnoceného stromu je s ním totožná.
\end{corollary}

Minimální kostra je zvlášť užitečná právě při návrhů elektrických obvodů, kdy je
potřeba zařídit, aby všechna připojená zařízení čerpala co nejmenší množství
energie. Protože elektřina proudí rychlostí světla, délka kabelu (pokud není
zrovna mezigalaktický) nás příliš netrápí, ale právě odpor či kvalita/vodivost
konkrétních spojů by mohly.

Další praktickou grafovou úlohou vedoucí na problém nalezení minimální kostry je
potřeba spojit vzdálené servery. Korporace mají obvykle mnoho různých serverů
rozmístěných po světě, jež spolu ale musejí sdílet data. Problém je v tom, že
propojení mezi servery by nejen mělo vést k nejmenší možné prodlevě při přenosu
dat (od toho \textbf{minimální}), ale nesmí ani obsahovat cykly (od toho
\textbf{kostra}). Kdyby totiž cykly obsahovalo, pak by se při přenosu dat stalo,
že by aspoň jeden server v tomto cyklu dostal aspoň dvakrát stejnou informaci z
dvou různých zdrojů, ale v odlišný čas. Taková situace vede nezbytně dříve nebo
později ke korupci dat; řekněme, když daný server už s obdrženou informací začal
po přijetí provádět výpočet. Pro více detailů k tomuto využití minimálních
koster vizte
\href{https://en.wikipedia.org/wiki/Spanning_Tree_Protocol}{Spanning Tree
Protocol}.

\subsubsection{Kruskalův algoritmus}
\label{sssec:kruskaluv-algoritmus}

Na problém nalezení minimální kostry souvislého ohodnoceného grafu existuje
skoro až zázračně přímočarý algoritmus, pojmenovaný po americkém matematiku,
Josephu B. Kruskalovi. Jeho základní myšlenkou je prostě začít s grafem $K$
obsahujícím všechny vrcholy z $V$ a přidávat hrany od těch s nejnižším
ohodnocením po ty s nejvyšším tak dlouho, dokud nevznikne souvislý graf. Je
potřeba pouze dávat pozor na cykly. K tomu stačí si pamatovat stromy (jako
množiny vrcholů), které přidáváním hran vytvářím, a povolit přidání hrany
jedině v případě, že spojuje dva různé stromy.

Pro zápis v pseudokódu vizte \myref{algoritmus}{alg:kruskal}. Manim s průběhem
algoritmu s náhodně vygenerovaným hodnocením je k dispozici
\href{https://raw.githubusercontent.com/Tesser3kt/GEVO/main/Seminar/animations/media/videos/graphs/2160p60/SpanningTreeExample.mp4}{zde}.

\pagebreak

\begin{algorithm}
 \caption{Kruskalův algoritmus.}
 \label{alg:kruskal}

 \SetKwInOut{Input}{input}
 \SetKwInOut{Output}{output}
 \SetKw{KwReturn}{return}

 \Input{souvislý ohodnocený graf $G = (V,E,w)$, kde $V = \{v_1,\ldots,v_n\}$}
 \Output{množina hran $E'$ minimální kostry grafu $G$}
 \BlankLine
 \emph{Inicializace}\;
 $E' \leftarrow \emptyset$\;
 \emph{Množina hran, které nelze přidat (jinak by vznikl cyklus)}\;
 $X \leftarrow \emptyset$\;
 \For{$i \leftarrow 1$ \KwTo $n$} {
  \emph{Každý strom nejprve obsahuje pouze jediný vrchol}\;
  $T_i \leftarrow \{v_i\}$\;
 }
 \emph{Množina indexů pro pamatování sloučených stromů}\;
 $I \leftarrow \{1,\ldots,n\}$\;
 \BlankLine
 \emph{Přidávám hrany, dokud mám pořád víc než jeden strom}\;
 \While{$\#I > 1$} {
  $e \leftarrow \text{libovolná hrana s nejnižším } w(e) \text{, která není v }
  E' \text{ ani v } X$\;
  $i \leftarrow \text{index v } I \text{ takový, že } s(e) \in T_i$\;
  $j \leftarrow \text{index v } I \text{ takový, že } t(e) \in T_j$\;
  \BlankLine
  \If{$i = j$}{
   \emph{Hrana spojuje vrcholy ve stejném stromě, jejím přidáním by vznikl
   cyklus}\;
   $X \leftarrow X \cup \{e\}$\;
  }
  \Else {
   \emph{Hrana spojuje různé stromy. Přidávám ji do kostry}\;
   $E' \leftarrow E' \cup \{e\}$\;
   $T_i \leftarrow T_i \cup T_j$\;
   $I \leftarrow I \setminus \{j\}$\;
  }
 }
 \KwReturn $E'$\;
\end{algorithm}

\begin{claim}
 \label{claim:kruskal-korektni}
 Kruskalův algoritmus je korektní.
\end{claim}
\begin{proof}
 Potřebujeme ověřit, že
 \begin{enumerate}
  \item algoritmus provede pouze konečný počet kroků;
  \item algoritmus vrátí správnou odpověď.
 \end{enumerate}
 Případ (1) je zřejmý, protože $\# E < \infty$, čili algoritmus přidá do $E'$
 pouze konečně mnoho hran. Navíc, graf $G$ je z předpokladu souvislý, a tedy
 vždy existuje hrana $e$ spojující dva různé stromy $T_i$ a $T_j$.

 V případě (2) uvažme, že $K = (V,E',w)$ není minimální kostra $G$. V~takovém
 případě se mohlo stát, že
 \begin{enumerate}[label=(\alph*)]
  \item $K$ není strom -- tedy buď není souvislý nebo obsahuje cyklus;
  \item existuje podmnožina $E'' \subseteq E$ taková, že $K' = (V,E'',w)$ je strom a
   \[
    \sum_{e \in E''}^{} w(e) < \sum_{e \in E'}^{} w(e).
   \]
 \end{enumerate}
 Případ (a) lze vyloučit snadno, neboť souvislost $K$ plyne ihned ze
 souvislosti $G$, tedy, jak již bylo řečeno v bodě (1), vždy lze nalézt hranu
 spojující do té doby dva různé stromy. Pokud $K$ obsahuje cyklus, tak
 algoritmus musel v jednom kroku spojit hranou dva vrcholy ze stejného stromu,
 což je spor.

 Pokud nastal případ (b), pak musejí existovat hrany $e'' \in E'' \setminus E'$
 a $e' \in E' \setminus E''$ takové, že $w(e'') < w(e')$. Protože algoritmus
 zkouší přidávat hrany vždy počínaje těmi s nejmenší vahou, musel v nějakém
 kroku narazit na hranu $e''$ a zavrhnout ji. To ovšem znamená, že hrana $e''$ 
 spojila dva různé vrcholy téhož stromu a graf $K' = (V,E'',w)$ obsahuje
 cyklus. To je spor s předpokladem, že $K'$ je strom. Tedy, taková podmnožina
 $E'' \subseteq E$ nemůže existovat a $K$ je vskutku minimální kostra $G$.
\end{proof}

\begin{warning}
 Minimální kostra grafu \textbf{není jednoznačně určena}! Všimněte si, že na
 řádku 12 \hyperref[alg:kruskal]{Kruskalova algoritmu} volím
 \textbf{libovolnou} hranu, která má ze všech zatím nepřidaných nejnižší váhu a
 jejímž přidáním nevznikne cyklus.
\end{warning}

\begin{remark}
 V \hyperref[def:minimalni-kostra]{definici minimální kostry} požadujeme, aby
 $G$ byl souvislý graf. Pokud tomu však tak není, je pořád možné zkonstruovat
 minimální kostru pro každou část $G$, která souvislá je (pro každou jeho tzv.
 \emph{komponentu souvislosti}), zkrátka tím, že
 \hyperref[alg:kruskal]{Kruskalův algoritmus} spouštíme opakovaně.
\end{remark}

