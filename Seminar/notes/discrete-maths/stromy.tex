\subsection{Stromy}
\label{ssec:stromy}

Chvíli se budeme bavit o stromech -- ano, těch s listy a kořenem. Stromy jsou
speciální typy grafů, které se takto nazývají ne nadarmo. Jsou to totiž grafy, u
kterých si člověk může zvolit jakýsi \uv{počáteční} vrchol (zvaný \emph{kořen}),
z nějž se po cestě (ve smyslu \myref{definice}{def:cesta}) vždy dostane do
jednoho z \uv{koncových} vrcholů, tzv. \emph{listů}. Příklad stromu je na
\myref{obrázku}{fig:priklad-stromu}.

\begin{figure}[h]
 \centering
 \begin{tikzpicture}
  \tikzset{vertex/.style = {shape=circle,fill,text=white,minimum size=6pt,inner sep=1pt}}
  \tikzset{->-/.style={decoration={
   markings,
   mark=at position #1 with {\arrow{>[scale=0.8]}}},postaction={decorate}}}

  \node[vertex,myred,text=white] (1) at (0, 0) {};

  \node[vertex] (2) at (-2, -1) {};
  \node[vertex] (3) at (2, -1) {};

  \node[vertex] (4) at (-3, -2) {};
  \node[vertex] (5) at (-1, -2) {};
  \node[vertex] (6) at (1, -2) {};
  \node[vertex] (7) at (3, -2) {};

  \node[vertex,myblue] (8) at (-3.5, -3) {};
  \node[vertex,myblue] (9) at (-2.5, -3) {};
  \node[vertex,myblue] (10) at (-1.5, -3) {};
  \node[vertex,myblue] (11) at (-0.5, -3) {};

  \node[vertex,myblue] (12) at (0.5, -3) {};
  \node[vertex,myblue] (13) at (1.5, -3) {};
  \node[vertex,myblue] (14) at (2.5, -3) {};
  \node[vertex,myblue] (15) at (3.5, -3) {};

  \foreach \i in {2,3} {
   \draw[thick] (1) -- (\i);
  }
  \foreach \i in {4,5} {
   \draw[thick] (2) -- (\i);
  }
  \foreach \i in {6,7} {
   \draw[thick] (3) -- (\i);
  }
  \foreach \i in {8,9} {
   \draw[thick] (4) -- (\i);
  }
  \foreach \i in {10,11} {
   \draw[thick] (5) -- (\i);
  }
  \foreach \i in {12,13} {
   \draw[thick] (6) -- (\i);
  }
  \foreach \i in {14,15} {
   \draw[thick] (7) -- (\i);
  }
 \end{tikzpicture}
 \caption{Příklad stromu. Kořen je značen \clr{červeně} a listy \clb{modře}.}
 \label{fig:priklad-stromu}
\end{figure}

\begin{warning}
 Listy stromu jsou určeny jednoznačně jeho strukturou (jsou to ty jediné
 vrcholy, do nichž cesty od kořene mohou pouze vést a nikoli jimi procházet). Za
 kořen lze však volit libovolný vrchol, klidně i jeden z listů. Strom z
 \myref{obrázku}{fig:priklad-stromu} může proto vypadat jako na
 \myref{obrázku}{fig:strom-jiny-koren}.

 Někdy není ani nutno o kořenu a listech stromu hovořit, pokud je pro naše účely
 irelevantní rozlišovat jednotlivé vrcholy. Součástí definice stromu (kterou si
 záhy odvodíme) kořen ani listy nejsou.
\end{warning}

\begin{figure}[h]
 \centering
 \begin{tikzpicture}
  \tikzset{vertex/.style = {shape=circle,fill,text=white,minimum size=6pt,inner
  sep=1pt}}
  \tikzset{->-/.style={decoration={
   markings,
   mark=at position #1 with {\arrow{>[scale=0.8]}}},postaction={decorate}}}

  \node[vertex,myred,text=white] (2) at (0, 0) {};

  \node[vertex] (4) at (-3, -1) {};
  \node[vertex] (5) at (0, -1) {};
  \node[vertex] (1) at (3, -1) {};

  \node[vertex,myblue] (8) at (-4, -2) {};
  \node[vertex,myblue] (9) at (-2, -2) {};
  \node[vertex,myblue] (10) at (-1, -2) {};
  \node[vertex,myblue] (11) at (1, -2) {};
  \node[vertex] (3) at (3, -2) {};

  \node[vertex] (6) at (2, -3) {};
  \node[vertex] (7) at (4, -3) {};

  \node[vertex,myblue] (12) at (1.5, -4) {};
  \node[vertex,myblue] (13) at (2.5, -4) {};
  \node[vertex,myblue] (14) at (3.5, -4) {};
  \node[vertex,myblue] (15) at (4.5, -4) {};
  
  \foreach \i in {2,3} {
   \draw[thick] (1) -- (\i);
  }
  \foreach \i in {4,5} {
   \draw[thick] (2) -- (\i);
  }
  \foreach \i in {6,7} {
   \draw[thick] (3) -- (\i);
  }
  \foreach \i in {8,9} {
   \draw[thick] (4) -- (\i);
  }
  \foreach \i in {10,11} {
   \draw[thick] (5) -- (\i);
  }
  \foreach \i in {12,13} {
   \draw[thick] (6) -- (\i);
  }
  \foreach \i in {14,15} {
   \draw[thick] (7) -- (\i);
  }
 \end{tikzpicture}
 \caption{Strom z \myref{obrázku}{fig:priklad-stromu} s jinou volbou kořene.}
 \label{fig:strom-jiny-koren}
\end{figure}

Nyní si rozmyslíme dvě ekvivalentní definice stromu.

Za první podmínku, abychom mohli graf nazvat stromem, budeme považovat fakt, že
od kořene se dá dostat po hranách do každého z listů. Ekvivalentně, že z každého
vrcholu se dá cestou dostat do každého vrcholu, protože za kořen lze, jak jsme
nahlédli, volit kterýkoli vrchol, a cestu z~kořene do vrcholu můžeme zkrátit
tak, aby končila v nějakém vrcholu, jímž původně procházela. Grafy splňující
tuto podmínku slují \emph{souvislé}.

\begin{definition}[Souvislý graf]
\label{def:souvisly-graf}
 Graf $G = (V,E)$ nazveme \emph{souvislým}, pokud pro každé dva vrcholy $v,w \in
 V$ existuje cesta z $v$ do $w$, tedy cesta $v_1v_2 \cdots v_n$, kde $v_1 = v$ a
 $v_n = w$.
\end{definition}

Samotný název \uv{strom} plyne z faktu, že se jako graf pouze \uv{větví}, tedy
že při cestě směrem od (libovolného) kořene se jeden může pouze přibližovat k
listům, ale nikoli se dostat zpět blíže ke kořeni. To lze snadno zařídit tak, že
zakážeme cykly. Totiž, neexistuje-li v grafu cyklus, pak se po libovolné cestě z
kteréhokoli vrcholu můžeme od tohoto vrcholu pouze vzdalovat. Takové grafy
nazveme, přirozeně, \emph{acyklické}.

\begin{definition}[Acyklický graf]
\label{def:acyklicky-graf}
 Graf $G = (V,E)$ nazveme \emph{acyklický}, pokud neobsahuje cyklus o aspoň dvou
 vrcholech (samotné vrcholy jsou totiž z \hyperref[def:cyklus]{definice} vždy
 cykly).
\end{definition}

\begin{definition}[Strom]
\label{def:strom}
 Graf $G = (V,E)$ nazveme \emph{stromem}, je-li souvislý a acyklický.
\end{definition}

Na začátku sekce jsme slíbili ještě ekvivalentní definici stromu; ta je důvodem,
proč jsou stromy užitečné, především v informatice.

Ukazuje se totiž, že neexistence cyklů spolu se souvislostí způsobují, že mezi
dvěma vrcholy stromu vede vždy \textbf{přesně jedna cesta}. Po chvíli zamyšlení
snad toto nepřichází jako nijak divoké tvrzení. Přeci, pokud by mezi vrcholy
vedly cesty dvě, pak vrchol, kde se rozpojují, a vrchol, kde se opět spojují, by
byly součástí cyklu uvnitř stromu, který jsme výslovně zakázali. Třeba
překvapivější je fakt, že platí i opačná implikace.

\begin{claim}[Ekvivalentní definice stromu]
 \label{claim:ekvivalentni-definice-stromu}
 Graf $G = (V,E)$ je stromem ve smyslu \myref{definice}{def:strom} právě tehdy,
 když mezi každými dvěma vrcholy $G$ vede přesně jedna cesta.
\end{claim}

\begin{proof}
 Dokazujeme dvě implikace. Obě budeme dokazovat v jejich \emph{kontrapozitivní}
 formě, tedy jako obrácenou implikaci mezi negacemi výroků. Lidsky, dokážeme, že
 (1) když existují vrcholy, mezi kterými nevede žádná nebo vede více než jedna
 cesta, pak $G$ není strom, a (2) když $G$ není strom, tak existují vrcholy,
 mezi kterými nevede žádná cesta nebo vede více než jedna.
 \begin{enumerate}
  \item Pokud existují vrcholy, mezi kterými nevede cesta, pak $G$ není
   souvislý, což odporuje definici stromu. Budeme tedy předpokládat, že existují
   vrcholy $v,w$, mezi kterými vedou různé cesty $v_1 \cdots v_n$ a $v'_1 \cdots
   v'_m$, kde $v_1 = v'_1 = v$ a $v_n = v'_m = w$. Myšlenka důkazu je najít
   cyklus obsahující vrchol, kde se cesty rozdělují, a vrchol, kde se opět
   spojují. Vizte \myref{obrázek}{fig:cast-1-dukazu-definice-stromu}.

   Ať $r$ (od \textbf{r}ozpojení) je \textbf{největší} index takový, že $v_i =
   v'_i$ pro všechna $i \leq r$ (čili $v_r = v'_r$ je vrchol, ve kterém se cesty
   rozpojují). Ten určitě existuje, protože cesty se v nejhorším případě dělí už
   ve vrcholu $v$.

   Podobně, ať $s$ (od \textbf{s}pojení) je \textbf{nejmenší} index takový, že
   existuje $k \in \Z$ splňující $v_j = v'_{j+k}$ pro všechna $j \geq s$. Čili,
   vrchol $v_s = v'_{s+k}$ je vrchol, ve kterém se cesty opět spojily. Ovšem,
   mohlo se tak stát v okamžiku, kdy jsme po jedné cestě prošli více nebo méně
   vrcholů než po druhé -- tento počet vyjadřuje ono číslo $k$. Takový vrchol
   jistě existuje, v nejhorším je to přímo koncový vrchol $w$.

   Zřejmě platí $r < s$, jinak by cesty nebyly různé. Potom je ovšem například
   posloupnost vrcholů
   \[
    (v_r,v_{r+1},\ldots,v_s = v'_{s+k},v'_{s+k-1},\ldots,v'_r = v_r)
   \]
   cyklem v $G$. Tedy ani v tomto případě $G$ není strom.
  \item Když $G$ není strom, tak není souvislý nebo obsahuje cyklus. Když $G$
   není souvislý, tak existují vrcholy, mezi nimiž nevede v $G$ cesta, což
   protiřečí podmínce, aby mezi každým párem vrcholů vedla přesně jedna.

   Budeme tedy předpokládat, že $G$ obsahuje cyklus $v_1v_2 \cdots v_n$ (tedy
   $v_n = v_1$ a $n \geq 2$). Pak ovšem pro libovolné indexy $i < j \leq n$ jsou
   posloupnosti
   \[
    (v_i,v_{i+1},\ldots,v_j) \quad \text{a} \quad (v_i,v_{i-1},\ldots,v_1 =
    v_n,v_{n-1},\ldots,v_j)
   \]
   dvě různé cesty mezi $v_i$ a $v_j$. Vizte
   \myref{obrázek}{fig:cast-2-dukazu-definice-stromu}.
 \end{enumerate}
 Tím je důkaz dokončen.
\end{proof}

\begin{figure}[h]
 \centering
 \begin{subfigure}{.45\textwidth}
  \centering
  \begin{tikzpicture}
   \tikzset{vertex/.style = {shape=circle,fill,text=white,minimum size=6pt,inner
   sep=1pt}}
   \tikzset{->-/.style={decoration={
    markings,
    mark=at position #1 with {\arrow{>[scale=1]}}},postaction={decorate}}}

   \node[vertex] (1) at (0,0) {};
   \node[vertex] (2) at (0.5,-1) {};
   \node[vertex,myred,minimum size=9pt] (3) at (0.5,-2) {};

   \node[vertex] (4) at (-0.5,-3) {};

   \node[vertex] (5) at (1,-2.5) {};
   \node[vertex] (6) at (1,-3.5) {};

   \node[vertex,myblue,minimum size=9pt] (7) at (0.25,-4) {};
   \node[vertex] (8) at (-1,-4.5) {};

   \draw[->-=.55,thick] (1) -- (2);
   \draw[->-=.55,thick] (2) -- (3);
   \draw[->-=.55,thick] (3) -- (4);
   \draw[->-=.55,thick] (3) -- (5);
   \draw[->-=.55,thick] (4) -- (7);
   \draw[->-=.55,thick] (5) -- (6);
   \draw[->-=.55,thick] (6) -- (7);
   \draw[->-=.55,thick] (7) -- (8);

   \node[right = -.5mm of 1] {$v = v_1 = v'_1$};
   \node[right = -.5mm of 2] {$v_2 = v'_2$};
   \node[above left = -3mm and -.5mm of 3,text=myred] {$v_3 = v'_3$};
   \node[left = -.5mm of 4]{$v_4$};
   \node[right = -.5mm of 5]{$v'_4$};
   \node[right = -.5mm of 6]{$v'_5$};
   \node[below right = -3mm and -.5mm of 7,text=myblue]{$v_5 = v'_6$};
   \node[left = -.5mm of 8]{$v'_7 = v_6 = w$};
  \end{tikzpicture}
  \caption{Část (1) důkazu \myref{tvrzení}{claim:ekvivalentni-definice-stromu}.
  Zde $\clr{r = 3}, \clb{s = 5}$ a $k = 1$. Sestrojený cyklus je
  $v_3v_4v_5v'_5v'_4v_3'$.}
  \label{fig:cast-1-dukazu-definice-stromu}
 \end{subfigure}
 \hfill
 \begin{subfigure}{.45\textwidth}
  \centering
  \begin{tikzpicture}
   \tikzset{vertex/.style = {shape=circle,fill,text=white,minimum size=6pt,inner
   sep=1pt}}
   \tikzset{->-/.style={decoration={
    markings,
    mark=at position #1 with {\arrow{>[scale=1]}}},postaction={decorate}}}

   \node[vertex] (1) at (0,0) {};
   \node[vertex] (2) at (0.5,-1) {};
   \node[vertex] (3) at (0.5,-2) {};

   \node[vertex,minimum size=9pt,myred] (4) at (-0.5,-3) {};

   \node[vertex] (5) at (1,-2.5) {};
   \node[vertex,minimum size=9pt,myblue] (6) at (1,-3.5) {};

   \node[vertex] (7) at (0.25,-4) {};
   \node[vertex] (8) at (-1,-4.5) {};

   \draw[thick] (1) -- (2);
   \draw[thick] (2) -- (3);
   \draw[->-=.55,thick] (4) -- (3);
   \draw[->-=.55,thick] (3) -- (5);
   \draw[->-=.55,thick] (4) -- (7);
   \draw[->-=.55,thick] (5) -- (6);
   \draw[->-=.55,thick] (7) -- (6);
   \draw[thick] (7) -- (8);

   \node[above left = -3mm and -.5mm of 3] {$v_6 = v_1$};
   \node[left = -.5mm of 4,text=myred]{$v_2$};
   \node[right = -.5mm of 5]{$v_5$};
   \node[right = -.5mm of 6,text=myblue]{$v_4$};
   \node[below right = -2mm and -.5mm of 7]{$v_3$};
  \end{tikzpicture}
  \caption{Část (2) důkazu \myref{tvrzení}{claim:ekvivalentni-definice-stromu}.
  Zde $\clr{i = 2}, \clb{j = 4}$ a sestrojené cesty jsou $v_2v_3v_4$ a
  $v_2v_1v_5v_4$.}
  \label{fig:cast-2-dukazu-definice-stromu}
 \end{subfigure}
 \caption{Ilustrace k důkazu
 \myref{tvrzení}{claim:ekvivalentni-definice-stromu}.}
 \label{fig:ilustrace-k-ekvivalentni-definici-stromu}
\end{figure}
