\subsubsection{Minimální kostra}
\label{sssec:minimalni-kostra}

Ne všechny hrany jsou si rovny. Kterési se rodí krátké, jiné dlouhé; kterési
štíhlé, jiné otylé; kterési racionální, jiné iracionální.

Často nastávají situace, kdy jeden potřebuje hranám grafu přiřadit nějakou
hodnotu, obvykle číselnou, která charakterizuje klíčovou vlastnost této hrany.
Při reprezentaci dopravní sítě grafem to může být délka silnice či její
vytížení, při reprezentaci elektrických obvodů pak například odpor. V teorii
grafů takové přiřazení hodnoty hranám grafu sluje \emph{ohodnocení}.

\begin{definition}[Ohodnocený graf]
\label{def:ohodnoceny-graf}
 Ať $G = (V,E)$ je graf. Libovolné zobrazení $w: E \to \R^{+} = (0,\infty)$ 
 nazveme \emph{ohodnocením} grafu $G$. Trojici $(V,E,w)$, kde $w$ je ohodnocení
 $G$, nazveme \emph{ohodnoceným grafem}.
\end{definition}

\begin{remark}
 \label{rmrk:ohodnoceni-neohodnoceneho}
 Každý graf $G = (V,E)$ lze triviálně ztotožnit s ohodnoceným grafem $(V,E,w)$,
 kde $w: E \to \R^{+}$ je konstantní zobrazení. Obvykle se volí konkrétně $w
 \equiv 1$, tedy zobrazení $w$ takové, že $w(e) = 1$ pro každou $e \in E$.
\end{remark}

Porozumění struktuře ohodnocených grafů může odpovědět na spoustu zajímavých
(jak prakticky tak teoreticky) otázek. Můžeme se kupříkladu ptát, jak se nejlépe
(vzhledem k danému ohodnocení) dostaneme cestou z~jednoho vrcholu do druhého.
Slovo \uv{nejlépe} zde chápeme pouze intuitivně. V závislosti na zpytovaném
problému můžeme požadovat, aby cesta třeba minimalizovala či maximalizovala
součet hodnot všech svých hran přes všechny možné cesty mezi danými vrcholy.
Jsou však i případy, kdy člověk hledá cestu, která je nejblíže \uv{průměru}.

Abychom pořád neříkali \uv{součet přes všechny hrany cesty}, zavedeme si pro
toto často zkoumané množství název \emph{váha cesty}. Čili, je-li $\mathcal{P}
\coloneqq e_1 \cdots e_n$ cesta v nějakém ohodnoceném grafu $G$, pak její vahou
rozumíme výraz
\[
 w(\mathcal{P}) \coloneqq \sum_{i=1}^{n} w(e_i).
\]
Zápis $w(\mathcal{P})$ můžeme vnímat buď jako zneužití zavedeného značení, nebo
jako fakt, že jsme zobrazení $w$ rozšířili z množiny všech hran na množinu všech
cest v grafu $G$ (kde samotné hrany jsou z \hyperref[def:cesta]{definice} též
cesty).

\begin{remark}
 Záměrně jsme užili slovního spojení \emph{váha} cesty místo snad přirozenějšího
 \emph{délka} cesty. V teorii grafů se totiž délkou cesty myslí obyčejně počet
 hran (nebo vrcholů), které obsahuje. Délka cesty $\mathcal{P} = e_1 \cdots e_n$
 je tudíž $n$ (resp. $n + 1$), bo obsahuje $n$ hran (resp. $n + 1$ vrcholů).

 Tento úzus svědčí účelu \hyperref[rmrk:ohodnoceni-neohodnoceneho]{předchozí
 poznámky}. Pokud totiž každý graf bez ohodnocení vnímáme vlastně jako
 ohodnocený graf, kde každá hrana má váhu přesně $1$, pak váha každé cesty je
 rovna její délce.
\end{remark}

První (a nejjednodušší) problém, kterým se budeme zabývat, je nalezení
\emph{minimální kostry} (angl. \emph{spanning tree}).

\begin{definition}[Minimální kostra]
\label{def:minimalni-kostra}
 Ať $G = (V,E,w)$ je \textbf{souvislý} ohodnocený graf. Ohodnocený graf $K =
 (V',E',w)$ nazveme \emph{minimální kostrou} grafu $G$, pokud je souvislý, $V' =
 V$ (tedy $K$ obsahuje všechny vrcholy $G$), $E' \subseteq E$ a
 \[
  \sum_{e \in E'}^{} w(e)
 \]
 je minimální vzhledem ke všem možným volbám podmnožiny $E' \subseteq E$.
 Lidsky řečeno, graf $K$ spojuje všechny vrcholy $G$ tím \uv{nejlevnějším}
 způsobem vzhledem k ohodnocení $w$.
\end{definition}

\begin{figure}[h]
\centering
 \begin{tikzpicture}[scale=2]
  \tikzset{vertex/.style = {shape=circle,fill,text=white,minimum size=9pt,inner
  sep=1pt}}
  \tikzset{->-/.style={decoration={ markings, mark=at position #1 with
  {\arrow{>[scale=1]}}},postaction={decorate}}}
  \foreach \weightx/\x in {3/0, 4/1, 1/2} {
   \foreach \weighty/\y [evaluate=\weighty as \weight using {int(\weightx +
   \weighty)}] in {1/0, 2/1, 0/2} {
    \draw[thick] (\x,\y) to node[above,color=myblue] {$\weightx$} (\x+1,\y);
    \draw[thick] (\x,\y) to node[right,color=myblue] {$\weight$} (\x,\y+1);
   }
  }
  \foreach \weight/\x in {1/0, 2/1, 4/2} {
   \draw[thick] (\x,3) to node[above,color=myblue] {$\weight$} (\x+1,3);
  }
  \foreach \weight/\y in {5/0, 1/1, 3/2} {
   \draw[thick] (3,\y) to node[right,color=myblue] {$\weight$} (3,\y+1);
  }

  \draw[line width=1mm, color=myred] (0, 3) -- (1, 3);
  \draw[line width=1mm, color=myred] (2, 2) -- (2, 3);
  \draw[line width=1mm, color=myred] (2, 2) -- (3, 2);
  \draw[line width=1mm, color=myred] (2, 0) -- (3, 0);
  \draw[line width=1mm, color=myred] (2, 1) -- (3, 1);
  \draw[line width=1mm, color=myred] (3, 1) -- (3, 2);

  \draw[line width=1mm, color=myred] (2, 0) -- (2, 1);
  \draw[line width=1mm, color=myred] (1, 3) -- (2, 3);

  \draw[line width=1mm, color=myred] (3, 3) -- (3, 2);
  \draw[line width=1mm, color=myred] (0, 3) -- (0, 2);
  \draw[line width=1mm, color=myred] (0, 2) -- (1, 2);
  \draw[line width=1mm, color=myred] (0, 1) -- (1, 1);
  \draw[line width=1mm, color=myred] (0, 0) -- (1, 0);

  \draw[line width=1mm, color=myred] (1, 0) -- (2, 0);
  \draw[line width=1mm, color=myred] (0, 0) -- (0, 1);

  \foreach \x in {0, 1, 2, 3} {
   \foreach \y in {0, 1, 2, 3} {
    \node[vertex,fill=myred] (v\x\y) at (\x, \y) {};
   }
  }
 
 \end{tikzpicture}
 \caption{\clr{Minimální kostra} grafu s ohodnocením \clb{$w$}.}
 \label{fig:minimalni-kostra}
\end{figure}

\begin{observation}
 Minimální kostra ohodnoceného grafu je strom.
\end{observation}
\begin{proof}
 Kdyby minimální kostra nebyla strom, pak buď není souvislá, což jsme výslovně
 zakázali, nebo obsahuje cyklus. Tudíž se mezi nějakými dvěma vrcholy dá jít po
 více než jedné cestě, a proto můžeme přinejmenším jednu hranu z kostry
 odebrat. Protože každá hrana má kladné ohodnocení, snížili jsme tím součet
 hodnot všech hran. To je spor.
\end{proof}

\begin{corollary}
 Minimální kostra ohodnoceného stromu je s ním totožná.
\end{corollary}

Minimální kostra je zvlášť užitečná právě při návrhů elektrických obvodů, kdy je
potřeba zařídit, aby všechna připojená zařízení čerpala co nejmenší množství
energie. Protože elektřina proudí rychlostí světla, délka kabelu (pokud není
zrovna mezigalaktický) nás příliš netrápí, ale právě odpor či kvalita/vodivost
konkrétních spojů by mohly.

Další praktickou grafovou úlohou vedoucí na problém nalezení minimální kostry je
potřeba spojit vzdálené servery. Korporace mají obvykle mnoho různých serverů
rozmístěných po světě, jež spolu ale musejí sdílet data. Problém je v tom, že
propojení mezi servery by nejen mělo vést k nejmenší možné prodlevě při přenosu
dat (od toho \textbf{minimální}), ale nesmí ani obsahovat cykly (od toho
\textbf{kostra}). Kdyby totiž cykly obsahovalo, pak by se při přenosu dat stalo,
že by aspoň jeden server v tomto cyklu dostal aspoň dvakrát stejnou informaci z
dvou různých zdrojů, ale v odlišný čas. Taková situace vede nezbytně dříve nebo
později ke korupci dat; řekněme, když daný server už s obdrženou informací začal
po přijetí provádět výpočet. Pro více detailů k tomuto využití minimálních
koster vizte
\href{https://en.wikipedia.org/wiki/Spanning_Tree_Protocol}{Spanning Tree
Protocol}.
