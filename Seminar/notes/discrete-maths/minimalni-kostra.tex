\subsubsection{Minimální kostra}
\label{sssec:minimalni-kostra}

Ne všechny hrany jsou si rovny. Kterési se rodí krátké, jiné dlouhé; kterési
štíhlé, jiné otylé; kterési racionální, jiné iracionální.

Často nastávají situace, kdy jeden potřebuje hranám grafu přiřadit nějakou
hodnotu, obvykle číselnou, která charakterizuje klíčovou vlastnost této hrany.
Při reprezentaci dopravní sítě grafem to může být délka silnice či její
vytížení, při reprezentaci elektrických obvodů pak například odpor. V teorii
grafů takové přiřazení hodnoty hranám grafu sluje \emph{ohodnocení}.

\begin{definition}[Ohodnocený graf]
\label{def:ohodnoceny-graf}
 Ať $G = (V,E)$ je graf. Libovolné zobrazení $w: E \to \R^{+} = (0,\infty)$ 
 nazveme \emph{ohodnocením} grafu $G$. Trojici $(V,E,w)$, kde $w$ je ohodnocení
 $G$, nazveme \emph{ohodnoceným grafem}.
\end{definition}

\begin{remark}
 Každý graf $G = (V,E)$ lze triviálně ztotožnit s ohodnoceným grafem $(V,E,w)$,
 kde $w: E \to \R^{+}$ je konstantní zobrazení. Obvykle se volí konkrétně $w
 \equiv 1$, tedy zobrazení $w$ takové, že $w(e) = 1$ pro každou $e \in E$.
\end{remark}

Porozumění struktuře ohodnocených grafů může odpovědět na spoustu zajímavých
(jak prakticky tak teoreticky) otázek. Můžeme se kupříkladu ptát, jak se nejlépe
(vzhledem k danému ohodnocení) dostaneme cestou z jednoho vrcholu do druhého.
Slovo \uv{nejlépe} zde chápeme pouze intuitivně. V závislosti na zpytovaném
problému můžeme požadovat, aby cesta třeba minimalizovala či maximalizovala
součet hodnot všech svých hran přes všechny možné cesty mezi danými vrcholy.
Jsou však i případy, kdy člověk hledá cestu, která je nejblíže \uv{průměru}.

Abychom pořád neříkali \uv{součet přes všechny hrany cesty}, zavedeme si pro
toto často zkoumané množství název \emph{váha cesty}. Čili, je-li $\mathcal{P}
\coloneqq e_1 \cdots e_n$ cesta v nějakém ohodnoceném grafu $G$, pak její vahou
rozumíme výraz
\[
 w(\mathcal{P}) \coloneqq \sum_{i=1}^{n} w(e_i).
\]
Zápis $w(\mathcal{P})$ můžeme vnímat buď jako zneužití zavedeného značení, nebo
jako fakt, že jsme zobrazení $w$ rozšířili z množiny všech hran na množinu všech
cest v grafu $G$ (kde samotné hrany jsou z \hyperref[def:cesta]{definice} též
cesty).

První takový problém, kterým se budeme zabývat, je nalezení \emph{minimální
kostry} (angl. \emph{spanning tree}).

\begin{definition}[Minimální kostra]
\label{def:minimalni-kostra}
 Ať $G = (V,E,w)$ je \textbf{souvislý} ohodnocený graf. Ohodnocený graf $K =
 (V',E',w)$ nazveme \emph{minimální kostrou} grafu $G$, pokud je souvislý, $V' =
 V$ (tedy $K$ obsahuje všechny vrcholy $G$), $E' \subseteq E$ a
 \[
  \sum_{e \in E'}^{} w(e)
 \]
 je minimální vzhledem ke všem možným volbám podmnožiny $E' \subseteq E$.
 Lidsky řečeno, graf $K$ spojuje všechny vrcholy $G$ tím \uv{nejlevnějším}
 způsobem vzhledem k ohodnocení $w$.
\end{definition}

\begin{figure}[h]
\centering
 \begin{tikzpicture}[scale=2]
  \tikzset{vertex/.style = {shape=circle,fill,text=white,minimum size=9pt,inner
  sep=1pt}}
  \tikzset{->-/.style={decoration={ markings, mark=at position #1 with
  {\arrow{>[scale=1]}}},postaction={decorate}}}
  \foreach \weightx/\x in {3/0, 4/1, 1/2} {
   \foreach \weighty/\y [evaluate=\weighty as \weight using {int(\weightx +
   \weighty)}] in {1/0, 2/1, 0/2} {
    \draw[thick] (\x,\y) to node[above,color=myblue] {$\weightx$} (\x+1,\y);
    \draw[thick] (\x,\y) to node[right,color=myblue] {$\weight$} (\x,\y+1);
   }
  }
  \foreach \weight/\x in {1/0, 2/1, 4/2} {
   \draw[thick] (\x,3) to node[above,color=myblue] {$\weight$} (\x+1,3);
  }
  \foreach \weight/\y in {5/0, 1/1, 3/2} {
   \draw[thick] (3,\y) to node[right,color=myblue] {$\weight$} (3,\y+1);
  }

  \draw[line width=1mm, color=myred] (0, 3) -- (1, 3);
  \draw[line width=1mm, color=myred] (2, 2) -- (2, 3);
  \draw[line width=1mm, color=myred] (2, 2) -- (3, 2);
  \draw[line width=1mm, color=myred] (2, 0) -- (3, 0);
  \draw[line width=1mm, color=myred] (2, 1) -- (3, 1);
  \draw[line width=1mm, color=myred] (3, 1) -- (3, 2);

  \draw[line width=1mm, color=myred] (2, 0) -- (2, 1);
  \draw[line width=1mm, color=myred] (1, 3) -- (2, 3);

  \draw[line width=1mm, color=myred] (3, 3) -- (3, 2);
  \draw[line width=1mm, color=myred] (0, 3) -- (0, 2);
  \draw[line width=1mm, color=myred] (0, 2) -- (1, 2);
  \draw[line width=1mm, color=myred] (0, 1) -- (1, 1);
  \draw[line width=1mm, color=myred] (0, 0) -- (1, 0);

  \draw[line width=1mm, color=myred] (1, 0) -- (2, 0);
  \draw[line width=1mm, color=myred] (0, 0) -- (0, 1);

  \foreach \x in {0, 1, 2, 3} {
   \foreach \y in {0, 1, 2, 3} {
    \node[vertex,fill=myred] (v\x\y) at (\x, \y) {};
   }
  }
 
 \end{tikzpicture}
 \caption{\clr{Minimální kostra} grafu s ohodnocením \clb{$w$}.}
 \label{fig:minimalni-kostra}
\end{figure}

\begin{observation}
 Minimální kostra ohodnoceného grafu je strom.
\end{observation}
\begin{proof}
 Kdyby minimální kostra nebyla strom, pak buď není souvislá, což jsme výslovně
 zakázali, nebo obsahuje cyklus. Tudíž se mezi nějakými dvěma vrcholy dá jít po
 více než jedné cestě, a proto můžeme přinejmenším jednu hranu z kostry
 odebrat. Protože každá hrana má kladné ohodnocení, snížili jsme tím součet
 hodnot všech hran. To je spor.
\end{proof}

\begin{corollary}
 Minimální kostra ohodnoceného stromu je s ním totožná.
\end{corollary}

Minimální kostra je zvlášť užitečná v tzv. \uv{facility location problems},
v~překladu přibližně \uv{úlohy umístění střediska}. Při výstavbě nové
nemocnice, školy, elektrárny apod. je třeba zařídit, aby jí žádné cílové
objekty (obce, okresy, domy, \ldots) nebyly příliš vzdáleny.

Držme se příkladu nemocnice. Sídliště (či obecně části města s vysokou
koncentrací obyvatel), o nichž se očekává, že svým působením pokryje, můžeme
reprezentovat jako vrcholy ohodnoceného grafu. Další vrcholy tohoto grafu budou
význačné dopravní uzly, přes které se cestou mezi sídlišti projíždí (z jednoho
sídliště k dopravnímu uzlu či k jinému sídlišti může samozřejmě existovat více
různých cest). Hrany budou v tomto případě nejkratší silniční spoje mezi
vrcholy a ohodnocené budou průměrným časem jízdy.

Ještě, než začneme hledat umístění nemocnice, uvědomíme si, že musí ležet na
minimální kostře takového grafu. Prve si tedy ukážeme, jak vůbec hledat
minimální kostru grafu, a pak zformalizujeme pojem \uv{ideálního} umístění
nemocnice.


