\subsection{Graf jako algebraická struktura}
\label{ssec:graf-jako-algebraicka-struktura}

Snad poněkud tajemný název sekce v sobě skrývá jiný pohled na graf, než jsme
chovali doposud. Mimo jejich využití v modelování systémů, které lze
reprezentovat jako sítě uzlů a spojnic, jsou grafy též velmi užitečné ve více
\uv{abstraktních} údech matematiky. Představují totiž v jistém smyslu
\emph{nejvolnější} strukturu na množině, která je vůbec ještě užitečná.

Abychom osvětlili, co tímto výrokem míníme, uvážíme \emph{ještě další} pohled na
hrany grafu $G$. Záměrně jsme teď neuvedli množinu hran, neb o nich vůbec
nechceme takto přemýšlet. Samozřejmě, stále potřebujeme mít nějakou množinu $V$,
na níž onu strukturu stavíme; tu někdy přezdíváme \emph{bázovou}, protože
skutečně tvoří jakýsi \uv{základ} sestrojené struktury.

Intuitivně je příjemné nahlížet na množinové struktury jako na stavebnice.
Bázová množina jsou její díly (každý máme k dispozici, kolikrát chce\-me) a
způsob, kterým do sebe díly zapadají, je právě ona struktura.

Aniž si to pravděpodobně uvědomujete, narazili jste už v matematice na celou
řadu struktur. Jako příklad uveďme množinu celých čísel $\Z$ s operací
násobení. Zde $\Z$ je bázová množina a operace $ \cdot $ je struktura na $\Z$.
Definuje \emph{jeden možný způsob}, jak do sebe díly nazvané celá čísla
zapadají. Možná jste někdy přemýšleli o tom, co vlastně znamená slovo
\uv{operace}. V případě $ \cdot $ hovoříme o \textbf{binární} operaci, tedy o
operaci na \textbf{dvou prvcích}. Přirozeně, nic nám nebrání definovat si
operace na libovolném počtu celých čísel. Asi nejvíce přímočarý způsob, jak
definovat (binární) operaci, je přes zobrazení. Čili, $ \cdot $ je zobrazení
\[
 \cdot :\Z \times \Z \to \Z,
\]
které každou dvojici $(x,y) \in \Z \times \Z$ zobrazí na $x \cdot y$.

Zkusme teď podobným způsobem definovat hrany na množině $V$. Binární operace na
$V$ by znamenala zobrazení $V \times V \to V$, tedy zobrazení, jež dvěma
vrcholům přiřadí třetí vrchol. To asi není v tomto případě příliš směrodatné.
Hrany však vždy existují mezi dvěma vrcholy, tedy volba množiny $V \times V$
jako domény se zdá smysluplná. Otázkou je, co má být kodoménou. Jedna možnost by
byla opravdu uvážit nějakou další množinu hran $E$ a směřovat zobrazení do ní.
Tento pohled je vlastně opačný k~našemu \hyperref[def:graf-potreti]{třetímu
pojetí grafu}. Místo toho, abychom přiřazovali dva vrcholy jedné hraně, tak
přiřazujeme hranu páru vrcholů.

Existuje však mnohem přímější způsob. Přeci, abych dal najevo, že mezi párem
vrcholů vede hrana, nemusím vybírat žádnou \emph{konkrétní} hranu z~předem
definované množiny. Všechny hrany jsou stejné! Stačí mi si pouze u každých dvou
vrcholů pamatovat, jestli mezi nimi vede hrana, či nikoliv. To jest, stačí mi
libovolná dvouprvková podmnožina, jejíž jeden prvek znamená \uv{Mezi těmito
vrcholy nevede hrana,} a ten druhý naopak \uv{Mezi těmito vrcholy hrana vede}.

Obvyklou volbou (zvláště v informatice) je množina $\{0,1\}$, kde $0$ tradičně
značí, že hrana neexistuje, a $1$, že ano. Příznivci čisté logiky možná uvítají
množinu $\{\bot,\top\}$, kde $\bot$ je logická konstanta \uv{lež} a $\top$ je
logická konstanta \uv{pravda}. My se budeme držet čtenářům spíše přirozenější
volby, $\{0,1\}$.

Čili, \emph{hranami} mezi vrcholy z množiny $V$ myslíme strukturu na $V$ danou
zobrazením
\[
 \varepsilon:V \times V \to \{0,1\}.
\]
Konečně vysvětlíme, co myslíme tím, že takováto struktura je v podstatě
nejvolnější možná. Totiž, když pro libovolnou dvojici $v,w \in V$ změníme obraz
$\varepsilon(v,w)$ třeba z $0$ na $1$, stále tím dostaneme validní strukturu
hran na $V$ ve smyslu svojí definice. Čili, \textbf{úplně každé} zobrazení $V
\times V \to \{0,1\}$ postaví strukturu na množině $V$. To je intuitivně
ekvivalentní tomu, že každé dva díly stavebnice do sebe zapadají jakýmkoli
způsobem. Je zřejmé, že \uv{volnější} stavebnice než taková už neexistuje.

Naopak, vraťme se k příkladu celých čísel s operací násobení. Co by se stalo,
kdybychom se ráno vzbudili a rozhodli se, že odteď $2 \cdot 3 = 5$, ale veškeré
ostatní vlastnosti násobení (jako třeba i komutativita a asociativita) zůstanou
beze změny? Tato jedna úprava by zcela rozbourala celou strukturu násobení na
$\Z$, protože najednou by například nebylo možné definovat sudá a lichá čísla
($2$ by dělila všechny násobky $5$), $6$ by byla prvočíslo (její rozklad býval
$2 \cdot 3$), $10$ by se rovnalo $12$, jelikož
\[
 10 = 2 \cdot 5 = 2 \cdot (2 \cdot 3) = (2 \cdot 2) \cdot 3 = 4 \cdot 3 = 12,
\]
a způsobila nekonečně mnoho dalších trhlin. Intuitivně, násobení na $\Z$ je
stavebnice, kde do sebe každé dva díly zapadají přesně jediným způsobem.

Po tomto neformálním úvodu se na chvíli ponoříme do hlubin struktury zvané
\emph{graf}, povíme si, co znamená \uv{podstruktura}, že graf se dá též vnímat
jako prostor, a že můžeme přes zobrazení skákat mezi různými grafy.

V následujícím nemusíme hovořit o hranách jako o zobrazení $\varepsilon:V \times
V \to \{0,1\}$ popsaném výše. Naše původní představa množiny $E$ postačuje. Není
však špatné tuto myšlenku uchovat v hlavě, k čemuž slouží následující, extrémně
snadné, cvičení.

\begin{exercise}
 Rozmyslete si, jak upravit zobrazení
 \[
  \varepsilon:V \times V \to \{0,1\}
 \]
 definující hranovou strukturu na množině vrcholů tak, aby zahrnovalo i všechny
 \textbf{ohodnocené} grafy.
\end{exercise}

\subsubsection{Podgrafy, souvislost a metrika}
\label{sssec:podgrafy-souvislost-a-metrika}

V této relativně krátké podsekci dáme formální tvář představě, že
\begin{enumerate}
 \item nějaký graf je \uv{uvnitř} druhého;
 \item graf je souvislý a rozpadá se na tzv. \emph{komponenty souvislosti}, tedy
  maximální souvislé části;
 \item graf je \emph{metrický} prostor, tedy prostor, ve kterém lze měřit
  vzdálenosti.
\end{enumerate}

Začneme bodem (1), vedoucím na pojem \emph{podgrafu}. Jeho definice je opravdu
nejjednodušší možná, požadujeme pouze, aby vrcholy a hrany podgrafu tvořili
podmnožinu vrcholů a hran většího grafu.

\begin{definition}[Podgraf]
 \label{def:podgraf}
 Ať $G = (V,E)$ je graf. Řekneme, že graf $H = (V',E')$ je \emph{podgrafem} $G$,
 pokud
 \[
  V' \subseteq V \quad \text{a} \quad E' \subseteq E.
 \]
\end{definition}

\begin{remark}
 Žádáme čtenáře, aby sobě povšimli, že podgraf \textbf{nemusí zachovat} hranovou
 strukturu svého nadgrafu. Přesněji, \hyperref[def:podgraf]{definice podgrafu}
 neobsahuje podmínku, že mezi vrcholy $H$ musí vést hrana, pokud mezi těmi
 samými vrcholy v $G$ hrana vedla.

 Taková definice je z pohledu algebraika zhola zbytečná, bať odpudivá. Hranová
 struktura je zásadní součástí definice grafu a měla by být dodržena. Z tohoto
 důvodu se nehodí říkat, že by podgraf byl \emph{podstrukturou} svého nadgrafu,
 kterakžekolivěk vágně je ono slovo vyloženo.
\end{remark}

Předchozí poznámka motivuje definici \emph{indukovaného podgrafu}, podgrafu,
jemuž je přikázáno původní strukturu zachovat. Žargonový výraz \uv{indu\-kovaný}
v~tomto kontextu obyčejně znamená přibližně \uv{plynoucí z}. Čili, indukovaný
podgraf je podgraf, jehož struktura \textbf{plyne ze} struktury vyššího grafu.

Tuto podmínku lze formulovat snadno. Díváme-li se na hrany jako na podmnožiny
systému dvouprvkových podmnožin $V$, tedy jako na množinu $E \subseteq
\binom{V}{2}$, pak požadavek, aby nějaký podgraf $H = (V',E')$ grafu $G = (V,E)$
obsahoval spolu s každou dvojicí vrcholů i hranu mezi nimi, pokud je v~$E$, lze
vyjádřit zkrátka tak, že nařídíme, aby $E' = E \cap \binom{V'}{2}$, tedy aby
$E'$ byla vlastně množina $E$, ve které necháme jen ty hrany, které vedou mezi
vrcholy z $V'$.

\begin{definition}[Indukovaný podgraf]
 \label{def:indukovany-podgraf}
 Ať $G = (V,E)$ je graf a $H = (V',E')$ jeho podgraf. Řekneme, že $H$ je
 \emph{indukovaný} (grafem $G$), pokud \textbf{zachovává hranovou strukturu na
 $G$}, to jest,
 \[
  E' = E \cap \binom{V'}{2}.
 \]
\end{definition}

\begin{remark}
 Je dlužno nahlédnout, že indukovaný podgraf grafu $G = (V,E)$ je jednoznačně
 určen svoji množinou vrcholů. Totiž, vyberu-li z $V$ podmnožinu $V'$, pak mezi
 všemi vrcholy z $V'$ musejí v indukovaném podgrafu vést všechny hrany, které
 mezi nimi vedou i v původním grafu. Množina $E'$ je tudíž kompletně popsána
 množinami $V'$ a $E$.
\end{remark}

\begin{figure}[h]
 \centering
 \begin{subfigure}{.47\textwidth}
  \centering
  \begin{tikzpicture}
   \node[vertex] (v1) at (30:2) {};
   \node[vertex,myred,minimum size=11pt] (v2) at (90:2) {};
   \node[vertex,myred,minimum size=11pt] (v3) at (150:2) {};
   \node[vertex,myred,minimum size=11pt] (v4) at (210:2) {};
   \node[vertex,myred,minimum size=11pt] (v5) at (270:2) {};
   \node[vertex] (v6) at (330:2) {};

   \draw[thick] (v1) -- (v2);
   \draw[thick] (v2) -- (v3);
   \draw[ultra thick,myred] (v3) -- (v4);
   \draw[thick] (v4) -- (v5);
   \draw[thick] (v5) -- (v6);
   \draw[thick] (v6) -- (v1);
   
   \draw[thick] (v1) -- (v4);
   \draw[thick] (v3) -- (v6);
   \draw[ultra thick,myred] (v2) -- (v5);
  \end{tikzpicture}
  \caption{\clr{Podgraf}, který není indukovaný.}
  \label{subfig:neindukovany-podgraf}
 \end{subfigure}
 \begin{subfigure}{.47\textwidth}
  \centering
  \begin{tikzpicture}
   \node[vertex,myred,minimum size=11pt] (v1) at (30:2) {};
   \node[vertex] (v2) at (90:2) {};
   \node[vertex,myred,minimum size=11pt] (v3) at (150:2) {};
   \node[vertex,myred,minimum size=11pt] (v4) at (210:2) {};
   \node[vertex] (v5) at (270:2) {};
   \node[vertex,myred,minimum size=11pt] (v6) at (330:2) {};
   
   \draw[thick] (v1) -- (v2);
   \draw[thick] (v2) -- (v3);
   \draw[ultra thick,myred] (v3) -- (v4);
   \draw[thick] (v4) -- (v5);
   \draw[thick] (v5) -- (v6);
   \draw[ultra thick,myred] (v6) -- (v1);
   
   \draw[ultra thick,myred] (v1) -- (v4);
   \draw[thick] (v2) -- (v5);
   \draw[ultra thick,myred] (v3) -- (ve);
  \end{tikzpicture}
  \caption{\clr{Indukovaný} podgraf.}
  \label{subfig:indukovany-podgraf}
 \end{subfigure}
 \caption{Rozdíl mezi podgrafem a \emph{indukovaným} podgrafem.}
 \label{fig:indukovany-podgraf}
\end{figure}

Každý graf se přirozeně rozkládá na své maximální souvislé indukované podgrafy,
tzv. \emph{komponenty souvislosti}. Toto slovo jsme zde zmínili mnohokrát v
různých kontextech, však vždy bez řádné definice. Důvodem je fakt, že samotná
definice komponent souvislosti není zcela bez práce.

Nabízejí se dva přirozené přístupy, jejichž ekvivalenci si postupně ukážeme.
První, snad více informatický přístup, je definovat komponentu souvislosti jako
\textbf{maximální souvislý podgraf}, tedy takový (indukovaný) podgraf, mezi
každým párem jehož vrcholů vede cesta a je největší takový; to jest, k žádnému z
ostatních vrcholů vyššího grafu z vrcholů tohoto podgrafu cesta nevede. Tento
postup jsme nazvali \emph{informatickým}, neb popisuje, jak se algoritmicky
komponenty souvislosti v grafu hledají. Zkrátka tak, že začneme v libovolném
vrcholu, pokračujeme z něj do jeho sousedů a pak zase do jejich sousedů tak
dlouho, dokud to lze. V moment, kdy už se nikam dál z původního vrcholu nemůžeme
dostat, našli jsme tu \textbf{jednu} komponentu souvislosti, která obsahuje
počáteční vrchol.

Druhý přirozený přístup je ryze matematický a algoritmicky obtížně
realizovatelný. Zase je výhodný pro svou explicitnost a snadné využití v
důkazech. Obvyklý způsob, jak rozdělit množinu (v tomto případě množinu vrcholů,
$V$) na podmnožiny, je užitím \hyperref[def:trida-ekvivalence]{tříd
ekvivalence}. Protože dva vrcholy leží ve stejné komponentě souvislosti právě
tehdy, když mezi nimi vede cesta, nabízí se pro rozkouskování množiny využít
právě relaci \uv{býti cestou mezi vrcholy}. Jediný problém dlí v tom, že není na
první pohled zřejmé, jedná-li se o ekvivalenci.

Definujme na množině vrcholů $V$ grafu $G = (V,E)$ relaci $ \sim $ předpisem
\[
 u \sim v \Leftrightarrow \text{v $G$ vede cesta mezi $u$ a $v$}.
\]
Dokážeme si, že $ \sim $ je ekvivalence.

\begin{lemma}
 \label{lem:cesta-ekvivalence}
 Ať $G = (V,E)$ je graf a $ \sim $ je relace na $V$ dána výše. Pak $ \sim $ je
 ekvivalence na $V$.
\end{lemma}
\begin{proof}
 Dle \hyperref[def:ekvivalence]{definice ekvivalence} potřebujeme ověřit, že $
 \sim $ je
 \begin{enumerate}[label=(\alph*)]
  \item reflexivní, čili $v \sim v \; \forall v \in V$;
  \item symetrická, čili $u \sim v \Rightarrow v \sim u \; \forall u,v \in
   V$;
  \item transitivní, čili $u \sim v \wedge v \sim w \Rightarrow u \sim w$ pro
   všechny $u,v,w \in V$.
 \end{enumerate}
 Body (a) a (b) jsou v zásadě triviální. Pro každé $v \in V$ platí $v \sim v$,
 protože samotný vrchol je z \hyperref[def:cesta]{definice} též cesta. To
 dokazuje (a). Pro důkaz (b) ať $u \sim v$ a $u=v_0v_1\cdots v_n=v$ je cesta z
 $u$ do $v$. Pak $v=v_nv_{n-1}\cdots v_1v_0=u$ je cesta z $v$ do $u$, a tedy
 $v \sim u$.

 Jediný důkaz (c) není samozřejmý. Uvědomme si, že \textbf{spojení cest není
 vždy cesta}. Jeho princip spočívá v tom, že máme-li dánu cestu $u=u_0u_1\cdots
 u_n=v$ z $u$ do $v$ a cestu $v=v_0v_1\cdots v_m=w$ z $v$ do $w$, pak jdeme po
 první cestě tak dlouho, dokud nenarazíme na tu druhou. Tu pak následujeme až do
 $w$. Formálně, ať $k \leq n$ je \textbf{nejmenší} takové, že $u_k$ leží na
 cestě mezi $v$ a $w$. Takové $k$ musí existovat, protože zcela jistě
 přinejmenším vrchol $v$ leží jak na cestě z $u$ do $v$, tak na cestě z $v$ do
 $w$. Ať $j \leq m$ je index takový, že $u_k = v_j$. První cestu budeme
 následovat až do $u_k = v_j$ a potom budeme pokračovat po cestě druhé. V
 symbolech, kýžená cesta mezi $u$ a $w$ je v takovém případě
 \[
  u=u_0u_1\cdots u_{k-1}u_k=v_jv_{j+1}\cdots v_n=w.
 \]
 Tím je důkaz transitivity $ \sim $ hotov.
\end{proof}

Předchozí lemma opravňuje následující definici \emph{komponenty souvislosti}.

\begin{definition}[Komponenta souvislosti]
 \label{def:komponenta-souvislosti}
 Ať $G = (V,E)$ je graf, $ \sim $ je ekvivalence na $V$ z
 \myref{lemmatu}{lem:cesta-ekvivalence} a 
 \[
  V = \bigcup_{i = 1}^{n} V_i
 \]
 je rozklad $V$ na $n \in \N$ tříd ekvivalence $V_i$ podle $ \sim $. Indukované
 podgrafy $G_i = (V_i, E \cap \binom{V_i}{2})$ nazýváme \emph{komponenty
 souvislosti} grafu $G$.
\end{definition}

\begin{remark}
 Připomínáme, že rozklad na třídy ekvivalence
 \[
  V = \bigcup_{i = 1}^{n} V_i
 \]
 znamená, že $V$ má podle $ \sim $ přesně $n$ tříd ekvivalence, pro každé $i
 \leq n$ existuje vrchol $v_i \in V$ takový, že
 \[
  V_i = \{u \in V \mid u \sim v_i\},
 \]
 tedy $V_i$ je množina všech vrcholů, do nichž vede cesta z $v_i$, a $V_i \cap
 V_j = \emptyset$ pro každý pár $i \neq j$, to jest z $v_i$ do $v_j$ nevede
 žádná cesta.
\end{remark}

\begin{observation}
 Komponenty souvislosti $G_i$ grafu $G$ z
 \hyperref[def:komponenta-souvislosti]{definice výše} jsou právě všechny
 maximální souvislé indukované podgrafy grafu $G$.
\end{observation}
\begin{proof}
 Dokazujeme dvě implikace.

 Nejprve ať $G_i$ je nějaká komponenta souvislosti $G$. Graf $G_i$ je zřejmě
 indukovaný a souvislý z definice. Pro spor ať existuje vrchol $v \notin V_i$,
 do kterého vede cesta z nějakého vrcholu $u \in V_i$. Spor máme okamžitě, neboť
 $V_i$ z definice obsahuje všechny vrcholy, do nichž vede cesta z $u$. 

 Naopak, ať $H = (V',E')$ je maximální indukovaný souvislý podgraf $G$. Vezměme
 libovolný $v \in V'$. Pak je ale $H = [v]_{ \sim }$, tedy třída ekvivalence $
 \sim $ obsahující $v$. Vskutku, z maximality $H$ platí $u \in H$, kdykoli $v
 \sim u$. Protože sjednocení všech $V_i$ je množina vrcholů $V$ a tyto $V_i$
 jsou po dvou disjunktní, existuje nutně přesně jedno $i \leq n$ takové, že $v
 \in V_i$. Pak $H = V_i$.
\end{proof}

\subsubsection{Grafové homomorfismy}
\label{sssec:grafove-homomorfismy}

Trochu času věnujeme přemýšlení o tom, co může znamenat \emph{zobrazení} mezi
grafy. Tato podsekce má primárně býti jakýmsi úvodem do abstraktní algebry, kde
zaujímají zobrazení mezi strukturami snad největšího významu. O homomorfismech
grafů se dle názoru autora hovoří spíše okrajově a aplikací nejsou přehršle.

Ať tedy $G = (V_G,E_G)$ a $H = (V_H,E_H)$ jsou dva grafy. Zápis
$\varphi:G \to H$, tj. zobrazení z $G$ do $H$, sice smysl dává, neb $G$ a $H$
jsou (jako všechno v teorii množin) též množiny, ale není pro rozvoj teorie
nijak zajímavý. Takové zobrazení totiž může klidně posílat třeba hrany v $G$ na
vrcholy v $H$ a tvořit jiné podobné zrůdnosti.

V celé sekci se díváme na graf jako na množinu vrcholů se strukturou danou
hranami. Tenhle pohled zůstává užitečným i nyní. Totiž, \emph{homomorfismem} (z
řec. homós, \uv{stejný}, a morphé, \uv{tvar, forma}) mezi obecně množinami se
strukturou se myslí \textbf{zobrazení mezi těmito množinami, které zachovává
strukturu}.

V případě grafů toto znamená, že \emph{homomorfismus} $\varphi:G \to H$ je
zobrazení $\varphi_V:V_G \to V_H$ z vrcholů $G$ do vrcholů $H$, které
\uv{zachovává} hrany. To je lze vyložit zkrátka tak, že když mezi vrcholy v $G$
vede hrana, tak mezi jejich obrazy při zobrazení $\varphi_V$ vede též hrana.

\begin{definition}[Grafový homomorfismus]
 \label{def:grafovy-homomorfismus}
 Buďtež $G = (V_G,E_G)$ a $H = (V_H,E_H)$ dva grafy. \emph{Homomorfismem} z $G$
 do $H$, značeným běžně $\varphi:G \to H$, myslíme zobrazení
 \[
  \varphi_V:V_G \to V_H
 \]
 takové, že
 \[
  (u,v) \in E_G \Rightarrow (\varphi_V(u),\varphi_V(v)) \in E_H.
 \]
 Obvykle, bo jest tomu buřt, ztotožňujeme homomorfismus $\varphi:G \to H$ se
 zobrazením $\varphi_V:V_G \to V_H$ mezi bázovými množinami a oboje značíme
 zkrátka $\varphi$.
\end{definition}

S \hyperref[def:grafovy-homomorfismus]{touto definicí} je jistý problém. Uvažme
grafy $G$ a $H$ dané \myref{obrázkem}{fig:problem-s-definici-homomorfismu}.

\begin{figure}[h]
 \centering
 \begin{subfigure}{.47\textwidth}
  \centering
  \begin{tikzpicture}[scale=1.5]
   \node[vertex] (a) at (0,0) {};
   \node[vertex] (b) at (0,1) {};
   \node[vertex] (c) at (1,1) {};
   
   \draw[thick] (a) -- (b);
   \draw[thick] (b) -- (c);
   \draw[thick] (a) -- (c);
   
   \node[below left=-1mm and -1mm of a] {$a$};
   \node[above left=-1mm and -1mm of b] {$b$};
   \node[above right=-1mm and -1mm of c] {$c$};
  \end{tikzpicture}
  \caption{Graf $G$.}
  \label{subfig:graf-g}
 \end{subfigure}
 \begin{subfigure}{.47\textwidth}
  \centering
  \begin{tikzpicture}[scale=1.5]
   \node[vertex] (1) at (0,0) {};
   \node[vertex] (2) at (1,0) {};
   \node[vertex] (3) at (1,1) {};
   \node[vertex] (4) at (0,1) {};

   \draw[thick] (1) -- (2);
   \draw[thick] (2) -- (3);
   \draw[thick] (3) -- (4);
   \draw[thick] (4) -- (1);
   \draw[thick] (1) -- (3);
   
   \node[below left=-1mm and -1mm of 1] {$1$};
   \node[below right=-1mm and -1mm of 2] {$2$};
   \node[above right=-1mm and -1mm of 3] {$3$};
   \node[above left=-1mm and -1mm of 4] {$4$};
  \end{tikzpicture}
  \caption{Graf $H$.}
  \label{subfig:graf-h}
 \end{subfigure}
 \caption{Problém s definicí homomorfismu $G \to H$.}
 \label{fig:problem-s-definici-homomorfismu}
\end{figure}

Zde $V_G = \{a,b,c\}$ a $V_H = \{1,2,3,4\}$. Jeden možný homomorfismus $G \to H$
je třeba dán zobrazením $\varphi:V_G \to V_H$ takovým, že
\begin{equation*}
 \begin{split}
  \varphi(a) &= 1, \\
  \varphi(b) &= 4, \\
  \varphi(c) &= 3.
 \end{split}
\end{equation*}
Pak je opravdu splněna podmínka z \myref{definice}{def:grafovy-homomorfismus},
neboť mezi každýma dvěma vrcholoma z $a,b,c$ vede hrana a mezi každýma dvěma
vrcholoma z~$1,4,3$ rovněž vede hrana.

Uvažme však jen trochu jiný homomorfismus $\psi: G \to H$, který je dán
rovnostmi
\begin{equation*}
 \begin{split}
  \psi(a) &= 1, \\
  \psi(b) &= 3, \\
  \psi(c) &= 3; \\
 \end{split}
\end{equation*}
čili vlastně \uv{slepuje} vrcholy $b$ a $c$ do vrcholu $3$. Všimněte si, že
takovéto zobrazení není podle \myref{definice}{def:grafovy-homomorfismus}
homomorfismem. Hrany $(a,b)$ a $(a,c)$ jsou sice zachovány, ale podmínka z téže
definice pro hranu $(b,c)$ říká, že
\[
 (b,c) \in E_G \Rightarrow (\psi(b),\psi(c)) = (3,3) \in E_H,
\]
což zřejmě nelze, neboť podle naší \hyperref[def:graf-poprve]{první definice
grafu} je $E$ relace na $V$, která je symetrická a \textbf{antireflexivní}, což
znamená, že v grafu nepovolujeme smyčky, kterou by $(3,3)$ jistě byla.

Tato peripetie má několik možných katastrof. V zájmu rozvoje matematického
myšlení drahých čtenářů si některé rozebereme.

\begin{enumerate}
 \item Nic nedělat, držet se původní definice a nepovažovat $\psi$ za
  homomorfismus. To je jistě jedno možné řešení a vyniká množstvím potřebné
  práce k jeho dosažení. Lidsky řečeno vlastně znamená, že slepovat vrcholy mohu
  jedině tehdy, když mezi nimi nevede hrana. Toto řešení je zároveň nejvíce
  omezující.
 \item Modifikovat \hyperref[def:grafovy-homomorfismus]{definici homomorfismu}
  tak, aby ignorovala tyto případy. Tedy, upravit podmínku zachování hran
  následovně:
  \[
   ((u,v) \in E_G \wedge \psi(u) \neq \psi(v)) \Rightarrow (\psi(u),\psi(v)) \in
   E_H.
  \]
  Takové řešení je \emph{extrémně nepřirozené}. Totiž, když z grafu $H$ na
  \myref{obrázku}{subfig:graf-h} vynecháme hranu $(b,c)$ a homomorfismus $\psi$
  ponecháme beze změny, pak je v obou případech (tedy s hranou $(b,c)$ i bez ní)
  obrazem grafu $H$ v $G$ úsečka $13$. Byť obrazy dvou různých grafů stejným
  homomorfismem byly stejné, jesti algebraicky nepřijatelné.
 \item Zvolnit definici grafu tak, aby povolovala smyčky. To by znamenalo
  konkrétně odstranit z \hyperref[def:graf-poprve]{první definice grafu}
  podmínku antireflexivity relace $E$. Zůstala by tedy pouze symetrickou a
  dvojice $(v,v)$ pro $v \in V$ by byly smyčkami. Ovšem,
  \hyperref[def:graf-podruhe]{druhá definice grafů} by se stala nepoužitelnou,
  neboť dvouprvková podmnožina $V$ nemůže obsahovat dvakrát týž prvek. Z
  \hyperref[def:graf-potreti]{třetí definice grafu} by stačilo rovněž odstranit
  pouze podmínku (a) zakazující shodnost zdroje a cíle téže hrany.

  Při adopci tohoto řešení by graf $G$ musel obsahovat smyčku $(3,3)$, aby mohl
  předpis $\psi$ zůstat beze změny. V takovém případě by obrazem grafu $H$
  \textbf{bez} hrany $(b,c)$ byla opět úsečka $13$, ale obrazem grafu $H$
  \textbf{se} hranou $(b,c)$ by byla úsečka $13$ spolu se smyčkou na vrcholu
  $3$.
\end{enumerate}
Obě řešení (1) i (3) jsou rozumná a v různých situacích různě užitečná. Dle naší
zkušenosti je operace \uv{slepení} vrcholu příliš důležitá pro práci s~grafovou
strukturou a pročež je spíše dáváno přednosti řešení (3). Ovšem, jistě existují
aplikace, kdež mají homomorfismy své místo, ale přesto jsou smyčky nežádoucí.

Podsekci kvapně zakončíme zcela bezvýznamnou definicí. Účel jeho zařazení dlí,
podobně jako zbytku podsekce, v prezentaci a povrchovém prozkoumání pojmu
\emph{homomorfismu}.

Homomorfismy mimořádných vlastností (jako prostota, surjektivita) mají své
vlastní názvy, které zde pro pořádek uvedeme. Reálně nás v dalším textu bude
zajímat ovšem jedině pojem \emph{isomorfismu}, bijektivního homomorfismu.

\begin{definition}
 \label{def:morfismy}
 Ať $\varphi:G \to H$ je homomorfismus grafů. Je-li
 \begin{itemize}
  \item $\varphi$ prostý, pak sluje \emph{\textbf{mono}morfismus} (z řec. mono-,
   \uv{jeden, jediný});
  \item $\varphi$ na, pak sluje \emph{\textbf{epi}morfismus} (z řec. epi-, \uv{nad, přes,
   na});
  \item $\varphi$ bijekce, pak sluje \emph{\textbf{iso}morfismus} (z řec. iso-,
   \uv{roven});
  \item $H = G$, pak sluje \emph{\textbf{endo}morfismus} (z řec. éndon,
   \uv{vnitřek, vnitř\-ní});
  \item $\varphi$ endomorfismus a isomorfismus, pak sluje
   \emph{\textbf{auto}morfismus} (z~řec. auto-, \uv{sebe-}).
 \end{itemize}
\end{definition}

\begin{remark}
 Je-li $\varphi:G \to H$ isomorfismus, pak je to speciálně bijekce $V_G \to
 V_H$, tedy $\# V_G = \# V_H$ a graf $H$ je vlastně graf $G$ s přejmenovanými
 vrcholy.

 Je-li $\varphi:G \to G$ automorfismus, pak je to speciálně permutace na $V_G$,
 čili vlastně jen proházení vrcholů $G$.
\end{remark}


