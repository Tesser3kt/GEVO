\section*{Seznam cvičení}
\label{sec:seznam-cviceni}

% Add list of exercises to toc
\addcontentsline{toc}{section}{Seznam cvičení}

\subsubsection*{\hyperref[sec:uvodni-pojmy]{Úvodní pojmy}}

\begin{enumerate}
 \item Dokažte \myref{Tvrzení}{claim:vlastnosti-trid-ekvivalence}.
 \item Dokažte, že relace (ne nutně ekvivalence!) $R$ je transitivní, právě
  tehdy když $R \circ R \subseteq R$.
 \item Vyřešte následující úlohy rozprostřené po \myref{sekci}{ssec:zobrazeni}.
  Konkrétně,
  \begin{itemize}
   \item dokažte, že mezi dvěma konečnými množinami
   různé velikosti neexistuje žádná bijekce.
   \item pro množinu $A$ velikosti $n$ určete počet různých bijektivních
    zobrazení ${f:[n] \cong A}$. \item určete, jakou podmínku splňují zobrazení
    $f:A \to B$, ke kterým existuje zobrazení inverzní.
  \end{itemize}
 \item Dokažte, že každá lineární funkce $f:\R \to \R$, tedy funkce daná
  předpisem
  \[
   f(x) = ax + b \quad \text{pro } a,b \in \R,a \neq 0
  \]
  je bijektivní zobrazení.
 \item Nechť $A$ je konečná množina. Zformulujte důkaz, že libovolné zobrazení
  ${f:A \to A}$, které je definované pro každé $x \in A$, je \textbf{prosté,
  právě tehdy když je na}.
 \item Najděte příklad zobrazení $f:\N \to \N$ definovaného na celém $\N$, které
  je
 \begin{enumerate}
  \item prosté, ale není na.
  \item na, ale není prosté.
 \end{enumerate}
 \item Udělejte cvičení rozmístěná po \myref{sekci}{ssec:usporadani}. Konkrétně,
  \begin{enumerate}
   \item dokažte, že Hasseho diagram každého lineárního uspořádání má stejný
    tvar jako diagram na \myref{obrázku}{fig:hasse-linear}.
   \item dokažte, že relace dělitelnosti $ \mid $ je uspořádání na každé
    podmnožině přirozených čísel.
  \end{enumerate}
 \item Explicitně popište všechny relace (na libovolné množině), které jsou
  zároveň ekvivalencí a (částečným) uspořádáním.
 \item Řekněme, že $R$ a $S$ jsou uspořádání na množině $A$. Které z
  následujících relací jsou také uspořádáními na $A$?
  \begin{itemize}[itemsep=0pt]
  \item $R \cap S$ 
  \item $R \cup S$ 
  \item $R \setminus S$
  \item $R \circ S$
 \end{itemize}
 \item Dokažte indukcí, že
  \[
   \sum_{i=1}^{n} i 2^{i} = (n - 1)2^{n+1} + 2.
  \]
 \item Tak zvaná \emph{Fibonacciho} posloupnost je definována tak, že další člen
  dostanu jako součet dvou předchozích. Formálně, $F_0 = 0, F_1 = 1$ a $F_n =
  F_{n - 1} + F_{n - 2}$, kde $n \in \N$. Dokažte indukcí, že
  \[
   F_n \leq \left( \frac{1+\sqrt{5}}{2} \right) ^{n-1}
  \]
  pro všechna $n \geq 0$.
 \item Nakresleme $n$ přímek v rovině, a to tak, že
  \begin{itemize}
   \item žádné 2 nejsou rovnoběžné a
   \item žádné 3 se neprotínají v jednom bodě.
  \end{itemize}
  Dokažte indukcí, že takhle nakreslené přímky rozdělují rovinu na $n(n+1) / 2 +
  1$ částí.
\end{enumerate}

\subsubsection*{\hyperref[sec:pocitani]{Počítání}}
\begin{enumerate}
 \item  Dokažte \myref{Tvrzení}{claim:pocet-prostych-zobrazeni} indukcí podle
  velikosti množiny $A$.
 \item Určete počet všech uspořádaných dvojic $(A,B)$, kde $A \subseteq B
  \subseteq \{1,\ldots,n\}$.
 \item Spočtěte složení $\sigma\tau$ a $\tau\sigma$, když
  \begin{enumerate}
   \item $\sigma = (143)(26), \tau = (146)(253)$,
   \item $\sigma = (14)(25)(36), \tau = (123456)$,
   \item $\sigma = (145)(263), \tau = (154)(236)$.
  \end{enumerate}
 \item Určete řád permutace $\sigma$, kde
  \begin{enumerate}
   \item $\sigma = (1345)$,
   \item $\sigma = (1346)(28)(579)$.
  \end{enumerate}
 \item(těžké) Určete číslo $C_n(S_X)$, kde $\# X = 100$ a $n \leq 50$, tedy
  počet všech permutací na $100$ číslech s aspoň jedním cyklem délky menší nebo
  rovné $50$.

  Samozřejmě jich je $100! - C_{ \geq 51}(S_X)$, ale cílem úlohy je spočítat je
  nějak chytře, aby člověk dostal hezčí vzoreček.
 \item Ať $\sigma \in S_n$, tedy $\sigma$ je permutace množiny $\{1,\ldots,n\}$.
  Řekneme, že $\sigma$ \emph{invertuje} dvojici $(i,j)$, kde $i,j \in
  \{1,\ldots,n\}$, když $i<j$, ale $\sigma(i)>\sigma(j)$.

  Definujme
   \[
    I(\sigma) \coloneqq \{(i,j) \in \{1,\ldots,n\}^2 \mid \sigma \text{ invertuje
    } (i,j)\},
  \]
  čili $I(\sigma)$ je množina všech dvojic $(i,j)$, které $\sigma$ invertuje.
  Uvědomme si, že $I(\sigma)$ je podmnožinou $\{1,\ldots,n\}^2$, čili relací na
  $\{1,\ldots,n\}$.
  \begin{enumerate}
   \item Dokažte, že $I(\sigma)$ je transitivní relace na $\{1,\ldots,n\}$ pro
    každou permutaci $\sigma \in S_n$.
   \item (těžké) Navrhněte algoritmus, který pro danou permutaci $\sigma \in
    S_n$ spočte $\# I(\sigma)$.
   \item Spočtěte počet invertovaných dvojic, čili $\# I(\sigma)$, permutací
    $\sigma = (134)(579)(26)$.
  \end{enumerate}
 \item Dokažte, že
  \[
   \sum_{i=0}^{n} \binom{n}{i}^2 = \binom{2n}{n}.
  \]
  \textbf{Hint}: použijte \myref{lemma}{lemma:stejne-doplnku}.
 \item Dokažte vzorec
  \[
   \sum_{k=r}^{n} \binom{k}{r} = \binom{n+1}{r+1}
  \]
  pro pevné $r \in \N$ indukcí podle $n \in \N$.
 \item(těžké) Kolik existuje podmnožin $\{1,\ldots,n\}$, které neobsahují žádná
  dvě po sobě jdoucí čísla. Formálně, určete velikost množiny
  \[
   \{A \subseteq \{1,\ldots,n\} \mid \{i,j\} \nsubseteq A, \text{ kdykoli }
   |i-j|=1\}.
  \]
 \item Ať $p$ je prvočíslo a $k,n$ přirozená čísla.
  \begin{enumerate}
   \item Dokažte, že pro $k<p$ je $\binom{p}{k}$ dělitelné $p$.
   \item Dokažte, že $\binom{n}{p}$ je dělitelné $p$ právě tehdy, když
    $\left\lfloor n / p \right\rfloor$ je dělitelné $p$, kde $\left\lfloor \cdot
    \right\rfloor$ značí \emph{dolní celou část}.
  \end{enumerate}
 \item Budeme vybírat $k$-tice předmětů z $n$ druhů předmětů. Budeme uvažovat
  různé typy výběru podle toho, jestli vybíráme $k$-tice uspořádané, nebo
  neuspořádané (tj. podmnožiny) a též podle toho, zda každého druhu je vždy jen
  jeden předmět, či nikoli. Doplňte následující tabulku:
  \begin{table}[h]
   \centering
   \begin{tabular}{c|c|c}
    & Jen 1 předmět & Libovolně mnoho předmětů\\
    & každého druhu & každého druhu\\
    \midrule
    Uspořádané& &\\
    $k$-tice& &\\
    \midrule
    Neuspořádané & &\\
    $k$-tice& &
   \end{tabular}
   \caption{Výběr $k$-tic předmětů z $n$ druhů předmětů.}
   \label{table:vyber-k-z-n-2}
  \end{table}
 \item Kolika způsoby můžeme postavit $7$ čarodějnic a $5$ vodníků do řady tak,
  aby $2$ vodníci nikdy nestáli vedle sebe? \item Jeden z oblíbených a celkem
  rychlých rozkladů čísla na prvočísla je tzv. \emph{Eratosthenovo síto}.
 
  Funguje na principu vyškrtávání násobků čísel. Konkrétně, algoritmus prochází
  seznam čísel až do nějaké hranice, a kdykoli narazí na ještě neodškrtnuté
  číslo, odškrtne z tohoto seznamu všechny jeho násobky kromě něho samotného.
  Sami si rozmyslete, že tímto způsobem zůstanou ve výsledném seznamu pouze
  prvočísla.

  My si tady rozmyslíme pouze zjednodušenou verzi. Spočtěte, kolik zůstane
  čísel mezi $1$ a $1000$ potom, co vyškrtáme všechny násobky čísel $2,3,5$ a
  $7$.
 \item Určete počet přirozených čísel menších než $100$, které nedělí druhá
  mocnina žádného přirozeného čísla (kromě $1$).
 \item Kolika způsoby lze seřadit do řady 5 Čechů, 4 Maďary a 3 Rusy, aby
  všichni příslušníci jednoho národa nikdy nestáli hned za sebou?
\end{enumerate}

\subsubsection*{\hyperref[sec:teorie-grafu]{Teorie grafů}}

\begin{enumerate}
 \item Nakreslete graf $G = (V,E)$, kde
  \begin{itemize}
   \item $V = \{1,\ldots,5\}, E = \{\{1,2\}, \{1,3\}, \{1,5\}, \{2,3\},
    \{3,4\},\{4,5\}\}$.
   \item $V = \{1,\ldots,5\}$, $E = \binom{V}{2}$.
   \item $V = \{1,\ldots,8\}$, $E = \{e_1,\ldots,e_8\}$ a
   \begin{itemize}
    \item $t(e_i) = s(e_{i+1}) = i + 1$ pro všechna $i \leq 7$,
    \item $t(e_8) = s(e_1) = 1$.
   \end{itemize}
  \end{itemize}
 \item Popište všechny grafy $G = (V,E)$, kde $E$ je relace na $V$, která je
  antireflexivní, symetrická (to je součástí definice grafu) a navíc
  \textbf{transitivní}.
 \item Ať $V$ je konečná množina a $E$ je relace na $V$, která je antireflexivní
  a symetrická. Definujme navíc na $E$ další relaci $ \sim $ předpisem
  \[
   (v,v') \sim (w,w') \Leftrightarrow (v,v') = (w,w') \vee (v,v') = (w',w). 
  \]
  Dokažte, že pak existuje bijekce mezi $[E]_{ \sim }$ a množinou
  \[
   E' \coloneqq \{\{v,v'\} \mid (v,v') \in E\},
  \]
  čili mezi množinou tříd ekvivalence $E$ podle $ \sim $ a množinou, kterou
  dostanu tak, že z uspořádaných dvojic v $E$ udělám neuspořádané dvojice, tj.
  dvouprvkové podmnožiny. Pro intuici vizte poznámku pod
  \myref{definicí}{def:graf-poprve}.
 \item Spočtěte, kolik existuje grafů na $n$ vrcholech.
 \item Ať $\mathcal{P}_1 = e_1^{1}e_2^{1}\cdots e_n^{1}$ a $\mathcal{P}_2 =
  e_1^2e_2^2\cdots e_n^2$ jsou cesty. Za předpokladu, že $t(e_n^{1}) = s(e_1^2)$,
  definujeme jejich \emph{sloučení}, které zapíšeme třeba jako $\mathcal{P}_1
  \oplus \mathcal{P}_2$, přirozeně jako posloupnost hran
  \[
   \mathcal{P}_1 \oplus \mathcal{P}_2 \coloneqq e_1^{1}e_2^{1}\cdots
   e_n^{1}e_1^2e_2^2\cdots e_n^2.
  \]
  Určete, pro jaké cesty (v obecném grafu) $\mathcal{P}_1, \mathcal{P}_2$ platí,
  že
  \begin{itemize}
   \item $\mathcal{P}_1 \oplus \mathcal{P}_2 = \mathcal{P}_2 \oplus
    \mathcal{P}_1$;
   \item je $\mathcal{P}_1 \oplus \mathcal{P}_2$ cesta;
   \item je $\mathcal{P}_1 \oplus \mathcal{P}_2$ tah;
   \item je $\mathcal{P}_1 \oplus \mathcal{P}_2$ sled;
   \item je $\mathcal{P}_1 \oplus \mathcal{P}_2$ cyklus.
  \end{itemize}
 \item Dokažte, že je-li $T = (V,E)$ strom, pak $\# E = \# V - 1$.
 \item Spočtěte, kolik existuje stromů na $n$ vrcholech.
 \item (těžké) Ať $V$ je množina $n$ bodů v rovině. Pro každé dva body $x,y \in
  V$ definujme váhu hrany $xy$ jako vzdálenost bodů $x$ a $y$. Čili, pokud $x =
  (x_1,x_2)$ a $y = (y_1,y_2)$, pak
  \[
   w(xy) \coloneqq \sqrt{(x_1 - y_1)^2 + (x_2 - y_2)^2}.
  \]
  Vzniklý graf označme obyčejně $G$.
  \begin{enumerate}[label=(\alph*)]
   \item Ukažte, že v každé minimální kostře $G$ vede z každého vrcholu maximálně
    6 hran.
   \item Ukažte, že existuje minimální kostra $G$, jejíž hrany se (jakožto úsečky
    v rovině) nekříží.
  \end{enumerate}
 \item \emph{Úplným grafem} na $n$ vrcholech myslíme graf $G = (V,
  \binom{V}{2})$, tedy graf, mezi každým párem jehož vrcholů vede hrana. Takový
  graf se obvykle značí $K_n$. Najděte minimální kostru $K_n$ a spočtěte její
  váhu (tj, součet vah všech jejích hran), je-li $V = \{1,\ldots,n\}$ a
  \begin{enumerate}[label=(\alph*)]
   \item $w(ij) = \max(i,j)$,
   \item $w(ij) = i + j$,
  \end{enumerate}
  pro všechny páry $i,j \leq n$.
 \item Dokažte, že když $w$ (tj. ohodnocení $G$) je prosté zobrazení, pak je
  minimální kostra $G$ určena jednoznačně.
 \item Dokažte, že \hyperref[alg:floyd-warshall]{Floydův-Warshallův} algoritmus
  selže, když připustíme i záporná ohodnocení hran.
 \item Vyřešte \hyperref[prob:rodinny-vylet]{úlohu o rodinném výletu} pro graf
  daný \myref{obrázkem}{fig:trinec-orlova}.
 \item Rozmyslete si, jak upravit zobrazení
  \[
   \varepsilon:V \times V \to \{0,1\}
  \]
  definující hranovou strukturu na množině vrcholů tak, aby zahrnovalo i všechny
  \textbf{ohodnocené} grafy.
 \item (těžké) Kolik nejvíce hran může mít graf s $n$ a $k$ komponentami
  souvislosti?
 \item Ať $V$ je \textbf{konečná} množina a $d:V \times V \to \{0,1,2,\ldots\}$
  je metrika na $V$. Dokažte, že pak existuje graf $G = (V,E)$ takový, že
  $d_G(u,v) = d(u,v)$ pro všechny $u,v \in V$.
 \item Navrhněte efektivní algoritmus, který pro zadaný graf $G = (V,E)$
  rozhodne, zda je souvislý, a pokud není, najde jeho rozklad na komponenty
  souvislosti.
\end{enumerate}
