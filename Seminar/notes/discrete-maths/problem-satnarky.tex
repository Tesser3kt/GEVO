\subsubsection{Problém šatnářky}
\label{sssec:problem-satnarky}

Kapitolu o kombinatorickém počítání završíme klasickým \emph{problémem
šatnářky}. Jeho tradiční znění je následující.

V jeden chladný podvečer přijde skupina pánů do luxusního podniku. Před vstupem
do lokálu si každý z nich odloží svoji černou buřinku (neb jiné se nenosí) v
pánské šatně. Večírek se však neobyčejně prodlužuje a šatnářka, navíc ten den
dosti nevyspalá, počne únavou podřimovati. Když ctihodní pánové konečně
doflámují, sotva bdělá šatnářka jim při odchodu z šatny náhodně rozdá -- jejíma
zaslepenýma očima nerozlišitelné -- černé buřinky. S jakou pravděpodobností
odchází každý pán s cizí buřinkou?

Očíslujeme si pány přirozenými čísly od $1$ do $n$ a jejich buřinky taktéž; čili
pánovi číslo $i$ připadá buřinka číslo $i$. Náhodné rozdání buřinek odpovídá
výběru permutace $\sigma \in S_n$, neboli bijektivního zobrazení  $\sigma:
\{1,\ldots,n\} \to \{1,\ldots,n\}$. Podle \cref{prop:pocet-permutaci-na-mnozine}
je počet těchto roven $n!$. Jednu konkrétní volbu rozloučení buřinek učiní tedy
šatnářka ze zadání problému s pravděpodobností $1 / n!$. Označíme-li si počet
permutací, které žádné číslo nezobrazují na totéž, jako  $\check{s}(n)$, pak
řešením problému šatnářky je číslo $\check{s}(n) / n!$.

Tou netriviální otázkou zůstává, jak $\check{s}(n)$ určit. Postup, který se zde
jmeme popsat, využívá \uv{trik} častý především v teorii pravděpodobnosti. Bývá
totiž obvykle výrazně jednodušší spočítat počet objektů, které \textbf{splňují}
nějakou vlastnost, raději než objektů, které onu vlastnost \textbf{nesplňují}.
Oba postupy jsou logicky ekvivalentní, neboť počet objektů nesplňujících
vlastnost $V$ je roven rozdílu počtu všech objektů a počtu těch, které vlastnost
$V$ splňují. Výpočetně se však postupy svou obtížností často výrazně liší.

Místo toho, abychom určovali počet permutací $\sigma \in S_n$, jež nezobrazí
žádný prvek na sebe, budeme počítat permutace, které zobrazí aspoň jeden prvek
na sebe. Takovému prvku se pak obvykle říká \emph{pevný bod} dané permutace.

\begin{definition}[Pevný bod]
 \label{def:pevny-bod}
 Ať $\sigma \in S_n$. Prvek $i  \in \{1,\ldots,n\}$ nazveme \emph{pevným bodem}
 permutace $\sigma$, pokud $\sigma(i) = i$.
\end{definition}

Pro dané $i  \in \{1,\ldots,n\}$ si ještě označíme
\[
 \mathcal{P}_i \coloneqq \{\sigma \in S_n \mid \sigma(i) = i\},
\]
čili $\mathcal{P}_i$ je množina permutací, jejichž pevným bodem (\textbf{možná
ne jediným!}) je $i$. Snadno si rozmyslíme, že množina všech permutací, které
mají \textbf{aspoň jeden pevný bod}, je právě
\[
 \mathcal{P}_1 \cup \mathcal{P}_2  \cup \ldots \cup \mathcal{P}_n =
 \bigcup_{i=1}^{n} \mathcal{P}_i.
\]
Vskutku, permutace $\sigma$ leží v $\bigcup_{i=1}^{n} \mathcal{P}_i$ právě
tehdy, když existuje nějaké ${i  \in \{1,\ldots,n\}}$ takové, že $\sigma(i) =
i$, což je z \hyperref[def:pevny-bod]{definice} právě tehdy, když existuje
nějaký její pevný bod. Čili, $\# \bigcup_{i=1}^{n} \mathcal{P}_i$ vyjadřuje
počet permutací, které zobrazí aspoň jeden prvek na ten samý. Z diskuse výše
plyne, že
\[
 \check{s}(n) = n! - \# \bigcup_{i=1}^{n} \mathcal{P}_i.
\]
Problém šatnářky vyřešíme tím, že určíme $\# \bigcup_{i=1}^{n} \mathcal{P}_i$. Z
\myref{věty}{thm:princip-inkluze-a-exkluze} můžeme počítat
\begin{equation*}
 \label{eq:aspon-jeden-pevny-bod}
 \tag{$\square$}
 \# \bigcup_{i=1}^{n} \mathcal{P}_i = \sum_{I \subseteq \{1,\ldots,n\}}^{}
 (-1)^{\# I-1}\# \bigcap_{i \in  I}^{} \mathcal{P}_i.
\end{equation*}
Dalším krokem ke zdárnému řešení problému je uvědomění, že výraz $\# \bigcap_{i
\in  I}^{} \mathcal{P}_i$ záleží pouze na $\# I$, nikoli na konkrétních
indexech, které se v~$I$ vyskytují. Přeci, $\# \bigcap_{i \in  I}^{}
\mathcal{P}_i$ značí počet permutací s přesně $\# I$ pevnými body. Všimněme si,
že ten je určen pouze samotným počtem pevných bodů, a ne jejich konkrétní
hodnotou.

Toto pozorování napovídá, že v součtu na pravé straně
\eqref{eq:aspon-jeden-pevny-bod} můžeme sloučit ty sčítance, pro něž je $\# I$
stabilní. Řečeno přímočařeji, stačí, když budeme sčítat pouze přes
\textbf{velikosti} všech podmnožin, a nemusíme sčítat přes individuální
podmnožiny, neboť na jejich konkrétních prvcích nezáleží (pouze na jejich
velikostech).

Provedeme pro přehlednost ještě poslední přeznačení. Výraz $\mathsf{P}(k)$ bude
značit \emph{počet} permutací s přesně $k$ pevnými body. Díky předchozímu
odstavci bychom též mohli být formální a psát, že
\[
 \mathsf{P}(k) = \sum_{\substack{I \subseteq \{1,\ldots,n\}\\ \#I = k}}
 \#\bigcap_{i \in  I}^{} \mathcal{P}_i,
\]
čili (dokonce správně) tvrdit, že počet permutací s $k$ pevnými body dostaneme
tak, že sečteme počty permutací s pevnými body v podmnožině $I$ přes všechny
$k$-prvkové podmnožiny $I$.

Nyní můžeme rovnost \eqref{eq:aspon-jeden-pevny-bod} přepsat do mnohem
jednodušší podoby
\[
 \# \bigcup_{i=1}^{n} \mathcal{P}_i = \sum_{k=1}^{n} (-1)^{k-1} \mathsf{P}(k).
\]
Opět přeloženo do lidské řeči říkáme, že počet permutací s aspoň jedním pevným
bodem můžeme určit jako součet (až na znaménko) všech permutací s přesně $k$
pevnými body přes všechna $k$ od $1$ do $n$. To dává smysl.

Dokážeme-li určit $\mathsf{P}(k)$, jsme hotovi. K tomu slouží následující lemma.

\begin{lemma}
 \label{lemma:permutace-s-k-pevnymi-body}
 Ať $k \in\N, k \leq n$. Platí
 \begin{equation*}
  \label{eq:k-pevnych-bodu}
  \tag{$\bullet$}
  \mathsf{P}(k) = \binom{n}{k}(n-k)! = \frac{n!}{k!}.
 \end{equation*}
\end{lemma}
\begin{proof}
 Za důkaz vřele děkujeme laureátovi Fieldsovy medaile, prof. Mgr. et Ing.
 Jáchymovi Löwenhöfferu, RNDr. et Ph. D., CSc., DSc. Neodvažujeme se však přímo
 parafrázovat, tedy jej s odpuštěním mírně přeformulujeme.

 Permutace s přesně $k$ libovolnými pevnými body je určena výběrem těchto $k$ 
 pevných bodů z $n$ čísel a poté výběrem uspořádání zbývajících $n-k$ čísel.
 Výběr $k$ prvků z $n$ mohu podle
 \myref{tvrzení}{claim:pocet-k-prvkovych-podmnozin} učinit $\binom{n}{k}$
 způsoby. Zvolit permutaci zbývajících $n-k$ prvků mohu podle
 \myref{tvrzení}{prop:pocet-permutaci-na-mnozine} $(n-k)!$ způsoby. Celkem tedy
 mohu určit jednu permutaci na $n$ prvcích s $k$ pevnými body přesně
 $\binom{n}{k}(n-k)!$ způsoby. To dokazuje první rovnost v
 \eqref{eq:k-pevnych-bodu}.

 Ta druhá plyne z \myref{lemmatu}{lem:vzorec-pro-kombinacni-cislo}, neboť
 \[
  \binom{n}{k}(n-k)! = \frac{n!}{k!(n-k)!}(n-k)! = \frac{n!}{k!}.\qedhere
 \]
\end{proof}

Dopracovali jsme se k závěru problému. Díky
\hyperref[lemma:permutace-s-k-pevnymi-body]{předchozímu lemmatu} máme po
dosazení do \eqref{eq:aspon-jeden-pevny-bod} vzorec
\[
 \# \bigcup_{i=1}^{n} \mathcal{P}_i = \sum_{k=1}^{n} (-1)^{k-1} \frac{n!}{k!}.
\]
Čili,
\[
 \check{s}(n) = n! - \# \bigcup_{i=1}^{n} \mathcal{P}_i = n! - \sum_{k=1}^{n}
 (-1)^{k-1} \frac{n!}{k!}.
\]
Vytknutí čísla $n!$ dá snad hezčí vyjádření
\[
 \check{s}(n) = n! \sum_{k=0}^{n} (-1)^{k} \frac{1}{k!}.
\]
Je triviální dokázat, že součet výše konverguje k $1 / e$, kde písmeno $e$ značí
tzv. \href{https://cs.wikipedia.org/wiki/Eulerovo_%C4%8D%C3%ADslo}{Eulerovo
číslo}. Formálně zapsáno,
\[
 \sum_{k=0}^{n} (-1)^{k} \frac{1}{k!} \xrightarrow{n \to \infty}
 \frac{1}{e}.
\]
Neformálně tento vztah říká, že čím je vyšší počet přišedších pánů do podniku,
tím víc se pravděpodobnost, že šatnářka vrátí každému cizí buřinku, blíží k
číslu $1 / e$. Je tomu tak pro to, že
\[
 \frac{\check{s}(n)}{n!} = \frac{n!\sum_{k=0}^{n} (-1)^{k} \frac{1}{k!}}{n!} =
 \sum_{k=0}^{n} (-1)^{k} \frac{1}{k!} \xrightarrow{n \to \infty} \frac{1}{e},
\]
kde, pamatujte, $\check{s}(n) / n!$ je přesně řešení problému šatnářky.
