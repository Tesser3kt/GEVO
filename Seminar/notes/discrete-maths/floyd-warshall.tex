\subsubsection{Floydův-Warshallův algoritmus}
\label{sssec:floyduv-warshalluv-algoritmus}

Vlastně jedinou výzvou na cestě k vyřešení FLP je umět aspoň rámcově efektivně
najít vzdálenost mezi všemi dvojicemi vrcholů. Bohužel, žádná pěkná věta jako
Pythagorova, která umožňuje okamžitě počítat vzdálenosti bodů v Eukleidovských
prostorech, v teorii grafů neexistuje a existovat nemůže.

Připomeňme, že v ohodnoceném grafu $G = (V,E,w)$ je vzdálenost vrcholů $u,v \in
V$ \hyperref[def:vzdalenost-v-grafu]{definována} jako váha nejkratší cesty.
Poznamenejme, že zde slovo \uv{nejkratší} chápeme ve smyslu \emph{váhy} cesty
(tedy součtu vah všech jejích hran), nikoli ve smyslu \emph{délky} cesty (tedy
počtu jejích hran). Asi bychom měli říkat \uv{nejlehčí} cesta, to je ale poněkud
nepřirozené...

Floydův-Warshallův algoritmus nesouvisí přímo s FLP. Je to algoritmus, který
nalezne vzdálenost (tedy váhu nejkratší cesty) mezi všemi dvojicemi vrcholů. Je
však zřejmé, jak znalost této informace vede okamžitě k řešení EFLP, resp. SFLP.
V moment, kdy známe vzdálenost každého vrcholu od každého, stačí pouze spočítat
excentricitu, resp. výstřednost, každého vrcholu a vybrat pouze ty s minimální.

Jistě není překvapením, že zkoušet z každého vrcholu všechny možné cesty do
všech ostatních vrcholů a z nich vybírat ty nejkratší, není dvakrát efektivní.
Floydův-Warshallův algoritmus stojí na dvou principech, jimž se především v
programování říká \emph{rekurze} a \emph{dynamické programování}. Jejich úplné
pochopení a nabytí schopnosti využívat může být časově náročné, ale
Floydův-Warshallův algoritmus jich využívá velmi přímočaře. Postupně si
rozebereme, že ze znalosti váhy nejkratší cesty mezi dvěma vrcholy,
\textbf{která využívá jen nějakou podmnožinu ostatních vrcholů}, lze zvětšováním
této podmnožiny získat nakonec váhu nejkratší cesty mezi těmito vrcholy v celém
grafu (odtud \emph{rekurze}). Dále, ze znalosti vzdálenosti mezi určitými
dvojicemi vrcholů můžeme rychle určit vzdálenost mezi párem, u kterého jsme ji
zatím neznali (odtud \emph{dynamické programování}).

Naší prací v této podsekci je dát předchozímu odstavci formální podobu. Ať $G =
(V,E,w)$ je souvislý ohodnocený graf, kde $V = \{v_1,\ldots,v_n\}$. Definujme
zobrazení $\delta: \{1,\ldots,n\}^3 \to \R^+$ následovně. Ať $\delta(i,j,k)$
je váha nejkratší cesty mezi $v_i$ a $v_j$ \textbf{využívající pouze vrcholy z
podmnožiny vrcholů} $\{v_1,\ldots,v_k\}$ (samozřejmě, kromě počátečního $v_i$ a
koncového $v_j$). I když je $G$ souvislý, tak taková cesta nemusí vždy
existovat; v takovém případě je $\delta(i,j,k) = \infty$. Je snadné si uvědomit,
že $d_G(v_i,v_j) = \delta(i,j,n)$, čili vzdálenost mezi vrcholy v grafu $G$ je
váha nejkratší cesty, která smí využít jeho všechny vrcholy.

Je zřejmé, že $\delta(i,j,k) \leq \delta(i,j,k-1)$, neboť máme jeden vrchol
navíc, a přes ten může vést nějaká kratší cesta. Základní, a vlastně jedinou,
myšlenkou Floydova-Warshallova algoritmu je pozorování, že když nastane situace,
kdy ${\delta(i,j,k) < \delta(i,j,k-1)}$, pak ta kratší cesta musí využívat
vrchol $v_k$. Ovšem, takovou cestu lze rozdělit na dvě -- na cestu z $v_i$ do
$v_k$ a cestu $v_k$ do $v_j$. Původní cesta $v_i\cdots v_k \cdots v_j$ využívala
pouze vrcholy z $\{v_1,\ldots,v_k\}$, takže obě její části, $v_1 \cdots v_k$ i
$v_k \cdots v_j$ využívají pouze vrcholy z~$\{v_1,\ldots,v_{k-1}\}$. Zároveň to
musejí být právě ty nejkratší cesty mezi $v_i$ a $v_k$ a $v_k$ a $v_j$, jinak by
celková cesta $v_1 \cdots v_k \cdots v_j$ nebyla ta nejkratší. Mrkněte na
\myref{obrázek}{fig:delta-i-j-k}.

\begin{figure}[h]
 \centering
 \begin{tikzpicture}
  \tikzset{vertex/.style = {shape=circle,fill,text=white,minimum size=6pt,inner
  sep=1pt}}
  \tikzset{->-/.style={decoration={ markings, mark=at position #1 with
  {\arrow{>[scale=1]}}},postaction={decorate}}}

  \node[vertex,myred,minimum size=9pt] (vi) at (-4,0) {};
  \node[left=0mm of vi,myred] {$v_i$};
  \node[vertex,myred,minimum size=9pt] (vj) at (4,0) {};
  \node[below right=-11pt and 0mm of vj,myred] {$v_j$};

  \node[vertex] (v1) at (-2,0) {};
  \node[above left=-1mm and -1mm of v1] {$v_1$};
  \node[vertex,mygreen] (v2) at (0,1) {};
  \node[above=0mm of v2,mygreen] {$v_2$};
  \node[vertex,mygreen] (v3) at (2,1) {};
  \node[above=0mm of v3,mygreen] {$v_3$};
  \node[vertex,myblue] (v4) at (1,-1) {};
  \node[below=0mm of v4,myblue] {$v_4$};

  \draw[thick] (vi) to node[midway,below] {$2$} (v1);

  \draw[thick,mygreen] (v1) to node[pos=0.45,above] {$2$} (v2);
  \draw[thick,mygreen] (v2) to node[midway,above] {$1$} (v3);
  \draw[thick,mygreen] (v3) to node[pos=0.55,above] {$2$} (vj);

  \draw[thick,myblue] (v1) to node[pos=0.45,below] {$3$} (v4);
  \draw[thick,myblue] (v4) to node[pos=0.55,below] {$1$} (vj);
 \end{tikzpicture}
 \caption{Zde $\delta(\clr{i},\clr{j},\clg{3}) = 7$, ale
 $\delta(\clr{i},\clr{j},\clb{4}) = 6$.}
 \label{fig:delta-i-j-k}
\end{figure}

Odtud plyne, že $\delta(i,j,k)$ je buď
\begin{itemize}
 \item rovna $\delta(i,j,k-1)$, pokud přes vrchol $v_k$ nevede žádná kratší
  cesta z~$v_i$ do $v_j$, nebo
 \item rovna součtu vah nejkratší cesty z $v_i$ do $v_k$ a nejkratší cesty z
  $v_k$ do $v_j$, pokud přes vrchol $v_k$ nějaká kratší cesta z $v_i$ do $v_j$
  vede. V symbolech je tento součet roven $\delta(i,k,k-1) + \delta(k,j,k-1)$.
\end{itemize}
Předchozí body se dají shrnout do jednoho zápisu, neboť $\delta(i,j,k)$, jakožto
váha \textbf{nejkratší} cesty, je vždy menším z obou množství. Konkrétně,
vzoreček stojící za celým Floydovým-Warshallovým algoritmem je tento:
\begin{equation*}
 \label{eq:delta-i-j-k}
 \tag{$\triangle$}
 \delta(i,j,k) = \min(\delta(i,j,k-1), \delta(i,k,k-1) + \delta(k,j,k-1)).
\end{equation*}
Možná už vás napadá, jak samotný algoritmus bude fungovat. Přeci, čísla
$\delta(i,j,k)$ mohu spočítat pro všechny dvojice $(i,j) \in \{1,\ldots,n\}^2$ v
moment, kdy znám $\delta(i,j,k-1)$ opět pro všechny dvojice $(i,j)$. Jenže,
$\delta(i,j,0)$ je váha hrany mezi $v_i$ a $v_j$, pokud existuje, jinak
$\infty$. Symbolicky,
\[
 \delta(i,j,0) =
 \begin{cases}
  w(v_iv_j), &\text{pokud } v_iv_j \in E,\\
  \infty, &\text{jinak}.
 \end{cases}
\]
To znamená, že znám hodnotu $\delta(i,j,0)$ pro všechny dvojice $(i,j)$, a tedy
postupným zvyšováním $k$ umím spočítat i všechny hodnoty $\delta(i,j,k)$, a tím
i nakonec, pro $k = n$, vzdálenosti mezi všemi páry vrcholů.

Nyní, když jsme myšlenku algoritmu doufám objasnili, zbývá jeho postup nějak
rozumně zapsat. Naštěstí to lze velmi přímočaře, dokonce výrazně snadněji než
například \hyperref[alg:kruskal]{Kruskalův algoritmus}.

Vlastně nám jde o to postupně nacházet kratší a kratší cesty mezi všemi
dvojicemi vrcholů. Ten fakt, že hledáme vzdálenosti mezi vskutku \textbf{všemi}
dvojicemi, nám umožňuje díky závěrům v předchozích odstavcích toto dělat
efektivně.

Algoritmus vytváří zobrazení $D: V^2 \to [0,\infty]$, kterýmť si vlastně
\uv{pama\-tuje} váhu zatím nejkratší nalezené cesty mezi každými dvěma vrcholy.
Na začátku algoritmu je tudíž $D(v,v) \coloneqq 0$ pro všechny $v \in V$ a
$D(u,v) \coloneqq w(uv)$ pro všechny dvojice $(u,v) \in V^2$, mezi kterými vede
hrana. Pro všech\-ny ostatní dvojice $(u,v)$ pokládáme $D(u,v) \coloneqq
\infty$.

Opírajíce sebe o předšedší úvahy víme, že budeme muset učinit v algoritmu $n$
kroků (to je počet vrcholů $G$) a v $k$-tém kroku vždy počítat hodnoty
$\delta(i,j,k)$ pro všechny dvojice $(i,j)$. Přejeme si tudíž, aby hodnoty z
předchozího kroku, tj. $\delta(i,j,k-1)$, byly před spuštěním $k$-tého kroku
všechny již zakódovány v zobrazení $D$, jakožto čísla $D(v_i,v_j)$,
představující \textbf{zatím} nejkratší známé cesty mezi všemi dvojicemi vrcholů.
Rovnost \eqref{eq:delta-i-j-k} se pročež překládá (\textbf{v rámci $k$-tého
kroku algoritmu}) do rovnosti
\[
 D(v_i,v_j) = \min(D(v_i,v_j), D(v_i,v_k) + D(v_k, v_j)).
\]
Po skončení $n$-tého kroku bude tedy zobrazení $D$ přesně odpovídat zobrazení
$d_G$. Konečně můžeme přikročit k samotnému pseudokódu, jenž si prohlédněte
\hyperref[alg:floyd-warshall]{níže}.

\begin{algorithm}
 \caption{Floydův-Warshallův algoritmus.}
 \label{alg:floyd-warshall}

 \SetKwInOut{Input}{input}
 \SetKwInOut{Output}{output}
 \SetKw{KwReturn}{return}

 \Input{souvislý ohodnocený graf $G = (V,E,w)$, kde $V = \{v_1,\ldots,v_n\}$}
 \Output{zobrazení $D:V^2 \to [0,\infty]$ takové, že $D \equiv d_G$}
 \BlankLine
 \emph{Inicializace}\;
 \For{$i \leftarrow 1$ \KwTo $n$} {
  \For{$j \leftarrow 1$ \KwTo $n$} {
   \If{$i = j$} {
    \emph{Vzdálenost od vrcholu k němu samému je $0$}\;
    $D(v_i,v_i) \leftarrow 0$\;
   }
   \Else{
    \If{$v_iv_j \in E$} {
     \emph{Vzdálenost mezi různými vrcholy je váha hrany, pokud existuje}\;
     $D(v_i,v_j) \leftarrow w(v_iv_j)$\;
    }
    \Else {
     \emph{Jinak nastavíme hodnotu na $\infty$}\;
     $D(v_i,v_j) \leftarrow \infty$\;
    }
   }
  }
 }
 \BlankLine
 \emph{Postupně pro každé $k$ od $1$ do $n$ budeme zmenšovat hodnoty
 $D(v_i,v_j)$ pro všechny dvojice $(i,j)$}\;
 \For{$k \leftarrow 1$ \KwTo $n$} {
  \For{$i \leftarrow 1$ \KwTo $n$} {
   \For{$j \leftarrow 1$ \KwTo $n$} {
    $D(v_i,v_j) \leftarrow \min(D(v_i,v_j), D(v_i,v_k) + D(v_k,v_j))$\;
   }
  }
 }
 \KwReturn $D$\;
\end{algorithm}

\begin{exercise}
 Dokažte, že \hyperref[alg:floyd-warshall]{Floydův-Warshallův} algoritmus selže,
 když připustíme i záporná ohodnocení hran.
\end{exercise}

