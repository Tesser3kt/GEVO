\subsubsection{Floydův-Warshallův algoritmus}
\label{sssec:floyduv-warshalluv-algoritmus}

Vlastně jedinou výzvou při na cestě vyřešení FLP je umět aspoň rámcově efektivně
najít vzdálenost dvou vrcholů. Bohužel, žádná pěkná věta jako Pythagorova, která
umožňuje okamžitě počítat vzdálenosti bodů v Eukleidovských prostorech, v teorii
grafů neexistuje a existovat nemůže.

Připomeňme, že v ohodnoceném grafu $G = (V,E,w)$ je vzdálenost vrcholů $u,v \in
V$ \hyperref[def:vzdalenost-v-grafu]{definována} jako váha nejkratší cesty.
Poznamenejme, že zde slovo \uv{nejkratší} chápeme ve smyslu \emph{váhy} cesty
(tedy součtu vah všech jejích hran), nikoli ve smyslu \emph{délky} cesty (tedy
počtu jejích hran). Asi bychom měli říkat \uv{nejlehčí} cesta, to je ale poněkud
nepřirozené...

Floydův-Warshallův algoritmus nesouvisí přímo s FLP. Je to algoritmus, který
nalezne vzdálenost (tedy váhu nejkratší cesty) mezi všemi dvojicemi vrcholů. Je
však zřejmé, jak znalost této informace vede okamžitě k řešení EFLP, resp. SFLP.
V moment, kdy známe vzdálenost každého vrcholu od každého, stačí pouze spočítat
excentricitu, resp. výstřednost, každého vrcholu a vybrat pouze ty s minimální.

Jistě není překvapením, že zkoušet z každého vrcholu všechny možné cesty do
všech ostatních vrcholů a z nich vybírat ty nejkratší není dvakrát efektivní.
Floydův-Warshallův algoritmus stojí na dvou principech, jimž se především v
programování říká \emph{rekurze} a \emph{dynamické programování}. Jejich úplné
pochopení a nabytí schopnosti využívat může být časově náročné, ale
Floydův-Warshallův algoritmus jich využívá velmi přímočaře. Postupně si
rozebereme, že ze znalosti váhy nejkratší cesty mezi dvěma vrcholy,
\textbf{která využívá jen nějakou podmnožinu ostatních vrcholů}, lze zvětšováním
této podmnožiny získat nakonec váhu nejkratší cesty mezi těmito vrcholy v celém
grafu (odtud \emph{rekurze}). Dále, ze znalosti vzdálenosti mezi určitými
dvojicemi vrcholů můžeme rychle určit vzdálenost mezi párem, u kterého jsme ji
zatím neznali (odtud \emph{dynamické programování}).

Naší prací v této podsekci je dát předchozímu odstavci formální podobu. Ať $G =
(V,E,w)$ je souvislý ohodnocený graf, kde $V = \{v_1,\ldots,v_n\}$. Definujme
funkci $\delta(i,j,k): \{1,\ldots,n\}^3 \to \R^+$ následovně. Ať $\delta(i,j,k)$
je váha nejkratší cesty mezi $v_i$ a $v_j$ \textbf{využívající pouze vrcholy z
podmnožiny vrcholů} $\{v_1,\ldots,v_k\}$ (samozřejmě, kromě počátečního $v_i$ a
koncového $v_j$). I když je $G$ souvislý, tak taková cesta nemusí vždy
existovat, v takovém případě je $\delta(i,j,k) = \infty$.

Je zřejmé, že $\delta(i,j,k) \leq \delta(i,j,k-1)$, neboť máme jeden vrchol
navíc, a přes ten může vést nějaká kratší cesta. Základní, a vlastně jedinou,
myšlenkou Floydova-Warshallova algoritmu je pozorování, že když nastane situace,
kdy ${\delta(i,j,k) < \delta(i,j,k-1)}$, pak ta kratší cesta musí využívat
vrchol $v_k$. Ovšem, tuto cestu lze v takovém případě rozdělit na dvě -- na
cestu z $v_i$ do $v_k$ a na cestu $v_k$ do $v_j$. Původní cesta $v_i\cdots v_k
\cdots v_j$ využívala pouze vrcholy z $\{v_1,\ldots,v_k\}$, takže obě její
části, $v_1 \cdots v_k$ i $v_k \cdots v_j$ využívají pouze vrcholy
z~$\{v_1,\ldots,v_{k-1}\}$. Zároveň to musejí být právě ty nejkratší cesty mezi
$v_i$ a $v_k$ a $v_k$ a $v_j$, jinak by celková cesta $v_1 \cdots v_k \cdots
v_j$ nebyla ta nejkratší. Mrkněte na obrázek

\begin{figure}[h]
 \centering
 \begin{tikzpicture}
  \tikzset{vertex/.style = {shape=circle,fill,text=white,minimum size=6pt,inner
  sep=1pt}}
  \tikzset{->-/.style={decoration={ markings, mark=at position #1 with
  {\arrow{>[scale=1]}}},postaction={decorate}}}

  \node[vertex,myred,minimum size=9pt] (vi) at (-4,0) {};
  \node[left=0mm of vi,myred] {$v_i$};
  \node[vertex,myred,minimum size=9pt] (vj) at (4,0) {};
  \node[below right=-11pt and 0mm of vj,myred] {$v_j$};

  \node[vertex] (v1) at (-2,0) {};
  \node[above left=-1mm and -1mm of v1] {$v_1$};
  \node[vertex,mygreen] (v2) at (0,1) {};
  \node[above=0mm of v2,mygreen] {$v_2$};
  \node[vertex,mygreen] (v3) at (2,1) {};
  \node[above=0mm of v3,mygreen] {$v_3$};
  \node[vertex,myblue] (v4) at (1,-1) {};
  \node[below=0mm of v4,myblue] {$v_4$};

  \draw[thick] (vi) to node[midway,below] {$2$} (v1);

  \draw[thick,mygreen] (v1) to node[pos=0.45,above] {$2$} (v2);
  \draw[thick,mygreen] (v2) to node[midway,above] {$1$} (v3);
  \draw[thick,mygreen] (v3) to node[pos=0.55,above] {$2$} (vj);

  \draw[thick,myblue] (v1) to node[pos=0.45,below] {$3$} (v4);
  \draw[thick,myblue] (v4) to node[pos=0.55,below] {$1$} (vj);
 \end{tikzpicture}
 \caption{Zde $\delta(\clr{i},\clr{j},\clg{3}) = 7$, ale
 $\delta(\clr{i},\clr{j},\clb{4}) = 6$.}
 \label{fig:delta-i-j-k}
\end{figure}
