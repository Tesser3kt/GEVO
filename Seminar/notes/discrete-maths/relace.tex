\subsection{Relace}
\label{ssec:relace}

Pojem \emph{relace} zobecňuje věci jako zobrazení (se kterým jste se setkali,
ale říkali jste mu bůhvíproč funkce) nebo uspořádání (které taky znáte, jen vám
bůhvíproč neprozradili, oč jde).

Základní myšlenkou je to, že i relace -- vztahy mezi objekty se dají pomocí
množin (a jejich součinu) úspěšně definovat. Celá matematika, kterou jste dosud
poznali, je založená na \emph{teorii množin}, jinak řečeno, \textbf{všechno} je
množina.

\begin{definition}[Relace]
 Jsou-li $A,B$ množiny, pak \textbf{relací} mezi $A$ a $B$ nazveme
 \emph{libovolnou} podmnožinu $A \times B$. Je-li $A = B$, pak $R$ nazýváme
 relací na $A$.
\end{definition}

Pojem relace v matematice je založen na konceptu, že vztah mezi množinami je
dokonale popsán výpisem všech dvojic prvků, které v tom vztahu jsou. To se
trochu liší od běžného chápání slova \uv{vztah}. Asi byste nebyli úplně
spokojení, kdybychom vám tvrdili, že vztah manželský na množině všech lidí je to
samé, co výpis všech manželských párů. Z toho důvodu bude asi lepší se držet
latinské verse, \uv{relace}.

Protože nejstarší typy relací, mezi nimi třebas $<$ nebo $=$, lidé používali
ještě před vznikem samotné teorie množin, značení je zde trochu matoucí. Fakt,
že dvojice $(x,y) \in A \times B$ je v relaci $R$, nezapisujeme (jak by se
čekalo) $(x,y) \in R$, ale spíš $xRy$. Podobně jako nepíšeme $(x,y) \in \; <$,
ale $x < y$.

Jako spoustu věcí v matematice, relace je dobré si umět vizualizovat. Ukážeme si
teď tři standardní způsoby, jak si lidé relace kreslí.

\subsubsection{Kreslení relací}
\label{sssec:kresleni-relaci}

Po celou podsekci budeme předpokládat, že máme množiny $\clr{A = \{1, 2, 3,
4\}}$ a $\clb{B = \{a, b, c\}}$.

Jedním ze způsobů, jak se dají kreslit relace, je \emph{mříž}. Uvážíme relaci
\[
 \clg{R = \{(1, a), (1, b), (2, c), (3, a), (3, b), (3, c), (4, b)\}}
\]
mezi $\clr{A}$ a $\clb{B}$. Vizualizaci součinu $\clr{A} \times \clb{B}$ a
relace $\clg{R}$ pomocí mříže vidíte na \hyperref[fig:relace-mriz]{obrázku
\ref*{fig:relace-mriz}}.

\begin{figure}[h]
 \centering
 \begin{tikzpicture}
  \node (1) at (0, 0) {\clr{1}};
  \node (2) at (1, 0) {\clr{2}};
  \node (3) at (2, 0) {\clr{3}};
  \node (4) at (3, 0) {\clr{4}};

  \node (a) at (-1, 1) {\clb{a}};
  \node (b) at (-1, 2) {\clb{b}};
  \node (c) at (-1, 3) {\clb{c}};

  \node[circle,draw,fill=black,minimum size=2mm,inner sep=0pt,outer
  sep=0pt]
  (1a) at (0, 1) {};
  \node[circle,draw,fill=black,minimum size=2mm,inner sep=0pt,outer
  sep=0pt]
  (2a) at (1, 1) {};
  \node[circle,draw,fill=black,minimum size=2mm,inner sep=0pt,outer
  sep=0pt]
  (3a) at (2, 1) {};
  \node[circle,draw,fill=black,minimum size=2mm,inner sep=0pt,outer
  sep=0pt]
  (4a) at (3, 1) {};
  \node[circle,draw,fill=black,minimum size=2mm,inner sep=0pt,outer
  sep=0pt]
  (1b) at (0, 2) {};
  \node[circle,draw,fill=black,minimum size=2mm,inner sep=0pt,outer
  sep=0pt]
  (1c) at (0, 3) {};
  \node[circle,draw,fill=black,minimum size=2mm,inner sep=0pt,outer
  sep=0pt]
  (2b) at (1, 2) {};
  \node[circle,draw,fill=black,minimum size=2mm,inner sep=0pt,outer
  sep=0pt]
  (2c) at (1, 3) {};
  \node[circle,draw,fill=black,minimum size=2mm,inner sep=0pt,outer
  sep=0pt]
  (3b) at (2, 2) {};
  \node[circle,draw,fill=black,minimum size=2mm,inner sep=0pt,outer
  sep=0pt]
  (3c) at (2, 3) {};
  \node[circle,draw,fill=black,minimum size=2mm,inner sep=0pt,outer
  sep=0pt]
  (4b) at (3, 2) {};
  \node[circle,draw,fill=black,minimum size=2mm,inner sep=0pt,outer
  sep=0pt]
  (4c) at (3, 3) {};

  \node[circle,draw=mygreen,thick] at (1a.center) {};
  \node[circle,draw=mygreen,thick] at (1b.center) {};
  \node[circle,draw=mygreen,thick] at (2c.center) {};
  \node[circle,draw=mygreen,thick] at (3a.center) {};
  \node[circle,draw=mygreen,thick] at (3b.center) {};
  \node[circle,draw=mygreen,thick] at (3c.center) {};
  \node[circle,draw=mygreen,thick] at (4b.center) {};
 \end{tikzpicture}
 \caption{Kreslení relace $\clg{R} \subseteq \clr{A} \times \clb{B}$ pomocí
 mříže.}
 \label{fig:relace-mriz}
\end{figure}

Ještě jeden užitečný způsob kreslení, který funguje pro obecné relace, je
kreslení pomocí šipek.  V zásadě si člověk zobrazí obě množiny jako sloupce bodů
a mezi příslušnými body kreslí šipky. Například jako na
\hyperref[fig:relace-sipky]{obrázku~\ref*{fig:relace-sipky}}.

\begin{figure}[h]
 \centering
 \begin{tikzpicture}
  \node[circle,draw,fill=black,minimum size=2mm,inner sep=0pt,outer
  sep=0pt]
  (1) at (0, 3) {};
  \node[circle,draw,fill=black,minimum size=2mm,inner sep=0pt,outer
  sep=0pt]
  (2) at (0, 2) {};
  \node[circle,draw,fill=black,minimum size=2mm,inner sep=0pt,outer
  sep=0pt]
  (3) at (0, 1) {};
  \node[circle,draw,fill=black,minimum size=2mm,inner sep=0pt,outer
  sep=0pt]
  (4) at (0, 0) {};
  \node[circle,draw,fill=black,minimum size=2mm,inner sep=0pt,outer
  sep=0pt]
  (a) at (3, 3) {};
  \node[circle,draw,fill=black,minimum size=2mm,inner sep=0pt,outer
  sep=0pt]
  (b) at (3, 2) {};
  \node[circle,draw,fill=black,minimum size=2mm,inner sep=0pt,outer
  sep=0pt]
  (c) at (3, 1) {};

  \node[left=2mm of 1] {\clr{1}};
  \node[left=2mm of 2] {\clr{2}};
  \node[left=2mm of 3] {\clr{3}};
  \node[left=2mm of 4] {\clr{4}};
  \node[right=2mm of a] {\clb{a}};
  \node[right=2mm of b] {\clb{b}};
  \node[right=2mm of c] {\clb{c}};
 
  \node[right=.25mm of 1.center] (1-east) {};
  \node[right=.25mm of 2.center] (2-east) {};
  \node[right=.25mm of 3.center] (3-east) {};
  \node[right=.25mm of 4.center] (4-east) {};
  \node[left=.25mm of a.center] (a-west) {};
  \node[left=.25mm of b.center] (b-west) {};
  \node[left=.25mm of c.center] (c-west) {};


  \draw[-latex,mygreen,thick] (1-east) -- (a-west);
  \draw[-latex,mygreen,thick] (1-east) -- (b-west);
  \draw[-latex,mygreen,thick] (2-east) -- (c-west);
  \draw[-latex,mygreen,thick] (3-east) -- (a-west);
  \draw[-latex,mygreen,thick] (3-east) -- (b-west);
  \draw[-latex,mygreen,thick] (3-east) -- (c-west);
  \draw[-latex,mygreen,thick] (4-east) -- (b-west);

  \node (R) at (1.5, 3.5) {$\clg{R}$};
 \end{tikzpicture}
 \caption{Kreslení relace $\clg{R} \subseteq \clr{A} \times \clb{B}$ pomocí
 šipek.}
 \label{fig:relace-sipky}
\end{figure}

Tenhle způsob se může zdát méně přehledný než mříž, ale má svoje nesporné
využití, především v oblasti \emph{skládání} relací, kterým se budeme zabývat za
chvíli.

Ještě před tím si ale ukážeme způsob, jak přehledně kreslit relace na nějaké
množině. Řekněme, že tentokrát je třeba
\[
 \clg{R \coloneqq \{(1, 1), (1, 2), (2, 3), (3, 4), (3, 3), (4, 1), (4, 2)\}}
\]
relace na množině $\clr{A}$. Množinu $\clr{A}$ si nakreslíme jako body v rovině
a relaci $\clg{R}$ jako šipky a smyčky. Vizte
\hyperref[fig:relace-sipky-a-smycky]{obrázek~\ref*{fig:relace-sipky-a-smycky}}.

\begin{figure}[h]
 \centering
 \begin{tikzpicture}
  \node[circle,draw,fill=black,minimum size=2mm,inner sep=0pt,outer
  sep=0pt]
  (1) at (0, 0) {};
  \node[circle,draw,fill=black,minimum size=2mm,inner sep=0pt,outer
  sep=0pt]
  (2) at (2, 0) {};
  \node[circle,draw,fill=black,minimum size=2mm,inner sep=0pt,outer
  sep=0pt]
  (3) at (4, 0) {};
  \node[circle,draw,fill=black,minimum size=2mm,inner sep=0pt,outer
  sep=0pt]
  (4) at (6, 0) {};

  \node[below=2mm of 1] {$\clr{1}$};
  \node[below=2mm of 2] {$\clr{2}$};
  \node[below=2mm of 3] {$\clr{3}$};
  \node[below=2mm of 4] {$\clr{4}$};

  \draw[thick,-latex,mygreen,bend right] (1) to (2);
  \draw[thick,-latex,mygreen,bend right] (2) to (3);
  \draw[thick,-latex,mygreen,bend right] (3) to (4);
  \draw[thick,-latex,mygreen,bend right=30] (4) to (1);
  \draw[thick,-latex,mygreen,bend right=30] (4) to (2);

  \draw[-latex,thick,mygreen] (1.east) to [out=60,in=120,looseness=10] (1.west);
  \draw[-latex,thick,mygreen] (3.east) to [out=60,in=120,looseness=10] (3.west);
 \end{tikzpicture}
 \caption{Kreslení relace $\clg{R}$ na $\clr{A}$ pomocí šipek a smyček.}
 \label{fig:relace-sipky-a-smycky}
\end{figure}

\subsubsection{Skládání relací}
\label{sssec:skladani-relaci}

V této podsekci si řekneme, co znamená, že dvě (nebo více) relace složíme
dohromady. Tato operace se dá vnímat jako jakési \uv{zobecnění} skládání
zobrazení/funkcí. Jak si ale ukážeme, zobrazení jsou speciálním typem relací,
takže tahle představa není úplně vhodná.

Pro jednoduchost se budeme soustředit na relace na nějaké množině $A$. Tohle
ovšem není nutné; mám-li relaci $R \subseteq A \times B$ a relaci $S \subseteq B
\times C$, vždy je mohu složit a dostat relaci mezi $A$ a $C$.

Skládání relací není nijak divoká věc a vztahy (například mezi lidmi) v~životě
běžně skládáme, ale málokdy se na to asi díváme tímto způsobem. Například,
řekněme, že \clr{Adéla} má přítelkyni \clb{Simona} a \clb{Simona} má přítelkyni
\clg{Terezu}. Když složíme relace \uv{býti přítelkyně \clr{Adély}} a \uv{býti
přítelkyně \clb{Simony}} dostaneme relaci, ve které je \clg{Tereza} přítelkyně
\clr{Adély}. Na druhý příklad, třeba samotné přísloví \uv{Nepřítel mého
nepřítele je můj přítel.}, se dá vyložit jako skládání relací.

Teď formálně.

\begin{definition}[Složení relací]
 \label{def:slozeni-relaci}
 Mějme množinu $A$ a relace $R,S \subseteq A \times A = A^2$. Složením relací
 $R$ a $S$ nazveme množinu
 \[
  \{(x,z) \in A^2 \mid \exists y \in A: xRy \wedge yRz\}
 \]
 a značíme ji $R \circ S$.
\end{definition}

Řečeno asi možná třeba trošku lidštěji, když pro dané $x,z \in A$ najdu takový
prvek $y \in A$, že dvojice $(x,y)$ je v relaci $R$ a dvojice $(y,z)$ je v
relaci $S$, pak $(x,z)$ je v relaci $R \circ S$. Vlastně $(x,y)$ a $(y,z)$
slepím dohromady skrze $y$.

\begin{example}
 Řekněme, že je $A = \{1,2,3,4\}$ a máme relace
 \begin{align*}
  \clg{R} &\coloneqq \clg{\{(1,2),(1,3),(2,2),(2,4)\}},\\
  \clp{S} & \coloneqq \clp{\{(1,3),(2,1),(2,2),(3,1),(4,3)\}}
 \end{align*}
 na $A$. V \hyperref[sssec:kresleni-relaci]{podsekci o kreslení relací} jsme
 zmínili, že šipky jsou velmi užitečné při skládání. Teď uvidíte proč. Když si
 obě relace nakreslíme přímo vedle sebe, dostaneme
 \hyperref[fig:skladani-relaci]{obrázek~\ref*{fig:skladani-relaci}}.
 \begin{figure}[H]
  \centering
  \begin{tikzpicture}
   \node[circle,draw,fill=black,minimum size=2mm,inner sep=0pt,outer
   sep=0pt]
   (1) at (0, 3) {};
   \node[circle,draw,fill=black,minimum size=2mm,inner sep=0pt,outer
   sep=0pt]
   (2) at (0, 2) {};
   \node[circle,draw,fill=black,minimum size=2mm,inner sep=0pt,outer
   sep=0pt]
   (3) at (0, 1) {};
   \node[circle,draw,fill=black,minimum size=2mm,inner sep=0pt,outer
   sep=0pt]
   (4) at (0, 0) {};
   
   \node[circle,draw,fill=black,minimum size=2mm,inner sep=0pt,outer
   sep=0pt]
   (11) at (3, 3) {};
   \node[circle,draw,fill=black,minimum size=2mm,inner sep=0pt,outer
   sep=0pt]
   (22) at (3, 2) {};
   \node[circle,draw,fill=black,minimum size=2mm,inner sep=0pt,outer
   sep=0pt]
   (33) at (3, 1) {};
   \node[circle,draw,fill=black,minimum size=2mm,inner sep=0pt,outer
   sep=0pt]
   (44) at (3, 0) {};

   \node[circle,draw,fill=black,minimum size=2mm,inner sep=0pt,outer
   sep=0pt]
   (111) at (6, 3) {};
   \node[circle,draw,fill=black,minimum size=2mm,inner sep=0pt,outer
   sep=0pt]
   (222) at (6, 2) {};
   \node[circle,draw,fill=black,minimum size=2mm,inner sep=0pt,outer
   sep=0pt]
   (333) at (6, 1) {};
   \node[circle,draw,fill=black,minimum size=2mm,inner sep=0pt,outer
   sep=0pt]
   (444) at (6, 0) {};

   \node[left=2mm of 1] {$1$};
   \node[left=2mm of 2] {$2$};
   \node[left=2mm of 3] {$3$};
   \node[left=2mm of 4] {$4$};

   \node[right=2mm of 111] {$1$};
   \node[right=2mm of 222] {$2$};
   \node[right=2mm of 333] {$3$};
   \node[right=2mm of 444] {$4$};

   \node[above=1mm of 11] {$1$};
   \node[above=1mm of 22] {$2$};
   \node[above=1mm of 33] {$3$};
   \node[above=1mm of 44] {$4$};

   \node[right=.25mm of 1.center] (1-east) {};
   \node[right=.25mm of 2.center] (2-east) {};
   \node[right=.25mm of 3.center] (3-east) {};
   \node[right=.25mm of 4.center] (4-east) {};

   \node[left=.25mm of 111.center] (111-west) {};
   \node[left=.25mm of 222.center] (222-west) {};
   \node[left=.25mm of 333.center] (333-west) {};
   \node[left=.25mm of 444.center] (444-west) {};

   \node[left=.25mm of 11.center] (11-west) {};
   \node[left=.25mm of 22.center] (22-west) {};
   \node[left=.25mm of 33.center] (33-west) {};
   \node[left=.25mm of 44.center] (44-west) {};

   \node[right=.25mm of 11.center] (11-east) {};
   \node[right=.25mm of 22.center] (22-east) {};
   \node[right=.25mm of 33.center] (33-east) {};
   \node[right=.25mm of 44.center] (44-east) {};

   \node at (1.5, 3.5) {$\clg{R}$};
   \draw[-latex,mygreen,thick] (1-east) -- (22-west);
   \draw[-latex,mygreen,thick] (1-east) -- (33-west);
   \draw[-latex,mygreen,thick] (2-east) -- (22-west);
   \draw[-latex,mygreen,thick] (2-east) -- (44-west);

   \node at (4.5, 3.5) {$\clp{S}$};
   \draw[-latex,mypurple,thick] (11-east) -- (333-west);
   \draw[-latex,mypurple,thick] (22-east) -- (111-west);
   \draw[-latex,mypurple,thick] (22-east) -- (222-west);
   \draw[-latex,mypurple,thick] (33-east) -- (111-west);
   \draw[-latex,mypurple,thick] (44-east) -- (333-west);
  \end{tikzpicture}
  \caption{Složení relací $\clg{R}$ a $\clp{S}$.}
  \label{fig:skladani-relaci}
 \end{figure}

 V roli $x$ z \hyperref[def:slozeni-relaci]{Definice~\ref*{def:slozeni-relaci}}
 je zde první sloupec, v roli $y$ druhý a v roli $z$ třetí. Čili, prvek $(x,z)$
 bude v relaci $R \circ S$ jenom tehdy, když najdu v~prostředním sloupci prvek
 $y$ (aspoň jeden, ale klidně víc), přes který dokážu po šipkách dojít z $x$ do
 $z$.

 Z obrázku je teď už zřejmé, že
 \[
  R \circ S = \{(1, 1), (1, 2), (2, 1), (2, 2), (2, 3)\}.
 \]
\end{example}
