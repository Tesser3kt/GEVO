\subsection{Matematická indukce}
\label{ssec:matematicka-indukce}

Indukce je základní důkazovou technikou v diskrétní matematice. Je to jeden
možný, ale zcela jistě nejoblíbenější, způsob, jak dokazovat libovolná tvrzení o
přirozených číslech, která jsou vlastně právě tím číselným oborem, který studuje
diskrétní matematika.

Princip indukce spočívá v tom, že přirozená čísla jsou definována v zásadě velmi
jednoduše. Libovolná množina, která má nějaký \uv{základní prvek} (třeba
jedničku) a spolu s každým prvkem má i jeho bezprostředního následníka (třeba to
číslo o jedna větší), je automaticky \uv{ta samá množina} jako přirozená čísla.

Pokud byste měli chuť se podívat na formální definici přirozených čísel a
dalších souvisejících věcí, doporučujeme vyhledat klíčová slova \emph{Peanova
aritmetika}, která je vlastně (možná kecám, ale myslím, že nejmenším možným)
systémem axiomů (kategoricky platných výroků), jenž buduje ryze logický základ
pro aritmetiku.

My si ale vystačíme s následujícím zjednodušením.

\begin{claim}[Definice přirozených čísel]
 Nechť $A$ je množina, která splňuje, že
 \begin{itemize}
  \item $1 \in A$,
  \item je-li $n \in A$, pak rovněž $n+1 \in A$.
 \end{itemize}
 Potom $A = \N$.
\end{claim}
\begin{proof}
 Nedokazuje se, je to axiom (konkrétně pátý) Peanovy aritmetiky.
\end{proof}

Žádáme, abyste si dali chvíli a zamysleli nad významem tvrzení. Zevrubně řečeno
říká, že, pokud umím dokázat, že
\begin{itemize}
 \item tvrzení platí pro první přirozené číslo a
 \item za předpokladu, že tvrzení platí pro $n$, platí pro $n + 1$,
\end{itemize}
pak dané tvrzení platí pro všechna přirozená čísla. Tyhle dva důkazy totiž
dohromady dávají následující (nekonečný) řetězec důkazů:
\begin{enumerate}
 \item (Nějaké) tvrzení platí pro $n = 1$.
 \item Jestliže tvrzení platí pro $n = 1$, pak platí pro $n = 2$.
 \item Jestliže tvrzení platí pro $n = 2$, pak platí pro $n = 3$.

 \centering $\vdots$
\end{enumerate}

Princip indukce asi není přehnaně složitý, ale získat dostatek zkušenosti, aby
jej člověk uměl neomylně aplikovat, je výrazně obtížnější. Pár příkladů snad s
tímto krokem pomůže. Budeme je záměrně formulovat jako lemmata či tvrzení,
jelikož indukce je v prvé řadě důkazová technika. Doporučujeme, abyste důkazy
četli se zvýšenou pozorností.

\begin{lemma}
 Pro každé $n \in \N$ platí
 \[
  \sum_{i=0}^{n} 2^{i} = 2^{n+1} - 1.
 \]
\end{lemma}
\begin{proof}
 Dokazujeme indukcí. Protože suma začíná od $0$, je prvním prvkem, pro který
 musí tvrzení platit, v tomto případě právě $n = 0$. Dosazením zjistíme, že
 \[
  \sum_{i=0}^{0} 2^{i} = 2^{0} = 1 = 2^{0 + 1} - 1
 \]
 tedy tvrzení platí pro $n = 0$. Předpokládáme, že tvrzení platí pro všech\-na
 přirozená čísla až do nějakého $n \in \N$ a z tohoto předpokladu odvodíme, že
 platí i pro $n + 1$. Počítáme
 \[
  \sum_{i=0}^{n + 1} 2^{i} = \sum_{i=0}^{n} 2^{i} + 2^{n+1}.
 \]
 Ovšem, z předpokladu dostaneme
 \[
  \sum_{i=0}^{n} 2^{i} = 2^{n+1}-1,
 \]
 což dohromady s předchozím výpočtem dává
 \[
  \sum_{i=0}^{n + 1} 2^{i} = \clr{\sum_{i=0}^{n} 2^{i}} + 2^{n+1} = \clr{2^{n +
  1} - 1} + 2^{n+1} = 2 \cdot 2^{n+1} - 1 = 2^{n+2} - 1,
 \]
 jak jsme chtěli ukázat. Důkaz je podle principu indukce ukončen.
\end{proof}

\begin{lemma}
 \label{lemma:deleni-trojkou}
 Pro všechna $n \in \N$ platí, že
 \[
  3 \mid n \Rightarrow 3 \mid n^2,
 \]
 tedy, pokud $3$ dělí $n$, pak $3$ dělí $n^2$.
\end{lemma}
Tady by se jistě leckdo rád odvolal třeba na prvočíselné rozklady. Je ale dobré
si uvědomit, že fakt, že každé přirozené číslo má jednoznačný rozklad na
prvočísla, není samozřejmý. Ve skutečnosti zabere určitou práci toto dokázat. Či
vy jste viděli nějaký přímočarý důkaz, že jdou čísla rozkládat na prvočísla?
Opravdu lze \emph{každé} číslo rozložit na prvočísla a opravdu to lze
\emph{pouze jediným způsobem}?

\begin{enhproof}[lemmatu~\ref{lemma:deleni-trojkou}]
 Dokazujeme indukcí.

 První přirozené číslo, pro které má smysl tvrzení dokázat, je $n = 3$. Pak
 vskutku $3 \mid n = 3$ a $3 \mid n^2 = 9$.

 Předpokládejme, že výrok $3 \mid n \Rightarrow 3 \mid n^2$ platí pro nějaké
 $n \in \N$. Nejbližší další přirozené číslo po $n$, které je dělitelné $3$, je
 $n + 3$. Tedy, víme, že když $3 \mid n$, pak $3 \mid n + 3$ a také $3 \mid
 n^2$. Z těchto dvou faktů odvodíme, že $3 \mid (n+3)^2$.

 Máme $(n + 3)^2 = n^2 + 6n + 9$. Protože $3 \mid 6$ a $3 \mid 9$, také $3 \mid
 6n + 9$. Předpokládáme, že $3 \mid n^2$, dohromady tudíž $3 \mid n^2 + 6n + 9 =
 (n+3)^2$, jak jsme chtěli.
\end{enhproof}

Tři cvičení nakonec.
\newpage
\begin{exercise}
 Dokažte indukcí, že
 \[
  \sum_{i=1}^{n} i 2^{i} = (n - 1)2^{n+1} + 2.
 \]
\end{exercise}

\begin{exercise}[Fibonacciho čísla a zlatý řez]
 Tak zvaná \emph{Fibonacciho} posloupnost je definována tak, že další člen
 dostanu jako součet dvou předchozích. Formálně, $F_0 = 0, F_1 = 1$ a $F_n =
 F_{n - 1} + F_{n - 2}$, kde $n \in \N$. Dokažte indukcí, že
 \[
  \frac{F_n}{F_{n-1}} \leq \frac{1+\sqrt{5}}{2}
 \]
 pro všechna $n \in \N$.

 Číslu  $(1+\sqrt{5}) / 2$ se někdy říká hodnota \uv{zlatého řezu} (protože je
 to v jistém smyslu \emph{ideální} poměr mezi délkami dvěma bezprostředních
 úseček -- internet poví víc). Jestli si někdy ukážeme limity, pak dokážeme
 tento výsledek zdokonalit v tom smyslu, že platí
 \[
  \frac{F_n}{F_{n-1}} \xrightarrow{n \to \infty} \frac{1+\sqrt{5}}{2}. 
 \]
\end{exercise}

\begin{exercise}
 Nakresleme $n$ přímek v rovině, a to tak, že
 \begin{itemize}
  \item žádné 2 nejsou rovnoběžné a
  \item žádné 3 se neprotínají v jednom bodě.
 \end{itemize}
 Dokažte indukcí, že takhle nakreslené přímky rozdělují rovinu na $n(n+1) / 2 +
 1$ částí.
\end{exercise}
