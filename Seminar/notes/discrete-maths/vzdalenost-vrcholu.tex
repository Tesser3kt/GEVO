\subsection{Vzdálenost vrcholů}
\label{ssec:vzdalenost-vrcholu}

Sekci motivujeme úlohou \uv{jako ze života}, která ve mě budí jistou míru
nostalgie, jelikož jsem ji řešil na konci prvního ročníku v rámci zkoušky
z~programování.

\begin{problem}[Rodinný výlet]
 \label{prob:rodinny-vylet}
 Je prodloužený víkend a slezská rodina Koláčků plánuje cyklistický výlet
 oblastí Karviná. Jedou děda Koláček se svou chotí, babičkou Koláčkovou, a
 jejich čtyři uřvaná rozmazlená vnoučata -- Matouš, Marek, Lukáš a Jan.

 Z Třince do Orlové dánť jest směr a, i přes relativní nenáročnost trasy,
 vnoučata ustavičně fňukají, že chtějí jet tou nejkratší trasou. Děda Koláček,
 dobrodinec ten od kosti, snaží se vnoučatům vyhovět a nejkratší trasu úpěnlivě
 hledá. Do vřavy se přidává babička Koláčková, která ví, že trasa z Třince do
 Orlové vede přes mnoho malých vesnic, mnohože~z nich hostí aspoň jednu hospodu.

 Vědouc velmi dobře, že každá hospoda zbrzdí cestu na nejméně půl hodiny, trvá
 babička Koláčková na tom, aby se vybraná cesta z Třince do Orlové hospodám
 vyhnula. Křik vnoučat brzy ji však přesvědčí, že délka trasy převažuje nad
 množstvím hospod, které po cestě potkají.

 Rodina Koláčková tudíž stojí před nelehkým úkolem vybrat ze všech nejkratších
 cest z Třince do Orlové tu, která se vyhne co nejvíce hospodám.

 Brzy znaven, děda Koláček posílá nezkrotná vnoučata s konečně neosedlanými
 bicykly domů a výlet do Orlové se odkládá na příští školní prázdniny. Pomozme
 mu jej zorganizovat předem.
\end{problem}

Úlohu si modelujeme ohodnoceným grafem $G = (V,E,w)$. Vrcholy představovují
vesnice, případně města, mezi Třincem a Orlovou, z nichž některé jsou označeny
vykřičníkem mínícím přítomnost jedné či více hospod.

Nejprve se soustředíme na nejpodstatnější část úlohy, tou jest nalezení
nejkratší cest mezi Třincem a Orlovou, kterážto města si v zájmu strohosti
označíme písmeny $t,o \in V$.
