\subsection{Vzdálenost vrcholů}
\label{ssec:vzdalenost-vrcholu}

Sekci motivujeme úlohou \uv{jako ze života}, která ve mně budí jistou míru
nostalgie, jelikož jsem ji řešil na konci prvního ročníku v rámci zkoušky
z~programování. Dovolil jsem si ji ovšem literárně obohatit.

\begin{problem}[Rodinný výlet]
 \label{prob:rodinny-vylet}
 Je prodloužený víkend a slezská rodina Koláčků plánuje cyklistický výlet
 oblastí Karviná. Jedou děda Koláček se svou chotí, babičkou Koláčkovou, a
 jejich čtyři uřvaná rozmazlená vnoučata -- Matouš, Marek, Lukáš a Jan.

 Z Třince do Orlové dánť jest směr a, i přes relativní nenáročnost terénu,
 vnoučata ustavičně fňukají, že chtějí jet tou nejkratší trasou. Děda Koláček,
 dobrodinec ten od kosti, snaží se vnoučatům vyhovět a nejkratší trasu úpěnlivě
 hledá. Do vřavy se přidává babička Koláčková, která ví, že trasa z Třince do
 Orlové vede přes mnoho malých vesnic, mnohože~z nich hostí aspoň jednu hospodu.

 Vědouc velmi dobře, že každá hospoda zbrzdí cestu na nejméně půl hodiny a že i
 při mimoděčné zmínce o blízké hospodě si děda Koláček počíná mnouti pivní
 pupek, trvá babička Koláčková na tom, aby vybraná cesta z Třince do Orlové
 vedla přes vesnice, které jsou tak daleko od hospod, jak je to jen možné. Křik
 vnoučat brzy ji však přesvědčí, že délka trasy převažuje nad množstvím hospod,
 ke kterým se po cestě přiblíží.

 Koláčkovic rodina tudíž stojí před nelehkým úkolem vybrat ze všech nejkratších
 cest z Třince do Orlové tu, která je hospodám co nejvíce vzdálena.

 Brzy znaven, děda Koláček posílá nezkrotná vnoučata s konečně neosedlanými
 bicykly domů a výlet do Orlové se odkládá na příští školní prázdniny. Pomozme
 mu jej zorganizovat předem.
\end{problem}

Úlohu si modelujeme ohodnoceným grafem $G = (V,E,w)$. Vrcholy představují
vesnice, případně města, mezi Třincem a Orlovou, z nichž některé jsou označeny
vykřičníkem mínícím přítomnost jedné či více hospod.

\begin{figure}[h]
 \centering
 \begin{tikzpicture}[scale=1.5]
  \node[vertex,myred] (t) at (0, 0) {};
  \node[vertex,mypurple] (v11) at (1, 1) {};
  \node[vertex] (v12) at (1, -1) {};
  \node[vertex] (v21) at (2, 0.5) {};
  \node[vertex,mypurple] (v22) at (2, -0.5) {};
  \node[vertex] (v23) at (2, -2) {};
  \node[vertex] (v31) at (3, 1.5) {};
  \node[vertex] (v32) at (3, 0) {};
  \node[vertex,mypurple] (v33) at (3, -1) {};
  \node[vertex,mypurple] (v41) at (4, 2) {};
  \node[vertex] (v42) at (4, 1) {};
  \node[vertex] (v43) at (4, -0.5) {};
  \node[vertex] (v44) at (4, -1.5) {};
  \node[vertex,myblue] (o) at (5, 0) {};

  \draw[thick] (t) to node[midway,draw,circle,fill=white,inner sep=1pt]
   {\footnotesize $4$} (v11);
  \draw[thick] (t) to node[midway,circle,draw,fill=white,inner sep=1pt]
   {\footnotesize $3$} (v12);
  \draw[thick] (v11) to node[midway,circle,draw,fill=white,inner sep=1pt]
   {\footnotesize $2$} (v21);
  \draw[thick] (v11) to node[midway,circle,draw,fill=white,inner sep=1pt]
   {\footnotesize $7$} (v22);
  \draw[thick] (v12) to node[midway,circle,draw,fill=white,inner sep=1pt]
   {\footnotesize $3$} (v22);
  \draw[thick] (v12) to node[midway,circle,draw,fill=white,inner sep=1pt]
   {\footnotesize $6$} (v23);
  \draw[thick] (v23) to node[midway,circle,draw,fill=white,inner sep=1pt]
   {\footnotesize $11$} (v44);
  \draw[thick] (v11) to node[midway,circle,draw,fill=white,inner sep=1pt]
   {\footnotesize $12$} (v31);
  \draw[thick] (v21) to node[midway,circle,draw,fill=white,inner sep=1pt]
   {\footnotesize $5$} (v31);
  \draw[thick] (v21) to node[midway,circle,draw,fill=white,inner sep=1pt]
   {\footnotesize $3$} (v32);
  \draw[thick] (v22) to node[midway,circle,draw,fill=white,inner sep=1pt]
   {\footnotesize $4$} (v32);
  \draw[thick] (v23) to node[midway,circle,draw,fill=white,inner sep=1pt]
   {\footnotesize $5$} (v33);
  \draw[thick] (t) to node[midway,circle,draw,fill=white,inner sep=1pt]
   {\footnotesize $8$} (v22);
  \draw[thick] (v22) to node[midway,circle,draw,fill=white,inner sep=1pt]
   {\footnotesize $6$} (v23);
  \draw[thick] (v32) to node[midway,circle,draw,fill=white,inner sep=1pt]
   {\footnotesize $3$} (v33);
  \draw[thick] (v31) to node[midway,circle,draw,fill=white,inner sep=1pt]
   {\footnotesize $4$} (v41);
  \draw[thick] (v31) to node[midway,circle,draw,fill=white,inner sep=1pt]
   {\footnotesize $3$} (v42);
  \draw[thick] (v32) to node[midway,circle,draw,fill=white,inner sep=1pt]
   {\footnotesize $5$} (v42);
  \draw[thick] (v32) to node[midway,circle,draw,fill=white,inner sep=1pt]
   {\footnotesize $3$} (v43);
  \draw[thick] (v33) to node[midway,circle,draw,fill=white,inner sep=1pt]
   {\footnotesize $4$} (v43);
  \draw[thick] (v33) to node[midway,circle,draw,fill=white,inner sep=1pt]
   {\footnotesize $3$} (v44);
  \draw[thick] (v21) to node[midway,circle,draw,fill=white,inner sep=1pt]
   {\footnotesize $7$} (v42);
  \draw[thick] (v41) to node[midway,circle,draw,fill=white,inner sep=1pt]
   {\footnotesize $14$} (o);
  \draw[thick] (v42) to node[midway,circle,draw,fill=white,inner sep=1pt]
   {\footnotesize $8$} (o);
  \draw[thick] (v32) to node[midway,circle,draw,fill=white,inner sep=1pt]
   {\footnotesize $11$} (o);
  \draw[thick] (v43) to node[midway,circle,draw,fill=white,inner sep=1pt]
   {\footnotesize $5$} (o);
  \draw[thick] (v44) to node[midway,circle,draw,fill=white,inner sep=1pt]
   {\footnotesize $9$} (o);

  \node[below left=0mm of t,myred] (tlabel) {$t$};
  \node[below right=0mm of o,myblue] (olabel) {$o$};
 
  \node[above left=-1mm and -1mm of v11,mypurple] {$\mathbf{!}$};
  \node[above right=-1mm and -2mm of v22,mypurple] {$\mathbf{!}$};
  \node[above left=-3mm and -1mm of v33,mypurple] {$\mathbf{!}$};
  \node[above right=-1mm and -1mm of v41,mypurple] {$\mathbf{!}$};
  
 \end{tikzpicture}
 \caption{Graf obcí mezi \clr{Třincem} a \clb{Orlovou}.}
 \label{fig:trinec-orlova}
\end{figure}

Nejprve se soustředíme na nejpodstatnější část úlohy, tou jest nalezení
nejkratší cest mezi Třincem a Orlovou, kterážto města si v zájmu strohosti
označíme písmeny $t,o \in V$.

První, co by člověka mohlo napadnout je spustit
\hyperref[alg:floyd-warshall]{Floydův-Warshallův algoritmus}. Zde je však nutno
si uvědomit, že ten není možné modifikovat tak, aby hledal vzdálenost mezi dvěma
konkrétními vrcholy. Totiž, z~popisu jeho průběhu plyne, že váhu nejkratší cesty
mezi dvěma vrcholy, která vede pouze přes vrcholy z množiny
$\{v_1,\ldots,v_k\}$, umím spočítat teprve tehdy, když znám váhy nejkratších
cest používajících pouze vrcholy $v_1,\ldots,v_{k-1}$ mezi \textbf{všemi
dvojicemi vrcholů} v grafu $G = (\{v_1,\ldots,v_n\}, E, w)$. Odtud plyne, že
Floydův-Warshallův algoritmus vždy najde vzdálenosti mezi všemi dvojicemi
vrcholů.

Samozřejmě je též možné najít \textbf{všechny} cesty mezi danými dvěma vrcholy a
z nich vybrat tu nejkratší. Tento postup je však ještě výrazně méně efektivní
než Floydův-Warshallův algoritmus.

Není však radno zoufat, bo algoritmus pro relativně efektivní nalezení
vzdálenosti mezi určenými dvěma vrcholy v grafu existuje. Nosí jméno svého
tvůrce, holandského informatika a programátora Edsgera W. Dijk\-stry (čteno
\uv{dajkstry}). Není složitý, ale zcela jistě je méně přímočarý než
\hyperref[alg:kruskal]{Kruskalův} i
\hyperref[alg:floyd-warshall]{Floydův-Warshallův} algoritmus.

Je ironické, že najít efektivní algoritmus na výpočet vzdálenosti mezi dvě\-ma
vrcholy je výrazně obtížnější než na výpočet vzdálenosti mezi všemi dvojicemi
vrcholů.

\subsubsection{Dijkstrův algoritmus}
\label{sssec:dijkstruv-algoritmus}

Dijkstrův algoritmus má mnoho společného s
\hyperref[alg:floyd-warshall]{Floydovým-Warshallovým algoritmem}, například
rovněž v každém kroku zlepšuje zatím nejlepší známou vzdálenost od
\emph{zdrojového} vrcholu k ostatním. Jeho základní myšlenka je však jiná.
Dijkstrův algoritmus využívá tzv. \uv{průchod do šířky}, což znamená, že se
\uv{šíří} od jednoho zvoleného \emph{zdrojového vrcholu} nejprve do jeho sousedů
(tj. do vrcholů s ním spojených hranou). Při průchodu si u každého vrcholu
pamatuje zatím nejlepší nalezenou cestu ze zdrojového vrcholu do něj, a kdykoli
vstoupí do dalšího vrcholu, ověří, zda cesta, po které se do něj dostal, není
kratší než dosavad nejlepší.

Tento postup sám není nijak revoluční a průchod do šířky je asi tak starý jako
grafy samotné. Myšlenka, se kterou Edsger Dijkstra přišel, však na svou dobu
revoluční byla, neboť se jednalo o tehdy nejefektivnější způsob nalezení
vzdálenosti mezi vrcholy v ohodnoceném grafu. U každého vrcholu si pamatoval
váhu zatím nejkratší nalezené cesty od zdroje k němu. Uvědomil si, že
\textbf{pokud vybírá vrcholy vždy od těch s minimální uloženou hodnotou}, tak v
moment, \textbf{kdy projde všechny sousedy nějakého vrcholu}, pak už \textbf{je
zapamatovaná hodnota v tomto vrcholu nejlepší možná}, tedy se jedná o skutečnou
vzdálenost od zdroje k tomuto vrcholu a už se do něj nikdy nemusí znovu vracet.

Důvod, proč tomu tak je, není na první pohled zřejmý a objasníme ho podrobně v
důkazu správnosti Dijkstrova algoritmu.

Ještě před slovním popisem a představením pseudokódu je dobré se zamyslet, o
kolik více efektivní je Dijkstrův způsob oproti prostému průchodu do šířky. Pro
úplnost:
\begin{itemize}
 \item Průchod do šířky vždy putuje od vrcholu ke všem jeho sousedům, od těch
  zase k jejich sousedům (mezi nimiž je samozřejmě i onen původní vrchol) a tak
  dále.
 \item Dijkstrův algoritmus nahlédne z právě vybraného vrcholu do jeho sousedů,
  ale pak nepokračuje nutně jimi. Naopak, právě vybraný vrchol již \uv{uzamkne}
  a pokračuje nějakým s minimální uloženou zatím nejkratší cestou ze zdroje do
  něj.
\end{itemize}
Kvůli tomu, že jeden ze sousedů každého vrcholu je přesně ten samý vrchol,
projdu v naprosto nejhorším případě při běžném průchodu do šířky každý vrchol
tolikrát, kolik přes něj vede cest.

Naopak, Dijkstrův algoritmus \emph{zakazuje} vracet se do vrcholů, jejichž
všichni sousedi již byli navštíveni. Tedy, každý vrchol je navštíven v nejhorším
tolikrát, kolik má sousedů. Je snadné si rozmyslet, o kolik je v průměrném
případě počet sousedů vrcholu nižší než počet cest jím procházejících.

Největší časový rozdíl mezi obvyklým průchodem do šířky a Dijkstrovým algoritmem
je vidět v tzv. \emph{úplných grafech}, tedy grafech, ve kterých je každý vrchol
spojen s každým. V úplném grafu na $n$ vrcholech, který se často značí $K_n$,
definuje libovolná podmnožina vrcholů cestu. Již dlouho víme, že všech podmnožin
množiny o $n$ prvcích je $2^{n}$. Tedy, algoritmus průchodu do šířky udělá při
hledání vzdálenosti mezi dvěma vrcholy v úplném grafu vždy $2^{n}$ kroků. Oproti
tomu, není těžké dokázat (ale my to tu dělat nebudeme), že Dijkstrův algoritmus
udělá v grafu na $n$ vrcholech vždy nejvýše $n^2$ kroků.

Na \myref{obrázku}{fig:uplny-graf} vidíte úplný graf na $5$ vrcholech. Upřímně
doporučuji, abyste si tvrzení předchozího odstavce vyzkoušeli v praxi.

\begin{figure}[h]
 \centering
 \begin{tikzpicture}
  \node[vertex,myred] (1) at (90:2) {};
  \node[vertex] (2) at (162:2) {};
  \node[vertex] (3) at (234:2) {};
  \node[vertex,myblue] (4) at (306:2) {};
  \node[vertex] (5) at (18:2) {};
  
  \foreach \i in {1,2,...,5} {
   \foreach \j in {1,2,...,5} {
    \ifthenelse{\equal{\i}{\j}}{}{
     \tikzmath{
      integer \w;
      \w = \i + \j;
     }
     \draw[thick] (\i) to node[midway,circle,draw,fill=white,inner sep=1pt]
      {\footnotesize $\w$} (\j);
    }
   }
  }
 \end{tikzpicture}

 \caption{Ohodnocený úplný graf na pěti vrcholech s vyznačeným \clr{zdrojovým} a
 \clb{cílovým} vrcholem.}
 \label{fig:uplny-graf}
\end{figure}

Nyní přistoupíme ke slovnímu popisu Dijkstrova algoritmu. Algoritmus striktně
vzato nepočítá vzdálenost mezi dvěma vrcholy, anobrž vzdálenost od
\emph{zdrojového vrcholu} ke všem ostatním.

Podobně jako \hyperref[alg:floyd-warshall]{Floydův-Warshallův algoritmus}, i
Dijkstrův algoritmus vytváří postupně zobrazení $d:V \to [0,\infty]$ takové, že
po jeho skončení platí $d(v) = d_G(s,v)$ pro všechna $v \in V$ a nějaký pevně
zvolený zdrojový vrchol $s \in V$. Na začátku algoritmu je pročež $d(s) = 0$ a
$d(v) = \infty$ pro všechny $s \neq v \in V$. V každém kroku algoritmu navíc
existuje množina $X$ \uv{zakázaných} vrcholů, do kterých už není možné se dívat
ani vracet. V moment, kdy procházím sousedy právě vybraného vrcholu, ignoruji
ty, které již byly někdy zakázány.

Jeden krok algoritmu vypadá takto:
\begin{enumerate}
 \item Označ jako zvolený vrchol libovolný $v \in V$ \textbf{s minimální
  $d(v)$}.
 \item Pro každého souseda $u$ zvoleného vrcholu $v$ porovnej zatím nejkratší
  známou cestu z $s$ do $u$, to jest $d(u)$, s váhou zatím nejkratší cesty z $s$
  do $u$ vedoucí přes $v$, to jest $d(v) + w(uv)$. Je-li druhá hodnota menší,
  změň $d(u)$ na $d(v) + w(uv)$.
 \item Vrchol $v$ označ jako zakázaný.
 \item Pokud ještě existuje vrchol, který není zakázaný, opakuj (1).
\end{enumerate}

Jak jsme již psali -- fakt, že jako další vrchol v pořadí volíme ten s minimální
známou nejkratší cestou z $s$, umožňuje se nikdy do vrcholů, jejichž sousedy
projdeme, nevracet, není samozřejmý. Výklad o obecném chodu Dijkstrova algoritmu
zakončíme prezentací jeho \hyperref[alg:dijkstra]{pseudokódu} pro
\textbf{souvislý} ohodnocený graf $G = (V,E,w)$ s pevně zvoleným zdrojovým
vrcholem $s \in V$. Ihned poté dokážeme jeho správnost.

Pro stručnost vyjádření zavedeme ještě značení
\[
 n(v) \coloneqq \{u \in V \mid uv \in E\},
\]
čili označíme výrazem $n(v)$ (od angl. \emph{\textbf{n}eighbour}), množinu všech
sousedů vrcholu $v \in V$.

\begin{algorithm}
 \caption{Dijkstrův algoritmus}
 \label{alg:dijkstra}

 \SetKwInOut{Input}{input}
 \SetKwInOut{Output}{output}
 \SetKw{KwReturn}{return}

 \Input{souvislý ohodnocený graf $G = (V,E,w)$ s počátečním vrcholem $s \in V$}
 \Output{zobrazení $d:V \to [0,\infty]$ takové, že $d \equiv d_G(s,-)$}
 \BlankLine
 \emph{Inicializace}\;
 $d(s) \leftarrow 0$\;
 \For{$v \in V \setminus \{s\}$} {
  $d(v) \leftarrow \infty$\;
 }
 $X \leftarrow \emptyset$\;
 \BlankLine
 \While{$X \neq V$} {
  $v \leftarrow $ libovolný vrchol s minimální $d(v)$ takový, že $v \notin X$\;
  \For{$u \in n(v) \setminus X$} {
   \If{$d(u) > d(v) + w(uv)$} {
    $d(u) \leftarrow d(v) + w(uv)$\;
   }
  }
  $X \leftarrow X \cup \{v\}$\;
 }
 \KwReturn{$d$}\;
\end{algorithm}

\pagebreak

\begin{claim}
 \label{claim:dijsktra-korektni}
 Dijkstrův algoritmus je korektní.
\end{claim}
\begin{proof}
 Musíme ověřit, že Dijkstrův algoritmus je konečný a počítá správně.

 Konečnost je zřejmá. Vrchol $v \in V$ s minimální $d(v)$ vždy existuje (i kdyby
 ta hodnota měla být $\infty$), a tedy nastane chvíle, kdy v $X$ jsou všechny
 vrcholy grafu $G$. V ten moment $X = V$, cyklus končí a algoritmus vrací
 zobrazení $d$.
 
 Korektnost ukážeme indukcí podle počtu zakázaných vrcholů, tedy podle $\# X$.
 Naším indukčním předpokladem bude, jako tomu obvykle v důkazech indukcí bývá,
 že algoritmus funguje. To přesně znamená, že v každém kroku
 \begin{enumerate}[label=(\alph*)]
  \item je $d(v) = d_G(s,v)$ pro všechny vrcholy $v \in X$;
  \item je $d(u)$ váha nejkratší cesty z $s$ do $u$ vedoucí pouze přes vrcholy v
   $X$, pro všechna $u \in V \setminus X$.
 \end{enumerate}
 Předpoklad (a) říká v lidské mluvě, že zakazování vrcholů si opravdu můžeme
 dovolit, tedy že do zakázaných vrcholů již nejde nalézt kratší cestu.

 Význam předpokladu (b) je těžší dekódovat. Představme si spíše, co by se stalo,
 kdyby \textbf{neplatil}. Pak by existovala nějaká kratší cesta do nezakázaného
 vrcholu $u$ přes zakázané vrcholy. Ovšem, přesně pro to, že tyto vrcholy jsou
 již zakázané, nemohu během zbytku průběhu algoritmu již nikdy tuto cestu
 objevit. Pokud by se stalo, že zrovna tato cesta je tou nejkratší, pak již
 zároveň nemohu nalézt hodnotu $d_G(s,u)$ a algoritmus selže.

 Počáteční případ $\# X = 1$ je snadný. V tomto stádiu jsme zrovna zakázali
 zdrojový vrchol $s$ a pro každého jeho souseda $u$ jsme položili $d(u) =
 w(su)$, jelikož $d(s) = 0$ a $d(u)$ bývala hodnota $\infty$. Tedy, zakázaný
 vrchol je jedině $s$ a $d(s) = d_G(s,s) = 0$, čímž jsme ověřili platnost (a).
 Jediné vrcholy, do nichž vede cesta přes zakázané vrcholy, tedy přes vrchol
 $s$, jsou sousedé $s$, a váha nejkratší cesty je zřejmě $w(su)$ pro každého
 souseda $u \in n(s)$. To dokazuje (b).

 Nyní rozebereme případ pro obecnou $\# X$ s předpokladem platnosti (a) a (b)
 pro $\# X - 1$. Nejprve si uvědomíme, že v tomto případě je výrok (b) platný
 automaticky.
\end{proof}
