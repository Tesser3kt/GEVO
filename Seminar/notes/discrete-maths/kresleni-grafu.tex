\subsection{Kreslení grafů}
\label{ssec:kresleni-grafu}

Čtenáře může překvapit, že \emph{kreslení} grafů je matematicky formální postup.
Na druhou stranu to však divné není, neboť mnoho aplikací grafů plyne právě z
přirozeného vnímání grafu jako množiny bodů v rovině spojených úsečkami.

Naší zábavou v této závěrečné sekci bude prezentovat onen postup a po té se
rozhovoříme o tzv. \emph{rovinných} grafech, grafech, jež lze nakreslit, aniž se
křivky představující úsečky v rovině kříží.

Více prakticky orientované čtenáře, kterým, stejně jako i ostatním vedlejším
vrstvám akademické komunity, je tento text samozřejmě též určen, by snad
zajímalo, k čemu je kreslení grafů dobré.

Jedním konkrétním příkladem ze stavby počítačů je návrh logických obvodů v
procesorech. Je v zájmu výrobců procesorů snížit počet křížení logických obvodů
na naprosté minimum, neboť každé křížení znamená nutnost zvýšit procesor o další
vrstvu zlata a silikonu, což zhoršuje rychlost přenosu a zvedá cenu výroby.

Více matematické aplikace pak zahrnují mimo mnohé další například (stá\-le
nevyřešený!)
\href{https://en.wikipedia.org/wiki/Tur%C3%A1n%27s_brick_factory_problem}{Brick
Factory Problem} nebo též
\href{https://www.sciencedirect.com/science/article/abs/pii/S0252960212600472}{různé
úlohy v teorii uzlů}.

První netriviální výzvou je dojít k rozumné definici kreslení grafu. Potřebujeme
nějakým způsobem přenést množinu vrcholů $V$ grafu $G$ na body v rovině a hrany
na křivky spojující tyto body. Záměrně jsme použili slovo \uv{křivka} místo
\uv{úsečka}, neboť není těžké si rozmyslet (a my to později též učiníme), že
mnoho grafů lze nakreslit bez křížení hran, pokud tyto kreslíme jako křivky či
oblouky, ale nikoli kreslíme-li je jako úsečky.


