\documentclass[a4paper,11pt,dvipsnames,twoside]{book}

% Language %
\usepackage[czech,english]{babel}

% Encoding %
\usepackage[utf8]{inputenc}
\usepackage[T1]{fontenc}

% Layout %
\usepackage[margin=1.5in]{geometry}

% Header & Footer TODO %
\renewcommand{\chaptermark}[1]{\markboth{#1}{}}
\renewcommand{\sectionmark}[1]{\markright{\thesection\ #1}}
\usepackage{fancyhdr}
\pagestyle{fancy}
\fancyhf{}
\fancyhead[LE,RO]{\thepage}
\fancyhead[RE]{\textit{\nouppercase{\leftmark}}}
\fancyhead[LO]{\textit{\nouppercase{\rightmark}}}
\fancypagestyle{plain}{%
 \fancyhf{}
 \renewcommand{\headrulewidth}{0pt}
}
\setlength{\headheight}{16pt}

% AMS LaTeX %
\usepackage{amsmath,amsthm,mathtools}
\usepackage{dsfont}

% Fonts %
\usepackage[libertine,theoremfont]{newtx}

% Make Titles sans-serif %
\usepackage{sectsty}
\allsectionsfont{\sffamily}

% Graphics %
\usepackage[dvipsnames]{xcolor}
\usepackage{graphicx}
\usepackage{tikz}
\usetikzlibrary{arrows,automata,positioning}

% Tables %
\usepackage{booktabs}
\usepackage{multirow}

% Hyperlinks %
\usepackage{hyperref}

% Figures %
\usepackage{caption}
\usepackage{subcaption}

% Itemize and Enumerate %
\usepackage{enumitem}
\setlist{topsep=0pt}
\setlist[enumerate,1]{label=(\arabic*).}
\setlist[enumerate,2]{label=(\alph*)}

% Patch for hyphens and amsthm/mdframed %
\usepackage{regexpatch}

\makeatletter
% Change the `-` delimiter to an active character
\xpatchparametertext\@@@cmidrule{-}{\cA-}{}{}
\xpatchparametertext\@cline{-}{\cA-}{}{}
\makeatother

\makeatletter
\AtBeginDocument{%
\@ifpackageloaded{amsthm}%
 {%
  \renewrobustcmd\mdf@patchamsthm{%
   \chardef\kludge@catcode@hyphen=\catcode`\-
   \catcode`\-=12
   \let\mdf@deferred@thm@head\deferred@thm@head
   \pretocmd{\deferred@thm@head}{\@inlabelfalse}%
      {\mdf@PackageInfo{mdframed detected package amsthm ^^J%
                        changed the theorem header of amsthm\MessageBreak}%
      }{%
       \mdf@PackageError{mdframed detected package amsthm ^^J%
                         changed the theorem header of amsthm
                         failed\MessageBreak}%
       }%
   \catcode`\-=\kludge@catcode@hyphen
     }%
 }{}%
}
\makeatother

% Math Operators %
\DeclareMathOperator{\dom}{dom}
\DeclareMathOperator{\img}{im}
\DeclareMathOperator{\id}{\mathds{1}}

% Number Sets %
\newcommand{\N}{\mathbb{N}}
\newcommand{\Z}{\mathbb{Z}}
\newcommand{\Q}{\mathbb{Q}}
\newcommand{\R}{\mathbb{R}}
\newcommand{\C}{\mathbb{C}}
 
% Colors %
\newcommand{\clr}{\textcolor{BrickRed}}
\newcommand{\clb}{\textcolor{Royalblue}}
\newcommand{\clg}{\textcolor{ForestGreen}}
\newcommand{\clm}{\textcolor{Fuchsia}}
\newcommand{\clv}{\textcolor{violet}}
\newcommand{\clbr}{\textcolor{Sepia}}
\newcommand{\cly}{\textcolor{Dandelion}}

% American Paragraph Skip %
\setlength{\parindent}{0pt}
\setlength{\parskip}{1em}

% Theorem definitions %
\usepackage[xcolor,framemethod=tikz,ntheorem]{mdframed}

%% General Theorem Style %%
\mdfdefinestyle{mythmstyle}{%
 topline=false,
 leftline=false,
 rightline=false,
 bottomline=true,
 frametitlefont=\sffamily\bfseries,
 frametitlerule=false,
 linewidth=1pt,
 frametitleaboveskip=6pt,
 theoremtitlefont=\normalfont\sffamily,
 font=\itshape,
 theoremseparator={ ---},
 innertopmargin=9pt,
 innerbottommargin=8pt
}
\mdfdefinestyle{mydefstyle}{%
 topline=false,
 leftline=false,
 rightline=false,
 bottomline=true,
 frametitlefont=\sffamily\bfseries,
 frametitlerule=false,
 linewidth=1pt,
 frametitleaboveskip=6pt,
 theoremtitlefont=\normalfont\sffamily,
 theoremseparator={ ---},
 innertopmargin=9pt,
 innerbottommargin=8pt
}
\mdfdefinestyle{myprfstyle}{%
 topline=false,
 leftline=true,
 rightline=false,
 bottomline=true,
 frametitlefont=\bfseries,
 frametitlerule=false,
 linewidth=1pt,
 frametitleaboveskip=0pt,
 frametitlebelowskip=0pt,
 skipabove=0pt,
 theoremtitlefont=\normalfont\sffamily,
 theoremseparator={},
 innertopmargin=9pt,
 innerbottommargin=8pt
}

% Theorems %
\mdtheorem[%
 style=mythmstyle,
 linecolor=Orange,
 frametitlebackgroundcolor=Orange!70,
 backgroundcolor=Orange!10
]{theorem}{Věta}[section]
\mdtheorem[%
 style=mythmstyle,
 linecolor=RoyalBlue,
 frametitlebackgroundcolor=RoyalBlue!50,
 backgroundcolor=RoyalBlue!10
]{proposition}[theorem]{Tvrzení}
\mdtheorem[%
 style=mythmstyle,
 linecolor=ForestGreen,
 frametitlebackgroundcolor=ForestGreen!60,
 backgroundcolor=ForestGreen!10
]{lemma}[theorem]{Lemma}
\mdtheorem[%
 style=mythmstyle,
 linecolor=YellowGreen,
 frametitlebackgroundcolor=YellowGreen!70,
 backgroundcolor=YellowGreen!10
]{observation}[theorem]{Pozorování}
\mdtheorem[%
 style=mythmstyle,
 linecolor=RoyalPurple,
 frametitlebackgroundcolor=RoyalPurple!50,
 backgroundcolor=RoyalPurple!10
]{corollary}[theorem]{Důsledek}

% Definition-like stuff %
\mdtheorem[%
 style=mydefstyle,
 linecolor=Aquamarine,
 frametitlebackgroundcolor=Aquamarine!50,
 backgroundcolor=Aquamarine!10
]{definition}[theorem]{Definice}
\mdtheorem[%
 style=mydefstyle,
 linecolor=VioletRed,
 frametitlebackgroundcolor=VioletRed!50,
 backgroundcolor=VioletRed!10
]{example}[theorem]{Příklad}
\mdtheorem[%
 style=mydefstyle,
 linecolor=Mahogany,
 frametitlebackgroundcolor=Mahogany!40,
 backgroundcolor=Mahogany!10
]{remark}[theorem]{Poznámka}
\mdtheorem[%
 style=mydefstyle,
 linecolor=Red,
 frametitlebackgroundcolor=Red!80,
 backgroundcolor=Red!10
]{warning}[theorem]{Výstraha}
\mdtheorem[%
 style=mydefstyle,
 linecolor=Gray,
 frametitlebackgroundcolor=Gray!50,
 backgroundcolor=Gray!10
]{exercise}[theorem]{Cvičení}

% Proofs, each for different color %
% New qed with hfill %
\renewcommand{\qed}{\hfill\ensuremath{\blacksquare}}
\mdtheorem[%
 style=myprfstyle,
 linecolor=Orange
]{thmproof}{Důkaz}[section]
\mdtheorem[%
 style=myprfstyle,
 linecolor=RoyalBlue
]{propproof}{Důkaz}[section]
\mdtheorem[%
 style=myprfstyle,
 linecolor=ForestGreen
]{lemmaproof}{Důkaz}[section]
\mdtheorem[%
 style=myprfstyle,
 linecolor=YellowGreen
]{corproof}{Důkaz}[section]
\mdtheorem[%
 style=myprfstyle,
 linecolor=Aquamarine
]{defproof}{Důkaz}[section]
