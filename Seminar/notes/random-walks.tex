\documentclass[a4paper,11pt]{article}

\usepackage[czech]{babel}
% Colors %
\usepackage[dvipsnames]{xcolor}

% Page Layout %
\usepackage[margin=1.5in]{geometry}

% Fancy Headers %
\usepackage{fancyhdr}
\fancyhf{}
\cfoot{\thepage}
\rhead{}
\renewcommand{\headrulewidth}{0pt}
\setlength{\headheight}{16pt}

% Math
\usepackage{mathtools}
\usepackage{amssymb}
\usepackage{faktor}
\usepackage{import}
\usepackage{caption}
\usepackage{subcaption}
\usepackage{wrapfig}
\usepackage{booktabs}

% Theorems
\usepackage{amsthm}
\usepackage{thmtools}

\theoremstyle{remark}
\newtheorem*{example}{Příklad}

% Title %
\title{\Huge\textsf{Náhodné cesty}\\
 \Large\textsf{ve 2D a ve 3D}
 \author{}
 \date{}
}

% Table of Contents %
\usepackage{hyperref}
\hypersetup{
 colorlinks=true,
 linktoc=all,
 linkcolor=blue
}

% Tables %
\usepackage{booktabs}

% Enumerate %
\usepackage{enumerate}

% Operators %
\DeclareMathOperator{\Ker}{Ker}
\DeclareMathOperator{\Img}{Im}
\DeclareMathOperator{\End}{End}
\DeclareMathOperator{\Aut}{Aut}
\DeclareMathOperator{\Inn}{Inn}

% Common operators %
\newcommand{\R}{\mathbb{R}}
\newcommand{\N}{\mathbb{N}}
\newcommand{\Z}{\mathbb{Z}}
\newcommand{\Q}{\mathbb{Q}}
\newcommand{\C}{\mathbb{C}}
\renewcommand{\P}{\mathbf{P}}
\newcommand{\E}{\mathbf{E}}

% American Paragraph Skip %
\setlength{\parindent}{0pt}
\setlength{\parskip}{1em}

% Document %
\pagestyle{fancy}
\begin{document}

\maketitle
\thispagestyle{fancy}

Traduje se, že japonský matematik Kakutani Shizuo řekl v 70. letech,

\begin{center}
 \emph{\uv{Opilý muž se vždy vrátí domů, ale opilý pták může bloudit navždy.}}
\end{center}

Za tímhle rčením je schována pravděpodobnostní úloha. Řekněme, že opilý muž
vyjde z domu a, bo je opilý, chodí náhodně. Slovo `náhodně' tu chápeme v tom
smyslu, že po každém kroku, který je pro jednoduchost vždy stejně dlouhý (i když
to na výsledku úlohy nic nemění), pokračuje s pravděpodobností 1/4, nebo 25 \%,
v jednom ze 4 směrů. Opilý pták, bo létá, volí po každém kroku z 6 různých
směrů.

Tvrdíme, že opilý muž se po nějakém počtu učiněných kroků vždycky vrátí zpět do
svého domu, ale ptákovi se může stát, že už cestu do hnízda zpátky nikdy
nenajde.

Myslím si, že tohle je jedna z úloh, kdy naše přirozená lidská intuice selhává.
Je to pravděpodobně proto, že nám lidem obecně schází intuice pro `nekonečné'
věci. Cílem tohoto (snad krátkého) textu je spočítat, že pronesené tvrzení je
pravdivé.

Trocha formalismu. Budeme předpokládat, že opilá zvířata (ano, toto slovo
zahrnuje i druh Člověka moudrého) se pohybují v reálném prostoru (buď 2D nebo
3D) a jedním \emph{krokem} je vždy posun přesně o $1$ ve směru souřadnicových
os. \emph{Cestou} délky $k$ ve 2D budeme myslet \textbf{konečnou} posloupnost
přesně $k$ učiněných kroků. Cesta je pak \emph{náhodná}, pokud má náhodnou délku
a směr každého dalšího kroku je volen náhodně (ve smyslu prvního odstavce).

\begin{figure}[t]
 \centering
 \def\svgwidth{.5\textwidth}
 \import{figs/}{path-in-2D.pdf_tex}
 \caption{Kus \textcolor{green}{náhodné cesty} ve 2D se začátkem v
 \textcolor{red}{počátku}.}
\end{figure}

Teď můžeme úlohu přeformulovat tak, že pravděpodobnost toho, že se opilý muž
vrátí, když chodí náhodně, je 1, nebo 100 \%, a pravděpodobnost toho, že se
opilý pták vrátí, když bude létat náhodně, je \textbf{ostře} menší než 100 \%.

Pravděpodobnost toho, že se stane nějaká skutečnost/jev $X$, budu psát jako
$\P(X)$. Symbolicky, pravděpodobnost návratu můžu psát třeba jako
$\P(\text{návrat})$. \textcolor{red}{Pamatujte, že pravděpodobnost návratu
\textbf{nezávisí} na volbě cesty!} Naopak, prav\-děpodobnost návratu vyjadřuje,
jakou šanci mám, že se vrátím, když zvolím nějakou náhodnou cestu.

Tuhle pravděpodobnost potřebujeme nějak hezky vyjádřit.

V tom nám brání následující pozorování. Hodnota $\P(\text{návrat})$ nerozlišuje
mezi cestami, podle toho, kolikrát se vrátí. Cesta, která projde počátkem 50krát
je v~tomto smyslu stejná jako ta, která se vrátí jednou. Protože cesta, která se
vrátí dvakrát, lze rozdělit do dvou cest, každáže z nich se vrátí jednou, budeme
chtít úlohu převést na počítání s cestami, které se vrací přesně jednou.

Jako první krok k tomu si definujme náhodnou veličinu (to je pravděpodobnostní
termín, stačí si to představovat jako něco, co má při každém `experimentu'
náhodnou hodnotu) $V$ jako `počet návratů'. Tedy například $\P(V=4)$ představuje
pravděpodobnost, že se vrátím do počátku přesně 4krát.

Jako druhý rok uvážíme tzv. `střední hodnotu' $V$, což je velmi nevhodný český
název pro údaj, kolikrát očekáváme, že se vrátíme. Pro intuitivní představu
slouží následující příklad.

\vspace{\parskip}
\hrule
\begin{example}
 Řekněme, že házím kostkou a výsledek hodu sleduje náhodná veličina $H$. Pak
 samozřejmě $\P(H = k) = 1 / 6$ pro každé číslo $k$ od 1 do 6. Je rozumné
 očekávat, že hodím 1 s pravděpodobností 1/6, 2 s pravděpodobností 1/6 atd.
 Takže moje střední hodnota bodu bude
 \begin{align*}
  \E[H] &= 1 \cdot \frac{1}{6} + 2 \cdot \frac{1}{6} + 3 \cdot \frac{1}{6} + 4
  \cdot \frac{1}{6} + 5 \cdot \frac{1}{6} + 6 \cdot \frac{1}{6}\\
  &= 1 \cdot \P(H = 1) + 2 \cdot \P(H = 2) + 3 \cdot \P(H = 3)\\
  &+4 \cdot \P(H = 4) + 5 \cdot \P(H = 5) + 6 \cdot \P(H = 6).
 \end{align*}
 Když to spočtete, vyjde vám $\E[H] = 3.5$. Všimněte si, že to je přesně
 aritmetický průměr čísel 1 až 6. To není náhoda. Když je pravděpodobnost každé
 možnosti stejná, je střední hodnota vždycky aritmetický průměr.

 Intuitivně, střední hodnota je číslo, ke kterému se náhodná veličina `nejvíce
 kloní'.

 Znetvořme herní kostku tak, aby pravděpodobnost toho, že hodím 2 byla 0.4, že
 hodím 3 taky 0.4, a u všech ostatních čísel 0.05. Tvrdím, že teď bude střední
 hodnota hodu někde mezi 2 a 3, ale blíž 3, protože 2 a 3 jsou
 nejpravděpodobnější, ale 3 má napravo od sebe ještě tři další hodnoty, zatímco
 2 má nalevo od sebe jen jednu. Schválně, máme
 \begin{align*}
  \E[H] &= 1 \cdot \P(H = 1) + 2 \cdot \P(H = 2) + 3 \cdot \P(H = 3)\\
  &+4 \cdot \P(H = 4) + 5 \cdot \P(H = 5) + 6 \cdot \P(H = 6)\\
  &=1 \cdot 0.05 + 2 \cdot 0.4 + 3 \cdot 0.4 + 4 \cdot 0.05 + 5 \cdot 0.05 + 6
  \cdot 0.05 = 2.8.
 \end{align*}
\end{example}
\hrule
\vspace{\parskip}
Zpět k náhodným cestám. Z příkladu s kostkou je doufám aspoň částečně jasné, že
očekávám, že se vrátím přesně $k$-krát s pravděpodobností toho, že počet
návratů je $k$ pro každé přirozené číslo $k \in \N$. To znamená, že
\[
 \E[V] = 1 \cdot \P(V = 1) + 2 \cdot \P(V = 2) + 3 \cdot \P(V = 3) + \ldots,
\]
kdy sčítám nekonečně mnoho členů, protože jsem nedal žádné omezení na délku
cesty ani počet návratů. Pomocí symbolu $\Sigma$ to můžu taky napsat jako
\[
 \E[V] = \sum_{k=1}^{\infty} k \cdot \P(V=k),
\]
což přesně znamená, že spolu sčítám všechny výrazy $k \cdot \P(V=k)$, kde $k$
(což je počet návratů) postupně probíhá všechna přirozená čísla.

Jak z tohohle součtu vykutat pravděpodobnost návratu $\P(\text{návrat})$, která
nás reálně zajímá. Ten součet nahoře si můžeme napsat trochu jinak. Konkrétně,
protože sčítanec $\P(V=k)$ se objevuje přesně $k$-krát, rozdělím si součet na

\begin{center}
 \begin{tabular}{cccccccc}
  $\P(V=1)$ & $+$ & & & & & & \\[1ex]
  $\P(V=2)$ & $+$ & $\P(V=2)$ & $+$ & & & & \\[1ex]
  $\P(V=3)$ & $+$ & $\P(V=3)$ & $+$ & $\P(V=3)$ & $+$ & & \\[1ex]
  $\P(V=4)$ & $+$ & $\P(V=4)$ & $+$ & $\P(V=4)$ & $+$ & $\P(V=4)$ & \\[1ex]
  $\vdots$ & & $\vdots$ & & $\vdots$ & & $\vdots$ & $\ddots$ \\
 \end{tabular}
\end{center}

Všimněte si, že v prvním sloupci mám součet
\[
 \P(V=1) + \P(V=2) + \P(V=3) + \ldots,
\]
neboli pravděpodobnost, že se vrátím jednou nebo se vrátím dvakrát nebo se
vrátím třikrát atd. To je ale přesně pravděpodobnost toho, že se vrátím
\textbf{aspoň jednou}. Neboli, ten součet v prvním sloupci je \textbf{přesně}
$\P(\text{návrat})$, protože to, že se vrátím, znamená, že se vrátím aspoň
jednou.

Hodnotu $\P(\text{návrat})$ si označím pro strohost zápisu jako $r$.

Jak souvisí součet v druhém sloupci s $r$? No, úplně stejnou úvahou dojdeme k
tomu, že součet v druhém sloupci mi dává pravděpodobnost, že se vrátím
\textbf{aspoň dvakrát}.


\end{document}
