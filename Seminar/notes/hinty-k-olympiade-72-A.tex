\documentclass[a4paper,11pt]{article}

% Lang %
\usepackage[czech]{babel}

% Colors %
\usepackage[dvipsnames]{xcolor}

% Page Layout %
\usepackage[margin=1.5in]{geometry}

% Fancy Headers %
\usepackage{fancyhdr}
\fancyhf{}
\cfoot{}
\rhead{}
\renewcommand{\headrulewidth}{0pt}
\setlength{\headheight}{16pt}

% Math
\usepackage{mathtools}
\usepackage{amssymb}
\usepackage{faktor}
\usepackage{import}
\usepackage{caption}
\usepackage{subcaption}
\usepackage{wrapfig}

% Theorems
\usepackage{amsthm}
\usepackage{thmtools}

% Title %
\title{\Huge\textsf{}\\
 \Large\textsf{}
 \author{}
 \date{}
}

% Table of Contents %
\usepackage{hyperref}
\hypersetup{
 colorlinks=true,
 linktoc=all,
 linkcolor=blue
}

% Tables %
\usepackage{booktabs}

% Enumerate %
\usepackage{enumerate}

% Operators %
\DeclareMathOperator{\Ker}{Ker}
\DeclareMathOperator{\Img}{Im}
\DeclareMathOperator{\End}{End}
\DeclareMathOperator{\Aut}{Aut}
\DeclareMathOperator{\Inn}{Inn}

% Common operators %
\newcommand{\R}{\mathbb{R}}
\newcommand{\N}{\mathbb{N}}
\newcommand{\Z}{\mathbb{Z}}
\newcommand{\Q}{\mathbb{Q}}
\newcommand{\C}{\mathbb{C}}

% American Paragraph Skip %
\setlength{\parindent}{0pt}
\setlength{\parskip}{1em}

% Document %
\pagestyle{fancy}
\begin{document}

Pár hintů k úlohám z mat. olympiády (72. ročník, kategorie A).

\begin{enumerate}
 \item Zamyslete se, jestli se ta rovnice nedá nějak zjednodušit. Dále,
  $\lfloor a \rfloor$ je vždycky celé číslo pro každé $a \in \R$. Co se tím
  mohu dozvědět o té další proměnné?
 \item Trojúhelník $AB'C'$ je dvakrát větší než $ABC$. V trojúhelnících se
  všechny vzdálenosti zvětšují lineárně, to znamená, že $AB'C'$ má vzdálenosti
  mezi všemi body (ne jen vrcholy) dvakrát větší.
 \item Nejdřív si položte jednodušší otázku. Hledáme \textbf{nejmenší} možný
  počet tahů. Je dobré začít tím, že si vytvořím nějakou (sice nereálnou ale
  aspoň přibližnou) spodní hranici toho, kolik budu potřebovat tahů. Co kdybych
  mohl žetony posouvat i skrz ostatní žetony (zkrátka na libovolné sousední 
  políčko). Kolik potřebuji v takovém případě minimálně tahů, abych každý žeton
  dostal tam, kde má na konci být? Nedá se tahle situace náhodou rozšířit na tu
  původní?
 \item Jaký má medián posloupnost $1,2,3,\ldots,k$ a posloupnost
  $1,2,3,\ldots,n$? Jaké číslo vás jako první napadne, že by měl být medián
  \textbf{podílu} těchhle posloupností? Ten tip je správně, ale je třeba to
  dokázat. Já jsem postupoval tak, že jsem ukázal, že to číslo má stejný počet
  zlomků nalevo jako napravo. Ale určitě to jde i chytřejc.
 \item Tuhle úlohu jsem řešil hloupě, takže spíš budu radši, když to zkusíte
  jinak. Nenapadlo mě jiné řešení než to prostě upočítat. Položte si ten
  trojúhelník do reálné roviny a rozmyslete si, že si můžete počítání hodně
  zjednodušit. Například si mohu nějak pěkně volit souřadnice jednoho z
  vrcholů. Taky, protože se všechno v trojúhelníku zvětšuje lineárně, mohu si i
  zvolit libovolně délku jedné strany. Najděte si vzorečky pro osu úhlu a
  ortocentrum (průnik výšek) a pak si zahrajte na kalkulačku. \textbf{Ale
  zkuste to prosím nejdřív jinak}.
 \item (\textcolor{red}{těžká úloha, hlavně část (b)}).
  \begin{enumerate}[(a)]
   \item Každé číslo v té posloupnosti musí být dělitelné \textbf{nějakým}
    prvočíslem.  Stačí vám proto ukázat, že když nějaké prvočíslo dělí nějaký
    člen, pak nemůže dělit žádný vyšší. Protože je nekonečně mnoho členů a
    každé různé prvočíslo dělí nejvýše jeden, musí těch prvočísel dělících
    jeden člen být nekonečně mnoho.
   \item Tuhle úlohu jsem řešil s trochou znalosti teorie čísel, takže neznám
    zatím žádné čistě \uv{středoškolské} řešení. Ale aspoň startovní bod, který
    by měl fungovat, vám prozradím. Řeště to sporem. Představte si, že by jen
    konečně mnoho prvočísel nedělilo žádný člen té posloupnosti. Protože jich
    je jen konečně mnoho, existuje mezi nimi nějaké nejvyšší (třeba $p$). Co
    pak můžeme říct o všech prvočíslech větších než $p$?
  \end{enumerate}
\end{enumerate}

\end{document}
