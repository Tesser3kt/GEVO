\documentclass[12pt,twoside]{article}
\usepackage{geometry}
\geometry{a4paper,margin=1.5cm,footskip=2em}

\usepackage[czech]{babel}
\usepackage[table]{xcolor}

\usepackage{pgfpages}
\pgfpagesuselayout{2 on 1}[a4paper,border shrink=0pt,landscape]

\usepackage{fontawesome5}
\usepackage{ragged2e}
\usepackage{parskip}

\usepackage{booktabs,makecell,xltabular}

\usepackage[T1]{fontenc}
\usepackage[lf,default]{FiraSans}
\usepackage{zi4}

\usepackage{regexpatch}
\usepackage[os=mac]{menukeys}
\renewmenumacro{\keys}[+]{shadowedroundedkeys}
\renewmenumacro{\menu}[>]{angularmenus}
\xpatchcmd*{\SPACE}{2em}{1em}{}{}

\renewcommand{\tabularxcolumn}[1]{m{#1}}
\renewcommand{\arraystretch}{1.4}
\arrayrulecolor{gray!60!white}

\makeatletter
\renewcommand{\maketitle}{{\centering\sffamily{\LARGE\bfseries\@title}\par\vskip\baselineskip{\large\@date}\par}\vskip\baselineskip}
% nifty commands by Paul Gaborit from http://tex.stackexchange.com/a/236891/226
\def\setmenukeyswin{\def\tw@mk@os{win}}
\def\setmenukeysmac{\def\tw@mk@os{mac}}
\makeatother

\usepackage{hyperref}
\urlstyle{same}

\title{\LaTeX~(asi ty nejužitečnější) snippety}
\author{Adam Klepáč}
\date{}

\begin{document}

\maketitle

\emph{Některé snippety se aktivují Tabem a některé automaticky, jen co ho
dopíšete (hlavně ty matematické). Některé fungují jen v matice. Abych to
odlišil, když}
\begin{itemize}
 \item \emph{je u snippetu \keys{\tab}, tak se aktivuje Tabem};
\item \emph{je u snippetu \keys{\$}, tak funguje jen v math modu (jak display,
 tak inline)}.
\end{itemize}

\textbf{Snippety mají tzv. \uv{tabstopy}, to jsou místa označená v tom
automaticky vytvořeném textu, kam Vim skočí při zmáčknutí \keys{\tab}. Je
důležité vždycky snippet dokončit; to znamená \uv{doTabovat} až na konec, jinak
se vám stane, že když budete chtít odsadit, tak vám místo toho Vim skočí někam
nahoru, aby dokončil snippet.}
\bigskip

\begin{xltabular}{\textwidth}{
	>{\setmenukeyswin}c@{\hspace{2em}}
	>{\renewcommand\cellalign{cl}\RaggedRight\arraybackslash}r}

 	\multicolumn{2}{c}{\large \textbf{Sekce, kapitoly, atd.}}\\*
	\toprule

	part | chap | sec | ssec | sssec \keys{\tab} & 
	\makecell[r]{
	 Snippety pro nadpisy\\
	 s automatickým labelem.
 	}\\*
	\bottomrule

	& \\*
	\multicolumn{2}{c}{\large \textbf{Environments}}\\*
	\toprule

 	mk & Inline math (\$...\$)\\*
 	\midrule
 	md & Display math (\textbackslash [ ... \textbackslash ])\\*
 	\midrule
 	beg \keys{\tab} & \makecell[r]{
	 Obecný \textbackslash begin\{\} ... \textbackslash end\{\}.\\
	 Název se automaticky kopíruje i do end.}\\*
	\midrule
	 ben | bit \keys{\tab} &
	\makecell[r]{
	 enumerate/itemize\\
	 prvním \keys{\tab} = nastavení seznamu.\\
	 Samo se hodí do [].\\
	 Enter = automaticky \textbackslash item.\\
	 Druhý Enter ho zruší, když chcete odstavec.
	}\\
	\midrule

	fig | tikz \keys{\tab} & Prázdná figure/figure s tikzpicture.\\
	\midrule

 	table | tabular \keys{\tab} & Prázdná table/table s tabularx.\\
 	\midrule

 	eq | eq* \keys{\tab} & equation/equation*\\
 	\midrule

 	split \keys{\tab} & equation* se split uvnitř\\
 	\midrule

 	== \keys{\$} &
 	\makecell[r]{
 	 Změní se na \&= \textbackslash\textbackslash.\\
 	 Funguje jenom v splitu.\\
 	 Zarovnává dlouhé rovnosti podle =.
 	}\\
 	\midrule

	\makecell{
	 thm | prop | claim | lem\\
	 def | cor | prob | exam\\
	 rmrk | prf | warn | obs
	}\keys{\tab} &
	\makecell[r]{
	 Všechna možná  prostředí -- \\
	 -- theorem, lemma atd.\\
	 První Tab je skok za název prostředí,\\
	 když chcete napsat název.\\
	 Druhý Tab je na label, automaticky\\
	 vygenerovaný z názvu.
	}\\
	\bottomrule
	& \\
	&\\

	\multicolumn{2}{c}{\large \textbf{Závorky}}\\
	\toprule
	lr( | lr[ | lr\{ | lr< \keys{\tab}\keys{\$} & Závorky, co se přizpůsobí
 	velikosti.\\
 	\midrule

 	() \keys{\$}& To samé, co lr(, akorát automatické.\\
 	\midrule

 	set \keys{\$} & Vytvoří závorky \{ \}.\\
 	\midrule

 	norm \keys{\$} & Vytvoří dvojité svislé čáry $\|\ldots\|$.\\
 	\midrule

 	lr| | lrn \keys{\tab}\keys{\$} & Přizpůsobivé $|\ldots|$ / $\|\ldots\|$.\\
 	\midrule

 	<> \keys{\$} & Špičaté závorky $\langle \ldots \rangle$.\\
 	\bottomrule

 	& \\
 	\multicolumn{2}{c}{\large \textbf{Dekorátory textu}}\\
 	\toprule

 	em | bf | al | tt \keys{\tab} & \textbackslash emph\{\}, \textbackslash
 	textbf\{\} atd.\\
 	\midrule

 	tt \keys{\$} & \textbackslash text\{\} v math modu.\\
 	\midrule

	cal | bb | frak | mrm | bf | sf \keys{\$} & Prázdný \textbackslash
 	mathcal\{\}, \textbackslash mathbb\{\} apod.\\
 	\midrule

 	písmeno + cal | bb | frak | rm | bf | sf \keys{\$}&
 	\makecell[r]{
 	 \textbackslash mathcal\{\}, \textbackslash mathbb\{\} apod.\\
 	 s tím písmenem uvnitř.\\
 	 Např. z Acal se stane \textbackslash	mathcal\{A\}.
 	}\\
 	\midrule

 	vec | bar | tld | ndl | hat \keys{\$} & Prázdný \textbackslash vec\{\},
 	\textbackslash bar\{\} apod.\\
 	\midrule

 	písmeno + \textbackslash. | bar | tld | \textbackslash\_ | hat \keys{\$} &
 	\textbackslash vec\{\}, \textbackslash bar\{\} atd. s písmenem uvnitř.\\
 	\bottomrule

 	& \\
 	\multicolumn{2}{c}{\large \textbf{Obyčejné funkce}}\\
 	\toprule

 	sq \keys{\$} & \textbackslash sqrt\{\}\\
 	\midrule

 	nsq \keys{\$} & n-tá odmocnina \textbackslash sqrt[]\{\}\\
 	\midrule

 	\makecell{
 	 sin | cos | tan | cot | csc | sec\\
 	 arcsin | arctan | arccot\\
 	 exp | log\\
 	 det | deg | dim\\
 	 ker | img | dom | codom\\
 	 \#
 	} \keys{\$} & \makecell[r]{
 	 Před všechna se hodí automaticky \textbackslash.\\
 	 Jestli vás to zajímá, arccos tam není,\\
 	 protože cc je snippet pro podmnožinu.\\
 	 Pro to zatím nemám řešení.
 	}\\
 	\bottomrule

	& \\
 	\multicolumn{2}{c}{\large \textbf{Inline math symboly}}\\
 	\toprule

 	... \keys{\$} & \textbackslash ldots\\
 	\midrule

 	nf \keys{\$} & nekonečno\\
 	\midrule

	=> | =< \keys{\$} & implikace zleva doprava/zprava doleva.\\
	\midrule

	iff \keys{\$} & ekvivalence\\
	\midrule

	aa | vv \keys{\$} & logické a/nebo\\
	\midrule

	neg \keys{\$} & negace\\
	\midrule

	AA | EE \keys{\$} & pro všechny/existuje\\
	\midrule

	!= | <= | >= \keys{\$} & 
	\makecell[r]{
	 není rovno/\\
	 menší nebo rovno/\\
	 větší nebo rovno
	}\\
	\midrule

 	:= \keys{\$} & Definítko (\textbackslash coloneqq)\\
 	\midrule

 	<< | >> \keys{\$} & Mnohem menší/mnohem větší.\\
 	\midrule

 	-= | $\sim$= | $\sim\sim$ \keys{\$} & 
 	\makecell[r]{
	 Identická rovnost/\\
	 isomorfismus (bijekce)/\\
	 podobnost.
	}\\
 	\midrule

 	+- | -+ \keys{\$} & Plus minus/minus plus\\
 	\midrule

	oo \keys{\$} & Složení zobrazení $ \circ $.\\
	\midrule

	xx \keys{\$} & Direktní součin $ \times $.\\
	\midrule

	** \keys{\$} & Násobení $ \cdot $.\\
	\midrule

	\textbackslash\textbackslash\textbackslash | Nn | UU \keys{\$} & Množinové
	minus/průnik/sjednocení.\\
	\midrule

 	-> (to) \keys{\$} & Zobrazení z ... do ... ($ \to $).\\
 	\midrule

 	!> \keys{\$} & Zobrazuje se na ... ($ \mapsto $).\\
 	\midrule

 	inn | notin \keys{\$} & Je/není prvkem ...\\
 	\midrule

 	cc | scc | ncc \keys{\$} & 
 	\makecell[r]{
 	 podmnožina/\\
 	 vlastní podmnožina/\\
 	 není podmnožinou
 	}\\
 	\midrule

 	div | ndiv \keys{\$} & Dělí/nedělí ...\\
 	\midrule

 	|| \keys{\$} &
 	\makecell[r]{
 	 Dělítko v definici množiny,\\
 	 třeba $\{(x,y) \in R^2 \mid x^2+y^2 = 1\}$.
 	}\\
 	\midrule

 	sgn \keys{\$} & (-1) na něco\\
 	\bottomrule
	
	& \\
 	\multicolumn{2}{c}{\large \textbf{Display math symboly}}\\
 	\toprule

 	// \keys{\$} & \textbackslash frac\{...\}\{...\}\\
 	\midrule

 	ucelený text + / \keys{\$} & \makecell[r]{
 	 \textbackslash frac\{ten text\}\{...\}\\
	 Za ucelený text se považuje leccos,\\	
	 třeba i věci jako $\arcsin^{5}(x_1^2 + x_3^3)$.
	}\\
	\midrule

	(cokoliv) + / \keys{\$} & 
 	\makecell[r]{
 	 \textbackslash frac\{cokoliv\}\{...\}\\
 	 Vezme veškerý text až po první\\
 	 otevřenou závorku, která doplňuje\\
 	 počet všech uzavřených závorek.
 	}\\
 	\midrule

 	\makecell{
 	 binom\\
 	 ucelený text + bin\\
 	 (cokoliv) + bin
 	}\keys{\$} & 
 	\makecell[r]{
 	 Funguje stejně jako frac, akorát pro\\
 	 \textbackslash binom\{...\}\{...\}
 	}\\
 	\midrule

 	sum \keys{\$} & \makecell[r]{
 	 Velký součet.\\
 	 první \keys{\tab} = to dole\\
	 druhý \keys{\tab} = to nahoře\\
	 třetí \keys{\tab} = to vevnitř
	}\\
	\midrule

	nsum \keys{\$} & Součet od i=1 do n.\\
	\midrule

	isum \keys{\$} & Součet přes indexovou množinu.\\
	\midrule

	prod | nprod | iprod \keys{\$} & To samé, akorát pro součin.\\
	\midrule

	uuu | nuu | iuu \keys{\$} & To samé, akorát pro sjednocení.\\
	\midrule

	nnn | nn1 | nni \keys{\$} & To samé, akorát pro průnik.\\
	\midrule

	int | dint \keys{\$} & 
	\makecell[r]{
	 Integrál bez/s dx na konci.\\
	 Funguje stejně jako suma.
	}\\
	\midrule

	lim | flim \keys{\$} &
	\makecell[r]{
	 Limita posloupnosti/funkce.\\
	 první \keys{\tab} = proměnná\\
	 druhý \keys{\tab} = kam
	}\\
	\bottomrule

	& \\
	\multicolumn{2}{c}{\large \textbf{Dolní indexy}}\\
	\toprule

 	ucelený text (nebo závorky) + číslo \keys{\$} &
 	\makecell[r]{
 	 Automatický dolní index.\\
 	 Třeba z x2 se stane x\_2\\
   a z (f + g)6 se stane (f+g)\_6.
  }\\
  \midrule

  ucelený text / závorky + víc čísel \keys{\$} &
  \makecell[r]{
   To samé, akorát to navíc dá\\
   \{\} kolem těch čísel.
  }\\
  \midrule

  \makecell{
   ucelený text / závorky + \\
	 ii | jj | kk | ll | mm | nn
	} \keys{\$} &
 	\makecell[r]{
 	 Dolní index s obvyklými písmeny.\\
 	 Třeba z Aii se stane A\_i.
 	}\\
 	\midrule

 	\_\_ \keys{\$} & Změní se v \_\{\}.\\
 	\bottomrule
	
	& \\
 	\multicolumn{2}{c}{\large \textbf{Horní indexy}}\\
 	\toprule

 	sr | cb | inv \keys{\$} & \^{}2 | \^{}3 | \^{}\{-1\}\\
 	\midrule

 	td \keys{\$} & \^\{...\}\\
 	\bottomrule

 	& \\
 	\multicolumn{2}{c}{\large \textbf{Nezařazené věci}}\\
 	\toprule

 	pac \keys{\tab} & \textbackslash usepackage[...]\{...\}\\
	\midrule

	url | cite | eqref | foot \keys{\tab} & Přidá \textbackslash~před a \{\} za.\\
	\midrule

	href | myref \keys{\tab} & Přidá \textbackslash~před a \{\} \{\} za.\\
	\midrule

	xref \keys{\tab} & \textbackslash hyperref[...]\{...\}.\\
	\bottomrule
\end{xltabular}

\end{document}
