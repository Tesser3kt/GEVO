\documentclass[12pt,twoside]{article}
\usepackage{geometry}
\geometry{a4paper,margin=1.5cm,footskip=2em}

\usepackage[czech]{babel}
\usepackage[table]{xcolor}

\usepackage{pgfpages}
\pgfpagesuselayout{2 on 1}[a4paper,border shrink=0pt,landscape]

\usepackage{fontawesome5}
\usepackage{ragged2e}
\usepackage{parskip}

\usepackage{booktabs,makecell,xltabular}

\usepackage[T1]{fontenc}
\usepackage[lf,default]{FiraSans}
\usepackage{zi4}

\usepackage{regexpatch}
\usepackage[os=mac]{menukeys}
\renewmenumacro{\keys}[+]{shadowedroundedkeys}
\renewmenumacro{\menu}[>]{angularmenus}
\xpatchcmd*{\SPACE}{2em}{1em}{}{}

\renewcommand{\tabularxcolumn}[1]{m{#1}}
\renewcommand{\arraystretch}{1.4}
\arrayrulecolor{gray!60!white}

\makeatletter
\renewcommand{\maketitle}{{\centering\sffamily{\LARGE\bfseries\@title}\par\vskip\baselineskip{\large\@date}\par}\vskip\baselineskip}
% nifty commands by Paul Gaborit from http://tex.stackexchange.com/a/236891/226
\def\setmenukeyswin{\def\tw@mk@os{win}}
\def\setmenukeysmac{\def\tw@mk@os{mac}}
\makeatother

\usepackage{hyperref}
\urlstyle{same}

\title{AstroNvim pár klávesových zkratek}
\author{Adam Klepáč}
\date{}

\begin{document}

\maketitle

\centering
\emph{Bacha! Vim je case-sensitive.}

\emph{Věci s * jsou podle mě extra užitečný.}

\bigskip

\begin{xltabular}{\textwidth}{
		>{\setmenukeyswin}c @{\hspace{2em}}
		>{\renewcommand\cellalign{cl}\RaggedRight\arraybackslash}r}

	\multicolumn{2}{c}{\textbf{\large Zkratky v normal modu}}\\
	\toprule

	\keys{:} & Proveď příkaz.\\*
	\midrule

	\keys{Esc}* & Zpět do normal mode.\\*
	\midrule

	\keys{h | j | k | l} nebo
	\keys{\arrowkeyleft~|~\arrowkeydown~|~\arrowkeyup~|~\arrowkeyright}* &
	Doleva/dolu/nahoru/doprava.\\*
	\midrule

	\keys{a}* & Insert mode (za zvýrazněný symbol).\\*
	\midrule

	\keys{i}* & Insert mode (před zvýrazněný symbol).\\*
	\midrule

	\keys{v}* & Visual mode.\\*
	\midrule

	\keys{V}* & Visual mode vodorovný výběr. Označí současný řádek.\\*
	\midrule

	\keys{\ctrl + v} & Visual mode svislý výběr.\\*
	\midrule

	\keys{H~|~M~|~L} & Vršek/střed/spodek obrazovky.\\*
	\midrule

	\keys{w~|~e} &
	\makecell[r]{Skoč na začátek/konec nejbližšího dalšího slova.\\
	 Co je ve Vimu \uv{slovo} je u symbolů někdy trochu random.}
	 \\*
	\midrule

	\keys{b~|~ge} & Skoč na začátek/konec nejbližšího předchozího slova.\\*
	\midrule

	\keys{0}* & Skoč na začátek řádku.\\*
	\midrule

	\keys{\^{}}* & Skoč na první symbol na řádku, co není mezera.\\*
	\midrule

	\keys{\$}* & Skoč na konec řádku.\\*
	\midrule

	\keys{g\_} & Skoč na poslední symbol řádku, co není mezera.\\*
	\midrule

	\keys{gg} & Skoč na začátek souboru.\\*
	\midrule
	
	\keys{G} & Skoč na konec souboru.\\*
	\midrule

	\keys{\%}* & 
	\makecell[r]{Skoč na druhou z páru závorek(třeba z `[' na `]').\\ Taky ze
	 začátku bloku na konec, třeba z \textbackslash begin\{...\} na
	\textbackslash end\{...\}}.\\*
	\midrule

	\keys{\{~|~\}} & Skoč na předchozí/další odstavec.\\*
	\midrule

	\keys{o}* & Nový řádek pod tímto + rovnou insert mode.\\*
	\midrule

	\keys{O}* & Nový řádek nad tímto + rovnou insert mode.\\*
	\midrule

	\keys{gw} & Reflow (co já vím, jak je to česky) vybraný text (opraví
	zarovnání).\\*
	\midrule

	\keys{gwip}* & Reflow tenhle odstavec.\\*
	\midrule

	\keys{u}* & Undo.\\*
	\midrule

	\keys{U} & Undo jen poslední řádek.\\*
	\midrule

	\keys{\ctrl + r}* & Redo.\\*
	\midrule

	\keys{<<~|~>>}* & \makecell[r]{Posuň tenhle řádek o Tab doleva/doprava.\\
	Když chcete třeba čtyři doleva, stačí \keys{4<<}.}\\*
	\midrule

	\keys{yy}* & Zkopíruj tenhle řádek.\\*
	\midrule

	\keys{p}* & Vlož za současnou pozici.\\*
	\midrule

	\keys{P}* & Vlož před současnou pozici.\\*
	\midrule

	\keys{dd}* & \makecell[r]{Smaž tenhle řádek.\\
	 Pozor, ve Vimu je smazat vždycky \uv{vyjmout},\\
	takže se ten text zároveň zkopíruje.}\\*

	\bottomrule
\end{xltabular}

\begin{xltabular}{\textwidth}{
		>{\setmenukeyswin}c @{\hspace{2em}}
		>{\renewcommand\cellalign{cl}\RaggedRight\arraybackslash}r}

	\multicolumn{2}{c}{\textbf{\large Zkratky ve visual modu}}\\
	\toprule

	\keys{ib} & Vyber všechno uvnitř `()'.\\*
	\midrule

	\keys{iB}* & Vyber všechno uvnitř `\{\}'.\\*
	\midrule

	\keys{<~|~>}* &
	\makecell[r]{Posuň vybrané o jeden Tab doleva/doprava.\\
	Když potřebujete třeba o čtyři doleva, stačí zmáčknout \keys{4<}.}\\*
	\midrule

	\keys{y}* & Zkopíruj vybrané.\\*
	\midrule

	\keys{d}* & Smaž vybrané (zároveň zkopíruje).\\*

	\bottomrule
\end{xltabular}

\begin{xltabular}{\textwidth}{
		>{\setmenukeyswin}c @{\hspace{2em}}
		>{\renewcommand\cellalign{cl}\RaggedRight\arraybackslash}r}

	\multicolumn{2}{c}{\textbf{\large AstroNvim zkratky (vesměs v normal modu)}}\\
	\toprule

	\keys{\ctrl + \arrowkeyup~|~\arrowkeydown~|~\arrowkeyleft~|~\arrowkeyright} &
	Zvětši/zmenši daným směrem aktivní okno.\\*
	\midrule

	\keys{\ctrl + k~|~j~|~h~|~l} & Přepni na okno nahoře/dole/vlevo/vpravo.\\*
	\midrule

	\keys{\ctrl + s} & Ulož soubor.\\*
	\midrule

	\keys{\ctrl + q} & Ukonči Neovim.\\*
	\midrule

	\keys{\SPACE + /}* & Odkomentuj řádek/vybrané řádky (umí poznat jazyk).\\*
	\midrule

	\keys{\SPACE + n} & Nový soubor (v otevřené složce).\\*
	\midrule

	\keys{\SPACE + c} & Zavři tenhle buffer.\\*
	\midrule

	\keys{\textbackslash} & Rozděl okno vodorovně.\\*
	\midrule

	\keys{|} & Rozděl okno svisle.\\*
	\midrule

	\keys{]t~|~[t} & Přepni na tab vpravo/vlevo.\\*
	\midrule

	\keys{]b~|~[b}* & Přepni na buffer vpravo/vlevo.\\*
	\midrule

	\keys{\SPACE + e}* & Otevři/zavři NeoTree (prohlížeč souborů).\\*
	\midrule

	\keys{\SPACE + o}* & Skoč na NeoTree (prohlížeč souborů).\\*
	\midrule

	\keys{\SPACE + fb}* & Prohledávač otevřených bufferů.\\*
	\midrule

	\keys{\SPACE + ff}* & Prohledávač souborů ve složce.\\*
	\midrule

	\keys{\SPACE + fk}* & Prohledávač klávesových zkratek.\\*
	\midrule

	\keys{\SPACE + fo}* & Prohledávač nedávných souborů.\\*
	\midrule

	\keys{\SPACE + th} & Rozdělí vodorovně okno a otevře dole terminál (jako v
	Codu).\\*
	\midrule

	\keys{\SPACE + vc} & \makecell[r]{Spustí automatickou kompilaci LaTeXu.\\
	Soubor se vždycky při uložení zkompiluje.}\\*
	\bottomrule
\end{xltabular}

\end{document}


