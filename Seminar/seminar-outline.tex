\documentclass[a4paper,11pt]{article}

\usepackage[czech,english]{babel}
% Fonts %
\usepackage{fouriernc}
\usepackage[T1]{fontenc}

% Colors %
\usepackage[dvipsnames]{color}
\usepackage[dvipsnames]{xcolor}

% Page Layout %
\usepackage[margin=1.5in]{geometry}

% Fancy Headers %
\usepackage{fancyhdr}
\fancyhf{}
\cfoot{\thepage}
\rhead{}
\renewcommand{\headrulewidth}{0pt}
\setlength{\headheight}{16pt}

% Math
\usepackage{mathtools}
\usepackage{amssymb}
\usepackage{faktor}
\usepackage{import}
\usepackage{caption}
\usepackage{subcaption}
\usepackage{wrapfig}
\usepackage{enumitem}
\setlist{topsep=0pt}

\usepackage{tikz}
\usetikzlibrary{cd,positioning,babel,shapes,decorations.text,
 decorations.pathmorphing}
\usepackage{tkz-base}
\usepackage{tkz-euclide}

% Theorems
\usepackage[thmmarks, amsmath, thref]{ntheorem}
\usepackage{thmtools}

\theoremsymbol{\ensuremath{\blacksquare}}
\newtheorem*{solution}{Possible solution.}

% Title %
\title{\Huge\textsf{Math Homework -- PreIB 3.AB 2 \& 3}\\
 \Large\textsf{Trigonometric Functions}
 \author{Áďa Klepáčů}
 \date{\today}
}

% Table of Contents %
\usepackage{hyperref}
\hypersetup{
 colorlinks=true,
 linktoc=all,
 linkcolor=blue
}

% Tables %
\usepackage{booktabs}
\usepackage{tabularx}

% Patch for hyphens
\usepackage{regexpatch}
\makeatletter
% Change the `-` delimiter to an active character
\xpatchparametertext\@@@cmidrule{-}{\cA-}{}{}
\xpatchparametertext\@cline{-}{\cA-}{}{}
\makeatother

\newcolumntype{s}{>{\centering\arraybackslash}p{.4\textwidth}}

% Operators %
\DeclareMathOperator{\Ker}{Ker}
\DeclareMathOperator{\Img}{Im}
\DeclareMathOperator{\End}{End}
\DeclareMathOperator{\Aut}{Aut}
\DeclareMathOperator{\Inn}{Inn}

% Common operators %
\newcommand{\R}{\mathbb{R}}
\newcommand{\N}{\mathbb{N}}
\newcommand{\Z}{\mathbb{Z}}
\newcommand{\Q}{\mathbb{Q}}
\newcommand{\C}{\mathbb{C}}

\newcommand{\clr}{\textcolor{BrickRed}}
\newcommand{\clb}{\textcolor{RoyalBlue}}
\newcommand{\clg}{\textcolor{ForestGreen}}
\newcommand{\clm}{\textcolor{Fuchsia}}
\newcommand{\clv}{\textcolor{violet}}
\newcommand{\clbr}{\textcolor{Sepia}}
\newcommand{\cly}{\textcolor{Dandelion}}

% American Paragraph Skip %
\setlength{\parindent}{0pt}
\setlength{\parskip}{1em}

% Document %
\pagestyle{fancy}
\begin{document}

\section*{Higher Mathematics \\ \large\sffamily Seminar Outline 2024/25}
\thispagestyle{fancy}

\subsection*{Description}

The aim of the seminar is to acquaint students with the typical content of
introductory courses to university-level mathematics. The difference between
high-school mathematics and university mathematics is stark and the shift of
focus from mostly mechanical counting to pure logic and problem solving leaves
freshmen often bemused and frustrated, inviting failure. Advance knowledge of
the habitual curriculum alongside common proof methods and logical arguments
should lessen the burden and, perhaps, bring to light the beauty of the art that
is mathematics.

\subsection*{Expected Outcomes}
By the end of the course, students will have gained the knowledge (and hopefully
comprehension) of
\begin{itemize}
 \item first-order logic (the very `language' of modern mathematics),
 \item basics of set theory (the very `foundation' of modern mathematics),
 \item selected parts of a university-level introductory course in pure
  mathematics (the exact course will be elected by the attendees),
\end{itemize}
and also the ability to
\begin{itemize}
 \item construct logical proofs using common methods,
 \item use reasoning and intuition to solve (chiefly non-algorithmic) problems.
\end{itemize}

\subsection*{Topics}

Depending on the chosen course, the following will be discussed (two courses are
detailed here; more can be added at the students' behest).

\subsubsection*{Linear Algebra}
\begin{enumerate}
 \item Systems of linear equations.
 \item Fields \& vector spaces.
 \item Linear maps \& matrices.
 \item Determinant.
 \item Scalar product \& orthogonal projection.
 \item Eigenvalues and eigenvectors, diagonalization.
\end{enumerate}

\subsubsection*{Elementary Number Theory}
\begin{enumerate}
 \item Prime numbers \& divisibility.
 \item GCD \& Euclid's algorithm.
 \item Congruences \& Chinese Remainder Theorem.
 \item Quadratic residues.
 \item Diophantine equations \& rational points on curves.
\end{enumerate}

\end{document}
