\documentclass[a4paper,11pt]{article}

\usepackage[english,czech]{babel}
% Fonts %
\usepackage{fouriernc}
\usepackage[T1]{fontenc}

% Colors %
\usepackage[dvipsnames]{xcolor}

% Colored boxes %
\usepackage[many]{tcolorbox}

% Page Layout %
\usepackage[margin=.5in]{geometry}

% Fancy Headers %
\usepackage{fancyhdr}
\fancyhf{}
\cfoot{\thepage}
\rhead{}
\renewcommand{\headrulewidth}{0pt}
\setlength{\headheight}{16pt}

% Math
\usepackage{mathtools}
\usepackage{amssymb}
\usepackage{faktor}
\usepackage{import}
\usepackage{caption}
\usepackage{subcaption}
\usepackage{wrapfig}
\usepackage{enumitem}
\setlist{topsep=0pt}
\setlist[enumerate,1]{label=(\arabic*)}

\usepackage{tikz}
\usetikzlibrary{cd,positioning,babel,shapes}
\usepackage{tkz-base}
\usepackage{tkz-euclide}
\tikzset{
  vertex/.style = {shape=circle,fill,text=white,minimum size=9pt,inner
  sep=1pt}
}

% Theorems
\usepackage[thmmarks, amsmath, thref]{ntheorem}
\usepackage{thmtools}

\theoremsymbol{\ensuremath{\blacksquare}}
\newtheorem*{solution}{Possible solution.}

% Title %
\title{\Huge\textsf{Ústní zkouška}\\
 \Large\textsf{z Úvodu do matematické analýzy, části prvé}\\
 \vspace*{1em}
 Verze: r3kt\\
 \author{Přednášející: His Divine Wisdom Sir Adam Clypatch}
 \date{19. ledna 2024}
}

% Table of Contents %
\usepackage{hyperref}
\hypersetup{
 colorlinks=true,
 linktoc=all,
 linkcolor=blue
}

% Tables %
\usepackage{booktabs}
\usepackage{tabularx}

% Patch for hyphens
\usepackage{regexpatch}
\makeatletter
% Change the `-` delimiter to an active character
\xpatchparametertext\@@@cmidrule{-}{\cA-}{}{}
\xpatchparametertext\@cline{-}{\cA-}{}{}
\makeatother

\newcolumntype{s}{>{\centering\arraybackslash}p{.4\textwidth}}

% Operators %
\DeclareMathOperator{\img}{im}
\DeclareMathOperator{\dom}{dom}
\DeclareMathOperator{\codom}{codom}

% Common operators %
\newcommand{\R}{\mathbb{R}}
\newcommand{\N}{\mathbb{N}}
\newcommand{\Z}{\mathbb{Z}}
\newcommand{\Q}{\mathbb{Q}}
\newcommand{\C}{\mathbb{C}}

\newcommand{\clr}{\textcolor{red}}
\newcommand{\clb}{\textcolor{blue}}
\newcommand{\clg}{\textcolor{green}}
\newcommand{\clm}{\textcolor{magenta}}
\newcommand{\clv}{\textcolor{violet}}
\newcommand{\clbr}{\textcolor{Sepia}}

% American Paragraph Skip %
\setlength{\parindent}{0pt}
\setlength{\parskip}{1em}

% Table array stretch
\renewcommand{\arraystretch}{1.5}

% Document %
\pagestyle{empty}

\newcommand{\defthis}[1]{%
\par Definujte tento pojem: \emph{#1}.%
\vspace{\parskip}%
\hrule%
}

\newcommand{\provethisez}[1]{%
\par Dokažte následující (snadné) tvrzení.\\[.5em]%
\emph{#1}%
\vspace{\parskip}%
\hrule%
}

\newcommand{\provethishard}[1]{%
\par Dokažte následující tvrzení.\\[.5em]%
\emph{#1}%
\vspace{\parskip}%
\hrule%
}

\newcommand{\calcthislimez}[1]{%
\par Spočtěte následující limitu.\\[.5em]%
\emph{#1}%
\vspace{\parskip}%
\hrule%
}

\newcommand{\calcthisderez}[1]{%
\par Spočtěte derivaci následující funkce.\\[.5em]%
\emph{#1}%
\vspace{\parskip}%
\hrule%
}

\newcommand{\calcthisintez}[1]{%
\par Spočtěte následující integrál.\\[.5em]%
\emph{#1}%
\vspace{\parskip}%
\hrule%
}

\begin{document}
 \hrule
 \defthis{oboustranná limita funkce}
 \defthis{spojitost funkce v bodě}
 \defthis{extrém reálné funkce na intervalu}
 \defthis{derivace reálné funkce v bodě}
 \defthis{exponenciála a logaritmus}
 \defthis{funkce sinus a cosinus}
 \defthis{obecná mocnina}
 \defthis{Taylorův polynom daného stupně reálné funkce v bodě}
 \defthis{Symbol malé $o$}
 \defthis{primitivní funkce na intervalu}

 \clearpage
 
 \hrule
 \provethisez{
  Ať má reálná funkce $f$ \textbf{konečnou} limitu v bodě $a \in
  \R^{*}$. Pak je $f$ na jistém prstencovém okolí $a$ omezená.
 }
 \provethisez{
  Ať $a \in \R^{*}$ a $f,g$ jsou reálné funkce. Platí-li
  \[
   \lim_{x \to a} f(x) > \lim_{x \to a} g(x),
  \]
  pak $f(x) > g(x)$ pro každé $x$ z jistého prstencového okolí $a$.
 }
 \provethisez{
  Existují reálné funkce $f,g$ a čísla $a,A,B \in \R^{*}$, že
  \[
   \lim_{x \to a} g(x) = A \quad \text{a} \quad \lim_{y \to A} f(y) = B,
  \]
  ale přesto $\lim_{x \to a} (f \circ g)(x) \neq B$.
 }
 \provethisez{
  Reálná funkce spojitá na uzavřeném intervalu je na něm omezená.
 }
 \provethisez{
  Ať $f$ je reálná funkce, $a \in \R$ a existuje $f'(a)$. Pak existuje
  \[
   \lim_{x \to a} \frac{f(x) - f(a)}{x - a}
  \]
  a je rovna $f'(a)$.
 }
 \provethisez{
  Ať $f$ je reálná funkce, $a \in \R$ a existuje konečná $f'(a)$. Pak je $f$ v
  bodě $a$ spojitá.
 }
 \provethisez{
  Ať má funkce $f$ v bodě $a$ extrém a ať $f'(a)$ existuje. Pak $f'(a) = 0$.
 }
 \provethisez{
  Ať $I \subseteq \R$ je interval a funkce $f$ má všude na $I$ zápornou
  derivaci. Pak je $f$ na $I$ klesající.\\
  \textbf{Hint}: Lagrangeova věta o střední hodnotě.
 }
 \provethisez{
  Pro všechna $x \in \R$ platí $\exp'x = \exp x$.\\
  \textbf{Hint}: Použijte rovnosti
  \[
   \exp(x + y) = \exp x \cdot \exp y \quad \text{a} \quad \lim_{x \to 0}
   \frac{\exp x - 1}{x} = 1.
  \]
 }
 \provethisez{
  Pro všechna $x,y > 0$ platí
  \[
   \log(xy) = \log x + \log y.
  \]
  \textbf{Hint}: Vlastnosti exponenciály.
 }

 \clearpage
 \hrule
 \provethisez{
  Ať $n \in \N$ a $f:M \to \R$ má v $a \in M$ derivace všech řádů do $n$ včetně.
  Pak
  \[
   \lim_{x \to a} \frac{f(x) - T^{f,a}_n(x)}{(x-a)^{n}} = 0.
  \]
  \textbf{Hint}: Indukcí podle $n$ užitím rovnosti $(T^{f,a}_n)' =
  T^{f',a}_{n-1}$.
 }
 \provethisez{
  Jsou-li $f_1,f_2,g_1,g_2$ reálné funkce a $f_1 = o(g_1), f_2 = o(g_2)$, pak
  $f_1f_2 = o(g_1g_2)$.
 }
 \provethisez{
  Ať $f,g:(a,b) \to \R$ jsou reálné funkce a $F,G$ jsou primitivní k $f,g$ na
  $(a,b)$. Pak
  \[
   \int fG = FG - \int Fg.
  \]
 }
 \provethisez{
  Ať $a < b, \alpha < \beta \in \R$, $f:(a,b) \to \R$ a
  $\varphi:(\alpha,\beta) \to (a,b)$ jsou reálné funkce, přičemž $\varphi'$
  existuje konečná na $(a,b)$. Ať $F$ je primitivní k $f$ na $(a,b)$. Pak
  \[
   \int (f \circ \varphi)(t) \cdot \varphi'(t) \, \mathrm{d}t = (F \circ
   \varphi)(t)
  \]
  pro $t \in (\alpha,\beta)$.
 }

 \clearpage
 \hrule
 \calcthislimez{
  \[
   \lim_{x \to 1} \frac{x^{2} - 1}{2 x^{2} - x - 1}.
  \]
 }
 \calcthislimez{
  \[
   \lim_{x \to 1} \frac{x^{m} - 1}{x^{n} - 1},
  \]
  kde $m,n \in \N$.
 }
 \calcthisderez{
  \[
   f(x) = \frac{\exp x - \exp (-x)}{\exp x + \exp(-x)}.
  \]
 }
 \calcthisderez{
  \[
   f(x) = \log(\log x - 3) + \arcsin \left( \frac{x-5}{2} \right).
  \]
 }
 \calcthislimez{
  \[
   \lim_{x \to 0} \frac{\tan x - x}{x - \sin x}.
  \]
 }
 \calcthislimez{
  \[
   \lim_{x \to 0^{+}} x^{x}.
  \]
 }
 \calcthislimez{
  \[
   \lim_{x \to 0} \frac{\cos x - \exp(-x^2 / 2)}{x^{4}}.
  \]
 }
 \calcthislimez{
  \[
   \lim_{x \to 0} \frac{\log(\cos x)}{x^2}.
  \]
 }
 \calcthisintez{
  \[
   \int \frac{\exp x}{2 + \exp x} \, \mathrm{d}x
  \]
  pro $x \in \R$.
 }
 
 \clearpage
 \hrule
 \calcthisintez{
  \[
   \int \arcsin x \, \mathrm{d}x
  \]
  pro $x \in (-1,1)$.
 }
 \calcthisintez{
  \[
   \int \frac{\log^2 x}{x} \, \mathrm{d}x
  \]
  pro $x > 0$.
 }
 \calcthisintez{
  \[
   \int \frac{x^2}{(x^2 + 2x + 2)^2} \, \mathrm{d}x
  \]
  pro $x \in \R$.
 }

 \clearpage
 \hrule
 \provethishard{
  Ať $f:\R \to \R$ je funkce. Potom je množina
  \[
   M \coloneqq \{x \in \R \mid f \text{ má v } x \text{ ostré lokální maximum}\}
  \]
  spočetná.\\
  \textbf{Návod}:
  \begin{enumerate}
   \item Z definice extrému existuje pro každé $x \in M$ okolí $P(x,\delta_x)$,
    na němž platí $f(y) < f(x), y \in P(x,\delta_x)$. Volme
    \[
     M_n \coloneqq \left\{x \in M \mid \delta_x > \frac{1}{n}\right\}.
    \]
    Pro každé $n \in \N$ najděte nalezněte číslo $\eta_n$ takové, že
    \[
     M_n = \coprod_{x \in M_n} B(x,\eta_n).
    \]
   \item Dokažte, že $M_n$, jakožto disjunktní sjednocení otevřených intervalů,
    je spočetná.
   \item Dokažte, že $M$ je spočetná.
  \end{enumerate}
 }
 
  Řekneme, že bod $a \in M$ je \textbf{inflexním bodem} funkce $f:M \to \R$,
  jestliže existuje konečná $f'(a)$ a $\delta>0$ takové, že buď
  \[
   \forall x \in (a-\delta,a): f(x) > T^{f,a}_1(x) \quad \text{a} \quad \forall
   x \in (a,a+\delta): f(x) < T^{f,a}_1(x),
  \]
  nebo
  \[
   \forall x \in (a-\delta,a): f(x) < T^{f,a}_1(x) \quad \text{a} \quad \forall
   x \in (a,a+\delta): f(x) > T^{f,a}_1(x).
  \]
  Slovy: bod $a$ je inflexním bodem $f$, když hodnoty $f$ v bodech vlevo od $a$
  leží pod, resp. nad, tečnou $f$ v bodě $a$ a hodnoty $f$ v bodech vpravo od
  $a$ leží nad, resp. pod, tečnou $f$ v bodě $a$.

  \emph{Dokažte, že když $a$ je inflexním bodem $f$, pak $f''(a)$ buď neexistuje
  nebo je rovna $0$.}\\
  \emph{
   \textbf{Návod}: Uvažujte pro spor, že třeba $f''(a) > 0$, pak použijte
   definici derivace a Lagrangeovu větu o střední hodnotě, abyste ukázali, že
   $a$ není inflexním bodem.
  }
  \vspace{\parskip}
  \hrule
  \emph{Ať $f:(a,b) \to \R$ je spojitá a $\varphi:(a,b) \to \R$ má konečnou nenulovou
  derivaci všude na $(a,b)$. Potom je $\varphi((a,b))$ interval; označíme jej
  $(\alpha,\beta)$. Položme $g(t) \coloneqq (f \circ \varphi ^{-1})(t) \cdot
  (\varphi ^{-1})'(t)$ a nechť $G$ je primitivní ke $g$ na $(\alpha,\beta)$.
  Dokažte, že pak}
  \[
   \int f(x) \, \mathrm{d}x = (G \circ \varphi)(x)
  \]
  \emph{pro $x \in (a,b)$.}\\
  \emph{\textbf{Návod}:
  \begin{enumerate}
   \item Použitím Darbouxovy vlastnosti (tj. vlastnosti zobrazování intervalu na
    interval) spojitých funkcí k důkazu, že
    \begin{enumerate}
     \item $\varphi((a,b))$ je vskutku interval;
     \item platí buď $\varphi' > 0$ nebo $\varphi' < 0$ na celém $(a,b)$.
    \end{enumerate}
   \item Dokažte, že existuje $\varphi ^{-1}:(\alpha,\beta) \to (a,b)$.
   \item Zaderivujte si.
  \end{enumerate}
 }
 \vspace{\parskip}
 \hrule
\end{document}
