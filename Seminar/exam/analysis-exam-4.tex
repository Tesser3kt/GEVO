\documentclass[a4paper,11pt]{article}

\usepackage[czech,english]{babel}
% Fonts %
\usepackage{fouriernc}
\usepackage[T1]{fontenc}

% Colors %
\usepackage[dvipsnames]{xcolor}

% Colored boxes %
\usepackage[many]{tcolorbox}

% Page Layout %
\usepackage[margin=1in]{geometry}

% Fancy Headers %
\usepackage{fancyhdr}
\fancyhf{}
\cfoot{\thepage}
\rhead{}
\renewcommand{\headrulewidth}{0pt}
\setlength{\headheight}{16pt}

% Math
\usepackage{mathtools}
\usepackage{amssymb}
\usepackage{faktor}
\usepackage{import}
\usepackage{caption}
\usepackage{subcaption}
\usepackage{wrapfig}
\usepackage{enumitem}
\setlist{topsep=6pt}
\setlist[enumerate,1]{label=(\arabic*)}

\usepackage{tikz}
\usetikzlibrary{cd,positioning,babel,shapes}
\usepackage{tkz-base}
\usepackage{tkz-euclide}
\tikzset{
  vertex/.style = {shape=circle,fill,text=white,minimum size=9pt,inner
  sep=1pt}
}

% Theorems
\usepackage[thmmarks, amsmath, thref]{ntheorem}
\usepackage{thmtools}

\theoremsymbol{\ensuremath{\blacksquare}}
\newtheorem*{solution}{Possible solution.}

% Title %
\title{\Huge\textsf{Ústní zkouška}\\
 \Large\textsf{z Úvodu do matematické analýzy, části prvé}\\
 \vspace*{1em}
 Verze: oMegareKt\\
 \author{Přednášející: His Divine Wisdom Sir Adam Clypatch}
 \date{16. února 2024}
}

% Table of Contents %
\usepackage{hyperref}
\hypersetup{
 colorlinks=true,
 linktoc=all,
 linkcolor=blue
}

% Tables %
\usepackage{booktabs}
\usepackage{tabularx}

% Patch for hyphens
\usepackage{regexpatch}
\makeatletter
% Change the `-` delimiter to an active character
\xpatchparametertext\@@@cmidrule{-}{\cA-}{}{}
\xpatchparametertext\@cline{-}{\cA-}{}{}
\makeatother

\newcolumntype{s}{>{\centering\arraybackslash}p{.4\textwidth}}

% Operators %
\DeclareMathOperator{\img}{im}
\DeclareMathOperator{\dom}{dom}
\DeclareMathOperator{\codom}{codom}

% Common operators %
\newcommand{\R}{\mathbb{R}}
\newcommand{\N}{\mathbb{N}}
\newcommand{\Z}{\mathbb{Z}}
\newcommand{\Q}{\mathbb{Q}}
\newcommand{\C}{\mathbb{C}}

\newcommand{\clr}{\textcolor{red}}
\newcommand{\clb}{\textcolor{blue}}
\newcommand{\clg}{\textcolor{green}}
\newcommand{\clm}{\textcolor{magenta}}
\newcommand{\clv}{\textcolor{violet}}
\newcommand{\clbr}{\textcolor{Sepia}}

% American Paragraph Skip %
\setlength{\parindent}{0pt}
\setlength{\parskip}{1em}

% Table array stretch
\renewcommand{\arraystretch}{1.5}

% Document %
\pagestyle{fancy}
\begin{document}
 \maketitle
 \begin{tcolorbox}[boxsep=3mm,arc=0mm,toprule=1pt,bottomrule=1pt,leftrule=-0.1mm,
   rightrule=-0.1mm,colframe=red!90!black]
  \vspace*{-2pt}
  \begin{center}
   \textbf{NENÍ-LI ŘEČENO JINAK, VŠECHNY POJMY A DŮKAZY FORMULUJTE PEČLIVĚ
   S~DŮRAZEM NA FORMÁLNÍ SPRÁVNOST.}
  \end{center}
 \end{tcolorbox}
 \vspace*{\fill}
 \begin{center}
  \begin{tabular}{c|c}
   \textsf{\textbf{Část}} & \textsf{\textbf{Hodnocení}}\\
   \toprule
   \textcolor{CornflowerBlue}{Základní definice} & 0 / 0\\
   \textcolor{Emerald}{Lehké úlohy a důkazy} & \hspace{2ex}/ 6\\
   \textcolor{BrickRed}{Těžké ulohy a důkazy} & \hspace{2ex} / 12
  \end{tabular}
 \end{center}
 \vspace*{\fill}
 \clearpage
 \begin{tcolorbox}[title=\textsf{Základní
   definice (0 bodů)},arc=0mm,boxsep=3mm,bottomrule=1pt,toprule=3pt,leftrule=-0.1mm,
   rightrule=-0.1mm,colframe=CornflowerBlue!80!white,
   colback=CornflowerBlue!5!white]
  \emph{Neznalost základních definic znamená bezpodmínečné nesložení
   zkoušky.}
  \begin{enumerate}
   \item Monoid.
   \item Limita posloupnosti.
   \item Monotónní posloupnost.
   \item Minimum a maximum.
   \item Interval a typy intervalů.
  \end{enumerate}
 \end{tcolorbox}
 \clearpage

 \begin{tcolorbox}[title=\textsf{Lehké úlohy a důkazy (6
  bodů)},arc=0mm,boxsep=3mm,bottomrule=1pt,toprule=3pt,leftrule=-0.1mm,
  rightrule=-0.1mm,colframe=Emerald!80!white,colback=Emerald!5!white]
  \emph{Pojmy užité v úlohách nemusíte definovat. Používáte-li k řešení úlohy
  nebo k důkazu předchozí tvrzení, zformulujte je.}
  \begin{enumerate}
   \item Uvažme množinu $\R$ jako množinu tříd ekvivalence $ \simeq $ všech
    konvergentních racionálních posloupností. Dokažte, že zobrazení
    \begin{align*}
     \xi: \Q &\hookrightarrow \R,\\
     q & \mapsto [(q)]_{ \simeq }
    \end{align*}
    které racionálnímu číslu přiřadí třídu ekvivalence posloupnosti samých čísel
    $q$, je prosté.
   \item Dokažte, že jsou-li posloupnosti $a,b:\N \to \R$ konvergentní, pak je
    konvergentní i posloupnost $a + b$.
   \item Spočtěte
    \[
     \lim_{n \to \infty} \frac{3n^{4} - 7n^2 + 5}{6 - 4n^{4}}.
    \]
  \end{enumerate}
 \end{tcolorbox}
 \clearpage
 \begin{tcolorbox}[breakable,title=\textsf{Těžké úlohy a důkazy (12
  bodů)},arc=0mm,boxsep=3mm,bottomrule=1pt,toprule=1pt,leftrule=-0.1mm,
  rightrule=-0.1mm,colframe=BrickRed!80!white,colback=BrickRed!5!white]
  \emph{Nemusíte dokonale zformulovat svá řešení. Obecná idea rozvinutá
  důležitými detaily postačuje.}
  \begin{enumerate}
   \item Uvažte posloupnost danou rekurzivním předpisem
    \begin{align*}
     a_1 &= 2,\\
     a_{n+1} &= \frac{a_n}{2} + \frac{1}{2a_n}.
    \end{align*}
    Spočtěte $\lim_{n \to \infty} a_n$.
   
    \textbf{Návod:}
    \begin{enumerate}
     \item Dokažte, že $a_n$ je klesající a zdola omezená.
     \item Podle věty o limitě monotónní posloupnosti má $a_n$ limitu, označme
      ji $L$. Využijte vzorce pro $a_{n+1}$ (v závislosti na $a_n$) a faktu, že
      $\lim_{n \to \infty} a_{n+1} = \lim_{n \to \infty} a_n$ pro nalezení
      rovnice k výpočtu $L$.
    \end{enumerate}
   \item Nechť $a:\N \to \R$ je posloupnost a $A \in \R$. Dokažte, že $\lim_{n
    \to \infty} a_n = A$ právě tehdy, když $|a_n - A| \geq \varepsilon$ jen pro
    konečně mnoho $n \in \N$, tj. když je množina $\{n \in \N \mid |a_n-A| \geq
    \varepsilon\}$ konečná.
   
    \textbf{Návod:}
    \begin{enumerate}
     \item K důkazu implikace $ \Leftarrow $ využijte toho, že každá neprázdná
      konečná množina má maximum.
     \item K důkazu $ \Rightarrow $ stačí definice limity.
    \end{enumerate}
  \end{enumerate}
 \end{tcolorbox}
\end{document}
