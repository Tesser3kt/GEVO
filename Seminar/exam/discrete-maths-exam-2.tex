\documentclass[a4paper,11pt]{article}

\usepackage[czech,english]{babel}
% Fonts %
\usepackage{fouriernc}
\usepackage[T1]{fontenc}

% Colors %
\usepackage[dvipsnames]{xcolor}

% Colored boxes %
\usepackage[many]{tcolorbox}

% Page Layout %
\usepackage[margin=1in]{geometry}

% Fancy Headers %
\usepackage{fancyhdr}
\fancyhf{}
\cfoot{\thepage}
\rhead{}
\renewcommand{\headrulewidth}{0pt}
\setlength{\headheight}{16pt}

% Math
\usepackage{mathtools}
\usepackage{amssymb}
\usepackage{faktor}
\usepackage{import}
\usepackage{caption}
\usepackage{subcaption}
\usepackage{wrapfig}
\usepackage{enumitem}
\setlist{topsep=6pt}
\setlist[enumerate,1]{label=(\arabic*)}

\usepackage{tikz}
\usetikzlibrary{cd,positioning,babel,shapes}
\usepackage{tkz-base}
\usepackage{tkz-euclide}
\tikzset{
  vertex/.style = {shape=circle,fill,text=white,minimum size=9pt,inner
  sep=1pt}
}

% Theorems
\usepackage[thmmarks, amsmath, thref]{ntheorem}
\usepackage{thmtools}

\theoremsymbol{\ensuremath{\blacksquare}}
\newtheorem*{solution}{Possible solution.}

% Title %
\title{\Huge\textsf{Ústní zkouška}\\
 \Large\textsf{z Úvodu do diskrétní matematiky}\\
 \vspace*{1em}
 Verze: ez clap\\
 \author{Přednášející: His Divine Benevolence Sir Adam Clypatch}
 \date{1. června 2023}
}

% Table of Contents %
\usepackage{hyperref}
\hypersetup{
 colorlinks=true,
 linktoc=all,
 linkcolor=blue
}

% Tables %
\usepackage{booktabs}
\usepackage{tabularx}

% Patch for hyphens
\usepackage{regexpatch}
\makeatletter
% Change the `-` delimiter to an active character
\xpatchparametertext\@@@cmidrule{-}{\cA-}{}{}
\xpatchparametertext\@cline{-}{\cA-}{}{}
\makeatother

\newcolumntype{s}{>{\centering\arraybackslash}p{.4\textwidth}}

% Operators %
\DeclareMathOperator{\img}{im}
\DeclareMathOperator{\dom}{dom}
\DeclareMathOperator{\codom}{codom}

% Common operators %
\newcommand{\R}{\mathbb{R}}
\newcommand{\N}{\mathbb{N}}
\newcommand{\Z}{\mathbb{Z}}
\newcommand{\Q}{\mathbb{Q}}
\newcommand{\C}{\mathbb{C}}

\newcommand{\clr}{\textcolor{red}}
\newcommand{\clb}{\textcolor{blue}}
\newcommand{\clg}{\textcolor{green}}
\newcommand{\clm}{\textcolor{magenta}}
\newcommand{\clv}{\textcolor{violet}}
\newcommand{\clbr}{\textcolor{Sepia}}

% American Paragraph Skip %
\setlength{\parindent}{0pt}
\setlength{\parskip}{1em}

% Table array stretch
\renewcommand{\arraystretch}{1.5}

% Document %
\pagestyle{fancy}
\begin{document}
 \maketitle
 \begin{tcolorbox}[boxsep=3mm,arc=0mm,toprule=1pt,bottomrule=1pt,leftrule=-0.1mm,
   rightrule=-0.1mm,colframe=red!90!black]
  \vspace*{-2pt}
  \begin{center}
   \textbf{NENÍ-LI ŘEČENO JINAK, VŠECHNY POJMY A DŮKAZY FORMULUJTE PEČLIVĚ
   S~DŮRAZEM NA FORMÁLNÍ SPRÁVNOST.}
  \end{center}
 \end{tcolorbox}
 \vspace*{\fill}
 \begin{center}
  \begin{tabular}{c|c}
   \textsf{\textbf{Část}} & \textsf{\textbf{Hodnocení}}\\
   \toprule
   \textcolor{CornflowerBlue}{Základní definice} & 0 / 0\\
   \textcolor{Emerald}{Lehké úlohy a důkazy} & \hspace{2ex}/ 6\\
   \textcolor{BrickRed}{Těžké ulohy a důkazy} & \hspace{2ex} / 15
  \end{tabular}
 \end{center}
 \vspace*{\fill}
 \clearpage
 \begin{tcolorbox}[title=\textsf{Základní
   definice (0 bodů)},arc=0mm,boxsep=3mm,bottomrule=1pt,toprule=1pt,leftrule=-0.1mm,
   rightrule=-0.1mm,colframe=CornflowerBlue!80!white,
   colback=CornflowerBlue!5!white]
  \emph{Neznalost základních definic znamená bezpodmínečné nesložení
  zkoušky.}
  \begin{enumerate}
   \item Logický výrok.
   \item Sjednocení, průnik a rozdíl dvou libovolných množin $A,B$ užitím
    logických spojek a kvantifikátorů.
   \item Sjednocení $n \in \N$ libovolných množin $A_i$, kde $i \in
    \{1,\ldots,n\}$, užitím logických spojek a kvantifikátorů.
   \item Konečná množina a velikost konečné množiny.
   \item Ekvivalence a třída ekvivalence (relaci není třeba definovat).
   \item Zobrazení (opět, relaci není třeba definovat).
   \item Kodoména a doména zobrazení. Vzor a obraz prvku při zobrazení.
   \item Složení zobrazení.
   \item Permutace, řád permutace. \textbf{Neformálně} cyklický zápis permutace.
   \item Kombinační číslo.
   \item Graf (libovolná z definic).
   \item Sled, tah a cesta v grafu (opět, libovolné z definic).
   \item Souvislý graf, ohodnocený graf, strom.
   \item Vzdálenost vrcholů v grafu, EFLP a SFLP.
  \end{enumerate}
 \end{tcolorbox}
 \clearpage

 \begin{tcolorbox}[title=\textsf{Lehké úlohy a důkazy (6
  bodů)},arc=0mm,boxsep=3mm,bottomrule=1pt,toprule=1pt,leftrule=-0.1mm,
  rightrule=-0.1mm,colframe=Emerald!80!white,colback=Emerald!5!white]
  \emph{Pojmy užité v úlohách nemusíte definovat. Používáte-li k řešení úlohy
  nebo k důkazu předchozí tvrzení, zformulujte je.}
  \begin{enumerate}
   \item O zobrazení $f: A \to B$ řekneme, že je \emph{na}, když platí
    následující výrok:
    \[
     \forall b \in B \; \exists a \in A: f(a) = b.
    \]
    Dokažte, že $f$ je \textbf{na} právě tehdy, když $\img f = \codom f$.
   \item Ať $A,B,C$ jsou libovolné množiny. Je inkluze
   \[
    (A \setminus B) \setminus C \subseteq A \setminus (B \cap C)
   \]
   vždy platná? Pokud ano, dokažte. Pokud ne, najděte protipříklad.
  \item Zformulujte důkaz, že pro každé dvě \textbf{konečné} množiny $A,B$ platí
   \[
    \# (A \times B) = \# A \cdot \# B.
   \]
  \item Ve třídě 4.A je 27 studentů. Rozhodli se poškádlit svého třídního a
   náhodným losováním změnit svůj obvyklý zasedací pořádek. Spočtěte
   pravděpodobnost, že si přesně 7 žáků vylosovalo své obvyklé místo.
  \item Ať $G = (V,E)$ je graf. Dokažte, že v $G$ existuje \textbf{cesta} mezi
   $u$ a $v$ právě tehdy, když v $G$ existuje \textbf{sled} mezi $u$ a $v$ pro
   libovolné dva vrcholy $u,v \in V$.
  \item Užitím Dijkstrova algoritmu, nalezněte vzdálenosti všech vrcholů od
   počátečního vrcholu $\textcolor{BrickRed}{s}$ v ohodnoceném grafu určeném
   následujícím obrázkem.
   \begin{center}
    \begin{tikzpicture}[scale=1.5]
     \node[vertex,BrickRed] (s) at (0,0) {};
     \node[vertex] (v11) at (1,1) {};
     \node[vertex] (v12) at (1,-1) {};
     \node[vertex] (v21) at (2,0) {};
     \node[vertex] (v31) at (3,1) {};
     \node[vertex] (v32) at (3,-1) {};
     \node[vertex] (v41) at (4,0) {};

     \node[left=0mm of s,BrickRed] {\Large$s$};

     \draw[thick] (s) to node[midway,circle,draw,fill=white,inner sep=1pt]
      {\footnotesize $2$} (v11);
     \draw[thick] (v11) to node[midway,circle,draw,fill=white,inner sep=1pt]
      {\footnotesize $1$} (v31);
     \draw[thick] (s) to node[midway,circle,draw,fill=white,inner sep=1pt]
      {\footnotesize $1$} (v12);
     \draw[thick] (v11) to node[midway,circle,draw,fill=white,inner sep=1pt]
      {\footnotesize $2$} (v21);
     \draw[thick] (v12) to node[midway,circle,draw,fill=white,inner sep=1pt]
      {\footnotesize $2$} (v21);
     \draw[thick] (v12) to node[midway,circle,draw,fill=white,inner sep=1pt]
      {\footnotesize $6$} (v32);
     \draw[thick] (v21) to node[midway,circle,draw,fill=white,inner sep=1pt]
      {\footnotesize $1$} (v31);
     \draw[thick] (v21) to node[midway,circle,draw,fill=white,inner sep=1pt]
      {\footnotesize $3$} (v32);
     \draw[thick] (v21) to node[midway,circle,draw,fill=white,inner sep=1pt]
      {\footnotesize $3$} (v41);
     \draw[thick] (v31) to node[midway,circle,draw,fill=white,inner sep=1pt]
      {\footnotesize $2$} (v41);
     \draw[thick] (v32) to node[midway,circle,draw,fill=white,inner sep=1pt]
      {\footnotesize $1$} (v41);
    \end{tikzpicture}
   \end{center}
 \end{enumerate}
 \end{tcolorbox}
 \clearpage
 \begin{tcolorbox}[breakable,title=\textsf{Těžké úlohy a důkazy (15
  bodů)},arc=0mm,boxsep=3mm,bottomrule=1pt,toprule=1pt,leftrule=-0.1mm,
  rightrule=-0.1mm,colframe=BrickRed!80!white,colback=BrickRed!5!white]
  \emph{Nemusíte dokonale zformulovat svá řešení. Obecná idea rozvinutá
  důležitými detaily postačuje.}
  \begin{enumerate}
   \item \emph{Krychlí} dimenze $n \in \N$ myslíme graf na $2^{n}$ vrcholech
    číslovaných binárními čísly od $0$ do $2^{n} - 1$. Budeme předpokládat, že
    každé binární číslo je doplněno nulami na $n$ cifer, tedy například nultý
    vrchol krychle dimenze $5$ je binární číslo $00000$, první je $00001$ atd.
    
    Vzdáleností, značenou písmenem $d$, mezi binárními čísly nazveme počet míst,
    kde se jejich cifry liší. Tedy například $d(00100, 11100) = 2$ a
    $d(01100,10010) = 4$. Hrana vede mezi dvěma vrcholy $v$ a $w$ právě tehdy,
    když $d(v,w) = 1$.
   
    Dokažte indukcí, že počet hran krychle dimenze $n$ je přesně $n \cdot
    2^{n-1}$.

    Jen pro zajímavost (tedy, stačí se zamyslet, ne dokazovat), odpovídá
    vzdálenost binárních čísel vzdálenosti mezi příslušnými vrcholy?
   \item (jiný důkaz principu inkluze a exkluze) Ať $A,B$ jsou množiny a $A
    \subseteq B$. \emph{Charakteristickou funkcí} množiny $A$ myslíme zobrazení
    \[
     \chi_A: B \to \{0,1\}
    \]
    takové, že $\chi_A(x) = 1$, když $x \in A$, a $\chi_A(x) = 0$, když $x \notin
    A$.
   
    Platí (to \textbf{nedokazujte}) následující vzorec:
    \begin{equation*}
     \label{eq:poly-identity}
     \tag{$*$}
     \prod_{i=1}^n (1 + x_i) = \sum_{I \subseteq \{1,2,\ldots,n\}} \left(
     \prod_{i \in I} x_i \right).
    \end{equation*}
    Ať nyní $A_i$, $i \in \{1,2,\ldots,n\}$, jsou konečné množiny a položme
    \[
     A \coloneqq \bigcup_{i = 1}^{n} A_i.
    \]
    Charakteristickou funkci množiny $A_i$ označíme jednoduše jako $\chi_i$.
    \begin{enumerate}[label=(\alph*)]
     \item Dokažte, že pro každé $a \in A$ platí
      \[
       \prod_{i=1}^n (1-\chi_i(a)) = 0.
      \] 
     \item Dosaďte $x_i = -\chi_i(a)$ do vzorce \eqref{eq:poly-identity} a
      výslednou sumu napište.
     \item Užitím výsledků z (a) a (b) dokažte, že
      \[
       \sum_{I \subseteq \{1,2,\ldots,n\}} (-1)^{\# I}\left( \sum_{a \in A}
       \prod_{i \in I} \chi_i(a) \right) = 0.
      \]
      \textbf{Hint:} Sčítejte výsledek z (b) přes všechny prvky $a \in A$ a pak
      prohoďte pořadí sumace.
     \item Dokažte, že $\prod_{i \in I} \chi_i$ je charakteristická funkce
      množiny $\bigcap_{i \in I} A_i$.
     \item Užitím (d), dokažte, že
      \[
       \sum_{a \in A} \prod_{i \in I} \chi_i(a) = \# \left( \bigcap_{i \in I}
       A_i \right).
      \]
     \item Speciálně, pro $I = \emptyset$ je $\prod_{i \in \emptyset}
      \chi_i(a)$ prázdný součin, a tedy má hodnotu $1$. Pročež platí
      \[
       \sum_{a \in A} \prod_{i \in \emptyset} \chi_i(a) = \sum_{a \in A} 1 = \#
       A.
      \]
      Dosaďte do (c) výsledek z (e) a rozložte na základě výpočtu výše vhodným
      způsobem vzniklou sumu, abyste dokázali princip inkluze a exkluze.
    \end{enumerate}
   \item Zformulujte Kruskalův algoritmus a dokažte jeho správnost.
  \end{enumerate}
 \end{tcolorbox}
\end{document}
