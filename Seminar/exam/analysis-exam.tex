\documentclass[a4paper,11pt]{article}

\usepackage[czech,english]{babel}
% Fonts %
\usepackage{fouriernc}
\usepackage[T1]{fontenc}

% Colors %
\usepackage[dvipsnames]{xcolor}

% Colored boxes %
\usepackage[many]{tcolorbox}

% Page Layout %
\usepackage[margin=1in]{geometry}

% Fancy Headers %
\usepackage{fancyhdr}
\fancyhf{}
\cfoot{\thepage}
\rhead{}
\renewcommand{\headrulewidth}{0pt}
\setlength{\headheight}{16pt}

% Math
\usepackage{mathtools}
\usepackage{amssymb}
\usepackage{faktor}
\usepackage{import}
\usepackage{caption}
\usepackage{subcaption}
\usepackage{wrapfig}
\usepackage{enumitem}
\setlist{topsep=6pt}
\setlist[enumerate,1]{label=(\arabic*)}

\usepackage{tikz}
\usetikzlibrary{cd,positioning,babel,shapes}
\usepackage{tkz-base}
\usepackage{tkz-euclide}
\tikzset{
  vertex/.style = {shape=circle,fill,text=white,minimum size=9pt,inner
  sep=1pt}
}

% Theorems
\usepackage[thmmarks, amsmath, thref]{ntheorem}
\usepackage{thmtools}

\theoremsymbol{\ensuremath{\blacksquare}}
\newtheorem*{solution}{Possible solution.}

% Title %
\title{\Huge\textsf{Ústní zkouška}\\
 \Large\textsf{z Úvodu do matematické analýzy, části prvé}\\
 \vspace*{1em}
 Verze: reKt\\
 \author{Přednášející: His Divine Wisdom Sir Adam Clypatch}
 \date{19. ledna 2024}
}

% Table of Contents %
\usepackage{hyperref}
\hypersetup{
 colorlinks=true,
 linktoc=all,
 linkcolor=blue
}

% Tables %
\usepackage{booktabs}
\usepackage{tabularx}

% Patch for hyphens
\usepackage{regexpatch}
\makeatletter
% Change the `-` delimiter to an active character
\xpatchparametertext\@@@cmidrule{-}{\cA-}{}{}
\xpatchparametertext\@cline{-}{\cA-}{}{}
\makeatother

\newcolumntype{s}{>{\centering\arraybackslash}p{.4\textwidth}}

% Operators %
\DeclareMathOperator{\img}{im}
\DeclareMathOperator{\dom}{dom}
\DeclareMathOperator{\codom}{codom}

% Common operators %
\newcommand{\R}{\mathbb{R}}
\newcommand{\N}{\mathbb{N}}
\newcommand{\Z}{\mathbb{Z}}
\newcommand{\Q}{\mathbb{Q}}
\newcommand{\C}{\mathbb{C}}

\newcommand{\clr}{\textcolor{red}}
\newcommand{\clb}{\textcolor{blue}}
\newcommand{\clg}{\textcolor{green}}
\newcommand{\clm}{\textcolor{magenta}}
\newcommand{\clv}{\textcolor{violet}}
\newcommand{\clbr}{\textcolor{Sepia}}

% American Paragraph Skip %
\setlength{\parindent}{0pt}
\setlength{\parskip}{1em}

% Table array stretch
\renewcommand{\arraystretch}{1.5}

% Document %
\pagestyle{fancy}
\begin{document}
 \maketitle
 \begin{tcolorbox}[boxsep=3mm,arc=0mm,toprule=1pt,bottomrule=1pt,leftrule=-0.1mm,
   rightrule=-0.1mm,colframe=red!90!black]
  \vspace*{-2pt}
  \begin{center}
   \textbf{NENÍ-LI ŘEČENO JINAK, VŠECHNY POJMY A DŮKAZY FORMULUJTE PEČLIVĚ
   S~DŮRAZEM NA FORMÁLNÍ SPRÁVNOST.}
  \end{center}
 \end{tcolorbox}
 \vspace*{\fill}
 \begin{center}
  \begin{tabular}{c|c}
   \textsf{\textbf{Část}} & \textsf{\textbf{Hodnocení}}\\
   \toprule
   \textcolor{CornflowerBlue}{Základní definice} & 0 / 0\\
   \textcolor{Emerald}{Lehké úlohy a důkazy} & \hspace{2ex}/ 6\\
   \textcolor{BrickRed}{Těžké ulohy a důkazy} & \hspace{2ex} / 12
  \end{tabular}
 \end{center}
 \vspace*{\fill}
 \clearpage
 \begin{tcolorbox}[title=\textsf{Základní
   definice (0 bodů)},arc=0mm,boxsep=3mm,bottomrule=1pt,toprule=3pt,leftrule=-0.1mm,
   rightrule=-0.1mm,colframe=CornflowerBlue!80!white,
   colback=CornflowerBlue!5!white]
  \emph{Neznalost základních definic znamená bezpodmínečné nesložení
  zkoušky.}
  \begin{enumerate}
   \item Konvergentní racionální posloupnost (včetně definice racionální
    posloupnosti).
   \item Reálné číslo. Vysvětlete též, v jakém smyslu jsou $\Q$ podmnožinou
    $\R$.
   \item Celé číslo.
   \item Limita posloupnosti.
   \item Supremum a infimum.
  \end{enumerate}
 \end{tcolorbox}
 \clearpage

 \begin{tcolorbox}[title=\textsf{Lehké úlohy a důkazy (6
  bodů)},arc=0mm,boxsep=3mm,bottomrule=1pt,toprule=3pt,leftrule=-0.1mm,
  rightrule=-0.1mm,colframe=Emerald!80!white,colback=Emerald!5!white]
  \emph{Pojmy užité v úlohách nemusíte definovat. Používáte-li k řešení úlohy
  nebo k důkazu předchozí tvrzení, zformulujte je.}
  \begin{enumerate}
   \item Dokažte, že relace $ \sim $ na $\Z \times \Z \setminus \{0\}$ daná
    předpisem
    \[
     (a,b) \sim (c,d) \overset{def.}{\Longleftrightarrow} a \cdot d = b \cdot c
    \]
    je ekvivalence a že operace $+$ a $ \cdot $ na třídách ekvivalence $ \sim $
    dané předpisy
    \begin{align*}
     [(a,b)]_{ \sim } + [(c,d)]_{ \sim } &\coloneqq [(a \cdot d + b \cdot c, b
     \cdot d)]_{ \sim },\\
     [(a,b)]_{ \sim } \cdot [(c,d)]_{ \sim } & \coloneqq [(a \cdot c,b \cdot
     d)]_{ \sim }
    \end{align*}
    jsou dobře definované.
   \item Dokažte, že každá konvergentní posloupnost (reálných čísel) je
    omezená.
   \item Spočtěte
    \[
     \lim_{n \to \infty} \sqrt{\frac{9 + n^2}{4n^2}}.
    \]
    Uveďte všechna tvrzení, jež používáte, a ověřte jejich předpoklady.
  \end{enumerate}
 \end{tcolorbox}
 \clearpage
 \begin{tcolorbox}[breakable,title=\textsf{Těžké úlohy a důkazy (12
  bodů)},arc=0mm,boxsep=3mm,bottomrule=1pt,toprule=1pt,leftrule=-0.1mm,
  rightrule=-0.1mm,colframe=BrickRed!80!white,colback=BrickRed!5!white]
  \emph{Nemusíte dokonale zformulovat svá řešení. Obecná idea rozvinutá
  důležitými detaily postačuje.}
  \begin{enumerate}
   \item Alternativní důkaz Bolzanovy-Weierstraßovy věty.
   \begin{enumerate}
    \item Dokažte, že každá posloupnost $a:\N \to \R$ má \emph{monotónní}
     podposloupnost.
    \begin{enumerate}
     \item Předpokládejte nejprve, že pro každé $m \in \N$ má množina $\{a_n
      \mid n \geq m\}$ maximum. Využijte tohoto předpokladu k sestrojení
      nerostoucí posloupnosti $b:\N \to \R$ jako posloupnosti maxim stále
      menších podmnožin prvků posloupnosti $a$. Zkuste to \emph{induktivně}.
     \item Nyní naopak předpokládejte, že existuje $m \in \N$, pro které množina
      $\{a_n \mid n \geq m\}$ maximum nemá. V tomto případě rovněž $\{a_n \mid
      n \geq m'\}$ nemá maximum pro všechna $m' \geq m$. Induktivní konstrukcí
      velmi obdobnou té z~bodu i. sestrojte podposloupnost $b$ posloupnosti $a$,
      jež je rostoucí.
    \end{enumerate}
   \item Dokažte Bolzanovu-Weierstraßovu větu (tedy tvrzení, že každá omezená
    posloupnost má konvergentní podposloupnost) užitím bodu (a) a tvrzení, že
    každá monotónní omezená posloupnost je konvergentní.
   \end{enumerate}
  \item Posloupnost $a:\N \to \R$ je zadána rekurentně vztahy
   \begin{align*}
    a_1 & \coloneqq 1,\\
    a_{n+1} & \coloneqq \frac{1}{1+a_n}.
   \end{align*}
   Spočtěte $\lim a$. Návod:
   \begin{enumerate}
    \item Dokažte, že všechny členy $a$ jsou dobře definovány.
    \item Definujme funkce
     \[
      f(x) \coloneqq \frac{1}{1+x}, \quad g(x) \coloneqq (f \circ f)(x).
     \]
     Dokažte, že
     \begin{itemize}
      \item rovnice $g(x) = x$ má na intervalu $[0,1]$ přesně jedno řešení.
       Označme je $c$.
      \item platí $x < g(x) < c$ pro $x \in [0,c)$ a $c < g(x) < x$ pro $x \in
       (c,1]$.
      \item platí $a_2 < c < a_1$ a $g(a_k) = a_{k+2}$.
      \item podposloupnost lichých členů $a$ je klesající a zdola omezená a
       podposloupnost sudých členů je rostoucí a shora omezená.
     \end{itemize}
    \item Podle bodu (b) jsou posloupnosti $(a_{2k})_{k=1}^{\infty}$ a
     $(a_{2k+1})_{k=1}^{\infty}$ monotónní a omezené, tudíž mají limitu. Označme
     \[
      A \coloneqq \lim_{k \to \infty} a_{2k}, \quad B \coloneqq \lim_{k \to
      \infty} a_{2k+1}.
     \]
     Dokažte, že $g(A) = A$ a $g(B) = B$.
    \item Odvoďte, že z bodu (c) plyne, že $\lim a = c$.
   \end{enumerate}
  \end{enumerate}
 \end{tcolorbox}
\end{document}
