\documentclass[a4paper,11pt]{article}

\usepackage[czech,english]{babel}
% Fonts %
\usepackage{fouriernc}
\usepackage[T1]{fontenc}

% Colors %
\usepackage[dvipsnames]{xcolor}

% Colored boxes %
\usepackage[many]{tcolorbox}

% Page Layout %
\usepackage[margin=1in]{geometry}

% Fancy Headers %
\usepackage{fancyhdr}
\fancyhf{}
\cfoot{\thepage}
\rhead{}
\renewcommand{\headrulewidth}{0pt}
\setlength{\headheight}{16pt}

% Math
\usepackage{mathtools}
\usepackage{amssymb}
\usepackage{faktor}
\usepackage{import}
\usepackage{caption}
\usepackage{subcaption}
\usepackage{wrapfig}
\usepackage{enumitem}
\setlist{topsep=6pt}
\setlist[enumerate,1]{label=(\arabic*)}

\usepackage{tikz}
\usetikzlibrary{cd,positioning,babel,shapes}
\usepackage{tkz-base}
\usepackage{tkz-euclide}
\tikzset{
  vertex/.style = {shape=circle,fill,text=white,minimum size=9pt,inner
  sep=1pt}
}

% Theorems
\usepackage[thmmarks, amsmath, thref]{ntheorem}
\usepackage{thmtools}

\theoremsymbol{\ensuremath{\blacksquare}}
\newtheorem*{solution}{Possible solution.}

% Title %
\title{\Huge\textsf{Ústní zkouška}\\
 \Large\textsf{z Úvodu do matematické analýzy, části prvé}\\
 \vspace*{1em}
 Verze: r3kt\\
 \author{Přednášející: His Divine Wisdom Sir Adam Clypatch}
 \date{19. ledna 2024}
}

% Table of Contents %
\usepackage{hyperref}
\hypersetup{
 colorlinks=true,
 linktoc=all,
 linkcolor=blue
}

% Tables %
\usepackage{booktabs}
\usepackage{tabularx}

% Patch for hyphens
\usepackage{regexpatch}
\makeatletter
% Change the `-` delimiter to an active character
\xpatchparametertext\@@@cmidrule{-}{\cA-}{}{}
\xpatchparametertext\@cline{-}{\cA-}{}{}
\makeatother

\newcolumntype{s}{>{\centering\arraybackslash}p{.4\textwidth}}

% Operators %
\DeclareMathOperator{\img}{im}
\DeclareMathOperator{\dom}{dom}
\DeclareMathOperator{\codom}{codom}

% Common operators %
\newcommand{\R}{\mathbb{R}}
\newcommand{\N}{\mathbb{N}}
\newcommand{\Z}{\mathbb{Z}}
\newcommand{\Q}{\mathbb{Q}}
\newcommand{\C}{\mathbb{C}}

\newcommand{\clr}{\textcolor{red}}
\newcommand{\clb}{\textcolor{blue}}
\newcommand{\clg}{\textcolor{green}}
\newcommand{\clm}{\textcolor{magenta}}
\newcommand{\clv}{\textcolor{violet}}
\newcommand{\clbr}{\textcolor{Sepia}}

% American Paragraph Skip %
\setlength{\parindent}{0pt}
\setlength{\parskip}{1em}

% Table array stretch
\renewcommand{\arraystretch}{1.5}

% Document %
\pagestyle{fancy}
\begin{document}
 \maketitle
 \begin{tcolorbox}[boxsep=3mm,arc=0mm,toprule=1pt,bottomrule=1pt,leftrule=-0.1mm,
   rightrule=-0.1mm,colframe=red!90!black]
  \vspace*{-2pt}
  \begin{center}
   \textbf{NENÍ-LI ŘEČENO JINAK, VŠECHNY POJMY A DŮKAZY FORMULUJTE PEČLIVĚ
   S~DŮRAZEM NA FORMÁLNÍ SPRÁVNOST.}
  \end{center}
 \end{tcolorbox}
 \vspace*{\fill}
 \begin{center}
  \begin{tabular}{c|c}
   \textsf{\textbf{Část}} & \textsf{\textbf{Hodnocení}}\\
   \toprule
   \textcolor{CornflowerBlue}{Základní definice} & 0 / 0\\
   \textcolor{Emerald}{Lehké úlohy a důkazy} & \hspace{2ex}/ 6\\
   \textcolor{BrickRed}{Těžké ulohy a důkazy} & \hspace{2ex} / 12
  \end{tabular}
 \end{center}
 \vspace*{\fill}
 \clearpage
 \begin{tcolorbox}[title=\textsf{Základní
   definice (0 bodů)},arc=0mm,boxsep=3mm,bottomrule=1pt,toprule=3pt,leftrule=-0.1mm,
   rightrule=-0.1mm,colframe=CornflowerBlue!80!white,
   colback=CornflowerBlue!5!white]
  \emph{Neznalost základních definic znamená bezpodmínečné nesložení
  zkoušky.}
  \begin{enumerate}
   \item Okruh a těleso.
   \item Racionální číslo.
   \item Konvergentní posloupnost.
   \item Rozšířená reálná osa.
   \item Interval a typy intervalů.
  \end{enumerate}
 \end{tcolorbox}
 \clearpage

 \begin{tcolorbox}[title=\textsf{Lehké úlohy a důkazy (6
  bodů)},arc=0mm,boxsep=3mm,bottomrule=1pt,toprule=3pt,leftrule=-0.1mm,
  rightrule=-0.1mm,colframe=Emerald!80!white,colback=Emerald!5!white]
  \emph{Pojmy užité v úlohách nemusíte definovat. Používáte-li k řešení úlohy
  nebo k důkazu předchozí tvrzení, zformulujte je.}
  \begin{enumerate}
   \item Dokažte, že každá posloupnost má \emph{nejvýše} jednu limitu.
   \item Dokažte, že $\Q$ jsou hustá v $\R$, tedy že pro každé $\varepsilon>0$ a
    každé $x \in \R$ existuje $r \in \Q$ splňující $|x - r|<\varepsilon$.\\
    \textbf{Hint:} Využijte definici $\R$ jako tříd ekvivalence konvergentních
    racionálních posloupností.
   \item Spočtěte
   \[
    \lim_{n \to \infty} \left( -\frac{1}{2} \right)^{n}.
   \]
   Ověřte předpoklady všech tvrzení, která k výpočtu používáte.
  \end{enumerate}
 \end{tcolorbox}
 \clearpage
 \begin{tcolorbox}[breakable,title=\textsf{Těžké úlohy a důkazy (12
  bodů)},arc=0mm,boxsep=3mm,bottomrule=1pt,toprule=1pt,leftrule=-0.1mm,
  rightrule=-0.1mm,colframe=BrickRed!80!white,colback=BrickRed!5!white]
  \emph{Nemusíte dokonale zformulovat svá řešení. Obecná idea rozvinutá
  důležitými detaily postačuje.}
  \begin{enumerate}
   \item Ať $c > 0$ a $a:\N \to \R$ je posloupnost dána rekurentním vztahem
   \begin{align*}
    a_0 & \coloneqq \sqrt{c},\\
    a_{n+1} & \coloneqq \sqrt{a_n + c}.
   \end{align*}
   Spočtěte $\lim a$. Návod:
   \begin{enumerate}
    \item Dokažte, že posloupnost $a$ je dobře definovaná. To znamená, že $a_n
     \in \R$ pro všechna $n \in \N$.
    \item Dokažte, že $a$ je rostoucí.
    \item Dokažte, že $a$ je shora omezená.
    \item Z bodů (b) a (c) plyne (užitím věty o limitě monotónní posloupnosti),
     že $a$ má limitu. Označme ji $A$. Spočtěte tuto limitu pomocí vhodné
     kvadratické rovnice využivše rovností
     \[
      a^2_{n+1} = a_n + c, \quad \lim_{n \to \infty} a_{n+1} = A \quad \text{a}
      \quad \lim_{n \to \infty} (a_n + c) = A + c.
     \]
   \end{enumerate}
  \item Dokažte \emph{Borelovu větu}: Ať $I$ je \textbf{uzavřený} interval a
   $\{S_i \mid i \in \N\}$ je množina \textbf{otevřených} intervalů splňující $I
   \subseteq \bigcup_{i \in \N} S_i$. Pak existuje \textbf{konečná} množina $F
   \subseteq \N$ taková, že $I \subseteq \bigcup_{i \in F} S_i$.

   Návod (vřele doporučujeme si při důkaze kreslit):
   \begin{enumerate}
    \item Označme $I = [a,b]$. Ať $M$ je množina takových prvků $x \in [a,b]$,
     pro něž existuje konečná $F_x \subseteq \N$ taková, že $[a,x] \subseteq
     \bigcup_{i \in F_x} S_i$. Uvědomte si, že pro důkaz Borelovy věty stačí
     ukázat, že $b \in M$.
    \item Dokažte, že $M$ má supremum, které leží v $\R$. K tomu je třeba
     ukázat, že není prázdná a že je shora omezená. Existence suprema pak plyne
     z úplnosti $\R$. Položme $y \coloneqq \sup M$. Nyní je třeba ukázat, že
     $y \in M$ a $y = b$.
    \item Z předpokladu $y \in I \subseteq \bigcup_{i \in \N} S_i$, a tedy
     existuje \textbf{otevřený} interval $S_j$ takový, že $y \in S_j$. Ovšem,
     protože je tento interval otevřený, $y$ nemůže být jeho mi\-nimem. Dokažte,
     že z tohoto plyne existence prvku $z \in M \cap S_j$ (nezapomeňte, že $y$
     je supremem $M$).
    \item Pro tento prvek z definice $M$ existuje konečná množina $F_z$ taková,
     že $[a,z] \subseteq \bigcup_{i \in F_z} S_i$. Dokažte, že
     \[
      x \in \left( \bigcup_{i \in F_z} S_i \right) \cup S_j
     \]
     pro každé $x \in [a,y]$. Argumentujte, že tento fakt již znamená, že $y \in
     M$.
    \item Dokážeme, že $y = b$. Předpokládejme pro spor, že $y < b$. Pak
     existuje $\varepsilon>0$ takové, že $y + \varepsilon < b$, což znamená, že
     umíme najít (proč?) otevřený interval $S_k$ takový, že $[y,y+\varepsilon]
     \subseteq S_k$. Užitím faktu, že $y \in M$, najděte konečnou množinu
     $\mathcal{S} \subseteq \{S_i \mid i \in \N\}$ otevřených intervalů takovou,
     že $[a,y + \varepsilon] \subseteq \mathcal{S}$ (nezapomeňte, že takové
     množiny umíte triviálně najít zvlášť pro $[a,y]$ a $[y,y+\varepsilon]$).
    \item Argumentujte, že existence $\mathcal{S}$ z bodu (e) je sporem s
     definicí $y$ jako suprema $M$. Tedy, $y = b$ a důkaz je hotov.
   \end{enumerate}
  \end{enumerate}
 \end{tcolorbox}
\end{document}
