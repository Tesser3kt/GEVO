\documentclass[a4paper,11pt]{article}

\usepackage[czech,english]{babel}
% Fonts %
\usepackage{fouriernc}
\usepackage[T1]{fontenc}

% Colors %
\usepackage[dvipsnames]{xcolor}

% Colored boxes %
\usepackage[many]{tcolorbox}

% Page Layout %
\usepackage[margin=1in]{geometry}

% Fancy Headers %
\usepackage{fancyhdr}
\fancyhf{}
\cfoot{\thepage}
\rhead{}
\renewcommand{\headrulewidth}{0pt}
\setlength{\headheight}{16pt}

% Math
\usepackage{mathtools}
\usepackage{amssymb}
\usepackage{faktor}
\usepackage{import}
\usepackage{caption}
\usepackage{subcaption}
\usepackage{wrapfig}
\usepackage{enumitem}
\setlist{topsep=6pt}
\setlist[enumerate,1]{label=(\arabic*)}

\usepackage{tikz}
\usetikzlibrary{cd,positioning,babel,shapes}
\usepackage{tkz-base}
\usepackage{tkz-euclide}
\tikzset{
  vertex/.style = {shape=circle,fill,text=white,minimum size=9pt,inner
  sep=1pt}
}

% Theorems
\usepackage[thmmarks, amsmath, thref]{ntheorem}
\usepackage{thmtools}

\theoremsymbol{\ensuremath{\blacksquare}}
\newtheorem*{solution}{Possible solution.}

% Title %
\title{\Huge\textsf{Ústní zkouška}\\
 \Large\textsf{z Úvodu do diskrétní matematiky}\\
 \vspace*{1em}
 Verze: R.I.P.\\
 \author{Přednášející: His Divine Benevolence Sir Adam Clypatch}
 \date{23. června 2023}
}

% Table of Contents %
\usepackage{hyperref}
\hypersetup{
 colorlinks=true,
 linktoc=all,
 linkcolor=blue
}

% Tables %
\usepackage{booktabs}
\usepackage{tabularx}

% Patch for hyphens
\usepackage{regexpatch}
\makeatletter
% Change the `-` delimiter to an active character
\xpatchparametertext\@@@cmidrule{-}{\cA-}{}{}
\xpatchparametertext\@cline{-}{\cA-}{}{}
\makeatother

\newcolumntype{s}{>{\centering\arraybackslash}p{.4\textwidth}}

% Operators %
\DeclareMathOperator{\img}{im}
\DeclareMathOperator{\dom}{dom}
\DeclareMathOperator{\codom}{codom}

% Common operators %
\newcommand{\R}{\mathbb{R}}
\newcommand{\N}{\mathbb{N}}
\newcommand{\Z}{\mathbb{Z}}
\newcommand{\Q}{\mathbb{Q}}
\newcommand{\C}{\mathbb{C}}

\newcommand{\clr}{\textcolor{red}}
\newcommand{\clb}{\textcolor{blue}}
\newcommand{\clg}{\textcolor{green}}
\newcommand{\clm}{\textcolor{magenta}}
\newcommand{\clv}{\textcolor{violet}}
\newcommand{\clbr}{\textcolor{Sepia}}

% American Paragraph Skip %
\setlength{\parindent}{0pt}
\setlength{\parskip}{1em}

% Table array stretch
\renewcommand{\arraystretch}{1.5}

% Document %
\pagestyle{fancy}
\begin{document}
 \maketitle
 \begin{tcolorbox}[boxsep=3mm,arc=0mm,toprule=1pt,bottomrule=1pt,leftrule=-0.1mm,
   rightrule=-0.1mm,colframe=red!90!black]
  \vspace*{-2pt}
  \begin{center}
   \textbf{NENÍ-LI ŘEČENO JINAK, VŠECHNY POJMY A DŮKAZY FORMULUJTE PEČLIVĚ
   S~DŮRAZEM NA FORMÁLNÍ SPRÁVNOST.}
  \end{center}
 \end{tcolorbox}
 \vspace*{\fill}
 \begin{center}
  \begin{tabular}{c|c}
   \textsf{\textbf{Část}} & \textsf{\textbf{Hodnocení}}\\
   \toprule
   \textcolor{CornflowerBlue}{Základní definice} & 0 / 0\\
   \textcolor{Emerald}{Lehké úlohy a důkazy} & \hspace{2ex}/ 6\\
   \textcolor{BrickRed}{Těžké ulohy a důkazy} & \hspace{2ex} / 15
  \end{tabular}
 \end{center}
 \vspace*{\fill}
 \clearpage
 \begin{tcolorbox}[title=\textsf{Základní
   definice (0 bodů)},arc=0mm,boxsep=3mm,bottomrule=1pt,toprule=1pt,leftrule=-0.1mm,
   rightrule=-0.1mm,colframe=CornflowerBlue!80!white,
   colback=CornflowerBlue!5!white]
  \emph{Neznalost základních definic znamená bezpodmínečné nesložení
  zkoušky.}
  \begin{enumerate}
   \item Implikace ($ \Rightarrow $) pomocí logických spojek \emph{a} ($ \wedge
    $) a \emph{nebo} ($ \vee $).
   \item Sjednocení, průnik a rozdíl dvou libovolných množin $A,B$ užitím
    logických spojek a kvantifikátorů.
   \item Průnik $n \in \N$ libovolných množin $A_i$, kde $i \in
    \{1,\ldots,n\}$, užitím logických spojek a kvantifikátorů.
   \item Podmnožina $B$ množiny $A$ a množina všech podmnožin množiny $A$.
   \item Ekvivalence a třída ekvivalence (relaci není třeba definovat).
   \item \emph{Prosté} zobrazení, zobrazení \emph{na} a \emph{bijekce}
    (zobrazení není třeba definovat).
   \item Kodoména a doména zobrazení. Vzor a obraz prvku při zobrazení.
   \item \textbf{Neformálně} princip matematické indukce.
   \item Permutace, řád permutace. \textbf{Neformálně} cyklický zápis permutace.
   \item Kombinační číslo.
   \item Graf (libovolná z definic).
   \item Sled, tah a cesta v grafu (opět, libovolné z definic).
   \item Vzdálenost vrcholů v grafu.
   \item Excentricita vrcholu a Jordanovo centrum.
  \end{enumerate}
 \end{tcolorbox}
 \clearpage

 \begin{tcolorbox}[title=\textsf{Lehké úlohy a důkazy (6
  bodů)},arc=0mm,boxsep=3mm,bottomrule=1pt,toprule=1pt,leftrule=-0.1mm,
  rightrule=-0.1mm,colframe=Emerald!80!white,colback=Emerald!5!white]
  \emph{Pojmy užité v úlohách nemusíte definovat. Používáte-li k řešení úlohy
  nebo k důkazu předchozí tvrzení, zformulujte je.}
  \begin{enumerate}
   \item Ať $A$ je množina a $ \sim $ je \emph{ekvivalence} na $A$. Pro $x,y \in
    A$ značíme $[x]_{ \sim }$ a $[y]_{ \sim }$ třídy ekvivalence $x$ a $y$ podle
    $ \sim $. Dokažte, že buď $[x]_{ \sim } = [y]_{ \sim }$, nebo $[x]_{ \sim
    } \cap [y]_{ \sim } = \emptyset$.
   \item Ať $A,B$ jsou libovolné množiny. Najděte množinu $C$ (v závislosti na
    $A,B$) takovou, aby platila rovnost
    \[
     (A \cap B) \cup C = (A \cup B) \cap C.
    \]
  \item Zformulujte důkaz, že pro každá dvě přirozená čísla $k,n$, kde $k \leq
   n$, platí vzorec
   \[
    \binom{n}{k} = \binom{n}{n-k}.
   \]
  \item Učitelé základní školy U Smrťáka často berou své žáky do zoologické
   zahrady a varují je, aby v každém případě stály tak blízko klecím, jak to jen
   lze. Zjistivše, že fyzické metody jeví sebe ovšem neúčinnými, souhlasně
   vydali se cestou psychického teroru. Za tímto účelem se dnešního výletu do
   zoo účastní i učitelka matematiky. Ona zadala dětem následující hádanku:
  
   \uv{Tak, milánkové. Živočichy umíme rozdělit na ty, kteří žijí pod vodou, na
    souši, a ve vzduchu. Tady v zahradě žije 30 živočichů ve vodě, 50 na souši a
    20 ve vzduchu. Z těchto, 10 živočichů je schopno žít jak na souši, tak ve
    vzduchu, 5 ve vodě i na souši a 2 ve vzduchu i ve vodě. Je tu dokonce jeden
    zvlášť přizpůsobivý živočich, který umí žít v libovolném prostředí.

    Tak, milánkové, kdo mi poví, kolik je tady v zahradě celkem živočichů, smí
    si jít hrát s tygrem.}
  \item Ať $K_n$ je úplný graf na $n$ vrcholech, čili graf, mezi každým párem
   jehož vrcholů vede hrana. Nalezněte minimální kostru $K_n$ pro každé $n \in
   \N$ a spočtěte její váhu za předpokladu, že množina vrcholů je $V \coloneqq
   \{1,\ldots,n\}$ a váha každé hrany $ij \in E$ je dána funkcí $w(ij) \coloneqq
   \max(i,j)$.
  \item V grafu daném obrázkem níže
   \begin{center}
    \begin{tikzpicture}[scale=2]
     \node[vertex] (v1) at (0,0) {};
     \node[vertex] (v2) at (-30:1) {};
     \node[vertex] (v3) at (30:1) {};
     \node[vertex] (v4) at (30:2) {};
     \node[vertex] (v5) at (-30:2) {};

     \draw[thick] (v1) to node[midway,circle,draw,fill=white,inner sep=1pt]
      {\footnotesize $1$} (v2);
     \draw[thick] (v1) to node[midway,circle,draw,fill=white,inner sep=1pt]
      {\footnotesize $1$} (v3);
     \draw[thick] (v2) to node[midway,circle,draw,fill=white,inner sep=1pt]
      {\footnotesize $3$} (v3);
     \draw[thick] (v3) to node[midway,circle,draw,fill=white,inner sep=1pt]
      {\footnotesize $2$} (v4);
     \draw[thick] (v2) to node[midway,circle,draw,fill=white,inner sep=1pt]
      {\footnotesize $2$} (v5);
     \draw[thick] (v4) to node[midway,circle,draw,fill=white,inner sep=1pt]
      {\footnotesize $9$} (v5);
    \end{tikzpicture}
   \end{center}
   nalezněte \textbf{všechna} řešení EFLP i SFLP \textbf{užitím
   Floydova-Warshallova algoritmu}. Existuje vrchol, který je řešením obou? 
 \end{enumerate}
 \end{tcolorbox}
 \clearpage
 \begin{tcolorbox}[breakable,title=\textsf{Těžké úlohy a důkazy (15
  bodů)},arc=0mm,boxsep=3mm,bottomrule=1pt,toprule=1pt,leftrule=-0.1mm,
  rightrule=-0.1mm,colframe=BrickRed!80!white,colback=BrickRed!5!white]
  \emph{Nemusíte dokonale zformulovat svá řešení. Obecná idea rozvinutá
  důležitými detaily postačuje.}
  \begin{enumerate}
   \item Staří Egypťané měli zajímavý způsob zápisu zlomků. Každý zlomek byl pro
    ně součtem zlomků s čitatelem $1$ a navzájem různými jmenovateli, například
    \[
     \frac{3}{5} = \frac{1}{2} + \frac{1}{10}, \quad \frac{4}{7} = \frac{1}{2} +
     \frac{1}{14} \quad \text{nebo} \quad \frac{1}{2} = \frac{1}{2}.
    \]
    Algoritmus rozpisu libovolného zlomku tímto způsobem pracuje následovně. Ať
    jsou dána dvě přirozená čísla $m,n \in \N$ taková, že $m < n$, tedy zlomek
    $m / n$ je menší než $1$.
    \begin{enumerate}[topsep=0pt]
     \item Polož $m_0 \coloneqq m$, $n_0 \coloneqq n$ a $i \coloneqq 0$.
     \item Spočti
      \[
       z_i \coloneqq \frac{1}{\left\lceil n_i/m_i \right\rceil}.
      \]
      a
      \[
       \frac{m_{i+1}}{n_{i+1}} \coloneqq \frac{m_i}{n_i} - z_i.
      \]
     \item Je-li $m_{i+1}= 1$, polož $z_{i+1} \coloneqq m_{i+1} / n_{i+1}$ a
      skonči. Jinak polož $i \coloneqq i + 1$ a opakuj část (b).
    \end{enumerate}
    Po skončení algoritmu je rozkladem zlomku $m / n$ na
    zlomky s různými jmenovateli a čitateli rovny $1$ právě
    \[
     \frac{m}{n} = z_0 + z_1 + \ldots + z_{i+1}.
    \]
    Dokažte indukcí, že je tento algoritmus korektní (tj. skončí a dá správný
    výsledek).\\
    \emph{Poznámka}: Výraz $\left\lceil n \right\rceil$ značí \textbf{horní
    celou část} čísla $n$, tj. nejmenší přirozené číslo větší než $n$.
   \item Dokažte, že počet všech zobrazení mezi konečnými množinami $A$ a $B$ je
    přesně $\# B^{\# A}$.
   \item \emph{Trojúhelníkem} myslíme graf $K_3$, tj. úplný graf na třech
    vrcholech. Ať $G$ je libovolný graf na $n$ vrcholech. Takový graf má nejvíce
    $\binom{n}{2}$ hran, v takovém případě je to $K_n$. Co se ovšem stane, když
    zakážeme trojúhelníky, tedy když $G$ \textbf{nyní nesmí obsahovat} $K_3$
    jako podgraf? Už jistě nemůže mít $\binom{n}{2}$ hran, neboť úplné grafy
    $K_n$ pro $n \geq 3$ trojúhelníky obsahují.

    Označme výrazem $T(n)$ maximální počet hran, který může mít graf bez
    trojúhelníků na $n$ vrcholech. Platí $T(n) = \left\lfloor n^2 / 4
    \right\rfloor$, to však není snadné dokázat. V této úloze dokážete
    zeslabenou verzi, tj. že $T(n) \geq \left\lfloor n^2 / 4 \right\rfloor$,
    neboli, že existují grafy bez trojúhelníků na $n$ vrcholech mající aspoň
    $\left\lfloor n^2 / 4 \right\rfloor$ hran.
    \begin{enumerate}
     \item Ať $K_{a,b}$ značí úplný bipartitní graf na $a+b$ vrcholech. Tím
      myslíme graf $K_{a,b} = (V_1 \cup V_2,E)$ takový, že $V_1$ a $V_2$ jsou
      disjunktní množiny vrcholů, $\# V_1 = a$, $\# V_2 = b$ a z každého vrcholu
      z $V_1$ vede hrana do každého vrcholu z $V_2$, ale mezi dvěma vrcholy ze
      stejné množiny žádné hrany nevedou. Dokažte, že $K_{a,b}$ má přesně $a
      \cdot b$ hran.
     \item Dokažte, že $K_{a,b}$ neobsahuje trojúhelník.
     \item Najděte čísla $a,b \in \N$ taková, že $a+b=n$ a $a \cdot b \geq
     \left\lfloor n^2 / 4 \right\rfloor$.\\
      \textbf{Hint:} Rozlište dva případy: $n$ sudé a $n$ liché.
     \item Odůvodněte, proč z (a) až (c) už plyne původní tvrzení.
    \end{enumerate}
  \end{enumerate}
 \end{tcolorbox}
\end{document}
