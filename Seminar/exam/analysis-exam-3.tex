\documentclass[a4paper,11pt]{article}

\usepackage[czech,english]{babel}
% Fonts %
\usepackage{fouriernc}
\usepackage[T1]{fontenc}

% Colors %
\usepackage[dvipsnames]{xcolor}

% Colored boxes %
\usepackage[many]{tcolorbox}

% Page Layout %
\usepackage[margin=1in]{geometry}

% Fancy Headers %
\usepackage{fancyhdr}
\fancyhf{}
\cfoot{\thepage}
\rhead{}
\renewcommand{\headrulewidth}{0pt}
\setlength{\headheight}{16pt}

% Math
\usepackage{mathtools}
\usepackage{amssymb}
\usepackage{faktor}
\usepackage{import}
\usepackage{caption}
\usepackage{subcaption}
\usepackage{wrapfig}
\usepackage{enumitem}
\setlist{topsep=6pt}
\setlist[enumerate,1]{label=(\arabic*)}

\usepackage{tikz}
\usetikzlibrary{cd,positioning,babel,shapes}
\usepackage{tkz-base}
\usepackage{tkz-euclide}
\tikzset{
  vertex/.style = {shape=circle,fill,text=white,minimum size=9pt,inner
  sep=1pt}
}

% Theorems
\usepackage[thmmarks, amsmath, thref]{ntheorem}
\usepackage{thmtools}

\theoremsymbol{\ensuremath{\blacksquare}}
\newtheorem*{solution}{Possible solution.}

% Title %
\title{\Huge\textsf{Ústní zkouška}\\
 \Large\textsf{z Úvodu do matematické analýzy, části prvé}\\
 \vspace*{1em}
 Verze: 0m3g4r3kt\\
 \author{Přednášející: His Divine Wisdom Sir Adam Clypatch}
 \date{16. února 2024}
}

% Table of Contents %
\usepackage{hyperref}
\hypersetup{
 colorlinks=true,
 linktoc=all,
 linkcolor=blue
}

% Tables %
\usepackage{booktabs}
\usepackage{tabularx}

% Patch for hyphens
\usepackage{regexpatch}
\makeatletter
% Change the `-` delimiter to an active character
\xpatchparametertext\@@@cmidrule{-}{\cA-}{}{}
\xpatchparametertext\@cline{-}{\cA-}{}{}
\makeatother

\newcolumntype{s}{>{\centering\arraybackslash}p{.4\textwidth}}

% Operators %
\DeclareMathOperator{\img}{im}
\DeclareMathOperator{\dom}{dom}
\DeclareMathOperator{\codom}{codom}

% Common operators %
\newcommand{\R}{\mathbb{R}}
\newcommand{\N}{\mathbb{N}}
\newcommand{\Z}{\mathbb{Z}}
\newcommand{\Q}{\mathbb{Q}}
\newcommand{\C}{\mathbb{C}}

\newcommand{\clr}{\textcolor{red}}
\newcommand{\clb}{\textcolor{blue}}
\newcommand{\clg}{\textcolor{green}}
\newcommand{\clm}{\textcolor{magenta}}
\newcommand{\clv}{\textcolor{violet}}
\newcommand{\clbr}{\textcolor{Sepia}}

% American Paragraph Skip %
\setlength{\parindent}{0pt}
\setlength{\parskip}{1em}

% Table array stretch
\renewcommand{\arraystretch}{1.5}

% Document %
\pagestyle{fancy}
\begin{document}
 \maketitle
 \begin{tcolorbox}[boxsep=3mm,arc=0mm,toprule=1pt,bottomrule=1pt,leftrule=-0.1mm,
   rightrule=-0.1mm,colframe=red!90!black]
  \vspace*{-2pt}
  \begin{center}
   \textbf{NENÍ-LI ŘEČENO JINAK, VŠECHNY POJMY A DŮKAZY FORMULUJTE PEČLIVĚ
   S~DŮRAZEM NA FORMÁLNÍ SPRÁVNOST.}
  \end{center}
 \end{tcolorbox}
 \vspace*{\fill}
 \begin{center}
  \begin{tabular}{c|c}
   \textsf{\textbf{Část}} & \textsf{\textbf{Hodnocení}}\\
   \toprule
   \textcolor{CornflowerBlue}{Základní definice} & 0 / 0\\
   \textcolor{Emerald}{Lehké úlohy a důkazy} & \hspace{2ex}/ 6\\
   \textcolor{BrickRed}{Těžké ulohy a důkazy} & \hspace{2ex} / 12
  \end{tabular}
 \end{center}
 \vspace*{\fill}
 \clearpage
 \begin{tcolorbox}[title=\textsf{Základní
   definice (0 bodů)},arc=0mm,boxsep=3mm,bottomrule=1pt,toprule=3pt,leftrule=-0.1mm,
   rightrule=-0.1mm,colframe=CornflowerBlue!80!white,
   colback=CornflowerBlue!5!white]
  \emph{Neznalost základních definic znamená bezpodmínečné nesložení
  zkoušky.}
  \begin{enumerate}
   \item Přirozená čísla.
   \item Konvergentní posloupnost.
   \item Limita posloupnosti v $ \pm \infty$.
   \item Infimum a supremum.
   \item Délka intervalu.
  \end{enumerate}
 \end{tcolorbox}
 \clearpage

 \begin{tcolorbox}[title=\textsf{Lehké úlohy a důkazy (6
  bodů)},arc=0mm,boxsep=3mm,bottomrule=1pt,toprule=3pt,leftrule=-0.1mm,
  rightrule=-0.1mm,colframe=Emerald!80!white,colback=Emerald!5!white]
  \emph{Pojmy užité v úlohách nemusíte definovat. Používáte-li k řešení úlohy
  nebo k důkazu předchozí tvrzení, zformulujte je.}
  \begin{enumerate}
   \item Trojúhelníková nerovnost.
   \item Dokažte, že když $a,b:\N \to \R$ jsou posloupnosti, $\lim a = A \in \R$
    a $\lim b = B \in \R$, pak
    \[
     \lim (a + b) = A + B.
    \]
   \item Spočtěte
    \[
     \lim_{n \to \infty} (\sqrt{n+1}-\sqrt{n}).
    \]
  \end{enumerate}
 \end{tcolorbox}
 \clearpage
 \begin{tcolorbox}[breakable,title=\textsf{Těžké úlohy a důkazy (12
  bodů)},arc=0mm,boxsep=3mm,bottomrule=1pt,toprule=1pt,leftrule=-0.1mm,
  rightrule=-0.1mm,colframe=BrickRed!80!white,colback=BrickRed!5!white]
  \emph{Nemusíte dokonale zformulovat svá řešení. Obecná idea rozvinutá
  důležitými detaily postačuje.}
  \begin{enumerate}
   \item Ať $a:\N \to \R$ je posloupnost a $b:\N \to \R$ je posloupnost
    \uv{průměrů} posloupnosti $a$, tj.
    \[
     b_n \coloneqq \frac{1}{n}\sum_{k=1}^n a_k.
    \]
    \begin{enumerate}
     \item Dokažte, že když $\lim a = 0$, pak i $\lim b = 0$. Platí i opačná
      implikace? Dokažte nebo uveďte protipříklad.
     \item Dokažte s použitím bodu (a), že když $\lim a = L \in \R$, pak rovněž
      $\lim b = L$.
    \end{enumerate}
    \textbf{Návod:} Pro dané $\varepsilon>0$ a $n_0 \in \N$, od kterého dále již
    platí $|a_n|<\varepsilon$, rozložte členy posloupnosti $b_n$ na dvě složky,
    které lze obě seshora odhadnout číslem závislým na $\varepsilon$. V bodě (b)
    využijte faktu, že když $\lim_{n \to \infty} a_n = L$, pak $\lim_{n \to
    \infty} (a_n - L) = 0$.
   \item Dokažte, že
    \[
     \lim_{n \to \infty} \frac{4^{n}}{n!} = 0.
    \]
    \textbf{Návod:}
    \begin{enumerate}
     \item Ukažte, že posloupnost $a_n \coloneqq 4^{n} / n!$ je klesající a
      zdola omezená, a tedy má z~věty o konvergenci monotónních posloupností
      limitu.
     \item Vyjádřete $a_{n+1}$ pomocí $a_n$ a využijte získaného vzorce pro
      výpočet $\lim_{n \to \infty} a_{n} = \lim_{n \to \infty} a_{n+1}$.
    \end{enumerate}
  \end{enumerate}
 \end{tcolorbox}
\end{document}
