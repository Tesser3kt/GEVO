\documentclass[a4paper,11pt]{article}

\usepackage[czech,english]{babel}
% Fonts %
\usepackage{fouriernc}
\usepackage[T1]{fontenc}

% Colors %
\usepackage[dvipsnames]{color}
\usepackage[dvipsnames]{xcolor}

% Page Layout %
\usepackage[margin=1.5in]{geometry}

% Fancy Headers %
\usepackage{fancyhdr}
\fancyhf{}
\cfoot{\thepage}
\rhead{}
\renewcommand{\headrulewidth}{0pt}
\setlength{\headheight}{16pt}

% Math
\usepackage{mathtools}
\usepackage{amssymb}
\usepackage{faktor}
\usepackage{import}
\usepackage{caption}
\usepackage{subcaption}
\usepackage{wrapfig}
\usepackage{enumitem}
\setlist{topsep=0pt}

\usepackage{tikz}
\usetikzlibrary{cd,positioning,babel,shapes,calc}
\usepackage{tkz-base}
\usepackage{tkz-euclide}

% Theorems
\usepackage[thmmarks, amsmath, thref]{ntheorem}
\usepackage{thmtools}

\theoremsymbol{\ensuremath{\blacksquare}}
\newtheorem*{solution}{Possible solution.}

% Title %
\title{\Huge\textsf{Tree Diagrams}\\
 \Large\textsf{Exam -- PreIB 4.AB 4}
 \author{Áďa Klepáčů}
 \date{\today}
}

% Table of Contents %
\usepackage{hyperref}
\hypersetup{
 colorlinks=true,
 linktoc=all,
 linkcolor=blue
}

% Tables %
\usepackage{booktabs}
\usepackage{tabularx}
\usepackage{multirow} 

% Patch for hyphens
\usepackage{regexpatch}
\makeatletter
% Change the `-` delimiter to an active character
\xpatchparametertext\@@@cmidrule{-}{\cA-}{}{}
\xpatchparametertext\@cline{-}{\cA-}{}{}
\makeatother

\newcolumntype{s}{>{\centering\arraybackslash}p{.4\textwidth}}

% Operators %
\DeclareMathOperator{\Ker}{Ker}
\DeclareMathOperator{\Img}{Im}
\DeclareMathOperator{\End}{End}
\DeclareMathOperator{\Aut}{Aut}
\DeclareMathOperator{\Inn}{Inn}

% Common operators %
\newcommand{\R}{\mathbb{R}}
\newcommand{\N}{\mathbb{N}}
\newcommand{\Z}{\mathbb{Z}}
\newcommand{\Q}{\mathbb{Q}}
\newcommand{\C}{\mathbb{C}}

\newcommand{\clr}{\textcolor{red}}
\newcommand{\clb}{\textcolor{blue}}
\newcommand{\clg}{\textcolor{green}}
\newcommand{\clm}{\textcolor{magenta}}
\newcommand{\clv}{\textcolor{violet}}
\newcommand{\clbr}{\textcolor{Sepia}}

% American Paragraph Skip %
\setlength{\parindent}{0pt}
\setlength{\parskip}{1em}

% Document %
\pagestyle{fancy}
\begin{document}

\maketitle
\thispagestyle{empty}

\begin{center}
 \hrule
 \textbf{\clr{\uppercase{Don't forget to explain your reasoning when
 appropriate!}}}
 \vspace{2ex}
 \hrule
\end{center}

\section*{Problem 1.}

In a factory, three machines -- A, B and C -- are used to make biscuits.

Machine A makes 25 \% of the biscuits, B makes 45 \% and C the rest. In
addition, about 2 \% of all the biscuits made by A are broken, 3 \% of those
made by B are broken and 5 \% of those made by C are broken.

\begin{enumerate}
 \item Draw a tree diagram representing the problem.
 \item Calculate the probability that a randomly picked biscuit made by machine
  A is not broken.
 \item Calculate the probability that a randomly picked biscuit is broken.
 \item Assuming that a biscuit is broken, what's the probability it was
  \textbf{not} made by machine B?
\end{enumerate}

\section*{Problem 2.}

A bag contains 5 black and 2 white balls. Grace picks a ball at random and then
replaces it (meaning, she replaces black with white or white with black). Grace
then picks a second ball.

\begin{enumerate}
 \item Draw a tree diagram representing the problem.
 \item Compute the probability that Grace picks 2 black balls.
\end{enumerate}
 
\end{document}
