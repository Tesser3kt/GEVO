\documentclass[a4paper,11pt]{article}

\usepackage[czech,english]{babel}
% Fonts %
\usepackage{fouriernc}
\usepackage[T1]{fontenc}

% Colors %
\usepackage[dvipsnames]{color}
\usepackage[dvipsnames]{xcolor}

% Page Layout %
\usepackage[margin=1.5in]{geometry}

% Fancy Headers %
\usepackage{fancyhdr}
\fancyhf{}
\cfoot{\thepage}
\rhead{}
\renewcommand{\headrulewidth}{0pt}
\setlength{\headheight}{16pt}

% Math
\usepackage{mathtools}
\usepackage{amssymb}
\usepackage{faktor}
\usepackage{import}
\usepackage{caption}
\usepackage{subcaption}
\usepackage{wrapfig}
\usepackage{enumitem}
\setlist{topsep=0pt}

\usepackage{tikz}
\usetikzlibrary{cd,positioning,babel,shapes,calc}
\usepackage{tkz-base}
\usepackage{tkz-euclide}

% Theorems
\usepackage[thmmarks, amsmath, thref]{ntheorem}
\usepackage{thmtools}

\theoremsymbol{\ensuremath{\blacksquare}}
\newtheorem*{solution}{Possible solution.}

% Title %
\title{\Huge\textsf{Intro To Statistics}\\
 \Large\textsf{Exam}
 \author{Áďa Klepáčů}
 \date{\today}
}

% Table of Contents %
\usepackage{hyperref}
\hypersetup{
 colorlinks=true,
 linktoc=all,
 linkcolor=blue
}

% Tables %
\usepackage{booktabs}
\usepackage{tabularx}
\usepackage{multirow} 

% Patch for hyphens
\usepackage{regexpatch}
\makeatletter
% Change the `-` delimiter to an active character
\xpatchparametertext\@@@cmidrule{-}{\cA-}{}{}
\xpatchparametertext\@cline{-}{\cA-}{}{}
\makeatother

\newcolumntype{s}{>{\centering\arraybackslash}p{.4\textwidth}}

% Operators %
\DeclareMathOperator{\Ker}{Ker}
\DeclareMathOperator{\Img}{Im}
\DeclareMathOperator{\End}{End}
\DeclareMathOperator{\Aut}{Aut}
\DeclareMathOperator{\Inn}{Inn}

% Common operators %
\newcommand{\R}{\mathbb{R}}
\newcommand{\N}{\mathbb{N}}
\newcommand{\Z}{\mathbb{Z}}
\newcommand{\Q}{\mathbb{Q}}
\newcommand{\C}{\mathbb{C}}

\newcommand{\clr}{\textcolor{red}}
\newcommand{\clb}{\textcolor{blue}}
\newcommand{\clg}{\textcolor{green}}
\newcommand{\clm}{\textcolor{magenta}}
\newcommand{\clv}{\textcolor{violet}}
\newcommand{\clbr}{\textcolor{Sepia}}

% American Paragraph Skip %
\setlength{\parindent}{0pt}
\setlength{\parskip}{1em}

% Document %
\pagestyle{fancy}
\begin{document}

\maketitle
\thispagestyle{fancy}

\begin{center}
 \hrule
 \textbf{\clr{\uppercase{Don't just cite the content of my presentation when
 explaining. Always apply your analysis to the experiment at hand.}}}
 \vspace{2ex}
 \hrule
\end{center}
 
We've conducted a small experimental comparison of the minimal required level of
education and average annual salary between multiple jobs (in different fields).
The results are as follows:
\begin{table}[ht]
 \centering
 \begin{tabular}{c|c|c}
  \textbf{Job} & \textbf{Required Education} & \textbf{Annual Salary}\\
  \toprule
  \multirow{2}*{Janitor} & Elementary School & \multirow{2}*{\$40,517} \\
                         & (at least 7 years) & \\
  \midrule
  \multirow{2}*{Police Officer} & High School Diploma + Training &
  \multirow{2}*{\$61,986} \\
                                & (at least 13.5 years) & \\
  \midrule
  \multirow{2}*{Registered Nurse} & BSN or ADN degree &
  \multirow{2}*{\$101,841}\\
                                  & (at least 15 years) & \\
  \midrule
  \multirow{2}*{Software Developer} & Bachelor's Degree + Internship &
  \multirow{2}*{\$117,509}\\
                                    & (at least 18 years) \\
  \midrule                                  
  \multirow{2}*{Theoretical Mathematician} & Ph.D. + postdoctorate &
  \multirow{2}*{\$101,080}\\
                                           & (at least 25 years) &
 \end{tabular}
 \caption*{Experiment Data}
\end{table}
\begin{enumerate}
 \item We provide three ways to quantify education:
  \begin{enumerate}
   \item the fewest number of years to achieve it.
   \item earning a degree (or some sort of training) means $+1$. That is, for
    instance, a \emph{high school diploma} is $2$ because it requires finishing
    two levels of education (elementary + high). In the same vein, \emph{Ph.D. +
    postdoc} would be the number $6$ (elementary + high + bachelor + master +
    Ph.D. + postdoc).
   \item relative length -- meaning, say, elementary school equals $1$ and high
    school + training equals $13.5 / 7$ because it is that many times longer than
    elementary school.
  \end{enumerate}
  Choose one and fill it into the table. \textbf{Explain why you think it is the
  best statistical predictor}.
 \item Calculate the mean required education and the mean annual salary.
  \textbf{Explain what type of mean you use for each output and why}.
 \item Determine \textbf{the median} of the annual salary. Is it very different
  (relatively speaking) from the mean? Why is it so?
 \item Calculate the standard deviation for both outputs. What information does
  it give us? \textbf{Explain in the context of this experiment}.
 \item Calculate the correlation of required education and average salary. How
  strongly do they correlate and what can you deduce from it?
\end{enumerate}
\end{document}
