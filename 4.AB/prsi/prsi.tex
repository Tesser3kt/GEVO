\documentclass[a4paper,11pt]{article}

% Babel %
\usepackage[czech]{babel}

% Colors %
\usepackage[dvipsnames]{xcolor}

% Page Layout %
\usepackage[margin=1.5in]{geometry}

% Fancy Headers %
\usepackage{fancyhdr}
\fancyhf{}
\renewcommand{\headrulewidth}{0pt}
\setlength{\headheight}{16pt}

% Math
\usepackage{mathtools}
\usepackage{amssymb}
\usepackage{faktor}
\usepackage{import}
\usepackage{caption}
\usepackage{subcaption}
\usepackage{wrapfig}

% Theorems
\usepackage{amsthm}
\usepackage{thmtools}

% Title %
\title{\Huge\textsf{}\\
 \Large\textsf{}
 \author{}
 \date{}
}

% Table of Contents %
\usepackage{hyperref}
\hypersetup{
 colorlinks=true,
 linktoc=all,
 linkcolor=blue
}

% Tables %
\usepackage{booktabs}

% Enumerate %
\usepackage{enumitem}
\setlist[itemize]{topsep=0pt}
% Operators %
\DeclareMathOperator{\Ker}{Ker}
\DeclareMathOperator{\Img}{Im}
\DeclareMathOperator{\End}{End}
\DeclareMathOperator{\Aut}{Aut}
\DeclareMathOperator{\Inn}{Inn}

% Common operators %
\newcommand{\R}{\mathbb{R}}
\newcommand{\N}{\mathbb{N}}
\newcommand{\Z}{\mathbb{Z}}
\newcommand{\Q}{\mathbb{Q}}
\newcommand{\C}{\mathbb{C}}

% American Paragraph Skip %
\setlength{\parindent}{0pt}
\setlength{\parskip}{1em}

% Document %
\pagestyle{fancy}
\begin{document}

\thispagestyle{fancy}

\textbf{Domácí úkol.} V pseudokódu naprogramujte karetní hru \emph{prší} pro tři
hráče. Pravidla hry jsou následující.
\begin{itemize}
 \item Na začátku má každý hráč 4 karty.
 \item Hráči se střídají po tahu.
 \item Tah sestává z
  \begin{itemize}
   \item odhození jedné karty,
   \item přibrání jedné karty z balíčku,
   \item přibrání dvou karet z balíčku nebo
   \item stání.
  \end{itemize}
 \item Hráč přibírá dvě karty, pokud hráč před ním odhodil 7 a sám nemá žádnou 7
  v ruce. \textcolor{red}{Hráč přibírá vždy jen dvě karty na hozenou 7, bez
  ohledu na to, kolikátá odhozená 7 to je!}
 \item Hráč stojí, pokud hráč před ním odhodil A a sám nemá žádné A v ruce.
 \item Hráč odhazuje, pokud má v ruce kartu stejné barvy nebo hodnoty, které je
  poslední odhozená karta, a nenastal žádný z případů výše.
 \item Hráč přibírá jednu kartu, když nenastal žádný z případů výše.
\end{itemize}

\vspace{\parskip}
\hrule

Vaším úkolem je napsat algoritmus, který běží, dokud má každý hráč v ruce aspoň
jednu kartu. Algoritmus se musí v každém tahu zachovat podle pravidel výše a
podle toho, jaká karta byla odhozena v předešlém tahu. Nemusíte řešit balíček
karet; klidně počítejte s tím, že nikdy nedojde. Máte jen naprogramovat chování
hráče v každém tahu.

K dispozici máte
\begin{itemize}
 \item proměnné $p_1,p_2,p_3$ určující počet karet v ruce každého hráče. Na
  začátku jsou všechny rovny $4$ a sami je musíte v každém tahu aktualizovat.
 \item množinu $\mathcal{B}$ karetních barev, tj., $\mathcal{B} = \{\heartsuit,
  \diamondsuit, \spadesuit, \clubsuit\}$.
 \item množinu $\mathcal{H}$ karetních hodnot, tj., $\mathcal{H} = \{7, 8, 9,
  10, J, Q, K, A\}$.
\end{itemize}
Zbytek je na vás. Můžete si definovat kolik nových proměnných a procedur chcete.
Chování hráče může být libovolně složité, ale spodní hranice je taková, že
kdokoli, kdo sleduje váš algoritmus, musí být schopen úspěšně odehrát hru prší.

\vspace{\parskip}
\hrule

\textbf{Speciální karty}
\begin{itemize}
 \item 7 libovolné barvy nutí následujícího hráče brát dvě karty z balíčku,
  pokud sám nemá v ruce 7.
 \item A libovolné barvy nutí následujícího hráče stát, pokud sám nemá A v ruce.
 \item K$\spadesuit$ nutí následujícího hráče brát pět karet z balíčku za každé
  situace.
 \item Q libovolné barvy umožňuje hráči na tahu změnit barvu pro následujícího
  hráče. \textbf{Q lze hodit, stejně jako všechny ostatní karty, pouze na jiné
  Q nebo na kartu stejné barvy}.
\end{itemize}

\textbf{Pár rad a komentářů závěrem}
\begin{itemize}
 \item Pište v češtině, \textbf{co nejméně to jde}. Skoro všechno by se mělo dát
  napsat pomocí proměnných, podmínek, cyklů a procedur. Na instrukce psané
  větami se nedívám vlídně.
 \item Často se stane, že hráč může odhodit více různých karet.
  \textcolor{red}{Když nechcete programovat žádnou strategii, můžete použít
  \textbf{náhodný} výběr.} Na počítači se dá celkem snadno naprogramovat náhodná
  volba, takže to vám dovoluji. Do pseudokódu to můžete napsat třeba jako
  proceduru \texttt{ZvolNáhodně} nebo tak nějak.
 \item Vytvořte si proměnnou, kde si budete pamatovat poslední odhozenou kartu
  na kupce. Žádné pod ní vás nezajímají.
 \item Všimněte si, že pravidla hry jsou psána návodně, aby vám pomohla s
  postavením nějaké aspoň částečně elegantní podmínky.
 \item Pseudokód pište klidně na papír. Patlat se s formátováním na počítači je
  zbytečné.
 \item Než začnete psát, \textcolor{red}{\textbf{MYSLETE!}} Když si špatně
  rozmyslíte, jak má ten algoritmus vypadat, v půlce práce se věci obtížně
  spravují.
 \item Tohle není úkol, který se dá stihnout večer před deadlinem. Zkuste si
  práci rozdělit aspoň trochu.
\end{itemize}

\end{document}
