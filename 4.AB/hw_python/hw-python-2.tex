\documentclass[a4paper,11pt]{article}

\usepackage[czech]{babel}
% Fonts %
\usepackage{fouriernc}
\usepackage[T1]{fontenc}

% Colors %
\usepackage[dvipsnames]{color}
\usepackage[dvipsnames]{xcolor}

% Page Layout %
\usepackage[margin=1.5in]{geometry}

% Fancy Headers %
\usepackage{fancyhdr}
\fancyhf{}
\cfoot{\thepage}
\rhead{}
\renewcommand{\headrulewidth}{0pt}
\setlength{\headheight}{16pt}

% Math
\usepackage{mathtools}
\usepackage{amssymb}
\usepackage{faktor}
\usepackage{import}
\usepackage{caption}
\usepackage{subcaption}
\usepackage{wrapfig}
\usepackage{enumitem}
\setlist{topsep=0pt}

\usepackage{tikz}
\usetikzlibrary{cd,positioning,babel,shapes,calc}
\usepackage{tkz-base}
\usepackage{tkz-euclide}

% Theorems
\usepackage[thmmarks, amsmath, thref]{ntheorem}
\usepackage{thmtools}

\theoremsymbol{\ensuremath{\blacksquare}}
\newtheorem*{solution}{Possible solution.}

% Title %
\title{\Huge\textsf{Homework -- PreIB 3.AB 4}\\
 \Large\textsf{Triangulations and Symmetries of Regular Polygons}
 \author{Áďa Klepáčů}
 \date{\today}
}

% Table of Contents %
\usepackage{hyperref}
\hypersetup{
 colorlinks=true,
 linktoc=all,
 linkcolor=blue
}

% Tables %
\usepackage{booktabs}
\usepackage{tabularx}

% Patch for hyphens
\usepackage{regexpatch}
\makeatletter
% Change the `-` delimiter to an active character
\xpatchparametertext\@@@cmidrule{-}{\cA-}{}{}
\xpatchparametertext\@cline{-}{\cA-}{}{}
\makeatother

\newcolumntype{s}{>{\centering\arraybackslash}p{.4\textwidth}}

% Operators %
\DeclareMathOperator{\Ker}{Ker}
\DeclareMathOperator{\Img}{Im}
\DeclareMathOperator{\End}{End}
\DeclareMathOperator{\Aut}{Aut}
\DeclareMathOperator{\Inn}{Inn}

% Common operators %
\newcommand{\R}{\mathbb{R}}
\newcommand{\N}{\mathbb{N}}
\newcommand{\Z}{\mathbb{Z}}
\newcommand{\Q}{\mathbb{Q}}
\newcommand{\C}{\mathbb{C}}

\newcommand{\clr}{\textcolor{red}}
\newcommand{\clb}{\textcolor{blue}}
\newcommand{\clg}{\textcolor{green}}
\newcommand{\clm}{\textcolor{magenta}}
\newcommand{\clv}{\textcolor{violet}}
\newcommand{\clbr}{\textcolor{Sepia}}

% American Paragraph Skip %
\setlength{\parindent}{0pt}
\setlength{\parskip}{1em}

% Document %
\pagestyle{fancy}
\begin{document}

\thispagestyle{fancy}

Napište program \textbf{v Pythonu} (nebo v jiném programovacím jazyce, pokud
chcete), který řeší následující úlohy.

\section*{DNA}

V řetězcích deoxyribonukleové kyseliny jsou symboly \texttt{A} a \texttt{T}
komplementární a stejně tak symboly \texttt{C} a \texttt{G}. Napište funkci,
která dostane jednu část DNA (jako \texttt{string}) a vrátí její komplementární
část, tj. prohodí mezi sebou \texttt{A} a \texttt{T}, \texttt{C} a \texttt{G}.

\emph{Příklady:}
\begin{itemize}
 \item Komplement k řetězci \texttt{ATTGC} je \texttt{TAACG}.
 \item Komplement k řetězci \texttt{GTAT} je \texttt{CATA}.
\end{itemize}

\section*{Růst populace}

V malé vesničce je na začátku roku \texttt{1000} obyvatel. Průměrně vzroste
každým rokem její populace o \texttt{2 \%} a navíc se přistěhuje \texttt{50}
nových obyvatel. Napište funkci, která dostane parametrem budoucí počet obyvatel
(jako \texttt{int}) a vrátí počet let, za který přeroste populace vesničky tento
daný počet obyvatel.

\emph{Příklad:} Když je zadaný budoucí počet obyvatel \texttt{1200}, pak
populace doroste tohoto počtu za 3 roky.
\begin{itemize}
 \item Po 1. roce je ve vesničce \texttt{1000 * 1.02 + 50 = 1070} obyvatel.
 \item Po 2. roce je ve vesničce \texttt{1070 * 1.02 + 50 = 1141} obyvatel.
 \item Po 3. roce je ve vesničce \texttt{1141 * 1.02 + 50 = 1213} obyvatel.
\end{itemize}

\end{document}
