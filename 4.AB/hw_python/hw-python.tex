\documentclass[a4paper,11pt]{article}

\usepackage[czech]{babel}
% Fonts %
\usepackage{fouriernc}
\usepackage[T1]{fontenc}

% Colors %
\usepackage[dvipsnames]{color}
\usepackage[dvipsnames]{xcolor}

% Page Layout %
\usepackage[margin=1.5in]{geometry}

% Fancy Headers %
\usepackage{fancyhdr}
\fancyhf{}
\cfoot{\thepage}
\rhead{}
\renewcommand{\headrulewidth}{0pt}
\setlength{\headheight}{16pt}

% Math
\usepackage{mathtools}
\usepackage{amssymb}
\usepackage{faktor}
\usepackage{import}
\usepackage{caption}
\usepackage{subcaption}
\usepackage{wrapfig}
\usepackage{enumitem}
\setlist{topsep=0pt}

\usepackage{tikz}
\usetikzlibrary{cd,positioning,babel,shapes,calc}
\usepackage{tkz-base}
\usepackage{tkz-euclide}

% Theorems
\usepackage[thmmarks, amsmath, thref]{ntheorem}
\usepackage{thmtools}

\theoremsymbol{\ensuremath{\blacksquare}}
\newtheorem*{solution}{Possible solution.}

% Title %
\title{\Huge\textsf{Homework -- PreIB 3.AB 4}\\
 \Large\textsf{Triangulations and Symmetries of Regular Polygons}
 \author{Áďa Klepáčů}
 \date{\today}
}

% Table of Contents %
\usepackage{hyperref}
\hypersetup{
 colorlinks=true,
 linktoc=all,
 linkcolor=blue
}

% Tables %
\usepackage{booktabs}
\usepackage{tabularx}

% Patch for hyphens
\usepackage{regexpatch}
\makeatletter
% Change the `-` delimiter to an active character
\xpatchparametertext\@@@cmidrule{-}{\cA-}{}{}
\xpatchparametertext\@cline{-}{\cA-}{}{}
\makeatother

\newcolumntype{s}{>{\centering\arraybackslash}p{.4\textwidth}}

% Operators %
\DeclareMathOperator{\Ker}{Ker}
\DeclareMathOperator{\Img}{Im}
\DeclareMathOperator{\End}{End}
\DeclareMathOperator{\Aut}{Aut}
\DeclareMathOperator{\Inn}{Inn}

% Common operators %
\newcommand{\R}{\mathbb{R}}
\newcommand{\N}{\mathbb{N}}
\newcommand{\Z}{\mathbb{Z}}
\newcommand{\Q}{\mathbb{Q}}
\newcommand{\C}{\mathbb{C}}

\newcommand{\clr}{\textcolor{red}}
\newcommand{\clb}{\textcolor{blue}}
\newcommand{\clg}{\textcolor{green}}
\newcommand{\clm}{\textcolor{magenta}}
\newcommand{\clv}{\textcolor{violet}}
\newcommand{\clbr}{\textcolor{Sepia}}

% American Paragraph Skip %
\setlength{\parindent}{0pt}
\setlength{\parskip}{1em}

% Document %
\pagestyle{fancy}
\begin{document}

\thispagestyle{fancy}

Napište program \textbf{v Pythonu} (nebo v jiném programovacím jazyce, pokud
chcete), který řeší vám přidělené úlohy.

\subsection*{Podmínky}
\begin{itemize}
 \item Váš program musí \textbf{přesně} řešit zadanou úlohu. To mimo jiné
  znamená, že funguje a naprosto dodržuje, co je vstupem a výstupem.
 \item S řešením úloh vám \textbf{\clr{nesmí}} pomáhat žádná umělá inteligence.
  Spolupráce s jiným žákem je fajn, pokud o ní vím dopředu a odsouhlasím ji.
  Nebudu podporovat párování lidí, z nichž jeden ví, co dělá, a druhý v tom
  plave.
 \item K řešení používejte ideálně pouze části Pythonu, které jsme si
  vysvětlili. \textbf{Použití jakýchkoli funkcí nebo datových struktur, které
  výrazně zjednodušují úlohu, neberu!} Příkladem takových funkcí je třeba
  \texttt{sort}, která seřadí seznam, nebo \texttt{reverse}, která ho otočí.
 \item \textbf{Součástí vašeho programu budou komentáře!} Těmi vysvětlujete, co
  děláte a \textbf{hlavně} proč.
 \item Program odevzdáte do Classroomu s koncovkou .py (tj. \textbf{\clr{ne}}
  jako textový soubor a už vůbec ne jako obrázek). Program, který bude obsahovat
  chyby, kvůli kterým nejde spustit, hodnotím rovnou 0 \%. \textbf{Program vám
  vrátím k přepracování maximálně dvakrát, pokud je odevzdaný včas!}
 \item Pokud vám jakákoli část zadání nebo obecně cokoli o úkolu není jasné,
  \textbf{pište/přijďte a ptejte se}. Pozdní odevzdání se slovy, \uv{Nechápu, co
  se po mně chtělo.} neberu.
\end{itemize}

\subsection*{Úlohy}
\begin{enumerate}
 \item Napište funkci, která dostane jako parametr \texttt{list} (seznam) čísel
  typu \texttt{int} (celé číslo) a vrátí součet \textbf{všech kladných čísel v
  tomto seznamu} jako \texttt{int}. Pokud je seznam prázdný, vrátí \texttt{0}.\\
  \textbf{Příklad:} pro seznam \texttt{[2, -1, 3, -2, 5]} vaše funkce vrátí
  \texttt{10}.
 \item Napište funkci, která dostane parametrem \texttt{string}, odebere jeho
  první a poslední prvek a výsledek vrátí opět jako \texttt{string}. Můžete
  počítat s tím, že obdržený string má aspoň dva prvky.\\
  \textbf{Hint}: \texttt{for} cyklus prochází \texttt{string} prvek po prvku
  stejně jako třeba seznam.
  \textbf{Příklad:} pro string \texttt{"čuně"} vaše funkce vrátí \texttt{"un"}.
 \item Napište funkci, která dostane parametrem kladné číslo (jako \texttt{int})
  a vrátí součin všech kladných čísel, která jsou mu menší nebo rovna (opět jako
  \texttt{int}). \textbf{Nepoužívejte funkci \texttt{range}!}\\
  \textbf{Příklad:} pro číslo \texttt{7} vaše funkce vrátí \texttt{5040}.
 \item Napište funkci, která dostane parametrem rok (po Kristu, jako
  \texttt{int}) a vrátí století, do kterého je tento rok zařazen (opět jako
  \texttt{int}). Pamatujte, že třeba rok 1705 je v 18. století, ale rok 400 je
  stále ve čtvrtém.\\
  \textbf{Příklad}: pro rok \texttt{1609} vaše funkce vrátí \texttt{17}.
 \item Napište funkci, která najde \uv{jehlu v kupce sena}. Dostane parametrem
  \texttt{list} (seznam) stringů, který obsahuje \textbf{přesně jeden} string
  \texttt{"jehla"}. Vrátí \textbf{pozici/pořadí} v zadaném seznamu, kde se tento
  string nachází (jako \texttt{int}).\\
  \textbf{Příklad:} pro seznam \texttt{["bla", "ano", "jejda", "jehla", "blb"]}
  vaše funkce vrátí \texttt{3}.
 \item Napište funkci, která dostane tři parametry -- hodiny, minuty, vteřiny --
  jako tři celá čísla (\texttt{int}). Tyto parametry značí množství času, které
  uběhlo od půlnoci. Vaše funkce vrátí \textbf{počet vteřin} (jako
  \texttt{int}), který uběhl od půlnoci. Nepředstavuje-li trojice parametrů
  denní čas (například je-li počet hodin větší než 23), vrátí \texttt{-1}.\\
  \textbf{Příklad}: pro vstup \texttt{4, 20, 11} vaše funkce vrátí
  \texttt{15611} a pro vstup \texttt{20, 69, 420} vrátí \texttt{-1}.
 \item Napište funkci, která dostane dva parametry -- prvek (jakéhokoli typu) a
  \texttt{list}. Funkce vrátí \texttt{True}, pokud prvek leží v obdrženém
  seznamu, jinak vrátí \texttt{False}.\\
  \textbf{Příklad:} pro prvek \texttt{7} a seznam \texttt{["nic", 2, 3.5, "zas
  nic"]} vrátí funkce \texttt{False}.
 \item Napište funkci, která dostane jeden parametr -- seznam (\texttt{list})
  stringů -- a vrátí seznam dvojic (\texttt{tuple}) všech stringů v seznamu
  spolu s jejich délkou.\\
  \textbf{Příklad}: pro seznam \texttt{["nic", "slovo", "kočka"]} má vaše funkce
  vrátit \texttt{[("nic", 3), ("slovo", 5), ("kočka", 5)]}.
\end{enumerate}

\end{document}
