\documentclass[a4paper,11pt]{article}

\usepackage[czech,english]{babel}
% Fonts %
\usepackage{fouriernc}
\usepackage[T1]{fontenc}
\usepackage{sectsty}
\allsectionsfont{\sffamily}

% Colors %
\usepackage[dvipsnames]{color}
\usepackage{xcolor}

% Page Layout %
\usepackage[margin=1in]{geometry}

% Fancy Headers %
\usepackage{fancyhdr}
\fancyhf{}
\rhead{}
\cfoot{\thepage}
\renewcommand{\headrulewidth}{0pt}
\setlength{\headheight}{16pt}

% Math
\usepackage{mathtools}
\usepackage{amssymb}
\usepackage{faktor}
\usepackage{import}
\usepackage{caption}
\usepackage{subcaption}
\usepackage{wrapfig}
\usepackage{enumitem}
\usepackage{tikz}
\usetikzlibrary{cd,positioning,babel,shapes,fit}
\usepackage{tkz-base}
\usepackage{tkz-euclide}

% Theorems
\usepackage{thmtools}
\usepackage[thmmarks, amsmath, thref]{ntheorem}

% Title %
% \title{\Huge\textsf{Math Exam -- PreIB 3.AB 2}\\
%  \Large\textsf{Systems of Linear Equations}
%  \author{Áďa Klepáčů}
%  \date{March 6, 2023}
% }

% Table of Contents %
\usepackage{hyperref}
\hypersetup{
 colorlinks=true,
 linktoc=all,
 linkcolor=blue
}

% Tables %
\usepackage{booktabs}
\usepackage{tabularx}
\newcolumntype{Y}{>{\centering\arraybackslash}X}

% Patch for hyphens
\usepackage{regexpatch}
\makeatletter
% Change the `-` delimiter to an active character
\xpatchparametertext\@@@cmidrule{-}{\cA-}{}{}
\xpatchparametertext\@cline{-}{\cA-}{}{}
\makeatother

\newcolumntype{s}{>{\centering\arraybackslash}p{.4\textwidth}}

% Operators %
\DeclareMathOperator{\Ker}{Ker}
\DeclareMathOperator{\Img}{Im}
\DeclareMathOperator{\End}{End}
\DeclareMathOperator{\Aut}{Aut}
\DeclareMathOperator{\Inn}{Inn}

% Common operators %
\newcommand{\R}{\mathbb{R}}
\newcommand{\N}{\mathbb{N}}
\newcommand{\Z}{\mathbb{Z}}
\newcommand{\Q}{\mathbb{Q}}
\newcommand{\C}{\mathbb{C}}

\newcommand{\tr}{\textcolor{red}}
\newcommand{\tb}{\textcolor{blue}}
\newcommand{\tg}{\textcolor{green}}
\newcommand{\tm}{\textcolor{magenta}}
\newcommand{\tv}{\textcolor{violet}}

% American Paragraph Skip %
\setlength{\parindent}{0pt}
\setlength{\parskip}{1em}

% Document %
\pagestyle{fancy}
\begin{document}

% \maketitle
\thispagestyle{fancy}

\section{Jak funguje GEVO E-Shop}
\label{sec:jak-funguje-e-shop}

\subsection{Databáze}
\label{ssec:databaze}

\textbf{Pravidla tvorby databází} (s klesající důležitostí):
\begin{enumerate}[topsep=0pt]
 \item Na údaje z cizí tabulky se \textbf{zásadně odkazuje pomocí
  identifikátoru}, který je jedinečný a dostane ho každý řádek od databáze
  automaticky.
 \item V každé buňce je \textbf{jen jeden údaj}!
 \item V žádné tabulce není \textbf{víc sloupců, než je třeba}!
 \item V žádné tabulce není \textbf{víc řádků, než je třeba}!
\end{enumerate}

Potřebujeme si pamatovat tři typy údajů:
\begin{itemize}[topsep=0pt]
 \item uživatele,
 \item produkty,
 \item košík.
\end{itemize}

U uživatele si potřebujeme pamatovat pouze jméno a heslo. Pozor! Heslo si
pamatuju zásadně nějak zakódované, aby se nedalo z databáze vyčíst, kdyby se k
jejím údajům někdo dostal. Když se uživatel přihlašuje, tak to heslo, co zadá,
taky zakóduju a zkontroluju, že je ten kód stejný jako ten uložený. Košík
\textbf{nepatří} do tabulky uživatelů. Ukládat s uživateli košík by znamenalo v
jedné buňce ukládat seznam produktů i s jejich za\-koupeným množstvím. To
naprosto protiřečí bodu 1 pravidel tvorby databází.

U produktů potřebujeme znát název, cenu, počet kusů na skladě a popis.

Nejtěžší vymyslet je košík. Chybný postup by byl ukládat pro každého uživatele
\textbf{seznam} za\-koupených produktů. Seznam totiž není jeden údaj! Místo toho
bude jeden řádek v tabulce košíku odpovídat jen jednomu produktu. To znamená, že
si musíme pamatovat na každém řádku, o který jde produkt (pomocí
\textbf{identifikátoru} z tabulky produktů), kdo ho koupil (pomocí
identifikátoru z~tabulky uživatelů), a zakoupené množství.

\begin{figure}[h]
 \centering
 \begin{tikzpicture}
  \node[anchor=west] (uziv) at (-5,2) {
   \begin{tabular}{|c|c|c|}
    \multicolumn{3}{l}{\textcolor{red}{Tabulka uživatelů}}\\
    \textsf{\textbf{id}} & \textsf{\textbf{jméno}} & \textsf{\textbf{heslo}}\\
    \toprule
    1 & cr1ng3M3m3r & pE\@\%qXlL8axV\textbackslash1\$P\\
    \midrule
    2 & tvojemama & M5x\textbackslash'\{MjR8yR"J7\textbackslash\\
    \midrule
    3 & bigBadBoi & p/gC8Y3cJi;hZ\}6U
   \end{tabular}
  };
 \node[anchor=west] (prod) at (-5,-2) {
   \begin{tabular}{|c|c|c|c|c|}
    \multicolumn{5}{l}{\textcolor{blue}{Tabulka produktů}}\\
    \textsf{\textbf{id}} & \textsf{\textbf{název}} & \textsf{\textbf{cena}} &
    \textsf{\textbf{na skladě}} & \textsf{\textbf{popis}}\\
    \toprule
    1 & papoušek & 99 & 8 & Umí papouškovat.\\
    \midrule
    2 & prsten & 1399 & 2 & Konec svobody.\\
    \midrule
    3 & 100 \% z IVT & 9999 & 0 & Nope.
   \end{tabular}
  };
 \node[anchor=east] (cart) at (10, 2) {
   \begin{tabular}{|c|c|c|c|}
    \multicolumn{4}{l}{Tabulka košíku}\\
    \textsf{\textbf{id}} & \textsf{\textbf{\textcolor{blue}{id prod.}}} &
    \textsf{\textbf{\textcolor{red}{id uživ.}}} & \textsf{\textbf{množství}}\\
    \toprule
    1 & \textcolor{blue}{2} & \textcolor{red}{1} & 1\\
    \midrule
    2 & \textcolor{blue}{2} & \textcolor{red}{3} & 1\\
    \midrule
    3 & \textcolor{blue}{1} & \textcolor{red}{3} & 5
   \end{tabular}
  };
  \draw[circle,blue,thick] (-4.5, -2.51) circle (.25);
  \draw[circle,blue,thick] (5.25, 1.49) circle (.25);
  \draw[dashed,blue,->,thick] (-4.3,-2.41) -- (5.05, 1.39);

  \draw[circle,red,thick] (-4.5, 0.8) circle (.25);
  \draw[circle,red,thick] (7, 1.49) circle (.25);
  \draw[dashed,red,->,thick] (-4.3, 0.7) to[bend right=20] (6.8, 1.39);
 \end{tikzpicture}
\end{figure}

\subsection{Uživatelské rozhraní}
\label{ssec:uzivatelske-rozhrani}

Uživatelské rozhraní jsou jednoduše řečeno okna a tlačítka. Je třeba vymyslet,
co uživatel uvidí, kam bude moct psát a co se stane když klikne na tlačítko.

V E-Shopu chceme uživateli zobrazit tři okna:
\begin{itemize}[topsep=0pt]
 \item přihlášení/registrace,
 \item seznam produktů,
 \item košík/souhrn objednávky.
\end{itemize}

Okno s přihlášením/registrací může vypadat třeba takhle:
\begin{figure}[h]
 \centering
 \begin{tikzpicture}
  \node[draw,thick,minimum height=5mm,minimum width=4cm] (jmeno) at (0, 0) {};
  \node[draw,thick,minimum height=5mm,minimum width=4cm] (heslo) at (0, -0.75) {};

  \node[left=of jmeno] (jmenolabel) {Jméno:};
  \node[left=of heslo] (heslolabel) {Heslo:};

  \node[draw,red,thick,minimum height=7mm,minimum width=3cm] (prihlasit) at (-3,
   -2) {Přihlásit};
  \node[draw,blue,thick,right=4mm of prihlasit,minimum height=7mm,minimum
   width=3cm] (registrovat) {Registrovat};

  \node[draw,inner sep=4mm,fit=(jmenolabel) (heslo) (prihlasit)] {};
 \end{tikzpicture}
\end{figure}

Při kliknutí na tlačítko \textcolor{red}{Přihlásit} se musejí stát následující
věci.
\begin{figure}[h]
 \centering
 \begin{tikzpicture}[scale=0.8]
  \node (userexist) at (0, 0) {Je uživatel v kolonce \uv{Jméno} v databázi?};

  \node (usererror) at (4, -2) {Ukaž chybu/nabídni registraci.};
  \node (userok) at (-4, -2) {Zakóduj to, co je v kolonce \uv{Heslo}.};

  \node[align=center] (passcheck) at (-4, -4) {Zkontroluj, že kód odpovídá
   tomu uloženému\\ v řádku tabulky uživatelů se zadaným jménem.};

  \node (passerror) at (-2, -6) {Ukaž chybu.};
  \node (passok) at (-6, -6) {Přihlaš uživatele.};

 \draw[->] (userexist) to node[midway,above right] {ne} (usererror);
 \draw[->] (userexist) to node[midway,above left] {ano} (userok);
 \draw[->] (userok) -- (passcheck);
 \draw[->] (passcheck) to node[midway,above left] {ano} (passok);
 \draw[->] (passcheck) to node[midway,above right] {ne} (passerror);
 \end{tikzpicture}
\end{figure}

Při kliknutí na tlačítko \textcolor{blue}{Registrovat} se musí stát toto:
\begin{figure}[h]
 \centering
 \begin{tikzpicture}[scale=0.8]
  \node (userexist) at (0, 0) {Je uživatel v kolonce \uv{Jméno} v databázi?};

  \node (usererror) at (-4, -2) {Ukaž chybu.};
  \node (userok) at (4, -2) {Zakóduj to, co je v kolonce \uv{Heslo}.};

  \node[align=center] (save) at (4, -4) {Vytvoř nový řádek v tabulce uživatelů\\
   se zadaným jménem a zakódovaným heslem.};

  \draw[->] (userexist) to node[midway,above left] {ano} (usererror);
  \draw[->] (userexist) to node[midway,above right] {ne} (userok);
  \draw[->] (userok) -- (save);
 \end{tikzpicture}
\end{figure}

\newpage
Okno s produkty může zas vypadat takhle:
\begin{figure}[h]
 \centering
 \begin{tikzpicture}
  \node (prod1nazev) at (0, 0) {\sffamily Papoušek};
  \node[below=2mm of prod1nazev] (prod1popis) {\footnotesize Umí papouškovat.};
  \node[below=2mm of prod1popis] (prod1cena) {99 Kč s DPH};
  \node[below=1mm of prod1cena] (prod1ks) {Na skladě: 8 ks};

  \node[draw,rectangle,red,below left=3mm and -16.5mm of prod1ks,minimum
   height=7mm, minimum width=2cm] (prod1koupit) {Koupit};
  \node[draw,rectangle,red,right=6mm of prod1koupit,minimum height=7mm,minimum
   width=5mm] (prod1pocet) {};
  \node[right=0mm of prod1pocet] (prod1uzivks) {ks};
  \node[draw,thick,rectangle,inner sep=3mm,xshift=-1mm,fit=(prod1nazev)
   (prod1koupit) (prod1uzivks),minimum height=5cm] (rect1) {};

  \node (prod2nazev) at (5, 0) {\sffamily Prsten};
  \node[below=2mm of prod2nazev] (prod2popis) {\footnotesize Konec svobody.};
  \node[below=2mm of prod2popis] (prod2cena) {1399 Kč s DPH};
  \node[below=1mm of prod2cena] (prod2ks) {Na skladě: 2 ks};

  \node[draw,rectangle,red,below left=3mm and -16.5mm of prod2ks,minimum
   height=7mm, minimum width=2cm] (prod2koupit) {Koupit};
  \node[draw,rectangle,red,right=6mm of prod2koupit,minimum height=7mm,minimum
   width=5mm] (prod2pocet) {};
  \node[right=0mm of prod2pocet] (prod2uzivks) {ks};
  \node[draw,thick,rectangle,inner sep=3mm,xshift=-1mm,fit=(prod2nazev)
   (prod2koupit) (prod2uzivks),minimum height=5cm] (rect2) {};

  \node (prod3nazev) at (10, 0) {\sffamily 100 \% z IVT};
  \node[below=2mm of prod3nazev] (prod3popis) {\footnotesize Nope.};
  \node[below=2mm of prod3popis] (prod3cena) {9999 Kč s DPH};
  \node[below=1mm of prod3cena] (prod3ks) {Na skladě: 0 ks};

  \node[draw,rectangle,red,below left=3mm and -16.5mm of prod3ks,minimum
   height=7mm, minimum width=2cm] (prod3koupit) {Koupit};
  \node[draw,rectangle,red,right=6mm of prod3koupit,minimum height=7mm,minimum
   width=5mm] (prod3pocet) {};
  \node[right=0mm of prod3pocet] (prod3uzivks) {ks};
  \node[draw,thick,rectangle,inner sep=3mm,xshift=-1mm,fit=(prod3nazev)
   (prod3koupit) (prod3uzivks),minimum height=5cm] (rect3) {};

  \node[draw,rectangle,inner sep=6mm,fit=(rect1) (rect3)] {};
 \end{tikzpicture}
\end{figure}

Při kliknutí na tlačítko \textcolor{red}{Koupit} u kteréhokoliv produktu se musí
stát toto:
\begin{figure}[h]
 \centering
 \begin{tikzpicture}
  \node[align=center] (prodenough) at (0, 0) {Je v řádku s tímto produktem v
   tabulce produktů číslo\\ \uv{na skladě} větší než to, které zadal uživatel?};
  \node (proderror) at (4, -2) {Ukaž chybu.};
  \node[align=center] (prodok) at (-4, -2) {Odečti počet zadaný uživatelem ze
   sloupce\uv{na skladě} na\\ řádku odpovídajícím produktu v tabulce produktů.};
  \node[align=center] (cartcheck) at (-4, -4) {Existuje v tabulce košíku řádek s
   ID\\ přihlášeného uživatele a ID tohoto produktu?};
  \node[align=center] (cartexist) at (-1, -7) {Vytvoř nový řádek v tabulce
   košíku\\ s ID přihlášeného uživatele, ID\\ produktu a zakoupeným množstvím.};
  \node[align=center] (cart!exist) at (-7, -7) {Připiš k počtu na
   tomto řádku\\ množství produktu zakoupené\\ přihlášeným uživatelem.};

  \draw[->] (prodenough) to node[midway,above right] {ne} (proderror);
  \draw[->] (prodenough) to node[midway,above left] {ano} (prodok);
  \draw[->] (prodok) -- (cartcheck);
  \draw[->] (cartcheck) to node[midway,above left] {ano} (cart!exist);
  \draw[->] (cartcheck) to node[midway,above right] {ne} (cartexist);
 \end{tikzpicture}
\end{figure}

Košík jsem líný dělat, ten si domyslete sami. Je tam třeba vymyslet, jak funguje
tlačítko \textcolor{red}{Odebrat} u každé položky a jak zobrazit pouze produkty,
které zakoupil přihlášený uživatel.

\section{Váš úkol -- rezervační systém letenek}
\label{sec:vas-ukol-rezervacni-system-letenek}

\begin{center}
 \textcolor{red}{\textbf{ÚKOL SMÍTE VYPRACOVAT VE DVOJICÍCH!}}
\end{center}

Máte za úkol vymyslet \textbf{databázovou strukturu} a \textbf{rozhraní} k
zjednodušenému programu na rezervaci letenek. Návrh obojího může být samozřejmě
schematický a používat obrázky, podobně jako jsem popisoval ten e-shop.

Program musí umět pracovat s následujícím.
\begin{enumerate}[topsep=0pt]
 \item \textbf{uživatelské účty}
 \begin{enumerate}[topsep=0pt]
  \item Uživatel se může registrovat a posléze přihlásit.
  \item Uživatel má pod svým účtem uloženy zakoupené letenky.
  \item Uživatel si musí se zakoupením letenky zvolit místo v letadle, nebo
   zvolit mož\-nost automatického výběru.
  \item Uživatel má možnost změnit vybrané místo v letadle u každé rezervované
   letenky na kterékoliv volné. \textcolor{red}{\textbf{Třídy (Business,
   Economy, ...) neřešíme!}}
  \item Uživatel smí své letenky za poplatek (závislý na době před
   odletem) stornovat, \textbf{pokud před odletem zbývají méně než dva dny!}
  \item \textbf{\textcolor{red}{Check-in (tj. víza, pasy, zavazadla apod.)
   neřešíme!}}
 \end{enumerate}
 \item \textbf{seznam letů}
 \begin{enumerate}[topsep=0pt]
  \item \textcolor{red}{\textbf{Přestupy a zpáteční letenky neřešíme! Všechna
   letadla létají přímo.}}
  \item Program musí umět zobrazit lety na základě:
  \begin{itemize}[topsep=0pt]
   \item ceny letenky,
   \item počtu volných míst (když si uživatel přeje rezervovat let pro více
    osob),
   \item data a času odletu,
   \item data a času příletu,
   \item místa odletu,
   \item místa příletu,
   \item letecké společnosti,
   \item libovolné kombinace těchto údajů.
  \end{itemize}
  \item Program neukazuje letadla, ve kterých už není místo.
 \end{enumerate}
 \item \textbf{letenky}
 \begin{enumerate}[topsep=0pt]
  \item Cena každé (nezakoupené) letenky se snižuje, čím méně času zbývá do
   odletu.
  \item Ke každé letence se váže místo v odpovídajícím letadle.
  \item Každou zakoupenou letenku lze stornovat až do doby dvou dnů před odletem
   za po\-platek zvyšující se s úbytkem času před odletem.
 \end{enumerate}
\end{enumerate}

\end{document}
