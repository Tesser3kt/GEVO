\documentclass[a4paper,11pt]{article}

\usepackage[czech,english]{babel}
% Fonts %
\usepackage{fouriernc}
\usepackage[T1]{fontenc}
\usepackage{sectsty}
\allsectionsfont{\sffamily}

% Colors %
\usepackage[dvipsnames]{color}
\usepackage{xcolor}

% Page Layout %
\usepackage[margin=1in]{geometry}

% Fancy Headers %
\usepackage{fancyhdr}
\fancyhf{}
\rhead{}
\renewcommand{\headrulewidth}{0pt}
\setlength{\headheight}{16pt}

% Math
\usepackage{mathtools}
\usepackage{amssymb}
\usepackage{faktor}
\usepackage{import}
\usepackage{caption}
\usepackage{subcaption}
\usepackage{wrapfig}
\usepackage{enumitem}
\usepackage{tikz}
\usetikzlibrary{cd,positioning,babel,shapes}
\usepackage{tkz-base}
\usepackage{tkz-euclide}

% Theorems
\usepackage{thmtools}
\usepackage[thmmarks, amsmath, thref]{ntheorem}

% Title %
% \title{\Huge\textsf{Math Exam -- PreIB 3.AB 2}\\
%  \Large\textsf{Systems of Linear Equations}
%  \author{Áďa Klepáčů}
%  \date{March 6, 2023}
% }

% Table of Contents %
\usepackage{hyperref}
\hypersetup{
 colorlinks=true,
 linktoc=all,
 linkcolor=blue
}

% Tables %
\usepackage{booktabs}
\usepackage{tabularx}
\newcolumntype{Y}{>{\centering\arraybackslash}X}

% Patch for hyphens
\usepackage{regexpatch}
\makeatletter
% Change the `-` delimiter to an active character
\xpatchparametertext\@@@cmidrule{-}{\cA-}{}{}
\xpatchparametertext\@cline{-}{\cA-}{}{}
\makeatother

\newcolumntype{s}{>{\centering\arraybackslash}p{.4\textwidth}}

% Operators %
\DeclareMathOperator{\Ker}{Ker}
\DeclareMathOperator{\Img}{Im}
\DeclareMathOperator{\End}{End}
\DeclareMathOperator{\Aut}{Aut}
\DeclareMathOperator{\Inn}{Inn}

% Common operators %
\newcommand{\R}{\mathbb{R}}
\newcommand{\N}{\mathbb{N}}
\newcommand{\Z}{\mathbb{Z}}
\newcommand{\Q}{\mathbb{Q}}
\newcommand{\C}{\mathbb{C}}

\newcommand{\tr}{\textcolor{red}}
\newcommand{\tb}{\textcolor{blue}}
\newcommand{\tg}{\textcolor{green}}
\newcommand{\tm}{\textcolor{magenta}}
\newcommand{\tv}{\textcolor{violet}}

% American Paragraph Skip %
\setlength{\parindent}{0pt}
\setlength{\parskip}{1em}

% Document %
\pagestyle{fancy}
\begin{document}

% \maketitle
\thispagestyle{fancy}

\section{Jak funguje GEVO E-Shop}
\label{sec:jak-funguje-e-shop}

\subsection{Databáze}
\label{ssec:databaze}

\textbf{Pravidla tvorby databází} (s klesající důležitostí):
\begin{enumerate}[topsep=0pt]
 \item Na údaje z cizí tabulky se \textbf{zásadně odkazuje pomocí
  identifikátoru}, který je jedinečný a dostane ho každý údaj od databáze
  automaticky.
 \item V každé buňce je \textbf{jen jeden údaj}!
 \item V žádné tabulce není \textbf{víc sloupců, než je třeba}!
 \item V žádné tabulce není \textbf{víc řádků, než je třeba}!
\end{enumerate}

Potřebujeme si pamatovat tři typy údajů:
\begin{itemize}[topsep=0pt]
 \item uživatele,
 \item produkty,
 \item košík.
\end{itemize}

U uživatele si potřebujeme pamatovat pouze jméno a heslo. Pozor! Heslo si
pamatuju zásadně nějak zakódované, aby se nedalo z databáze vyčíst, kdyby se k
jejím údajům někdo dostal. Když se uživatel přihlašuje, tak to heslo, co zadá,
taky zakóduju a zkontroluju, že je ten kód stejný jako ten uložený. Košík
\textbf{nepatří} do tabulky uživatelů. Ukládat s uživateli košík by znamenalo v
jedné buňce ukládat seznam produktů i s jejich za\-koupeným množstvím. To
naprosto protiřečí bodu 1 pravidel tvorby databází.

U produktů potřebujeme znát název, cenu, počet kusů na skladě a popis.

Nejtěžší vymyslet je košík. Chybný postup by byl ukládat pro každého uživatele
\textbf{seznam} za\-koupených produktů. Seznam totiž není jeden údaj! Místo toho
bude jeden řádek v tabulce košíku odpovídat jen jednomu produktu. To znamená, že
si musíme pamatovat na každém řádku, o který jde produkt (pomocí
\textbf{identifikátoru} z tabulky produktů), kdo ho koupil (pomocí
identifikátoru z~tabulky uživatelů), a zakoupené množství.

\begin{figure}[h]
 \centering
 \begin{tikzpicture}
  \node[anchor=west] (uziv) at (-5,2) {
   \begin{tabular}{|c|c|c|}
    \multicolumn{3}{l}{Tabulka uživatelů}\\
    \textsf{\textbf{id}} & \textsf{\textbf{jméno}} & \textsf{\textbf{heslo}}\\
    \toprule
    1 & cr1ng3M3m3r & pE\@\%qXlL8axV\textbackslash1\$P\\
    \midrule
    2 & tvojemama & M5x\textbackslash'\{MjR8yR"J7\textbackslash\\
    \midrule
    3 & bigBadBoy & p/gC8Y3cJi;hZ\}6U
   \end{tabular}
  };
 \node[anchor=west] (prod) at (-5,-2) {
   \begin{tabular}{|c|c|c|c|c|}
    \multicolumn{5}{l}{Tabulka produktů}\\
    \textsf{\textbf{id}} & \textsf{\textbf{název}} & \textsf{\textbf{cena}} &
    \textsf{\textbf{na skladě}} & \textsf{\textbf{popis}}\\
    \toprule
    1 & papoušek & 99 & 8 & Umí papouškovat.\\
    \midrule
    2 & prsten & 1399 & 2 & Konec svobody.\\
    \midrule
    3 & 100 \% z IVT & 9999 & 0 & Nope.
   \end{tabular}
  };
 \node[anchor=east] (cart) at (10, 2) {
   \begin{tabular}{|c|c|c|c|}
    \multicolumn{4}{l}{Tabulka košíku}\\
    \textsf{\textbf{id}} & \textsf{\textbf{id prod.}} & \textsf{\textbf{id
    uživ.}} & \textsf{\textbf{množství}}\\
   \end{tabular}
  }
 \end{tikzpicture}
\end{figure}

\end{document}
