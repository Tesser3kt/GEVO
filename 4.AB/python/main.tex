% Modelo de slides para projetos de disciplinas do Abel
\documentclass[aspectratio=169,11pt]{beamer}

\usetheme[%
 sectionpage=none,%
 subsectionpage=progressbar%
 ]{metropolis}
\useoutertheme{infolines}
\setbeamersize{text margin left=1cm,text margin right=1cm}
\setbeamertemplate{part page}
{
  \begin{centering}
    \begin{beamercolorbox}[sep=16pt,center]{part title}
      \usebeamerfont{part title}\insertromanpartnumber.~\insertpart\par
    \end{beamercolorbox}
  \end{centering}
}

\usepackage[czech]{babel}
\usepackage{graphicx}
\usepackage{enumitem}
\usepackage{amsmath}
\usepackage{mathtools}
\usepackage{tcolorbox}
\usepackage[linesnumbered,vlined,commentsnumbered,czech]{algorithm2e}
\SetKwInOut{Input}{input}\SetKwInOut{Output}{output}
\SetKw{KwReturn}{return}\SetKw{KwFrom}{from}
\SetKw{KwAnd}{and}\SetKw{KwOr}{or}
% skip line number
\let\oldnl\nl
\newcommand{\nonl}{\renewcommand{\nl}{\let\nl\oldnl}}

\tcbset{%
 sharp corners=all,%
 boxsep=7pt,%
 fonttitle=\bfseries,%
 colback=gray!30!white,%
 colframe=mDarkTeal,%
 boxrule=1pt,%
 center,%
 width=.9\textwidth%
}

\title{PYTHON}
\date{\today}
\author{Adam Klepáč}
\institute[GEVO]{Gymnázium Evolution Jižní Město}

% enumerate global settings
\setlist[enumerate,1]{label=\arabic*.}
\setlist[enumerate,2]{label=\alph*)}
\setlist[itemize,1]{label=\textbullet}

\begin{document}

\maketitle

\part[Programovací jazyky]{Programovací jazyky}

\begin{frame}[plain]
 \partpage
\end{frame}

\begin{frame}
 \frametitle{Obsah}
 \tableofcontents
\end{frame}

\section[Jak mluvit s počítačem?]{Jak mluvit s počítačem?}

\begin{frame}{Nejnižší forma komunikace}
 \begin{tcolorbox}[title=Strojový kód]
  Strojový kód je jazyk sestávající pouze ze \alert{základních instrukcí} pro
  CPU.
 \end{tcolorbox}
\end{frame}

\begin{frame}{Nejvyšší forma komunikace}
 \begin{tcolorbox}[title=Programovací jazyk]
  Programovací jazyk je jakýkoli jazyk, který lze \alert{automaticky přeložit}
  do strojového kódu.
 \end{tcolorbox}
\end{frame}

\begin{frame}{Nač programovací jazyky?}
 \begin{itemize}
  \item<1-> Strojový kód je člověku nečitelný.
  \item<2-> Programovací jazyky se čím dál více přibližují lidské řeči.
  \item<3-> Programovací jazyky jsou rozšířitelné -- umožňují přidání nových
   konceptů (proměnných, podmínek, ...)
  \item<4-> V programovacích jazycích lze některé běžné paměťové operace CPU
   automatizovat (rekurze, garbage collector, ...)
 \end{itemize}
\end{frame}

\begin{frame}{Jaká je cena?}
 \begin{itemize}
  \item<1-> Překlad prog. jazyků je automatický -- vzniká spousta přebytečného
   strojového kódu.
  \item<2-> Přebytečné instrukce zpomalují běh programu.
  \item<3-> Velká práce s údržbou -- každá nová funkce programovací jazyka
   vyžaduje mnoho testování správnosti překladu do stroj. kódu
  \item<4-> V různých jazycích jsou stejné funkce psané jinak.
 \end{itemize}
\end{frame}

\begin{frame}{Typy programovací jazyků}
 \begin{enumerate}[label=(\arabic*)]
  \item<1-> strojový kód,
  \item<2-> assemblery (jazyky symbolických adres):
   \begin{itemize}
    \item symbolické reprezentace CPU instrukcí
    \item zkratky pro běžné operace
    \item žádná automatizace
   \end{itemize}
  \item<3-> high-level (vysokoúrovňové) programovací jazyky:
   \begin{itemize}
    \item pokročilé řídící sekvence -- proměnné, podmínky, cykly, ...
    \item automatická správa běhu -- procedury, funkce
    \item částečně automatická správa paměti -- pole, třídy, ...
   \end{itemize}
 \end{enumerate}
\end{frame}

\begin{frame}[plain]
 \centering\Huge Díky za pozornost.
\end{frame}

\end{document}
