% Modelo de slides para projetos de disciplinas do Abel
\documentclass[aspectratio=169,11pt]{beamer}

\usetheme[%
 sectionpage=none,%
 subsectionpage=progressbar%
 ]{metropolis}
\useoutertheme{infolines}
\setbeamersize{text margin left=1cm,text margin right=1cm}
\setbeamertemplate{part page}
{
  \begin{centering}
    \begin{beamercolorbox}[sep=16pt,center]{part title}
      \usebeamerfont{part title}\insertromanpartnumber.~\insertpart\par
    \end{beamercolorbox}
  \end{centering}
}

\usepackage[czech]{babel}
\usepackage{graphicx}
\usepackage{enumitem}
\usepackage{amsmath}
\usepackage{mathtools}
\usepackage{tcolorbox}
\usepackage[linesnumbered,vlined,commentsnumbered,czech]{algorithm2e}
\SetKwInOut{Input}{input}\SetKwInOut{Output}{output}
\SetKw{KwReturn}{return}\SetKw{KwFrom}{from}
\SetKw{KwAnd}{and}\SetKw{KwOr}{or}
% skip line number
\let\oldnl\nl
\newcommand{\nonl}{\renewcommand{\nl}{\let\nl\oldnl}}

\tcbset{%
 sharp corners=all,%
 boxsep=7pt,%
 fonttitle=\bfseries,%
 colback=gray!30!white,%
 colframe=mDarkTeal,%
 boxrule=1pt,%
 center,%
 width=.9\textwidth%
}

\title{PYTHON}
\date{\today}
\author{Adam Klepáč}
\institute[GEVO]{Gymnázium Evolution Jižní Město}

% enumerate global settings
\setlist[enumerate,1]{label=\arabic*.}
\setlist[enumerate,2]{label=\alph*)}
\setlist[itemize,1]{label=\textbullet}
\setlist[itemize,2]{label=$\circ$}

\begin{document}

\maketitle

\section[Programovací jazyky]{Jak mluvit s počítačem?}

\begin{frame}[plain]
 \sectionpage
\end{frame}

\begin{frame}{Nejnižší forma komunikace}
 \begin{tcolorbox}[title=Strojový kód]
  Strojový kód je jazyk sestávající pouze ze \alert{základních instrukcí} pro
  CPU.
 \end{tcolorbox}
\end{frame}

\begin{frame}{Nejvyšší forma komunikace}
 \begin{tcolorbox}[title=Programovací jazyk]
  Programovací jazyk je jakýkoli jazyk, který lze \alert{automaticky přeložit}
  do strojového kódu.
 \end{tcolorbox}
\end{frame}

\begin{frame}{Nač programovací jazyky?}
 \begin{itemize}
  \item<1-> Strojový kód je člověku nečitelný.
  \item<2-> Programovací jazyky se čím dál více přibližují lidské řeči.
  \item<3-> Programovací jazyky jsou rozšířitelné -- umožňují přidání nových
   konceptů (proměnných, podmínek, ...)
  \item<4-> V programovacích jazycích lze některé běžné paměťové operace CPU
   automatizovat (rekurze, garbage collector, ...)
 \end{itemize}
\end{frame}

\begin{frame}{Jaká je cena?}
 \begin{itemize}
  \item<1-> Překlad prog. jazyků je automatický -- vzniká spousta přebytečného
   strojového kódu.
  \item<2-> Přebytečné instrukce zpomalují běh programu.
  \item<3-> Velká práce s údržbou -- každá nová funkce programovací jazyka
   vyžaduje mnoho testování správnosti překladu do stroj. kódu
  \item<4-> V různých jazycích jsou stejné funkce psané jinak.
 \end{itemize}
\end{frame}

\begin{frame}{Typy programovací jazyků}
 \begin{enumerate}[label=(\arabic*)]
  \item<1-> strojový kód,
  \item<2-> assemblery (jazyky symbolických adres):
   \begin{itemize}
    \item symbolické reprezentace CPU instrukcí
    \item zkratky pro běžné operace
    \item žádná automatizace
   \end{itemize}
  \item<3-> high-level (vysokoúrovňové) programovací jazyky:
   \begin{itemize}
    \item pokročilé řídící sekvence -- proměnné, podmínky, cykly, ...
    \item automatická správa běhu -- procedury, funkce
    \item částečně automatická správa paměti -- pole, třídy, ...
   \end{itemize}
 \end{enumerate}
\end{frame}

\begin{frame}{Kam patří Python?}
 \begin{itemize}
  \item<1-> Python je high-level programovací jazyk.
  \item<2-> Python $\rightarrow$ C $\rightarrow$ (Assembly $\rightarrow$) stroj.
   kód 
  \item<3-> Python je \alert{interpretovaný} (vs. kompilovaný) programovací
   jazyk -- to znamená, že počítač překládá Python za běhu programu.
  \item<4-> Python má automatickou správu paměti a dokonce vás ani nenutí
   typovat.
 \end{itemize}
\end{frame}

\part[Programování v Pythonu]{Programování v Pythonu}

\begin{frame}[plain]
 \partpage
\end{frame}

\section[Datové typy]{Datové typy}

\begin{frame}[plain]
 \sectionpage
\end{frame}

\begin{frame}{Co to je?}
 \begin{tcolorbox}[title=Datový typ]
  \alert{Datový typ} je doslova typ (forma, podoba, ...) informace uložené
  v~paměti počítače.
 \end{tcolorbox}
 \pause
 \begin{itemize}
  \item<2-> Narozdíl od pseudokódu, v programovacích jazycích musíte kromě názvu
   proměnné uvádět i její typ.
  \item<3-> Základní typy v Pythonu jsou \texttt{int, float, str, set, list,
   tuple, dict}
 \end{itemize}
\end{frame}

\begin{frame}{Měnné vs. neměnné}
 \begin{itemize}
  \item<1-> Python rozlišuje mezi \alert{měnnými} (mutable) a \alert{neměnnými}
   (immutable) datovými typy.
  \item<2-> Do struktury měnných typů (seznamy, slovníky, ...) můžete zasahovat
   během programu, ale do struktury neměnných (čísla, slova, ...) nikoliv.
 \end{itemize}
\end{frame}

\subsection[Číselné typy]{Číselné typy}

\begin{frame}{Celá čísla}
 \begin{tcolorbox}[title=Datový typ \texttt{int}]
  Zkratkou \alert{\texttt{int}} (z angl. \alert{int}eger) Python označuje typ
  celých čísel, tj. čísel bez desetinné části.
 \end{tcolorbox}
\end{frame}

\begin{frame}{Celá čísla}
 Python umí následující operace na celých číslech.
 \begin{itemize}
  \item součet (\texttt{+});
  \item rozdíl (\texttt{-});
  \item součin (\texttt{*});
  \pause
  \item<2-> celočíselný podíl (\texttt{//}), např. \texttt{11 // 3 == 3};
  \item<3-> zbytek po dělení (\texttt{\%}), např. \texttt{11 \% 3 == 2};
  \item<4-> mocninu (\texttt{**}), např. \texttt{4 ** 3 == 64}.
 \end{itemize}
\end{frame}

\begin{frame}{Desetinná čísla}
 \begin{tcolorbox}[title=Datový typ \texttt{float}]
  Zkratka \alert{\texttt{float}} (z angl. \alert{float}ing point) označuje v
  Pythonu typ desetinných čísel.
 \end{tcolorbox}
 \pause
 \emph{Poznámka.} Celá čísla jsou samozřejmě zároveň desetinná. Aby je Python v
 tomto případě rozlišil, píše \texttt{2.0} pro \uv{desetinné číslo} dva a
 \texttt{2} pro \uv{celé číslo} dva.
\end{frame}

\begin{frame}{Desetinná čísla}
 Python umí následující operace na desetinných číslech.
 \begin{itemize}
  \item součet (\texttt{+});
  \item rozdíl (\texttt{-});
  \item součin (\texttt{*});
  \item podíl (\texttt{/});
  \item mocninu (\texttt{**}).
 \end{itemize}
\end{frame}

\begin{frame}{celá $\leftrightarrow$ desetinná} 
 \begin{itemize}
  \item<1-> Slova \texttt{int} a \texttt{float} jsou zároveň názvy
   funkcí/procedur v Pythonu pro převod mezi číselnými typy.
  \item<2-> \texttt{\alert{int}(x: float)} vrátí tzv. \uv{celou část} z
   \texttt{x}; např. \texttt{\alert{int}(3.9) == 3}.
  \item<3-> \texttt{\alert{float}(x: int)} převede celé číslo \texttt{x} na
   desetinné prostě tak, že k němu přidá \uv{\texttt{.0}}. Takže třeba
   \texttt{\alert{float}(3) == 3.0}.
 \end{itemize}
\end{frame}

\section[Řetězce]{Řetězce}

\begin{frame}[plain]
 \sectionpage
\end{frame}

\begin{frame}{Řetězce (stringy)}
 \begin{tcolorbox}[title=Datový typ \texttt{str}]
  Zkratkou \alert{\texttt{str}} (z angl. \alert{str}ing) Python označuje typ
  \uv{řetězců znaků}, tj. posloupností v zásadě libovolných symbolů.
 \end{tcolorbox}
 \begin{itemize}
  \item<2-> Stringy se píší do uvozovek, buď jednoduchých
   (\texttt{\textquotesingle}) nebo dvojitých (\texttt{"}). Na výběru nezáleží,
   ale string musí začínat končit stejnou uvozovkou.
  \item<3-> Python používá pro kódování textu UTF-8 (\textbf{U}nicode
   \textbf{T}ransformation \textbf{F}ormat -- 8-bit). Tedy umí rozpoznat každý
   znak v tomto kódování.
 \end{itemize}\end{frame}

\begin{frame}{Řetězce (stringy)}
 Python umí následující operace na řetězcích.
 \begin{itemize}
  \item<1-> součet/spojení (\texttt{+} nebo mezera)
   \begin{itemize}
    \item např. \texttt{"auto"} \texttt{+} \texttt{"bus"} \texttt{==}
     \texttt{"autobus"}
    \item např. \texttt{"mrt"} \texttt{"vola"} \texttt{==} \texttt{"mrtvola"}
   \end{itemize}
  \item<2-> součin/opakování (\texttt{*}): např. \texttt{"hehe"} \texttt{* 3}
   \texttt{==} \texttt{"hehehehehehe"}
  \item<3-> výběr prvku (\texttt{str[pořadí prvku]}): např. \texttt{"python"[2]}
   \texttt{==} \texttt{"t"}.\\
   \alert{Pozor!} Python čísluje od 0.
 \end{itemize}
\end{frame}

\begin{frame}{stringy $\leftrightarrow$ čísla}
 \begin{itemize}
  \item Zkratka \texttt{str} je zároveň procedura na převod dané proměnné na
   string. V případě čísel máme
   \begin{itemize}
    \item \texttt{str(x: int)} převede celé číslo na string. Třeba
     \texttt{str(3)} \texttt{==} \texttt{"3"}.
    \item \texttt{str(x: float)} převede desetinné číslo na string. Např. 
     \texttt{str(3.14159)} \texttt{==} \texttt{"3.14159"}.
   \end{itemize}
   \alert{Pozor!} Python neřeší, jestli je ve stringu číslo. Takže třeba
   \texttt{"1"} \texttt{+} \texttt{"1"} \texttt{==} \texttt{"11"}, ale
   \texttt{1} \texttt{+} \texttt{1} \texttt{==} \texttt{2}.
 \end{itemize}
\end{frame}

\begin{frame}{stringy $\leftrightarrow$ čísla}
 Procedury \texttt{int} a \texttt{float} taky převádějí stringy na číslo, pokud
 to lze. Např.
 \begin{itemize}
  \item<1-> \texttt{int("69")} \texttt{==} \texttt{69},
  \item<2-> \texttt{float("3.14159")} \texttt{==} \texttt{3.14159}, ale
  \item<3-> \texttt{float("hehe")} i \texttt{int("9.11")} hodí chybu.
 \end{itemize}
\end{frame}

\section[Seznamy]{Seznamy}

\begin{frame}[plain]
 \sectionpage
\end{frame}

\begin{frame}
 \frametitle{Seznamy}
 \begin{tcolorbox}[title=Datový tip \texttt{list}]
  Slovem \texttt{\alert{list}} označuje Python seznam; vlastně množinu, kde
  každý prvek má jednoznačné pořadí. \textbf{Prvky v seznamu mohu nahrazovat,
  přidávat a odebírat}.
 \end{tcolorbox}
 \pause
 \begin{itemize}
  \item<2-> Seznamy se píší do \alert{hranatých závorek} \texttt{[]} a prvky
   oddělují čárkami. Třeba \texttt{[2,~"hora", 4, 7]} je seznam se čtyřmi
   prvky.
  \item<3-> Prvkem seznamu může být další seznam. Třeba  \texttt{[1, [2, "tři"],
   4]} je seznam, jehož druhým prvkem je seznam \texttt{[2, "tři"]}.
 \end{itemize}
\end{frame}

\begin{frame}
 \frametitle{Seznamy}
 Python umí následující operace na seznamech.
 \begin{itemize}
  \item<1-> součet/spojení (\texttt{+})
   \begin{itemize}
    \item např. \texttt{[69, 420]} \texttt{+} \texttt{[911, 1337]} \texttt{==}
     \texttt{[69, 420, 911, 1337]}.
   \end{itemize}
  \item<2-> součin/opakování (\texttt{*})
   \begin{itemize}
    \item např. \texttt{[1, 2]} \texttt{*} \texttt{4} \texttt{==} \texttt{[1, 2,
     1, 2, 1, 2, 1, 2]}.
   \end{itemize}
  \item<3-> výběr prvku (\texttt{list[pořadí prvku]})
   \begin{itemize}
    \item např. \texttt{[1, 4, 7, "hroch"][2]} \texttt{==} \texttt{7}.
    \item např. \texttt{[1, 4, 7, "hroch"][-1]} \texttt{==} \texttt{"hroch"}.
   \end{itemize}
   \alert{Pozor!} Python čísluje buď od \texttt{0} nahoru (začátek $\rightarrow$
   konec) nebo od \texttt{-1} dolu (konec $\rightarrow$ začátek)
 \end{itemize}
\end{frame}

\section[N-tice]{N-tice}

\begin{frame}[plain]
 \sectionpage
\end{frame}

\begin{frame}
 \frametitle{N-tice}
 \begin{tcolorbox}[title=Datový typ \texttt{tuple}]
  Slovem \texttt{\alert{tuple}} Python označuje n-tici, neboli posloupnost n
  prvků. \textbf{Prvky n-tice nemohu nahrazovat, přidávat ani odebírat}.
 \end{tcolorbox}
 \begin{itemize}
  \item<2-> N-tice se píší buď kulatou závorkou s prvky oddělenými čárkou, třeba
   \texttt{(1, 2)}, nebo v mnoha případech i bez závorky, třeba \texttt{1, 2}.
  \item<3-> Python umí stejné operace na n-ticích jako na seznamech.
 \end{itemize}
\end{frame}

\section[Slovníky]{Slovníky}

\begin{frame}[plain]
 \sectionpage
\end{frame}

\begin{frame}
 \frametitle{Slovníky}
 \begin{tcolorbox}[title=Datový typ \texttt{dict}]
  Zkratkou \texttt{\alert{dict}} (z angl. \alert{dict}ionary) označuje Python
  typ slovníku, tj. množiny hodnot, které jsou zařazeny pod klíči.
  \textbf{Slovník umožňuje nahrazování, přidávání i odebírání klíčů i hodnot}.
 \end{tcolorbox}
 \begin{itemize}
  \item<2-> Slovník se píše do složených závorek \texttt{\{\}} a prvky jsou v
   podobě \texttt{klíč: hodnota} odděleny čárkami. Např. \texttt{\{(1, 2):
   "kočka", 3: [4, 5], "pes": 6\}}.
  \item<3-> Hodnotou může být cokoli, ale klíč musí být \alert{neměnný datový
   typ} (číslo, slovo, n-tice apod.).
 \end{itemize}
\end{frame}

\begin{frame}
 \frametitle{Slovníky}
 Slovníky nelze sčítat/spojovat ani násobit/opakovat. Jedinou základní operací
 je výběr prvku příkazem \texttt{dict[}klíč\texttt{]}. Pár příkladů:
 \begin{itemize}
  \item<2-> \texttt{\{"kočka": 2, "pes": 3\}["pes"]} \texttt{==} \texttt{3}.
  \item<3-> \texttt{\{(1, 2, 3): "teď", 4: 5\}[(1, 2, 3)]} \texttt{==}
   \texttt{"teď"}.
  \item<4-> \texttt{\{0: "nula", 1: "jedna", 2: "dva"\}[1]} \texttt{==}
   \texttt{"jedna"}.
 \end{itemize}
\end{frame}

\section[\texttt{list}, \texttt{tuple} a \texttt{dict} jako
procedury]{\texttt{list}, \texttt{tuple} a \texttt{dict} jako procedury}

\begin{frame}[plain]
 \sectionpage
\end{frame}

\begin{frame}
 \frametitle{\texttt{list} jako procedura}
 Procedure/funkce \texttt{\alert{list}} umožňuje převod jiného datového typu na
 seznam, pokud to (podle Pythonu) dává smysl. Obecné pravidlo je, že Python umí
 převést na seznam jen ty datové typy, \textbf{které jsou číslované}.
 \begin{itemize}
  \item<1-> \texttt{list(x: \alert{int}|\alert{float})} hodí chybu,
  \item<2-> \texttt{list(x: \alert{str})} převede řetězec na seznam jeho znaků,
   \begin{itemize}
    \item např. \texttt{list("kočka")} \texttt{==} \texttt{["k", "o", "č", "k",
     "a"]}.
   \end{itemize}
  \item<3-> \texttt{list(x: \alert{tuple})} převede n-tici na seznam se stejnými
   prvky,
   \begin{itemize}
    \item např. \texttt{list((1, 2, 3))} \texttt{==} \texttt{[1, 2, 3]}.
   \end{itemize}
  \item<4-> \texttt{list(x: \alert{dict})} převede slovník na seznam klíčů.
   \begin{itemize}
    \item např. \texttt{list(\{"pes": "haf", "kočka": "mňau"\}} \texttt{==}
     \texttt{["pes", "kočka"]}.
   \end{itemize}
 \end{itemize}
\end{frame}

\begin{frame}{\texttt{tuple} jako procedura}
 Procedura/funkce \alert{\texttt{tuple}} funguje v zásadě stejně jako
 \texttt{list}. Tzn.
 \pause
 \begin{itemize}
  \item \texttt{tuple(x: \alert{int}|\alert{float})} hodí chybu,
  \item \texttt{tuple(x: \alert{str})} udělá z řetězce n-tici jeho symbolů,
  \item \texttt{tuple(x: \alert{list})} převede seznam na n-tici se stejnými
   prvky.
  \item \texttt{tuple(x: \alert{dict})} převede slovník na n-tici jeho klíčů.
 \end{itemize}
\end{frame}

\begin{frame}{\texttt{dict} jako procedura}
 Procedura/funkce \alert{dict} lze použít pouze na převod seznamu nebo n-tice,
 jejichž \textbf{každý prvek má délku 2} (tj. dvojice nebo seznam o dvou
 prvcích). Příklady:
 \begin{itemize}
  \item \texttt{dict([("pes", 2), ("kočka", 3)])} \texttt{==} \texttt{\{"pes":
   2, "kočka": 3\}}.
  \item \texttt{dict((["pes", 2], ["kočka", 3]))} \texttt{==} \texttt{\{"pes":
   2, "kočka": 3\}}.
 \end{itemize}
\end{frame}

\begin{frame}[plain]
 \centering\Huge Díky za pozornost.
\end{frame}

\end{document}
