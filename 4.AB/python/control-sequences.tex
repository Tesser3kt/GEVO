\begin{frame}{Co tím myslím?}
 \begin{block}{Co myslím řídící sekvencí.}
  \begin{itemize}
   \item<1-> podmínky (\alert{if} $\rightarrow$ \alert{elif} $\rightarrow$
    \alert{else}),
   \item<2-> cykly (\alert{for} nebo \alert{while}),
   \item<3-> procedury/funkce (\alert{def}).
  \end{itemize}
 \end{block}
\end{frame}

\subsection[Proměnné]{Proměnné}

\begin{frame}{Proměnné v Pythonu}
 \begin{itemize}
  \item<1-> Proměnné v Pythonu se dají pojmenovat v podstatě jakoukoli
   posloupností znaků (až na výjimky).
  \item<2-> Nemusíte Pythonu říkat, jaký má proměnná datový typ; on si to určí
   sám.
  \item<3-> Táž proměnná může být v průběhu programu různých typů.
  \item<4-> Hodnota se do proměnné ukládá jednoduchým \texttt{=}.\\
   \alert{Pozor!} Tohle \texttt{=} \textbf{nemá nic společného se stejným
   symbolem v matematice}. Čte se \alert{zprava doleva}.
   \begin{itemize}
    \item např. \texttt{\mlb{number}} \texttt{=} \texttt{\mlg{3}} znamená \uv{do
     \texttt{\mlb{number}} dosaď \texttt{\mlg{3}}} a
    \item \texttt{\mlb{number} = \mlg{number + 2}} znamená \uv{do
     \texttt{\mlb{number}} dosaď \texttt{\mlg{number + 2}}}.
   \end{itemize}
 \end{itemize}
\end{frame}

\begin{frame}{Proměnné -- příklady}
 \begin{block}{Příklad s čísly}
  \vspace{6pt}
  Program\\
  $\vcenter{\rule{.5\textwidth}{0.4pt}}$\\
  \texttt{\mlb{first\_number} = 4}\\
  \texttt{\mlg{second\_number} = 5}\\
  \texttt{print(\mlb{first\_number} * \mlg{second\_number})}\\
  $\vcenter{\rule{.5\textwidth}{0.4pt}}$\\
  vytiskne \texttt{20}.
 \end{block}
\end{frame}

\begin{frame}{Proměnné -- příklady}
 \begin{block}{Příklad se stringy}
  \vspace{6pt}
  Program\\
  $\vcenter{\rule{.55\textwidth}{0.4pt}}$\\
  \texttt{\mlb{first_word} = "kocour"}\\
  \texttt{\mlg{second_word} = "kočka"}\\
  \texttt{print(\mlb{first_word}[3] + \mlg{second_word}[-2])}\\
  $\vcenter{\rule{.55\textwidth}{0.4pt}}$\\
  vytiskne \texttt{"ok"}.
 \end{block}
\end{frame}

\begin{frame}{Proměnné -- příklady}
 \begin{block}{Příklad se seznamy}
  \vspace{6pt}
  Program\\
  $\vcenter{\rule{.55\textwidth}{0.4pt}}$\\
  \texttt{\mlb{inner_list} = [4, "blb"]}\\
  \texttt{\mlg{outer_list} = ["ano", inner_list, 5, 6]}\\
  \texttt{print(\mlg{outer_list})}\\
  $\vcenter{\rule{.55\textwidth}{0.4pt}}$\\
  vytiskne \texttt{["ano", [4, "blb"], 5, 6]}.
 \end{block}
\end{frame}

\subsection[Podmínky]{Podmínky}

\begin{frame}{Podmínky v Pythonu}
 \begin{itemize}
  \item<1-> Podmínky se píší ve tvaru
   \begin{center}
    \fbox{
     \texttt{\mlb{if}} nějaká
     podmínka\texttt{\mlb{:}}
    }
   \end{center}
   Pro další možnosti pište \texttt{\mlb{elif}} (zkráceno z \mlb{el}se \mlb{if})
   a nakonec \texttt{\mlb{else}}.
  \item<2-> Kód, který se má za dané podmínky vykonat, \textbf{musí být
   odsazen!} Ideálně odsazujte klávesou \texttt{Tab}.
  \item<3-> Každá (správně napsaná) podmínka je v Pythonu vyhodnocena buď jako
   pravda (\texttt{\mlb{True}}), nebo lež (\texttt{\mlb{False}}).
 \end{itemize}
\end{frame}

\begin{frame}{Tvoření podmínky -- vnitřek}
  Uvnitř podmínky budeme nejčastěji používat operátory
  \begin{itemize}
   \item<1-> \texttt{\mlb{in}} (doslova \uv{v} -- testuje, jestli to nalevo je
    uvnitř toho napravo)
    \begin{itemize}
     \item Např. \texttt{("s"}~\texttt{\mlb{in} "synek")} \texttt{==}
      \texttt{\mdb{True}}, ale
     \item \texttt{(3 \mlb{in} [1, 2, 4, 5])} \texttt{==} \texttt{\mdb{False}}.
    \end{itemize}
   \item<2-> \texttt{\mlb{==}} (testuje, jestli je nalevo to samé, co napravo).
    \textbf{Tohle je ten ekvivalent jednoduchého \texttt{=} v matice.} V Pythonu
    jednoduché \texttt{=} dosazuje do proměnných!
    \begin{itemize}
     \item Např. \texttt{("sova"[2]} \texttt{\mlb{==}} \texttt{"v")} \texttt{==}
      \texttt{\mdb{True}}.
    \end{itemize}
   \item<3-> \texttt{\mlb{<,>,<=,>=}} (porovnání toho, co je nalevo, s tím, co
    je napravo). Symboly \texttt{\mlb{<=}} a \texttt{\mlb{>=}} značí \uv{menší
    nebo rovno} a \uv{větší nebo rovno}, resp.
    \begin{itemize}
     \item Např. \texttt{(5 \mlb{>} 3)} \texttt{==} \texttt{\mdb{True}} a
     \item \texttt{("c"~\mlb{<=} "f")} \texttt{==} \texttt{\mdb{True}}.
    \end{itemize}
  \end{itemize}
\end{frame}

\begin{frame}{Tvoření podmínky -- vnějšek}
 \textbf{Vně} podmínek budeme používat (logické) operátory
 \begin{itemize}
  \item<1-> \texttt{\mlb{not}} (doslova \uv{ne} -- logický opak podmínky)
   \begin{itemize}
    \item Např. \texttt{\mlb{not} (5 > 3)} \texttt{==} \texttt{\mdb{False}} a
    \item \texttt{\mlb{not} ("x"~in "kocour")} \texttt{==} \texttt{\mdb{True}}.\\
     Místo \texttt{\mlb{not} ("x"~in "kocour")} lze psát (přirozenějc)
     \texttt{"x"~\mlb{not} in "kocour"}.
   \end{itemize}
  \item<2-> \mlb{and} (doslova \uv{a} -- musí platit obě podmínky)
   \begin{itemize}
    \item Např. \texttt{(5 > 3 \mlb{and} "s"~in "synek")} \texttt{==}
     \texttt{\mdb{True}},
    \item \texttt{(3 <= 4 \mlb{and} "x"~in "kocour")} \texttt{==}
     \texttt{\mdb{False}} a
    \item \texttt{(1 in [2, 3] \mlb{and} "x"~in "kocour")} \texttt{==}
     \texttt{\mdb{False}}.
   \end{itemize}
 \end{itemize}
\end{frame}

\begin{frame}{Tvoření podmínky -- vnějšek}
 \textbf{Vně} podmínek budeme používat (logické) operátory
 \begin{itemize}
  \item \mlb{or} (doslova \uv{nebo} -- musí platit \textbf{alespoň} jedna z
   podmínek)
   \begin{itemize}
    \item Např. \texttt{(5 > 3 \mlb{or} "s"~in "synek")} \texttt{==}
     \texttt{\mdb{True}},
    \item \texttt{(3 <= 4 \mlb{or} "x"~in "kocour")} \texttt{==}
     \texttt{\mdb{True}} a
    \item \texttt{(1 in [2, 3] \mlb{or} "x"~in "kocour"} \texttt{==}
     \texttt{\mdb{False}}.
   \end{itemize}
 \end{itemize}
\end{frame}

\begin{frame}{Příklad -- liché číslo}
 Program, který určuje, jestli je číslo liché, může vypadat třeba takto.
 $\vcenter{\rule{.5\textwidth}{0.4pt}}$\\
 \texttt{\mlg{number} = 5}\\
 \texttt{\mlb{if} \mlg{number} \% 2 == 1\mlb{:}}\\
 \hspace{4ex}\texttt{print(str(\mlg{number}) + "~je liché.")}\\
 \texttt{\mlb{else:}}\\
 \hspace{4ex}\texttt{print(str(\mlg{number}) + "~je sudé.")}\\
 $\vcenter{\rule{.5\textwidth}{0.4pt}}$
\end{frame}

\subsection[Cykly]{Cykly}

\begin{frame}{\texttt{for} cyklus v Pythonu}
 \begin{itemize}
  \item<1-> \texttt{for} cyklus se v Pythonu píše
   \begin{center}
    \fbox{
     \texttt{\mlb{for}} prvek \texttt{\mlb{in}}
     seznam/n-tice/slovník\texttt{\mlb{:}}
    }
   \end{center}
   a kód uvnitř cyklu se odsazuje.\\
   \alert{Pozor!} Python prochází seznam a n-tici po prvcích, ale
   \textbf{slovník po klíčích}.
  \item<2-> Proměnná pro cyklus se může jmenovat jakkoliv. Python do ní během
   cyklu postupně dosazuje všechny prvky seznamu/n-tice, klíče slovníku a po
   jeho konci ji zapomene.
  \item<3-> Důležitá je funkce \texttt{\textcolor{violet}{range}(\mdb{n}: int)},
   která vrací seznam přirozených čísel menších než \texttt{\mdb{n}}. Např.
   \texttt{\textcolor{violet}{range}(5)} \texttt{==} \texttt{[0, 1, 2, 3, 4]}.
 \end{itemize}
\end{frame}

\begin{frame}{Příklad -- průchod seznamem}
 Program, který vytiskne každý prvek seznamu krát dva lze napsat jako
 $\vcenter{\rule{.65\textwidth}{0.4pt}}$
 \texttt{\mlg{random_stuff} = [1, "hračka", [2, 3], (4, 5)]}\\
 \texttt{\mlb{for} \mdb{wtv} \mlb{in} \mlg{random_stuff}\mlb{:}}\\
 \hspace{4ex}\texttt{print(\mdb{wtv} * 2)}\\
 $\vcenter{\rule{.65\textwidth}{0.4pt}}$
 \pause
 nebo použitím \texttt{\textcolor{violet}{range}} jako
 $\vcenter{\rule{.65\textwidth}{0.4pt}}$
 \texttt{\mlg{random_stuff} = [1, "hračka", [2, 3], (4, 5)]}\\
 \texttt{\mlb{for} \mdb{index} \mlb{in} \textcolor{violet}{range}(4)\mlb{:}}\\
 \hspace{4ex}\texttt{print(\mlg{random_stuff}[\mdb{index}] * 2)}
 $\vcenter{\rule{.65\textwidth}{0.4pt}}$
\end{frame}

\begin{frame}{\texttt{while} cyklus v Pythonu}
 \begin{itemize}
  \item<1-> \texttt{while} cyklus se v Pythonu píše
   \begin{center}
    \fbox{
     \texttt{\mlb{while} podmínka\mlb{\texttt{:}}}
    }
   \end{center}
   a obsah cyklu je odsazený.
  \item<2-> Stavění podmínek ve \texttt{\alert{while}} cyklu je stejné jako v
   \texttt{\alert{if}}.
 \end{itemize}
\end{frame}

\begin{frame}{Příklad -- mocniny dvojky}
 Program, který vypíše všechny mocniny dvojky menší než dané číslo
 \texttt{\mdb{limit}}, může vypadat třeba takhle.
 $\vcenter{\rule{.55\textwidth}{0.4pt}}$
 \texttt{\mdb{limit} = 69 ** 69}\\
 \texttt{\mlg{power_of_two} = 2}\\
 \texttt{\mlb{while} \mlg{power_of_two} < \mdb{limit}\mlb{:}}\\
 \hspace{4ex}\texttt{print(\mlg{power_of_two})}\\
 \hspace{4ex}\texttt{\mlg{power_of_two} = \mlg{power_of_two} * 2}
 $\vcenter{\rule{.55\textwidth}{0.4pt}}$
\end{frame}

\subsection[Funkce]{Funkce}

\begin{frame}{Funkce/procedury v Pythonu}
 \begin{itemize}
  \item<1-> V Pythonu se funkce píší
   \begin{center}
    \fbox{
     \texttt{\mlb{def}} jméno funkce\texttt{\mlb{(}}jména
     parametrů\texttt{\mlb{):}}
    }
   \end{center}
   a obsah funkce je odsazený.
  \item<2-> \texttt{\mlb{def}} je z angl. \alert{def}ine.
  \item<3-> Pro ukončení funkce a vrácení nějaké hodnoty slouží
   \begin{center}
    \fbox{
     \texttt{\mlb{return} hodnota}
    }
   \end{center}
   \alert{Pozor!} Funkce \textbf{nemusí vracet nic}. Jakmile provede svůj obsah,
   skončí sama, i když \texttt{\mlb{return}} nikam nenapíšete.
 \end{itemize}
\end{frame}

\begin{frame}{Příklady funkcí}
 Funkce, co dostane jméno a příjmení a vrátí je spojená dohromady, se dá napsat
 třeba takhle.
 $\vcenter{\rule{.55\textwidth}{0.4pt}}$
 \texttt{\mlb{def} whole_name\mlb{(}\mlg{name}\mlb{,} \mlg{surname}\mlb{):}}\\
 \hspace{4ex}\texttt{\mlb{return} \mlg{name}~+~"~"~+~\mlg{surname}}
 $\vcenter{\rule{.55\textwidth}{0.4pt}}$
 \pause
 Další funkce, co dostane věk a připojí za něj \uv{let}, vypadá
 $\vcenter{\rule{.55\textwidth}{0.4pt}}$
 \texttt{\mlb{def} age_to_string\mlb{(}\mlg{age}\mlb{):}}\\
 \hspace{4ex}\texttt{\mlb{return} str(\mlg{age})~+~"~let"}\\
 $\vcenter{\rule{.55\textwidth}{0.4pt}}$
\end{frame}

\begin{frame}{Příklad -- využití funkcí z před. slidu}
 Řekněme, že máme daný seznam \texttt{\mdb{data}} trojic \texttt{(}jméno,
 příjmení, věk\texttt{)}, kde jméno a příjmení jsou stringy a věk je int. Pomocí
 funkcí z předchozího slidu ho pěkně vytiskneme.
 $\vcenter{\rule{.6\textwidth}{0.4pt}}$
 \texttt{\mlb{for} (\mlg{name}, \mlg{surname},
 \mlg{age})~in~\mdb{data}\mlb{:}}\\
 \hspace{4ex}\texttt{whole_name =
 \textcolor{violet}{whole_name(}\mlg{name}\textcolor{violet}{,}
 \mlg{surname}\textcolor{violet}{)}}\\
 \hspace{4ex}\texttt{age =
 \textcolor{violet}{age_to_string(}\mlg{age}\textcolor{violet}{)}}\\
 \hspace{4ex}\texttt{print(whole_name~+~",~"~+~age)}
 $\vcenter{\rule{.6\textwidth}{0.4pt}}$
\end{frame}
