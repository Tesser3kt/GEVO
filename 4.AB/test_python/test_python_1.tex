\documentclass[a4paper,11pt]{article}

\usepackage[czech]{babel}
% Fonts %
\usepackage{fouriernc}
\usepackage[T1]{fontenc}

% Colors %
\usepackage[dvipsnames]{color}
\usepackage[dvipsnames]{xcolor}

% Page Layout %
\usepackage[margin=1.5in]{geometry}

% Fancy Headers %
\usepackage{fancyhdr}
\fancyhf{}
\cfoot{\thepage}
\rhead{}
\renewcommand{\headrulewidth}{0pt}
\setlength{\headheight}{16pt}

% Math
\usepackage{mathtools}
\usepackage{amssymb}
\usepackage{faktor}
\usepackage{import}
\usepackage{caption}
\usepackage{subcaption}
\usepackage{wrapfig}
\usepackage{enumitem}
\setlist{topsep=0pt}

\usepackage{tikz}
\usetikzlibrary{cd,positioning,babel,shapes,calc}
\usepackage{tkz-base}
\usepackage{tkz-euclide}

% Theorems
\usepackage[thmmarks, amsmath, thref]{ntheorem}
\usepackage{thmtools}

\theoremsymbol{\ensuremath{\blacksquare}}
\newtheorem*{solution}{Possible solution.}

% Title %
\title{\Huge\textsf{Homework -- PreIB 3.AB 4}\\
 \Large\textsf{Triangulations and Symmetries of Regular Polygons}
 \author{Áďa Klepáčů}
 \date{\today}
}

% Table of Contents %
\usepackage{hyperref}
\hypersetup{
 colorlinks=true,
 linktoc=all,
 linkcolor=blue
}

% Tables %
\usepackage{booktabs}
\usepackage{tabularx}

% Patch for hyphens
\usepackage{regexpatch}
\makeatletter
% Change the `-` delimiter to an active character
\xpatchparametertext\@@@cmidrule{-}{\cA-}{}{}
\xpatchparametertext\@cline{-}{\cA-}{}{}
\makeatother

\newcolumntype{s}{>{\centering\arraybackslash}p{.4\textwidth}}

% Operators %
\DeclareMathOperator{\Ker}{Ker}
\DeclareMathOperator{\Img}{Im}
\DeclareMathOperator{\End}{End}
\DeclareMathOperator{\Aut}{Aut}
\DeclareMathOperator{\Inn}{Inn}

% Common operators %
\newcommand{\R}{\mathbb{R}}
\newcommand{\N}{\mathbb{N}}
\newcommand{\Z}{\mathbb{Z}}
\newcommand{\Q}{\mathbb{Q}}
\newcommand{\C}{\mathbb{C}}

\newcommand{\clr}{\textcolor{red}}
\newcommand{\clb}{\textcolor{blue}}
\newcommand{\clg}{\textcolor{green}}
\newcommand{\clm}{\textcolor{magenta}}
\newcommand{\clv}{\textcolor{violet}}
\newcommand{\clbr}{\textcolor{Sepia}}

% American Paragraph Skip %
\setlength{\parindent}{0pt}
\setlength{\parskip}{1em}

% Document %
\pagestyle{fancy}
\begin{document}

\thispagestyle{fancy}

\hrule
\vspace*{-1em}
\begin{center}
 \clr{\textbf{Z NÁSLEDUJÍCÍCH ÚLOH SI VYBERTE JEDNU. PŘI ŘEŠENÍ ÚLOH 
  NESMÍTE VYUŽÍT CIZÍ POMOCI -- ANI LIDSKÉ ANI UMĚLÉ!}}
\end{center}
\hrule

\subsection*{1. Správné závorky}

Napište funkci, která dostane parametrem \texttt{vyraz} algebraický výraz, který
může obsahovat (pouze kulaté) závorky. O tomto výrazu rozhodne, \textbf{zda je
správně uzávorkovaný}. To pro nás znamená jen to, že je v něm stejně levých
závorek \texttt{"("} jako pravých \texttt{")"} a že při čtení zleva doprava v
žádnou chvíli nenarazím na víc pravých závorek než levých.

\textbf{Hint:} Čtěte výraz \texttt{for} cyklem a pamatujte si ve dvou proměnných
počet levých a pravých závorek. V průběhu cyklu vždycky zkontrolujte, že počet
pravých nikdy není větší než počet levých. Na konci zkontrolujte, že obou typů
závorek máte stejně.

\emph{Příklady}:
\begin{itemize}
 \item Výraz \texttt{((x + y) * 6) + 5} je správně uzávorkovaný.
 \item Výraz \texttt{(20 + 3) * 5)} správně uzávorkovaný není, protože má víc
  \texttt{")"} než \texttt{"("}.
 \item Výraz \texttt{)z - (20 + 5) * 3} je špatně uzávorkovaný, protože před
  první \texttt{"("} už je jedna \texttt{")"}.
\end{itemize}

\clearpage

\subsection*{2. Stejná písmena}

Napište funkci, která dostane dva parametry -- \texttt{slovo1} a \texttt{slovo2}
a spočítá, \textbf{kolik mají \texttt{slovo1} a \texttt{slovo2} stejných písmen
na stejných místech}.

\textbf{Hint:} Procházejte \texttt{slovo1} a \texttt{slovo2} současně pomocí
proměnné, ve které si budete pamatovat pořadí písmene v obou slovech (buď
\texttt{while} cyklem nebo \texttt{for} cyklem s~funkcí \texttt{range}). Když se
písmena s tímto pořadím budou shodovat, zvyšte počet shodných písmen o jedna.

\emph{Příklady:}
\begin{itemize}
 \item Pro slova \texttt{"pes"} a \texttt{"les"} vaše funkce vrátí \texttt{2}.
 \item Pro slova \texttt{"stůl"} a \texttt{"vůl"} vaše funkce vrátí \texttt{0},
  protože mají sice společná písmena \texttt{"ů"} a \texttt{"l"}, ale jsou na
  jiných místech.
\end{itemize}

\end{document}
