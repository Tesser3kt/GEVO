\documentclass[a4paper,11pt]{article}

\usepackage[czech]{babel}
% Fonts %
\usepackage{fouriernc}
\usepackage[T1]{fontenc}

% Colors %
\usepackage[dvipsnames]{color}
\usepackage[dvipsnames]{xcolor}

% Page Layout %
\usepackage[margin=1.5in]{geometry}

% Fancy Headers %
\usepackage{fancyhdr}
\fancyhf{}
\cfoot{\thepage}
\rhead{}
\renewcommand{\headrulewidth}{0pt}
\setlength{\headheight}{16pt}

% Math
\usepackage{mathtools}
\usepackage{amssymb}
\usepackage{faktor}
\usepackage{import}
\usepackage{caption}
\usepackage{subcaption}
\usepackage{wrapfig}
\usepackage{enumitem}
\setlist{topsep=0pt}

\usepackage{tikz}
\usetikzlibrary{cd,positioning,babel,shapes,calc}
\usepackage{tkz-base}
\usepackage{tkz-euclide}

% Theorems
\usepackage[thmmarks, amsmath, thref]{ntheorem}
\usepackage{thmtools}

\theoremsymbol{\ensuremath{\blacksquare}}
\newtheorem*{solution}{Possible solution.}

% Table of Contents %
\usepackage{hyperref}
\hypersetup{
 colorlinks=true,
 linktoc=all,
 linkcolor=blue
}

% Tables %
\usepackage{booktabs}
\usepackage{tabularx}

% Patch for hyphens
\usepackage{regexpatch}
\makeatletter
% Change the `-` delimiter to an active character
\xpatchparametertext\@@@cmidrule{-}{\cA-}{}{}
\xpatchparametertext\@cline{-}{\cA-}{}{}
\makeatother

\newcolumntype{s}{>{\centering\arraybackslash}p{.4\textwidth}}

% Operators %
\DeclareMathOperator{\Ker}{Ker}
\DeclareMathOperator{\Img}{Im}
\DeclareMathOperator{\End}{End}
\DeclareMathOperator{\Aut}{Aut}
\DeclareMathOperator{\Inn}{Inn}

% Common operators %
\newcommand{\R}{\mathbb{R}}
\newcommand{\N}{\mathbb{N}}
\newcommand{\Z}{\mathbb{Z}}
\newcommand{\Q}{\mathbb{Q}}
\newcommand{\C}{\mathbb{C}}

\newcommand{\clr}{\textcolor{red}}
\newcommand{\clb}{\textcolor{blue}}
\newcommand{\clg}{\textcolor{green}}
\newcommand{\clm}{\textcolor{magenta}}
\newcommand{\clv}{\textcolor{violet}}
\newcommand{\clbr}{\textcolor{Sepia}}

% American Paragraph Skip %
\setlength{\parindent}{0pt}
\setlength{\parskip}{1em}

% Document %
\pagestyle{empty}
\begin{document}

\thispagestyle{empty}

\hrule
\vspace*{-1em}
\begin{center}
 \clr{\textbf{Z NÁSLEDUJÍCÍCH ÚLOH SI VYBERTE JEDNU. PŘI ŘEŠENÍ ÚLOH 
  NESMÍTE VYUŽÍT CIZÍ POMOCI -- ANI LIDSKÉ ANI UMĚLÉ!}}
\end{center}
\hrule

\section*{1. Isogramy}

Řekneme, že nějaké slovo je \emph{isogram}, když se v něm neopakuje žádné
písmeno. Napište funkci, která dostane libovolné slovo (jako \texttt{string}) a
rozhodne, zda je to isogram, tj. vrátí \texttt{True}, pokud ano, a
\texttt{False}, pokud ne.\\
\textbf{Hint:} Slovníky jsou dobrým způsobem, jak si ukládat počet výskytů
písmene ve slově.

\emph{Příklady:}
\begin{itemize}
 \item Slovo \emph{přímočarý} je isogram, vaše funkce vrátí \texttt{True}.
 \item Slovo \emph{palentologie} isogram není, vaše funkce vrátí \texttt{False}.
\end{itemize}

\clearpage

\section*{2. Kód kreditky}

Při nákupu kartou po internetu jsou během ověřování z důvodů bezpečnosti obvykle
skryta všechna číslice kreditní karty až na poslední čtyři.

Kód kreditní karty je \texttt{string} \textbf{o přesně 16 znacích}. Napište
funkci, která obdrží jeden \texttt{string}. Pokud má tento \texttt{string} 16
znaků (\textbf{nemusíte kontrolovat, zda jsou to číslice}), vaše funkce nahradí
prvních 12 znaků symbolem \texttt{\#} a výsledný string vrátí. Pokud 16 znaků
nemá, vrátí vaše funkce pouze \texttt{"Toto není kód kreditní karty."}.\\
\textbf{Hint:} Vyrobte si prázdný \texttt{string} a použijte \texttt{for} cyklus
buď s funkcí \texttt{range} nebo s~\texttt{enumerate}.

\emph{Příklady:}
\begin{itemize}
 \item Pro string \texttt{"4556364607935616"} vrátí vaše funkce
  \texttt{"\#\#\#\#\#\#\#\#\#\#\#\#5616"}.
 \item Pro string \texttt{"64607935616"} vrátí vaše funkce \texttt{"Toto není
  kód kreditní karty."}.
 \item Pro string \texttt{"psineradikocicky"} vrátí vaše funkce
  \texttt{"\#\#\#\#\#\#\#\#\#\#\#\#icky"}.
\end{itemize}

\end{document}
