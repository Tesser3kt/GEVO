\documentclass[a4paper,11pt]{article}

\usepackage[czech]{babel}
% Fonts %
\usepackage{fouriernc}
\usepackage[T1]{fontenc}

% Colors %
\usepackage[dvipsnames]{color}
\usepackage[dvipsnames]{xcolor}

% Page Layout %
\usepackage[margin=1.5in]{geometry}

% Fancy Headers %
\usepackage{fancyhdr}
\fancyhf{}
\cfoot{\thepage}
\rhead{}
\renewcommand{\headrulewidth}{0pt}
\setlength{\headheight}{16pt}

% Math
\usepackage{mathtools}
\usepackage{amssymb}
\usepackage{faktor}
\usepackage{import}
\usepackage{caption}
\usepackage{subcaption}
\usepackage{wrapfig}
\usepackage{enumitem}
\setlist{topsep=0pt}

\usepackage{tikz}
\usetikzlibrary{cd,positioning,babel,shapes,calc}
\usepackage{tkz-base}
\usepackage{tkz-euclide}

% Theorems
\usepackage[thmmarks, amsmath, thref]{ntheorem}
\usepackage{thmtools}

\theoremsymbol{\ensuremath{\blacksquare}}
\newtheorem*{solution}{Possible solution.}

% Table of Contents %
\usepackage{hyperref}
\hypersetup{
 colorlinks=true,
 linktoc=all,
 linkcolor=blue
}

% Tables %
\usepackage{booktabs}
\usepackage{tabularx}

% Patch for hyphens
\usepackage{regexpatch}
\makeatletter
% Change the `-` delimiter to an active character
\xpatchparametertext\@@@cmidrule{-}{\cA-}{}{}
\xpatchparametertext\@cline{-}{\cA-}{}{}
\makeatother

\newcolumntype{s}{>{\centering\arraybackslash}p{.4\textwidth}}

% Operators %
\DeclareMathOperator{\Ker}{Ker}
\DeclareMathOperator{\Img}{Im}
\DeclareMathOperator{\End}{End}
\DeclareMathOperator{\Aut}{Aut}
\DeclareMathOperator{\Inn}{Inn}

% Common operators %
\newcommand{\R}{\mathbb{R}}
\newcommand{\N}{\mathbb{N}}
\newcommand{\Z}{\mathbb{Z}}
\newcommand{\Q}{\mathbb{Q}}
\newcommand{\C}{\mathbb{C}}

\newcommand{\clr}{\textcolor{red}}
\newcommand{\clb}{\textcolor{blue}}
\newcommand{\clg}{\textcolor{green}}
\newcommand{\clm}{\textcolor{magenta}}
\newcommand{\clv}{\textcolor{violet}}
\newcommand{\clbr}{\textcolor{Sepia}}

% American Paragraph Skip %
\setlength{\parindent}{0pt}
\setlength{\parskip}{1em}

% Document %
\pagestyle{empty}
\begin{document}

\thispagestyle{empty}

\hrule
\vspace*{-1em}
\begin{center}
 \clr{\textbf{Z NÁSLEDUJÍCÍCH ÚLOH SI VYBERTE JEDNU. PŘI ŘEŠENÍ ÚLOH 
  NESMÍTE VYUŽÍT CIZÍ POMOCI -- ANI LIDSKÉ ANI UMĚLÉ!}}
\end{center}
\hrule

\subsection*{1. Závod}

Napište funkci, která dostane jako parametry dva seznamy -- \texttt{auto1} a
\texttt{auto2} a rozhodne, které auto ujede větší vzdálenost. Obdržené seznamy
obsahují úseky -- dvojice \texttt{(rychlost, čas)}, které znamenají, že dané
auto jelo po dobu \texttt{čas} (ve vteřinách) rychlostí \texttt{rychlost} (v
metrech za vteřinu). Funkce má vrátit \texttt{1}, když \texttt{auto1} ujede
větší vzdálenost, jinak má vrátit \texttt{2}.

\textbf{Hint:} Na začátku funkce si vytvořte proměnné pro vzdálenosti obou aut.
Procházejte \texttt{for} cyklem oba seznamy postupně (nebo jedním \texttt{while}
cyklem oba najednou) a vždy k zapamatovaným vzdálenostem přičtěte
\texttt{rychlost * čas}. Na konci cyklu budeme mít spočtenou celou uraženou
vzdálenost.

\emph{Příklad:} Je-li
\begin{itemize}
 \item \texttt{auto1 = [(40, 10), (35, 15), (50, 5)]},
 \item \texttt{auto2 = [(30, 5), (35, 10), (50, 10), (25, 5)]},
\end{itemize}
pak
\begin{itemize}
 \item \texttt{auto1} urazí vzdálenost \texttt{40 * 10 + 35 * 15 + 50 * 5 =
  1175},
 \item \texttt{auto2} urazí vzdálenost \texttt{30 * 5 + 35 * 10 + 50 * 10 + 25 *
  5 = 925},
\end{itemize}
takže vyhraje \texttt{auto1} a funkce vrací \texttt{1}.

\clearpage

\subsection*{2. Platný e-mail}

Napište funkci, která dostane parametrem string \texttt{email} a rozhodne (tj.
vrátí \texttt{True} nebo \texttt{False}), zda se jedná o platnou e-mailovou
adresu. Podmínky platnosti jsou pro nás následující:
\begin{enumerate}
 \item \texttt{email} obsahuje \texttt{"@"} i \texttt{"."}.
 \item mezi \texttt{"@"} a \texttt{"."} je aspoň jeden další symbol.
 \item \texttt{email} začíná a končí jiným symbolem než \texttt{"@"} a
  \texttt{"."}.
\end{enumerate}

\textbf{Hint:} Podmínky 1 a 3 jsou jednoduché, stačí použít \texttt{in} a
podívat se na první a poslední prvek stringu \texttt{email}. U podmínky číslo
\texttt{2} je potřeba zjistit pořadí \texttt{"@"} a \texttt{"."} (funkci
\texttt{find} \textbf{nepoužívejte}) a podívat se, že nejsou hned u sebe.

\emph{Příklady:}
\begin{itemize}
 \item \texttt{adam.klepac@gevo.cz} je platná e-mailová adresa.
 \item \texttt{@gmail.com} není platná, protože začíná na \texttt{"@"}.
 \item \texttt{whoever@gmail} není platná, protože neobsahuje \texttt{"."}.
 \item \texttt{whoever2@.com} není platná, protože \texttt{"@"} a \texttt{"."}
  jsou hned u sebe.
\end{itemize}

\end{document}
