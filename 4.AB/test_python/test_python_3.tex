\documentclass[a4paper,11pt]{article}

\usepackage[czech]{babel}
% Fonts %
\usepackage{fouriernc}
\usepackage[T1]{fontenc}

% Colors %
\usepackage[dvipsnames]{color}
\usepackage[dvipsnames]{xcolor}

% Page Layout %
\usepackage[margin=1.5in]{geometry}

% Fancy Headers %
\usepackage{fancyhdr}
\fancyhf{}
\cfoot{\thepage}
\rhead{}
\renewcommand{\headrulewidth}{0pt}
\setlength{\headheight}{16pt}

% Math
\usepackage{mathtools}
\usepackage{amssymb}
\usepackage{faktor}
\usepackage{import}
\usepackage{caption}
\usepackage{subcaption}
\usepackage{wrapfig}
\usepackage{enumitem}
\setlist{topsep=0pt}

\usepackage{tikz}
\usetikzlibrary{cd,positioning,babel,shapes,calc}
\usepackage{tkz-base}
\usepackage{tkz-euclide}

% Theorems
\usepackage[thmmarks, amsmath, thref]{ntheorem}
\usepackage{thmtools}

\theoremsymbol{\ensuremath{\blacksquare}}
\newtheorem*{solution}{Possible solution.}

% Table of Contents %
\usepackage{hyperref}
\hypersetup{
 colorlinks=true,
 linktoc=all,
 linkcolor=blue
}

% Tables %
\usepackage{booktabs}
\usepackage{tabularx}

% Patch for hyphens
\usepackage{regexpatch}
\makeatletter
% Change the `-` delimiter to an active character
\xpatchparametertext\@@@cmidrule{-}{\cA-}{}{}
\xpatchparametertext\@cline{-}{\cA-}{}{}
\makeatother

\newcolumntype{s}{>{\centering\arraybackslash}p{.4\textwidth}}

% Operators %
\DeclareMathOperator{\Ker}{Ker}
\DeclareMathOperator{\Img}{Im}
\DeclareMathOperator{\End}{End}
\DeclareMathOperator{\Aut}{Aut}
\DeclareMathOperator{\Inn}{Inn}

% Common operators %
\newcommand{\R}{\mathbb{R}}
\newcommand{\N}{\mathbb{N}}
\newcommand{\Z}{\mathbb{Z}}
\newcommand{\Q}{\mathbb{Q}}
\newcommand{\C}{\mathbb{C}}

\newcommand{\clr}{\textcolor{red}}
\newcommand{\clb}{\textcolor{blue}}
\newcommand{\clg}{\textcolor{green}}
\newcommand{\clm}{\textcolor{magenta}}
\newcommand{\clv}{\textcolor{violet}}
\newcommand{\clbr}{\textcolor{Sepia}}

% American Paragraph Skip %
\setlength{\parindent}{0pt}
\setlength{\parskip}{1em}

% Document %
\pagestyle{empty}
\begin{document}

\thispagestyle{empty}

\hrule
\vspace*{-1em}
\begin{center}
 \clr{\textbf{Z NÁSLEDUJÍCÍCH ÚLOH SI VYBERTE JEDNU. PŘI ŘEŠENÍ ÚLOH 
  NESMÍTE VYUŽÍT CIZÍ POMOCI -- ANI LIDSKÉ ANI UMĚLÉ!}}
\end{center}
\hrule

\subsection*{1. Přelévání vody}

Napište funkci, která dostane jako parametry seznam čísel
\texttt{objemy\_sklenic} a jedno číslo \texttt{objem\_nadoby}. Funkce rozhodne,
kolik sklenic (v pořadí daném seznamem \texttt{objemy\_sklenic}) plných vody lze
přelít do prázdné nádoby o objemu \texttt{objem\_nadoby}. Tento počet vrátí jako
celé číslo.

\textbf{Hint:} Vytvořte si proměnnou, ve které si budete pamatovat počet už
přelitých sklenic. Pak procházejte seznam \texttt{objemy\_sklenic} a čísla z něj
odčítejte od čísla \texttt{objem\_nadoby}, dokud zůstává kladným. Pak vraťte ten
počet přelitých sklenic.

\emph{Příklad}: Pro parametry
\begin{itemize}
 \item \texttt{objemy\_sklenic = [3.14, 2.78, 1.4]},
 \item \texttt{objem\_nadoby = 6},
\end{itemize}
vrátí funkce $2$, protože se do nádoby vejde objem jen prvních dvou sklenic.

\clearpage

\subsection*{2. Součet vnitřního seznamu}

Napište funkci, která dostane jako parametr seznam jménem \texttt{seznam}. Máte
zaručeno, že tento seznam má vždy na čtvrté pozici (tj. na indexu \texttt{3})
další seznam plný celých čísel. Funkce nahradí v seznamu \texttt{seznam} tento
vnitřní seznam součtem jeho prvků a takto upravený seznam vrátí. \clr{K určení
součtu prvků nepoužívejte funkci \texttt{sum}!}

\textbf{Hint:} Stačí tradičním způsobem (přes pomocnou proměnnou) sečíst prvky
ve vnitřním seznamu a pak prvek na místě \texttt{3} v seznamu \texttt{seznam}
touto proměnnou nahradit.

\emph{Příklad}: Pro seznam
\begin{center}
 \texttt{seznam = ["pes", 4.56, "nic", [5, 1, 4], "neco"]}
\end{center}
vrátí funkce seznam
\begin{center}
 \texttt{["pes", 4.56, "nic", 10, "neco"]},
\end{center}
protože součet seznamu \texttt{[5, 1, 4]} je \texttt{10}.

\end{document}
