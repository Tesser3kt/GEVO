\question[50]
Máte daný seznam se sudým počtem celých čísel (tím myslím \uv{Vytvořte si seznam
	se sudým počtem celých čísel.}). Tiskněte jeho prvky na přeskáčku vždycky
zepředu a pak zezadu -- takže nejdřív vytisknu první prvek, pak poslední, pak
druhý, pak předposlední atd.\\
Například pro seznam \texttt{[1, 2, 3, 4, 5, 6]} vytisknu čísla \texttt{1, 6, 2,
	5, 3, 4} v tomhle pořadí.\\
\textcolor{red}{Délku seznamu můžete psát do programu manuálně jako celé číslo.}