\documentclass{exam}
\usepackage[czech]{babel}
\usepackage{xcolor}
\usepackage{enumitem}
\usepackage{amsmath}
\usepackage{mathtools}
\usepackage{booktabs}
\usepackage{logicpuzzle}
\pointname{ \%}

\begin{document}
Jméno: Jáchym Löwenhöffer
\begin{questions}
	\question[0.1]
	\emph{Latinský čtverec velikosti $n$} je pole $n \times n$ takové, že každý
 prvek z množiny libovolných $n$ prvků (obvykle se používá $A,B,C,\ldots$, proto
 \uv{latinský}) je v každém sloupci a v každém řádku přesně jednou. Například
 \begin{center}
  \begin{logicpuzzle}[rows=3,columns=3]
   \setrow{3}{$A$,$B$,$C$}
   \setrow{2}{$B$,$C$,$A$}
   \setrow{1}{$C$,$A$,$B$}
  \end{logicpuzzle}
 \end{center}
 je latinský čtverec velikosti $3$.

 Napiš funkci, která dostane pole (seznam seznamů) velikosti $n \times n$,
 kde prvky jsou čísla od $1$ do $n$ a určí, zda se jedná o latinský čtverec.

 \question[0.01][* celkem hard]
 Napiš funkci, která pro dané $n \leq 7$ určí počet všech různých latinských
 čtverců velikosti $n$.

 \textbf{Hint:} Ani to nezkoušej na papíře. Pro $n = 7$ jich je
 $61,479,419,904,000$.

 \question[0.001]
 \emph{Řecko-latinský čtverec} velikosti $n$ je pole $n \times n$ obsahující na
 každém místě dvojici prvků ze součinu dvou disjunktních množin velikosti $n$
 tak, že každý prvek z jedné množiny je v každém sloupci a v každém řádku právě
 jednou a navíc je každá dvojice prvků právě jednou v celém čtverci. Například
 \begin{center}
  \begin{logicpuzzle}[rows=3,columns=3]
   \setrow{3}{$A\alpha$,$B\beta$,$C\gamma$}
   \setrow{2}{$B\gamma$,$C\alpha$,$A\beta$}
   \setrow{1}{$C\beta$,$A\gamma$,$B\alpha$}
  \end{logicpuzzle}
 \end{center}
 je řecko-latinský čtverec velikosti $3$. Jsou to jakoby dva přes sebe přeložené
 latinské čtverce tak, aby odpovídající dvojice v každém políčku byly unikátní.

 Dva latinské čtverce nazveme \emph{vzájemně ortogonální (kolmé)}, když jejich
 přeložení přes sebe je řecko-latinský čtverec.

 Napiš funkci, která dostane dva latinské čtverce jako pole $n \times n$ 
 čísel od $1$ do $n$ a určí, zda jsou vzájemně ortogonální.

 \question[0.0001][** extra hard]
 Napiš funkci, která pro dané $n \leq 6$ určí počet všech (neuspořádáných)
 dvojic vzájemně ortogonálních latinských čtverců.
\end{questions}
\end{document}
