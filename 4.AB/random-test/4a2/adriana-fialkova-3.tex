\question[70]
Napište proceduru, která dostane jako parametry seznam celých čísel
\texttt{seznam} a jedno další celé číslo \texttt{x}. Procedura bude tisknout
prvky seznamu \texttt{seznam}, dokud nenarazí na číslo, které je větší než
\texttt{x}. To, ani další čísla za ním už nevytiskne. Když jsou všechna čísla
v seznamu menší než \texttt{x}, vytiskne všechny prvky seznamu.

\textbf{Hint:} Tady bych použil \texttt{while} cyklus. V nějaké proměnné
\texttt{i} bych si
ukládal pořadí v seznamu, které bych pokaždé zvýšil o jedna. Nezapomeňte v
podmínce \texttt{while} cyklu kontrolovat jak to, že prvek na pozici \texttt{i}
není větší než \texttt{x}, tak to, že pozice \texttt{i} nepřeroste délku
seznamu.
\textcolor{red}{Nesmíte použít příkaz \texttt{break}!}