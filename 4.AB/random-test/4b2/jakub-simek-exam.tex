\documentclass{exam}
\usepackage[czech]{babel}
\usepackage{xcolor}
\usepackage{enumitem}
\usepackage{hyperref}
\pointname{ \%}

\begin{document}
Jméno: Jakub Šimek
\begin{questions}
	\question[100]
	Napiš program, který v zadaném textovém souboru \texttt{file.txt}
	vycenzoruje všechna slova určená v jiném souboru \texttt{censor.txt}. Slova
	v \texttt{censor.txt} jsou oddělena čárkami bez mezer. Vycensurovaný text je
	uložen do souboru \texttt{file\_censored.txt} v \textbf{dokonale stejném
		formátování} jako byl původní text. Konečně, všechna vycensurovaná slova
	jsou vytištěna do konzole spolu s jejich počtem v původním textu.

	Censura probíhá tak, že dané slovo je nahrazeno tolika hvězdičkami
	(\texttt{*}), jaká je délka slova. Censura musí ignorovat velká, malá
	písmena i čárky a háčky. Tedy, je-li například v \texttt{censor.txt} slovo
	\texttt{reholnik}, pak program musí vycensurovat všechna slova
	\texttt{řeholník} i \texttt{Řeholník} v původním textu.

	Soubor \texttt{censor.txt} si vyrob sám, \texttt{file.txt} je
	\href{https://drive.google.com/file/d/1hSUMrPpdCYT3i1POkVOjx2itOmyUwW6i/view?usp=sharing}{tady}.
\end{questions}
\end{document}
