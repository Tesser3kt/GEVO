\documentclass[a4paper,11pt]{article}

\usepackage[czech]{babel}
% Fonts %
\usepackage{fouriernc}
\usepackage[T1]{fontenc}

% Colors %
\usepackage[dvipsnames]{color}
\usepackage[dvipsnames]{xcolor}

% Page Layout %
\usepackage[margin=1.5in]{geometry}

% Fancy Headers %
\usepackage{fancyhdr}
\fancyhf{}
\cfoot{\thepage}
\rhead{}
\renewcommand{\headrulewidth}{0pt}
\setlength{\headheight}{16pt}

% Math
\usepackage{mathtools}
\usepackage{amssymb}
\usepackage{faktor}
\usepackage{import}
\usepackage{caption}
\usepackage{subcaption}
\usepackage{wrapfig}
\usepackage{enumitem}
\setlist{topsep=0pt}

\usepackage{tikz}
\usetikzlibrary{cd,positioning,babel,shapes,calc}
\usepackage{tkz-base}
\usepackage{tkz-euclide}

% Theorems
\usepackage[thmmarks, amsmath, thref]{ntheorem}
\usepackage{thmtools}

\theoremsymbol{\ensuremath{\blacksquare}}
\newtheorem*{solution}{Possible solution.}

% Title %
\title{\Huge\textsf{Homework -- PreIB 3.AB 4}\\
 \Large\textsf{Triangulations and Symmetries of Regular Polygons}
 \author{Áďa Klepáčů}
 \date{\today}
}

% Table of Contents %
\usepackage{hyperref}
\hypersetup{
 colorlinks=true,
 linktoc=all,
 linkcolor=blue
}

% Tables %
\usepackage{booktabs}
\usepackage{tabularx}

% Patch for hyphens
\usepackage{regexpatch}
\makeatletter
% Change the `-` delimiter to an active character
\xpatchparametertext\@@@cmidrule{-}{\cA-}{}{}
\xpatchparametertext\@cline{-}{\cA-}{}{}
\makeatother

\newcolumntype{s}{>{\centering\arraybackslash}p{.4\textwidth}}

% Operators %
\DeclareMathOperator{\Ker}{Ker}
\DeclareMathOperator{\Img}{Im}
\DeclareMathOperator{\End}{End}
\DeclareMathOperator{\Aut}{Aut}
\DeclareMathOperator{\Inn}{Inn}

% Common operators %
\newcommand{\R}{\mathbb{R}}
\newcommand{\N}{\mathbb{N}}
\newcommand{\Z}{\mathbb{Z}}
\newcommand{\Q}{\mathbb{Q}}
\newcommand{\C}{\mathbb{C}}

\newcommand{\clr}{\textcolor{red}}
\newcommand{\clb}{\textcolor{blue}}
\newcommand{\clg}{\textcolor{green}}
\newcommand{\clm}{\textcolor{magenta}}
\newcommand{\clv}{\textcolor{violet}}
\newcommand{\clbr}{\textcolor{Sepia}}

% American Paragraph Skip %
\setlength{\parindent}{0pt}
\setlength{\parskip}{1em}

% Document %
\pagestyle{fancy}
\begin{document}

\thispagestyle{fancy}

\clr{\textbf{\uppercase{Z následujících úloh si vyberte jednu!}}}

\subsection*{Bitka, bitka, bitka!}
Máte dány dvě množiny lidí -- $\mathtt{L_1}$ a $\mathtt{L_2}$. Dále máte funkci
\texttt{potkajise}, která dostane jako parametry jednoho člověka z
$\mathtt{L_1}$ a jednoho člověka z $\mathtt{L_2}$ a rozhodne o nich, jestli se
potkají, tj. vrátí \texttt{true}, když se potkají, a \texttt{false}, když ne.
Naposled máte funkci \texttt{nesnasi}, která dostane jako parametry dva lidi (ať
už z $\mathtt{L_1}$ nebo z $\mathtt{L_2}$) a rozhodne, jestli \textbf{ten první
 nesnáší toho druhého}.

Vaším úkolem je napsat algoritmus, který vrátí dvě množiny lidí -- \texttt{B} a
\texttt{U}. V množině \texttt{B} jsou lidi, kteří se s někým poperou -- to se
stane tehdy, když se potkají dva lidé, kteří se vzájemně nesnáší. V množině
\texttt{U} budou lidi, kteří utečou -- to se stane tehdy, když se potkají dva
lidé a pouze jeden nesnáší toho druhého; ten druhý před ním uteče.

\emph{Příklad}: Pro následující zadání (funkce \texttt{\clb{potkajise}} a
\texttt{\clr{nesnasi}} reprezentujeme šipkami)
\begin{center}
 \begin{tikzpicture}[scale=1.5]
  \node (z) at (0,1) {Zbyšek};
  \node (j) at (0,2) {Jarmila};
  \node at (0,3) {$L_1$};

  \node (k) at (4,0) {Kvido};
  \node (h) at (4,1) {Horymír};
  \node (a) at (4,2) {Alois};
  \node at (4,3) {$L_2$};

  \draw[thick,blue,<->,dashed,bend left=30] (z) to (h);
  \draw[thick,blue,<->,dashed,bend right=30] (z) to (k);
  \draw[thick,blue,<->,dashed] (j) -- (a);
  \node at (2,2.6) {\clb{\texttt{potkajise}}};

  \draw[thick,red,->,dashed,bend left=15] (a) to (j);
  \draw[thick,red,->,dashed] (h) -- (z);
  \draw[thick,red,->,dashed,bend right=15] (j) to (h);
  \draw[thick,red,->,dashed] (k) -- (z);
  \node at (2,-0.5) {\clr{\texttt{nesnasi}}};
 \end{tikzpicture}
\end{center}
je správná odpověď
\begin{align*}
 \mathtt{B} &= \{\text{Zbyšek},\text{Horymír}\},\\
 \mathtt{U} &= \{\text{Jarmila},\text{Zbyšek}\}.
\end{align*}
\clearpage

\subsection*{Divná Morseovka}

Armáda Spolkové republiky GEVO používá pro tajnou komunikaci tzv. Divnou
Morseovku. Každá zakódovaná zpráva obsahuje pouze symboly \clr{\textbf{.}} a
\clr{\textbf{--}}, ale obsahuje \textbf{přesně dvakrát tolik \clr{.}, jak
\clr{--}}. Navíc, ještě divněji, musí \textbf{každá zpráva začínat dvěma
\clr{--} a končit třemi \clr{.}}.

Dvě zprávy v Divné Morseovce považujeme za \textbf{stejné}, když \textbf{mají
stejný počet \clr{.} a \clr{--} bez ohledu na pořadí}.

Na vstupu dostanete \clr{\textbf{pouze funkce}} $znak_1$ a $znak_2$, které umí
ze zadaných dvou zpráv vytáhnout znak na pozici $i$ jako $znak_1(i)$ nebo
$znak_2(i)$. \clr{Nemáte funkci, která by vám řekla délku zprávy!} Funkce
$znak_1$ a $znak_2$ vrací prázdnou množinu, když znak na pozici $i$ neexistuje.

\begin{enumerate}
 \item Napište funkci/proceduru $DivnaMorseovka(znak)$, která dostane parametrem
  zprávu jako funkci $znak$ a rozhodne, zda se jedná o zprávu v Divné Morseovce.
  Doporučuji, aby funkce vracela i počet \clr{\textbf{.}} a \clr{\textbf{--}} ve
  zprávě. Výsledný algoritmus bude jednodušší.
 \item Použijte tuhle proceduru v algoritmu, který dostane funkce $znak_1$ a
  $znak_2$ a, \textbf{za předpokladu, že se jedná o zprávy v Divné Morseovce},
  rozhodne, zda \textbf{jsou stejné}.
\end{enumerate}

\emph{Příklady:}
\begin{itemize}
 \item \clr{\textbf{-- -- . . -- -- -- . . .}} a \clr{\textbf{-- -- -- -- . . --
  . . .}} jsou dvě zprávy Divné Morseovce, které jsou stejné.
 \item \clr{\textbf{-- -- . . .}} není zpráva v Divné Morseovce, protože má víc
  \clr{\textbf{.}} než \clr{\textbf{--}}.
 \item \clr{\textbf{-- . . -- .}} není zpráva v Divné Morseovce, protože nezačíná
  na \clr{\textbf{-- --}} ani nekončí na \clr{\textbf{. . .}}.
 \item \clr{\textbf{-- -- . -- -- . . .}} a \clr{\textbf{-- -- -- . . .}} jsou
  obě zprávy v Divné Morseovce, které nejsou stejné.
\end{itemize}

\end{document}
