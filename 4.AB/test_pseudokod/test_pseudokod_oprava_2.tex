\documentclass[a4paper,11pt]{article}

\usepackage[czech]{babel}
% Fonts %
\usepackage{fouriernc}
\usepackage[T1]{fontenc}

% Colors %
\usepackage[dvipsnames]{color}
\usepackage[dvipsnames]{xcolor}

% Page Layout %
\usepackage[margin=1.5in]{geometry}

% Fancy Headers %
\usepackage{fancyhdr}
\fancyhf{}
\cfoot{\thepage}
\rhead{}
\renewcommand{\headrulewidth}{0pt}
\setlength{\headheight}{16pt}

% Math
\usepackage{mathtools}
\usepackage{amssymb}
\usepackage{faktor}
\usepackage{import}
\usepackage{caption}
\usepackage{subcaption}
\usepackage{wrapfig}
\usepackage{enumitem}
\setlist{topsep=0pt}

\usepackage{tikz}
\usetikzlibrary{cd,positioning,babel,shapes,calc}
\usepackage{tkz-base}
\usepackage{tkz-euclide}

% Theorems
\usepackage[thmmarks, amsmath, thref]{ntheorem}
\usepackage{thmtools}

\theoremsymbol{\ensuremath{\blacksquare}}
\newtheorem*{solution}{Possible solution.}

% Title %
\title{\Huge\textsf{Homework -- PreIB 3.AB 4}\\
 \Large\textsf{Triangulations and Symmetries of Regular Polygons}
 \author{Áďa Klepáčů}
 \date{\today}
}

% Table of Contents %
\usepackage{hyperref}
\hypersetup{
 colorlinks=true,
 linktoc=all,
 linkcolor=blue
}

% Tables %
\usepackage{booktabs}
\usepackage{tabularx}

% Patch for hyphens
\usepackage{regexpatch}
\makeatletter
% Change the `-` delimiter to an active character
\xpatchparametertext\@@@cmidrule{-}{\cA-}{}{}
\xpatchparametertext\@cline{-}{\cA-}{}{}
\makeatother

\newcolumntype{s}{>{\centering\arraybackslash}p{.4\textwidth}}

% Operators %
\DeclareMathOperator{\Ker}{Ker}
\DeclareMathOperator{\Img}{Im}
\DeclareMathOperator{\End}{End}
\DeclareMathOperator{\Aut}{Aut}
\DeclareMathOperator{\Inn}{Inn}

% Common operators %
\newcommand{\R}{\mathbb{R}}
\newcommand{\N}{\mathbb{N}}
\newcommand{\Z}{\mathbb{Z}}
\newcommand{\Q}{\mathbb{Q}}
\newcommand{\C}{\mathbb{C}}

\newcommand{\clr}{\textcolor{red}}
\newcommand{\clb}{\textcolor{blue}}
\newcommand{\clg}{\textcolor{green}}
\newcommand{\clm}{\textcolor{magenta}}
\newcommand{\clv}{\textcolor{violet}}
\newcommand{\clbr}{\textcolor{Sepia}}

% American Paragraph Skip %
\setlength{\parindent}{0pt}
\setlength{\parskip}{1em}

% Document %
\pagestyle{fancy}
\begin{document}

\thispagestyle{fancy}

\section*{Stěhování}

Vaším úkolem je napsat algoritmus, který naplní krabice věcmi.

Na vstupu máte danou funkci \texttt{vec(poradi)}, která dostane parametrem
přirozené číslo \texttt{poradi} a vrátí vám věc s tímto pořadím jako trojici
přirozených čísel \texttt{(šířka,výška,hloubka)} představující rozměry této
věci. V případě, že věc s daným pořadím již neexistuje, vrátí funkce
\texttt{vec} prázdnou množinu.

Napište algoritmus, který pro tento vstup dá na výstupu \textbf{počet krabic},
které potřebuji, abych do nich uskladnil všechny věci, za následujících
předpokladů:
\begin{itemize}
 \item Každá krabice má rozměry $100 \times 50 \times 100$ v pořadí
  $\text{šířka} \times \text{výška} \times \text{hloubka}$.
 \item Věc se vejde do krabice, pokud žádný z jejích rozměrů spolu s rozměry
  věcí již přítomných v dané krabici nepřekročí žádný z rozměrů této krabice.
 \item Věci do krabic vkládám v tom pořadí, v kterém mi je dává funkce
  \texttt{vec}. Jakmi\-le se již věc do krabice nevejde, beru krabici novou.
 \item Krabic mám k dispozici libovolné množství.
 \item Abyste se neupsali, smíte spolu porovnávat trojice čísel prostě pomocí
  $<$ a $>$ a smíte je od sebe rovněž odečítat. Tedy, napíši-li
  \vspace*{-1em}
  \begin{verbatim}
   (40,30,10) <= (100,50,100)
  \end{verbatim}
  \vspace*{-2.3em}
  myslím tím, že \texttt{40 <= 100}, \texttt{30 <= 50} a \texttt{10 <= 100}.
  Podobně,
  \vspace*{-1em}
  \begin{verbatim}
   (100,50,100) - (40,30,10) = (60,20,90).
  \end{verbatim}
\end{itemize}

\end{document}
