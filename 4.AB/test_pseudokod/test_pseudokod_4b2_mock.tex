\documentclass[a4paper,11pt]{article}

\usepackage[czech]{babel}
% Fonts %
\usepackage{fouriernc}
\usepackage[T1]{fontenc}

% Colors %
\usepackage[dvipsnames]{color}
\usepackage[dvipsnames]{xcolor}

% Page Layout %
\usepackage[margin=1.5in]{geometry}

% Fancy Headers %
\usepackage{fancyhdr}
\fancyhf{}
\cfoot{\thepage}
\rhead{}
\renewcommand{\headrulewidth}{0pt}
\setlength{\headheight}{16pt}

% Math
\usepackage{mathtools}
\usepackage{amssymb}
\usepackage{faktor}
\usepackage{import}
\usepackage{caption}
\usepackage{subcaption}
\usepackage{wrapfig}
\usepackage{enumitem}
\setlist{topsep=0pt}

\usepackage{tikz}
\usetikzlibrary{cd,positioning,babel,shapes,calc}
\usepackage{tkz-base}
\usepackage{tkz-euclide}

% Theorems
\usepackage[thmmarks, amsmath, thref]{ntheorem}
\usepackage{thmtools}

\theoremsymbol{\ensuremath{\blacksquare}}
\newtheorem*{solution}{Possible solution.}

% Title %
\title{\Huge\textsf{Homework -- PreIB 3.AB 4}\\
 \Large\textsf{Triangulations and Symmetries of Regular Polygons}
 \author{Áďa Klepáčů}
 \date{\today}
}

% Table of Contents %
\usepackage{hyperref}
\hypersetup{
 colorlinks=true,
 linktoc=all,
 linkcolor=blue
}

% Tables %
\usepackage{booktabs}
\usepackage{tabularx}

% Patch for hyphens
\usepackage{regexpatch}
\makeatletter
% Change the `-` delimiter to an active character
\xpatchparametertext\@@@cmidrule{-}{\cA-}{}{}
\xpatchparametertext\@cline{-}{\cA-}{}{}
\makeatother

\newcolumntype{s}{>{\centering\arraybackslash}p{.4\textwidth}}

% Operators %
\DeclareMathOperator{\Ker}{Ker}
\DeclareMathOperator{\Img}{Im}
\DeclareMathOperator{\End}{End}
\DeclareMathOperator{\Aut}{Aut}
\DeclareMathOperator{\Inn}{Inn}

% Common operators %
\newcommand{\R}{\mathbb{R}}
\newcommand{\N}{\mathbb{N}}
\newcommand{\Z}{\mathbb{Z}}
\newcommand{\Q}{\mathbb{Q}}
\newcommand{\C}{\mathbb{C}}

\newcommand{\clr}{\textcolor{red}}
\newcommand{\clb}{\textcolor{blue}}
\newcommand{\clg}{\textcolor{green}}
\newcommand{\clm}{\textcolor{magenta}}
\newcommand{\clv}{\textcolor{violet}}
\newcommand{\clbr}{\textcolor{Sepia}}

% American Paragraph Skip %
\setlength{\parindent}{0pt}
\setlength{\parskip}{1em}

% Document %
\pagestyle{fancy}
\begin{document}

\thispagestyle{fancy}

\clr{\textbf{\uppercase{Z následujících úloh si vyberte jednu!}}}


\subsection*{Princ a drak}

Vyrobte simulaci boje prince s drakem. Na vstupu máte dvě proměnné --
\texttt{princ} a \texttt{drak}. K tomu máte tři funkce -- \texttt{hp},
\texttt{damage} a \texttt{heal}, které dostanou parametrem prince nebo draka a
vrátí celé číslo.
\begin{itemize}
 \item Funkce \texttt{hp} značí, kolik životů ještě princi či drakovi zbývá.
 \item Funkce \texttt{damage} značí, kolik životů sebere útokem drak princi a
  princ drakovi.
 \item Funkce \texttt{heal} značí, kolik životů se princi či drakovi doplní.
\end{itemize}

Vaším úkolem je napsat algoritmus, který simuluje boj prince s drakem a vrátí
vítěze (tedy proměnnou \texttt{drak} nebo proměnnou \texttt{princ}). \textbf{Boj
je rozdělen na kola} a probíhá následovně:
\begin{enumerate}
 \item Každé kolo ubude z \texttt{hp} draka a prince \texttt{damage} toho
  druhého.
 \item Každé \textbf{třetí} kolo se \texttt{hp} draka i prince doplní o jejich
  \texttt{heal}.
 \item V moment, kdy \texttt{hp} prince či draka klesne pod \texttt{1}, tento
  zemře a druhý vyhrává.
\end{enumerate}

\emph{Příklad:} Pro vstup
\begin{center}
 \begin{tabular}{cccc}
  & \texttt{hp} & \texttt{damage} & \texttt{heal}\\
  \toprule
  Princ & \texttt{100} & \texttt{60} & \texttt{10}\\
  \midrule
  Drak & \texttt{200} & \texttt{20} & \texttt{30}\\
 \end{tabular}
\end{center}
bude boj probíhat takto:
\begin{center}
 \begin{tabular}{ccc}
  & \texttt{hp(princ)} & \texttt{hp(drak)} \\
  \toprule
  Na začátku & \texttt{100} & \texttt{200}\\
  \midrule
  1. kolo & \texttt{80} & \texttt{140}\\
  \midrule
  2. kolo & \texttt{60} & \texttt{80}\\
  \midrule
  3. kolo & \texttt{50} & \texttt{50}\\
  \midrule
  4. kolo & \texttt{30} & \texttt{-10}
 \end{tabular}
\end{center}
Čili algoritmus odpoví \texttt{princ}.

\clearpage

\subsection*{Dárky}

Darovanému koni na zuby nekoukej, a dej ho někomu jinému.

Na vstupu máte dány množinu dárků \texttt{D} a množinu lidí \texttt{L}. Z
množiny \texttt{L} do množiny \texttt{D} vede funkce \texttt{daroval}, která
vrací dárek, jež vám dal příslušný člověk. Například
"\texttt{daroval(}Zbyšek\texttt{) = }whisky"~znamená, že Zbyšek vám dal whisky.
Konečně, máte dánu proměnnou \texttt{nejdrazsi}, ve které je uložen nejdražší
dárek z množiny \texttt{D}.

Vaším rozdílem je všechny dárky kromě toho nejdražšího rozdat \textbf{jiným}
lidem, než vám je darovali. Formálně to znamená sestavit funkci \texttt{daruju},
která pro každý dárek kromě nejdražšího z množiny \texttt{D} rozhodne, který
člověk z množiny \texttt{L} ho dostane. Dosazování hodnot do funkce můžete v
pseudokódu napsat například jako "\texttt{daruju(}whisky\texttt{)}
\textleftarrow~Jarmila". Tím řeknete, že whisky darujete Jarmile.

\emph{Příklad:} Pro následující vstup
\begin{align*}
 \mathtt{D} &= \{\text{auto}, \text{plyšák}, \text{past na myši}\}\\
 \mathtt{L} &= \{\text{Kvido}, \text{Horymír}, \text{Řehoř}\}\\
 \mathtt{daruju} & : \text{Horymír} \to \text{auto}, \text{Kvido} \to
 \text{plyšák}, \text{Řehoř} \to \text{past na myši}\\
 \mathtt{nejdrazsi} &= \text{plyšák},
\end{align*}
je \textbf{jeden možný} správný výstup
\[
 \mathtt{daruju}: \text{past na myši} \to \text{Horymír}, \text{auto} \to
 \text{Kvido}.
\]

\end{document}
