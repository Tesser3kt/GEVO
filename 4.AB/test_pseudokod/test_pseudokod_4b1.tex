\documentclass[a4paper,11pt]{article}

\usepackage[czech]{babel}
% Fonts %
\usepackage{fouriernc}
\usepackage[T1]{fontenc}

% Colors %
\usepackage[dvipsnames]{color}
\usepackage[dvipsnames]{xcolor}

% Page Layout %
\usepackage[margin=1.5in]{geometry}

% Fancy Headers %
\usepackage{fancyhdr}
\fancyhf{}
\cfoot{\thepage}
\rhead{}
\renewcommand{\headrulewidth}{0pt}
\setlength{\headheight}{16pt}

% Math
\usepackage{mathtools}
\usepackage{amssymb}
\usepackage{faktor}
\usepackage{import}
\usepackage{caption}
\usepackage{subcaption}
\usepackage{wrapfig}
\usepackage{enumitem}
\setlist{topsep=0pt}

\usepackage{tikz}
\usetikzlibrary{cd,positioning,babel,shapes,calc}
\usepackage{tkz-base}
\usepackage{tkz-euclide}

% Theorems
\usepackage[thmmarks, amsmath, thref]{ntheorem}
\usepackage{thmtools}

\theoremsymbol{\ensuremath{\blacksquare}}
\newtheorem*{solution}{Possible solution.}

% Title %
\title{\Huge\textsf{Homework -- PreIB 3.AB 4}\\
 \Large\textsf{Triangulations and Symmetries of Regular Polygons}
 \author{Áďa Klepáčů}
 \date{\today}
}

% Table of Contents %
\usepackage{hyperref}
\hypersetup{
 colorlinks=true,
 linktoc=all,
 linkcolor=blue
}

% Tables %
\usepackage{booktabs}
\usepackage{tabularx}

% Patch for hyphens
\usepackage{regexpatch}
\makeatletter
% Change the `-` delimiter to an active character
\xpatchparametertext\@@@cmidrule{-}{\cA-}{}{}
\xpatchparametertext\@cline{-}{\cA-}{}{}
\makeatother

\newcolumntype{s}{>{\centering\arraybackslash}p{.4\textwidth}}

% Operators %
\DeclareMathOperator{\Ker}{Ker}
\DeclareMathOperator{\Img}{Im}
\DeclareMathOperator{\End}{End}
\DeclareMathOperator{\Aut}{Aut}
\DeclareMathOperator{\Inn}{Inn}

% Common operators %
\newcommand{\R}{\mathbb{R}}
\newcommand{\N}{\mathbb{N}}
\newcommand{\Z}{\mathbb{Z}}
\newcommand{\Q}{\mathbb{Q}}
\newcommand{\C}{\mathbb{C}}

\newcommand{\clr}{\textcolor{red}}
\newcommand{\clb}{\textcolor{blue}}
\newcommand{\clg}{\textcolor{green}}
\newcommand{\clm}{\textcolor{magenta}}
\newcommand{\clv}{\textcolor{violet}}
\newcommand{\clbr}{\textcolor{Sepia}}

% American Paragraph Skip %
\setlength{\parindent}{0pt}
\setlength{\parskip}{1em}

% Document %
\pagestyle{fancy}
\begin{document}

\thispagestyle{fancy}

\clr{\textbf{\uppercase{Z následujících úloh si vyberte jednu!}}}

\section*{Nákup}
Máte dánu množinu produktů \texttt{produkty} a množinu obchodů \texttt{obchody}.
Dále máte dány funkce \texttt{cena} a \texttt{skladem}. Funkce \texttt{cena}
dostane jako parametry \textbf{jeden produkt} a \textbf{jeden obchod} a vrátí
\textbf{cenu tohoto produktu v tomto obchodě}. Funkce \texttt{skladem} dostane
též jako parametry produkt a obchod a vrátí \textbf{true/false}, podle toho,
jestli tento obchod má skladem tento produkt.

Pokud je tedy například cena produktu \texttt{boty} v obchodě \texttt{deichmann}
1000 Kč, pak \texttt{cena(boty,deichmann) = 1000}. Podobně, když například
obchod \texttt{deichmann} produkt \texttt{boty} na skladě má, pak
\texttt{skladem(boty,deichmann) = true}.

Vaším úkolem je napsat algoritmus, který najde \textbf{jeden obchod}, ve kterém
\textbf{mají na skladě všechny produkty z množiny \texttt{produkty}} a navíc je
\textbf{celková cena za nákup} v tomto obchodě nejnižší ze všech obchodů.

\vspace*{2em}

\hrule

\section*{Mocniny dvojky}
Na vstupu máte dáno jedno kladné celé číslo \texttt{A}. O tomto čísle máte
rozhodnout, zda se jedná o mocninu čísla \texttt{2} (a pokud ano, tak o kterou).
Čili na vstupu vrátíte buď číslo \texttt{n} takové, že $\mathtt{2^{n} = A}$,
nebo vypíšete \texttt{\uv{A není mocninou 2}}.

\clr{\textbf{Pozor!}} \textbf{Náš počítač umí JENOM základní aritmetické operace
-- tj. sčítání, odčítání, násobení a dělení. Neumí třeba ani mocnit!}

\end{document}
