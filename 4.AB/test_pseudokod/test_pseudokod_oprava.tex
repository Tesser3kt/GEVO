\documentclass[a4paper,11pt]{article}

\usepackage[czech]{babel}
% Fonts %
\usepackage{fouriernc}
\usepackage[T1]{fontenc}

% Colors %
\usepackage[dvipsnames]{color}
\usepackage[dvipsnames]{xcolor}

% Page Layout %
\usepackage[margin=1.5in]{geometry}

% Fancy Headers %
\usepackage{fancyhdr}
\fancyhf{}
\cfoot{\thepage}
\rhead{}
\renewcommand{\headrulewidth}{0pt}
\setlength{\headheight}{16pt}

% Math
\usepackage{mathtools}
\usepackage{amssymb}
\usepackage{faktor}
\usepackage{import}
\usepackage{caption}
\usepackage{subcaption}
\usepackage{wrapfig}
\usepackage{enumitem}
\setlist{topsep=0pt}

\usepackage{tikz}
\usetikzlibrary{cd,positioning,babel,shapes,calc}
\usepackage{tkz-base}
\usepackage{tkz-euclide}

% Theorems
\usepackage[thmmarks, amsmath, thref]{ntheorem}
\usepackage{thmtools}

\theoremsymbol{\ensuremath{\blacksquare}}
\newtheorem*{solution}{Possible solution.}

% Title %
\title{\Huge\textsf{Homework -- PreIB 3.AB 4}\\
 \Large\textsf{Triangulations and Symmetries of Regular Polygons}
 \author{Áďa Klepáčů}
 \date{\today}
}

% Table of Contents %
\usepackage{hyperref}
\hypersetup{
 colorlinks=true,
 linktoc=all,
 linkcolor=blue
}

% Tables %
\usepackage{booktabs}
\usepackage{tabularx}

% Patch for hyphens
\usepackage{regexpatch}
\makeatletter
% Change the `-` delimiter to an active character
\xpatchparametertext\@@@cmidrule{-}{\cA-}{}{}
\xpatchparametertext\@cline{-}{\cA-}{}{}
\makeatother

\newcolumntype{s}{>{\centering\arraybackslash}p{.4\textwidth}}

% Operators %
\DeclareMathOperator{\Ker}{Ker}
\DeclareMathOperator{\Img}{Im}
\DeclareMathOperator{\End}{End}
\DeclareMathOperator{\Aut}{Aut}
\DeclareMathOperator{\Inn}{Inn}

% Common operators %
\newcommand{\R}{\mathbb{R}}
\newcommand{\N}{\mathbb{N}}
\newcommand{\Z}{\mathbb{Z}}
\newcommand{\Q}{\mathbb{Q}}
\newcommand{\C}{\mathbb{C}}

\newcommand{\clr}{\textcolor{red}}
\newcommand{\clb}{\textcolor{blue}}
\newcommand{\clg}{\textcolor{green}}
\newcommand{\clm}{\textcolor{magenta}}
\newcommand{\clv}{\textcolor{violet}}
\newcommand{\clbr}{\textcolor{Sepia}}

% American Paragraph Skip %
\setlength{\parindent}{0pt}
\setlength{\parskip}{1em}

% Document %
\pagestyle{fancy}
\begin{document}

\thispagestyle{fancy}

\section*{Oprava železnice}

Na vstupu máte dánu množinu \texttt{kolej} obsahující její jednotlivé díly. Dále
máte dány funkce \texttt{potrebuje\_opravu} a \texttt{cas\_opravy}. Funkce
\texttt{potrebuje\_opravu} dostane parametrem díl koleje a vrátí \uv{Ano}, či
\uv{Ne} (tj. \texttt{true}, či \texttt{false}), podle toho, zda daný díl koleje
potřebuje opravu. Podobně, funkce \texttt{cas\_opravy} vrátí pro daný díl
koleje, \textbf{který potřebuje opravu}, čas, jejž tato oprava zabere. Posledním
vstupem je množina \texttt{vlaky} obsahující vlaky, které mají po koleji značené
\texttt{kolej} jet za čas, který vám pro daný vlak řekne funkce
\texttt{cas\_odjezdu}.

Vaším úkolem je spočítat čas, který zabere oprava všech dílů z množiny
\texttt{kolej}, a vrátit množinu \texttt{V} vlaků, které je třeba odklonit na
kolej jinou, neboť jejich \texttt{cas\_odjezdu} je nižší, než potřebný čas na
opravu.

\emph{Příklad}:\\
Pro množiny \texttt{kolej = \{a,b,c\}}, \texttt{vlaky = \{v1,v2\}} a funkce
\begin{center}
 \begin{tabular}{c|c|c|c}
  & \texttt{potrebuje\_opravu} & \texttt{cas\_opravy} & \texttt{cas\_odjezdu}\\
  \toprule
  \texttt{a} & \texttt{true} & \texttt{40} & $\diagup$ \\
  \midrule
  \texttt{b} & \texttt{false} & $\diagup$ & $\diagup$\\
  \midrule
  \texttt{c} & \texttt{true} & \texttt{15} & $\diagup$\\
  \midrule
  \texttt{v1} & $\diagup$ & $\diagup$ & \texttt{35}\\
  \midrule
  \texttt{v2} & $\diagup$ & $\diagup$ & \texttt{60}
 \end{tabular}
\end{center}
je správná odpověď \texttt{V = \{v1\}}, protože celkový čas opravy činí 55
minut, a tudíž je vlak \texttt{v1} třeba odklonit.

\end{document}
