\documentclass[a4paper,11pt]{article}

\usepackage[czech]{babel}
% Fonts %
\usepackage{fouriernc}
\usepackage[T1]{fontenc}

% Colors %
\usepackage[dvipsnames]{color}
\usepackage[dvipsnames]{xcolor}

% Page Layout %
\usepackage[margin=1.5in]{geometry}

% Fancy Headers %
\usepackage{fancyhdr}
\fancyhf{}
\cfoot{\thepage}
\rhead{}
\renewcommand{\headrulewidth}{0pt}
\setlength{\headheight}{16pt}

% Math
\usepackage{mathtools}
\usepackage{amssymb}
\usepackage{faktor}
\usepackage{import}
\usepackage{caption}
\usepackage{subcaption}
\usepackage{wrapfig}
\usepackage{enumitem}
\setlist{topsep=0pt}

\usepackage{tikz}
\usetikzlibrary{cd,positioning,babel,shapes,calc}
\usepackage{tkz-base}
\usepackage{tkz-euclide}

% Theorems
\usepackage[thmmarks, amsmath, thref]{ntheorem}
\usepackage{thmtools}

\theoremsymbol{\ensuremath{\blacksquare}}
\newtheorem*{solution}{Possible solution.}

% Title %
\title{\Huge\textsf{Homework -- PreIB 3.AB 4}\\
 \Large\textsf{Triangulations and Symmetries of Regular Polygons}
 \author{Áďa Klepáčů}
 \date{\today}
}

% Table of Contents %
\usepackage{hyperref}
\hypersetup{
 colorlinks=true,
 linktoc=all,
 linkcolor=blue
}

% Tables %
\usepackage{booktabs}
\usepackage{tabularx}

% Patch for hyphens
\usepackage{regexpatch}
\makeatletter
% Change the `-` delimiter to an active character
\xpatchparametertext\@@@cmidrule{-}{\cA-}{}{}
\xpatchparametertext\@cline{-}{\cA-}{}{}
\makeatother

\newcolumntype{s}{>{\centering\arraybackslash}p{.4\textwidth}}

% Operators %
\DeclareMathOperator{\Ker}{Ker}
\DeclareMathOperator{\Img}{Im}
\DeclareMathOperator{\End}{End}
\DeclareMathOperator{\Aut}{Aut}
\DeclareMathOperator{\Inn}{Inn}

% Common operators %
\newcommand{\R}{\mathbb{R}}
\newcommand{\N}{\mathbb{N}}
\newcommand{\Z}{\mathbb{Z}}
\newcommand{\Q}{\mathbb{Q}}
\newcommand{\C}{\mathbb{C}}

\newcommand{\clr}{\textcolor{red}}
\newcommand{\clb}{\textcolor{blue}}
\newcommand{\clg}{\textcolor{green}}
\newcommand{\clm}{\textcolor{magenta}}
\newcommand{\clv}{\textcolor{violet}}
\newcommand{\clbr}{\textcolor{Sepia}}

% American Paragraph Skip %
\setlength{\parindent}{0pt}
\setlength{\parskip}{1em}

% Document %
\pagestyle{fancy}
\begin{document}

\thispagestyle{fancy}

Cílem úkolu je nakreslit flowchart popisující algoritmus, který řeší úlohu. Úloh
je celkem 6, každý z vás dostane přiděleny 2 v tabulce, kterou jsem poslal
společně s tímhle PDFkem.

Zkuste flowcharty nepsat \textbf{příliš} neformálně a nezapomínejte na věci jako
určit počáteční hodnotu každého údaje, který si potřebujete pamatovat, předtím
než ho používáte atd.

Flowcharty odevzdávejte do classroomu v jakékoli podobě -- na papíře, v
libovolné kreslící appce...

\begin{enumerate}
 \item Nakreslete algoritmus, který dostane zlomek a zkrátí ho. Samozřejmě
  nemůžete používat vágní formulace jako \uv{pokud jsou čitatel i jmenovatel
  dělitelné stejným číslem} nebo \uv{najdi největšího společného dělitele}.
  Počítejte s tím, že jediné, co vykonavatel algoritmu umí, je pamatovat si
  čísla a pak sčítat, odčítat, násobit, dělit a poznat, jestli jedno číslo dělí
  druhé.
 \item Nakreslete algoritmus, který vypíše $n$-té Fibonacciho číslo. Fibonacciho
  čísla $f_n$ jsou definována tak, že $f_1 = 0$, $f_2 = 1$ a potom $f_n =
  f_{n-1} + f_{n-2}$, tj. každé další číslo je součtem dvou předchozích.
  Počítejte s tím, že vykonavatel algoritmu umí jenom pamatovat si čísla,
  sčítat, odčítat, násobit a dělit.
 \item Nakreslete algoritmus, který najde dopravní spoje mezi místy A a B přes
  maximálně jedno další místo C, a vybere z nich ten časově nejkratší. Počítejte
  s tím, že stojíte na zastávce A, víte, kolik je hodin, a znáte dopravní řád
  všech zastávek -- A, B i C. Ovšem, jediné, co vykonavatel algoritmu umí, je
  pamatovat si časové údaje a místa, sčítat a odčítat časové údaje a číst
  dopravní řád -- tj. zjistit, kdy a odkud který spoj vyjíždí a kdy a kam
  přijíždí. Pochopitelně, mezi každou dvojicí míst může vést víc než jeden přímý
  spoj. \textbf{Nezapomeňte přičítat i čas, který musíte na zastávkách čekat!}
 \item Nakreslete algoritmus, který seřadí seznam jmen podle abecedy. Jediné, co
  vykonavatel algoritmu umí, je pamatovat si jména a čísla, prohazovat jména v
  seznamu a rozhodnout, které jméno je lexikograficky menší (jakoby dřív v~
  abecedě).
 \item Nakreslete algoritmus, který počítá tržby v dané skupině poboček během
  jedné otevírací doby. Pro každou pobočku má za úkol
  \begin{itemize}
   \item zjistit celkový počet zákazníků na pobočce během otevírací doby
    (vykonavatel algoritmu umí poznat, když zákazník vstoupí do obchodu);
   \item spočítat celkovou tržbu za všechny zakoupené produkty (vykonavatel
    algoritmu ví, kolik stojí každý produkt, a umí poznat, když si ho zákazník
    koupí);
   \item spočítat průměrný zisk každé pobočky (jako tržba / počet zákazníků);
   \item určit pobočku s největší návštěvností (je-li jich víc, tak libovolnou
    z~nich).
  \end{itemize}
 Kromě schopností popsaných výše umí algoritmus pouze základní operace na
 číslech (aritmetické operace + porovnávání) a pamatovat si čísla. \textbf{Nelze
 si pamatovat tržby nebo počty zákazníků poboček pro každou pobočku}, tedy je
 třeba všechny údaje počítat souběžně.
\item Nakreslete algoritmus, který obdrží slovo a rozhodne, jestli je to
 \emph{palindrom} -- čte se pozpátku stejně jako dopředu. Vykonavatel algoritmu
 umí poznat, kolik má slovo písmen, umí rozhodnout, zda jsou dvě slova stejná, a
 umí slovo číst písmeno po písmeně. Umí také spojit dvě slova dohromady,
 speciálně umí přidat písmeno ke slovu, a to jak zepředu tak zezadu.
 \textbf{Neumí otáčet slova!} Samozřejmě si umí slova a písmena též pamatovat.
\end{enumerate}

\end{document}
