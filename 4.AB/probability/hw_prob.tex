\documentclass[a4paper,11pt]{article}

\usepackage[czech,english]{babel}
% Fonts %
\usepackage{fouriernc}
\usepackage[T1]{fontenc}

% Colors %
\usepackage[dvipsnames]{color}
\usepackage[dvipsnames]{xcolor}

% Page Layout %
\usepackage[margin=1.5in]{geometry}

% Fancy Headers %
\usepackage{fancyhdr}
\fancyhf{}
\cfoot{\thepage}
\rhead{}
\renewcommand{\headrulewidth}{0pt}
\setlength{\headheight}{16pt}

% Math
\usepackage{mathtools}
\usepackage{amssymb}
\usepackage{faktor}
\usepackage{import}
\usepackage{caption}
\usepackage{subcaption}
\usepackage{wrapfig}
\usepackage{enumitem}
\setlist{topsep=0pt}

\usepackage{tikz}
\usetikzlibrary{cd,positioning,babel,shapes,calc}
\usepackage{tkz-base}
\usepackage{tkz-euclide}

% Theorems
\usepackage[thmmarks, amsmath, thref]{ntheorem}
\usepackage{thmtools}

\theoremsymbol{\ensuremath{\blacksquare}}
\newtheorem*{solution}{Possible solution.}

% Title %
\title{\Huge\textsf{Homework -- PreIB 4.AB 4}\\
 \Large\textsf{Inclusion-Exclusion Principle and Conditional Probability}
 \author{Áďa Klepáčů}
 \date{\today}
}

% Table of Contents %
\usepackage{hyperref}
\hypersetup{
 colorlinks=true,
 linktoc=all,
 linkcolor=blue
}

% Tables %
\usepackage{booktabs}
\usepackage{tabularx}

% Patch for hyphens
\usepackage{regexpatch}
\makeatletter
% Change the `-` delimiter to an active character
\xpatchparametertext\@@@cmidrule{-}{\cA-}{}{}
\xpatchparametertext\@cline{-}{\cA-}{}{}
\makeatother

\newcolumntype{s}{>{\centering\arraybackslash}p{.4\textwidth}}

% Operators %
\DeclareMathOperator{\Ker}{Ker}
\DeclareMathOperator{\Img}{Im}
\DeclareMathOperator{\End}{End}
\DeclareMathOperator{\Aut}{Aut}
\DeclareMathOperator{\Inn}{Inn}

% Common operators %
\newcommand{\R}{\mathbb{R}}
\newcommand{\N}{\mathbb{N}}
\newcommand{\Z}{\mathbb{Z}}
\newcommand{\Q}{\mathbb{Q}}
\newcommand{\C}{\mathbb{C}}

\newcommand{\clr}{\textcolor{red}}
\newcommand{\clb}{\textcolor{blue}}
\newcommand{\clg}{\textcolor{green}}
\newcommand{\clm}{\textcolor{magenta}}
\newcommand{\clv}{\textcolor{violet}}
\newcommand{\clbr}{\textcolor{Sepia}}

% American Paragraph Skip %
\setlength{\parindent}{0pt}
\setlength{\parskip}{1em}

% Document %
\pagestyle{fancy}
\begin{document}

\maketitle
\thispagestyle{fancy}

\textbf{You are to always formalize the problem, solve it and (at least briefly)
explain your reasoning.}

\section*{Problem 1.}

I've received a bag full of juggling balls for Christmas. As my parents are evil
theoretical mathematicians, they refused to let me use them until I solve a
problem.

I was told the balls have up to three different colours but some have only one
or two. The number of balls which are white (but not necessarily only white) is
15, 12 have black colour and 8 have blue. Moreover, there are 6 balls which are
both black and white, 4 that are black and blue and 3 that are white and blue.
Only 2 balls are three-coloured. \emph{What is the probability that a random
ball I pick is both black and white?}

\section*{Problem 2.}
A university has an acceptance rate of 10 \%, meaning 1 of every 10 applicants
is accepted. Accepted applicants have the right to also apply for financial aid
in the form of scholarships. The chance that an accepted applicant receives the
scholarship is 2 \%, meaning on average only 1 out of 50 accepted students do
receive it. \emph{What is the probability that a random applicant is both
accepted and receives a scholarship?}

\section*{Problem 3.}
A family has two children but we don't know their sex. If we're given the
information that the first child is a boy, what is the probability that the
second one is also a boy?

\section*{Problem 4.}
\emph{Given the following statistics, calculate the probability that a woman
over fifty has breast cancer if she has a positive mammogram result?}
\begin{itemize}
 \item 1 \% of all women over fifty years old have breast cancer.
 \item 90 \% of women over fifty who have breast cancer test positive on
  mammograms.
 \item 8 \% of all breast cancer tests on women over fifty are false positives.
\end{itemize}

\end{document}
