\documentclass[a4paper,11pt]{article}

% Language %
\usepackage[czech]{babel}

% Colors %
\usepackage[dvipsnames]{xcolor}

% Page Layout %
\usepackage[margin=1in]{geometry}

% Fancy Headers %
\usepackage{fancyhdr}
\fancyhf{}
\cfoot{}
\rhead{}
\renewcommand{\headrulewidth}{0pt}
\setlength{\headheight}{16pt}

% Math
\usepackage{mathtools}
\usepackage{amssymb}
\usepackage{faktor}
\usepackage{import}
\usepackage{caption}
\usepackage{subcaption}
\usepackage{wrapfig}

% Theorems
\usepackage{amsthm}
\usepackage{thmtools}

% Title %
\title{\Huge\textsf{}\\
 \Large\textsf{}
 \author{}
 \date{}
}

% Table of Contents %
\usepackage{hyperref}
\hypersetup{
 colorlinks=true,
 linktoc=all,
 linkcolor=blue
}

% Tables %
\usepackage{booktabs}

% Enumerate %
\usepackage{enumitem}

% Pseudocode %
\usepackage[linesnumbered,vlined,czech]{algorithm2e}
\SetKwInOut{Input}{input}\SetKwInOut{Output}{output}
\SetKw{KwReturn}{return}\SetKw{KwFrom}{from}
\SetKw{KwAnd}{and}\SetKw{KwOr}{or}
% skip line number
\let\oldnl\nl
\newcommand{\nonl}{\renewcommand{\nl}{\let\nl\oldnl}}

% Operators %
\DeclareMathOperator{\Ker}{Ker}
\DeclareMathOperator{\Img}{Im}
\DeclareMathOperator{\End}{End}
\DeclareMathOperator{\Aut}{Aut}
\DeclareMathOperator{\Inn}{Inn}

% Common operators %
\newcommand{\R}{\mathbb{R}}
\newcommand{\N}{\mathbb{N}}
\newcommand{\Z}{\mathbb{Z}}
\newcommand{\Q}{\mathbb{Q}}
\newcommand{\C}{\mathbb{C}}

% American Paragraph Skip %
\setlength{\parindent}{0pt}
\setlength{\parskip}{1em}

% Document %
\pagestyle{fancy}
\begin{document}

\section*{Šablona k Prší}

Pro ty, kdo opravdu potřebují, posílám jakousi nedodělanou verzi šablony pro
Prší, kterou jsem nastínil na hodině. Komentáře (tj. objasňující věty, co
\textbf{nepatří} do pseudokódu) budou psány buď \textcolor{blue}{modře} nebo
\textcolor{red}{červeně} v závislosti na důležitosti.

Na začátku mám proměnné $p_1, p_2, p_3$, ve kterých je uložen počet karet
každého kráče (tj. na začátku jsou všechny rovny 4). Mám taky množiny barev a
hodnot jednotlivých karet.

Aby vypadal náhodný výběr z ruky a odebírání/přidávání karet trochu elegantně,
definuji si ještě proměnné $P_1,P_2,P_3$, což jsou množiny karet každého hráče.
Dál potřebuji přepínat mezi hráči, na to si definuji proměnnou $i$, ve které
budou vždycky čísla $1, 2, 3, 1, 2, 3, \ldots$ To znamená, že hráč na tahu bude
pokaždé $p_i$. Poslední, co budu potřebovat je proměnná, ve které je uložena
poslední hozená karta, třeba $k$. Můžu počítat s tím, že všechny tyhle proměnné
dostanu na vstupu, jelikož ty jsou dané ještě před tím, než hra vůbec začne.

Připomínám, že to celé \textbf{můžete dělat úplně jinak}. Tohle je jen způsob,
jak bych to řešil já.

Algoritmus je na další stránce.

\begin{algorithm}[h]
 \DontPrintSemicolon
 \SetSideCommentRight

 \nonl\textcolor{blue}{\tcc{Význam všech těchhle proměnných je výše.}}
 \Input{$p_1, p_2, p_3, P_1, P_2, P_3, i, k$}
 \nonl\textcolor{blue}{\tcc{Výstupem může být třeba vítěz, tj. číslo $1, 2$ nebo
 $3$.}}
 \Output{číslo výherce}
 \BlankLine

 \nonl\textcolor{red}{\tcc{Hra nekončí, dokud má každý hráč aspoň jednu kartu
 v~ruce, tj. dokud $p_1, p_2$ i $p_3$ je větší než nula. Ještě si vytvořím
 proměnnou $b$, kde budu ukládat barvu vybranou hráčem, který poslední hodil
 $Q$. Zatím v ní může být cokoliv, to je jedno.}}
 $b \leftarrow \text{nějaká barva}$\;
 \While{$(p_1 > 0)$ \KwAnd $(p_2 > 0)$ \KwAnd $(p_3 > 0)$}{
  \nonl\textcolor{red}{\tcc{Přijde mi nejlepší se nejdřív podívat, jestli vůbec
  můžu odhodit nějakou kartu. Pokud jo, pak prostě nějakou náhodnou vyhodím.
  Jenom, když nemůžu odhodit, řeším, co vlastně mám dělat (přibírat, stát
  apod.)}}
  \nonl\textcolor{blue}{\tcc{Jedinou výjimkou je $K\spadesuit$, kde vždycky
  přibírám 5 karet. Připomínám, že $k$ označuje poslední hozenou kartu.}}
  \If{$k = K\spadesuit$}{
   Přiber pět karet. Tenhle příkaz musíte rozvést. Pokud si pamatujete jen počty
   karet, stačí přičíst $5$; pokud i karty v ruce, vytvořte si ideálně nějakou
   přibírací proceduru.
  }
  \nonl\textcolor{blue}{\tcc{Na $Q$ můžu použít takový trik. Abych nemusel
  speciálně hledat karty k odhození pro $Q$, můžu prostě změnit jeho barvu na tu
  zvolenou (jako byste si představili, že to $Q$ má tu zvolenou barvu). Pak
  všechno funguje normálně.}}
  \If{$\text{hodnota } k = Q$}{
   $\text{barva } k \leftarrow b$
  }
  \nonl\textcolor{blue}{\tcc{Množina $X$ budou všechny karty, které můžu
  odhodit. Ideálně si napište proceduru, která vám odpoví, co můžete odhodit.
  Doporučuji for cyklem procházet ruku hráče (pokud ji máte) a zkoušet, jestli
  sedí barva nebo hodnota s poslední odhozenou kartou.}}
  $X \leftarrow \{\text{všechny karty, které může hráč $i$ odhodit}\}$\;
  \nonl\textcolor{red}{\tcc{Když $X$ není prázdná, tak můžu něco odhodit. Vyberu
  náhodně a neřeším. Odhození zase musíte rozvést. POZOR! Odhozenou kartu si
  musíte pamatovat, a pokud je to $Q$, musíte zvolit (třeba náhodně) i nějakou
  barvu.}}
  \If{$X \neq \emptyset$}{
   Hráč $i$ odhodí náhodnou kartu z $X$.
  }
  \Else{
   \nonl\textcolor{red}{\tcc{Tohle je asi ta nejdůležitější část. Musíte
   rozlišit případy pro $7$ a pro $A$. $Q$ už máme vyřešeno, protože si
   `představujeme', že má tu změněnou barvu.}}
   Hráč $i$ dělá, co má, v závislosti na poslední hozené kartě.
  }
  \nonl\textcolor{blue}{\tcc{Posunu se na dalšího hráče.}}
  $i \leftarrow i + 1$\;
  \If{$i > 3$}{ 
   $i \leftarrow 1$\;
  }
 }
 \nonl\textcolor{blue}{\tcc{Odpovím číslo výherce. Výherce je předchozí hráč,
 protože pro hráče na tahu už se cyklus nespustil. To znamená, že předchozí hráč
 musel odhodit poslední kartu.}}
 \If{$i > 1$}{
  \KwReturn $i - 1$
 }
 \Else{
  \KwReturn $3$
 }
\end{algorithm}

\end{document}
